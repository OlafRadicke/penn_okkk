\section*{C}

\articlesize

\begin{description}

 \item[Cadbury, John] $\ast$12.8.1801 \dag11.5.1889, Quaker,
Schokoladen-Fabrikant aus Birmingham (England).

 \item[Ceresole, Pierre] $\ast$17.8.1879 in Lausanne \dag23.10.1945 in Le Daley
bei Lausanne. Schweizer Pazifist, Quäker, Mathematiker und Gründer des Service
Civil International (SCI).

 \item[Cookworthy, William] $\ast$12.4.1705 in Kingsbridge, Devon, England;
\dag17.10.1780 in Plymouth. Quaker, Apotheker, Chemiker und Erfinder. Er gilt
als Pionier sowohl der Kaolin-Industrie in Cornwall und Devon als auch der
Porzellanherstellung in England.

 \item[Cooper, James Fenimore] $\ast$15.9.1789 \dag14.9.1851. Quaker und Autor
 des bekannten fünf "`Lederstrumpf"'-Romane. 

 \item[Cromwell, Oliver] $\ast$25.4.1599 in Huntingdon, \dag3.9.1658 in
Westminster, Gründer der englischen Republik, regierte als Lordprotektor
England, Schottland und Irland während der kurzen republikanischen Periode der
britischen Geschichte. Die Quaker genossen zum Teil seinen Schutz auf Grund
guter Beziehungen, die zum Teil aus der Zeit des englischen Bürgerkrigs stammen.
Denn nicht wenige Quaker waren frühr Mitglieder der \textit{New Model Army},
deren Oberbefehlshaber Oliver Cromwell war.

 \end{description}

\normalsize