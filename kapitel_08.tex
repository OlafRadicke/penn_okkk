
\chapter{8. Kapitel} \label{kap8}

\section{Zusammenfassung des 8. Kapitel}

\footnotesize
\begin{description}
\item[1. Abschnitt] Der Stolz trachtct nicht nur nach hohen Einsichten, sondern
auch nach Macht.
\item[2. Abschnitt] Dieser beweiset der Fall mit Korah, Dothan etc.
\item[3. Abschnitt] auch Absalons Ehrgeiz bestätigt es.
%<!-- Nummerierung offensichtlich falsch im Orginal -->
\item[4. Abschnitt] Nebukadnezar giebt ebenfalls ein Beispiel davon.
\item[5. Abschnitt] So auch Pisistratus, Alexander, Casar u. a.
\item[6. Abschnitt] Die Türken, die so viel Blut vergossen haben, um ihr stolzes
Streben nach Gewalt zu befriedigen, sind ein lebender Veweie.
\item[7. Abschnitt] Das letztere Jahrzehent der Chritenthumes übertrifft alle
jene Beispiele.
\item[8. Abschnitt] Der Ergeiz wohnet nicht allein an Höfen, sondern findet auch
in der Brust. der Privatpersonen Raum, und stört die Ruhe und Glückseligkeit der
Familien und Gesellschaften.
\item[9. Abschnitt] Groß ist der Friede Derer, die durch Hülfe der göttlichen
Gnade ihren Begierden Schranken setzen, und die Macht, die sie besitzen, nur zum
Besten ihrer Mitmenschen anwenden.

\end{description}
\normalsize

\section{1. Abschnitt}

Nun wollen wir eine andere Wirkung dieser bösen Eigenschaft des Stolzes
betrachten, welche die gewöhnlichste, auffallendste und schädlichste derselben
ist. Der Stolz trachtet beständig mit Eifer nach Macht; und dieses
leidenschaftliche Trachten der Menschen hat zu allen Zeiten die meiste Unruhe
und Zerstörung in der Welt angerichtet. Es würde eine unnöthige Bemühung seyn,
wenn ich weitläuftige Beweise hierüber beibringen wollte; da nicht nur die
Jahrbücher der Geschichte, sondern auch unsere eigenen Erfahrungen uns
überzeugen, daß die mehrsten Kriege der Völker unter sich und mit Anderer, die
mehrsten Entvölkerungen der Staaten und Zerstörungen der Städte, nebst allen
Verheerungen, Bedrückungen und Jammerscenen<!-- weis der Teufel was das Wort
meint!-->, welche die traurigen Folgen davon waren, aus den Wirkungen des
Ehrgeizes oder eines nach Ansehn, Ruhm und Macht strebenden Stolzes
hervorgegangen sind.

\section{2. Abschnitt}

So scheinbar auch der Vorwand seyn mochte, dessen Korah, Dathan und Abiram sieh
gegen Moses bedienten, so war es dennoch nur ihr neidisches Trachten nach seiner
großen Macht im israelitischen Lager, welches sie zur Verschwörung und zum
Aufruhe gegen ihn verleitete. Sie strebten nach seinem Ansehn, und rechneten es
ihm zum Verbrechen an, daß. sie es nicht besaßen. Sie wollten selbst die Händler
und Anführer des Volks seyn. Und die Folge davon war, daß der Allmächtige sie
sammt allen ihren unglücklichen Mitschuldigen auf eine merkwürdige Weise
vertilgte.

\section{3. Abschnitt}

So wollte auch Absahlon die Rechte des Volkes gegen die Tyrannei seines Vaters
und Königs in Schulz nehmen, oder vielmehr seinen Ehrgeiz mit diesem Vorwande
beschönigen. Allein seine Empörung zeigte, daß es sein stolzes Streben nach
Macht war, welches ihn zu dem schändlichen Entschlusse gebracht hatte, seine
Pflicht als Sohn und Unterthan der Befriedigung seines rastlosen Ehrgeizes
aufzuopfern. Dadurch zog er sich selbst einen elenden Teod und seinem Heere eine
blutige Niederlage zu.\footnote{1.Sam. 15}

\section{4. Abschnitt}

Nebukadnezar dienet zu einem merkwürdigen Beispiele der unbändigen Begierde nach
Macht und der übermütigen Selbsterhebung, welche der Stolz einflößt. Aufgeblasen
von seinen großen Fortschritten und von dem Besize eines mächtigen Reichs, hatte
er im Taumel seiner Größe vergessen, daß er sich nicht selbst geschaffen habe,
und daß es noch eine höhere Macht als die seinige gäbe. Er ließ ein Bild machen,
vor dem Alle sich beugen oder verbrannt werden sollten; und als Sadrach, Mesach
und Abednego sich weigerten, seinem Gebote zu gehorchen, rief er höhnend aus:
"`Laßt sehen, wer der Gott sei, der euch aus meiner Hand erretten
werde!"'\footnote{Dan. 3} Er ward durch die Standhaftigkeit jener edlen Männer
von der göttlichen Allmacht überzeugt; aber dennoch, und ungeachtet der
Auslegung, die ihm Daniel von seinem Traume gegeben hatte, war bald darauf sein
Herz schon wieder mit so stolzen Gefühlen seiner Macht erfüllt, daß sein Mund in
die ruhmredige Frage ausbrach: "`Ist nicht dieses; die große Babylon, die ich
durch meine große Macht erbauet habe, zum königlichen Hause und zur Ehre meiner
Herrlichkeit?"'\footnote{Dan. 4,27} Aber die heilige Schrift sagt uns, daß, ehe
er noch diese Worte ausgeredet hatte, eine Stimme vom Himmel den Stolz seinen
Geistes bestraft habe, und daß er, seiner Vernunft beraubt, aus der menschlichen
Gesellschaft verstoßen worden sei, und seine Nahrung wie die Thiere auf dem
Felde habe suchen müssen.

\section{5. Abschnitt}

Werfen wir einige Blicke in die Weltgeschichte, so fallen uns häufige Beispiele
in die Augen, welche die Schädlichkeit dieser Leidenschaft des Stolzes an den
Tag legen. Ich will nur einige derselben, um Derer willen, die sie vielleicht
nicht gelesen oder nicht beachtet haben, hier berühren. 

Solon machte Athen durch eine vortreffliche Gesetzverfassung frei; aber der
Ehrgeiz des Pifistratuts zerstörte fein Werk vor feinen Augen. Alexander, nicht
zufrieden mit seinem Reiche, überzog andere mit Krieg, und erfüllte alle Länder,
die er unterjochte, mit Raub und Mord. Es war daher eine ganz richtige Bemerkung
des. Mannes, den Alexander der Seeräuberei beschuldigte, daß Alexander selbst
der größte Seeräuber in der Welt sei. Eben dieser Ehrgeiz machte Cäsar zum
Verfüher seineis Vaterlandes, und bewog ihn, die Armee, die ihm zur
Vertheidignng desselben anvertrauet war, gegen seine Vorgesetzten zu führen, sie
zu überwältigen, und die Herrschaft an sich zu reißen, wodurch zugleich Freiheit
und Tugend aus dem römischen Freistaate verbannet wurden. Denn nun erhoben sich
Parteien gegen alles Gute in Rom, und jene Maßigkeit und Maßheit, welche die
Senatoren zuvor so ehrwürdig machten, wurden jetzt ihrer Sicherheit gefährlich;
indem Cäsar’s Nachfolger kaum Jemand mit dem Tode oder mit der Verbannung
verschonten, der nicht Schmeichler ihrer Ungerechtigkeit wurde und ihrer
schwelgerischen Lebensart nicht folgte.

\section{6. Abschnitt}

Auch die Türken liefern einen sprechenden Beweise für meine Behauptung; da sie,
um ihre Macht und Herrschaft auszudehnen, oft große Blutbäder angerichtet und
viele herrliche Gegenden verwüstet haben.

Dennoch sind sie hierin von abgefallenen Christen übertroffen worden, deren
Handlungen um so verwerflicher sind, da sie bessern Unterricht als jene genossen
und einen Lehrer gehabt haben, der ihnen andere und vortrefflichere Lehren und
auch ein besserers Beispiel gegeben hat. Freilich nennen ihn noch immer ihren
Herrn; allein sie lassen sich von ihrem Ehrgeize beherrschen, und lieben Ansehen
und Macht weit mehr als ihren Nächsten. Um diese zu erlangen, erlauben sie sich,
einander zu erwürgen; obgleich der Herr ihnen geboten hat, "`nicht nach Hoheit
und Herrschaft zu streben, sondern einander zu lieben und zu
dienen."'\footnote{Matth 18,1-9} Was aber die Trauerscenen noch schrecklicher
macht, ist, daß die Wuth dieser furchtbaren Leidenschaft das ehrgeizigen
Strebens nach Größe und Macht sogar die Bande der Natur zerreißt, und ihr die
zartesten Gefühle natürlicher Liebe und Zuneigung zum Opfer gebracht werden.
Daher finden wir so oft die Blätter der Geschichte mit Ermordungen von Aeltern,
Kindern, Geschwistern, Oheimen, Neffen, Vorgesetzten u. s. w. befleckt.

\section{7. Abschnitt}

Sehen wir uns in den entfernten Weltgegenden um, so hören wir selten von
Kriegen; in der Christenheit selten von Frieden. Hier muß oft der geringfügigste
Gegenstand zu einer Veranlassung des Streites dienen; und es ist kein Bündniß so
heilig oder unverletzlich , daß man nicht künstlich zu umgehen oder auszulösen
verstände, sobald nur von Ausdehnung des Gebietes die Rede ist. Es wird wenig in
Betrachtung gezogen welche, oder wie viele Menschen dabei ums Leben kommen, oder
zu Wittwen und Waisen gemacht werden, oder ihr Eigenthum und ihren
Lebenssunterhalt verlieren; welche Länder zu Grunde gerichtet, welche Städte und
Ortschaften verheeret und ausgeplündert werden; wenn nur dadurch die Ehrgeizigen
ihre Zwecke erreichen können. Wir wollen nur sechzig Jahre zurückgehen, und
schon dieser kleine Zeitraum wird uns viele Kriege erdlicken lassen, die
ungerecht begonnen wurden und mit großen Verwüstungen endeten. Es würde zu
langweilig seyn, auch ist mein Zweck nicht, mich weiter hierüber auszulassen; da
Andere es schon oft bemerkt haben, und fast Jedermann weiß, in wie vielen
blutigen und schrecklichen Kriegen Frankreich, Spanien, Deutschland, England und
Holland bisher verwickelt gewesen sind.

\section{8. Abschnitt}

Der Ehrgeiz wohnet jedoch nicht allein an Höfen und bei Regierungen; ein Streben
nach Ansehn und Macht ist jeder einzelnen Brust nur zu natürlich geworden. Wir
sehen täglich, wie die Menschen alle ihre Kräfte ausbieten, ihren Verstand
anstrengen und Einfluß zu bekommen suchen, um sich hervorzuthun und höhere
Stellen oder größere Titel zu erwerben; damit sie ein vornehmeres Ansehen und
mehr Ehre erlangen, über Andere ihres Gleichen ein Uebergewicht gewinnen, und so
mit Denen, die sonst ihre Obern waren, ins Gleichgewicht kommen und Macht
besitzen mögen, ihren Freunden Gesetze vorzuschreiben und an ihren Feinden sich
zu reichen. Hierin liegt die Ursache, warum wahres Christenthum, welches alles
dieses verbietet, von weltlich gesinnten Menschen so wenig geliebt wird. Sein
Reich ist nicht von dieser Welt. Und wenn sie auch noch so gut davon reden, so
ist dennoch die Welt der wahre Gegenstand ihrer Liebe. Man kann daher, ohne
lieblos zu urtheilen, mit Wahrheit sagen, daß die mehrsten Menschen sich zwar
äußerlich zum Christeuthume bekennen, mit ihren Herzen aber der Welt anhangen.
Statt daß sie "`zuerst nach dem Reiche Gotteet und nach seiner Gerechtigkeit
trachten"'\footnote{Math. 6,35} und in allem Uebrigen sich auf Gott verlassen
sollten, suchen sie zuerst sich des Reichthumes und der Ehre dieser Welt zu
versichern, und schieben die Sorge für des Heil ihrer unsterblichen Seelen bis
aufs Krankenletger oder bis auf die letzten Augenblicke ihres Lebens auf; wenn
sie wirklich noch ein künftiges Leben glauben.

\section{9. Abschnitt}

Groß ist aber endlich der Friede Derer, die ihrem Ehrgeize Schranken zu setzen
wissen, und gelernt haben, mit dem, was die Vorsehung ihnen angewiesen und
zugemessen hat, zufrieden zu seyn; die nicht nach Ansehn streben, sondern, wenn
sie es besitzen, dabei demüthig und wolhlthätig sind. Diese verstehen das
Trachten ihrer: Geistes den Vorschriften ihres Gewissens unterzuordnen, und
können zu jeder Zeit mit ruhigen Gemüthe die unruhigen Bewegungen der Welt
ermessen, mitten im Schwanken ihrer Ungewißheit und Unbeständigkeit fest stehen,
und ali': Solche, die Antheil an einer bessern Welt haben, getrost und froh die
gegenwärtige verlassen, wenn es Gott gefällt, sie in der von ihm ersehenen Zeit
von dem Schauplalze der Wirksamkeit abzurufen; wohingegen die Ehergeizigen, mit
dem Bewußtsehn ihrer Uebelthaten und unter der drückenden Last ihrer Schuld ins
Grab sinken, um vor einem Richterstuhle zu erscheinen, der weder durch Furcht zu
erschüttern noch durch Bestechung zu bewegen ist.
