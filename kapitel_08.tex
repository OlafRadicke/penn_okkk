
\chapter{Kapitel 8} \label{kap8}

\section{Zusammenfassung des 8. Kapitel}


\begin{description}
\item[1. Abschnitt] Der Stolz trachtet nicht nur nach hohen Einsichten, sondern
auch nach Macht.
\dotfill \textit{Seite~\pageref{kap8_ab1}}\\
\item[2. Abschnitt] Dieser beweist der Fall mit Korah, Dothan etc.
\dotfill \textit{Seite~\pageref{kap8_ab2}}\\
\item[3. Abschnitt] auch Absalons Ehrgeiz bestätigt es.
\dotfill \textit{Seite~\pageref{kap8_ab3}}\\
\item[4. Abschnitt] Nebukadnezar gibt ebenfalls ein Beispiel davon.
\dotfill \textit{Seite~\pageref{kap8_ab4}}\\
\item[5. Abschnitt] So auch Pisistratus, Alexander, Cäsar u.a.
\dotfill \textit{Seite~\pageref{kap8_ab5}}\\
\item[6. Abschnitt] Die Türken, die so viel Blut vergossen haben, um ihr stolzes
Streben nach Gewalt zu befriedigen, sind ein lebender Beweis.
\dotfill \textit{Seite~\pageref{kap8_ab6}}\\
\item[7. Abschnitt] Das letztere Jahrzehnt des Christentums übertrifft alle
jene Beispiele.
\dotfill \textit{Seite~\pageref{kap8_ab7}}\\
\item[8. Abschnitt] Der Ehrgeiz wohnt nicht allein an Höfen, sondern findet auch
in der Brust der Privatpersonen Raum und stört die Ruhe und Glückseligkeit der
Familien und Gesellschaften.
\dotfill \textit{Seite~\pageref{kap8_ab8}}\\
\item[9. Abschnitt] Groß ist der Friede derer, die durch Hilfe der göttlichen
Gnade ihren Begierden Schranken setzen, und die Macht, die sie besitzen, nur zum
Besten ihrer Mitmenschen anwenden.
\dotfill \textit{Seite~\pageref{kap8_ab9}}\\

\end{description}

\newpage

\section{1. Abschnitt} \label{kap8_ab1}

Nun wollen wir eine andere Wirkung dieser bösen Eigenschaft des Stolzes
betrachten, welche die gewöhnlichste, auffallendste und schädlichste derselben
ist.
\label{ref:08_01_stolz} \textbf{
    Der Stolz trachtet beständig mit Eifer nach Macht\index{Macht!-streben}, und
dieses leidenschaftliche Trachten der Menschen hat zu allen Zeiten die meiste
Unruhe und Zerstörung in der Welt angerichtet.
}Es würde eine unnötige Bemühung sein,
wenn ich weitläufige Beweise hierüber beibringen wollte, da nicht nur die
Jahrbücher der Geschichte, sondern auch unsere eigenen Erfahrungen uns
überzeugen, dass die meisten Kriege\index{Krieg} der Völker unter sich und mit
anderer, die
meisten Entvölkerungen der Staaten und Zerstörungen der Städte, nebst allen
Verheerungen, Bedrückungen und Jammerszenen,
welche die traurigen Folgen davon waren, aus den Wirkungen des
Ehrgeizes oder eines nach Ansehen, Ruhm und Macht strebenden Stolzes
hervorgegangen sind.

\section{2. Abschnitt} \label{kap8_ab2}

So scheinbar auch der Vorwand sein mochte, dessen Korah\personenindex{Korah},
Dathan\personenindex{Dathan} und Abiram\personenindex{Abiram} sie
gegen Moses\personenindex{Mose} bedienten, so war es dennoch nur ihr
neidisches Trachten nach seiner
großen Macht im israelischen Lager, welches sie zur Verschwörung und zum
Aufruhr gegen ihn verleitete. Sie strebten nach seinem Ansehen und rechneten es
ihm zum Verbrechen an, dass sie es nicht besaßen. Sie wollten selbst die Händler
und Anführer des Volks sein. Und die Folge davon war, dass der Allmächtige sie
samt allen ihren unglücklichen Mitschuldigen auf eine merkwürdige Weise
vertilgte.

\section{3. Abschnitt} \label{kap8_ab3}

So wollte auch Absahlon\personenindex{Absahlon} die Rechte des Volkes gegen
die Tyrannei seines Vaters
und Königs in Schutz nehmen, oder vielmehr seinen Ehrgeiz mit diesem Vorwand
beschönigen. Allein seine Empörung zeigte, dass es sein stolzes Streben nach
Macht war, welches ihn zu dem schändlichen Entschluss gebracht hatte, seine
Pflicht als Sohn und Untertan der Befriedigung seines rastlosen Ehrgeizes
aufzuopfern. Dadurch zog er sich selbst einen elenden Tod und seinem Heer eine
blutige Niederlage zu.\biblecite{1. Samuel 15.}{Samuel 1 15@1. Samuel 15}

\section{4. Abschnitt} \label{kap8_ab4}

Nebukadnezar\personenindex{Nebukadnezar} dient zu einem merkwürdigen Beispiele
der unbändigen Gier nach
Macht und der übermütigen Selbsterhebung, welche der Stolz einflößt.
Aufgeblasen
von seinen großen Fortschritten und von dem Besitz eines mächtigen Reichs hatte
er im Taumel seiner Größe vergessen, dass er sich nicht selbst geschaffen hat,
und dass es noch eine höhere Macht als die seinige gibt. Er ließ ein Bild
machen,
vor dem alle sich beugen oder verbrannt werden sollten, und als
Sadrach\personenindex{Sadrach}, Mesach\personenindex{Mesach}
und Abednego\personenindex{Abednego} sich weigerten, seinem Gebote zu
gehorchen, rief er höhnend aus:
\textit{"`Lasst sehen, wer der Gott sei, der euch aus meiner Hand retten
werde!"'}\biblecite{Daniel 3.}{Daniel 03@Daniel 3}
Er war durch die Standhaftigkeit jener edlen Männer
von der göttlichen Allmacht\index{Gott!Allmacht Gottes} überzeugt, aber dennoch,
und ungeachtet der
Auslegung, die ihm Daniel\personenindex{Daniel} von seinem Traum gegeben
hatte, war bald darauf sein
Herz schon wieder mit so stolzen Gefühlen seiner Macht erfüllt, dass sein Mund
in
die stolze Frage ausbrach:
\textit{"`Ist nicht dieses die große Babylon, die ich
durch meine große Macht erbaut habe, zum königlichen Hause und zur Ehre meiner
Herrlichkeit?"'}\biblecite{Daniel 4,27.}{Daniel 04@Daniel 4}
Aber die heilige Schrift\index{Heilige!Schrift} sagt uns, dass, ehe
er noch diese Worte ausgeredet hatte, eine Stimme vom Himmel den Stolz seinen
Geistes bestraft hat und dass er, seiner Vernunft beraubt, aus der menschlichen
Gesellschaft verstoßen wurde, und seine Nahrung wie die Tiere auf dem
Feld hat suchen müssen.

\section{5. Abschnitt} \label{kap8_ab5}

Werfen wir einige Blicke in die Weltgeschichte, so fallen uns häufig Beispiele
in die Augen, welche die Schädlichkeit dieser Leidenschaft des Stolzes an den
Tag legen. Ich will nur einige derselben, um derer willen, die sie
% Claus: wird der Leser zun gesiezt?
% Olaf: Nein 'sie' ist die Mehrzahl von
% 'häufig Beispiele'. Deshalb auch klein geschrieben
vielleicht
nicht gelesen oder nicht beachtet haben, hier berühren.

\medskip

Solon\personenindex{Solon} machte Athen\index{Orte:!Athen} durch eine
vortreffliche Gesetzverfassung frei, aber der
Ehrgeiz des Pifistratuts\personenindex{Pifistratuts} zerstörte sein Werk vor
seinen Augen. Alexander\personenindex{Alexander}, nicht
zufrieden mit seinem Reich, überzog andere mit Krieg und erfüllte alle Länder,
die er unterjochte, mit Raub und Mord. Es war daher eine ganz richtige Bemerkung
des Mannes, den Alexander der Seeräuberei beschuldigte, dass Alexander selbst
der größte Seeräuber in der Welt sei. Eben dieser Ehrgeiz machte
Cäsar\personenindex{Cäsar} zum
Verfüher seines Vaterlandes und bewog ihn, die Armee, die ihm zur
Verteidigung desselben anvertrauet war, gegen seine Vorgesetzten zu führen, sie
zu überwältigen und die Herrschaft an sich zu reißen, wodurch zugleich Freiheit
und Tugend aus dem römischen Freistaat\index{Römischer Freistaat} verbannt
wurden. Denn nun erhoben sich
Parteien gegen alles Gute in Rom\index{Orte:!Rom}, und jene Mäßigkeit und
Maßheit, welche die
Senatoren\index{Senator (Rom)} zuvor so ehrwürdig machten, wurden
jetzt ihrer Sicherheit gefährlich,
indem Cäsars Nachfolger kaum jemand mit dem Tod oder mit der Verbannung
verschonten, der nicht Schmeichler ihrer Ungerechtigkeit wurde und ihrer
verschwenderischen\editnote{'verschwenderisch' ersetzt durch
'verschwenderischen'.} Lebensart nicht folgte.

\section{6. Abschnitt} \label{kap8_ab6}

Auch die Türken\index{Türken} liefern einen sprechenden Beweise für
meine Behauptung, da sie,
um ihre Macht und Herrschaft auszudehnen, oft große Blutbäder angerichtet und
viele herrliche Gegenden verwüstet haben.

\medskip

\label{ref:08_06_heiden}\textbf{Dennoch sind sie hierin von abgefallenen
Christen\index{Christ!abgefallener} übertroffen worden, deren
Handlungen um so verwerflicher sind, da sie besseren Unterricht als jene
genossen
und einen Lehrer gehabt haben, der ihnen andere und vortrefflichere Lehren und
auch ein bessereres Beispiel gegeben hat. Freilich nennen sie ihn noch immer
ihren
Herrn, allein sie lassen sich von ihrem Ehrgeize beherrschen und lieben Ansehen
und Macht weit mehr als ihren Nächsten\index{Nächster, der}.} Um diese zu
erlangen, erlauben sie sich,
einander zu erwürgen,\index{Herrschaft!untersagte} obgleich der Herr ihnen
geboten hat,
\textit{"`nicht nach Hoheit
und Herrschaft zu streben, sondern einander zu lieben und zu
dienen."'}\biblecite{Matthäus 18,1-9.}{Matthäus 18}
Was aber die Trauerszenen noch schrecklicher
macht, ist, dass die Wut dieser furchtbaren Leidenschaft des ehrgeizigen
Strebens nach Größe und Macht sogar die Bande der Natur zerreißt und ihr die
zartesten Gefühle natürlicher Liebe und Zuneigung zum Opfer gebracht werden.
Daher finden wir so oft die Blätter der Geschichte mit Ermordungen von Eltern,
Kindern, Geschwistern, Onkeln, Neffen, Vorgesetzten usw. befleckt.

\section{7. Abschnitt} \label{kap8_ab7}

Sehen wir uns in den entfernten Weltgegenden um, so hören wir selten von
Kriegen\index{Krieg!bei Heiden}\index{Heide}, in der Christenheit
selten von Frieden\index{Friede!bei Christen}\index{Christ}. Hier
muss oft der geringfügigste
Gegenstand zu einer Veranlassung des Streits dienen, und es ist kein Bündnis so
heilig oder unverletzlich, dass man nicht künstlich zu umgehen oder auszulösen
verstände, sobald nur von Ausdehnung des Gebietes die Rede ist. Es wird wenig in
Betrachtung gezogen welche oder wie viele Menschen dabei ums Leben kommen, oder
zu Witwen und Waisen gemacht werden, oder ihr Eigentum und ihren
Lebenssunterhalt verlieren, welche Länder zu Grunde gerichtet, welche Städte und
Ortschaften verheert und ausgeplündert werden, wenn nur dadurch die Ehrgeizigen
ihre Zwecke erreichen können. Wir wollen nur sechzig Jahre zurückgehen, und
schon dieser kleine Zeitraum wird uns viele Kriege\index{Krieg} erblicken
lassen, die
ungerecht begonnen wurden und mit großen Verwüstungen endeten. Es würde zu
langweilig sein, auch ist mein Zweck nicht, mich weiter hierüber auszulassen, da
andere es schon oft bemerkt haben, und fast jedermann weiß, in wie vielen
blutigen und schrecklichen Kriegen Frankreich\index{Orte:!Frankreich},
Spanien\index{Orte:!Spanien}, Deutschland\index{Orte:!Deutschland},
England\index{Orte:!England} und
Holland\index{Orte:!Holland} bisher verwickelt gewesen sind.

\section{8. Abschnitt} \label{kap8_ab8}

Der Ehrgeiz wohnet jedoch nicht allein an Höfen und bei Regierungen, ein Streben
nach Ansehen und Macht ist jeder einzelnen Brust nur zu natürlich geworden. Wir
sehen täglich, wie die Menschen alle ihre Kräfte ausbieten, ihren Verstand
anstrengen und Einfluss zu bekommen suchen, um sich hervorzutun und höhere
Stellen oder größere Titel zu erwerben, damit sie ein vornehmeres Ansehen und
mehr Ehre erlangen, über andere ihresgleichen ein Übergewicht gewinnen, und so
mit denen, die sonst ihre Obern wären, ins Gleichgewicht kommen und Macht
besitzen mögen, ihren Freunden Gesetze vorzuschreiben und an ihren Feinden sich
zu rächen.
\label{ref:08_08_reich_gottes} \textbf{Hierin liegt die Ursache, warum wahres
Christentum, welches alles
dieses verbietet, von weltlich gesinnten Menschen so wenig geliebt wird.} Sein
Reich ist nicht von dieser Welt\index{Gott!Reich Gottes}\index{Reich!Gottes}.
Und wenn sie auch noch so gut davon reden, so
ist dennoch die Welt der wahre Gegenstand ihrer Liebe. \textbf{Man kann daher,
ohne
lieblos zu urteilen, mit Wahrheit sagen, dass die meisten Menschen sich zwar
äußerlich zum Christentume bekennen\index{Bekenntnis}, mit ihren Herzen aber der
Welt anhängen.}
Statt dass sie
\textit{"`zuerst nach dem Reich Gottes und nach seiner Gerechtigkeit
trachten"'}\biblecite{Matthäus 6,35.}{Matthäus 06@Matthäus 6}
und in allem Übrigen sich auf Gott verlassen
sollten, \textbf{suchen sie zuerst sich des Reichtumes und der Ehre dieser Welt
zu
versichern und schieben die Sorge für das Heil ihrer unsterblichen Seelen bis
aufs Krankenlager oder bis auf die letzten Augenblicke ihres Lebens auf, wenn
sie wirklich noch an ein künftiges Leben glauben.}

\section{9. Abschnitt} \label{kap8_ab9}

\label{ref:08_09_friedem} \textbf{Groß ist aber endlich der Friede derer,
die ihrem Ehrgeiz Schranken zu setzen wissen, und gelernt haben, mit dem, was
die Vorsehung\index{Vorsehung} ihnen angewiesen und
zugemessen hat, zufrieden\index{Zufriedenheit}\index{Bescheidenheit} zu sein,
die nicht nach Ansehen streben, sondern, wenn
sie es besitzen, dabei demütig und wolhltätig sind.} Diese verstehen das
Trachten ihren Geist den Vorschriften ihres Gewissens unterzuordnen, und
können zu jeder Zeit mit ruhigen Gemüt die unruhigen Bewegungen der Welt
ermessen, mitten im Schwanken ihrer Ungewissheit und Unbeständigkeit fest
stehen,
und als solche, die Anteil an einer bessern Welt haben, getrost und froh die
gegenwärtige verlassen\index{Sterben}, wenn es Gott gefällt, sie in der von ihm
ersehenen Zeit
von dem Schauplatz der Wirksamkeit abzurufen, \textbf{wohingegen die
Ehrgeizigen, mit
dem Bewusstsein ihrer Übeltaten und unter der drückenden Last ihrer
Schuld\index{Gewissen!schlechtes} ins
Grab sinken, um vor einem Richterstuhl\index{Gericht!Jüngstes} zu erscheinen,
der weder durch Furcht zu
erschüttern noch durch Bestechung zu bewegen ist.}



