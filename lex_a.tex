\section*{A}

\articlesize
\begin{description}

 \item[Abolitionisten] Einige Quaker waren bekannte Abolitionisten. So zum
Beispiel $\to$Benjamin Lay und $\to$John Woolman. Wobei selbige gegen häftie
Wiederstände in den eigenen Reihen ankämfen mussten.\endnote{"`Die
Aufzeichnungen von John Woolman. Aus der Zeit der Sklavenbefreiung"'. 2.
Auflage, Lorber U. Turm Verlag, 1964, ISBN 3799901019}

 \item[Absentisten] Höfliche Umschreibung für Mitglieder, die selten oder nie
 in Erscheinung treten. Abgeleitet von lateinisch \textit{absentia}: Abwesenheit.
 \endnote{Nachweislich benutzt wurde der Begriff während der
 Auseinandersetzungen in der Berliner Gruppe durch ein scheiben vom
 18.~April~2008, von Horst Konopatzky}.
 In anderen christlichen Kofessionen spricht man von "`U-Boot-Christ"'.\endnote{
Seite „U-Boot-Christ“. In: Wikipedia, Die freie Enzyklopädie. Bearbeitungsstand:
17.12.2009, 21:20 UTC. URL: \\
\texttt{http://de.wikipedia.org/w/index.php?title=U-Boot-Christ\&oldid=68141830}}

 \item[acting under concern] $\to$\textit{Concern}

 \item[Ältester] (engl. \textit{Elders}) Diese Amt gibt es für jede Verwaltungsebene
($\to$\textit{Jahresversammlung, Monatsversamm, Viertel- bzw.
Bezirksversammlung}). Haubt aufgabe sind seelsorgerische und organisatorische
Aufgaben. Für diese Aufgabe werden in der Regel erfahrene und angesehende
Mitglieder berufen. Die $\to$Ämterbesetzung erfolgt nicht per Wahl.

 \item[Amnesty International] Eine Organisation die sich für Menschenrechte
einsetzt. Einer der Gründer $\to$Eric Baker war Quaker. Die Organisation selbst
ist aber überkonfessionel.

 \item[Ämterbesetzung]  Die meisten Ämter werden nicht durch Wahlen besetzt,
sondern ein von der Versammlung bestellter Ausschuß wird beauftragt geeignete
Personen zu suchen, und der Versammlung vor zu schlagen. Die Versammlung prüft
in einer $\to$\textit{Geschäftsversammlung} den Vorschlag. Gibt es ein
$\to$\textit{Sense of the Meeting} wirde der vorgeschlagenen Person das Amt
anvertraut. Herrscht keine Einigkeit oder gar Ablehnung, wir der Ausschuß
gebeten einen neuen Vorschlag vor zu bereiten. Dazu kann der
$\to$\textit{Bennenungsausschuß} neu zusammengesetzt werden.

 \item[Andacht] oder auch \textit{"`stille Andacht"'} (englisch: \textit{meeting
for worschip}), wird der Gottesdienst genant bei Quakern. Findet meist Sonntags
statt (muss aber nicht). Die Andacht zentrum und tragenenes Element des
spirituellen Lebebs einer $\to$\textit{Versammlung}. Besonderes Merkmal der
Quakerandacht ist, das es keine Liturgie gibt. In "`Ohne Kreuz keine Krohne"'
Kapitel 5 ($\to$~Seite~\pageref{kap5_ab1}) geht William Penn ausführlich darauf
ein, warum das so ist. Praktisch ist es so, das in der regel ein
$\to$~\textit{Schreiber} die Leitung hat. Entweder dur eine kurze Ansprache wird
von ihm die Andacht begonnen, oder dadurch das er sich einen Moment hinstellt,
um zu signalisieren, das die Andacht begonnen hat und das Reden eingestellt
werden soll. Hat jeder seinen Platz eingenommen und ist zur Ruhe gekommen. setzt
sich der Schreiber. Im Verlauf der Andacht können Teilnehmer spontan aufstehen
und zur Versammlung sprechen. Diese Redebeiträge sollten insperiert sein vom
$\to$\textit{Innerein Licht}. Die Prüfung obliegt der Versammlung.
Möglicherweise wird das gesate nach einiger Zeit der schweigenen Prüfung von
einem weiteren Redner aufgegriffen. Oder es gibt verschiedenste Redebeiträge,
die keinerlei inhaltliche Verbindungen haben. Die Redezeit wird nicht formal
begrenst. Von $\to$~George Fox und $\to$~Ludwig Seebohm zum Beispiel ist bekannt
das sie zum Teil über eine Stunde redeten (nicht immer zum Wohlgefallen der
Anwesenden). Das sprechen/predigen ist jedem gestattet Männer, Freuen, Kinder
und selbst Nichtquakern. So ist bekannt, das in Deutschland und Holland auch
Mennoniten an Quakerandachten teilnamen und auch sprachen (dabei kam es auch
gelegendlich auch zu so häftigen Tumulten, das weltliche Ordnungskräfte
eingriffen, weil sich Anwohner über Lerm beschwerten)\endnote{Sünne Juterczenka,
"`Über Gott und die Welt - Entzeitvisionen, Reformdebatten, und die europäische
Quäkermission in der frühen Neuzeit"', Vandenhoeck\&Ruprecht, 2008, ISBN
978-3-525-35458-2, Dort Seite~188 und Seite~94}. Formal gibt es kein
festgelegtes Ende, bei der Anacht. Hier ist es Aufgabe des Schreibers
wahrzunehmen, ob die Zeit gekommen ist die Versammlung zu beänden.  Der
Gottesdiens des evangelikalen Flügels des Quakertums ($\to$\textit{Evangelical
Friends}) hat heute einen vorbestimmten Ablauf mit Gesang, Predigt und Gebet.
Ähnelt also äusserlich kaum noch der ursprünglichen Quakerandacht.

 \item[Andachsgruppe] (engl. Worship Group) Als eine Andachsguppe wird eine
 Gruppe bezeichnet, die keine atministrativen Organe wie eine Monatsversammlung
 oder Ämter wie Älteste hat.

 \item[Anliegen] $\to$\textit{Concern}


 \end{description}
\normalsize
