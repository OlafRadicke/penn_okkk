

\chapter{Kapitel 7} \label{kap7}

\section{Zusammenfassung des 7. Kapitels}

\begin{description}
\item[1. Abschnitt] Vom Stolz, der ersten Hauptleidenschaft des Menschen, sein
Ursprung.
\dotfill \textit{Seite~\pageref{kap7_ab1}}\\
\item[2. Abschnitt] Nähere Erklärung und Auseinandersetzung desselben.
\dotfill \textit{Seite~\pageref{kap7_ab2}}\\
\item[3. Abschnitt] Eine unerlaubte Begierde nach Wissen, der Adam nachgab,
führte das menschliche Elend herbei.
\dotfill \textit{Seite~\pageref{kap7_ab3}}\\
\item[4. Abschnitt] Dadurch verlor der Mensch seine Unschuld und Reinheit.\\
.\dotfill \textit{Seite~\pageref{kap7_ab4}}\\
\item[5. Abschnitt] Was für Menschen sich in Adams Zustande befinden.
\dotfill \textit{Seite~\pageref{kap7_ab5}}\\
\item[6. Abschnitt] Das Wissen bläht auf.
\dotfill \textit{Seite~\pageref{kap7_ab6}}\\
\item[7. Abschnitt] Von den üblen Wirkungen des falschen Wissens und Erkennens
und dem Nutzen der wahren Erkenntnis.
\dotfill \textit{Seite~\pageref{kap7_ab7}}\\
\item[8. Abschnitt] Kains Beispiel dient hierin zum Beweise.
\dotfill \textit{Seite~\pageref{kap7_ab8}}\\
\item[9. Abschnitt] Stolz der Juden, indem sie sich einbildeten, weiser zu sein
als
Moses, der Knecht Gottes war.
\dotfill \textit{Seite~\pageref{kap7_ab9}}\\
\item[10. Abschnitt] Die Folge davon war, dass sie die wahren Propheten
verfolgten.\\
.\dotfill \textit{Seite~\pageref{kap7_ab10}}\\
\item[11. Abschnitt] Die göttliche Erkenntnis Christi brachte Friede auf Erden.\\
.\dotfill \textit{Seite~\pageref{kap7_ab11}}\\
\item[12. Abschnitt] Von den blinden Führern und dem Schaden den sie angerichtet
haben.
\dotfill \textit{Seite~\pageref{kap7_ab12}}\\
\item[13. Abschnitt] Der Stolz, den die Menschen in ihrer falsche Erkenntnis und
eigene Weisheit sehen, macht sie unfähig das einfache Zeugnis vom Evangelium
anzunehmen.
\dotfill \textit{Seite~\pageref{kap7_ab13}}\\
\item[14. Abschnitt] Die falschen Christen übertreffen die Juden sowohl in ihrem
Stolz auf ihre höhere Erkenntnis, als auch in ihrer Abweichung von dem wahren
christlichen Leben.
\dotfill \textit{Seite~\pageref{kap7_ab14}}\\
\item[15. Abschnitt] So wie Adam und die Juden durch ihren Ehrgeiz und durch ihr
Trachten nach hohen Begriffen sich zu Grunde richteten, ebenso haben die
Bekenner des Christentum sich ihren Untergang dadurch zugezogen, dass sie,
nachdem sie die Furcht Gottes verloren hatten, sich selbst hohe
Glaubensbekenntnisse und schöne Formen der Gottesverehrung verfertigten, denen
jedermann bei Strafe, verbrannt zu werden, beipflichten sollte.
\dotfill \textit{Seite~\pageref{kap7_ab15}}\\
\item[16. Abschnitt] Die üblen Wirkungen, welche dieses in der sogenannten
Christenheit hervorgebracht hat.
\dotfill \textit{Seite~\pageref{kap7_ab16}}\\
\item[17. Abschnitt] Das Mittel, von diesem elenden Verfall zurückzukommen.\\
.\dotfill \textit{Seite~\pageref{kap7_ab17}}\\

\end{description}

\newpage

\section{1. Abschnitt} \label{kap7_ab1}

Nachdem ich nun in Ansehung der unerlaubten Eigenliebe, -- insofern dieselbe die
Menschen täuscht und verleitet, sich für wirkliche Christen und wahre Gläubige,
ja für Heilige zu halten, während sie doch mit dem Kreuze Christi und den
heiligen Wirkungen desselben gänzlich unbekannt sind,~-- mein Gewissen befreite
und zugleich in der Kürze gezeigt habe;
%Sinn??
worin die wahre Gottesverehrung
besteht, und was für einen Einflus die Kraft des Kreuzes dabei haben müsse,
wenn dieselbe dem allmächtigen Gott wohlgefällig sein soll, so werde ich nun,
mit dem Beistand des Herrn, mich ausführlicher über diejenigen Gegenstände der
unerlaubten Eigenliebe\index{Eigenliebe!unerlaubte} verbreiten, welche die
Hauptangelegenheit, Sorge und
Unterhaltung der Welt ausmachen.\label{ref:07_01_drei_haupteigenschaften}
\textbf{Diese sind nun unter den drei
Haupteigenschaften, dem Stolze\index{Stolz}, dem Gier\footnote{\texttt{'Geize'
ersetzt durch 'Gier'.}} und der
Verschwendung\footnote{\texttt{'Üppigkeit' ersetzt durch 'Verschwendung'.}}
\index{Verschwendung} begriffen, aus
welchen alle anderen bösen Neigungen, wie die Bäche und Ströme aus ihren
natürlichen Quellen herfließen, und deren Ertötung gewiss mächtige Wirkungen
des wahren Kreuzes erfordert}; da sie, obgleich sie hier zuletzt abgehandelt
werden, doch in der Erfahrung sich zuerst zeigen, und folglich auch früh der
Vernichtung des Kreuzes übergeben werden müssen. Denn sobald dieses geschieht,
treten an die Stelle jener verderbten Neigungen und Gewohnheiten die gesegneten
Früchte der so notwendigen Gemütsverbesserung, nämlich
Selbstüberwindung\footnote{\texttt{'Selbstverleugnung' ersetzt durch
'Selbstüberwindung'.}},
Demut\index{Demut}, Mäßigkeit\index{Mäßigkeit}, Liebe\index{Liebe},
Geduld\index{Geduld} und Himmlischgesinntheit mit allen anderen
geistlichen Tugenden, die den Nachfolgern Jesu, des vollkommenen
Gottesmenschen\index{Gottesmenschen},
wohl anstehen.

\medskip

\textbf{Das Streben\index{Menschen!Streben der} und die Liebe aller Menschen ist
entweder auf Gott oder auf sich
selbst gerichtet.} Diejenigen, welche Gott über alles lieben, sind immer
beschäftigt, ihre Eigenliebe den Geboten Gottes zu unterwerfen, und sie lieben
sich selbst nur in dieser Unterwerfung unter ihn, der der Herr über alles ist.
Diejenigen aber, die von der wahren Liebe zu Gott abgewichen sind, lieben sich
selbst mehr als Gott, denn einer von beiden wird immer der Gegenstand unserer
höchsten Liebe sein. Mit jener ungeordneten Selbstliebe verbindet der Apostel
sehr richtig Stolz und Aufgeblasenheit\index{Aufgeblasenheit}.\footnote{1.
Timotheus 3,2+4.}
\index{Bibelstellen:!Timotheus 1 03@1. Timotheus 3)}
Auch waren die
Engel\index{Engel} nicht sobald von ihrer Liebe,
Pflicht\index{Pflicht} und Ehrfurcht\index{Ehrfurcht} gegen Gott abgewichen,
als sie anfingen, sich selbst übermäßig zu lieben und hochzuschätzen, welches
sie verleitete, sich über ihren wahren Stand zu erheben und nach einem höheren
Rang in der Schöpfung\index{Schöpfung!Rangordnung} zu trachten. Dieses war die Wirkung
des Stolzes, und,
durch die traurige Abweichung von der Liebe zu Gott, der schreckliche Fall
derer, die in Ketten der Finsternis bis auf den großen Gerichtstag
Gottes\index{Gott!Gericht Gottes}\index{Gericht!Gottes}
aufbehalten werden.

\section{2. Abschnitt} \label{kap7_ab2}

Der Stolz\index{Stolz}, dieses verderbliche Übel, mit dessen Auseinandersetzung
dieses
Kapitel anfängt, begann auch das Elend des ganzen Menschengeschlechts. Er ist
eine so mächtige, aus ihren furchtbaren Wirkungen und traurigen Folgen so
allgemein bekannte Leidenschaft, dass jedes unveränderte Herz eine Erklärung
seines Wesens in sich trägt. Doch will ich mit Wenigem sagen: Der \textbf{Stolz
ist ein Übermaß von Selbstliebe, verbunden mit einer Geringschätzung
anderer\index{Geringschätzung!anderer} und mit einem Verlangen, über sie zu
herrschen\index{Herrschsucht}, welches ihn zum unruhigsten Übel in der
Welt macht.} \textbf{Durch vier Stücke hat der Stolz sich dem Menschen
vornehmlich zu
erkennen gegeben,} deren Folgen mit der Menge seiner Verbrechen ein gleiches Maß
von Elend verknüpfen. \textbf{Das erste} ist ein ungezügeltes Trachten nach
hoher
Erkenntnis\index{Erkenntnis}; \textbf{das zweite} ein ehrgeiziges Streben und
Ringen nach Macht\index{Macht}; \textbf{das dritte} ein übertriebenes Verlangen,
sich selbst geachtet und geehrt zu sehen,
\textbf{und endlich}, Prachtliebe oder Verschwendung\index{Verschwendung} in
Verzierungen der Hausgeräte und anderer weltlicher Dinge. \textbf{Wegen der
Wahrheit dieser Behauptung berufe ich mich auf den wahrhaften Zeugen des ewigen
Gottes\index{Gott!als Zeuge}, der in den Herzen aller Menschen
sich befindet.}

\section{3. Abschnitt} \label{kap7_ab3}

\label{ref:07_03_wissen_erbsuende}
\index{Schöpfungsgeschichte}\textbf{Was den ersten Punkt betrifft, so ist es
klar, dass ein ungezügeltes Verlangen
nach hoher Erkenntnis\index{Erkenntnis!Wissensdurst}
\index{Erkenntnis!Wissenschaftsfeindlichkeit} das Elend des Menschen
herbeiführte, und einen allgemeinen
Fall von der Herrlichkeit seines ursprünglichen Zustandes verursachte.
\index{Erbsünde}\index{Sündenfall}\index{Personen:!Adam} Adam
wollte gern noch weiser sein, als Gott ihn gemacht hatte. Es genügte ihm nicht,
seinen Schöpfer zu kennen und ihm die heilige Huldigung zu leisten, wozu sein
Dasein und seine Unschuld ihn aufforderten und antrieben. Er war nicht
zufrieden\index{Unzufriedenheit}, einen Verstand zu besitzen, der ihn über alle
Tiere auf dem Feld,
über die Vögel in der Luft und über die Fische im Meere erhob, und ihm die Macht
gab, über die ganze sichtbare Schöpfung Gottes zu herrschen, nein, er wollte
auch so weise wie Gott selbst sein.\footnote{1. Mose 2,19+20 und 1. Mose 3,5.}
\index{Bibelstellen:!Mose 1 02@1. Mose 2)}
\index{Bibelstellen:!Mose 1 03@1. Mose 3)}
Dieses unverzeihliche Trachten, dieser eben so törichte als ungerechte
Ehrgeiz\index{Ehrgeiz}, machte
ihn der von Gott empfangenen Wohltaten unwürdig.\footnote{1. Mose 3,34.}
\index{Bibelstellen:!Mose 1 03@1. Mose 3)}
Dieser vertrieb ihn aus dem Paradies\index{Paradies!Vertreibung}, und
anstatt Herr über die ganze Welt zu sein, war Adam der Elendeste auf der Erde.}

\section{4. Abschnitt} \label{kap7_ab4}

Seltsame Veränderung! \textbf{Statt Göttern gleich zu sein, fallen beide, Adam
und Eva,\index{Personen:!Adam}\index{Personen:!Eva}
niedriger als die Tiere,} mit denen verglichen sie als Götter geschaffen waren.
Die traurige Folge dieses großen Falles war eine bejammernswerte Verwandlung
der Unschuld in Schuld, eines Paradieses in eine Wildnis\index{Paradies}.
Das
Schlimmste aber
war noch, dass Adam und Eva in diesem unglücklichen Zustande statt des wahren
lebendigen Gottes einen anderen Gott annahmen, \textbf{denn der
Versucher\index{Satan}, der sie zu
diesem ganzen Unglück gereizt und verleitet hatte, versah sie nun mit einer
eiteln Erkenntnis\index{Erkenntnis} und schädlichen
Weisheit\index{Erkenntnis!schädliche Weisheit}, mit Geschicklichkeit und
Fertigkeit
im Lügen\index{Lüge} und zweideutigen Reden\index{Rede!zweideutige}, und mit
List, Ausreden\index{Ausreden} und Entschuldigungen\index{Entschuldigungen} zu
machen.} Das gerade, rechtschaffene, reine Herz des Menschen, dieses ihm
anerschaffene göttliche Ebenbild, nahm die Eigenschaften der krummen, sich
windenden und drohenden Schlange\index{Schlange}, das Bild des
ungerechten, unreinen Geistes an,
dessen Versuchungen ihn, durch seine eigene Nachgiebigkeit, und durch seinen
Ungehorsam gegen Gott, um seine paradiesische
Glückseligkeit\index{Glückseligkeit} brachten.

\section{5. Abschnitt} \label{kap7_ab5}

Dieser unglückliche Fall beschränkt sich aber nicht allein auf Adam, denn alle,
die das herrliche Gesetz Gottes in ihren Herzen übertreten haben, sind wirkliche
Kinder seines Ungehorsams. Sie haben, wie er, von der verbotenen
Frucht\index{Frucht, verbotene}
gegessen, nämlich das getan, was sie nicht hätten tun sollen, und das
unterlassen, was sie schuldig waren, zu tun. \textbf{Sie haben gegen das von
Gott
empfangene Maß des Lichts und der Erkenntnis gesündigt, den göttlichen Geist
betrübt} und daher auch die Erfüllung des schrecklichen Ausspruchs erfahren:
\textit{"`An dem Tage, an dem du davon isst, wirst du des Todes
sterben."'}\footnote{1. Mose 2,17.}
\index{Bibelstellen:!Mose 1 02@1. Mose 2)}
Das heißt: Sobald du tust, was du deiner Erkenntnis nach nicht tun
solltest, wirst du aufhören, in der Gunst Gottes\index{Gott!Gunst Gottes} zu
leben, und den Genuss des
seligen Friedens\index{Friede!selige} seines Geistes verlieren. Dieses ist das
Sterben jener
unschuldigen und heiligen Sehnsucht und Zuneigung, die Gott dem Menschen
anerschuf, der geistliche Tod\index{Tod!geistlicher}, der ihn kalt, erstarrt und
fühllos für die Liebe
Gottes, für den Einfluss seines Geistes, seiner Kraft und
Weisheit\index{Erkenntnis!göttliche Weisheit} macht, und ihn
des Lichts und der Freude seines Angesichts, der Überzeugung eines guten
Gewissens und des Zeugnisses und Beifalls des heiligen Geistes beraubt.

\section{6. Abschnitt} \label{kap7_ab6}

\label{ref:07_06_wissenschaft}
Die Erkenntnis Gottes, die Adam in seinem gefallenen Zustande besaß, bestand
also nicht mehr in einer täglichen Erfahrung der Liebe und des Werks Gottes in
seiner Seele, sondern bloß in einer Wissenschaft\index{Erkenntnis!Wissenschaft}
und Vorstellung von dem, was er
früher davon erkannt und erfahren hatte. \textbf{Da dieses nun nicht die wahre,
lebendige Weisheit\index{Erkenntnis!lebendige Weisheit}, die von oben
kommt, sondern gewissermaßen nur eine Abbildung
derselben ist, so kann sie den Menschen auch nicht in der Reinheit des Herzens
bewahren, sondern dient vielmehr dazu, dass sie ihn mit einer hohen Meinung von
sich selbst aufbläht, ihn stolz\index{Stolz} auf seine erhobenen Begriffe,
Einsichten und Widersprüche\index{Erdulden!Widersprüche} zu ertragen, ganz
unfähig macht. Dieses war der Zustand der
abgefallenen Juden\index{Personen:!Juden!abgefallene}, ehe Christus erschien,
und ist noch immer, seit seiner
Erscheinung, der Zustand der abgewichenen
Christen\index{Christ!abgewichener}, indem ihre Religion, wenn
man einige leibliche Übungen ausnimmt, teils nur in der Erinnerung dessen, was
sie ehemals von dem Werk Gottes in ihrem Innern erkannten und wovon sie
abgewichen sind, teils in einem bloß historischen Glauben und eingebildeten
Begriff, oder in wörtlicher Auslegung\index{Bibelauslegung!wörtliche}
\index{Gebote!wörtliche Auslegung} der Erfahrungen und Weissagungen jener
heiligen Männer und Frauen besteht, die zu allen Zeiten den Namen und Charakter
der wahren Kinder Gottes verdienten.}

\section{7. Abschnitt} \label{kap7_ab7}

So wie nun eine solche Erkenntnis Gottes nicht wahr und wesentlich ist, so
lehrt uns auch die Erfahrung, dass die Wirkungen, die sie hervorbringt, mit den
Früchten der wahren Weisheit beständig im Widerspruch stehen. Denn so wie diese
\textit{"`zuerst keusch, (rein) dann friedsam, dann gelinde ist, und sich sagen
lässt;"'}\footnote{Jakobus 3,17.}
\index{Bibelstellen:!Jakobus 03@Jakobus 3)}
so ist das Wissen entarteter, mit
Selbstüberwindung\footnote{\texttt{'Selbstverleugnung' ersetzt durch
'Selbstüberwindung'.}}
unbekannter Menschen, zuerst unrein und dann in allen Stücken das Gegentheil
der Weisheit von oben\index{Erkenntnis! göttliche Weisheit}\index{Gott!
göttliche Weisheit}. Denn es entsprang aus der Übertretung und erhält sich
nur in einem bösen und unreinen Gewissen\index{Gewissen!unreines} bei denen, die
dem Gesetze Gottes in
ihren Herzen ungehorsam sind und täglich Dinge tun, die sie nicht tun
sollten, weshalb sie auch vor dem Richterstuhl Gottes\index{Gott!Richterstuhl
Gottes} in den Herzen der
Menschen verurteilt werden. Denn das Licht der Gegenwart Gottes entdeckt die
verborgensten Werke der Finsternis\index{Werke!der Finsternis}, die geheimsten
Gedanken\index{Gedanken!geheime} und verstecktesten
Absichten\index{Absichten!versteckte} der ungöttlichen
Menschen\index{Ungöttlicher Mensch}. \textbf{Dieses ist die
\textit{"`falsch berühmte Kunst,"'}\footnote{1. Timotheus 6,20.}
\index{Bibelstellen:!Timotheus 1 06@1. Timotheus 6)} \label{ref:07_07_aroganz}
oder \textit{"`fälschlich sogenannte
Wissenschaft,"'}\index{Erkenntnis!falsche Wissenschaft}
die, so wie sie an sich unrein\index{Erkenntnis!unreine Wissenschaft} ist, die
Menschen auch\index{Ergeiz!falscher}unfriedsam, verdrießlich,
unbiegsam, launig, hartnäckig und verfolgungssüchtig macht, so dass sie andere,
die besser als sie sein wollen, nicht ertragen können, und diejenigen, die es
wirklich sind, hassen und schmähen.}

\section{8. Abschnitt} \label{kap7_ab8}

Dieser eifersüchtige Stolz\index{Stolz!eifersüchtiger}, diese
verabscheuungswürdige Leidenschaft voller Neid\index{Neid}
und Rachsucht\index{Rachsucht} war es, die Kain\index{Kain} zum
Brudermörder\index{Brudermord} machte. Wie? War denn seine
Religion und Gottesverehrung nicht eben so gut als die seines Bruders? Es fehlte
ja an keinem äußern Teile derselben. Kain\index{Opfer} opferte eben so wie
Abel, und sein
Opfer konnte an sich auch eben so gut als Abels Opfer sein, allein es scheint,
dass sein Herz es nicht war. -- Von so langer Zeit her hat Gott also schon auf
die innere Anbetung der Seele gesehen! -- Und was war nun die Folge dieser
Verschiedenheit des Herzens jener beiden Gottesverehrer? Kains Stolz empörte
sich. Er konnte es nicht ertragen, dass sein Bruder ihn übertraf. Sein Zorn
entbrannte. Er beschloss, die Verwerfung seines Opfers an dem Leben seines
Bruders zu rächen\index{Rache}, und ohne auf die natürlichen Bande der Liebe,
noch auf die
geringe Anzahl der damals lebenden Menschen Rücksicht zu nehmen, färbte der
Grausame seine Hände mit dem Blut seines Bruders.

\section{9. Abschnitt} \label{kap7_ab9}

Die Religion der ausgearteten Juden\index{Personen:!Juden!ausgeartete} brachte
keine bessere Früchte hervor. Denn
nachdem sie das innere Leben, die Kraft, den Geist des
Gesetzes\index{Gebote!Gesetz} verloren hatten,
blähten sie sich mit ihrer erlangten Erkenntnis desselben, und in dieser
eigenliebigen Gemütsverfassung dienten ihnen ihre Ansprüche auf
Abraham\index{Abraham} und
Moses\index{Personen:!Mose} und auf die göttlichen
Verheißungen\index{Gott!Verheißungen} nur dazu, dass sie ihren Stolz, ihre
unerträglichen Anmaßungen\index{Anmaßung} und ihre
Grausamkeit\index{Grausamkeit} auf's Höchste trieben, so dass sie
wahre höhere Erscheinungen, wenn sie damit heimgesucht
wurden\index{Heimsuchung!nicht ertragen}, nicht ertragen
konnten und die an sie gesandten Boten des Friedens\index{Bote des
Friedens} wie Wölfe und Tiger
behandelten.

\section{10. Abschnitt} \label{kap7_ab10}

Es ist auch sehr bemerkenswert, dass die falschen
Propheten\index{Prophet!falscher}, die sich beständig
als die heftigsten Gegner der Wahren bewiesen, diese immer als Falsche
verfolgten\index{Prophet!Verfolgung}, und die weltlichen Fürsten und
die arme verführte Menge durch ihren
Einflus auf dieselben als Werkzeug ihrer Bosheit gebrauchten. Daher geschah es,
dass ein heiliger Prophet voneinander gesagt, ein anderer zu Tod gesteinigt
wurde, usw. So stolz und steifsinnig\index{Starrsinn} macht die falsche
Wissenschaft\index{Erkenntnis!falsche Wissenschaft}
diejenigen, welche ihr nachtrachten, und dieses bewog den heiligen
\index{Personen:!Stephanus} Stephanus,
auszurufen:
\textit{"`Ihr Halsstarrigen und Unbeschnittenen an Herzen und Ohren! Ihr
widerstrebt allezeit dem heiligen Geiste. Wie euere Väter taten, so tut auch
ihr."'}\footnote{Apostelgeschichte 7,51.}
\index{Bibelstellen:!Apostelgeschichte 07@Apostelgeschichte 7)}

\section{11. Abschnitt} \label{kap7_ab11}

Die wahre Erkenntnis\index{Erkenntnis!wahre Erkenntnis} war mit Freuden von den
Engeln\index{Engel} verbreitet, \textit{"`welche Frieden auf Erden
und Wohlwollen gegen die Menschen"'}
verkündeten,\footnote{Johannes 11,18.}
\index{Bibelstellen:!Johannes 11)}
allein die Anhänger der falschen Wissenschaft
suchten diese Botschaft durch Verläumdungen zu verdunkeln. Christus sollte
durchaus ein Betrüger sein, und dieses sollte seine Macht, Wunder\index{Wunder}
zu tun,
beweisen, welches doch gerade das Gegentheil bewies. Sie suchten oft ihn zu
töten und führten ihre gottlose Absicht auch endlich aus. Und was war ihr
Beweggrund? -- Christus zeugte gegen ihre Heuchelei\index{Heuchelei}, gegen ihre
breiten
Denkzettel, gegen die Ehre von Menschen, die sie suchten,~-- doch sie geben ihre
Ursachen selbst in diesen Worten an:
\textit{"`Lassen wir ihn also, so werden alle an ihn
glauben;"'}\footnote{Johannes 11,18.}
\index{Bibelstellen:!Johannes 11)}
so wird er uns um unser Ansehen beim Volk
bringen, es wird ihm anhängen, und uns verlassen, und dann ist es um unsere
Macht und um unseren Ruf bei der Menge geschehen.

\section{12. Abschnitt} \label{kap7_ab12}

Eigentlich kam auch Christus, um ihre Ehre herabzubringen, ihr Rabbiwesen
umzustoßen und durch seine Gnade die Menschen zur innern Erkenntnis Gottes
anzuleiten, von der sie durch ihre Übertretungen abgewichen waren. Dadurch
sollten sie zur Einsicht der Betrügereien ihrer blinden
Führer\index{Führer!blinde} gelangen, die
durch ihre leeren Überlieferungen die Gerechtigkeit des
Gesetzes\index{Gebote!Gerechtigkeit des Gesetzes} vernichtet
hatten, und, weit entfernt, wahre Schriftgelehrte und rechte Ausleger des
Gesetzes zu sein, vielmehr wahre Kinder des argen Feindes waren, der von jeher
ein stolzer Lügner und grausamer Mörder war.

\section{13. Abschnitt} \label{kap7_ab13}

\label{ref:07_13_gelehrte}
\textbf{Da nun ihr Stolz auf ihre falsche Erkenntnis\index{Erkenntnis!falsche}
sie für die Annahme des einfachen
Evangeliums ganz unfähig gemacht hatte, so dankte Christus seinem Vater,
\textit{"`dass er die Geheimnisse desselben den Klugen und Weisen verborgen und
den Unmündigen geoffenbart habe."'}}
\footnote{Matthäus 11,25.}
\index{Bibelstellen:!Matthäus 11)}
Eben diese falsche Weisheit hatte
auch die Gemüter der
Athener\index{Athener}\footnote{\texttt{'Athenienser' ersetzt durch
'Athener'.}} zu einem solchen Grad
aufgebläht, dass sie
die Predigt des Apostels Paulus\index{Personen:!Paulus} als eitel und töricht
verwarfen. Allein dieser
Apostel, der eine vorzüglich gelehrte Erziehung seines Zeitalters genossen
hatte, drückt sich über jene von den Juden\index{Personen:!Juden} und
Griechen\index{Grieche!hoch geschätzte}
Weisheit sehr scharf und treffend aus, wenn er sagt:
\textit{"`Wo sind die Klugen? Wo
sind die Schriftgelehrten? Wo sind die Weltweisen? -- Hat nicht Gott die
Weisheit dieser Welt zur Torheit gemacht?"'}\footnote{1. Korinther 1,20.}
\index{Bibelstellen:!Korinther 1 01@1. Korinther 1)}
Und er
gibt weiter unten auch einen guten Grund dafür an:
\textit{"`Damit vor ihm kein Fleisch sich rühme."'}\footnote{1. Korinther 1,29.}
\index{Bibelstellen:!Korinther 1 01@1. Korinther 1)}
Das heißt: Gott wird die eingebildete
falsche Erkenntnis des Menschen zu Schanden machen, damit er in dieser Hinsicht
nichts habe, worauf er stolz sein könne, sondern alles der
Offenbarung\index{Offenbarung} des
göttlichen Geistes verdanken müsse. Der Apostel geht noch weiter und behauptet,
\textit{"`dass die Welt durch ihre Weisheit Gott nicht erkannte;"'}
welches so viel sagen
will, als dass diese Weisheit, so wie die Menschen sie anwenden, statt ihnen
behilflich zu sein, ihnen vielmehr an der wahren Erkenntnis Gottes hinderlich
ist. Und seine erste Epistel an seinen geliebten Timotheus schließt er mit den
Worten:
\textit{"`O Timotheus! Bewahre was dir anvertrauet ist und meide das
ungeistliche lose Geschwätz und das Gezänk der falsch berühmten
Kunst."'}\footnote{1. Timotheus 6,20.}
\index{Bibelstellen:!Timotheus 1 06@1. Timotheus 6)}
Dieses waren die herrschenden Gesinnungen der
Christen in den apostolischen Zeiten, als die göttliche Gnade ihnen die wahre
Erkenntnis Gottes verlieh und ihr Führer war.

\section{14. Abschnitt} \label{kap7_ab14}

\label{ref:07_14_vortschritt}
\textbf{Was für Fortschritte\index{Erkenntnis!Fortschritt} haben nun aber die
Jahrhunderte gemacht, welche auf die
apostolischen Zeiten folgten? Haben sie sich besser als die jüdischen Zeiten
gehalten? Nein! Nicht im geringsten. Die Bekenner des Christentums haben es
vielmehr sowohl hinsichtlich auf die hohen Ansprüche auf größere Erkenntnis, als
auch in Ansehung ihrer Abweichungen von dem wahren christlichen Leben, noch
ärger als die Juden\index{Personen:!Juden} gemacht.} Denn obgleich sie einen
bessern Vorgänger als die
Juden hatten, zu welchen Gott durch seinen Knecht Moses\index{Personen:!Mose}
redete, indem er sich
ihnen durch seinen geliebten Sohn, das Ebenbild seines Wesens, die Fülle aller
Sanftmut und Demut, mitteilte und obgleich ihnen nichts so sehr als die
Anbetung seines Namens und die Verehrung des Andenkens seiner gesegneten Jünger
und Apostel anzuliegen schien, so wurde dennoch ihre Abweichung von der innern
in der Seele sich offenbarenden Kraft Gottes und von dem reinen Leben des
Christentums so groß, dass ihre Ehrfurcht und Achtung fast in nichts anderes
als in leeren Formen und Zeremonien bestand. Sie zeigten freilich, wie die
Juden, ungemein viel Eifer, die Grabmäler der verstorbenen Heiligen
\index{Heiliger} zu schmücken
und ihre Bilder künstlich zu schnitzen, auch suchten sie nicht nur unter jedem
Vorwande alles sorgfältig zu sammeln und aufzubewahren, was als eine Reliquie
\index{Reliquie}
oder als ein Überbleibsel von ihren Personen betrachtet werden konnte, sondern
gaben auch tausend Dinge für Reliquien aus und empfahlen sie als solche, deren
Ursprung bloß erdichtet war, die nicht selten ins Lächerliche fielen, und im
Ganzen genommen mit dem Christentum nicht übereinstimmten, von den
wesentlichen und wichtigen Stücken des christlichen Gesetzes\index{Gebote!
christliches Gesetz} aber, von der
Liebe, Sanftmut und Selbstüberwindung\footnote{\texttt{'Selbstverleugnung'
ersetzt durch 'Selbstüberwindung'.}} waren sie entartet. Sie wurden
hochmütig, stolz, ruhmredig, unnatürlich oder gegen natürliche Zuneigung
unempfindlich, vorwitzig und streitsüchtig, so dass sie beständig die Kirche mit
zweifelhaften und spitzfindigen Fragen\index{Spitzfindigkeit} beunruhigten, und
der Menge Veranlassung
zu Mißhelligkeiten und Zänkereien gaben, woraus dann Parteien entstanden, die
endlich zum Blutvergießen\index{Christ!Blutvergießen unter sich}
schritten, --
gerade, als wenn sie dadurch, dass sie
sich zum Christentum bekannten, nur desto schlimmer geworden wären.

\medskip

\label{ref:07_14_ketzer}
\textbf{O des bejammernswerten Zustandes der vergeblichen Christen! Die, anstatt
dass
sie, der Lehre Christi und seiner Apostel gemäß,
\textit{"`ihre Feinde lieben und
diejenigen, die ihnen fluchen, segnen sollten,"'}
die Menschen lehren, unter dem
Vorwande eines christlichen Eifers, einander auf die unmenschlichste Weise zu
zerfleischen und umzubringen\index{Auseinandersetzungen!blutige}, die, anstatt
dass sie bereit sein sollten, ihr
eigenes Blut für das Zeugnis Jesu vergießen zu lassen, das Blut der Zeugen Jesu,
als vorgeblicher Ketzer\index{Ketzer}, vergießen.} So hat also die
listige Schlange, jener
schlaue böse Geist, der mit seinen Versuchungen Adam\index{Personen:!Adam} um
seine Unschuld
brachte, und die Juden\index{Personen:!Juden} verführte, von dem Gesetz
Gottes\index{Gebote!Gesetz Gottes}\index{Gott!Gesetz} abzuweichen, nun
auch durch eitle Lügen die Christen betrogen\index{Christ!Betrug} und
verleitet, das heilige
Gesetz des Geistes Christi in ihren Herzen\index{Gebote!Gesetz des Herzens} zu
verlassen, und \textit{Seine}
Sklaven\index{Sklave} zu werden,
\textit{"`der in den Herzen der Kinder des Ungehorsams
herrscht."'}\footnote{Epheser 2,2.}
\index{Bibelstellen:!Epheser 02@Epheser 2)}

\section{15. Abschnitt} \label{kap7_ab15}

Es ist bemerkenswert, dass, so wie der Stolz\index{Stolz}, der immer von
Aberglauben\index{Aberglaube} und
Starrsinn\index{Starrsinn} begleitet ist, Adam\index{Personen:!Adam} verleitete,
sich über den Stand, in welchen
Gott ihn gesetzt hatte, erheben zu wollen, und dass, so wie die
Juden\index{Personen:!Juden}, durch
denselben Stolz angetrieben, dass ihnen von Gott durch
Moses\index{Personen:!Mose} auf dem Berg
gegebene Vorbild zu übertreffen suchten, indem sie
\textit{"`ihre Pfosten an Gottes
Pfosten setzten, und ihre eigenen Menschengebote für wahre Lehren
ausgaben,"'}\footnote{Ezech. 43,8) Jesaja 29,13) Matthäus 15,9.}
\index{Bibelstellen:!Ezechiel 43)}
\index{Bibelstellen:!Jesaja 02@Jesaja 2)}
\index{Bibelstellen:!Matthäus 15)}
ebenso auch die
abgewichenen Namenschristen\index{Christ!Namenschristen}, durch dieselbe
Sünde
des Stolzes verleitet, mit
starrem Aberglauben\index{Aberglaube} und kühner Anmaßung\index{Anmaßung}, statt
der wahren geistlichen
Gottesverehrung und Kirchenzucht\index{Gebote!Kirchenzucht}
\index{Kirche!Kirchenzucht}bloß weltliche Zeremonien und Anordnungen
eingeführt, und mit solchen Neuerungen und überlieferten menschlichen Meinungen
vermischt haben, die, wie ihre zahlreichen Konzile\index{Konzil} und
verwickelten
Glaubensartikel\index{Gebote!Glaubensartikel} beweisen, offenbar
\textit{"`Früchte der irdischen Weißheit"'} sind.

\section{16. Abschnitt} \label{kap7_ab16}

So wie nun dieser unverantwortliche Stolz sie zuerst verleitete, die geistige
Beschaffenheit der christlichen Gottesverehrung\index{Gottesverehrung} so sehr
zu verkehren, dass sie
eher der bildlichen Verehrungsart\index{Verehrung!bildliche} der Juden und dem
pomphaften Gottesdienst
der Ägypter\index{Ägypter} ähnlich war, als mit der schlichten
Einfachheit\index{Einfachheit} der christlichen
Anordnung übereinstimmte, -- welche weder der Anbetung auf dem Berg noch der zu
Jerusalem\index{Orte:!Jerusalem} gleichen sollte~--, so trieb auch derselbe
Stolz, dieselbe
Anmaßung sie an, den Ruf ihrer großen Diana\index{Personen:!Diana} durch alle
nur erdenkliche
Mittel der Grausamkeit zu behaupten und aufrecht zu erhalten. Weder die sanften
Bitten, noch die demütigen Vorstellungen von Seiten derer, die sich genau an
die ursprüngliche Reinheit der christlichen Lehre\index{Lehre!christliche} und
Gottesverehrung hielten,
konnten diese hartherzigen Namenschristen\index{Christ!Namenschristen}
bewegen, sie von der Befolgung ihrer
unapostolischen Überlieferungen\index{Überlieferungen!unapostolische}
auszunehmen oder zu befreien. So wie die Lehrer
und Bischöfe\index{Bischof} dieser Ausgearteten anfingen, sich um die
mühsame Untersuchung und
Pflege der Herde Christi nicht mehr zu kümmern, wurden sie ehrgeizig,
habsüchtig und verschwenderisch\footnote{\texttt{'üppig' ersetzt durch
'verschwenderisch'.}}, so dass sie in ihrem Betragen eher stolzen, weltlichen
Herrschern, als demütigen, sich selbst
überwindenden\footnote{\texttt{'verleugnenden' ersetzt durch 'überwindenden'.}}
Jüngern Jesu glichen.
Und die Geschichte fast eines jeden Landes sagt uns, mit welchem Stolze, mit
welcher Grausamkeit\index{Grausamkeit}, und mit was für einem Blutsdurste, --
der sie die
ungewohntesten Martern und seltensten Foltern erfinden\index{Folter!erfinden}
ließ~--, sie die
heiligen Glieder Christi\index{Kirche!Glieder Christi}
verfolgt\index{Verfolgung} und aus der Welt geschafft, und noch
überdies mit so furchtbaren Bannflüchen\index{Bann!-flüche} belegt haben, dass
sie
dieselben,
insofern es ihnen möglich gewesen wäre, auch der himmlischen
Seligkeit\index{Seligkeit!rauben} beraubt
haben würden.

\medskip

Solche Schlachtopfer\index{Schlachtopfer} wurden immer von den wahren Christen
\textit{Märtyrer}\index{Märtyrer} genannt,
aber die falsche sogenannte Geistlichkeit\index{Geistlicher!falscher}
hat ihnen, den Verfolgungssüchtigen\index{Verfolgung}
ständig die Maus schieben muss.
Juden\index{Personen:!Juden} nachahmend, den Namen der
\textit{Gotteslästerer}\index{Gotteslästerer} und
\textit{Ketzer}\index{Ketzer} beigelegt. \index{Verfolgung!der
Rechtgläubigen} \label{ref:07_16_vervolgung}
\textbf{So haben diese Abtrünnigen\index{Abtrünniger} die Vorhersagung
unseres Herrn Jesu Christi erfüllt, der nicht sagte, dass die Menschen glauben
würden, den Göttern einen Dienst zu tun, wenn sie seine teueren Nachfolger, die
wahren Christen, töteten
-- welches sich auf die Verfolgung der Christen von Seiten der abgöttischen
Heiden\index{Heide} beziehen könnte~--; sondern, dass sie meinen
würden, Gott einen Dienst
damit zu tun,}\footnote{Johannes 16,2.}
\index{Bibelstellen:!Johannes 16)}
woraus klar hervorgeht, dass es von solchen
geschehen würde, die sich zu dem wahren Gott bekennen, wie auch allezeit die
abtrünnigen Christen vergeblich getan haben. Diese müssen also, ohne Zweifel,
jene Wölfe sein, von denen der Apostel vorhersagte,
\textit{"`dass sie unter die
Gläubigen kommen und die Herde nicht verschonen
würden,"'}\footnote{Apostelgeschichte 20,29.}
\index{Bibelstellen:!Apostelgeschichte 20)}
\textbf{wenn der von ihm angekündigte große Abfall seinen Anfang nähme, der
notwendig kommen müsste, damit der Glaube und die Treue der
Rechtschaffenen\index{Rechtschaffener}
geprüft und das große Geheimnis der Bosheit offenbar gemacht würde.}

\medskip

Ich schließe diesen Abschnitt mit der Behauptung der ganz unverkennbaren
Wahrheit\index{Wahrheit}, dass überall, wo der ausgeartete
Klerus\index{Klerus}, oder sogenannte
Geistlichkeit\index{Geistlichkeit}\index{Verfolgung!durch
Klerus}\index{Verbrechen!des Klerus}\index{Geschichte!des Klerus}, am meisten
Gewalt und Ansehen besaß und aus Fürsten und
Staaten den größten Einfluss hatte, auch immer die größten Verwirrungen,
Streitigkeiten, Bludfehden\footnote{\texttt{'Blutscenen' ersetzt durch
'Bludfehden'.}}, Einziehungen der Güter, Einkerkerungen und
Verbannungen\index{Verbannung} statt fanden, und zur Rechtfertigung dieser
Wahrheit berufe ich
mich auf die Geschichte aller Zeiten. Wie es in unserem Zeitalter hiermit steht,
überlasse ich den Leben den aus eigener Erfahrung und Beobachtung zu
beurteilen. Nur einige Tatsachen liegen uns so klar vor Augen, dass sie unserer
Bemerkung schwerlich entgehen können: Die Menschheit ist nicht
bekehrt\index{Bekehrung}, sondern
in einen so hohen Grade, auf eine so verfeinerte Art verderbt, dass die
Geschichte voriger Zeiten uns kein Beispiel davon liefert. Die Gottesverehrung
der Christen ist, augenscheinlich, in Zeremonie und Pomp ausgeartet. Der
geistliche Stand sucht, größtenteils unter dem Vorwand der geistlichen
Beförderung, nur weltliche Vorteile, indem gewöhnlich bei denen, die sich
demselben widmen, die Aussicht zu einer gemächlichen zeitlichen Versorgung ist
die
Haupttriebfeder in der Bestimmung ihrer Wahl, und fast jeder von ihnen findet
sich
bereit, seine gegenwärtige Stelle zu verlassen und um eine andere zu
werben, sobald diese ihm einen höhern Rang und reichlichere Einkünfte darbietet.
Es muss uns also klar einleuchten, dass Stolz und Gier\footnote{\texttt{'Geiz'
ersetzt durch 'Gier'.}} die beiden unglücklichen
Leidenschaften sind, von denen der gute fromme Petrus\index{Personen:!Petrus}
wohl voraussah, dass sie
ihnen zu Fallstricken werden würden, und durch welche sie so viel Unwissenheit,
Irrtum und Religionsverachtung in der Christenheit erzeugt haben.

\section{17. Abschnitt} \label{kap7_ab17}

\textbf{Das Mittel nun, aus dieser unglücklichen Abweichung und Verirrung
herauszukommen,
ist kein anderes, als dass man sich eine lebendige, seligmachende Erkenntnis der
Religion Jesu, nämlich eine innere Erfahrung\index{Erfahrung!innere} von dem
Werke Gottes in der Seele,
erwerbe. Diese zu erlangen, musst du, o Mensch, fleißig und sorgfältig auf die
\index{Heimsuchung}
"`Erscheinung der heilbringenden Gnade"'\footnote{Titus 2,11+12.}
\index{Bibelstellen:!Titus 02@Titus 2)}
in deiner Seele
achten und derselben Gehorsam leisten.} Sie wird dich von der breiten Straße auf
einen schmalen Weg leiten, von der Befriedigung deiner verderbten Neigungen dich
zurückhalten und dich zur Erfüllung deiner Pflichten\index{Pflicht} antreiben.
Sie wird dich
von den sündlichen Vergnügungen der Welt\index{Vergnügen!der Welt} zu einem
heiligen Leben\index{Leben!heiliges} der
Selbstüberwindung\footnote{\texttt{'Selbstverleugnung' durch 'Selbstüberwindung'
ersetzt.}}, von der Gewalt des Satans\index{Satan}, des argen Feindes
deiner
Glückseligkeit\index{Glückseligkeit}, zu Gott bekehren. Aber du musst dein
eigenes Leben, das Leben
deiner Eigenliebe\index{Eigenliebe!hassen}, einsehen und hassen,\footnote{Lukas
14,26) Johannes 12,25) Matthäus 10,39.}
\index{Bibelstellen:!Lukas 14)}
\index{Bibelstellen:!Johannes 12)}
\index{Bibelstellen:!Matthäus 10)}
du musst wachen und beten\index{Beten}, und auch fasten\index{fasten}, nicht auf
deinen Versucher\index{Satan},
sondern auf deinen Helfer und Erlöser\index{Erlöser} sehen, böse
Gesellschaft\index{Böse!Gesellschaft} meiden, dich oft
in die Einsamkeit\index{Einsamkeit} zurückziehen, und wie ein reiner
Pilger\index{Pilger} in dieser bösen Welt\index{Böse!Welt}
leben. Auf diese Weise wirst du zu der wahren Erkenntnis
Gottes\index{Gott!Erkenntnis} und Jesu Christi
gelangen, die deiner Seele ewiges Leben\index{Leben!ewiges}, eine wohlgegründete
Überzeugung von
dem, was sie selbst davon empfindet und erfährt, mit unzweifelhafter Gewissheit
erteilt. Die Herzen derer, die zu dieser Erfahrung gelangen, sind
\textit{"`getrost und fürchten sich nicht."'}\footnote{Psalm 112,7+8.}
\index{Bibelstellen:!Psalm 112)}



