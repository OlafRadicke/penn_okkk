
% STATUS=Unfertig
% FUSSNOTEN=1.Durchlauf
% ANFÜHRUNGSZEICHEN=ok
% KORSIVSCHRIFT=ok
% FETTSCHRIFT=ok
% RANDBEMERKUNGEN=weg
% MODERNE_RECHTSCHREIBUNG=noch nicht begonnen
% WORTERSÄTZUNG=ok
% VERSCHLAGWORTUNG(INDEX)=ok
% REFERENZIERUNG=abgeschlossen
% BIBELSTELLEN=ungeprüft

\chapter{7. Kapitel} \label{kap7}

\section{Zusammenfassung des 7. Kapitel}
\footnotesize
\begin{description}
\item[1. Abschnitt] Vom Stolz, der ersten Hauptleidenschaft des Menschen; sein
Ursprung. (Seite: \pageref{kap7_ab1})
\item[2. Abschnitt]  Nähere Erklärung und Auseinandersetzung desselben. (Seite: \pageref{kap7_ab2})
\item[3. Abschnitt] Eine unerlaubte Begierde nach Wissen, der Adam nachgab,
führte das menschliche Elend herbei. (Seite: \pageref{kap7_ab3})
\item[4. Abschnitt] Dadurch verlor der Mensch seine Unschuld und Reinheit. (Seite: \pageref{kap7_ab4})
\item[5. Abschnitt] Was für Menschen sich in Adams Zustande befinden. (Seite: \pageref{kap7_ab5})
\item[6. Abschnitt] Das Wissen blähet auf. (Seite: \pageref{kap7_ab6})
\item[7. Abschnitt] Von den übeln Wirkungen des falschen Wissens und Erkennens
und dem Nutzen der wahren Erkenntniß. (Seite: \pageref{kap7_ab7})
\item[8. Abschnitt] Kains Beispiel dienet hierin zum Beweise. (Seite: \pageref{kap7_ab8})
\item[9. Abschnitt] Stolz der Juden; indem sie sich dünkten, weise zu sein als
Moses, der Knecht Gottes war. (Seite: \pageref{kap7_ab9})
\item[10. Abschnitt] Die Folge davon war, das sie die wahren Propheten
verfolgten. (Seite: \pageref{kap7_ab10})
\item[11. Abschnitt] Die göttliche Erkenntnis Christi brachte Frieden auf Erden. (Seite: \pageref{kap7_ab11})
\item[12. Abschnitt] Von den blinden Führern und dem Schaden den sie angerichtet
haben. (Seite: \pageref{kap7_ab12})
\item[13. Abschnitt] Der Stolz den die Menschen in ihre falsche Erkenntniß und
eigene Weisheit sehen, macht sie unfähig das einfache Zeugniß vom Evangelium
anzunehmen. (Seite: \pageref{kap7_ab13})
\item[14. Abschnitt] Die falschen Christen übertreffen die Juden sowohl in ihrem
Stolze auf ihre höhere Ertenntniß, ale auch in ihrer Abweichung von dem wahren
christlichen Leben. (Seite: \pageref{kap7_ab14})
\item[15. Abschnitt] So wie Adam und die Juden durch ihren Ehrgeiz und durch ihr
Trachten nach hohen Begriffen sich zu Grunde richteten, eben so haben die
Bekenner des Christenthume sich ihren Untergang dadurch zugezogen, daß sie,
nachdem sie die Furcht Gottes verloren hatten, sich selbst hohe
Glaubensbekenntnisse und schöne Formen der Gottesverehrung verfertigten, denen
Jedermann bei Strafe, verbrannt zu werden, beipflichten sollte. (Seite: \pageref{kap7_ab15})
\item[16. Abschnitt] Die übeln Wirkungen, welche dieses in der sogenannten
Christenheit hervorgebracht hat. (Seite: \pageref{kap7_ab16})
\item[17. Abschnitt] Das Mittel, von diesem elenden Verfalle zurückzukommen. (Seite: \pageref{kap7_ab17})

\end{description}
\normalsize


\section{1. Abschnitt} \label{kap7_ab1}

Nachdem ich nun in Ansehung der unerlaubten Eigenliebe, -- insofern dieselbe die
Menschen täuscht und verleitet, sich für wirkliche Christen und wahre Gläubige,
ja für Heilige zu halten, während sie doch mit dem Kreuze Christi und den
heiligen Wirkungen desselben gänzlich unbekannt sind, -- mein Gewissen befreiete
und zugleich in der Kürze gezeigt habe; worin die wahre Gottesverehrung
bestehet, und was für einen Einfluß die Kraft des Kreuzes dabei haben müsse,
wenn dieselbe dem allmächtigen Gott wohlgefällig sein soll; so werde ich nun,
mit dem Beistande des Herrn, mich ausführlicher über diejenigen Gegenstände der
unerlaubten Eigenliebe \index{Eigenliebe, unerlaubte} verbreiten, welche die Hauptangelegenheit, Sorge und
Unterhaltung der Welt ausmachen. \textbf{Diese sind nun unter den drei
Haupteigenschaften, dem Stolze\index{Stolze}, dem Geitze\index{Geitze} und der Verschwendung \footnote{"`Üppigkeit"' ersetzt duch "`Verschwendung"' Prahlerei} \index{Verschwendung} begriffen, aus
welchen alle andern bösen Neigungen, wie die Bäche und Ströme aus ihren
natürlichen Quellen, herfließen, und deren Ertödtung gewiß mächtige Wirkungen
des wahren Kreuzes erfordert}; da sie, obgleich sie hier zuletzt abgehandelt
werden, doch in der Erfahrung sich zuerst zeigen, und folglich auch früh der
Vernichtung des Kreuzes übergeben werden müssen. Denn sobald dieses geschiehet,
treten an die Stelle jener verderbten Neigungen und. Gewohnheiten die gesegneten
Früchte der so nothwendigen Gemüthsverbesserung, nämlich Selbstüberwindung \footnote{"`Selbstverleugnung"' ersetzt durch "`Selbstüberwindung"'},
Demuth\index{Demuth}, Mäßigkeit\index{Mäßigkeit}, Liebe\index{Liebe}, Geduld\index{Geduld} und Himmlischgesinntheit mit allen andern
geistlichen Tugenden, die den Nachfolgern Jesu, des vollkommenen Gottesmenschen\index{Gottesmenschen},
wohl anstehen.

\medskip

\textbf{Das Streben\index{Streben der Menschen} und die Liebe aller Menschen ist entweder auf Gott oder auf sich
selbst gerichtet.} Diejenigen, welche Gott über Alles lieben, sind immer
beschäftigt, ihre Eigenliebe den Geboten Gottes zu unterwerfen, und sie lieben
sich selbst nur in dieser Unterwerfung unter ihn, der der Herr über Alles ist.
Diejenigen aber, die von der wahren Liebe zu Gott abgewichen sind, lieben sich
selbst mehr als Gott; denn einer von Beiden wird immer der Gegenstand. unserer
höchsten Liebe sein. Mit jener ungeordneten Selbstliebe verbindet der Apostel
sehr richtig Stolz und Aufgeblasenheit\index{Aufgeblasenheit}.
\footnote{1 Timotheus 3,2+4}
\index{Bibelstellen:!1 Timotheus 3)}
Auch waren die
Engel\index{Personen:!Engel} nicht sobald von ihrer Liebe, Pflicht\index{Pflicht} und Ehrfurcht\index{Ehrfurcht} gegen Gott abgewichen,
als sie anfingen, sich selbst übermäßig zu lieben und hochzuschätzen, welches
sie verleitete, sich über ihren wahren Stand zu erheben und nach einem höhern
Range in der Schöpfung\index{Range in der Schöpfung} zu trachten. Dieses war die Wirkung des Stolzes, und,
durch die traurige Abweichung von der Liebe zu Gott, der schreckliche Fall
Derer, die in Ketten der Finsterniß bis auf den großen Gerichtstag Gottes\index{Gericht Gottes}
aufbehalten werden.

\section{2. Abschnitt} \label{kap7_ab2}

Der Stolz\index{Stolz}, dieses verderbliche Übel, mit dessen Auseinanderselzung dieses
Kapitel anfängt, begann auch das Elend des ganzen Menschengeschlechts. Er ist
eine so mächtige, aus ihren furchtbaren Wirkungen und traurigen Folgen so
allgemein gekannte Leidenschaft, daß jedes unveränderte Herz eine Erklärung
seines Wesens in sich trägt. Doch will ich mit Wenigem sagen: der \textbf{Stolz ist ein Übermaß von Selbstliebe, verbunden mit einer Geringschätzung Anderer\index{Geringschätzung Anderer} und mit einem Verlangen, über sie zu herrschen\index{Herrschsucht}, welches ihn zum unruhigsten Übel in der
Welt macht.} \textbf{Durch vier Stücke hat der Stolz sich dem Menschen vornehmlich zu
erkennen gegeben,}deren Folgen mit der Menge seiner Verbrechen ein gleiches Maß
von Elend verknüpfen. \textbf{Das Erste} ist, ein ungezügeltes Trachten nach hoher
Erkenntniß\index{Erkenntniß}; \textbf{das Zweite}, ein ehrgeiziges Streben und Ringen nach Macht\index{Macht}; \textbf{das Dritte}, ein übertriebenes Verlangen, sich selbst geachtet und geehrt zu sehen;
\textbf{und endlich}, Prachtliebe oder Verschwendung\index{Verschwendung} in Verzierungen der Hausgeräthe und anderer weltlichen Dinge. \textbf{Wegen der Wahrheit dieser Behauptung berufe ich mich auf den wahrhaftigen Zeugen des ewigen Gottes\index{Gatt, als Zeuge}, der in den Herzen aller Menschen
sich befindet.}

\section{3. Abschnitt} \label{kap7_ab3}

\index{Schöpfungsgeschichte}\textbf{Was den ersten Punkt betrifft, so ist es klar, daß ein ungezügeltes Verlangen
nach hoher Erkenntniß\index{Wissensdurst} \index{Wissenschaftsfeintlichkeit} das Elend des Menschen herbeiführte, und einen allgemeinen
Fall von der Herrlichkeit seines ursprünglichen Zustandes verursachte. \index{Ersünde} \index{Sündenfall} \index{Personen:!Adam} Adam
wollte gern noch weiser sein, als Gott ihn gemacht hatte. Es genügte ihm nicht,
seinen Schöpfer zu kennen, und ihm die heilige Huldigung zu leisten, wozu sein
Dasein und seine Unschuld ihn aufforderten und antrieben. Er war nicht
zufrieden\index{Unzufriedenheit}, einen Verstand zu besitzen, der ihn über alle Thiere auf dem Felde,
über die Vögel in der Luft und über die Fische im Meere erhob, und ihm die Macht
gab, über die ganze sichtbare Schöpfung Gottes zu herrschen; nein, er wollte
auch so weise als Gott selbst sein.
\footnote{1. Mose 2,19+20 und 1. Mose 3,5}
\index{Bibelstellen:!1. Mose 2)}
\index{Bibelstellen:!1. Mose 3)}
Dieses unverzeihliche Trachten, dieser eben so törichte als ungerechte Ehrgeiz\index{Ehrgeiz}, machte
ihn der von Gott empfangenen Wohlthaten unwürdig.
\footnote{1. Mose 3,34}
\index{Bibelstellen:!1. Mose 3)}
Dieser vertrieb ihn aus dem Paradiese\index{Paradies, Vertreibung aus}; und anstatt Herr über die ganze Welt zu sein, ward Adam der Elendeste auf der Erde.}

\section{4. Abschnitt} \label{kap7_ab4}

Seltsame Veränderung! \textbf{Statt Göttern gleich zu sein, fallen beide, Adam und Eva,\index{Personen:!Adam} \index{Personen:!Eva}
niedriger als die Thiere,} mit denen verglichen sie als Götter geschaffen waren.
Die traurige Folge dieses großen Falles war eine bejammernswerthe Verwandlung
der Unschuld in Schuld, eines Paradieses in eine Wildniß\index{Paradies}. Das Schlimmste aber
war noch, daß Adam und Eva in diesem unglücklichen Zustande statt des wahren
lebendigen Gottes einen andern Gott annahmen; \textbf{denn der Versucher\index{Versucher, Der}, der sie zu
diesem ganzen Unglücke gereizt und verleitet hatte, versah sie nun mit einer
eiteln Erkenntniß\index{Erkenntniß} und schädlichen Weisheit\index{Weisheit, schädlichen}, mit Geschicklichkeit und Fertigkeit
im Lügen\index{Lügen} und zweideutigen Reden\index{Reden, zweideutige}, und mit List, Ausreden\index{Ausreden} und Entschuldigungen\index{Entschuldigunge} zu
machen.} Das gerade, rechtschaffene, reine Herz des Menschen, dieses ihm
anerschaffene göttliche Ebenbild, nahm die Eigenschaften der krummen, sich
windenden und drohenden Schlange\index{Personen:!Schlange, Die}, das Bild des ungerechten, unreinen Geistes an,
dessen Versuchungen ihn, durch seine eigene Nachgiebigkeit, und durch seinen
Ungehorsam gegen Gott, um seine paradiesische Glückseligkeit\index{Glückseligkeit} brachten.

\section{5. Abschnitt} \label{kap7_ab5}

Dieser unglückliche Fall beschränkt sich aber nicht allein auf Adam; denn Alle,
die das herrliche Gesetz Gottes in ihren Herzen übertreten haben, sind wirkliche
Kinder seines Ungehorsams. Sie haben, wie er, von der verbotenen Frucht\index{verbotene Frucht}
gegessen; nämlich das getan, was sie nicht hätten tun sollen, und das
unterlassen, was sie schuldig waren, zu tun. \textbf{Sie haben gegen das von Gott
empfangene Maß des Lichts und der Erkenntniß gesündigt, den göttlichen Geist
betrübt} und daher auch die Erfüllung des schrecklichen Ausspruchs erfahren:
\textit{"`Welches Tages du davon issest, wirst du des Todes sterben."'}
\footnote{1. Mose 2,17}
\index{Bibelstellen:!1. Mose 2)}
Das heißt: sobald du tust, was du deiner Erkenntniß nach nicht tun
solltest, wirst du aufhören, in der Gunst Gottes\index{Gunst Gottes} zu leben, und den Genuß des
seligen Friedens\index{Frieden, seligen} seines Geistes verlieren. Dieses ist das Sterben jener
unschuldigen und heiligen Sehnsucht und Zuneigung die Gott dem Menschen
anerschuf; der geistliche Tod\index{Tod, geistlicher}, der ihn kalt, erstarrt und fühllos für die Liebe
Gottes, für den Einfluß seines Geistes, seiner Kraft und Weisheit\index{Weisheit, göttliche} macht, und ihn
des Lichts und der Freude seines Angesichts, der Überzeugung eines guten
Gewissens, und des Mitzeugnisses und Beifalles des heiligen Geistes beraubt.

\section{6. Abschnitt} \label{kap7_ab6}

Die Erkenntniß Gottes, die Adam in seinem gefallenen Zustande besaß, bestand
also nicht mehr in einer täglichen Erfahrung der Liebe und des Werks Gottes in
seiner Seele, sondern bloß in einer Wissenschaft\index{Wissenschaft} und Vorstellung von Dem, was er
früher davon erkannt und erfahren hatte. \textbf{Da dieses nun nicht die wahre,
lebendige Weisheit\index{Weisheit, wahre/lebendige}, die von oben kommt, sondern gewissermaßen nur eine Abbildung
derselben ist, so kann sie den Menschen auch nicht in der Reinheit des Herzens
bewahren, sondern dienet vielmehr dazu, daß sie ihn mit einer hohen Meinung von
sich selbst aufblähet, ihn stolz\index{stolz} auf seine erhobenen Begriffe und Einsichten, und, Widersprüche\index{Widersprüche ertragen} zu ertragen, ganz unfähig macht. Dieses war der Zustand der
abgefallenen Juden\index{Personen:!Juden, abgefallene}, ehe Christus erschien, und ist noch immer, seit seiner
Erscheinung, der Zustand der abgewichenen Christen\index{Personen:!Christen, abgewichenen}; indem ihre Religion, wenn
man einige leibliche Übungen ausnimmt, teils nur in der Erinnerung dessen, was
sie ehemals von dem Werke Gottes in ihrem Innern erkannten, und wovon sie
abgewichen sind, teils in einem bloß historischen Glauben und eingebildeten
Begriffe, oder in wörtlicher Auslegung\index{Auslegung, wörtliche} der Erfahrungen und Weissagungen jener
heiligen Männer und Weiber bestehet, die zu allen Zeiten den Namen und Charakter
der wahren Kinder Gottes verdienten.}

\section{7. Abschnitt} \label{kap7_ab7}

So wie nun eine solche Erkenntniß Gottes nicht wahr und wesentlich ist, so
lehret uns auch die Erfahrung, daß die Wirkungen, die sie hervorbringt, mit den
Früchten der wahren Weisheit beständig im Widerspruche stehen. Denn so wie diese
\textit{"`zuerst keusch, (rein) dann friedsam, dann gelinde ist, und sich sagen
läßt;"'}
\footnote{Jakobus 3,17}
\index{Bibelstellen:!Jakobus 3)}
so ist das Wissen entarteter, mit Selbstüberwindung \footnote{"`Selbstverleugnun"' ersetzt duch "`Selbstüberwindung"'}
unbekannter Menschen, zuerst unrein, und dann in allen Stücken das Gegentheil
der Weisheit von oben\index{Weisheit, göttliche}. Denn es entsprang aus der Übertretung, und erhält sich
nur in einem bösen und unreinen Gewissen\index{Gewissen, unreines} bei Denen, die dem Gesetze Gottes in
ihren Herzen ungehorsam sind, und täglich Dinge tun, die sie nicht tun
sollten; weshalb sie auch vor dem Richterstuhle Gottes\index{Richterstuhle Gottes} in den Herzen der
Menschen verurtheilt werden. Denn das Licht der Gegenwart Gottes entdeckt die
verborgensten Werke der Finsterniß\index{Werke der Finsterniß}, die geheimsten Gedanken \index{Gedanken, geheime} und verstecktesten
Absichten\index{Absichten, versteckte} der ungöttlichen Menschen\index{Menschen, ungöttliche}. \textbf{Dieses ist die
\textit{"`falsch berühmte Kunst,"'}
\footnote{1 Timotheus 6,20}
\index{Bibelstellen:!1 Timotheus 6)}
oder  \textit{"`fälschlich sogenannte Wissenschasft,"'}\index{Wissenschasft, fälschlich}
die, so wie sie an sich unrein \index{Wissenschasft, unreine} ist, die Menschen auch \index{Ergeiz, falscher}unfriedsam, verdrießlich,
unbiegsam, launig, hartnäckig und verfolgungssüchtig macht, so daß sie Andere,
die besser als sie sein wollen, nicht ertragen können, und Diejenigen, die es
wirklich sind, hassen und schmähen.}

\section{8. Abschnitt} \label{kap7_ab8}

Dieser eifersüchtige Stolz\index{Stolz, eifersüchtiger}, diese verabscheuungswürdige Leidenschaft voll Neides\index{Neid}
und Rachsucht\index{Rachsucht} war es, die Kain \index{Personen:!Kain} zum Brudermörder \index{Braudermord} machte. Wie? war denn seine
Religion und Gottesverehrung nicht eben so gut als die seines Bruders? Es fehlte
ja an keinem äußern Teile derselben. Kain \index{Opfer} opferte eben so wie Abel, und sein
Opfer konnte an sich auch eben so gut als Abels Opfer sein; allein es scheint,
daß sein Herz es nicht war. -- Von so langer Zeit her hat Gott also schon aus
die innere Anbetung der Seele gesehen! -- Und was war nun die Folge dieser
Verschiedenheit des Herzens jener beiden Gottesverehrer? Kains Stolz empörte
sich. Er konnte es nicht ertragen, daß sein Bruder ihn übertraf. Sein Zorn
entbrannte. Er beschloß, die Verwerfung seines Opfers an dem Leben seines
Bruders zu rächen\index{Rache}; und ohne auf die natürlichen Bande der Liebe, noch auf die
geringe Anzahl der damals lebenden Menschen Rücksicht zu nehmen, färbte der
Grausame seine Hände mit dem Blute seines Bruders.

\section{9. Abschnitt} \label{kap7_ab9}

Die Religion der ausgearteten Juden\index{Personen:!Juden, ausgearteten} brachte keine bessere Früchte hervor. Denn
nachdem sie das innere Leben, die Kraft, den Geist des Gesetzes\index{Gesetz} verloren hatten,
bläheten sie sich mit ihrer erlangten Erkenntniß desselben; und in dieser
eigenliebigen Gemüthsverfassung dienten ihnen ihre Ansprüche aus Abraham\index{Personen:!Abraham} und
Moses\index{Personen:!Mose} und aus die göttlichen Verheißungen\index{Verheißungen, göttliche} nur dazu, daß sie ihren Stolz, ihre
unerträglichen Anmaßungen\index{Anmaßung} und ihre Grausamkeit\index{Grausamkeit} auf's Höchste trieben; so daß sie
wahre höhere Erscheinungen, wenn sie damit heimgesucht wurden\index{Heimsuchzng, nicht ertragen}, nicht ertragen
konnten, und die an sie gesandten Boten des Friedens\index{Boten des Friedens} wie Wölfe und Tiger
behandelten.

\section{10. Abschnitt} \label{kap7_ab10}

Es ist auch sehr bemerkenswerth, daß die falschen Propheten\index{Personen:!Propheten, falschen}, die sich beständig
als die heftigsten Gegner der wahren bewiesen, diese immer als falsche
verfolgten\index{Verfolgung der Propheten}, und die weltlichen Fürsten und die arme verführte Menge durch ihren
Einfluß auf dieselben als Werkzeuge ihrer Bosheit gebrauchten. Daher geschah es,
daß ein heiliger Prophet von einander gesagt, ein Anderer zu Tode gesteinigt
wurde, usw. So stolz und steifsinnig\index{Starrsingigkeit} macht die falsche Wissenschaft\index{Wissenschaft, falsche}
Diejenigen, welche ihr nachtrachten und dieses bewog den heiligen \index{Personen:!Stephanus} Stephanus,
auszurufen:
\textit{"`Ihr Halsstarrigen und Unbeschnittenen an Herzen und Ohren! Ihr widerstrebet allezeit dem heiligen Geiste. Wie eure Väter thaten, so tut auch ihr."'}
\footnote{Apostelgeschichte 7,51}
\index{Bibelstellen:!Apostelgeschichte 7)}

\section{11. Abschnitt} \label{kap7_ab11}

Die wahre, Erkenntniß\index{Erkenntniß, wahre} ward mit Freuden von den Engeln\index{Engel} verbreitet,  \textit{"`welche Frieden auf Erden und Wohlwollen gegen die Menschen"'}
verkündeten;
\footnote{Johannes 11,18}
\index{Bibelstellen:!Johannes 11)}
allein die Anhänger der falschen Wissenschaft
suchten diese Botschaft durch Verläumdungen zu verdunkeln. Christus sollte
durchaus ein Betrüger sein; und dieses sollte seine Macht, Wunder\index{Wunder} zu tun,
beweisen, welche doch gerade das Gegentheil bewies. Sie suchten oft ihn zu
tödten, und führten ihre gottlose Absicht auch endlich aus. Und was war ihr
Beweggrund? -- Christus zeugte gegen ihre Heuchelei\index{Heuchelei}; gegen ihre breiten
Denkzettel; gegen die Ehre von Menschen, die sie suchten, -- doch sie geben ihre
Ursachen selbst in diesen Worten an:
\textit{"`Lassen wir ihn also, so werden Alle an ihn glauben;"'}
\footnote{Johannes 11,18}
\index{Bibelstellen:!Johannes 11)}
so wird er uns um unser Ansehn beim Volke
bringen; es wird ihm anhangen, und uns verlassen, und dann ist es um unsere
Macht und um unsern Ruf bei der Menge geschehen.

\section{12. Abschnitt} \label{kap7_ab12}

Eigentlich kam auch Christus, um ihre Ehre herabzubringen, ihr Rabbiwesen
umzustoßen, und durch seine Gnade die Menschen zur innern Erkenntniß Gottes
anzuleiten, von der sie durch ihre Übertretungen abgewichen waren. Dadurch
sollten sie zur Einsicht der Betrügereien ihrer blinden Führer\index{Führer, blinde} gelangen, die
durch ihre leeren Überlieferungen die Gerechtigkeit des Gesetzes\index{Gerechtigkeit des Gesetzes} vernichtet
hatten, und, weit entfernt, wahre Schriftgelehrte und rechte Ausleger des
Gesetzes zu sein, vielmehr wahre Kinder des argen Feindes waren, der von jeher
ein stolzer Lügner und grausamer Mörder war.

\section{13. Abschnitt} \label{kap7_ab13}

\textbf{Da nun ihr Stolz auf ihre falsche Erkenntniß\index{Erkenntniß, falsche} sie für die Annahme des einfachen
Evangeliums ganz unfähig gemacht hatte, so dankte Christus seinem Vater,
\textit{"`daß er die Geheimnisse desselben den Klugen und Weisen verborgen und den Unmündigen geoffenbaret habe."'}}
\footnote{Matthäus 11,25}
\index{Bibelstellen:!Matthäus 11)}
Eben diese falsche Weisheit hatte
auch die Gemüther der Athener\index{Personen:!Athener} \footnote{"`Athenienser"' ersetzt duch "`Athener"'} zu einem solchen Grade aufgeblähet, daß sie
die Predigt des Apostels Paulus\index{Personen:!Paulus} als eitel und thöricht verwarfen. Allein dieser
Apostel, der eine vorzüglich gelehrte Erziehung seines Zeitalters genossen
hatte, drückt sich über jene von den Juden\index{Personen:!Juden} und Griechen\index{Personen:!Griechen so hoch geschätzte} 
Weisheit sehr scharf und treffend aus, wenn er sagt:
\textit{"`Wo sind die Klugen? Wo
sind die Schriftgelehrten? Wo sind die Weltweisen? -- Hat nicht Gott die
Weisheit dieser Welt zur Thorheit getnacht?"'}
\footnote{1. Korinther 1,20}
\index{Bibelstellen:!1. Korinther 1)}
Und er
giebt weiter unten auch einen guten Grund dafür an:
\textit{"`Damit vor ihm kein Fleisch sich rühme."'}
\footnote{1.Korinther 1,29}
\index{Bibelstellen:!1. Korinther 1)}
Das heißt: Gott wird die eingebildete
falsche Erkenntniß des Menschen zu Schanden machen, damit er in dieser Hinsicht
nichts habe, worauf er stolz sein könne, sondern Alles der Offenbarung\index{Offenbarung} des
göttlichen Geistes verdanken müsse. Der Apostel geht noch weiter und behauptet,
\textit{"`daß die Welt durch ihre Weisheit Gott nicht erkannte;"'}
welches so viel sagen
will, als daß diese Weisheit, so wie die Menschen sie anwenden, statt ihnen
behülflich zu sein, ihnen vielmehr an der wahren Erkenntniß Gottes hinderlich
ist. Und seine erste Epistel an seinen geliebten Timotheus schließt er mit den
Worten:
\textit{"`O Timotheus! bewahre was dir anvertrauet ist, und meide das
ungeistliche lose Geschwätz und das Gezänk der falsch berühmten
Kunst."'}
\footnote{1. Timotheus 6,20}
\index{Bibelstellen:!1. Timotheus 6)}
Dieses waren die herrschenden Gesinnungen der
Christen in den apostolischen Zeiten, als die göttliche Gnade ihnen die wahre
Erkenntniß Gottes verlieh und ihr Führer war.

\section{14. Abschnitt} \label{kap7_ab14}

\textbf{Was für Fortschritte \index{Fortschritt} haben nun aber die Jahrhunderte gemacht, welche auf die
apostolischen Zeiten folgten? Haben sie sich besser als die jüdischen Zeiten
gehalten? Nein! nicht im geringsten. Die Bekenner des Christenthums haben es
vielmehr sowohl hinsichtlich auf die hohen Ansprüche auf größere Erkenntniß, als
auch in Ansehung ihrer Abweichungen von dem wahren christlichen Leben, noch
ärger als die Juden \index{Personen:!Juden} gemacht.} Denn obgleich sie einen bessern Vorgänger als die
Juden hatten, zu welchen Gott durch seinen Knecht Moses \index{Personen:!Mose} redete; indem er sich
ihnen durch seinen geliebten Sohn, das Ebenbild seines Wesens, die Fülle aller
Sanftmuth und Demuth, mittheilte und obgleich ihnen nichts so sehr als die
Anbetung seines Namens und die Verehrung des Andenkens seiner gesegneten Jünger
und Apostel anzuliegen schien; so wurde dennoch ihre Abweichung von der innern
in der Seele sich offenbarenden Kraft Gottes und von dem reinen Leben des
Christenthumes so groß, daß ihre Ehrfurcht und Achtung fast in nichts Anders,
als in leeren Formen und Zeremonien bestand. Sie bezeigten freilich, wie die
Juden, ungemein viel Eifer, die Grabmäler der verstorbenen Heiligen \index{Personen:!Heilige} zu schmücken
und ihre Bilder künstlich zu schnitzenz auch suchten sie nicht nur unter jedem
Vorwande Alles sorgfältig zu sammeln und aufzubewahren, was als eine Reliquie \index{Reliquien}
oder als ein Überbleibsel von ihren Personen betrachtet werden konnte, sondern
gaben auch tausend Dinge für Reliquien aus und empfahlen sie als solche, deren
Ursprung bloß erdichtet war, die nicht selten ins Lächerliche fielen, und im
Ganzen genommen mit dem Christenthume nicht übereinstimmten; von den
wesentlichen und wichtigen Stücken des christlichen Gesetzes\index{Gesetz, christliches} aber, von der
Liebe, Sanftmuth und Selbstüberwindung\footnote{"`Selbstverleugnung"' ersetzt durch "`Selbstüberwindung"'} waren sie entartet. Sie wurden
hochmüthig, stolz, ruhmredig, unnatürlich oder gegen natürliche Zuneigung
unempfindlich, vorwitzig und streitsüchtig; so daß sie beständig die Kirche mit
zweifelhaften und spitzfindigen Fragen \index{Spitzfindigkeit} beunruhigten, und der Menge Veranlassung
zu Mißhelligkeiten und Zänkereien gaben, woraus dann Parteien entstanden, die
endlich zum Blutvergießen\index{Blutvergießen, unter Christen} schritten; -- gerade als wenn sie dadurch, daß sie
sich zum Christenthume bekenneten, nur desto schlimmer geworden wären.

\medskip

\textbf{O des bejammernswerthen Zustandes der vergeblichen Christen! die, anstatt daß
sie, der Lehre Christi und seiner Apostel gemäß,
\textit{"`ihre Feinde lieben und
Diejenigen, die ihnen fluchen, segnen sollten,"'}
die Menschen lehren, unter dem
Vorwande eines christlichen Eifers, einander auf die unmenschlichste Weise zu
zerfleischen und umzubringen\index{Auseinandersetzungen, blutige}; die, anstatt daß sie bereit sein sollten, ihr
eigenes Blut für das Zeugniß Jesu vergießen zu lassen, das Blut der Zeugen Jesu,
als vorgeblicher Ketzer\index{Ketzer}, vergießen.} So hat also die listige Schlange, jener
schlaue böse Geist, der mit seinen Versuchungen Adam\index{Personen:!Adam} um seine Unschuld
brachte, und die Juden\index{Personen:!Juden} verführte, von dem Gesetze Gottes\index{Gesetz Gottes} abzuweichen, nun
auch durch eitle Lügen die Christen betrogen\index{Betrug der Christen} und verleitet, das heilige
Gesetz des Geistes Christi in ihren Herzen\index{Gesetz des Herzens} zu verlassen, und \textit{Seine}
Sklaven\index{Sklaverei} zu werden,
\textit{"`der in den Herzen der Kinder des Ungehorsams
herrscht."'}
\footnote{Epheser 2,2}
\index{Bibelstellen:!Epheser 2)}

\section{15. Abschnitt} \label{kap7_ab15}

Es ist bemerkenswerth, daß, so wie der Stolz\index{Stolz}, der immer von Aberglauben\index{Aberglauben} und
Starrsinn\index{Starrsinn} begleitet ist, Adam\index{Personen:!Adam} verleitete, sich über den Stand, in welchen
Gott ihn gesetzt hatte, erheben zu wollen, und daß, so wie die Juden\index{Personen:!Juden}, durch
denselben Stolz angetrieben, das ihnen von Gott durch Moses\index{Personen:!Mose} auf dem Berge
gegebene Vorbild zu übertreffen suchten, indem sie
\textit{"`ihre Pfosten an Gottes
Pfosten setzten, und ihre eigenen Menschengebote für wahre Lehren
ausgaben,"'}
\footnote{Ezech. 43,8. Jesaja 29,13. Matthäus 15,9.}
\index{Bibelstellen:!Ezech. 43)}
\index{Bibelstellen:!Jesaja 2)}
\index{Bibelstellen:!Matthäus 15)}
eben so auch die
abgewichenen Namenchristen\index{Namenchristen}, durch dieselbe Sünde des Stolzes verleitet, mit
starrem Aberglauben\index{Aberglaube} und kühner Anmaßung\index{Anmaßung}, statt der wahren geistlichen
Gottesverehrung und Kirchenzucht\index{Kirchenzucht} bloß weltliche Zeremonien und Anordnungen
eingeführt, und mit solchen Neuerungen und überlieferten menschlichen Meinungen
vermischt haben, die, wie ihre zahlreichen Concilien\index{Concilien} und verwickelten
Glaubensartikel\index{Glaubensartikel} beweisen, offenbare \textit{"`Früchte der irdischen Weißheit"'} sind.

\section{16. Abschnitt} \label{kap7_ab16}

So wie nun dieser unverantwortliche Stolz sie zuerst verleitete, die geistige
Beschaffenheit der christlichen Gottesverehrung\index{Gottesverehrung} so sehr zu verkehren, daß sie
eher der bildlichen Verehrungsart\index{Verehrung, bildliche} der Juden und dem pomphaften Gottesdienste
der Egypter\index{Personen:!Egypter} ähnlich war, als mit der schlichten Einfachheit\index{Einfachheit} der christlichen
Anordnung übereinstimmte, -- welche weder der Anbetung auf dem Berge noch der zu
Jerusalem\index{Orte:!Jerusalem} gleichen sollte --; so trieb auch derselbe Stolz, dieselbe
Anmaßung sie an, den Ruf ihrer großen Diana\index{Personen:!Diana} durch alle nur erdenkliche
Mittel der Grausamkeit zu behaupten und aufrecht zu erhalten. Weder die sanften
Bitten, noch die demüthigen Vorstellungen von Seiten Derer, die sich genau an
die ursprüngliche Reinheit der christlichen Lehre\index{Lehre, christlichen} und Gottesverehrung hielten,
konnten diese hartherzigen Namenchristen\index{Namenchristen} bewegen, sie von der Befolgung ihrer
unapostolischen Überlieferungen\index{unapostolischen Überlieferungen} auszunehmen oder zu befreien. So wie die Lehrer
und Bischöfe\index{Bischöfe} dieser Ausgearteten anfingen, sich um die mühsame Untersuchung und
Pflege der Heerde Christi nicht mehr zu bekümmern, wurden sie ehrgeizig,
habssücbtig und verschwenderisch\footnote{"`üppig"' ersetzt durch "`verschwenderisch"'}, so daß sie in ihrem Betragen eher stolzen, weltlichen
Herrschern, als demüthigen, sich selbst überwindene\footnote{"`verleugnenden"' ersetzt durch "`überwindene"'} Jüngern Jesu glichen
Und die Geschichte fast eines jeden Landes sagt uns, mit welchem Stolze, mit
welcher Grausamkeit\index{Grausamkeit}, und mit was für einem Blutdurste, -- der sie die
ungewohntesten Martern und seltensten Foltern erfinden\index{Foltern erfinden} ließ --, sie dies
heiligen Glieder Christi\index{Glieder Christi} verfolgt\index{Verfolgung} und aus der Welt geschafft, und noch
überdies mit so furchtbaren Bannflüchen\index{Bannflüche} belegt haben, daß sie dieselben,
insofern es ihnen möglich gewesen wäre, auch der himmlischen Seligkeit\index{Seligkeit rauben} beraubt
haben würden.

\medskip

Solche Schlachtopfer\index{Schlachtopfer} wurden immer von den wahren Christen \textit{Märtyrer}\index{Märtyrer} genannt;
aber die falsch sogenannte Geistlichkeit\index{Geistlichkeit, falsche} hat ihnen, den Verfolgungssüchtigen\index{Verfolgun}
Juden\index{Personen:!Juden} nachahmend, den Namen der \textit{Gotteslästerer}\index{Gotteslästerer} und \textit{Ketzer}\index{Ketzer} beigelegt. \index{Verfolung der Rechtgläubigen}
\textbf{So haben diese Abtrünnigen\index{Abtrünnige} die Vorhersagung unsers Herrn Jesu Christi erfüllt, der nicht sagte, daß die Menschen glauben würden, den Göttern einen Dienst zu tun, wenn sie seine theuern Nachfolger, die wahren Christen, tödteten
-- welches sich auf die Verfolgung der Christen von Seiten der abgöttischen
Heiden\index{Heiden} beziehen könnte --; sondern, daß sie meinen würden, Gott einen Dienst
damit zu tun;}
\footnote{Johannes 16, 2.}
\index{Bibelstellen:!Johannes 16)}
woraus klar hervorgehet, daß es von solchen
geschehen würde, die sich zu dem wahren Gott bekennen, wie auch allezeit die
abtrünnigen Christen vergeblich getan haben. Diese müssen also, ohne Zweifel,
jene Wölfe sein, von denen der Apostel vorhersagte,
\textit{"`daß sie unter die
Gläubigen kommen und der Heerde nicht verschonen würden,"'}
\footnote{Apostelgeschichte 20,29.}
\index{Bibelstellen:!Apostelgeschichte 20)}
\textbf{wenn der von ihm angekündigte große Abfall seinen Anfang nähme, der
nothwendig kommen müßte, damit der Glaube und die Treue der Rechtschaffenen\index{Rechtschaffenen}
geprüft und; das große Geheimniß der Bosheit offenbar gemacht würde.}

\medskip

Ich schließe diesen Abschnitt mit der Behauptung der ganz unverkennbaren
Wahrheit\index{Wahrheit}, daß überall, wo die ausgeartete Klerisei\index{Klerisei}\index{Klerus}, oder sogenannte
Geistlichkeit\index{Geistlichkeit, die}\index{Vervolgung durch Klerus}\index{Verbrechen des Klerus}\index{Geschichte des Klerus}, am Meisten Gewalt und Ansehen besaß und aus Fürsten und
Staaten den größten Einfluß hatte, auch immer die größten Verwirrungen,
Streitigkeiten, Bludfeden\footnote{"`Blutscenen"' ersetzt durch "`Bludfeden"'}, Einziehungen der Güter, Einkerkerungen und
Verbannungen\index{Verbannungen} statt fanden; und zur Rechtfertigung dieser Wahrheit berufe ich
mich auf die Geschichte aller Zeiten. Wie es in unserm Zeitalter hiermit stehet,
überlasse ich den Leben den aus eigener Erfahrung und Beobachtung zu
beurtheilen. Nur einige Tatsachen liegen und so klar vor Augen, daß sie unserer
Bemerkung schwerlich entgehen können: Die Menschheit ist nicht bekehrt\index{Bekehrung}, sondern
in einen so hohen Grade, auf eine so verfeinerte Art verderbt, daß die
Geschichte voriger Zeiten uns kein Beispiel davon liefert. Die Gottesverehrung
der Christen ist, augenscheinlich, in Zeremonie und Pomp ausgeartet. Der
geistliche Stand sucht, größtenteils unter dem Vorwande der geistlichen
Beförderung, nur weltliche Vortheile; indem gewöhnlich bei Denen, die sich
demselben widmen, die Aussicht zu einer gemächlichen zeitlichen Versorgung die
Haupttriebfeder in der Bestimmung ihrer Wahl ist, und fast Jeder von ihnen sich
bereit findet, seine gegenwärtige Stelle zu verlassen und um eine andere zu
werben, sobald diese ihm einen höhern Rang und reichlichere Einkünfte darbietet.
Es muß uns also klar einleuchten, daß Stolz und Gier\index{"`Geiz"' ersetzt durch "`gier"'} die beiden unglücklichen
Leidenschaften sind, von denen der gute fromme Petrus\index{Personen:!Petrus} wohl voraussah, daß sie
ihnen zu Fallstricken werden würden, und durch welche sie so viel Unwissenheit,
Irrthum und Religionsverachtung in der Christenheit,erzeugt haben.

\section{17. Abschnitt} \label{kap7_ab17}

\textbf{Das Mittel nun, aus dieser unglücklichen Abweichung und Verirrung zurückkommen,
ist kein anderes, als daß man sich eine lebendige, seligmachende Erkenntniß der
Religion Jesu, nämlich eine innere Erfahrung\index{Erfahrung, innere} von dem Werke Gottes in der Seele,
erwerbe. Diese zu erlangen, mußt du, o Mensch! fleißig und sorgfältig auf die \index{Heimsuchung}
"`Erscheinung der heilbringenden Gnade"'
\footnote{Titus 2,11+12}
\index{Bibelstellen:!Titus 2)}
in deiner Seele
achten, und derselben Gehorsam leisten.} Sie wird dich von der breiten Straße auf
einen schmalen Weg leiten, von der Befriedigung deiner verderbten Neigungen dich
zurückhalten und zur Erfüllung deiner Pflichten\index{Pflichten} dich antreiben. Sie wird dich
von den sündlichen Vergnügungen der Welt\index{Vergnügungen der Welt} zu einem heiligen Leben\index{Leben, heiliges} der
Selbstüberwindung\footnote{"`Selbstverleugnung"' durch "`Selbstüberwindung"' ersetzt}, von der Gewalt des Satans\index{Personen:!Satan}, des argen Feindes deiner
Glückseligkeit\index{Glückseligkeit}, zu Gott bekehren. Aber du mußt dein eigenes Leben, das Leben
deiner Eigenliebe\index{Eigenliebe, hassen}, einsehen und hasse;
\footnote{Lukas 14,26) Johannes 12,25) Matthäus 10,39}
\index{Bibelstellen:!Lukas 14)}
\index{Bibelstellen:!Johannes 12)}
\index{Bibelstellen:!Matthäus 10)}
du mußt wachen und beten\index{beten}, und auch fasten\index{fasten}; nicht auf deinen Versucher\index{Personen:!Versucher, Der},
sondern auf deinen Helfer und Erlöser\index{Personen:!Erlöser, Der} sehen, böse Gesellschaft\index{Gesellschaft, böse} meiden, dich oft
in die Einsamkeit\index{Einsamkeit} zurückziehen, und wie ein reiner Pilger\index{} in dieser bösen Welt\index{Welt, Die böse}
leben. Auf diese Weise wirst du zu der wahren Erkenntniß Gottes\index{Erkenntniß Gottes} und Jesu Christi
gelangen, die deiner Seele ewiges Leben\index{ewiges Leben}; eine wohlgegründete Überzeugung von
dem, was sie selbst davon empfindet und erfährt, mit unzweifelhafter Gewißheit
ertheilet. Die Herzen Derer, die zu dieser Erfahrung gelangen, sind
\textit{"`getrost und fürchten sich nicht."'}
\footnote{Psalm 112,7+8}
\index{Bibelstellen:!Psalm 112)}
