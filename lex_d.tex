\section*{D}

\articlesize

\begin{description}

 \item[dark countries] Auf grund schlechter Erfahrungen (siehe: $\to$\textit{John
 'Love' Luffe}) mieden die frühen Quakermissionare katholische länder und nannten sie
 \textit{dark countries}. \textit{Dunkelheit} stand immer metaphorisch für dia
 abwesenheit von Gotteswirken. Menschen, die einen schlechten Lebenswandel führten,
 wurde oft vorgeworfen "`der finsternis verfallen"' zu sein. \endnote{Seite 74, in
 "`Über Gott und die Welt: Endzeitvisionen, Reformdebatten und die europäische
 Quäkermission in der Frühen Neuzeit"',
Band 143 von Veröffentlichungen des Max-Planck-Instituts für Geschichte, Max-Planck-
Institut für Geschichte (Göttingen), Autorin: Sünne Juterczenka, Verlag:
Vandenhoeck~\&~Ruprecht, 2008, ISBN: 3525354584 und 9783525354582, \\
\texttt{http://books.google.com/books?id=aP00obJjhT8C} }

\item[Datum-Format] Im frühen Quakertum wurden die Monatsbezeichnungen
(\textit{Januar, Februar, März}...) und die Wochentage (\textit{Sonntag,
Montag, Dienstag}...) als heitnische Begriffe abgelehnt. Statdessen wurde zum
Beispiel gesat "`\textit{7.3.1680}"' stat "`\textit{7. März 1680}"'. Für
Sonntag wurde zum Beispiel "`Ersttag"' und für Mittwoch "`Vierttag"' gesagt.
Einige wenige konservative Quaker tun das heute noch. Mit heitnischen Vornamen
hatten die frühen Quaker übrigens keine Probleme.


\item[Deutsche Jahresversammlung] In den 1920er Jahren gab es zwei Deutsche
Jahresversammlungen. Eine er konservative ($\to$\textit{Conservative Friends} mit
Schwerpunkt Süddeutschland, und eine liberal bürgerliche ($\to$\textit{Liberales
Quäkertum}) mit Berlin, als geograpisches Zentrum. Heute existiert nur noch die
liberal-bürgerliche Deutsche Jahresversammlung. Die Deutsche Jahresversammlung hatte
nie mehr als unter 600 Mitglieder. Heute etwa 250. Die Deutsche Jahresversammlung hat
zwei besonderheiten. 1.) Betreut sie keine selbständigen
$\to$\textit{Monatsversammlungen} sondern Einzelmitglieder. 2.) Ist sie die
Jahresversammlung von zwei Ländern: Deutschland und Östereich. Also müsste sie
eigendlich \textit{"`Deutsch-Östereicher Jahresversammlung"'} heißen.
\endnote{Siehe hierzu den Abschnitt "`Albrecht, Hans"' in "`Quäker aus Politik,
Wissenschaft und Kunst: Ein biographisches Lexikon"' Auflage 2, Seite 17-19, ISBN
978-3883094694. Zitat: "`Hans Albrecht ist ein bedeutender Mitbegründer der Deutschen
Jahresversammlung. Er verkörperte den bürgerlichen Flügel der Quäker."'(S.18). Und im
Kapitel "`Stackelberg, Freiherr Traugott von (1891-1970)"' auf Seite 195 Zitat "`[...]
Hinderungsgründe schienen ihnen [den Stackelbergs] vor allem die überwigend bürgerlich
Sozialorientierung der Quäker in Deutschland, [...] und ihre zu intellektuell
ausgeprägte Einstellung."'}
\endnote{Zu den Mitgliderzahlen siehe, Claus Bernet, Leben zwischen evangelischer
Theologie und Quäkertum, aus dem Heft Materialdienst, 02/2008, ISSN 0934-8522,
Seite~29}


\item[Diggers] (Zu Deutsch "`Buddler"') also ein Spottname für eine Splittergruppe der
$\to$\textit{Levellers}. Ihr Gründer war $\to$\textit{Gerrard Winstanley}. Selber
nannte sich die Gruppe "`True Levellers"'. Also "`wahre Gleichmacher"'. Zwischen 1649
und 1651 existierten  einige Gruppen von ihnen, die öffentliches Land besetzten und Gütergemeinschaft lebten. Vorbild war die Urgemeinde der Aphostelgeschichte.\endnote{Seite „Diggers“. In: Wikipedia, Die freie Enzyklopädie. Bearbeitungsstand:
2. Januar 2010, 22:42 UTC. URL: \\
\texttt{http://de.wikipedia.org/w/index.php?title=Diggers\&oldid=68745857}
(Abgerufen: 2. Januar 2010, 22:46 UTC) }

\item[Dissenter] Überbegriff von einer Reihe von Gruppen die während oder kurz
nach dem englischen Bürgerkrig endstanden, und sich als Gegenbewegung zu
etablierten Konfesionen, wie der römisch-katholischen Kirche oder den Anglikanern
verstanden. Hier einige Beispiele:

\begin{itemize}
 \item \textit{Adamites}
 \item \textit{Anabaptists}
 \item \textit{Baptisten}~*
 \item \textit{Barrowists}
 \item \textit{Behmenists}
 \item \textit{Brownists}
 \item \textit{Congregationalists}~*
 \item \textit{Countess of Huntingdon's Connexion}~*
 \item \textit{Enthusiasts}
 \item \textit{Familists}
 \item \textit{Fifth Monarchists}
 \item \textit{Grindletonians}
 \item \textit{Muggletonians}
 \item \textit{Kongregationalisten}
 \item \textit{Levellers}
 \item \textit{Presbyterianer}~*
 \item \textit{Puritans}
 \item \textit{Philadelphians}
 \item \textit{Sabbatarians}
 \item \textit{Seekers}
 \item \textit{Socinians}
 \item \textit{Renters} bzw. \textit{Renterismus}
 \item \textit{True Levellers} bzw. \textit{Diggers}
 \item \textit{Unitarians}~*
\end{itemize}
*Existieren noch Heute

 \item[Doppelmitgliedschaft] Gemeint ist die Mitgliedschaft in Meheren Gemeinschaften
 oder Kirchen gleichzeitig. Das wahr in frühen Quakertum unmöglich. Selbst das
 Heiraten
 ausserhalb der Gemeinschaft war absolute Ausname und wurde nicht gerne gesehen. Im
 Quakertum des 20. Jahrhundert in Deutschland wahren aber die meisten Quaker der 2.
 Deutschen Jahresversammlung (also die von 1925 mit dem Zentrum in Berlin) Mitglieder
 einer Weitern meist evangelischen oder katholischen Landeskirche. Das war eine
 Konzesion an das vorwiegend bürgerliche Specktrum. Bis Heute ist aber eine
 Doppelmitgliedschaft in den meisten Jahresversammlungen auf der Welt unüblich.
 \endnote{Seite "`Deutsche Jahresversammlung"'. In: Wikipedia, Die freie Enzyklopädie.
 Bearbeitungsstand: 24. Dezember 2009, 10:46 UTC. URL: \\
 \texttt{http://de.wikipedia.org/w/index.php?title=\\
 $\hookrightarrow$~Deutsche\_Jahresversammlung\&oldid=68383041}
\\ (Abgerufen: 5. Januar 2010, 19:06 UTC) }


 \end{description}

\normalsize