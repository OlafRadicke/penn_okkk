\chapter{Temenübersicht}


Der Umgang mit dem Inhaltsverzeichnis ist im Original etwas eigenwillig. Deshalb
nehme ich mir hier die Freiheit, hier noch mal eine eigene Übersicht für den
Leser zusammenzustellen.

\begin{enumerate}
 \item Der Sinn des Lebens
 \begin{flushright}...Seite \pageref{ref:vorw_sinndes_lebens}\end{flushright}
 \item Was ist über die Jahunderte aus dem Christentum geworden?
 \begin{flushright}...Seite \pageref{ref:01_02_urchristentum}\end{flushright}
 \item Die Juden verfolgten Jesus nur wenige Jahre. Die falschen Christen aber
bis zum heutigen Tage.
 \begin{flushright}...Seite
\pageref{ref:01_04_zweite_kreuzigung}\end{flushright}
 \item In der Sünde ist keiner besser als der Andere. Ob sie sich zum
Christentum bekennen oder nicht.
 \begin{flushright}...Seite \pageref{ref:01_05_in_suende_gleich}\end{flushright}
 \item Sie nennen sich Christen, und paktieren mit Bösen Menschen.
 \begin{flushright}...Seite
\pageref{ref:01_05_in_suende_verbinden}\end{flushright}
 \item Sie beweiräuchern und betrügen sich selbst, bis der Tod sie aus ihrem
Schlaf reissen wird.
 \begin{flushright}...Seite \pageref{ref:01_07_selbstbetrug}\end{flushright}
 \item Die Erlösung geschied durch Jesus Christus
 \begin{flushright}...Seite
\pageref{ref:01_10_jesus_der_erloeser}\end{flushright}
 \item Gott will niemand mit dem \textit{ewigen Tod} bestraffen müssen
 \begin{flushright}...Seite
\pageref{ref:02_02_kein_gefallen_am_tod}\end{flushright}
 \item Christus ist das Licht der Welt, das jedem Menschen erleuchtet, und ihm
seine Sünden vorhält.
 \begin{flushright}...Seite \pageref{ref:02_03_gott_in_jedem}\end{flushright}
 \item Die Christentum von Heute sind schlimmer als die Heiden und verdienen den
Namen \textit{Christ} nicht.
 \begin{flushright}...Seite
\pageref{ref:02_08_schlimmer_als_heiden}\end{flushright}
 \item Der schlimme Zustand der Christen rührt daher, das die nicht mehr auf das
Licht Christe achten.
 \begin{flushright}...Seite
\pageref{ref:02_08_ursache_des_abfalls}\end{flushright}
 \item Statt Gott im Geist an zu beten, entartet der Gottesdienst in pompösen
Zeromonien.
 \begin{flushright}...Seite \pageref{ref:02_08_falscher_gottesdienst}
\end{flushright}
 \item Statt lebendiger Erfahrung, besteht der Gottesdienst nur noch aus leeren
Zeromoniehen.
 \begin{flushright}...Seite
\pageref{ref:02_09_lebendige_erfahrung}\end{flushright}
 \item Durch die leeren Zeromoniehen wird alles noch schlimmer, weil sich die
Leute auch noch einbilden auf Grund ihrers Zeromonien etwas Besseres zu sein.
 \begin{flushright}...Seite
\pageref{ref:02_09_selbstueberschaetzung}\end{flushright}
 \item Der Weg zur wiederherstellung des wahren Christentums und zur Erlösung.
 \begin{flushright}...Seite \pageref{ref:02_10_abhilfe}\end{flushright}
 \item Das \textit{Kreuz} ist Bildlich zu verstehen.
 \begin{flushright}...Seite \pageref{ref:03_01_das_kreuz}\end{flushright}
 \item Die, die sich nicht beherschen wollen, halten das Christentum für
unsinnig.
 \begin{flushright}...Seite \pageref{ref:03_02_leidenschaft}\end{flushright}
 \item Das Kreuz ist dort, wo auch die Sünde ist. Im Herzen des Menschen.
 \begin{flushright}...Seite \pageref{ref:03_03_ort_des_kreuzes}\end{flushright}
 \item Das \textit{Reich des Bösenen} ist da, wo der Mensch in seinem Herz Raum
gibt für das Böse.
 \begin{flushright}...Seite \pageref{ref:03_04_reich_des_boesen}\end{flushright}
 \item Sich dem Willen Gottes zu unterwerffen, heißt das Kreuz auf sich zu
nehmen.
 \begin{flushright}...Seite
\pageref{ref:03_05_kreuz_auf_sich_nehmen}\end{flushright}
 \item Über die Bedeutung des Kelches, der Taufe und des Kreuzes.
 \begin{flushright}...Seite
\pageref{ref:04_04_kelch_taufe_kreuz}\end{flushright}
 \item Wahre Christen besigen ihren eigenen Egoismus und nicht andere Menschen
und Völker.
 \begin{flushright}...Seite \pageref{ref:04_05_besigen}\end{flushright}
 \item Welche Dinge ein Christ erlaubterweise genisen darf und wie er sie
genisen darf.
 \begin{flushright}...Seite \pageref{ref:04_07_vorteile}\end{flushright}
 \item Die Besitztümer die ein Christ hat, muss er immer nur als \textit{Mittlel
zum Zweck} betrachten. Wenn er sein Besitz anbetet, ist er Gott gegenüber
untreu.
 \begin{flushright}...Seite \pageref{ref:04_07_vorteile}\end{flushright}
 \item Wer nicht an seinen Besitztümern hängt, ist ein geistiger Pilger. Die
Materialisten, die auf diese Welt bauen, werden keine Wohnstätte im Himmel
haben.
 \begin{flushright}...Seite \pageref{ref:04_10_pilger} --
\pageref{ref:04_10_pilger_ende}\end{flushright}
 \item Die Tugenden die bei Jesus von den Christen bewundert wird, wird bei den
Menschen die sie Heute zeigen von den gleichen Leuten verlacht und verspottet.
 \begin{flushright}...Seite \pageref{ref:04_10_pilger}\end{flushright}
 \item Nicht ein Opfer ansich gefällt Gott, sondern die Haltung des Herzens, das
ein Opfer bring.
 \begin{flushright}...Seite \pageref{ref:04_13_opfer}\end{flushright}
 \item Weltliche Christen haben nicht nur nichts vom Christentum verstanden,
sondern sie sind auch noch Feinde des wahren Christentums.
 \begin{flushright}...Seite
\pageref{ref:04_13_weltliche_christen}\end{flushright}
 \item Für Weltliche Christen müsste die Entscheidung von Mose eigendlich
unverständlich bleigen. Die Weltlichen Christen währen in Ägypten geblieben und
nicht mit dem Jüdischen Volk ausgezogen.
 \begin{flushright}...Seite \pageref{ref:04_17_mose}\end{flushright}
 \item So wie alle grossen Gestalten in der Bibel sich in Selbstüberzindung
übten, so müssen wir es auch. Jesus ist für uns am Kreuz gestorben, um für uns
ein Vorbild zu sein. Aber nicht -- wie es einige Weltlichen Christen gerne
hätten -- um statt unserer zu leiden, so das wir es nicht mehr müssten!
 \begin{flushright}...Seite \pageref{ref:04_20_opfertod}\end{flushright}
 \item Jeder Mensch wird am Ende seines Lebens danach gerichtet werden, wie er
gelebt hat und nicht danach, zu was er sich nur mit dem Lippen bekannt hat.
 \begin{flushright}...Seite \pageref{ref:04_21_gericht}\end{flushright}
 \item Der menschliche Versatand kann Gottes Führung nicht ersetzen.
 \begin{flushright}...Seite \pageref{ref:04_22_vernunft}\end{flushright}
 \item Wenn man aus Egoismus herraus mit dem blossen Verstand entscheidet, wir
es gefährlich.
 \begin{flushright}...Seite \pageref{ref:04_22_vernunft_und_ego}\end{flushright}
 \item Die Art, mit der mit der \textit{Inneren Stimme} umgegangen werden
sollte.
 \begin{flushright}...Seite \pageref{ref:04_23_innere_stimme} --
\pageref{ref:04_23_innere_stimme_ende}\end{flushright}
 \item Die weltlichen Dinge sollte man gerig schätzen.
 \begin{flushright}...Seite \pageref{ref:04_23_dinge_der_welt}\end{flushright}
 \item Die Christen tun im Gottesdienst so, als würden sie ein
Unterhaltunsprogramm
 für einen Menschen veranstalten. Quasie ein göttliches Kasperletheater.
 \begin{flushright}...Seite \pageref{ref:05_02_kasperletheater}\end{flushright}
 \item Die wahre Anbetung ist die im Geiste.
 \begin{flushright}...Seite \pageref{ref:05_04_wahre_anbetung}\end{flushright}
 \item Die weltlichen Christen sind keine wahren Nachvolger. Egal wie sehr sie
Gott, Jesus und die Bibel loben sollten.
 \begin{flushright}...Seite \pageref{ref:05_05_wahre_nachfolge}\end{flushright}
 \item So wie der Mensch der Tempel Gottes ist -- und nicht der äussere Tempel
--
 so ist auch keine äussere Kirche, die Wohnstätte Gottes, sondern das innere des
Menschen.
 \begin{flushright}...Seite \pageref{ref:05_06_tempel}\end{flushright}
 \item Die weltlichen Christen kaben die Bedeutung des Kreuzes in das Gegenteil
verkehrt.
 \begin{flushright}...Seite \pageref{ref:05_07_kreuz}\end{flushright}
 \item Das tragen eines Schmuckkreuzes um den Hals, mach niemanden zu einem
besseren Menschen.
 \begin{flushright}...Seite \pageref{ref:05_08_kreuz}\end{flushright}
 \item Da ihr Kreuz nur ein totes Stück Schmuck ist, können sie noch soviel
damit schmücken und es wird keine Wirkung haben.
 \begin{flushright}...Seite \pageref{ref:05_09_kreuz}\end{flushright}
 \item Klöster macht Menscher er dumm, als tugendhaft.Wahre Christen haben ihre
Klsterzelle in sich und brauchen der Welt nicht zu entfliehen.
 \begin{flushright}...Seite \pageref{ref:05_11_kloster} --
\pageref{ref:05_11_kloster_ende}\end{flushright}
 \item Zurückgezogenheit an sich, ist aber eine gute Sache.
 \begin{flushright}...Seite \pageref{ref:05_13_zurueckgezogenheit} --
\pageref{ref:05_13_zurueckgezogenheit_ende}\end{flushright}
 \item Der richtige Gottesdienst brauch die innere Vorbereitung des Heiligen
Geistes.
 \begin{flushright}...Seite
\pageref{ref:06_03_gottesdiensvorbereitung}\end{flushright}
 \item Gebete dürfen nicht ausgedacht sein, sonndern müssen von einer Eingebung
kommen.
 \begin{flushright}...Seite \pageref{ref:06_04_gebet}\end{flushright}
 \item Die innere Haltung, die für ein Gebet nötig ist.
 \begin{flushright}...Seite \pageref{ref:06_04_gebetshaltung}\end{flushright}
 \item Über die sinnlosigkeit eines runtergebrabbelten
\textit{Vaterunser-Gebet}.
 \begin{flushright}...Seite \pageref{ref:06_07_sinnlose_gebete}\end{flushright}
 \item Wer sich moralisch verunreinigt, der kann nicht erwahrten das sein
Gotesdiens wohlgefallen bei Gott finden wird.
 \begin{flushright}...Seite
\pageref{ref:06_08_moralisch_verunreinigt}\end{flushright}
 \item Auch wenn es die Gebote von Gott selbst sind, die befolgt werden, nützt
das nichts,.wenn die Seele mit Sünde verunreinigt ist.
 \begin{flushright}...Seite \pageref{ref:06_09_gottesregeln}\end{flushright}
 \item Gebete von unaufrichtigen Menschen finden kein geöhr bei Gott
 \begin{flushright}...Seite \pageref{ref:06_09_gebetserhoerung}\end{flushright}
 \item Die Gemeinschaft der Gläubigen ist die wahre Kirche Gottes.
 \begin{flushright}...Seite \pageref{ref:06_12_wahre_kirche}\end{flushright}
 \item Die vier nötigen Dinge für den Gottesdienst.
 \begin{flushright}...Seite
\pageref{ref:06_13_vier_noetige_dinge}\end{flushright}
 \item Von Gott geprüft zu werden ist ein Unglöck.
 \begin{flushright}...Seite
\pageref{ref:06_13_auf_die_probe_gestellt}\end{flushright}
 \item Ohne Glauben ist richtiges predigen nicht möglich.
 \begin{flushright}...Seite \pageref{ref:06_14_predigt}\end{flushright}
 \item Über den Glauben und die Erfüllung von Gebeten.
 \begin{flushright}...Seite \pageref{ref:06_14_gebetserfuellung}\end{flushright}
 \item Die drei Haupeigenschaften, aus den sich alle anderen bösen Neigungen
ableiten, sind: Stolze, Gier und Verschwendung.
 \begin{flushright}...Seite
\pageref{ref:07_01_drei_haupteigenschaften}\end{flushright}
 \item Das streben nach Wissen, war die Ursache für den Sündenfall. 
 \begin{flushright}...Seite \pageref{ref:07_03_wissen_erbsuende}\end{flushright}
 \item Die Wissenschaft ist nur matter Abglanz der Weisheit Gottes.
 \begin{flushright}...Seite \pageref{ref:07_03_wissen_erbsuende}\end{flushright}
 \item Der Wissenschaftsbetrieb nach arogant und missgünstig.
 \begin{flushright}...Seite \pageref{ref:07_06_wissenschaft}\end{flushright}
 \item Gelehrten Menschen bleibt das Evangelum verborgen.
 \begin{flushright}...Seite \pageref{ref:07_13_gelehrte}\end{flushright}
 \item Der Vortschritt hat nichts gutes gebracht. Im Gegenteil.
 \begin{flushright}...Seite \pageref{ref:07_14_vortschritt}\end{flushright}
 \item Unter dem Vorwand des christlichen Eifer, wird die Lehre Christi
missachtet, und blutig andere Christen unter dem Vorwurf der Ketzerei verfolgt.
 \begin{flushright}...Seite \pageref{ref:07_14_ketzer}\end{flushright}
 \item Durch die Erduldung der Verfolgung duch falsche Christen, würden die
wahren Christen, ihre echte Nachvolge unter Beweis stellen.
 \begin{flushright}...Seite \pageref{ref:07_16_vervolgung}\end{flushright}
%  \item
%  \begin{flushright}...Seite \pageref{}\end{flushright}
%  \item
%  \begin{flushright}...Seite \pageref{}\end{flushright}
%  \item
%  \begin{flushright}...Seite \pageref{}\end{flushright}
 
%  \item 
%  \begin{flushright}...Seite \pageref{}\end{flushright}
\end{enumerate}