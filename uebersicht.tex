\chapter{Themenübersicht} \label{ref:theme_nuebersicht}


Der Umgang mit dem Inhaltsverzeichnis ist im Original etwas eigenwillig. Deshalb
nehme ich mir hier die Freiheit, hier nochmal eine eigene Übersicht für den
Leser zusammenzustellen.

\begin{enumerate}
 \item Der Sinn des Lebens
 \dotfill \textit{Seite~\pageref{ref:vorw_sinndes_lebens}}\\

 \item Was ist über die Jahrhunderte aus dem Christentum geworden?
 \dotfill \textit{Seite~\pageref{ref:01_02_urchristentum}}\\

 \item Die Juden verfolgten Jesus nur wenige Jahre. Die falschen Christen aber
bis zum heutigen Tage.
 \dotfill \textit{Seite~\pageref{ref:01_04_zweite_kreuzigung}}\\

 \item In der Sünde ist keiner besser als die Anderen. Ob sie sich zum
Christentum bekennen oder nicht.
 \dotfill \textit{Seite~\pageref{ref:01_05_in_suende_gleich}}\\

 \item Sie nennen sich Christen, und paktieren mit Bösen Menschen.
 \dotfill \textit{Seite
\pageref{ref:01_05_in_suende_verbinden}}\\

 \item Sie beweihräuchern und betrügen sich selbst, bis der Tod sie aus ihrem
Schlaf reissen wird.
 \dotfill \textit{Seite~\pageref{ref:01_07_selbstbetrug}}\\

 \item Die Erlösung geschieht durch Jesus Christus
 \dotfill \textit{Seite
\pageref{ref:01_10_jesus_der_erloeser}}\\

 \item Gott will niemanden mit dem \textit{ewigen Tod} bestrafen müssen
 \dotfill \textit{Seite
\pageref{ref:02_02_kein_gefallen_am_tod}}\\

 \item Christus ist das Licht der Welt, das jeden Menschen erleuchtet, und ihm
seine Sünden vorhält.
 \dotfill \textit{Seite~\pageref{ref:02_03_gott_in_jedem}}\\

 \item Die Christen von Heute sind schlimmer als die Heiden und verdienen den
Namen \textit{Christ} nicht.
 \dotfill \textit{Seite
\pageref{ref:02_08_schlimmer_als_heiden}}\\

 \item Der schlimme Zustand der Christen rührt daher, dass sie nicht mehr auf das
Licht Christi achten.
 \dotfill \textit{Seite
\pageref{ref:02_08_ursache_des_abfalls}}\\
 \item Statt Gott im Geist anzubeten, entartet der Gottesdienst in pompösen
Zeremonien.
 \dotfill \textit{Seite~\pageref{ref:02_08_falscher_gottesdienst}}\\

 \item Statt lebendiger Erfahrung, besteht der Gottesdienst nur noch aus leeren
Zeremonien. \dotfill \textit{Seite~\pageref{ref:02_09_lebendige_erfahrung}}\\

 \item Durch die leeren Zeremonien wird alles noch schlimmer, weil sich die
Leute auch noch einbilden, aufgrund ihrer Zeremonien etwas Besseres zu sein.
 \dotfill \textit{Seite
\pageref{ref:02_09_selbstueberschaetzung}}\\

 \item Der Weg zur Wiederherstellung des wahren Christentums und zur Erlösung.
 \dotfill \textit{Seite~\pageref{ref:02_10_abhilfe}}\\

 \item Das \textit{Kreuz} ist bildlich zu verstehen.
 \dotfill \textit{Seite~\pageref{ref:03_01_das_kreuz}}\\

 \item Die, die sich nicht beherrschen wollen, halten das Christentum für
unsinnig.
 \dotfill \textit{Seite~\pageref{ref:03_02_leidenschaft}}\\

 \item Das Kreuz ist dort, wo auch die Sünde ist. Im Herzen des Menschen.
 \dotfill \textit{Seite~\pageref{ref:03_03_ort_des_kreuzes}}\\

 \item Das \textit{Reich des Bösen} ist da, wo der Mensch in seinem Herzen Raum
gibt für das Böse.
 \dotfill \textit{Seite~\pageref{ref:03_04_reich_des_boesen}}\\

 \item Sich dem Willen Gottes zu unterwerfen, heißt das Kreuz auf sich zu
nehmen.
 \dotfill \textit{Seite
\pageref{ref:03_05_kreuz_auf_sich_nehmen}}\\

 \item Über die Bedeutung des Kelches, der Taufe und des Kreuzes.
 \dotfill \textit{Seite
\pageref{ref:04_04_kelch_taufe_kreuz}}\\

 \item Wahre Christen besiegen ihren eigenen Egoismus und nicht andere Menschen
und Völker.
 \dotfill \textit{Seite~\pageref{ref:04_05_besigen}}\\

 \item Welche Dinge ein Christ erlaubter Weise genießen darf und wie er sie
genießen darf.
 \dotfill \textit{Seite~\pageref{ref:04_07_vorteile}}\\

 \item Die Besitztümer die ein Christ hat, muss er immer nur als \textit{Mittel
zum Zweck} betrachten. Wenn er seinen Besitz anbetet, ist er Gott gegenüber
untreu.
 \dotfill \textit{Seite~\pageref{ref:04_07_vorteile}}\\

 \item Wer nicht an seinen Besitztümern hängt, ist ein geistiger Pilger. Die
Materialisten, die auf diese Welt bauen, werden keine Wohnstätte im Himmel
haben.
 \dotfill
 \textit{Seite~\pageref{ref:04_10_pilger}~--~\pageref{ref:04_10_pilger_ende}}\\

 \item Die Tugenden die bei Jesus von den Christen bewundert werden, werden bei
Menschen die sie heute zeigen, von den gleichen Leuten verlacht und verspottet.
 \dotfill \textit{Seite~\pageref{ref:04_10_pilger}}\\
 \item Nicht ein Opfer an sich gefällt Gott, sondern die Haltung des Herzens, das
ein Opfer bring.
 \dotfill \textit{Seite~\pageref{ref:04_13_opfer}}\\
 \item Weltliche Christen haben nicht nur Nichts vom Christentum verstanden,
sondern sie sind auch noch Feinde des wahren Christentums.
 \dotfill \textit{Seite
\pageref{ref:04_13_weltliche_christen}}\\

 \item Für weltliche Christen müsste die Entscheidung von Mose eigentlich
unverständlich bleiben. Die weltlichen Christen wären in Ägypten geblieben und
nicht mit dem Jüdischen Volk ausgezogen.
 \dotfill \textit{Seite~\pageref{ref:04_17_mose}}\\

 \item So wie alle großen Gestalten in der Bibel sich in Selbstüberwindung
übten, so müssen wir es auch tun. Jesus ist für uns am Kreuz gestorben, um für uns
ein Vorbild zu sein. Aber nicht -- wie es einige weltlichen Christen gerne
hätten -- um statt unser zu leiden, so dass wir es nicht mehr müssten!
 \dotfill \textit{Seite~\pageref{ref:04_20_opfertod}}\\

 \item Jeder Mensch wird am Ende seines Lebens danach gerichtet werden, wie er
gelebt hat und nicht danach, zu was er sich nur mit dem Lippen bekannt hat.
 \dotfill \textit{Seite~\pageref{ref:04_21_gericht}}\\

 \item Der menschliche Verstand kann Gottes Führung nicht ersetzen.
 \dotfill \textit{Seite~\pageref{ref:04_22_vernunft}}\\

 \item Wenn man aus Egoismus heraus mit dem bloßem Verstand entscheidet, wird
es gefährlich.
 \dotfill \textit{Seite~\pageref{ref:04_22_vernunft_und_ego}}\\

 \item Die Art, mit der mit der \textit{Inneren Stimme} umgegangen werden
sollte.
 \dotfill
 \textit{Seite~\pageref{ref:04_23_innere_stimme}~--~\pageref{ref:04_23_innere_stimme_ende}}\\

 \item Die weltlichen Dinge sollte man gering schätzen.
 \dotfill \textit{Seite~\pageref{ref:04_23_dinge_der_welt}}\\

 \item Die Christen tun im Gottesdienst so, als würden sie ein
Unterhaltungsprogramm
 für einen Menschen veranstalten. Quasi ein göttliches Kasperletheater.
 \dotfill \textit{Seite~\pageref{ref:05_02_kasperletheater}}\\

 \item Die wahre Anbetung ist die im Geiste.
 \dotfill \textit{Seite~\pageref{ref:05_04_wahre_anbetung}}\\

 \item Die weltlichen Christen sind keine wahren Nachfolger. Egal wie sehr sie
Gott, Jesus und die Bibel loben sollten.
 \dotfill \textit{Seite~\pageref{ref:05_05_wahre_nachfolge}}\\

 \item So wie der Mensch der Tempel Gottes ist -- und nicht der äußere Tempel
--
 so ist auch keine äußere Kirche die Wohnstätte Gottes, sondern das Innere des
Menschen.
 \dotfill \textit{Seite~\pageref{ref:05_06_tempel}}\\

 \item Die weltlichen Christen haben die Bedeutung des Kreuzes in das Gegenteil
verkehrt.
 \dotfill \textit{Seite~\pageref{ref:05_07_kreuz}}\\

 \item Das tragen eines Schmuckkreuzes um den Hals, macht niemanden zu einem
besseren Menschen.
 \dotfill \textit{Seite~\pageref{ref:05_08_kreuz}}\\

 \item Da ihr Kreuz nur ein totes Stück Schmuck ist, können sie noch soviel
damit schmücken und es wird keine Wirkung haben.
 \dotfill \textit{Seite~\pageref{ref:05_09_kreuz}}\\

 \item Klöster machen Menschen eher dumm als tugendhaft. Wahre Christen haben ihre
Klosterzelle in sich und brauchen der Welt nicht zu entfliehen.
 \dotfill \textit{Seite~\pageref{ref:05_11_kloster}~--~\pageref{ref:05_11_kloster_ende}}\\

 \item Zurückgezogenheit an sich ist aber eine gute Sache.
 \dotfill \textit{Seite~\pageref{ref:05_13_zurueckgezogenheit}~--~\pageref{ref:05_13_zurueckgezogenheit_ende}}\\

 \item Der richtige Gottesdienst braucht die innere Vorbereitung des Heiligen
Geistes.
 \dotfill \textit{Seite
\pageref{ref:06_03_gottesdiensvorbereitung}}\\

 \item Gebete dürfen nicht ausgedacht sein, sondern müssen von einer Eingebung
kommen.
 \dotfill \textit{Seite~\pageref{ref:06_04_gebet}}\\

 \item Die innere Haltung, die für ein Gebet nötig ist.
 \dotfill \textit{Seite~\pageref{ref:06_04_gebetshaltung}}\\

 \item Über die Sinnlosigkeit eines heruntergebrabbelten
\textit{Vaterunser-Gebetes}.
 \dotfill \textit{Seite~\pageref{ref:06_07_sinnlose_gebete}}\\

 \item Wer sich moralisch verunreinigt, der kann nicht erwarten, dass sein
Gottesdienst Wohlgefallen bei Gott finden wird.
 \dotfill \textit{Seite
\pageref{ref:06_08_moralisch_verunreinigt}}\\

 \item Auch wenn es die Gebote von Gott selbst sind, die befolgt werden, nützt
es nichts, wenn die Seele mit Sünde verunreinigt ist.
 \dotfill \textit{Seite~\pageref{ref:06_09_gottesregeln}}\\

 \item Gebete unaufrichtiger Menschen finden kein Gehör bei Gott
 \dotfill \textit{Seite~\pageref{ref:06_09_gebetserhoerung}}\\

 \item Die Gemeinschaft der Gläubigen ist die wahre Kirche Gottes.
 \dotfill \textit{Seite~\pageref{ref:06_12_wahre_kirche}}\\

 \item Die vier nötigen Dinge für den Gottesdienst.
 \dotfill \textit{Seite
\pageref{ref:06_13_vier_noetige_dinge}}\\

 \item Von Gott geprüft zu werden ist ein Unglück.
 \dotfill \textit{Seite
\pageref{ref:06_13_auf_die_probe_gestellt}}\\

 \item Ohne Glauben ist richtiges Predigen nicht möglich.
 \dotfill \textit{Seite~\pageref{ref:06_14_predigt}}\\

 \item Über den Glauben und die Erfüllung von Gebeten.
 \dotfill \textit{Seite~\pageref{ref:06_14_gebetserfuellung}}\\

 \item Die drei Haupteigenschaften, aus denen sich alle anderen bösen Neigungen
ableiten, sind: Stolz, Gier und Verschwendung.
 \dotfill \textit{Seite
\pageref{ref:07_01_drei_haupteigenschaften}}\\

 \item Das Streben nach Wissen war die Ursache für den Sündenfall.
 \dotfill \textit{Seite~\pageref{ref:07_03_wissen_erbsuende}}\\

 \item Die Wissenschaft ist nur matter Abglanz der Weisheit Gottes.
 \dotfill \textit{Seite~\pageref{ref:07_03_wissen_erbsuende}}\\

 \item Der Wissenschaftsbetrieb macht arrogant und missgünstig.
 \dotfill \textit{Seite~\pageref{ref:07_06_wissenschaft}}\\

 \item Gelehrten Menschen bleibt das Evangelium verborgen.
 \dotfill \textit{Seite~\pageref{ref:07_13_gelehrte}}\\

 \item Der Fortschritt hat nichts Gutes gebracht. Im Gegenteil.
 \dotfill \textit{Seite~\pageref{ref:07_14_vortschritt}}\\

 \item Unter dem Vorwand des christlichen Eifers wird die Lehre Christi
misachtet, und blutig werden andere Christen unter dem Vorwurf der Ketzerei verfolgt.
 \dotfill \textit{Seite~\pageref{ref:07_14_ketzer}}\\

 \item Durch die Erduldung der Verfolgung durch falsche Christen würden die
wahren Christen ihre echte Nachfolge unter Beweis stellen.
 \dotfill \textit{Seite~\pageref{ref:07_16_vervolgung}}\\

 \item Die Eitelkeit war oft Ursache für Krieg und Zerstörung
 \dotfill \textit{Seite~\pageref{ref:08_01_stolz}}\\

 \item Die, die sich Christen nennen, haben die Heiden auch noch übertroffen in
den Blutbädern und Verwüstungen, die sie angerichtet haben.
 \dotfill \textit{Seite~\pageref{ref:08_06_heiden}}\\

 \item Weil die wahren Christen nicht nach Macht und Einfluss in dieser Welt
streben, sind sie von Anderen so verhasst und verfolgt.
 \dotfill \textit{Seite~\pageref{ref:08_08_reich_gottes}}\\

 \item Diejenigen die ihrem Ehrgeiz Grenzen setzen, sind zufriedener.
 \dotfill \textit{Seite~\pageref{ref:08_09_friedem}}\\

 \item Der erste Grund, warum schmeichelhafte Begrüßungen abgelehnt werden. Die
Ablehnung von Titeln, Grußritualen und anderen Schmeicheleien rührt aus einer
göttlichen Eingebung.
 \dotfill \textit{Seite~\pageref{ref:09_05_offenbarung}}\\

 \item Die Gegner der Quaker verbeissen sich in ihrer Kritik an Äußerlichkeiten
der Quaker, ohne die viel wichtigeren sittlichen Lebensgrundsätze zu beachten.
 \dotfill \textit{Seite~\pageref{ref:09_06_grundsaetze}}\\

 \item Den Quakern wird vorgeworfen, sich nur deshalb so merkwürdig zu
verhalten, um sich darüber kenntlich zu machen.
 \dotfill \textit{Seite~\pageref{ref:09_07_vorwurff}}\\

 \item Zum Vorwurf, die Quaker nähmen es mit Belanglosigkeiten zu genau.
 \dotfill \textit{Seite~\pageref{ref:09_08_vorwurff}}\\

 \item So wie Jesus den Spott seiner Gegner ertragen hat, so wollen auch die
Quaker den Spott für ihren Lebensstil ertragen, in der Gewissheit, dafür am
Jüngsten Tag mit ihm zusammen zu regieren.
 \dotfill \textit{Seite~\pageref{ref:09_10_spott}}\\

 \item Die wahre Bedeutung des Begriffs \textit{Ehre}.
 \dotfill \textit{Seite~\pageref{ref:09_12_ehre}}\\

 \item Achtung und Wertschätzung muss man allen Menschen erweisen, auch wenn sie
unter Einem stehen.
 \dotfill \textit{Seite~\pageref{ref:09_18_ehre}}\\

 \item (Un-)Menschen, die Gott und seine Gebote in sich und bei Anderen
verachten, die soll man auch nicht ehren\footnote{\texttt{Anmerkung des
Herausgebers: Adolf Hitler würde ich z.B. zu solchen Unmenschen dazuzählen}}.
 \dotfill \textit{Seite~\pageref{ref:09_19_ehre}}\\

 \item Der zweite Grund, warum schmeichelhafte Begrüßungen abgelehnt werden, ist
dass es keine Begründung aus der Bibel gibt.
 \dotfill \textit{Seite~\pageref{ref:09_20_zeiter_grund}}\\

 \item Wir ehren den König, unsere Eltern,
unsere Herren, unsere Obrigkeiten
und Vorgesetzten, uns untereinander und alle
Menschen nach Gottes Vorschrift
 \dotfill \textit{Seite~\pageref{ref:09_20_koenig}}\\

 \item Was man "`feines Benehmen"' nennt in der Welt.
 \dotfill \textit{Seite~\pageref{ref:09_26_feines_benemen}}\\

 \item Titel dienen nur der Befriedigung der Eitelkeit.
 \dotfill \textit{Seite~\pageref{ref:09_31_heuchelei}}\\

 \item Dass Quaker alle Stände und Klassen abschaffen wollen, ist die logische
Konsequenz aus der Lehre der Bibel.
 \dotfill \textit{Seite
\pageref{ref:09_35_staende_abschaffen}}\\

 \item Quaker erweisen denen Ehre, die tugendhaft leben, und nicht denen, die
teure Kleider anhaben.
 \dotfill \textit{Seite~\pageref{ref:09_36_ehre_erweisen}}\\

 \item So verborgen die Religion der Quaker ist, so verborgen erweisen sie
anderen die Ehre.
 \dotfill \textit{Seite~\pageref{ref:09_37_ehre_erweisen}}\\

 \item Die Errettung der Armen wie der Reichen, kostet Christus die selbe Menge
Blutes.
 \dotfill \textit{Seite~\pageref{ref:09_39_erettung}}\\

 \item Auch die Jünger Christi unterschieden sich von anderen in der Art wie sie
sprachen.
 \dotfill \textit{Seite~\pageref{ref:10_08_sprache}}\\

 \item Wie kann man es für herablassend halten, geduzt zu werden, wenn man doch
den Höchsten selbst -- Gott -- mit \textit{"`DU"'} anredet? Oder betet ihr
"`...\textit{Ihr} Reich komme. \textit{Ihr} Wille geschehe."'?
 \dotfill \textit{Seite~\pageref{ref:10_08_duzen}}\\

 \item Die die den wahren Geist des Christentums haben, verschwenden ihre Zeit
nicht mit sinnlosen Dingen (Spielen, Besuchen, Tratsch usw.).
 \dotfill \textit{Seite~\pageref{ref:10_08_zeitvertreib}}\\

 \item Eine Beleidigung wird nicht dadurch erträglicher, dass sie von einem
höher Gestellten kommt oder von einem einfachen Menschen.
 \dotfill \textit{Seite~\pageref{ref:11_03_beleidigung}}\\

 \item Das Alter einer kirchlichen Tradition ist nicht die Garantie der
Richtigkeit der selben. Genauso wenig wie die Länge eines Stammbaums einer
Familie nicht die Garantie ist, dass diese moralisch besser sei.
 \dotfill \textit{Seite~\pageref{ref:11_04_abstammung}}\\

 \item Natürlich ist es so, dass der höhere Stand Vorteile bringt. Aber diese
Vorteile sollten zum Wohle aller eingesetzt werden.
 \dotfill \textit{Seite~\pageref{ref:11_07_standesvorteil}}\\

 \item Es wäre schon viel gewonnen, wenn einige nur die Hälfte der Zeit, die sie
auf sich selbst verwenden (mit Schminken, Anziehen und sich Herausputzen), Gott
widmeten und sich um die Armen zu kümmerten.
 \dotfill \textit{Seite~\pageref{ref:11_09_putzsucht}}\\

 \item Die Eitelkeit und die Gier nach Anerkennung ist auch oft schuld an
kaputten Ehen.
 \dotfill \textit{Seite~\pageref{ref:11_09_kaputte_ehen}}\\

 \item Möge ein jeder sich an seine eigene Vergänglichkeit erinnern und sich um
die wesentlichen Dinge im Leben kümmern!
 \dotfill \textit{Seite
\pageref{ref:11_10_juengstes_gericht}}\\

 \item \textbf{Über eitle Menschen...}

% BEGIN einer eingerückten Aufzählung
\begin{enumerate}
 \item Eitle Menschen können es nicht ertragen, anderen -- noch nicht einmal Gott --
zu dienen. Alles soll nur ihrem Zwecke dienen, als wären sie selbst der
Schöpfer.
 \dotfill \textit{Seite~\pageref{ref:12_01_egoisten}}\\

 \item Eitle Menschen wollen immer Recht behalten, aber wenn es hart auf hart
kommt, sind sie feige. Haben sie aber die Oberhand, sind sie erbarmungslos und
grausam.
 \dotfill \textit{Seite~\pageref{ref:12_02_eitle_menschen_streit}}\\

 \item Eitle Menschen sind zu keinem Mitgefühl fähig oder halten Mitgefühl sogar
für eine Schwäche.
 \dotfill \textit{Seite~\pageref{ref:12_02_eitle_menschen_mitgefuehl}}\\

 \item Als Ehemänner und Väter sind eitle Menschen meist Tyrannen in der
Familie. Ihre Frauen und Kinder behandeln sie wie Dienstboten.
 \dotfill \textit{Seite~\pageref{ref:12_03_eitle_menschen_ehe}}\\

 \item Das tugendhafte Verhalten anderer ziehen eitle Menschen meist in den Dreck,
weil sie ihnen den guten Ruf nicht gönnen, und sie sich nicht fähig sehen, es
ihnen gleich zu tun.
 \dotfill \textit{Seite~\pageref{ref:12_04_eitle_menschen_tugent}}\\

 \item Eitle Menschen können anderer Beleidigungen nicht entschuldigen, aber
selbst können sie es nicht unterlassen, ständig andere zu beleidigen.
 \dotfill \textit{Seite~\pageref{ref:12_04_eitle_menschen_beleidigung}}\\

 \item Eitle Menschen können niemandens Freund sein, weil es ihnen unerträglich
ist, in jemandes anderen Schuld zu stehen. Sie wollen ihre Umgebung immer nur in
tiefer Abhängigkeit sehen. Menschen werden von ihnen wie Vieh nach ihrem Wert
beurteilt.
 \dotfill \textit{Seite~\pageref{ref:12_05_eitle_menschen_freundschaft}}\\

 \item Eitle Menschen stürzen meist irgendwann über ihre eigen Eitelkeit,
nachdem sie lange Zeit genug anderen zur Strafe dienten.
 \dotfill \textit{Seite~\pageref{ref:12_06_eitle_menschen_sturz}}\\

 \item Eitle Menschen sind aber am unerträglichsten, wenn sie auch noch die
Religion für sich in Anspruch nehmen. Stolz und (wahre) Religion sind aber
völlig unvereinbar.
 \dotfill \textit{Seite~\pageref{ref:12_07_eitle_menschen_religion}}\\

 \item Religion missbrauchen eitle Menschen nur, um sich damit aufzuspielen und
bei anderen etwas zu gelten. Um Erlösung kann es ihnen dabei aber nicht gehen.
 \dotfill \textit{Seite~\pageref{ref:12_07_eitle_menschen_religion_2}}\\

 \item Als Geistliche glauben eitle Menschen, mit ihrem Amt über jeden
moralischen Tadel erhaben zu sein.
 \dotfill \textit{Seite~\pageref{ref:12_07_eitle_menschen_geislicher}}\\

 \item Die Beschwerde, dass sich Laien nicht in die Angelegenheiten der
Geistlichkeit einmischen sollen, ist unberechtigt. Umgekehrt ist es schlecht, wenn
sich die Geistlichkeit in die Angelegenheit der Laien einmischt.
 \dotfill \textit{Seite
\pageref{ref:12_08_eitle_menschen_leihen_vs_geisltichkeit}}\\

 \item Die eitlen Menschen, denen zu Lebzeiten kein Ort gut genug war,
werden sich eines Tages -- wie allen anderen -- mit der selben feuchten Erde
begnügen müssen.
 \dotfill \textit{Seite~\pageref{ref:12_10_eitle_menschen_tod}}\\

 \item Das Mittel zur Überwindung der Eitelkeit ist, sich dem Inneren Licht
zuzuwenden, und sich in diesem zu prüfen.
 \dotfill \textit{Seite~\pageref{ref:12_11_eitle_menschen_erloesung}}\\

 \item Man muss bereit sein, Opfer zu bringen. Alle wollen sich zwar mit Christus
zusammen freuen, aber wer ist bereit auch mit ihm zusammen zu leiden?
 \dotfill \textit{Seite~\pageref{ref:12_11_opfer}}\\

 \end{enumerate}
% ENDE einer eingerückten Aufzählung

 \item Die Akkumulation\footnote{Akkumulation ist das Anhäufen von Werten und
der Entzug derselben aus dem Wirtschaftskreislauf}. Der Habgierige schädigt die
Wirtschaft und die Allgemeinheit.
 \dotfill \textit{Seite~\pageref{ref:13_01_accumlation}}\\

 \item Reichtum an sich ist nichts unmoralisches, aber es verleitet schnell zu
unmoralischem Handeln.
 \dotfill \textit{Seite~\pageref{ref:13_05_reichtum}}\\

 \item Reiche Menschen, die nie genug bekommen können, sind von Gott Verlassene,
da sie ihr Heil offensichtlich im Geld und nicht in Gott suchen.
 \dotfill \textit{Seite~\pageref{ref:13_06_reichtum_unersaettlichkeit}}\\

 \item Wenn man wirtschaftlich erfolgreich war, sollte das ein Anlass sein, sich
aus dem Weltlichen zurückzuziehen, statt nach immer mehr zu gieren. Aber viele
wollen Geld nicht nur für ihren bloßen Unterhalt, sondern des Reichtums wegen.
 \dotfill \textit{Seite~\pageref{ref:13_07_reichtum_genuegsamkeit}}\\

 \item Der Grund warum viele Menschen wie die Sklaven schuften müssen für ihr
dürftiges Einkommen, ist die Maßlosigkeit der Reichen, die den Kleinen ihre
Einnahmequellen abgraben. Es wäre gut, wenn es Regeln gäbe, die der Expansion
von Unternehmen grenzen setzen würde. So dass die nächste Generation
auch noch Möglichkeiten zu ihrem Unterhalt finden kann. Die Alten
erfolgreichen Unternehmer sollten sich lieber irgendwann zurückziehen und an
ihr Lebensende und ihr Seelenheil denken.
 \dotfill \textit{Seite~\pageref{ref:13_08_reichtum_sklaverei}}\\

 \item Die in ihrer grenzenlosen Gier alles an sich raffen, sind Feinde des
Staates und Gift für die Allgemeinheit. Ihnen müssen per Gesetz Schranken gesetzt
werden.
 \dotfill \textit{Seite~\pageref{ref:13_14_reichtum_schaden}}\\

 \item Aus liebe zum Geld sparen die Raffgierigen an allem und bringen so
Häuser zum Einsturz, weil sie kein Geld für Reparaturen ausgeben wollen. Dinge
verfallen zu lassen (einschließlich der Gesundheit), rechnen sie sich sogar
noch als Tugend der Sparsamkeit an.
 \dotfill \textit{Seite~\pageref{ref:13_14_reichtum_einsturz}}\\

 \item Die Raffgierigen kritisieren gerne die Verschwendung anderer und
 erinnern daran, was man mit dem Geld alles sinnvolleres tun könne. Doch
 wenn es daran geht. Geld für die Dinge ausgegeben, sie er selbst vorgeschlagen
 haben, so finden sie andere Vorwände dies abzulehnen.
 \dotfill \textit{Seite~\pageref{ref:13_15_Kapitalisten_kritik}}\\

 \item Ist es nicht unwürdig, sich für die Gier nach Bodenschätzen
gegenseitig den Schädel einzuschlagen?
 \dotfill \textit{Seite~\pageref{ref:13_21_Kapitalisten_kritik}}\\

 \item Die Gierigen werden zu Sklaven ihrer Gier und dienen dem Geld. Doch
 sollte es eigentlich so sein, das das Geld seinem Besitzer Dienste leistet und nicht umgekehrt.
 \dotfill \textit{Seite~\pageref{ref:13_21_Kapitalisten_dienerschaft}}\\

 \item So wie für Wahnsinnige ein Vormund vom Gericht gestellt wird, so
müsste es auch einen Vormund für geldgierige Menschen geben, der ihr Vermögen für
sie verwaltet.
 \dotfill \textit{Seite~\pageref{ref:13_22_wahnsinnige}}\\

 \item Diejenigen die sich zu Christus bekennen, müssen auch das Leben eines
wahren Jüngers und Nachfolgers führen. Und diese schwelgten nicht in Luxus und
Vergnügungen.
 \dotfill \textit{Seite~\pageref{ref:14_01_wahre_nachfolger}}\\

 \item Es braucht keiner zu glauben, der Opfertod Christi könne ihm etwas
nützen, wenn er selbst in der Sünde verharrt.
 \dotfill \textit{Seite~\pageref{ref:14_01_wahre_nachfolger_suenetod}}\\

 \item Die Kleidung ist ein Zeichen der Sünde. Sie ist erst nötig geworden durch
den Sündenfall. Darum ist es umso verwerflicher, auch noch stolz auf seine
Kleidung zu sein und sie auffällig zu gestalten.
 \dotfill \textit{Seite~\pageref{ref:14_04_wahre_nachfolger_kleidung}}\\

 \item Bevor sich jemand \textit{Christ} nennt, sollte er erst einmal einem reinen
Lebenswandel folgen.
 \dotfill \textit{Seite~\pageref{ref:14_06_wahre_nachfolger_umkehr}}\\

 \item Früher bestand die \textit{Erholung} darin, Gott zu dienen, Gutes zu tun
und seinen Geschäften nachzugehen.
 \dotfill \textit{Seite~\pageref{ref:14_07_wahre_nachfolger_erholung}}\\

 \item Christus kam nicht, um das rationale Denken zu beseitigen. Im Gegenteil.
 \dotfill \textit{Seite~\pageref{ref:14_08_wahre_nachfolger_rational}}\\

 \item Es wäre für alle genug da, wenn sich einige mit etwas weniger begnügen
würden.
 \dotfill \textit{Seite~\pageref{ref:15_02_genug_fuer_alle}}\\

 \item Jeder wird ernten, was er gesät hat. Deshalb hört auf die innere Stimme
eures Richters! Sonst werdet ihr am Ende eures Lebens eine traurige und
schreckliche Ernte haben.
 \dotfill \textit{Seite~\pageref{ref:15_04_innere_stimme}}\\

 \item Über die angemessene Beschäftigung für Christen
 \dotfill \textit{Seite~\pageref{ref:15_05_freizeitbeschaeftigung}}\\

 \item Wie wollen denn die Leute die Ewigkeit in Gott ertragen, wenn ihnen in der
wenigen Zeit ihres Lebens, die Beschäftigung mit Gott zu langweilig ist?
 \dotfill \textit{Seite~\pageref{ref:15_06_langeweile}}\\

 \item Von allen überflüssigen Dingen ist das Schauspiel das schlimmste.
 \dotfill \textit{Seite~\pageref{ref:15_08_schauspiel}}\\

 \item Warum soll man das ganze Leid ertragen und sich der Sinnesfreuden
enthalten?
 \dotfill \textit{Seite~\pageref{ref:16_01_warum}}\\

 \item Wer an Jesus glaubt und ihm Gehorsam leistet, bekommt das ewige Reich und
die Krone.
 \dotfill \textit{Seite~\pageref{ref:16_01_warum_2}}\\

 \item Könnt ihr den Kelch trinken? Sonst könnt ihr nicht Jünger Christi sein.
 \dotfill \textit{Seite~\pageref{ref:16_03_kelch_tringen}}\\

 \item Für einen Jünger Christi muss die Welt gleichgültig werden.
 \dotfill \textit{Seite~\pageref{ref:16_05_weltliches}}\\

 \item Wer in den weltlichen Gelüsten verharrt, der kennt Christus nicht.
 \dotfill \textit{Seite~\pageref{ref:16_05_weltliches_2}}\\

 \item Über die Erziehung junger Menschen.
 \dotfill \textit{Seite~\pageref{ref:17_01_erziehung}}\\

 \item Über die Themen und Inhalte von Theaterstücken.
 \dotfill \textit{Seite~\pageref{ref:17_01_schauspiel}}\\

 \item Schauspiel verleitet zur Nachahmung.
 \dotfill \textit{Seite~\pageref{ref:17_01_schauspiel_2}}\\

 \item Nur ein Bruchteil von dem, was heute als gängiges und harmloses Vergnügen
angesehen wird, hätte Adam und Eva das Paradies gekostet.
 \dotfill \textit{Seite~\pageref{ref:17_02_adam_und_eva}}\\

 \item Gerade mal eine halbe Stunde leiern sie mit erzwungenem Eifer die Gebete
anderer herunter. Die Worte die sie dabei sprechen, haben überhaupt nichts mit
ihrer Gemütslage zu tun, und was sie im Gebet erbitten, wünschen sie in Wahrheit
nicht.
 \dotfill \textit{Seite~\pageref{ref:17_04_gottesdienst}}\\

 \item Zu dem Einwand, es gäbe auch sehr lehrreiche Schauspielstücke.
 \dotfill
 \textit{
 Seite~\pageref{ref:17_07_einwand}~--~\pageref{ref:17_07_einwand_ende}}\\

 \item Bei den Heiden wurde unsittliches Verhalten von Amts wegen bestraft. Bei
Christen wird man noch verhöhnt, wenn man ihnen Vorhaltungen wegen ihres
Lebenswandels macht.
 \dotfill \textit{Seite~\pageref{ref:17_08_sittenwaechter}}\\

 \item Durch den Beifall werden Menschen noch angestachelt, sich immer neue
Ausschweifungen einfallen zu lassen, statt sie zu sinnvoller Arbeit anzuhalten.
 \dotfill \textit{Seite~\pageref{ref:17_09_bedarf_wecken}}\\

 \item Zum Einwand, durch Schauspiel und Vergnügen würden Familien Lohn und
Auskommen haben.
 \dotfill \textit{Seite~\pageref{ref:17_10_einwand}}\\

 \item Zum Einwand, Gott hätte die Freude an den Dingen nicht zu unserer
Verdammnis geschaffen.
 \dotfill \textit{Seite~\pageref{ref:17_11_einwand_2}}\\

 \item Vielen dient die Unterhaltung auch zur Betäubung der inneren Stimme, die
sie anklagt.
 \dotfill \textit{Seite~\pageref{ref:17_11_beteubung}}\\

 \item Man sollte in sich Einkehr halten und prüfen, wessen Wille und Werk
man tut.
 \dotfill \textit{Seite~\pageref{ref:17_12_einkehr}}\\

 \item Zum Einwand, man könnte jederzeit die Vergnügungen unterlassen und würde
nicht an ihnen hängen.
 \dotfill \textit{Seite~\pageref{ref:18_02_einwand}}\\

 \item Selbst wenn man sich in den Vergnügungen mäßigen kann, so sollte man sie
trotzdem unterlassen, um ein Vorbild für die zu sein, die sich nicht mäßigen
können.
 \dotfill \textit{Seite~\pageref{ref:18_03_vorbild}}\\

 \item Wenn sich Christen vorbildlich kleiden und verhalten, steigt die Hemmung
der anderen, über das Maß zu schlagen.
 \dotfill \textit{Seite~\pageref{ref:18_08_vorbild_kleidung}}\\

 \item Tugendhaftes Verhalten ist nicht nur aus religiösen Gründen gut, sondern
auch der Staat profitiert davon. Deshalb sollte sich der Staat auch um die
Sitten kümmern.
 \dotfill \textit{Seite~\pageref{ref:18_09_gesellschaftlich}}\\

 \item Warum so lautstark die Stimme gegen Missbrauch und Torheiten erhoben
werden muss.
 \dotfill \textit{Seite~\pageref{ref:18_09_lautstark}}\\

 \item Es wird daran erinnert, dass ein großer Teil der Gesellschaft hart
arbeiten muss, damit ein kleiner Teil sich dem Vergnügen ergeben kann.
 \dotfill \textit{Seite~\pageref{ref:18_10_ungeraechtikeit}}\\

 \item Warum bedarf es so viel Überredung, um Menschen dazu zu bringen, für ihre
Glückseligkeit zu arbeiten?
 \dotfill \textit{Seite~\pageref{ref:18_11_ueberredung}}\\

 \item Seid ihr Feinde des Kreuzes?
 \dotfill \textit{Seite~\pageref{ref:18_11_feinde_des_kreuzes}}\\
\end{enumerate}
