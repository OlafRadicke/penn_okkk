\section*{B}

\articlesize

\begin{description}

 \item[Baker, Eric] $\ast$22.9.1920 \dag~July 1976. Quaker und Mitbegünder von $\to$\textit{Amnesty International}, deren Secretary-General er 1966 bis 1968 war. Er gründete \textit{Amnesty International} 1961 in London mit Peter Benenson zusammen. Für die Quaker betreute er von 1946 bis 1948 ein Zentrum in Delhi, India. In den spähten 1950er und den frühen 1960er Jahren war Baker für das \textit{the Friends Peace \& International Affairs Committee} (FPIAC) tätig. Gegen Ende seines Lebens war er im $\to$\textit{Yearly Meeting} der \textit{Religious Society of Friends in London} und dem \textit{Triennial Meeting of the Friends World Committee for Consultation} aktiv.

 \item[Barclay, Robert] $\ast$23.12.1648 in Gordonstown, Morayshire; \dag3.10.1690 in Ury bei Aberdeen. War Theologe und Quaker und eine der wichtigsten Figuren des jungen Quakertums. Er war der Erste, der die thologischen rundsätze systematisch auf akademischen neveu ausformulierte. Seine Schriften erfreuten sich unter Quakern großer beliebtheit. Kaum ein Quaker-Autor seiner Zeit wurde in so vielen Sprachen übersetzt und in hoher Auflage vertrieben. Seine Bücher wurden oft auch für Missionszwecke verschenkt. Sein Elternhaus war zunächst calvinistisch geprägt. Ihm wurde eine umfassende Bildung zu Teil, die ihn sogar bis an die Collegium Scotorum nach Paris führte. Sein Vater konvertierte 1666 zum Quakertum und er selbst schon ein Jahr spähter. Schon kurz darauf reiste er mit den führenden Persönlichkeiten des Quakertums wie $\to$William Penn und $\to$George Fox zu missionszwecken nach Holland und Deutschland. Hier bei traf er auch mit $\to$Pfalzgräfin Elisabeth, der Äbtissin von Herford zusammen. Und Auch er setzte sich wie sie für der Quaker bei dem König von England ein. Er versuchte sein Einfluss am Hof von Jakob II. dazu zu nutzen. Wie viele andere Quaker, sass auch er für fünf monate im Gefängnis. Sein Werk ist stark gepregt von von der kabbalistischen Mystik wie sie unter anderem $\to$George Keith vertrat. Seine \textit{Apologie} gilt als sein Hauptwerk und prägte das Selbstverständnis des frühen Quakertums.\endnote{Seite "`Robert Barclay (Quäker)"'. In: Wikipedia, Die freie Enzyklopädie. Bearbeitungsstand: 27. November 2009, 10:13 UTC. URL:
 \\ \texttt{http://de.wikipedia.org/w/index.php?title=Robert\_Barclay\_(Qu\%C3\%A4ker)\&oldid=67322327}
 (Abgerufen: 19. Dezember 2009, 13:10 UTC) } \endnote{Sünne Juterczenka, "`Über Gott und die Welt - Entzeitvisionen, Reformdebatten, und die europäische Quäkermission in der frühen Neuzeit"', Vandenhoeck\&Ruprecht, 2008, ISBN 978-3-525-35458-2} \endnote{Claus Bernet: Robert Barclay (Quäker). In: Biographisch-Bibliographisches Kirchenlexikon (BBKL). Band 1, Hamm 1975, Sp. 367–368, \\  \texttt{http://www.bbkl.de/b/barclay\_r.shtml}}

 \item[Barclays Bank PLC] Ein international agierendes Finanzunternehmen aus Großbritannien. Gegründet wurde die Bank schon im Jahr 1690. Von Londoner Quaker. Der Name \textit{Barclay} wurde erstmals 1736 verwendet. Zu ihrem Reichtum kam sie während des Kolonialismus. Die fühe Geschichte der Bank steht in Verbindung mit der Quaker-Famiele $\to$Gurney. Und diese wiederum mit der \textit{Gurney's bank}. Aus der Famili  Gurney ent stammen die beiden - für die Quakergeschichte wichtigen - Persönlichkeiten $\to$Joseph John Gurney und $\to$Elizabeth Fry. \endnote{\\ \texttt{http://www.nzz.ch/2007/03/28/al/articleF1WAM.html} NZZ vom 28. März 2007 und \\ \texttt{http://en.wikipedia.org/wiki/Elizabeth\_Fry}} \endnote{Barbara Müller, "`Widerstand gegen Barclays"', In "`Finanzplatz Informationen"' 3/2005, \\ \texttt{http://www.aktionfinanzplatz.ch/pdf/fpi/2005/3/Barclays.pdf}}

 \item[Beanite Friends]
 Diese sind durch ein unfreiwilligen Schisma entstanden. Diese, kann man sie als $\to$liberale Quaker bezeichnen. Sie haben eine umprogrammierte Andacht und keine Pastoren. Sind aber eher Christozentrisch. Sie haben sich Abgespalten von den Iowa Conservative Yearly Meeting, als sich diese immer mehr dem $\to$evangelikalen Quakertum zuwanden. Das war um das Jahr 1881.
 \endnote{\texttt{http://en.wikipedia.org/wiki/Beanite}} \endnote{Beanite Quakerism. (2009, October 11). In Wikipedia, The Free Encyclopedia. Retrieved 13:28, December 19, 2009, from \\ 
 \texttt{http://en.wikipedia.org/w/index.php?title=Beanite\_Quakerism\&oldid=319319549}
 }


 \item[Beerdigung] Die Beedigung bei Quakern findet ganz wie der $\to$Gottesdienst, ohne Liturgie statt. In Sterbeanzeigen hieß es oft \textit{"`...der ganannt wurde...[Name des Toten]"'}. Das hing damit zusammen, das geglaubt wurde, das erst am \textit{Jüngsten Tag} oder im Hadis von Gott der richtige Name bestimmt wurde. Das rührt daher, das dass Quakertum ursprünglich eine eschatologischen Erweckungsbewegungs-Charakter hatte. Von Aussenstehenden wurde die Quakerbeerdigungen als pietätlos empfunden und sorgten für Spott und Entrüstung. \endnote{Seite "`Quäkertum"'. In: Wikipedia, Die freie Enzyklopädie. Bearbeitungsstand: 11. Dezember 2009, 00:40 UTC. URL:
 \\ \texttt{ http://de.wikipedia.org/w/index.php?title=Qu\%C3\%A4kertum\&oldid=67859053  }
\\ (Abgerufen: 22. Dezember 2009, 22:08 UTC) } \endnote{Sünne Juterczenka, "`Über Gott und die Welt - Entzeitvisionen, Reformdebatten, und die europäische Quäkermission in der frühen Neuzeit"', Vandenhoeck\&Ruprecht, 2008, ISBN 978-3-525-35458-2, Seite 275}

 \item[Bennenungsausschuß] Da die Ämter nicht durch Wahlen besetzt werden, wird eine Gruppe von Mitgliedern beauftragt der $\to$\textit{Versammlung} vorschläge für die Ämterbesetzung zu unterbreiten. Über die Besetzung der Ausschuss entscheidet eigendlich die Versammlung. 
 
 \item[Bezirksversammlung] Eine Bezirksversammlung vereinigt einige $\to$\textit{Monatsversammlungen} zu einer oranisatorischen-infomellen Einheit, um die Monatsversammlungen untereinander zu versammeln. Die \textit{Bezirksversammlung} ist eine Besonderheit des deutschen Quakertums. In anderen Ländern verwendet man den Begriff \textit{Vierteljahresversammlung}. Das ist eigendlich auch die korekte Selbstbeschreibung. Denn die Selbige trifft sich vier mal im Jahr und ist informelles Bindeglied zwischen Monatsversammlung und $\to$\textit{Jahresversammlung}. Im deutschen Quakertum wird auf der ebene der Bezierks- bzw. Vierteljahresversammlung die Mitgliederverwaltung abgewickelt. Also die Aufnahmen, Ausschlüsse und so weiter abgewickelt. Aus dem ganz einfachen Grund, weil in Deutschland die die meisten Mitglieder keine eigene Monatsversammlung haben, zu der sie gehen können (oder wollen). In anderen Ländern, ist die Monatsversammlung das eigendliche Zentrum des Gemeinschaftlichen Lebens. Es gibt Länder die auf den \textit{Luxus} einer Jahresversammlung verzichten.\endnote{Dazu zählt zum Beispiel das Moscow Monthly Meeting (MM), das Ramallah MM, Brummana MM, Barcelona MM, Hong Kong MM, Seoul MM und das Kinshasa MM. Pink Dandelion, "`An Introduction to Quakerism"', Seite 177, ISBN: 052160088X}
 
 \item[Birthright Membership] Das frühe Quakertum verstand sich als \textit{"`Gemeinschaft der Gläubigen"'} und ihrer Mitglieder wahren ausschlisslich Konvertiken. Schon eine Generation spähter, mit den in die Gemeinschaft \textit{hineingeborenen}, entstand sowas wie "`geborene Quaker"'. In laufe der Zeit entstanden fast sowas wie Quaker-Dynastin wie bei den $\to$\textit{Gurneys}. Man begann von "`Birthright Friends"' zu sprechen. Das Gegenstück dazu ist quasie das \textit{"`Member by Convincement"'} (engl. ``conversion'' = ``Konvertieren) bzw. $\to$\textit{Quaker by Convincement}. Also dem Bekentnis zur Quakerüberzeugungen.
 
 \item[Birthright Friends] $\to$\textit{Birthright Membership}
 
 \item[Birthright Quaker] $\to$\textit{Birthright Membership}

 \item[Bunyan, John] $\ast$28. November 1628 zu Elstow bei Bedford, \dag31. August 1688 in London. Baptistenprediger. Bekannt durch seinen Hauptwerk \textit{"`Pilgerreise zur seligen Ewigkeit"'} (orig. \textit{The Pilgrim’s Progress from This World to That Which Is to Come}). Um 1656 bis 1657 lieferte er sich ein heftigen Schlagabtausch mit dem Quaker Edward Burrough, in dem sie Streitschriften veröffentlichten. Von John Bunyan stammte die Schriften \textit{"`Some Gospel Truths Opened in which he attacked Quaker beliefs"'} und \textit{"`A Vindication of Some Gospel Truths Opened"'}. Von Edward Burrough stammten \textit{"`The True Faith of the Gospel of Peace"'} und \textit{"`Truth (the Strongest of All) Witnessed Forth"'}. Darin griff er die Quaker dafür an, das sie das $\to$\textit{"`Innere Licht"'}, seiner Meinung nach, fälschlicher weise, über die Bibel stellten. Spähter behandelte auch noch mal $\to$George Fox die Auseinandersetzung mit John Bunyan in \textit{"`The Great Mystery of the Great Whore Unfolded"'}. Um so bemerkenswehrtet war, das sich viele Jahre spähter, die Quaker bei dem zweiten Gefängnisaufenthalt (Ab 1675) von John Bunyan erfolkreich für dessen Freilassung einsetzten.

 \item[Burrough, Edward] $\ast$1634 \dag1663 gehörte zu den so genanten $\to$\textit{Valiant Sixty}. $\to$\textit{Bunyan, John}


 \item[by convincement (Mitgliedschaft)] $\to$\textit{Quaker by Convincement}
 \end{description}

\normalsize

