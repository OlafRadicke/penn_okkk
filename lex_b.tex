\section*{B}

\articlesize

\begin{description}

 \item[Baker, Eric] $\ast$22.9.1920 \dag~July 1976. Quaker und Mitbegünder von $\to$\textit{Amnesty International}, deren Secretary-General er 1966 bis 1968 war. Er gründete \textit{Amnesty International} 1961 in London mit Peter Benenson zusammen. Für die Quaker betreute er von 1946 bis 1948 ein Zentrum in Delhi, India. In den spähten 1950er und den frühen 1960er Jahren war Baker für das \textit{the Friends Peace \& International Affairs Committee} (FPIAC) tätig. Gegen Ende seines Lebens war er im $\to$\textit{Yearly Meeting} der \textit{Religious Society of Friends in London} und dem \textit{Triennial Meeting of the Friends World Committee for Consultation} aktiv.

 \item[Barclay, Robert]

 \item[Barclays Bank PLC] Ein international agierendes Finanzunternehmen aus Großbritannien.

 \item[Beerdigung]

 \item[Bennenungsausschuß]
 
 \item[Bezirksversammlung]
 \item[Birthright Membership]
 \item[Birthright Friends] $\to$\textit{Birthright Membership}
 \item[Birthright Quaker] $\to$\textit{Birthright Membership}

 \item[Bunyan, John] $\ast$28. November 1628 zu Elstow bei Bedford, \dag31. August 1688 in London. Baptistenprediger. Bekannt durch seinen Hauptwerk \textit{"`Pilgerreise zur seligen Ewigkeit"'} (orig. \textit{The Pilgrim’s Progress from This World to That Which Is to Come}). Um 1656 bis 1657 lieferte er sich ein heftigen Schlagabtausch mit dem Quaker Edward Burrough, in dem sie Streitschriften veröffentlichten. Von John Bunyan stammte die Schriften \textit{"`Some Gospel Truths Opened in which he attacked Quaker beliefs"'} und \textit{"`A Vindication of Some Gospel Truths Opened"'}. Von Edward Burrough stammten \textit{"`The True Faith of the Gospel of Peace"'} und \textit{"`Truth (the Strongest of All) Witnessed Forth"'}. Darin griff er die Quaker dafür an, das sie das $\to$\textit{"`Innere Licht"'}, seiner Meinung nach, fälschlicher weise, über die Bibel stellten. Spähter behandelte auch noch mal $\to$George Fox die Auseinandersetzung mit John Bunyan in \textit{"`The Great Mystery of the Great Whore Unfolded"'}. Um so bemerkenswehrtet war, das sich viele Jahre spähter, die Quaker bei dem zweiten Gefängnisaufenthalt (Ab 1675) von John Bunyan erfolkreich für dessen Freilassung einsetzten.

 \item[Burrough, Edward] $\ast$1634 \dag1663 gehörte zu den so genanten $\to$\textit{Valiant Sixty}. $\to$\textit{Bunyan, John}
 \item[by convincement (Mitgliedschaft)]
 \end{description}

\normalsize

