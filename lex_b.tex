\section*{B}

\articlesize

\begin{description}


 \item[Barclays Bank PLC] Ein international agierendes Finanzunternehmen aus
Großbritannien. Gegründet wurde die Bank schon im Jahr 1690. Von Londoner
Quaker. Der Name \textit{Barclay} wurde erstmals 1736 verwendet. Zu ihrem
Reichtum kam sie während des Kolonialismus. Die fühe Geschichte der Bank steht
in Verbindung mit der Quaker-Famiele $\to$Gurney. Und diese wiederum mit der
\textit{Gurney's bank}. Aus der Famili  Gurney ent stammen die beiden - für die
Quakergeschichte wichtigen - Persönlichkeiten $\to$Joseph John Gurney und
$\to$Elizabeth Fry. ($\to$\textit{Gurney's bank})\endnote{\\
\texttt{http://www.nzz.ch/2007/03/28/al/articleF1WAM.html} \\
NZZ vom 28. März 2007 und \\
\texttt{http://en.wikipedia.org/wiki/Elizabeth\_Fry}}
\endnote{Barbara Müller, "`Widerstand gegen Barclays"', In "`Finanzplatz
Informationen"' 3/2005,
\\ \texttt{http://www.aktionfinanzplatz.ch/pdf/fpi/2005/3/Barclays.pdf}}

 \item[Beanite Friends]
 Diese sind durch ein unfreiwilligen Schisma entstanden. Diese, kann man sie als
$\to$liberale Quaker bezeichnen. Sie haben eine umprogrammierte Andacht und
keine Pastoren. Sind aber eher Christozentrisch. Sie haben sich Abgespalten von
den Iowa Conservative Yearly Meeting, als sich diese immer mehr dem
$\to$evangelikalen Quakertum zuwanden. Das war um das Jahr 1881.
 \endnote{\texttt{http://en.wikipedia.org/wiki/Beanite}} \endnote{Beanite
Quakerism. (2009, October 11). In Wikipedia, The Free Encyclopedia. Retrieved
13:28, December 19, 2009, from \\ 
 \texttt{
http://en.wikipedia.org/w/index.php?title=Beanite\_Quakerism\&oldid=319319549}
 }


 \item[Beerdigung] Die Beerdigung bei Quakern findet ganz wie der
$\to$Gottesdienst, ohne Liturgie statt. In Sterbeanzeigen hieß es oft
\textit{"`...der ganannt wurde...[Name des Toten]"'}. Das hing damit zusammen,
das geglaubt wurde, das erst am \textit{Jüngsten Tag} oder im Hadis von Gott der
richtige Name bestimmt wurde. Das rührt daher, das dass Quakertum ursprünglich
eine eschatologischen Erweckungsbewegungs-Charakter hatte. Von Aussenstehenden
wurde die Quakerbeerdigungen als pietätlos empfunden und sorgten für Spott und
Entrüstung. \endnote{Seite "`Quäkertum"'. In: Wikipedia, Die freie Enzyklopädie.
Bearbeitungsstand: 11. Dezember 2009, 00:40 UTC. URL:
 \\ \texttt{
http://de.wikipedia.org/w/index.php?title=Qu\%C3\%A4kertum\&oldid=67859053  }
\\ (Abgerufen: 22. Dezember 2009, 22:08 UTC) } \endnote{Sünne Juterczenka,
"`Über Gott und die Welt - Entzeitvisionen, Reformdebatten, und die europäische
Quäkermission in der frühen Neuzeit"', Vandenhoeck\&Ruprecht, 2008, ISBN
978-3-525-35458-2, Seite 275}

 \item[Bennenungsausschuß] Da die Ämter nicht durch Wahlen besetzt werden, wird
eine Gruppe von Mitgliedern beauftragt der $\to$\textit{Versammlung} vorschläge
für die Ämterbesetzung zu unterbreiten. Über die Besetzung der Ausschuss
entscheidet eigendlich die Versammlung. 

 \item[Beten] $\to$\textit{Hold in the Light}
 
 \item[Bezirksversammlung] Eine Bezirksversammlung vereinigt einige
$\to$\textit{Monatsversammlungen} zu einer oranisatorischen-infomellen Einheit,
um die Monatsversammlungen untereinander zu versammeln. Die
\textit{Bezirksversammlung} ist eine Besonderheit des deutschen Quakertums. In
anderen Ländern verwendet man den Begriff \textit{Vierteljahresversammlung}. Das
ist eigendlich auch die korekte Selbstbeschreibung. Denn die Selbige trifft sich
vier mal im Jahr und ist informelles Bindeglied zwischen Monatsversammlung und
$\to$\textit{Jahresversammlung}. Im deutschen Quakertum wird auf der ebene der
Bezierks- bzw. Vierteljahresversammlung die Mitgliederverwaltung abgewickelt.
Also die Aufnahmen, Ausschlüsse und so weiter abgewickelt. Aus dem ganz
einfachen Grund, weil in Deutschland die die meisten Mitglieder keine eigene
Monatsversammlung haben, zu der sie gehen können (oder wollen). In anderen
Ländern, ist die Monatsversammlung das eigendliche Zentrum des
Gemeinschaftlichen Lebens. Es gibt Länder die auf den \textit{Luxus} einer
Jahresversammlung verzichten.\endnote{Dazu zählt zum Beispiel das Moscow Monthly
Meeting (MM), das Ramallah MM, Brummana MM, Barcelona MM, Hong Kong MM, Seoul MM
und das Kinshasa MM. Pink Dandelion, "`An Introduction to Quakerism"', Seite
177, ISBN: 052160088X}
 
 \item[Birthright Membership] Das frühe Quakertum verstand sich als
\textit{"`Gemeinschaft der Gläubigen"'} und ihrer Mitglieder wahren
ausschlisslich Konvertiken. Schon eine Generation spähter, mit den ersten in die
Gemeinschaft \textit{hineingeborenen}, entstand sowas wie "`geborene Quaker"'
und man begann von "`Birthright Friends"' zu sprechen. Nicht in allen Zweigen
und Zeiten des Quakertums wurde das akzetiert. In laufe der Zeit entstanden aber
fast sowas wie \textit{Quaker-Dynastin} wie bei der Quaker-Famili den
$\to$\textit{Gurneys}, die sehr viel Gewicht hatte. Das Gegenstück dazu ist
quasie das \textit{"`Member by Convincement"'} (engl. ``conversion'' =
``Konvertieren) bzw. $\to$\textit{Quaker by Convincement}. Also der
Mitgliedschaft durch das Bekentnis zu Quakerüberzeugungen. Also dem bewusten und
überzeugten Übertritt.
 
 \item[Birthright Friends] $\to$\textit{Birthright Membership}
 
 \item[Birthright Quaker] $\to$\textit{Birthright Membership}


 \item[by convincement (Mitgliedschaft)] $\to$\textit{Quaker by Convincement}
 \end{description}

\normalsize

