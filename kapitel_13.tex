
\chapter{Kapitel 13} \label{kap13}

\section{Zusammenfassung des 13. Kapitel}

\begin{description}
\item[1. Abschnitt] Von der Gier\footnote{\texttt{'Geiz' hat in der Vergangenheit einen starken
Bedeutungswandel erfahren. Früher war in erster Linie das gemeint, was wir
heute 'Gier' nennen würden.
\\ \textbf{Siehe zum Bedeutungswandel auch:}
\\ \textit{Seite 'Geiz'. In: Wikipedia, Die freie Enzyklopädie.
Bearbeitungsstand: 4. August 2009, 10:48 UTC. URL:
\\ http://de.wikipedia.org/w/index.php?title=Geiz\&oldid=62956916
\\ (Abgerufen: 1. September 2009, 11:50 UTC) }
\\ \\ Desweiteren ersetze ich jetzt 'Geiz' durch 'Gier' ohne explizit darauf
hinzuweisen, da der Begriff zu oft verwendet wird!}}, als der zweiten Hauptleidenschaft des Menschen,
-- Bestimmung und Erklärung desselben.
\dotfill \textit{Seite~\pageref{kap13_ab1}}\\
\item[2. Abschnitt] Er besteht erstlich in einem Verlangen nach unerlaubten
Dingen.\\
.\dotfill \textit{Seite~\pageref{kap13_ab2}}\\
\item[3. Abschnitt] Wie in dem Beispiel mit David als er Urias Frau begehrte\\
.\dotfill \textit{Seite~\pageref{kap13_ab3}}\\
\item[4. Abschnitt] So auch mit Ahab, der Noboths Weinberg an sich zu bringen
suchte.
\dotfill \textit{Seite~\pageref{kap13_ab4}}\\
\item[5. Abschnitt] Zweitens besteht die Gier in einem unerlaubten Begehren
erlaubter Dinge.
\dotfill \textit{Seite~\pageref{kap13_ab5}}\\
\item[6. Abschnitt] Sie ist ein Kennzeichen der falschen Propheten.
\dotfill \textit{Seite~\pageref{kap13_ab6}}\\
\item[7. Abschnitt] Die Gier ist entehrend für die Religion.
\dotfill \textit{Seite~\pageref{kap13_ab7}}\\
\item[8. Abschnitt] Ein Feind der Obrigkeit.
\dotfill \textit{Seite~\pageref{kap13_ab8}}\\
\item[9. Abschnitt] Sie macht die Menschen zu Betrügern.
\dotfill \textit{Seite~\pageref{kap13_ab9}}\\
\item[10. Abschnitt] Sie leitet zur Unterdrückung.
\dotfill \textit{Seite~\pageref{kap13_ab10}}\\
\item[11. Abschnitt] Beispiel des Judas.
\dotfill \textit{Seite~\pageref{kap13_ab11}}\\
\item[12. Abschnitt] Auch des Simon Magus
\dotfill \textit{Seite~\pageref{kap13_ab12}}\\
\item[13. Abschnitt] Endlich besteht die Gier noch in nutzloser Anhäufung des
Geldes.
\dotfill \textit{Seite~\pageref{kap13_ab13}}\\
\item[14. Abschnitt] Der Gierige ist eine Volksplage.
\dotfill \textit{Seite~\pageref{kap13_ab14}}\\
\item[15. Abschnitt] Ihre Heuchelei.
\dotfill \textit{Seite~\pageref{kap13_ab15}}\\
\item[16. Abschnitt] Das Gold ist ihr Gott.
\dotfill \textit{Seite~\pageref{kap13_ab16}}\\
\item[17. Abschnitt] Der Gierige muss sparen, sollte es ihm auch das Leben
kosten\\
.\dotfill \textit{Seite~\pageref{kap13_ab17}}\\
\item[18. Abschnitt] Christus und seine Jünger verdammen die Gier.
\dotfill \textit{Seite~\pageref{kap13_ab18}}\\
\item[19. Abschnitt] Sünde und Gericht des Ananias und der Sapphire.
\dotfill \textit{Seite~\pageref{kap13_ab19}}\\
\item[20. Abschnitt] Wilhelm Lindals Abhandlung über diesen Gegenstand,
nachgewiesen.
\dotfill \textit{Seite~\pageref{kap13_ab20}}\\
\item[21. Abschnitt] Peter Charrons Zeugnis gegen die Gier, angeführt.
\dotfill \textit{Seite~\pageref{kap13_ab21}}\\
\item[22. Abschnitt] Abraham Cowlens witzige und scharfe Satire über dieselbe.\\
.\dotfill \textit{Seite~\pageref{kap13_ab22}}\\

\end{description}

\newpage

\section{1. Abschnitt} \label{kap13_ab1}

Ich komme nun zur Abhandlung der zweiten Hauptleidenschaft des Menschen, nämlich
der Gier\index{Gier} oder der Habsucht\index{Habsucht}, eines Übels, das wie
eine ansteckende Krankheit
in der Welt wütet und mit allen schädlichen Wirkungen begleitet ist, welche
die Menschen, sowohl einzeln als in gesellschaftlicher Verbindung, unglücklich
und elend machen. Dieses Laster ist mit dem, wovon ich eben geredet habe, mit
dem Stolz, so innig verbunden, dass man selten das eine ohne das andere
antrifft, indem Freigebigkeit den Stolzen fast ebenso gehässig als dem
Gierigen ist. Ich bestimme die Gier, die der Apostel
\textit{"`die Wurzel alles
Übels"'} nennt,\footnote{1. Timotheus 6,9+10.}
\index{Bibelstellen:!Timotheus 1 06@1. Timotheus 6)}
auf die Art, dass er in drei
Hauptzweigen erscheint, erstlich in einem Begehren unerlaubter Dinge, zweitens
in einen unerlaubten Begehren erlaubten Dinge, und drittens in
\label{ref:13_01_accumlation} \textbf{unnützer
Anhäufung des Geldes und anderer Dinge, wodurch der Gebrauch und die Wohltat
des einen oder anderen entweder einzelnen Personen oder der allgemeinen
Gesellschaft unnötigerweise entzogen wird}\index{Geld!Akkumulation}. Zuerst
werde ich das erörtern, was
die heilige Schrift darüber sagt, und die Beispiele anführen, die sie uns zur
Warnung gegen diese Laster aufstellt, und danach meine eigenen nebst einigen
Zeugnissen achtbarer Schriftsteller beibringen. Daraus wird sich dann erhellen, dass
das Kreuz Christi das Herz des Menschen ebensowohl von der Liebe zum Reichtum
als von jeder anderen Sünde, in welche er gefallen ist, befreien müsse.

\section{2. Abschnitt} \label{kap13_ab2}

Was erstlich das Begehren unerlaubter Dinge betrifft, so ist dieses
ausdrücklich von Gott selbst in dem Gesetz\index{Gott!Gesetz} verboten,
das er Moses\index{Personen:!Mose} auf dem
Berg Sinai\index{Orte:!Berge Sinai} als eine Richtschnur des Lebens, wonach
sein Volk, die Juden\index{Personen:!Juden},
wandeln sollten, mit den Worten gab:
\textit{"`Laß dich nicht gelüsten deines Nächsten
Haus. Laß dich nicht gelüsten deines Nächsten Ehefrau, noch seines Knechtes,
noch seiner Magd, noch seines Ochsen, noch seines Esels, noch alles dessen, was
dein Nächster hat."'}\footnote{2. Mose 20.}
\index{Bibelstellen:!Mose 2 20@2. Mose 20)}
Dieses bestätigte Gott durch Donner und
Blitz und andere schauerliche Zeichen\index{Zeichen!schauerliche}, um das Volk
mit desto größerer Ehrfurcht
zu erfüllen, damit es diese moralischen Vorschriften annehmen und halten, und
die Übertretung derselben desto schrecklicher finden würde. --
Micha\index{Personen:!Micha} brach zu
seiner Zeit in die laute Klage aus:
\textit{"`Sie reißen zu sich Äcker und nehmen
Häuser, welche ihnen gelüstet,"'}\footnote{Micha 2,2.}
\index{Bibelstellen:!Micha 02@Micha 2)}
sie nahmen aber auch ein
elendes Ende. Daher heißt es auch
\textit{"`Wehe denen, welchen mit Begierde gelüstet!"'}
Dieses ist der Punkt, welchen wir zu betrachten haben, und wir finden viele
merkwürdige Beispiele davon in der Schrift, von denen ich nur zwei in der Kürze
berühren will.

\section{3. Abschnitt} \label{kap13_ab3}

David\index{Personen:!David, König}, dieser sonst so gut Mann, fiel aus Mangel an
Wachsamkeit. Die Schönheit
der Ehefrau des Urias\index{Personen:!Urias Ehefrau} war eine zu harte
Versuchung für ihn, da er, von seiner
geistlichen Wache abgelockt, entwaffnet und wehrlos war. Nichts konnte ihn nun
von seinem Vorhaben mehr zurückhalten. Urias\index{Personen:!Urias} musste zu
einer der verzweifelten
Unternehmungen angestellt und einer Gefahr ausgesetzt werden, der er schwerlich
mit dem Leben entgehen konnte. Dieses geschah, um die Befriedigung der
unerlaubten Neigungen Davids auf eine Art zu beschleunigen, die nicht geradezu
das Ansehen einer offenbaren Ermordung hatte. Der Kunstgriff gelang, Urias blieb
im Gefechte, und seine Wittwe war unverzüglich Davids Ehefrau. Hieraus ging nun
Davids Habsucht offenbar hervor, aber kam er auch ungestraft\index{Gott!Strafe}
damit durch? Nein,
sein Verbrechen zog ihm schwere Gerichte und scharfe Züchtigungen zu.
"`Seine
Freude war bald in Angst und Bitterkeit des Geistes verkehrt. Seine Seele
wollte sich nicht trösten lassen, und die Fluten der Trübsal gingen über sein
Haupt. Seine Seele verschmachtete in ihm. Er war wie in Kot
versunken\footnote{\texttt{Heute würde man sagen: 'Er steckte in der Scheiße'.}},
er schrie, er weinte,
seine Augen glichen Tränenquellen. Seine Schuld lag schwer auf ihm, seine
Sünden, die rot wie Scharlach, waren, mussten weiß wie Schnee gewaschen
werden, oder er war auf ewig verloren."'\footnote{Psalm 51) 77) 42,7.) 69,2+14) 6,6+7.}
\index{Bibelstellen:!Psalm 051@Psalm 51)}
\index{Bibelstellen:!Psalm 077@Psalm 77)}
\index{Bibelstellen:!Psalm 042@Psalm 42)}
\index{Bibelstellen:!Psalm 069@Psalm 69)}
\index{Bibelstellen:!Psalm 006@Psalm 6)}
Endlich trug jedoch seine Reue\index{Reue} den Sieg davon. -- Siehe! Was für
traurige Folgen diese Art der Lüsternheit\index{Lüsternheit} und Habsucht nach
sich zieht, was für
Elend und Kummer sie begleitet. Und o! Möchte doch allen, die sich von einer
solchen Habsucht hinreißen lassen, ein Gefühl von Davids Reue und Kummer tief in
ihre Seelen dringen, damit sie auch Davids Heil und Erlösung erführen!
\textit{"`Richte
mich wieder auf!"'} schrie dieser gute Mann, denn er war sich ohne Zweifel eines
vorigen besseren Zustands bewusst. Ja, möge sein Fall auch anderen zur Warnung
dienen und sie lehren, in heiliger Furcht\index{Furcht!heilige} und Wachsamkeit
zu leben, damit sie
nicht auch sündigen und fallen. Denn David fiel durch Unbedachtsamkeit und
Unwachsamkeit, er stand nicht auf seiner Hut und hatte sich dem Kreuze
entfernt. Das Gesetz in seinem Herzen\index{Gebote!Gesetz im Herzen} war in dem
unglücklichen Augenblicke nicht
seines Fußes Leuchte und ein Licht auf seinem Wege. Er hatte seinen sicheren
Zufluchtsort, seine feste Burg verlassen, er sah sich überrascht, und gerade
diesen Zeitpunkt seiner Unwachsamkeit benutzte der Feind und überwältigte ihn.

\section{4. Abschnitt} \label{kap13_ab4}

Das zweite Beispiel liefert uns die Geschichte von
Naboths\index{Personen:!Nabot} Weinberg.
Ahab\index{Personen:!Ahab} und Isebel\index{Personen:!Isabel} begehrten ihn, und
derselbe Geist, der ihnen diesen unerlaubten
Gedanken einflößte, gab ihnen auch Mittel und Wege an die Hand, ihn auszuführen.
Naboth musste sterben, weil er ihnen seinen Weinberg nicht verkaufen wollte. Um
ihren Zweck zu erreichen, beschuldigten sie den schuldlosen Mann der
Gotteslästerung, und fanden auch zwei falsche Zeugen, zwei Kinder
Belials\index{Personen:!Belials Kinder}, die
gegen ihn zeugeten. So musste Naboth unter dem Vorwande des Namens Gottes, und
unter dem Schein des reinsten Eifers für seine Ehre, das Leben verlieren. Er
wurde zu Tode gesteinigt, und sobald Isebel die Nachricht von seinem Tod
erhielt, sagte sie zu Ahab:
\textit{"`Steh auf, und nimm den Weinberg ein, denn Naboth
lebt nicht mehr, sondern ist tot."'} Gott aber verfolgte beide mit strengem
Gericht. \textit{"`Auf der Stelle, da die Hunde das Blut Naboths geleckt
haben,"'}
sagte Elia\index{Personen:!Elia} im Namen des Herrn, \textit{"`sollen auch Hunde
dein Blut lecken. Ich will
Unglück über dich bringen und deine Nachkommen wegnehmen."'} Und von seiner
Ehefrau, Isebel, der Mitschuldigen an dieser Habsucht und Mordtat, fügte Elia
hinzu:
\textit{"`Die Hunde sollen ihr Fleisch an den Mauern von Jesreel fressen."'}
Hier
sehen wir die Schändlichkeit dieser Art Habsucht und auch die gerechte
Bestrafung\index{Bestrafung!gerechte} derselben. Möge es jeden abschrecken, der
ein Verlangen nährt,
unerlaubte Dinge, die Rechte und Besitzungen anderer, an sich zu bringen! Denn
Gott ist gerecht, der ein solches Begehren unfehlbar endlich bestrafen wird.
Jedoch ist vielleicht die Anzahl solcher Menschen, die nach dem unerlaubten
Besitz des Eigentums eines anderen trachten, nicht groß, denn viele tun es
darum nicht, oder scheuen sich, ihr Begehren laut werden zu lassen, weil sie die
Gesetze fürchten. Hingegen gibt es nur zu viele von einer anderen Gattung,
welche gegen die eben erwähnte Art der Habsucht zu eifern scheinen, um durch
ihren vorgeblichen Abscheu vor derselben die Beschuldigung einer anderen Gier
von sich abzulenken, die wir jetzt betrachten wollen.

\section{5. Abschnitt} \label{kap13_ab5}

Diese zweite Art der Habsucht oder der Gier, welche die allgemeinste ist,
besteht in einem unerlaubten Begehren erlaubter Dinge, vornehmlich des

 Reichtums. Geld zu besitzen, ist an sich eine erlaubte Sache, aber die
\textit{"`Geldliebe oder die Gier, ist die Wurzel allen Übels,"'} wenn anderes
der Mann
Gottes war redet. \label{ref:13_05_reichtum} \textbf{Auch ist
Reichtum\index{Reichtum} nicht unerlaubt, allein diejenigen,
\textit{"`die da reich werden wollen, fallen in mancherlei Versuchungen und
Stricke, und in
viele törichte und schädliche Lüste, welche sie ins Verderben
bringen,"'}}\footnote{1. Timotheus 6,9+10.}
\index{Bibelstellen:!Timotheus 1 06@1. Timotheus 6)}
wenn derselbe Apostel auch hierin die
Wahrheit sagt. Er nennt den Reichtum ungewiss, um die Torheit und die Gefahr
derer zu zeigen, die mit ihren Herzen danach trachten. In Gottes Augen ist
Gier ein Gräuel, und er hat allen, die sich desselben schuldig machen, große
Gerichte angekündigt\index{Gott!Vergeltung}. Dem Volk
Israel\index{Volk Israel} warf Gott ehemals Gier als eine
Ursache seiner Gerichte vor.
\textit{"`Ich war zornig über die Untugend ihres Geizes,
und schlug sie"'} sagt der Herr,\footnote{Jesaja 57,17.}
% Bibelstelle geprüft: Passt!
\index{Bibelstellen:!Jesaja 57)}
und an einem anderen Ort:
\textit{"`Denn sie geizten
 allesamt, Kleine und Große, und beide Propheten und
Priester, lehren allesammt falschen Gottesdienst. Darum will ich ihre Frauen den
Fremden geben, und ihre Äcker denen, die sie verjagen werden."'}\footnote{Jesaja 6,13) 8,10.}
% Bibelstelle geprüft: passt nicht
\index{Bibelstellen:!Jesaja 06@Jesaja 6)}
\index{Bibelstellen:!Jesaja 08@Jesaja 8)}
Auch klagte Gott,
"`dass ihre Augen und ihr Herz nur auf ihre Gier
gerichtet waren."'\footnote{Jesaja 22,17.}
\index{Bibelstellen:!Jesaja 22)}
% Bibelstelle geprüft: passt nicht!
% Was machen wir jetzt? Besprechen wir am Telo.
Diese Klagen über ihre Gier erneuerte
und wiederholte Gott durch Ezechiel\index{Personen:!Ezechiel} mit den Worten:
\textit{"`Und sie werden zu dir
in die Versammlung kommen, und vor dir sitzen, als mein Volk, und werden deine
Worte hören, aber nicht danach tun, sondern sie werden dich anpfeifen und
gleichwohl weiterleben nach ihrer Gier."'}\footnote{Ezechiel 33,31.}
\index{Bibelstellen:!Ezechiel 33)}
Darum forderte
Gott es auch bei der Wahl der Oberen als eine Haupteigenschaft ihres Charakters,
\textit{"`dass sie die Gier hassen sollten,"'} weil er das Elend voraussah, das
in einer
Gesellschaft oder in einem Staate daraus erfolgen würde,\index{Korruption}wenn
gierige Menschen
die oberste Gewalt in den Händen hätten, indem diese sich nur zu leicht durch
Eigennutz verleiten lassen, ihre Privatabsichten auf Kosten der öffentlichen
Wohlfahrt\index{Wohlfahrt!öffentliche} zu befördern. David flehte,
"`dass sein Herz zu den Zeugnissen des
Herrn und nicht zur Gier geneigt werden möchte."'\footnote{Psalm 119,36.}
\index{Bibelstellen:!Psalm 119)}
Und Salomo sagt: "`Wer die Gier hasst, der wird lange leben,"'\footnote{Weisheit Salomos 28,16.}
\index{Bibelstellen:!Weisheit Salomos 28)}
und er rechnet es für einen Fluch, wenn jemand ihm ergeben ist. Lukas
warf den Pharisäern\index{Pharisäer} das Laster der Gier als ein
Zeichen ihrer Gottlosigkeit\index{Zeichen!der Gottlosigkeit}
vor, und Christus sagt in diesem Evangelium zu seinen Jüngern:
\textit{"`Geht hin und
hütet euch vor der Gier,"'} dabei führt er einen Grund an, der trefflichen
Unterricht enthält: \textit{"`Denn niemand,"'} sagt er,
\textit{"`lebt davon, dass er viele Güter hat"'.}\footnote{Lukas 12,15.}
\index{Bibelstellen:!Lukas 12)}
Bei einer anderen Gelegenheit stellt Christus die Gier mit Ehebruch, Mord und
Gotteslästerung in eine Reihe.
\footnote{Matthäus 7,21+22.}
\index{Bibelstellen:!Matthäus 07@Matthäus 7)}
Man muss sich daher nicht wundern, wenn der Apostel Paulus dieses
Laster so häufig rügt. In seiner Epistel an die Römer stellt er die Gier mit
aller Ungerechtigkeit zusammen.\footnote{Römer 1,29.}
\index{Bibelstellen:!Römer 01@Römer 1)}
An die Epheser schreibt er,
\textit{"`das Geiz unter ihnen auch nicht einmal genannt werden müsse."'}\footnote{Epheser 5,3.}
\index{Bibelstellen:!Epheser 05@Epheser 5)}
Den Kolossern empfiehlt er, ihre Glieder zu töten, und indem er
verschiedene Sünden, als Hurerei, Unreinigkeit, etc. benennt, beschließt er
die Aufzählung derselben mit der Gier, den er Abgötterei\index{Abgötterei}
nennt, eine Sünde\index{Sünde},
die, wie uns bekannt ist, Gott aufs höchste beleidigt. Ja, dieser Apostel nennt
die Gier oder die Liebe zum Gelde \textit{"`eine Wurzel alles Bösen,"'}
\index{Böse!Wurzel des Böse} und sagt uns, dass
einige, die danach trachteten, \textit{"`vom Glauben abgeirrt sind, und sich
selbst
mit vielen Schmerzen durchstochen haben."'} \textit{"`Denn diejenigen, welche
reich
werden wollen,"'} sagt er, \textit{"`fallen in Versuchung und Stricke, und in
viele
törrichte und schädliche Lüste."'} Darum rief er seinem geliebten Freunde
Timotheus zu:\\textit{"`Aber du Gottesmensch! Fliehe! Jage der
Gerechtigkeit
nach, der Gottseligkeit, dem Glauben, der Liebe, der Geduld und der
Sanftmut."'}\footnote{1. Timotheus 6,9-11.}
\index{Bibelstellen:!Timotheus 1 06@1. Timotheus 6)}

\section{6. Abschnitt} \label{kap13_ab6}

Petrus\index{Personen:!Petrus} hatte dieselbe Gesinnung, denn er gab den Geiz
als ein Hauptkennzeichen
aus, woran die falschen Propheten\index{Prophet!falscher} und Lehrer,
die unter den Christen entstehen
würden, zu erkennen wären, \textit{"`welche,"'} sagt er, \textit{"`aus Geiz mit
erdichteten
Worten an euch hantieren (oder Handel mit euch treiben) werden."'}\footnote{2. Petrus 2,3.}
\index{Bibelstellen:!Petrus 2 02@2. Petrus 2)}
Endlich hinterlässt uns der Verfasser der Epistel an die Hebräer
unter anderen wichtigen Gegenstände, die er mit warmen Eifer einschärft, noch die
Ermahnung: \textit{"`Der Wandel sei ohne Gier."'} Doch lässt er es bei dieser
allgemeinen
Vorschrift nicht bewenden, sondern fügt noch hinzu:
\textit{"`Und lasst euch begnügen an
dem, das da ist, denn Gott hat gesagt: Ich will dich nicht verlassen, noch
versäumen."'}\footnote{Hebräer 13,5.}
\index{Bibelstellen:!Hebräer 13)} \label{ref:13_06_reichtum_unersaettlichkeit}
%hier habe ich doch mal was am label korrigiert.
\textbf{Können wir nun aber hieraus nicht auch
schließen, dass die Ungenügsamen\index{Ungenügsame}, die nach
Reichtum\index{Reichtum} trachten, Gott verlassen?
Dieser Schluss scheint vielleicht hart, allein er folgt ganz natürlich daraus, da
solche Menschen, die mit dem, was sie haben, nicht zufrieden sind, sondern immer
mehr verlangen, und daher, wenn es nur möglich ist, auch reich zu werden suchen,
offenbar nicht in dem Vertrauen und Aufsehen auf die göttliche Vorsehung leben,
zu welcher sie ermahnet sind, und ihnen folglich Gottseligkeit mit Genügsamkeit
verbunden, kein großer Gewinn zu sein scheint.}

\section{7. Abschnitt} \label{kap13_ab7}

\index{Maßlosigkeit}\index{Unersättlichkeit} Und wahrlich! \textbf{es ist eine
Schande für den Menschen, besonders für den
religös Gesinnten, dass er oft nicht weiß, wann er genug hat, wann er mit Gewinnen
aufhören und sich begnügen soll}, und dass er, obgleich ihm Gott eine reiche
Ernte
oder Einnahme nach der anderen zufließen lässt, doch diese Wohltaten nicht als
Beweggründe ansieht, \label{ref:13_07_reichtum_genuegsamkeit}
\textbf{sich endlich aus dem Gewühl der Welt zurückzuziehen,
sondern dieselben vielmehr als Einladungen betrachtet, sich noch tiefer in die
Geschäfte des Lebens einzulassen,} als wenn er, je mehr er habe, auch desto mehr
erwarten könne. \textbf{Er erneuert daher seine Begierde und strengt sich mehr
als
jemals an, um seinen Teil davon zu tragen, so lange es noch etwas zu erjagen
gibt. Gerade als ob Unruhe und nicht Abgeschiedenheit\index{Abgeschiedenheit},
Gewinnsucht\index{Gewinnsucht} und nicht
Genügsamkeit\index{Genügsamkeit} die Pflichten und die Quellen des
Trostes\index{Trost} eines Christen wären.}
Möchte doch dieses besser erwogen werden! Denn da die Gier den Augen und der
Wachsamkeit der Gesetze mehr als andere Laster entzogen ist, so ist er aus
\index{Gebote!Mangel an Gesetzen} Mangel an gesetzlicher Aufsicht und
Einschränkung desto gefährlicher.

\medskip

\textbf{Es ist nicht zu leugnen, dass sehr viele Menschen nicht des Unterhalts,
sondern
des Reichtums wegen Geld zu gewinnen suchen.} Es gibt einige, die eifrig
danach trachten, es aber auch bald wieder verschwenden\index{Verschwendung},
wenn sie es erlangt
haben. Dieses ist allerdings auch strafbar, allein bei weitem nicht so
verwerflich, als wenn jemand das Geld bloß um der Gelder willen liebt, welches
in der Tat eine der niedrigsten Leidenschaften\index{Leidenschaft!niedrigste}
ist, wovon das Herz eines
Menschen gefesselt werden kann. Dies ist ein größeres, und für die Seele
verderblicheres Laster, als in dem ganzen Verzeichnissen der sündlichen
Begierden des Menschen zu finden ist. Würde dieses recht erwogen, so müsste es
jeden zu der ernsten Untersuchung anleiten, inwiefern die Versuchung zur
Geldliebe einen Eingang in sein Herz gewonnen habe, und zwar um so mehr, da die
Fortschritte, die dieses Laster in dem menschlichen Gemüt macht, fast
unmerklich und seine Wirkungen daher desto gefährlicher sind. Tausende halten
vielleicht eine solche Warnung gegen die Gier bei ihnen am unrechten Ort, und
sind doch nichtsdesto weniger dieses Übels schuldig. Wie kann es auch anders
sein, wenn man sieht, wie oft diejenigen, die anfänglich wenig oder nichts
besaßen, nachdem sie sich Summen erworben haben, noch mit derselben Sorge und
ängstlichen Scharfsichtigkeit, mit welcher sie dieselben zusammenscharrten,
beständig fortwirken, um diese Summen zu vermehren, zu verdoppeln oder zu
verdreifachen. \textbf{Heißt dieses ein ruhiges und angenehmes Leben führen?
Oder heißt
dies auch wohl reich sein? Sehen wir nicht, wie diese Leute so früh auf sind,
und erst so spät zur Ruhe kommen? Wie ihre Gedanken beständig auf ihre Geschäfte
gerichtet und mit einer Menge von Gegenständen angefüllt sind? Wie sie dabei hin
und her eilen als ob sie beschäftigt wären, einem unschuldig Verurteilten das
Leben zu retten? Und dieses alles um eine unersättliche Leidenschaft zu
befriedigen, die ebenso verderblich für die Menschen als Gott zuwider ist, der
den Reichtum dazu gibt, dass man ihn wohl anwenden, nicht aber, damit man mit
seinem Herzen daran hängen solle, denn darin besteht der Missbrauch, den die
Menschen davon machen.} Soll endlich aber ein so beständiges Sorgen, scharfes
Nachdenken und tätiges Streben, um Geld zu gewinnen, bei denen, die zehnmal
mehr besitzen, als sie anfänglich hatten, und viel mehr, als sie verzehren
können oder nötig haben, nicht offenbar Geldliebe verraten, so weiß ich doch
nicht, wie jemand seine Liebe zu irgend einer Sache auf eine andere Art
deutlicher an den Tag legen könnte.

\section{8. Abschnitt} \label{kap13_ab8}

Bei obrigkeitlichen Personen ist der Geiz der Regierung\index{Regierung}
immer gefährlich , da er
zu Bestechungen\index{Bestechung} verleitet. Darum mussten diejenigen, die Gott
verordnete, solche
sein,
\textit{"`die ihn fürchteten, und die Gier hassten."'} Dann
schadet\index{Gemeinwohl schaden} er der
menschlichen Gesellschaft vornehmlich dadurch, dass er den alten
Geschäftsleuten\index{Geschäftsmann}
eingibt, die jungen Anfänger nicht aufkommen zu lassen.
\index{Ausbeutung}\label{ref:13_08_reichtum_sklaverei}
\textbf{Und eine Hauptursache,
warum viele Menschen zu wenig erwerben und genötigt sind, wie
Sklaven\index{Sklave} zu
arbeiten, um ihre Familien zu ernähren und ihr Auskommen zu finden, liegt
unstreitig darin, dass die Reichen nicht nachlassen, zu scharren, sondern immer
noch reicher werden wollen, und deswegen auch die kleineren Erwerbsquellen der
geringen Klasse an sich ziehen oder verstopfen. Es wäre daher zu wünschen, dass
man einen Maßstab festsetzen möchte, nach welchem jeder die Dauer und Ausdehnung
der Grenzen seiner Geschäfte beistimmen\index{Gewinnstreben begrenzen}, und nach
deren Erreichung er sie dann
unter diejenigen seiner Untergebenen, die es verdienten, verteilte. Dieses würde
den Jüngeren die Mittel gewähren, sich ihren Lebensunterhalt zu erwerben, und
den Alten Zeit verschaffen, darauf zu denken, wie sie eine Welt
verlassen\index{Welt!verlassen}
wollen, in der sie so geschäftig gewesen sind, und sich um ihr Los in jener Welt
zu bekümmern, um welches sie bisher so unbekümmert waren.}

\section{9. Abschnitt} \label{kap13_ab9}

Für die Regierungen\index{Regierung} entspringt aus der Gier der Untertanen
noch der Nachteil,
dass er diese verleitet, ihre Oberen durch Einführung oder Verheimlichung
schlechter oder verbotener Waren, durch Bewirkung oder Beförderung des Umlaufs
falscher Münzen, durch Schleichhandel, unrichtiges Maaß oder Gewicht, und auf
viele andere Weise zu beeinträchtigen und zu betrügen. \index{Betrug}


\section{10. Abschnitt} \label{kap13_ab10}

Nicht selten hat auch die Gier schon die zerstörendsten Übel in Familien
angerichtet. So sind, zum Beispiel dadurch, dass Güter oder Gelder den Händen solcher
Menschen anvertrauet wurden, deren Gier sie vermochte, einen eigennützige und
ungerechten Gebrauch davon zu machen, schon oft große Verluste und
Unterdrückungen entstanden. Ja, es war nur zu oft schon der Fall, dass solche
geizige Verwalter das Vermögen anderer, den rechtmäßigen Eigentümern, selbst
mittels des Geldes, das sie ihnen hätten auszahlen sollen, den Besitz ihres
Vermögens streitig machten, oder dasselbe an sich zogen.

\section{11. Abschnitt} \label{kap13_ab11}

Doch dies ist noch nicht alles. Die Gier macht die Menschen sogar zu Verrätern
an der Freundschaft. Denn, wo Bestechung\index{Bestechung} angewendet werden
muss, um eine
Schandtat zu verüben, oder jemand ins Elend zu stürzen, da lässt sich der
Gierige unfehlbar dadurch gewinnen, auch an seinem Freunde verräterisch zu
handeln. Ja, die Gier mordet oft Leib und Seele. Diese, indem er das Leben auf
Gott in ihr zerstört, denn, wo die Geldliebe das Gemüt des Menschen
beherrscht, da löscht sie alle Liebe zu etwas Besserm aus. Und
Mörder\index{Mörder} des Leibes
wird der Gierige ebenfalls aus Liebe zum Geld, die ihn zu Meuchelmord,
Vergiftungen und falschen Zeugnissen, usw. verleitet. -- Ich will zum
Beschluss noch die Sünde und das Schicksal zweier Gieriger, des
Judas\index{Personen:!Judas} und des
Simon Magus\index{Personen:!Simon Magnas}, berühren.

\medskip

In des Judas Herzen war der Samen der Religion unter die Dornen gefallen, seine
Geldliebe hatte ihn erstickt. Aus Stolz und Ärger suchten die Juden Jesus zu
töten. Allein es gelang ihnen nicht, bis die Gier, ihnen die Hände bot, ihr
Vorhaben auszuführen. Sie sahen, dass Judas den Geldbeutel trug, und vermuteten, dass
er auch wohl das Geld liebte. Sie beschlossen daher, ihn auf die Probe zu
setzen, und taten es auch. Man war über den Preis einig, und Judas
überlieferte seinen Herrn und Meister, der ihm nie etwas zu Leide getan hatte,
in die Hände seiner grausamsten Feinde. Um ihm jedoch Gerechtigkeit widerfahren
zu lassen, müssen wir auch bemerken, dass er das Geld, welches er für seine
Verräterei bekommen hatte, wieder zurückgab, und, um Rache an sich selbst zu
üben, hinging und sich erhängte\index{Personen:!Judas!Tod des}. Eine gottlose
Tat! Ein scheußliches Ende! Aber
was sagt ihr nun? Ihr Gierigen? Was denkt ihr von euerem Bruder Judas? -- War er
nicht ein böser Mensch? -- Handelte er nicht entsetzlich gottlos? -- O gewiss!
Werdet ihr sagen. Würdet ihr so gehandelt haben? -- Nein! Auf keinen Fall! Aber
die Juden\index{Personen:!Juden} sagten dasselbe von denen, welche die
Propheten\index{Prophet} getötet hatten, und
dennoch kreuzigten sie Jesus, den Sohn Gottes, der gekommen war, sie zu
erretten\index{Jesus!der Retter}
und selig zu machen, und es auch würde getan haben, wenn sie ihn angenommen und
den Tag ihrer Heimsuchung\index{Heimsuchung} nicht verworfen hätten. Lasdst euch
die Augen salben und vom Staub der Erde reinigen, damit ihr desto besser in euerem Gewissen
lesen und desto deutlicher erkennen könnet, ob nicht auch ihr, aus Geldliebe,
den Gerechten in euerem Inneren verraten habt und dadurch Bruder des Judas in
seiner Ungerechtigkeit geworden seid? -- Ich rede für die Sache Gottes und gegen
einen Götzen\index{Götzen}, darum ertraget mich mit Geduld. -- Habt ihr nicht
mit euerem
Trachten nach eueren geliebten Reichtümern dem Geiste Christi widerstrebt? Ja,
habt ihr nicht diesen guten Geist in euch selbst unterdrückt? O!
\textit{"`Untersucht
euch selbst, prüft euch selbst oder erkennt ihr euch selbst nicht? Denn wenn
Jesus Christus nicht in euch ist,"'}
wenn er nicht in euch die Herrschaft hat,
und über alles in der Welt von euch geliebt wird, \textit{"`so seid ihr
Untüchtige,"'}\index{Selbst!-kritik}\footnote{2. Korinther 13,5.}
\index{Bibelstellen:!Korinther 2 13@2. Korinther 13)}
oder Verworfene und befindet euch in einenm
verlorenen Zustande.

\section{12. Abschnitt} \label{kap13_ab12}

Der andere Gierige ist Simon\index{Personen:!Simon Magnas}, der Zauberer, der auch ein
Gläubiger war, allein
sein Glaube hatte, seines Geizes wegen, nicht tiefe Wurzel schlagen können. Er
hätte gern mit Petrus einen Handel abgeschlossen, damit
er die göttliche Gabe
hätte wieder verkaufen und auf diese Art einen guten Handelszweig daraus machen
können. Er schloß von sich auf Petrus, den er ohne Zweifel für einen Mann hielt,
der den Kniff, die Menschen zu berücken, noch besser als er selbst verstände,
der doch in Samaria\index{Orte:!Samaria} für die große Kraft Gottes galt, bis
die wahre Kraft Gottes,
durch Philippus\index{Personen:!Philippus} und Petrus offenbar wurde, und das
Volk aus dem Irrtum riss.
Aber was antwortete Petrus dem Simon, und was für ein Unheil sprach er über ihn
aus? \textit{"`dass du verdammt werdest mit deinem Geld!"'} sagte der Apostel,
\textit{"`du wirst
weder Teil noch Anfall (Los) an diesem Worte haben, denn ich sehe, du bist
voll bitterer Galle und verknüpft mit Ungerechtigkeit."'}\footnote{Apostelgeschichte 8,9-24.}
In der Tat ein schauderhafter Ausspruch!

\medskip

Außerdem dient die Gier auch noch der Verschwendung\index{Verschwendung} und
nimmt oft seinen
Ursprung aus derselben. Denn wenn einige Menschen viel haben, so lassen sie viel
ausgehen, und machen sich durch üppige Verschwendung wieder arm. Solche sind
dann immer gierig nach Gewinn, um desto mehr ausgeben zu können, welches jedoch
Mäßigkeit verhüten würde. Und wollten nur die Menschen in den Ausgaben für ihr
Essen\footnote{\texttt{'ihren Tisch' ersetzt durch 'ihr Essen'.}}, für ihre
Häuser, Möbel\index{Möbel}, Kleider\index{Kleider}, für ihre
Spiele\index{Spiele} und andere Vergnügungen
sich einschränken, oder könnten sie durch genaue Anwendung weiser Gesetze und
durch eine bessere Erziehung\index{Erziehung} davon abgehalten werden, so würden
sie nicht so
sehr der Versuchung\index{Versuchung} ausgesetzt sein, mit solchem Eifer nach
Geld zu trachten,
als sie gewöhnlich tun, weil es ihnen dann an den
Gelegenheiten\index{Gelegenheiten!Mangel an}, es zu
verschwenden, mangeln würde, denn der Geizhälse, die das Geld nur um des Geldes
willen lieben, gibt es doch wohl eigentlich nur wenige.

\section{13. Abschnitt} \label{kap13_ab13}

Dieses leitet mich zur Betrachtung der nidrigsten Gattung der Gier, welche die
abscheulichste von allen ist, nämlich der Anhäufung des Geldes oder der
Aufbewahrung desselben, um weder sich noch andere damit zu nutzen\index{Reichtum!unnützer}.
Die Menschen,
welche von dieser Art der Gier gefesselt sind, gehören zu denen, von welchen
Salomo sagt,
\textit{"`dass sie den großem Gute arm sind,"'}\footnote{Weisheit Salomos 13,7.}
welches in den Augen Gottes eine große Sünde\index{Sünde!große} ist. Gott klagt
über diejenigen,
\textit{"`die den Raub der Armen in ihren Häusern haben."'} Er sagt,
\textit{"`dass sie sein Volk
zertreten und die Person der Elenden zerschlagen."'}\footnote{Jesaja 3,13+14.}
\index{Bibelstellen:!Jesaja 03@Jesaja 3)}
Diese grämen sich, weil sie das ihnen Geraubte nicht wieder sehen. Aber der Herr
segnet die, \textit{"`welche sich der Dürftigen annehmen,"'}
\footnote{Psalm 41,1.}
\index{Bibelstellen:!Psalm 041@Psalm 41)}
und er
befiehlt allen,
\textit{"`ihr Herz nicht gegen ihren armen Bruder zu verhärten, und ihre
Hand nicht vor ihm zu verschließen, sondern ihm gern zu geben und zu
leihen,"'}\footnote{5. Mose 15,7-9.}
\index{Bibelstellen:!Mose 5 15@5. Mose 15)}
und zwar nicht allein dem geistlichen\index{Bruder!geistlicher},
sondern auch dem natürlichem Bruder\index{Bruder!natürlicher}.

\medskip

Der Apostel Paulus gab dem Timotheus vor dem Angesicht Gottes und Jesu Christi
diesen Auftrag:
\textit{"`Den Reichen dieser We1t gebiete, dass sie nicht stolz sind,
auch nicht auf den ungewissen Reichtum hoffen, sondern auf den lebendigen Gott,
der uns alles reichlich zu genießen gibt, dass sie Gutes tun und an guten
Werken reich werden."'}\footnote{1. Timotheus 6,17+18.}
\index{Bibelstellen:!Timotheus 1 06@1. Timotheus 6)}
Reichtümer sind nur zu leicht
im Stande, das Herz zu verderben\index{Geld!verdirbt den Charakter}, was aber
dagegen schützt, ist die Ausübung der
Mildtätigkeit\index{Mildtätigkeit}. Wer sie nicht dazu anwendet, der gebraucht
sie nicht zu dem
Zweck, wozu sie ihm gegeben wurden, und liebt sie nur um ihrer selbst willen,
nicht des Nutzens\index{Geld!Nutzen von} wegen, den er damit stiften kann. So
ist der Gierige mitten
unter seinen Schätzen arm. Er leidet Mangel, weil er sich vor Ausgaben fürchtet,
und seine Furcht\index{Furcht!der Reichen} nimmt nach dem Verhältnisse zu, wie
seine Hoffnun, zu gewinnen,
sich vermehrt. Auf diese Weise dient die Freude seines Herzens ihm zur Qual. Er
gleicht unter allen am meisten dem Knecht, der sein Pfund im Schweißtuch
verbag, denn er hat seine Pfunde, seine Schätze, in Säcken verwahrt, im Gewölbe,
unter dem Fußboden, hinter den Wänden, oder auch in Schuldverschreibungen und
Pfandbriefen versteckt, wo sie verborgen und wie vergraben liegen, ohne irgend
jemand nützlich zu sein.\footnote{\texttt{Anspielung auf Matthäus 25,14-30.}}
\index{Bibelstellen:!Matthäus 25)}

\section{14. Abschnitt} \label{kap13_ab14}

\label{ref:13_14_reichtum_schaden}
\index{Volkswirtschaft}\textbf{Ein solcher Gieriger ist ein wahres Ungeheuer,
ohne Menschengefühl, und, gleich
den Erdpolen, beständig kalt. Er ist ein Feind des Staates\index{Staat!Staatsfeind}, denn er saugt das
Geld aus. Ein Krankheitsstoff im politischen Körper, der den Umlauf des Blutes
hemmt, und daher durch irgend ein Reinigungsmittel des
Gesetzes\index{Gesetz} fortgeschafft
werden sollte, denn dieses Laster greift das Herz an und zerstört den
Zusammenhang aller Glieder.} -- Der Gierige haßt alle nützlichen
Künste\index{Kunst} und
Wissenschaften\index{Erkenntnis!Wissenschaft} als überflüssige Dinge, aus
Furcht, dass die Erlernung derselben
ihm etwas kosten würde. \index{Geiz!zerstörerischer}Daher ist sein Herz eben so
sehr als sein Geldbeutel
gegen Emfindsamkeit und Kunstfleiß verschlossen. \label{ref:13_14_reichtum_einsturz}
\textbf{Er lässt Häuser einstürzen, um
nur nicht die Ausgaben für Reparaturen zu haben. Und was seine schmale Kost,
seine abgetragenen Kleider und seine schlechten Möbel betrifft, so rechnet er
sich diese Stücke zur Mäßigkeit an. O was für ein Ungeheuer ist ein solcher
Mensch, der aus Liebe zum Geld, aber nicht aus Liebe zu Christo, das
Kreuz\index{Kreuz}
gegen sich selbst aufnehmen kann.}

\section{15. Abschnitt} \label{kap13_ab15}

\label{ref:13_15_Kapitalisten_kritik}
\textbf{Doch macht er auf seine Weise auch Anspruch auf Religion, denn er klagt
unaufhörlich über die Verschwendung\index{Verschwendung} der Menschen, um seinen
Geiz dadurch zu
bemänteln. Sieht er jemand eine kostbare Salbe auf das Haupt eines guten
Menschen ausgießen, so ist er gleich bereit, an die Dürftigkeit der Armen zu
erinnern, um seine Sparsamkeit zu zeigen und gerecht zu scheinen.\footnote{\texttt{Anspielung auf Markus 14,3-9.}}
\index{Bibelstellen:!Markus 14)}
Kommen die
Armen\index{Menschen!arme} aber zu ihm, so weiß er seinen Mangel an
Mildtätigkeit\index{Mildtätigkeit!Mangel an} damit zu bedecken,
dass er entweder die Gegenstände des Mitleids für unwürdig erklärt, oder auf die
Ursachen ihrer Armut\index{Armut!Ursachen} anspielt, oder vorschüzt, er könne
sein Geld auf solche,
die es besser verdienten, und zu weit edleren Zwecken verwenden.} -- Er, der nur
äußerst selten seine Börse öffnet, damit er nichts daraus verliere.

\section{16. Abschnitt} \label{kap13_ab16}

\index{Armut!der Reichen}Ein solcher Mensch ist elender als der Allerärmste,
denn er lebt in beständiger
Furcht, das zu verlieren, was ihm doch keinen Genuss gewährt, wohingegen die
Armen sich nicht fürchten dürfen, etwas zu verlieren, dessen sie sich nicht
erfreuen. So ist er arm durch Überschätzung seines Reichtums, und gewiss ist
er elend, da er mit vollem Geldbeutel in einem Speisehause Hunger leidet. Doch
wer weiß, ob er nicht, da das Geld sein Gott ist, vielleicht glaubt, es sei
unnatürlich, den Gegenstand seiner Anbetung zu verzehren!

\section{17. Abschnitt} \label{kap13_ab17}

\index{Gesundheit!aus Geiz ruinieren} Was endlich dieses Laster noch
verabscheuungswürdiger macht, ist, dass es dem
Menschen das Leben verkürzt. Ich habe selbst einige gekannt, die durch ihre
% hier erlaube ich mir mal, diese Mega-Sätze etwas zu trennen.
% Werde ich jetzt öfter machen.
Geldbegierde sich vor der Zeit ins Grab brachten. Diese konnten, ihrem
Grundsatz getreu, sich nicht entschließen, die Ausgabe für einen
Arzt\index{Arzt} zu
bewilligen, der vielleicht den armen Sklaven\index{Sklave} das Leben
gerettet hätte, und
starben also, bloß um Kosten zu sparen,~-- eine Standhaftigkeit, die sie zu
Märtyrern\index{Märtyrer!der Geldliebe} der Geldliebe erhebt!

\section{18. Abschnitt} \label{kap13_ab18}

Lasst uns nun aber auch sehen, was für Beispiele wir in der heiligen Schrift
antreffen, die solchen niederträchtigen
Schatzsammlern\index{Schatzsammler} und Geldverbergern zur
Bestrafung\index{Bestrafung} dienen. Wir finden, dass einst ein gutscheinender
\textbf{Jüngling zu Christo
kam, und nach dem Weg zum ewigen Leben fragte. Christus sagte ihm,
\textit{"`Er kenne ja die Gebote."'} Seine Antwort war:
\textit{"`er habe sie von Jugend auf gehalten."'} Es
scheint, er war kein ausschweifender Mensch.} -- Dies sind auch die Geizigen
gewöhnlich nicht, weil Ausschweifungen\index{Ausschweifung} mit Kosten verbunden
sind. -- \textbf{Doch
Christus erwiderte:
\textit{"`Eins fehlt dir noch. Geh, verkaufe alles, was du hast,
und gib es den Armen, so wirst du einen Schatz im Himmel haben, und dann komm, folge mir nach
und nimm das Kreuz auf."'} Christus griff ihn auf der
empfindlichsten Stelle an,} er traf sein Herz, das er durchschaute, und nun
zeigte es sich, inwiefern er das erste Gebot: \textit{"`Gott über alles zu
lieben,"'}
gehalten hatte, denn wir finden, \textit{"`der Jüngling war traurig und ging
betrübt
hinweg"'} und die Ursache seiner Traurigkeit, die uns dabei angegeben wird, war
die,
\textit{"`dass er viele Güter hatte."'}\footnote{Markus 10,19-23.}
\index{Bibelstellen:!Markus 10)}
Bei ihm trafen zwei einander entgegenstehende Wünsche zusammen, der eine war auf den Besitz des
Reichtums, der andere auf die Erlangung des ewigen Lebens\index{Leben!ewiges}
gerichtet, und
welcher von beiden behielt die Oberhand? -- Ach! Leider war seine Anhänglichkeit
an die irdischen Güter stärker als sein Verlangen nach dem ewigen Leben. Aber
was bemerkte Christus bei dieser Gelegenheit? -- \textit{"`Wie schwerlich,"'}
sagte er,
\textit{"`werden die Reichen in das Reich Gottes kommen,"'} Und gleich darauf
fügte er
hinzu:\\textit{"`Es ist leichter, dass ein Kamel durch ein Nadelöhr geht, als
dass ein
Reicher ins Reich Gottes kommt,"'}\footnote{Markus 10,23-25.}
\index{Bibelstellen:!Markus 10)}
das heißt: Wenn ein
solcher geiziger Reicher, dem es schwer wird, mit dem, was er besitzt, Gutes zu
tun, ins Reich Gottes\index{Reich!Gottes} kommt, das ist mehr als ein
gewöhnliches Wunder. O! Wer
wollte denn reich und geizig sein! -- Über solche Reiche sprach Christus das
Wehe aus, indem er sagte: \textit{"`Wehe euch, ihr Reichen! Ihr habt eueren
Trost
dahin."'}\footnote{Lukas 6,24.}
\index{Bibelstellen:!Lukas 06@Lukas 6)}
Wie? Haben die Reichen denn keinen Trost im
Himmel\index{Himmel} zu erwarten? Nein! Keinen. Es sei denn, -- dass sie sich
entschließen,
aller Anhänglichkeit an ihren Reichtum zu entsagen, sich von der Welt
loszureißen und von ihrem eitlen Wesen sich befreien zu lassen, so dass sie das,
was sie besitzen, unter ihren Füßen haben und das Geld ihr Diener, aber nicht
ihr Herr sei.

\section{19. Abschnitt} \label{kap13_ab19}

\index{Gütergemeinschaft} Das zweite Beispiel, das ich anführen will, ist die
schauderhafte Geschichte von
Ananias\index{Personen:!Ananias} und Sapphira\index{Personen:!Sapphira}. Es war
nämlich im Anfange der apostolischen Zeit
üblich, dass diejenigen, welche das Wort des Lebens annahmen, alles, was
sie besaßen, darbrachten und zu den Füßen der Apostel niederlegten, wovon auch
Josef, mit dem Zunamen Barnabas, ein Beispiel gab. Unter anderen verkauften auch
Ananias und seine Frau Sapphira, als sie sich zu der Wahrheit bekannten, ihr
Eigentum, behielten aber, aus Geiz, einen Teil von dem dafür gelösten Geld,
den sie nicht in die gemeinschaftliche Kasse fließen lassen wollten, für sich
zurück, und brachten den anderen Teil, als das Ganze, den sie zu den Füßen der
Apostel hinlegten. Allein Petrus, ein gerader und kühner Mann, sagte in der
Kraft und Majestät des heiligen Geistes:
\textit{"`Anania! Warum hat der Satan dein Herz
erfüllt, dass du dem heiligen Geist belügst, und entwendest etwas vom Geld des
Ackers? Hättest du ihn doch wohl behalten mögen, da du ihn hattest, und da er
verkauft war, war es auch in deiner Gewalt. Warum hast du denn solches in deinem
Herzen vorgenommen? Du hast nicht Menschen, sondern Gott belogen."'}\footnote{Apostelgeschichte 5.}
\index{Bibelstellen:!Apostelgeschichte 05@Apostelgeschichte 5)}
Und was hatte dieser Geiz und diese Heuchelei\index{Heuchelei} des Ananias zur
Folge?
-- Als er Petrus so reden hörte,
\textit{"`fiel er nieder und gab seinen Geist auf."'}
Ein gleiches Los traf seine Frau, die um den Betrug wußte, zu welchem der Geiz
sie beide verleitet hatte. Auch lesen wir,
\textit{"`dass über die ganze Gemeinde und über
alle, die es hörten, eine große Furcht kam."'}
Dieses sollte aber auch der Fall
bei denen sein, die es jetzt lesen, denn da dieses Gericht Gottes
\index{Gericht!Gottes}\index{Gott!Gericht}erfolgte, und
für uns aufgezeichnet war, damit wir uns vor ähnlichem Übel hüten mögen, was
für ein Ende wird es denn mit denen nehmen, die, als Bekenner des
Chrisientums\index{Bekenner des Christentums}, -- einer Religion,
welche die Menschen lehrt, der Welt zu
entsagen\index{Welt!Entsagen} und alles dem Willen und Dienst Christi und
seines Reichs aufzuopfern,~-- nicht bloß einen Teil, sondern das Ganze ihres Vermögens zurückbehalten, und
sich nicht von der geringsten Sache um Christi Willen trennen können. Ich bitte
Gott, dass er die Herzen meiner Leser zu einer ernsten Erwägung dieser
Gegenstände geneigt machen wolle! Es würde dem Ananiaes und der Sapphira ein
solches Gericht nicht widerfahren sein, wenn sie in der Gegenwart Gottes und
nach der völligen Liebe, Wahrheit und Aufrichtigkeit gehandelt hätten, wie ihnen
gebührte. Möchten daher doch alle sich des Lichts bedienen, das Christus ihnen
verliehen hat! Möchten sie in diesem Licht untersuchen und sehen, inwiefern
sie sich noch unter der Gewalt der Ungerechtigkeit des Geizes befinden! Denn,
wollten die Menschen nur gegen die Liebe zur Welt auf ihrer Hut sein, und sich
weniger von sichtbaren und vergänglichen Dingen\index{Vergänglichkeit} der
Erde fesseln lassen, so
würden sie bald anfangen, ihre Herzen auf dasjenige zu richten, was oben im
Himmel\index{Himmel} und von ewig dauernder Natur und Beschaffenheit ist. Dann
würden sie auch
anfangen, \textit{"`ein mit Christo in Gott verborgenes Leben"'} zu führen, ein
Leben,
das über die Ungewissheit der Zeit und über alle Unruhen und Veränderungen der
Sterblichkeit erhaben ist. Ja, möchten die Menschen nur erwägen, wie schwer es
ist, Reichtum zu erwerben, und wie ungewiss, ihn zu behalten, wie viel
Neid\index{Neid} er
erweckt, und dass er weder Weisheit verleihen, noch Krankheiten heilen, noch das
Leben verlängern, und noch viel weniger Frieden im Tod gewähren kann. Ja, die
wirklichen Vorteile, die der Reichtum dem Menschen verschafft, erstrecken sich
in der Tat kaum weiter, als auf Nahrung und Kleider, die man doch auch, ohne
reich zu sein, erlangen kann. Und die gute Anwendung, die bessere Menschen davon
machen, besteht darin, dass sie dem Elend und den Bedürfnissen anderer damit
abhelfen, indem sie sich als Haushalter der reichen Gaben der göttlichen
Vorsehung betrachten, die von ihrem Haushalten Rechenschaft geben müssen. Ich
sage, wenn wir solchen Betrachtungen ihr gehörigers Gewicht in unseren Gemütern
erstatteten, so würden wir uns nicht so eifrig bemühen, vergängliche Schätze zu
sammeln, und noch viel weniger, sie ängstlich zu verbergen und aufzubewahren. Und o!
Möchte das Kreuz Christi, dieser Geist und diese Kraft Gottes im Menschen, doch
mehr Raum in unseren Seelen gewinnen, damit wir dadurch immer mehr und mehr der
Welt gekreuzigt\index{Kreuzigung} würden, so dass auch uns die Welt
gekreuzigt wäre, und, wie in
den paradiesischen Tagen, die Erde wieder der Fußschemel, und die Schätze der
Erde die Diener, nicht die Götter des Menschen wären! -- Es haben schon viele
gegen das Laster des Geizes geschrieben, von denen ich hier drei anführen will.

\section{20. Abschnitt} \label{kap13_ab20}

Wilhelm Tinbal\index{Personen:!Tinbal, Wilhelm}, jener würdige Apostel der
englischen Reformation\index{Reformation!englische}, hat eine
vollständige Abhandlung unter dem Titel: \textit{"`Das Gleichnis vom ungerechten Mammon"'}
herausgegeben, worauf ich den Leser verweise. Der zweite Verfasser ist:

\section{21. Abschnitt} \label{kap13_ab21}

Peter Charron\index{Personen:!Charron, Peter}, ein besonders wegen seines
Buches über die Weisheit berühmter
französischer Schriftsteller. Es hat ein Kapitel gegen die Gier, worin er sagt:
\textit{"`Reichtum lieben und daran hängen, ist Gier, doch macht nicht allein
diese
Liebe und Anhänglichkeit, sondern vielmehr das allzueifrige Streben und Trachten
nach Reichtum eigentlich der Gier auf das Verlangen nach Reichtum, und das
Vergnügen, welches wir in seinem Besitze finden, gründet sich bloß auf unsere
Meinung davon. Das unmäßige Verlangen, reich zu werden, ist wie ein Krebsschaden
in der Seele, der mit giftigen Brande unsere natürliche Neigungen verzehrt und
uns mit bösartigen Säften anfüllt. Sobald dieses Verlangen sich unserer Herzen
bemächtigt hat, erlöscht alle wahre und natürliche Zuneigung zu unseren Eltern
und Freunden, sogar zu uns selbst, indem alles, was nicht auf Gewinn Bezug hat,
uns als nichtig erscheint. Ja, wir vernachlässigen und vergessen endlich, um der
vergänglichen Dinge Willen, uns selbst, unseren Körper und Geist, und, wie das
Sprichwort sagt, ‚verkaufen das Pferd um Heu anzuschaffen."'}

\textit{Die Gier ist die schändliche und niedrige Leidenschaft gemeiner Toren,
welche
Reichtum für das edelste Gut des Menschen halten, und Armut als das größte
Übel scheuen? Diese, nicht zufrieden mit den notwendigen Dingen des Lebens,
wägen die Güter der Erde in einer Goldwage, da doch die Natur uns lehrt, sie
nach dem Maßstab des Bedürfnisses zu messen. Welche Torheit kann aber wohl
größer sein, als dasjenige zu verehren und anzubeten, was selbst die Natur, als
etwas unseren Anblicks unwertes, unter unsere Füße gelegt hat, um uns zu
zeigen, dass es eher verachtet und mit Füßen getreten als angebetet werden müsse.
Dieses hat die Sünde des Menschen den Eingeweiden der Erde entrissen und ans
Licht gebracht, um sich damit zu töten. \label{ref:13_21_Kapitalisten_kritik}
\textbf{Wir durchwühlen das Innere der
Erde, um
etwas hervorzubringen, wofür wir kämpfen und streiten können, indem wir uns
nicht schämen, den Dingen, die in den niedrigsten Teilen der Erde liegen, einen
so hohen Werth beizulegen.} Die Natur scheint schon durch die Beschaffenheit des
Bodens, in welchem das Gold seine Bindung erhält, gewissermaßen das Elend derer,
die es lieb gewinnen, vorher verkündigt zu haben, denn sie hat es so geordnet,
dass in den Gegenden, wo man es findet, weder Gras noch Pf1anze, noch irgendetwas Brauchbares
wächst oder gedeiht, als habe sie uns dadurch andeuten
wollen, dass in den Gemütern, in welchen die Liebe zu diesem Metall eine
Vorherrschaft gewinnt, kein Funken von wahrer Ehre und Tugend aufkommen könne.
Und was kann auch niedriger sein, als wenn der Mensch sich dem unterwirft, und
dem sklavisch gehorcht, was ihn unterworfen und dienstbar sein sollte? Der
Reichtum ist des Weisen Diener, aber des Toren Herr.
\label{ref:13_21_Kapitalisten_dienerschaft} \textbf{Denn dem Gierigen dient
sein Reichtum nicht, sondern er dient ihm.} Man kann von ihm sagen: Er hat
Güter, wie man von jemand sagt: Er hat das Fieber, denn so wie jemand nicht das
Fieber, sondern das Fieber ihn besitzt, so besitzen und beherrschen auch den
Gierigen seine Güter, und nicht er sie. Ist wohl etwas schändicher als wenn der
Mensch das liebt, was weder gut ist, noch jemanden gut machen kann? Das so
gemein ist, dass er sich auch in den Händen der gottlosesten Menschen befindet?
Das oft sehr gute Sitten verdirbt, aber nie zu ihrer Veredlung dient? Ohne
welches schon so viele Weise glücklich wären, und wodurch so viele Gottlose zu
einem elenden Ende gebracht werden? Kurz, was ist scheußlicher, als Lebendige an
Tote binden, wie Mezentius\index{Personen:!Mezentius} tat, damit ihr Tod desto
schmachtender und
grausamer sein möchte? Nämlich, den Geist an den Auswurf und Abschaum der Erde
binden, um seine Seele mit tausend Qualen zu ängstigen, welche die beständigen
Begleiter einer leidenschaftlichen Liebe zum Geld sind, oder sich in den Banden
und Schlingen so schädlicher Dinge verwickeln, welche die heilige Schrift
Dornen, Diebe, die das Herz des Menschen stehlen, Stricke des Teufels,
Abgötterei und die Wurzel alles Bösen nennt. Und wahrlich! Wenn alle üblen
Wirkungen des Neides und Verdrusses, die der Reichtum in den Herzen der
Menschen erzeugt, recht erkannt und erwogen werden, so muss man ihn eher hassen
und fliehen, als lieben und suchen. Die Armut muss freilich vieles entbehren,
die Gier entbehrt aber alles! Der Gierige ist keines Menschen Freund, aber sein
eigener größter Feind!"'} -- So sagt Charron\index{Personen:!Charron, Peter}, ein
weiser und großer Mann. Mein
drittes Zeugnis nehme ich aus einem Schriftsteller, der wahrscheinlich einigen
seines Wizes wegen sehr gefallen wird. Möge man aber auch seine moralischen
Bemerkungen und seine reife Beurteilung ebenso sehr schätzen.

\section{22. Abschnitt} \label{kap13_ab22}

Abraham Cowley\index{Personen:!Cowley, Abraham}, ein witziger und talentvoller
Mann, liefert uns folgende
Bemerkungen über die Gier:
\textit{"`Es gibt zwei Gattungen der Gier. Die eine ist
nur eine Bastardart und besteht in einem eifrigen Streben nach Gewinn, nicht
aus Liebe zum Geld selbst, sondern um das Vergnügen zu haben, dasselbe auf
allen Wegen des Stolzes und der Üppigkeit wieder zu verschwenden. Die andere
Gattung ist die wahre Art der Gier, und wird mit Recht so genannt, denn sie
besteht in einer rastlosen und unersättlichen Begierde nach Reichtum, und zwar
aus keiner anderen Ursache und zu keinem anderen Zweck oder Gebrauch, als um
ihn auzuhäufen, zu bewahren und beständig zu vermehren. Der Geizige der ersten
Gattung ist einem gierigen Strauße
%Vogel Strauß?? Verstehe ich nicht.
% Vielleicht ein Übersetzungsfehler von Seeborm und
% es sollte die Elster sein...?
gleich, der jedes Stück Metall verschlingt,
doch in der Absicht, sich davon zu nähren, daher er sich dann auch alle Mühe
gibt, es zu kauen und zu verdauen. Der andere gleicht der törichten Krähe, die
nur Geld stiehlt, um es zu verstecken. \index{Volkswirtschaft!Schaden} Der
Erstere tut der menschlichen
Gesellschaft viel Schaden, doch einigen Personen zuweilen auch Gutes. Der
Letztere hingegen nützt niemand, ja, auch sich selbst nicht. Der Erstere kann
seine Handlungen weder vor Gott und den Engeln\index{Engel}, noch vor
vernünftigen Menschen
entschuldigen, der andere kann für das, was er tat, auch keinen Schatten von
Entschuldigung vorbringen. Er dient dem Mammon als Sklave, ohne Lohn. Der
Erstere weiß sich noch bei einigen beliebt, ja, sogar beneidenswert zu machen,
der Letztere ist allgemeiner Gegenstand des Hasses und der Verachtung. Er gibt
kein Laster, auf welches man viele Sinngedichte gemacht hätte, und das
besonders von Dichtern durch Satiren, Fabeln, Allegorien und witzige
Anspielungen in ein so verhaßtes Licht gestellt und von allen Seiten so scharf
getadelt worden wäre, als der Geiz. Doch erinnere ich mich keiner feinern
Bestrafung\index{Bestrafung} desselben, als der des Ovids\index{Personen:!Ovid}
in folgender Zeile:}

\medskip

\texttt{............Multa} \\
\texttt{Luxuriae desunt, omnia avaritiae.}

\medskip

\textit{Dein Luxus mangelt vieles, dem Geiz aber alles.}

\medskip

Diesem Ausspruch möchte ich noch ein paar Worte beifügen, und ihn so geben:
\textit{Der Armut mangelt manches, der Üppigkeit sehr viel,
allein der Gier Mangel ist ohne Maß und Ziel.}
\textit{Auch sagt jemand von einem tugendhaften und weisen Mann: Indem er
nichts
besitzt, hat er alles. Dieser ist also der wahre Gegenfüßler des Gierigen, der,
indem er alles hat, doch nichts besitzt. Und o! Wer kann wohl elender sein, als
derjenige, der, in Überfluss schmachtend, des Himmels Segen sich selbst zum
Fluch macht? Fühlt nicht der arme genügsame\index{Genügsamkeit}
Bettler\index{Bettler} beim Genuss einer geringen
Gabe sich glücklicher als der unersättliche Reiche, der bei allen seinen
Schätzen doch unaussprechlich arm und elend ist?}

\medskip

\textit{Es wundert mich, dass man noch kein Gesetz gegen die
Gier\index{Gebote!Gesetz gegen Gier} gemacht hat. Doch was
sage ich? Gegen ihn? – Für ihn, zu seinen Gunsten, wollte ich sagen.
\label{ref:13_22_wahnsinnige}\textbf{Denn da man
für alle Arten Wahnsinnige öffentliche Vorsorge trägt, so würde es unstreitig
auch sehr angemessen sein, wenn der König einige Personen ernennen würde, die das
Vermögen der Gierigen während ihrer Lebenszeit verwalteten,} -- denn ihre Erben
bedürfen gewöhnlich einer solchen Vorkehrung nicht,~-- und deren Geschäft es
wäre, dahin zu sehen, dass es ihnen nicht an dem nötigsten Unterhalt mangelte,
den ihr Stand und ihre Lage erfordert, weil sie so grausam sind, sich diesen
selbst zu versagen. Wir helfen ja müßigen Tagedieben und verstellten
Bettlern\index{Bettler!verstellte},
warum wollten wir uns denn nicht um diese wirklich armen Leute bekümmern, die
man übrigens, denke ich, in Rücksicht auf ihren Stand, auch mit gehöriger
Achtung behandeln müsste. -- Ich möchte unaufhörlich über diese Menschen reden,
aber das viele, was sich über sie sagen ließe, erstickt mich, und es geht mir
fast ebenso, wie ihnen: der Überfluss macht mich arm!"'}

\medskip

Dies sei genug von der Gier, -- diesem an der Seele nagenden Wurm und
Krebsschaden des Geistes.





