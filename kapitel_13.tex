
\chapter{13. Kapitel} \label{kap13}

\section{Zusammenfassung des 13. Kapitel}
\small
\begin{description}
\item[1. Abschnitt]  Vom Geize, alz der zweiten Hauptleidenschaft des Menschen;
-- Bestimmung und Erklärung desselben.
\item[2. Abschnitt] Er Besteht erstlich in einem Verlangen nach unerlaubten
Dingen.
\item[3. Abschnitt] Wie in dem Beispiele mit David als er Urias Weib begehrte
\item[4. Abschnitt] So auch mit Ahab, der Noboth's Weinberg an sich zu bringen
suchte.
\item[5. Abschnitt] Zweitens bestehet der Geiz in einem unerlaubten Begehren
erlaubter Dinge.
\item[6. Abschnitt] Er ist ein kennzeichen der falschen Propheten.
\item[7. Abschnitt] Der Geiz ist entehrend für die Religion.
\item[8. Abschnitt] Ein Feind der Obrigkeit.
\item[9. Abschnitt] Er macht die Menschen zu Betrügern.
\item[10. Abschnitt] Er leitet zur Unterdrückung.
\item[11. Abschnitt] Beispiel des Judas.
\item[12. Abschnitt] Auch des Simon Megus
\item[13. Abschnitt] Endlich bestehet der Geiz noch in nutzloser Anhäufung des
Geldes.
\item[14. Abschnitt] Der geizige ist eine Volksplage.
\item[15. Abschnitt] Seine Heuchelei.
\item[16. Abschnitt] Das Gold ist sein Gott.
\item[17. Abschnitt] Er muss sparen, sollte es ihm auch des Leben kosten
\item[18. Abschnitt] Christus und seine Jünger verdammen den Geiz.
\item[19. Abschnitt] Sünde und Gericht des Ananias und der Sapphire.
\item[20. Abschnitt] Wilhelm Tindal's Abhandlung über diesen Gegenstand,
nachgewiesen.
\item[21. Abschnitt] Peter Cbarron's Zeugnis gegen den Geiz, angeführt.
\item[22. Abschnitt] Abraham Cowlev's witzige und scharfe Satyre über denselben.

\end{description}
\normalsize


\section{1. Abschnitt}

Ich komme nun zur Abhandlung der zweiten Hauptleidenschaft des Menschen, nämlich
des Geilzes oder der Habsucht; eines Uedele, das wie eine ansteckende Krankheit
in der Welt wüthet, und mit allen schädlichen Wirkungen begleitet ist, welche
die Menschen, sowohl einzeln als in gesellschaftlicher Verbindung, unglüglich
und elend machen. Dieses Laster ist mit dem, wovon ich eben geredet habe, mit
dem Stolze, so innig verbunden, daß man selten das eine ohne das andere
antrifft; indem Freigebigkeit dein Stolzen fast eben so gehässig als dem
Geitzigen ist. Ich bestimme den Geiz, den der Apostel "`die Wurzel alles
Uebels"' nennt, \footnote{1 Tim. 6,9. 10.} auf die Art, daß er in drei
Hauptzweigen erscheint; erstlich: in einem Begehren unerlaubter Dinge; zweitens:
in einen unerlaubten Begehren erlaubten Dinge, und drittens: in unnützer
Anhäufung des Geldes und anderer Dinge, wodurch der Gebrauch und die Wohlthat
des einen oder andern entweder einzelnen Personen oder der allgemeinen
Gesellschaft unnöthigerweise entzogen wird. Zuerst werde ich Dao erörtern, was
die heilige Schrift darüber sagt, und die Beispiele anführen, die sie uns zur
Warnung gegen dieser Laster aufstellt, und hernach meine eigenen nebst einigen
Zeugnissen achtbarer Schriftsteller beibringen. Daraus wird dann erhellen, daß
das- Kreuz Christi das Herz des Menschen edensowohl von der Liebe zum Reichthume
als von jeder andern Sünde, in welche er gefallen ist, befreien müsse.

\section{2. Abschnitt}

Was, ersilich, das Begehren unerlaubter Dinge betrifft, so ist dieses
ausdrücklich von Gott selbst in dem Gesetze verboten, das er Mosers auf dem
Berge Sinai als eine Richtschnur des Lebens, wonach sein Volk, die Juden,
wandeln sollten, in den Worten gab: «"`Laß dich nicht gelüsten deines Nächsten
Hauses. Laß dich nicht gelüsten deines Nächsten Weibers, noch seines Knechtes,
noch seiner Magd, noch seines Ochsen, noch seines Esels, noch alles dessen, was
dein Nächster hat."'\footnote{2 Mos. 20} Dieses bestätigte Gott durch Donner und
Blitz und andere schauerliche Zeichen, um das Volk mit desto größerer Ehrfurcht
zu erfüllen, damit es diese moralischen Vorschriften annehmen und halten, und
die Übertretung der selben desto schrecklicher finden möchte. -- Micha brach zu
seiner Zeit in die laute Klage aus: "`Sie reißen zu sich Aecker, und nehmen
Häuser, welche sie gelüstet;"'\footnote{Micha 2,2} sie nahmen aber auch ein
elendes Ende. Daher heißt es auch "`Wehe Denen, welchen mit Begierde gelüstet!"'
Dieses ist der Punkt, welchen wir zu betrachten haben, und wir finden viele
merkwürdige Beispiele davon in der Schrift, von denen ich nur zwei in der Kürze
berühren will.

\section{3. Abschnitt}

David, dieser sonst so gut Mann, fiel aus Mangel an Wachsamkeit. Die Schönheit
der Ehefrau des Urias war eine zu harte Versuchung für ihn, da er, von seiner
geistlichen Wache abgelockt, entwaffnet und wehrlos war. Nichtes konnte ihn nun
von seinem Vorhaben mehr zurückhalten. Urias mußte zu einer der zweifelten
Unternehmungen angestellt und einer Gefahr ausgesetzt werden, der er schwerlich
mit dem Leben entgehen konnte. Dieses geschah, um die Befriedigung der
unerlaubten Neigungen Davids auf eine Art zu beschleunigen, die nicht geradezu
daß Ansehen einer offenbaren Ermordung hatte. Der Kunstgriff gelang; Urias blieb
im Gefechte, und seine Wittwe ward unverzüglich Davids Weib. Hieraus ging nun
Davids Habsucht offenbar hervor; Aber kam er auch ungestraft damit durch? Nein,
sein Verbrechen zog ihm schwere Gerichte und scharfe Züchtigungen zu. «"`Seine
Freude ward bald in Angst und Bitterkeit des Geistes verkehret. Seine Seele
wollte sich nicht trösten lassen, und die Fluthen der Trübsal gingen über sein
Haupt. Seine Seele verschmachtete in ihm. Er war. wie in Koth versunken <!---
Heute würde man sagen: "Er steckte in der Scheiße" -->; er schrie; er weinte;
seine Augen glichen Thränenquellen. Seine Schuld lag schwer auf ihm; seine
Sünden, die roth, wie Scharlach, waren, mußten weiß, wie Schnee, gewaschen
werden, oder er war auf ewig verloren."'\footnote{Ps. 51.-77.-42,7.-69,2.
14.-6,6. 7.} Endlich trug jedoch seine Reue den Sieg davon. -- Siehe! war für
traurige Folgen diese Art der Lüsternheit und Habsucht nach sich ziehet; wes für
Elend und Kummer sie begleitet Und o! möchte doch Allen, die sich von einer
solchen Habsucht hinreisen lassen, ein Gefühl von Davids Reue und Kummer tief in
ihre Seelen dringen, damit sie auch Davids Heil und Erlösung erführen! "`Richte
Mich wieder auf!"' schrie dieser gute Mann, denn er war sich ohne Zweifel eines
vorigen bessern Zustandes bewußt. Ja, möge sein Fall Auch andern zur Warnung
dienen, und sie lehren, in heiliger Furcht und Wachsamkeit zu leben, damit sie
nicht auch sündigen und fallen. Denn David fiel durch Unbedachtsamkeit und
Unwachsamkeit; er stand nicht auf seiner Hut, und hatte sich dem Kreuze
entfernt. Das Gesetz in seinem Herzen war in dem unglücklichen Augenblicke nicht
seines Fußes Leuchte und ein Licht auf seinem Wege. Er hatte seinen sichern
Zufluchtsort, seine feste Burg verlassen; er sah sich überrascht; und gerade
diesen Zeitpunkt seiner Unwachsamkeit benutzte der Feind, und überwältigte ihn.

\section{4. Abschnitt}

Das zweite Beispiel Liefert uns die Geschichte von Naboth’s Weinberge. 1)
%<!-- weiß der Deibel, watt die 1 zu bedeuten hat... -->
Ahab und Isebel begehrten ihn; und derselbe Geist, der ihnen diesen unerlaubten
Gedanken einflößte, gab ihnen auch Mittel und Wege an die Hand, ihn auszuführen.
Raboth mußte sterben, weil er ihnen seinen Weinberg nicht verkaufen wollte. Um
ihren Zweck zu erreichen, beschuldigten sie den schuldlosen Mann der
Gotteslästerung, und fanden auch zwei falsche Zeugen, zwei Kinder Belials, die
gegen ihn zeugeten. So mußte Naboth unter dem Vorwande des Namens Gottes, und
unter dem Scheine des reinsten Eifere für seine Ehre, das Leben verlieren. Er
ward zu Tode gesteinigt; und sobald Isebel die Nachricht don seinem Tode
erhielt, sagte sie zu Ahab: "`Stehe auf, und nimm den Weinberg ein; denn Naboth
lebt nicht mehr, sondern ist todt."' Gott aber verfolgte Beide mit strengen:
Gerichte. "`Auf der Stelle, da die Hunde das Blut Naboth’s gelecket haben,"'
sagte Elia im Namen des Herrn, "`sollen auch Hunde dein Blut lecken. Ich will
Unglück über dich bringen, und deine Nachkommen wegnehmen."' Und von seinem
Weibe, Isebel, der Mitschuldigen an dieser Habsucht und Mordthat, fügte Elia
hinzu: "`Die Hunde sollen ihr Fleisch an den Mauern von Jesreel fressen."' Hier
sehen wir die Schändlichkeit dieser Art Habsucht, und auch die gerechte
Bestrafung derselben. Möge es Jeden abschrecken, der ein Verlangen nähret,
unerlaubte Dinge, die Rechte und Besitzungen Anderer, an sich zu bringen! Denn
Gott ist gerecht, der ein solche Begehren unfehlbar endlich bestrafen wird.
Jedoch ist vielleicht die Anzahl solcher Menschen, die nach dem unerlaubten
Besitze des Eigenthumes eines Andern trachten, nicht groß, denn Viele thun es
darum nicht, oder scheuen sich, ihr Begehren laut werden zu lassen, weil sie die
Gesetze fürchten. Hingegen giebt es nur zu Viele von einer andern Gattung,
welche gegen die eben erwähnte Art der Habsucht zu eifern scheinen, um durch
ihren vorgeblichen Abscheu vor derselben die Beschuldigung eines andern Geitzes
von sich abzulehnen, den wir jetzt betrachten wollen.

\section{5. Abschnitt}

Diese zweite Art der Habsucht oder des Geitzes, welche die allgemeinste ist,
bestehet in einem unerlaubten Begehren erlaubter Dinge, vornehmlich des

 Reichthums. Geld zu besitzen, ist an sich eine erlaubte Sache; aber die
"`Geldliebe oder der Geiz; ist die Wurzel alles Übels,"' wenn anders der Mann
Gottes wahr redet. Auch ist Reichthum nicht unerlaubt; allein Diejenigen, "`die
da reich werden wollen, fallen in mancherlei Versuchungen und Stricke, und in
viele thörichte und schädliche Lüste, welche sie ins Verderben
dringen,"'\footnote{1 Tim 6,9. 10.} wenn derselbe Apostel auch hierin die
Wahrheit sagt. Er nennt den Reichthum ungewisi, um die Thorheit und die Gefahr
Derer zu zeigen, die mit ihren Herzen danach trachten. In Gottes Augen ist der
Geiz ein Greuel, und er hat Allen, die sich desselben schuldig machen, große
Gerichte angekündigt. Dein Volke Israel warf Gott ehemals den Geitz als eine
Ursache seiner Gerichte vor. "`Ich war zornig über die unrugend ihres Geitzes,
und schlug sie"' sagt der Herr;\footnote{Jes. 57,17} und an einem andern Orte:
"`Denn sie geitzten alle- sarmut, Kleine und Große,. und beide,- Propheten und
Priester, lehren allesammt falschen Gottesdienst. Darum will ich ihre Weiber den
Fremden geben, und ihre Aecker Denen, die sie verjagen werden."'\footnote{Jes.
6,13 8,10.} Auch klagte Gott, "`daß ihre Augen und ihr Herz nur auf ihren Geitz
gerichtet waren."'\footnote{Jes. 22,17} Diese Klagen über ihren Geitz erneuerte
und wiederholte Gott durch Ezechiel in den Worten: "`Und sie werden zu dir
kommen in die Versammlung, und vor dir sitzen, als mein Volk, und werden deine
Worte hören, aber nicht darnach thun; sondern sie werden dich anpfeifen und
gleichwohl fortleben nach ihrem Geitze."'\footnote{Ezech. 33,31} Darum forderte
Gott es auch bei der Wahl der Obern als eine Haupteigenschaft ihres Charaktere,
"`daß sie den Geiz; hassen sollten;"' weil er das Elend voraussah, das in einer
Gesellschaft oder in einem Staate daraus erfolgen würde, wenn geizige Menschen
die oberste Gewalt in Händen hätten; indem diese sich nur zu leicht durch
Eigennutz verleiten lassen, ihre Privatabsichten auf Kosten der öffentlichen
Wohlfahrt zu befördern. David flehete, "`daß sein Herz zu den Zeugnissen des
Herrn und nicht zum Geize geneigt werden möchte."'\footnote{Ps. 119, 36} Und
Salome sagt: "`Wer den Geiez hasset, der wird lange leben,"'\footnote{Sprichw.
28,16.} und er rechnet es für einen Fluch, wenn Jemand ihm ergeben ist. Lukas
warf den Pharisäern das Laster des Geizes als ein Zeichen ihrer Gottlosigkeit
vor, und Christus sagt in diesem Evangelisten zu seinen Jüngern: "`Gehet zu und
hütet euch vor dem Geize;"' dabei führt er einen Grund an, der trefflichen
Unterricht enthält: "`Denn Niemand,"' sagt er, "`lebt davon, daß er viele Güter
hat"'.\footnote{Luc. 12,15} Bei einer andern Gelegenheit stellt Christus den
Geilz mit Ehebruch, Mord und Gotteslästerung in eine Klasse.\footnote{Matth.
7,21, 22.}- Es ist daher nicht zu bewundern, wenn der Apostel Paulus dieses
Laster so häufig rügt. In seiner Epistel an die Römer stellt er den Geiz mit
aller Ungerechtigkeit zusammen.\footnote{Röm. 1,29} An die Epheser schreibt: er,
"`das Geiz unter ihnen auch nicht einmal genannt werden müsse."'\footnote{Ephes.
5,3.} Den Kolossern empfiehlt er, ihre Glieder zu tödten, und indem er
verschiedene Sünden, als: Hurerei, Unreinigkeit, etc.. benennt, beschließt er
die Aufzählung derselben mit dem Geize, den er Abgöteterei nennt; eine Sünde,
die, wie uns bekannt ist, Gott aufs höchste beleidigt. Ja, dieser Apostel nennt
den Geiz oder die Liebe zum Gelde "`eine Wurzel alles Bösen,"' und sagt uns, daß
Einige, die darnach trachteten, "`vom Glauben abgeirret sind, und sich selbst
mit vielen Schmerzen durchstochen haden."' "`Denn Diejenigen, welche reich
werden wollen,"' sagt er, "`fallen in Versuchung und Stricke, und in viele
thörichte und schädliche Lüste."' Darum rief er seinem geliebten Freunde
Timotheus zu: "`Aber du Gottesmensch! Fliehe Solches! Jage der Gerechtigkeit
nach; der Gottsseligkeit, dem Glauben, der Liebe, der Geduld und der
Sanflmuth."'\footnote{1 Tim. 6,9. 10. 11.}

\section{6. Abschnitt}

Petrus hatte dieselbe Gesinnung; denn er gab den Geiz als ein Hauptkennzeichen
au, woran die falschen Propheten und Lehrer, die unter den Christen entstehen
würden, zu erkennen wären,  "`welche,"' sagt er, "`aus Geiz mit erdichteten
Worten an euch handthieren (oder Handel mit euch treiben) werden."'\footnote{2
Petr. 2,3.} Endlich hintersläst uns der Verfasser der Epistel an die Ebräer
unter andern wichtigen Gegenstände, die er mit warmen Eifer einschärft, noch die
Errnahmmg: "`Der Wandel sei ohne Geiz."' Doch läßt er es bei dieser allgemeinen
Vorschrift nicht bewenden, sondern fügt noch hinzu: "`Und laßt euch begnügen an
dem, daß da ist; denn Gott hat gesagt: Ich will dich nicht verlassen, noch
versäumen."'\footnote{Ebr. 13,5} Können wir nun aber hieraus nicht auch
schließen, daß die Ungenügsamen, die nach Reichthum trachten, Gott verlassen?
Dieser Schluß scheint vielleicht hart, allein er folgt ganz natürlich daraus da
solche Menschen, die mit dem, was sie haben, nicht zufrieden sind, sondern immer
mehr verlangen, und daher, wenn es nur möglich ist, auch reich zu werden suchen,
offenbar nicht in dem Vertrauen und Aufsehen auf die göttliche Vorsehung leben,
zu welcher sie ermahnet sind, und ihnen folglich Gottseligkeit mit Genügsamkeit
verbunden, kein großer Gewinn zu sein scheint.

\section{7. Abschnitt}

Und wahrlich! es ist eine Schande für den Menschen, besonders für den
Religösgesinnten, daß er oft nicht weiß, wann er genug hat; wann er mit Gewinnen
aufhören und sich begnügen soll und daß er, obgleich ihm Gott eine reiche Ernte
oder Einnahme nach der andern zufließen läßt, doch diese Wohlthaten nicht als
Beweggründe ansiehet, sich endlich aus dem Gewühle der Welt zurückzuziehen,
sondern dieselben vielmehr als Einladungen betrachtet, sich noch tiefer in die
Geschäfte des Lebens einzulassen; als wenn er, jemehr er habe, auch desto mehr
erwariten könne. Er erneuert daher fseine Begierde und strengt sich mehr als
jemals an, um seinen Theil davon zu tragen, so lange es neoch etwas zu erjagen
giebt. Gerade als ab Unrnhe und nicht Abgeschiedenheit, Gewinnsucht und nicht
Genügsamkeit, die Pflichten und die Quellen des Trostes eines Christen wären.
Möchte doch dieses besser erwogen werden! Denn da der Geiz den Augen und der
Wachsamkeit der Gesetze mehr als andere Laster entzogen ist, so ist er aus
Mangel an gesetzlicher Aufsicht und Einschränkung desto gefährlicher.

Es ist nicht zu leugnen, daß sehr viele Menschen nicht des Unterhaltes, sondern
des Reichthums wegen, Geld zu gewinnen suchen. Es giebt Einige, die eifrig
darnach trachten, es aber auch bald wieder verschwenden, wenn sie es erlangt
haben. Dieses ist allerdings auch strafbar; allein bei weitem nicht so
verwerflich, als wenn Jemand das Geld bloß um des Gelder willen liebt, welches
in der That eine der niedrigsten Leidenschaften ist, wovon das Herz eines
Menschen gefesselt werden kann. Dieß ist ein größeres, und für die Seele
verderblicheres Laster, als in dem ganzen Verzeichnissen der sündlichen
Begierden des Menschen zu finden ist. Würde dieses recht erwogen, so müßte es
Jeden zu der ernsten Untersuchung anleiten, in wiefern die Versuchung zur
Geldliebe einen Eingang in sein Herz gewonnen habe; und zwar um so mehr, da die
Fortschritte, die diesrs Laster in dem menschlichen Gemüthe macht, fast
unmerklich und seine Wirkungen daher desto gefährlicher sind. Tausende halten
vielleicht eine solche Warnung gegen den Geiz bei ihnen am unrechten Orte, und
sind doch nichts destoweniger dieses Uebels schuldig. Wie kann es auch andere
seyn, wenn man siehet, wie oft Diejenigen, die anfänglich wenig oder nichts
besaßen, nachdem sie sich Summen erworben haben, noch mit derselben Sorge und
ängstlichen Scharfsichtigkeit, mit welcher sie dieselben zusammenscharreten,
beständig fortwirken, um diese Summen zu vermehren, zu verdoppeln oder zu
verdreifachen. Heißt dieses ein ruhiger und angenehmes Leben führen? Oder heißt
dies auch wohl reich seyn? Sehen wir nicht, wie diese Leute so früh auf sind,
und erst so spät zur Ruhe kommen? Wie ihre Gedanken beständig auf ihre Geschäfte
gerichtet und mit einer Menge von Gegenständen angefüllt sind? Wie sie dabei hin
und her eilen als od sie beschäftigt wären, einem unschuldig Verurtheilten das
Leben zu retten? Und dieses Allee um eine unersätiliche Leidenschaft zu
befriedigen, die eben so verderblich für die Menschen als Gott zuwider ist, der
den Reichthum dazu giebt, daß man ihn wohl anwenden, nicht aber, damit man mit
seinem Herzen daran hängen solle; denn darin bestehet der Mißbrauch, den die
Menschen davon machen. Soll endlich aber ein so beständiges Sorgen, scharfes
Nachdenken und thätiged Streben, um Geld zu gewinnen, bei Denen, die Zehnmal
mehr besitzen, als sie anfänglich hatten, und viel mehr, als sie verzehren
können oder nöthig haben, nicht offenbar Geldliebe verrathen, so weiß ich doch
nicht, wie Jemand seine Liebe zu irgend einer Sache auf eine andere Art
deutlicher an den Tag legen könnte.

\section{8. Abschnitt}

Bei obrigkeitlichen Personen ist der Geiz immer der Regierung gefährlich; da er
zu Bestechungen verleitet. Darum mußten Diejenigen, die Gott verordnete, Solche
seyn, "`die ihn fürchteten, und den Geiz; haßten."' Dann schadet er der
menschlichen Gesellschaft vornehmlich dadurch, daß er den alten Geschäftsleuten
eingiebt, die jungen Anfänger nicht aufkommen zu lassen. Und eine Hauptursache,
warum viele Menschen zu wenig erwerben, und genöthigt sind, wie Sklaven zu
arbeiten, um ihre Familien zu ernähren und ihr Auskommen zu finden, liegt
unstreitig darin, daß die Reichen nicht nachlassen, zu scharren, sondern immer
noch reicher werden wollen, und deswegen auch die kleinern Erwerbsquellen der
geringem Klasse an sich ziehen oder verstopfen. Es wäre daher zu wünschen, daß
man einen Maßstab festsetzen möchte, nach welchem Jeder die Dauer und Ausdehnung
der Grenzen seiner Geschäfte beistimmen, und nach deren Erreichung er sie dann
unter Diejenigen seiner Untergebenen, die es verdienten, vertheilte. Dieses wüde
den, Jüngern die Mittel gewähren, sich ihren Lebensunterhalt zu erwerben, und
den Alten Zeit verschaffen, darauf zu denken, wie sie eine Welt verlassen
wollen, in der sie so geschäftig gewesen sind, und sich um ihr Loos in jene Welt
zu bekümmern, um weiches sie bisher so unbekümmert waren.

\section{9. Abschnitt}

Für die Regierungen entspringt aus dem Geize der Unterthanen noch der Nachteil,
daß er diese verleitet, ihre Obern durch Einführung oder Verheimlichung
schlechter oder verbotener Wahren, durch Bewirkung oder Beförderung des Umlaufst
falscher Münzen, durch Schleichhandel, unrichtiges Maß oder Gewicht, und auf
viele andere Weise zu beeinträchtigen und zu betrügen.


\section{10. Abschnitt}

Nicht selten hat auch der geiz schon die zerstörendsten Ueber in Familien
angerichtet. So sind, z. B. dadurch, daß Güter oder Gelder den Händen solcher
Menschen anvertrauet wurden, deren Geiz sie vermochte, einen eigennützige und
ungerechten Gebrauch davon zu machen, schon oft große Verluste und
Unterdrückungen entstanden. Ja, es war nur zu oft schon der Fall, daß solche
geizige Verwalter des Vermögen Anderer, den rechtmäßigen Eigenthümern, selbst
mittels des Geldes, das sie ihnen hätten auszahlen sollen, den Besitz ihres
Vermögens streitig machten, oder dasselbe an sich zogen.

\section{11. Abschnitt}

Doch dies ist noch nicht Alles. Der Geiz macht die Menschen sogar zu Verräthern
an der Freundschaft. Denn, wo Bestechung angewendet werden muß, um eine
Schandthat zu verüben, oder Jemand ins Elend zu stürzen, da laßt sich der
Geizige unfehlbar dadurch gewinnen, auch an seinem Freunde verrätherisch zu
handeln. Ja der Geiz mordet oft Leib und Seele. Diese, indem er das Leben aus
Gott in ihr zerstöret; denn, wo die Geldliebe das Gemuth des Menschen
beherrscht, da löscht sie alle Liebe zu etwas Bessern aus. Und Mörder des Leibes
wird der Geizige ebenfalls aus Liebe zum Gelde, die ihn zu Meuchelmorden,
Vergiftungen und falschen Zeugnissen , u.s.w. verleitet. -- Ich will zum
Beschlusse noch die Sünde und das Schicksal zweier Geizigen, des Judas und des
Simon Magus,  berühren.

In des Judas Herzen war der Samen der Religion unter die Dornen gefallen; seine
Geldliebe hatte ihn erstickt; Aus Stolz und Aerger suchten die Juden Jesum zu
töten; allein es gelang ihnen nicht, bis der Geiz; ihnen die Hände bot, ihr
Vorhaben auszuführen. Sie sahen, daß Judas den Beutel trug, und vermutheten, daß
er auch wohl daß Geld liebte. Sie beschlossen daher, ihn auf die Probe zu
setzen, und thaten es auch. Man ward über den Preis einig; und Judas?
überlieferte seinen Herrn und Meister, der ihm nie etwaz zu Leide gethan hatte,
in die Hände seiner grausamsten Feinde. Um ihm jedoch Gerechtigkeit wiederfahren
zu lassen, müssen wir auch bemerken, dass er das Geld, welches er für seine
Verrätherei bekommen hatte, wieder zurückgeb, und, um Rache an sich selbst zu
üben, hinging und sich erhenkte. Eine gottlose That! Ein scheußliches Ende! Aber
was sagt ihr nun? ihr Geizigen? Was denkt ihr von euerem Bruder Judas? -- War er
nicht ein böser Mensch? -- Handelte er nicht entsetzlich gottlos? -- O! gewiß!
werdet ihr sagen. Würdet ihr so gehandelt haben? -- »Nein! auf keinen Fall! Aber
die Juden sagten dasselbe von Denen, welche die Propheten getödtet hatten; und
dennoch kreuzigten sie Jesum, den Sohn Gottes, der gekmmen war, sie zu erretten
und selig zu machen, und es auch würde gethan haben, wenn sie ihn angenommen und
den Tag ihrer Heinrsuchung nicht verworfen hatten. Laßt euch die Augen salben,
und vom Staube der Erde reinigen; damit ihr desto besser in euerem Gewissen
lesen und desto deutlicher erkennen könnet, ob nicht auch ihr, aus Geldliebe,
den Gerechten in euerein Innern verrathen habt, und dadurch Bruder des Judas in
seiner Ungerechtigkeit geworden seid? -- Ich rede für die Sache Gottes und gegen
einen Götzen; darum ertraget mich mit Geduld. -- Habt ihr nicht mit euerem
Trachten nach eueren geliebten Reichthümern dem Geiste Christi widerstrebt? ja,
habt ihr nicht diesen guten Geist in euch selbst unterdrückt? O! "`Untersuchet
euch selbst; prüfet euch selbst; oder erkennet ihr euch selbst nicht? Denn wenn
Jesus Christus nicht in euch ist,"' wenn Er nicht in euch die Herrschaft hat,
und über Alles in der Welt von euch geliebt. wird, "`so seid ihr
Untüchtige,"'<!-- Etwas schwer zu lesen. In meiner Kopie ist es besser zu
erkennen --> \footnote{2 Kor. 13,5} oder Verworfene, und befindet euch in einenm
verlorenen Zustande.

\section{12. Abschnitt}

Der andere Geileige ist Simon, der Zauberer, der auch ein Gläubiger war; allein
sein Glaube hatte, seines Geizes wegen, nicht tiefe Wurzel schlagen können. Er
hätte gern mit Petrus einen Handel abgeschlossen, damit er die göttliche Gabe
hätte wieder derkaufen und auf diese Art einen guten Handelezzweig daraus machen
können. Er schloß von sich auf Petrus, den er ohne Zweifel für einen Mann hielt,
der den Kniff, die Menschen zu berücken, noch besser als er selbst verstände,
der doch in Samaria für die große Kraft Gottes galt, bis die wahre Kraft Gottes;
durch Philippus und Petrus offenbar ward, und das Volk ans dem Irrthume riß.
Aber was antwortete Petrus dem Simon, und was für ein Uriheil sprach er über ihn
aus? «Daß du verdammet werdest mit deinem Gelde!" sagte der Apostel, "`du wirst
weder Theil noch Anfall (Loos) an diesem Worte haben; denn ich sehe, du bist
voll bitterer Galle und verknüpft mit Ungerechtigkeit."'\footnote{Apost. Gesch.
8,9-24.} In der That ein schauderhafter Ausspruch!

Außerdem dient der Geiz auch noch der Verschwendung und nimmt oft seinen
Ursprung aus derselben. Denn wenn einige Menschen viel haben, so lassen sie viel
ausgehen, und machen sich durch üppige Verschwendung wieder arm. Solche sind
dann immer geizig nach Gewinn, um desto mehr ausgeben zu können, welches jedoch
Mäßigkeit verhüten würde. Und wollten nur die Menschen in den Ausgaben für ihren
Tisch, für ihre Häuser, Möbeln, Kleider, für ihre Spiele und andere Vergnügungen
sich einschränken, oder könnten sie durch genaue Anwendung weiser Gesetze und
durch eine bessere Erziehung davon abgehalten werden, so würden sie nicht so
sehr der Versuchung ausgesetzt seyn, mit solchen Eifer nach Geld zu trachten,
als sie gewöhnlich thun, weil es ihnen dann an den Gelegenheiten, es zu
verschwenden, mangeln würde; denn der Geizhälse, die das Geld nur um des Geldes
willen lieben, giebt es doch wohl eigentlich nur wenige.

\section{13. Abschnitt}

Dieses leitet mich zur Betrachtung der nidrigsten Gattung des Geizes, welche die
abscheulichste von Allen ist; nämlich der Anhäiufung des Geldes, oder der
Aufbewahrung desselben, um weder sich noch Andere damit zu nützen. Die Menschen,
welche von dieser Art des Geikzes gefesselt sind, gehören zu Denen, von welchen
Salemo sagt, "`daß sie dei großem Gute arm sind,"'\footnote{Spr. Sal. 13,7.}
welches in den Augen Gottes eine große Sünde ist. Gott klagt über Diejenigen,
"`die den Raub der Armen in ihren Häusern haben."' Er sagt, "`daß sie sein Volk
zertreten und die Person der Elenden zerschlagen."'\footnote{Jes. 3,13. 14.}
Diese grämen sich, weil sie des ihnen Geraubte nicht wieder sehen. Aber der Herr
segnet Die, "`welche sich der Durftigen annehmen;"'\footnote{Ps. 41,1} und er
befiehlt Allen, "`ihr Herz nicht gegen ihren armen Bruder zu verhärten, und ihre
Hand nicht vor ihm zu verschließen, sondern ihn gern zu geben und zu
leihen;"'\footnote{5 Mos. 15, 7. 8. 9.} und zwar nicht allein dem geistlichen,
sondern auch dem natürlichem Bruder.

Der Apostel Paulus gab dem Timotheus vor dem Angesicht Gottes und Jesu Christi
diesen Auftrag: "`Den Reichen dieser We1t gebiete, das sie nicht stolz sind,
auch nicht auf den ungewissen Reichtum hoffen sondern auf den lebendigen Gott,
der uns Alles reichlich zu genießen giebt; daß sie Gutes thun und an guten
Werken reich werden."'\footnote{1 Tim. 6,17. 18} Reichtümer sind nur zu leicht
im Stande, daß Herz zu verderben; was aber dagegen schützt, ist die Ausübung der
Mildthätigkeit. Wer sie nicht dazu anwendet, der gebraucht sie nicht zu dem
Zwecke, wozu sie ihm gegeben wurden, und liebt sie nur um ihrer selbst willen,
nicht des Nutzens wegen, den er damit stiften kann. So ist der Geizige mitten
unter seinen Schätzen arm. Er leidet Mangel, weil er sich vor Ausgaben fürchtet,
und seine Furcht nimmt nach dem Verhältnisse zu, wie seine Hoffnun, zu gewinnen,
sich vermehrt. Auf diese Weise dient die Freude seines Herzens ihm zur Qual. Er
gleicht unter Allen am meisten dem Knechte, der sein Pfund im Schweistuche
verbag; denn er hat seine Pfunde, seine Schätze, in Säcken verwahrt, im Gewölbe,
unter dem Fußboden, hinter den Wänden, oder auch in Schulverschreibungen und
Pfandbriefen versteckt, wo sie verborgen und wie vergraben liegen, ohne irgend
Jemand nützlich zu seyn.

\section{14. Abschnitt}

Ein solcher Geiziger ist ein wahres Ungeheuer; Ohne Menschengefühl, und, gleich
den Polen, beständig kalt. Er ist ein Feind des Staatetes; denn er saugt das
Geld aus. Ein Krankheitsstoff im politischen Körper, der den Umlauf des Blutes
hemmt, und daher durch irgend ein Reinigungsmittel des Gesetze fortgeschafft
werden sollte; denn dieses Laster greift das Herz an und zerstört den
Zusammenhang aller Glieder. -- Der Geitzige haßt alle nützliche Künste und
Wissenschaften als überflüssige Dinge. aus Furcht, daß die Erlernung derselben
ihm etwas kosten Möchte. Daher ist sein Herz eben so sehr als sein Geldbeutel
gegen Erfindsamkeit und Kunstfleiß verschlossen. Er laßt Häuser einstürzen, um
nur nicht die Ausgaben für Reparaturen zu haben. Und was seine schmale Kost,
seine abgetragenen Kleider und seine schlechten Möbeln betrifft, so rechnet er
sich diese Stücke zur Mäßigkeit an. O! was für ein Ungeheuer ist ein solcher
Mensch, der aus Liebe zum Gelde, aber nicht aus Liebe zu Christo, das Kreuz
gegen sich selbst aufnehmen kann.

\section{15. Abschnitt}

Doch macht er auf seine Weise auch Anspruch auf Religion; denn er klagt
unaufhörlich über die Verschwendung der Menschen, um seinen Geiz dadurch zu
bemänteln. Siehet er Jemand eine kostbare Salbe auf das Haupt einen guten
Menschen ausgießen, so ist er gleich bereit, an die Dürftigkeit der Armen zu
erinnern, um seine Sparsamkeit zu zeigen und gerecht zu scheinen. Kommen die
Armen aber zu ihm, so weiß er seinen Mangel an Mildthätigkeit damit zu bedecken,
daß er entweder die Gegenstände des Mitleides füe unwürdig erklärt, oder auf die
Ursachen ihrer Armut anspielt, oder vorschüzt, er könne sein Geld auf Solche,
die es besser verdienten, und zu weit edlern Zwecken verwenden. -- Er, der nur
äußerst selten seine Börse öffnet, damit er nichts daraus verliere.

\section{16. Abschnitt}

Ein solcher Mensch ist elender als der Allerärmste; denn er lebt in beständiger
Furcht, das zu verlieren, was ihm doch keinen Genuß gewahrt, wohingegen die
Armen sich nicht fürchten dürfen, Etwas zu verlieren, dessen sie sich nicht
erfreuen. So ist er arm durch Uederschützung seines Reichthumes; und gewiß ist
er elend, da er mit vollem Geldbeutel in einem Speisehause Hunger leidet. Doch
wer weiß, ob er nicht, da das Geld sein Gott ist, vielleicht dafür hält, es sei
unnatürlich, den Gegenstand seiner Anbetung zu verzehren!

\section{17. Abschnitt}

Was endlich dieses Laster noch verabscheuungewürdiger macht, ist, daß es dem
Menschen das Leben verkürzt; und ich habe selbst Einige gekannt, die durch ihre
Geldbegierde sich vor der Zeit ins Grad brachten. Diese konnten, ihrem
Grundsaize getreu, sich nicht entschließen, die Ausgabe für einen Arzt zu
bewilligen, der vielleicht den armen Sklaven das Leben gerettet hätte, und
starben also, bloß um Kosten zu ersparen; -- eine Standhaftigkeit, die sie zu
Märtyrern der Geidliede erhebt!

\section{18. Abschnitt}

Laßt uns nun aber auch sehen, was für Beispiele wir in der heiligen Schrift
antreffen, die solchen niederträchtigen Schatzsammlern und Geldverbergern zur
Bestrafung dienen. Wir finden, daß einst ein gutscheinender Jüngling zu Christo
kam, und nach dem Wege zum ewigen Leben fragte. Christus; sagte ihm, "`Er kenne
ja die Gebote."' Seine Antwort war: "`er habe sie von Jugend auf gehalten."' Es
scheint, er war kein ausschweifender Mensch. -- Dies sind auch die Geizigen
gewöhnlich nicht, weil Ausschweifungen mit Kosten Vverbunden sind. -- Doch
Christus erwiederte: "`Eins fehlt dir noch. Geh! verkaufe Alles, was du hast,
und "`gieb es den Armen, so wirst du einen Schatz im Himmel haben; und dann komm
und folge mir nach, und nimm das Kreuz auf."' Christus griff ihn auf der
empfindlichsten Stelle an; er traf sein Herz, das er durchschauete, Und nun
zeigte es sich, in wie fern er das erste Gebot: "`Gott über Alles zu lieben,"'
gehalten hatte; denn wir finden, "`der Jüngling ward traurig und ging betrübt
hinweg!"' und die Ursache seiner Traurigkeit, die uns dabei angegeben wird, war
die, "`daß er viele Gütter hatte."'\footnote{Mark. 10,19-23} Bei ihrer trafen
zwei einander enrgegenstehende Wünsche zusammen; der eine war auf den Besitz des
Reichthumes, der andere auf die Erlangung des ewigen Lebens gerichtet; und
welcher von beiden behielt die Oberhand? -- Ach! leider war seine Anhänglichkeit
an die irdischen Güter starker als sein Verlangen nach dem ewigen. Leben. Aber
was bemerkte Christus bei dieser Gelegenheit? -- "`Wie schwerlich,"' sagte er,
"`werden die Reichen in das Reich Gottes kommen;"' Und gleich darauf fügte er
hinzu: "`Es ist leichter, daß ein Kameel durch ein Nadelohr gehe, als daß ein
Reicher ins Reich Gottes komme;"'\footnote{Mark. 10,23-25} das heißt: wenn ein
solcher geitziger Reicher, dem es schwer wird, mit dem, was er besitzt, Gutes zu
thun, ins Reich Gottes kommt, das ist mehr als ein gewöhnliches Wunder. O! wer
wollte benn reich und geizig feyn! -- Ueber solche Reiche sprach Christus das
Wehe aus, indem er sagte: "`Wehe euch, ihr Reichen! Ihr habt eueren Trost
dahin."'\footnote{Luk. 6,24} Wie? haben die Reichen denn keinen Trost im
Hinsnnel zu erwarten? Nein! Keinen. Es sei denn, -- daß sie sich entschließen,
aller Anhänglichkeit an ihren Reichthum zu entsagen, sich von der Welt
loszureißen und von ihrem eitlen Wesen sich befreien zu lassen; so daß sie daß,
was sie besitzen, unter ihren Füßen haben und daß Geld ihr Diener, aber nicht
ihr Herr sei.

\section{19. Abschnitt}

Das zweite Beispiel, daß ich anführen will, ist die schauderhafte Geschichte von
Ananias und Sapphira. Es war nämlich im Anfange der apostolischen Zeit
gebräuchlich, daß Diejenigen, welche das Wort des Lebens annahmen, Alles, was
sie besaßen, darbrachtens und zu den Füßen der Apostel niederlegten, wovon auch
Joses, mit dem Zunamen Barnabas, ein Beispiel gab. Unter Andern verkauften auch
Ananias und sein Weib Sapphira, als sie sich zu der Wahrheit bekannten, ihr
Eigenthum; behielten aber, aus Geiz, einen Theil von dem dafür gelöseten Gelde,
den sie nicht in die gemeinschaftliche Kasse wollten fließen lassen, für sich
zurück, Und brachten den andern Theil, als daß Ganze, den sie zu den Füßen der
Apostel hinlegten. Allein Petrus, ein gerader und kühner Mann, sagte in der
Kraft und Majestät des heiligen Geiste: "`Anania! warum hat der Satan dein Herz
erfüllet, daß du dem heiligen Geiste lügest, und entwendest etwas vom Gelde der
Ackers? Hättest du ihn doch wohl mögen behalten, da du ihn hattest; und da er
verkauft war, war es auch in deiner Gewalt. Warum hast du denn Solches in deinem
Herzen vorgenommen? Du hast nicht Menschen, sondern Gott gelogen."'\footnote{Ap.
Gesch. 5.} Und was hatte dieser Geiz und diese Heuchelei des Ananias zur Folge?
-- Als er Petrus so reden hörte, "`fiel er nieder und gab feinen Geist auf."'
Ein gleiches Loos traf seine Frau, die um den Betrug wußte, zu welchem der Geiz
sie beide verleitet halte. Auch lesen wir, "`daß über die ganze Gemeine und über
Alle, die es hörten, eine große Furcht kam."' Dieses sollte aber auch der Fall
bei Denen seyn, die es jetzt lesen; denn da dieser Gericht Gottes erfolgte, und
für uns aufgezeichnet ward, damit wir und vor ähnlichem Uebel hüten mögen, war
für ein Ende wird es denn mit Denen nehmen, die, als Bekenner des
Chrisienthumes, -- einer Religion, welche die Menschen lehret, der Welt zu
entsagen und Alles dem Willen und Dienste Christi und seines Reichs aufzuopfern,
-- nicht bloß einen Theil, sondern das Ganze ihres Vermögens zurückbehalten, und
sich nicht von der geringsten Sache um Christi willen trennen können. Ich bitte
Gott, daß er die Herzen meiner Leser zu einer ernsten Erwägung dieser
Gegenstände geneigt machen wolle! Es würde denm Ananiaes und der Sapphira ein
solchen Gericht nicht widerfahren seyn, wenn sie als in der Gegenwart Gottes und
nach der völligen Liebe, Wahrheit und Aufrichtigkeit gehandelt hätten, wie ihnen
gebührte. Möchten daher doch Alle sich des Lichtes bedienen, das Christus ihnen
verliehen hat! Möchten sie in diesem Lichte untersuchen und sehen, in wie fern
sie sich noch unter der Gewalt der Ungerechtigkeit des Geizes befinden! Denn,
wollten die Menschen nur gegen die Liebe zur Welt anf ihrer Hut stehen, und sich
weniger von sichtbaren und vergänglichen Dingen der Erde fesseln lassen, so
würden sie bald anfangen, ihre Herzen auf Dasjenige zu richten, was oben im
Himmel und von ewig dauernder Natur und Beschaffenheit ist. Dann würden sie auch
anfangen, "`ein mit Christo in Gott verborgenee Leben"' zu führen; ein Leben,
daß über die Ungewißheit der Zeit und über alle Unruhen und Veränderungen der
Sterblichkeit erhaben ist. Ja, möchten die Menschen nur erwägen, wie schwer es
ist, Reichtum zu erwerben, und wie ungewiß, ihn zu behalten; wie viel Neid er
erweckt; und daß er weder Weisheit verleihen, noch Krankheiten heilen, noch das
Leben verlängern, und noch vielweniger Frieden im Tode gewähren kann. Ja, die
wirklichen Vortheile, die der Reichthum dem Menschen verschafft, erstrecken sich
in der That kaum weiter, als auf Nahrung und Kleider, die man doch auch, ohne
reich zu seyn, erlangen kann. Und die gute Anwendung, die bessere Menschen davon
machen, bestehet darin, daß sie dem Elende und den Bedürfnissen Anderer damit
abhelfen; indem sie sich als Haushalter der reichen Gaben der göttlichen
Vorsehung betrachten, die von ihrem Haushalten Rechenschaft geben müssen. Ich
sage, wenn wir solchen Betrachtungen ihr gehörigers Gewicht in unsern, Gemüthern
verstatteten, so wurden wir uns nicht so eifrig bemühen, vergängliche Schätze zu
sammeln, und noch vielweniger sie ängstlich verbergen und aufbewahren. Und o!
möchte das Kreuz Christi, dieser Geist und diese Kraft Gottes im Menschen, doch
mehr Raum in unsern Seelen gewinnen, damit wir dadurch immer mehr und mehr der
Welt gekreuzigt würden, so daß auch uns die Welt gekreuzigt wäre, und, wie in
den paradiesischen Tagen, die Erde wieder der Fnßschemel, und die Schätze der
Erde die Diener, nicht die Götter des Menschen waren! -- Es haben schon Viele
gegen daß Laster des Geizes geschrieben, von denen ich hier drei anführen will.

\section{20. Abschnitt}

Wilhelm Tinbal, jener würdige Apostel der englischen Reformation, hat eine
vollständige Abhandlung unter dem Titel: Das Gleichniß vom ungerechten Mammon,
herausgegeben, woraus ich den Leser  Verweise. Der zweite Verfasser ist:

\section{21. Abschnitt}

Peter Charron, ein, besonders wegen seines. Buches über die Weisheit, berühmter,
französischer Schriftsteller. Er hat ein Kapitel gegen den Geiz, worin er sagt:
"`Reichthum lieben und daran hängen, ist Geiz; doch macht nicht allein diese
Liebe und Anhänglichkeit, sondern vielmehr das allzueifrige Streben und Trachten
nach Reichthum eigentlich den Geiz aus. Das Verlangen nach Reichthum, und das
Vergnügen, welches wir in seinem Besitze finden, gründet sich bloß auf unsere
Meinung davon. Daß unmäßige Verlangen, reich zu werden, ist wie ein Krebsschaden
in der Seele, der mit giftigen Brande unsere natürliche Neigungen ferzehrt, und
uns mir bösartigen Saften anfüllt. Sobald dieses Verlangen sich unserer Herzen
bemächtigt hat, erlöscht alle wahre und natürliche Zuneigung zu unsern Eltern
und Freunden, sogar zu und selbst; indem Alles, was nicht auf Gewinn Bezug hat,
uns als nichtig erscheint. Ja, wir vernachlässigen und vergessen endlich, um der
vergänglichen Dinge Willen, uns selbst, unsern Körper und Geist, und, wie daß
Sprichwort sagt, ‚verkaufen das Pferd um Heu anzuschaffen.‘

Der Geiz ist die schändlich und niedrige Leidenschaft gemeiner Thoren, welche
Reichthum für das edelste Gut des Menschen halten, und Armut als daß größte
Uebel scheuen? Diese, nicht zufrieden mit den nothwendigen Dingen des Lebens,
wägen die Güter der Erde in einer goldwage, da doch die Natur uns lehret, sie
nach dem Maßstabe des Bedürfnisses zu messen. Welche Thorheit kann aber wohl
größer seyn, als Desjenige zu verehren und anzubeten, was selbst die Natur, als
etwas unsere Anblickes Unwerthes, unter unsere Füße gelegt hat, um uns zu
zeigen, das es eher verachtet und mit Füßen getreten als angebetet werden müsse.
Dieses hat die Sünde des Menschen den Eingeweiden der Erde entrissen und ans
Licht gebracht, um sich damit zu töten. Wir durchwühlen das Innere der Erde, um
Etwas hervorzubringen, wofür wir kämpfen und streiten können, indem wir uns
nicht schämen, den Dingen, die in den niedrigsten Theilen der Erde liegen, einen
so hohen Werth beizulegen. Die Natur scheint schon durch die Beschaffenheit des
Bodens in welchem das Gold seine Bildung erhält, gewissermnßen das Elend Derer,
die es lieb gewinnen, vorher verkündigt zu haben; denn sie hat es so geordnet,
das in den Gegenden, wo man es findet, weder Gras noch Pf1anze, noch irgend
etwas Brauechbares wächst oder gedeihet, als habe sie uns dadurch andeuten
wollen, daß in den Gemüthern, in welchen die Liebe zu diesem Metalle eine
Vorherrschaft gewinnt, kein Funken von wahrer Ehre und Tugend aufkommen könne.
Und was kann auch niedriger seyn, als wenn der Mensch sich Dem unterwirft, und
Dem sklavisch gehorcht, was ihn unterworfen und dienstbar seyn sollte? Der
Reichtum ist des Weisen Diener, aber des Thoren Herr. Denn dem Geizigen dienet
sein Reichtum nicht; sondern Er dienet ihm. Man kann von ihm sagen: er hat
Güter, wie man von Jemand sagt: er hat das Fieber; denn so wie Jemand nicht das
Fieber, sondern das Fieber ihn besitzt, so besitzen und beherrschen auch den
Geizigen seine Güter, und er nicht sie. Ist wohl etwas schändicher als wenn der
Mensch daß liebt, was weder gut ist, noch Jemanden gut machen kann? das so
gemein ist, daß er sich auch in den Händen der gottlosesten Menschen befindet?
das oft sehr gute Sitten verdirbt, aber nie zu ihrer Veredlung dienet? ohne
welches schon so viele Weise glücklich waren, und wodurch so viele Gottlose zu
einem elenden Ende gebracht werden? Kurz, was ist scheußlicher, als Lebendige an
Todte binden, wie Mezentius that, damit ihr Tod desto schmachtender und
grausamer seyn möchte? nämlich, den Geist an den Auswurf und Abschaum der Erde
binden, um seine Seele mit tausend Qualen zu angstigen, welche die beständigen
Begleiter einer leidenschaftlichen Liebe zum Gelde sind; oder sich in den Banden
und Schlingen so schädlicher Dinge verwickeln, welche die heilige Schrift
''Dornen, Diebe,'' die das Herz des Menschen stehlen, ''Stricke des Teufels,
Abgötterei'' und die ''Wurzel alles Bösen'' nennt. Und wahrlich! wenn alle üble
Wirkungen des Neides und Verdrusses, die der Reichthum in den Herzen der
Menschen erzeugt, recht erkannt und erwogen werden, so muß man ihn eher hassen
und fliehen, als lieben und suchen. Die Armuth muß freilich Vieles entbehren,
der Geiz entbehrt aber Alles! Der Geizige ist keines Menschen Freund, aber sein
eigener größter Feind!"' -- So sagt Charron, ein weiser und großer Mann. Mein
drittes Zeugniß nehme ich aus einem Schriftsteller, der wahrscheinlich Einigen
seines Wizes wegen sehr gefallen wird. Möge man aber auch seine moralischen
Bemerkungen und seine reife Beutheilung eben so sehr schätzen.

\section{22. Abschnitt}

Abraham Cowley, ein witziger und talentvoller Mann, liefert uns folgende
Bemerkungen über den Geiz: "`Es giebt zwei Gattungen drei Geizes. Die eine ist
nur eine Bastardart, und bestehet in einem eifrigen Streben nach Gewinn, nicht
aus Liebe zum Gelde selbst, sondern um das Vergnügen zu haben, dasselbe auf
allen Wegen den Stolzes und der Ueppigkeit wieder zu verschwenden. Die andere
Gattung ist die wahre Art des Geizes, und wird mit Recht so genannt, denn sie
bestehet in einer rastlosen und unersättlichen Begierde nach Reichthum, und zwar
aus keiner andern Ursache, und zu keinem andern Zwecke oder Gebrauehe, als um
ihn auzuhäufen, zu bewahren und beständig zu vermehren. Der Geizige der ersten
Gattung ist ein gierigen Strauße gleich, der jedes Stuck Metall verschlingt;
doch in der Absicht, sich davon zu nähren; daher es sich denn auch alle Mühe
giebt, es zu kauen und zu verdauen. Der Andere gleicht der thörichten Krähe, die
nur Geld stiehlt, um es zu verstecken. Der Erstere thut der menschlichen
Gesellschaft viel Schaden; doch einigen Personen zuweilen auch Gutes. Der
Letztere hingegen nützt Niemand, ja, auch sich selbst nicht. Der Erstere kann
seine Handlungen weder vor Gott und den Engeln, noch vor vernünftigen Menschen
entschuldigen; der Andere kann für Das was er that, auch keinen Schatten von
Entschuldigung vorbringen; er dient dem Maunnon als Sklare, ohne Lohn. Der
Erstere weiß sich noch bei einigen beliebt, ja, sogar beneidenswerth zu machen;
der Letztere ist allgemeiner Gegenstand des Hasses und der Verachtung. Er giebt
kein Laster, auf welches man sie Viele Sinngedichte gemacht hätte, und das
besonders von Dichtern durch Satyren, Fabeln, Allegorien und witzige
Anspielungen in ein so verhaßtes Licht gestellt und von allen Seiten so scharf
getadelt worden wäre, als der Geitz. Doch erinnere ich mich keiner feinern
Bestrafung desselben, als der des ''Ovids'' in folgender Zeile:

<poem>
– – – – – – – – <tt>Multa
Luxuriae desunt, omnia avaritiae.</tt>

Dein Luxus mangelt vieles; dem Geize aber Alles.
</poem>
Diesem Ausspruche möchte ich noch ein paar Worte beifügen, und ihn so geben:
<poem>
Der Armuth mangelt Manches; der Ueppigkeit sehr Viel;
Allein des Geitzes Mangel ist ohne Maß und Ziel.
</poem>
Auch sagt Jemand von einem tugendhaften und weisen Manne: Indem er Nichts
besitzt, hat er Alles. Dieser ist also der wahre Gegenfüßler des Geitzigen, der,
indem er Alles hat, doch Nichts besitzt. Und o! wer kann wohl elender seyn, als
Derjenige, der, in Ueberflusse schmachtend, des Himmels Segen sich selbst zum
Fluche macht? Fühlt nicht der arme genügsame Bettler beim Genusse einer geringen
Gabe sich glücklicher, als der unersättliche Reiche, der bei allen seinen
Schätzen doch unaussprechlich arm und elend ist? –

Es wundert mich, daß man noch kein Gesetz gegen den Geitz gemacht hat. Doch was
sag’ ich? gegen ihn? – Für ihn, zu seinen Gunsten, wollt’ ich sagen. Denn da man
für alle Arten Wahnsinnige öffentliche Vorsorge trägt, so würde es unstreitig
auch sehr angemessen seyn, wenn der König einige Personen ernennete, die das
Vermögen der Geizigen während ihrer Lebenszeit veralteten, -- denn ihre Erben
bedürfen gewöhnlich einer solchen Vorkehrung nicht, -- und deren Geschäft es
wäre, dahin zu sehen, daß es ihnen nicht an dem nötigsten Unterhalte mangelte,
den ihr Stand und ihre Lage erfordert; weil sie so grausam sind, sich diesen
selbst zu versagen. Wir helfen ja müßigen Tagedieben und verstellten Bettlern;
warum wollten wir uns denn nicht um diese wirklich armen Leute bekümmern, die
man üdrigens, dünkt mich, in Rücksicht anf ihren Stand, auch mit gehöriger
Achtung behandeln müßte. – Ich möchte unauhärlich über diese Menschen reden,
aber das Viele, wias sich über sie sagen ließe, erstickt mich, und es geht mit
fast eben so, wie ihnen: der Ueberfluß macht mich arm!"'

Dieß sei genug vom Geize, -- diesem an der Seele nagenden Wurme und Krebsschaden
des Geistes.


