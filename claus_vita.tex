\pagebreak

% \addcontentsline{toc}{chapter}{Über Claus Bernet}
% \addcontentsline{toc}{chapter}{\protect\numberline{1}{Über Claus Bernet}}

% \chapter{Über Claus Bernet}
\section{Über Claus Bernet}
\begin{footnotesize}

Claus Bernet studierte Geschichtswissenschaften, Kunstgeschichte und
Stadtplanung in Berlin, Wien und Birmingham (UK). Die Dissertation im Bereich Frühe Neuzeit (magna cum laude)
erfolgte 2005 an der Martin-Luther-Universität Halle. Parallel dazu studierte er Erziehungswissenschaften an der Freien Universität Berlin.
2007 promovierte er mit einer Arbeit zum Erziehungsgedanken in nichtstaatlichen
Einrichtungen (summa cum laude).

\medskip

Lehrveranstaltungen zu Themen der Gewaltprävention (PAG) und zur neosokratischen
Gesprächsführung; 2002/03 Mitarbeit beim Zentrum für Demokratische Kultur in
Berlin. 2008/09 koordinierte Bernet das Interview-Projekt "`Erweckte
Geschichte"', ebenfalls in Berlin.

\medskip

Derzeit arbeitet er an einer Studie zu deutsch-italienischen Intellektuellen um
1900 mit einem Stipendium des Literaturarchivs Marbach 2008. Des weiteren ist er
seit Jahren tätig als Mitarbeiter für die Lebenshilfe e.V. in Berlin.

\medskip

Die Forschungsgebiete von Claus Bernet sind historische Friedenskirchen,
Pietismus, Stadtgeschichte und Genossenschaftswesen.

\medskip

Folgende Veröffentlichungen zur Quäkerforschung sind bisher von Bernet
erschienen:

\begin{itemize}
 \item Rufus Jones (1863-1948). Life and Bibliography of an American Scholar,
Writer, and Social Activist. With a Foreword by Douglas Gwyn, New York 2009.
 \item Quäker aus Politik, Kunst und Wissenschaft in Deutschland. 20.
Jahrhundert. Ein biographisches Lexikon, Nordhausen 2007. 2. Auflage Nordhausen
2008. 3. Auflage in Planung.
 \item Between Quietism and Radical Pietism: The German Quaker Settlement
Friedensthal. 1792-1814, Birmingham 2004 (Woodbrooke Journal Series, 14).
\end{itemize}

\medskip


Claus Bernet ist auch Herausgeber der Reihe \textit{"`Deutsche Quäkerschriften"'}. Zahlreiche
Beiträge von ihm sind in der Zeitschrift "`Quäker"' -- dem Organ der Deutschen
Jahresversammlung e.V. -- erschienen.
Er hält Vorträge in Deutschland, Großbritannien und den USA zu Themen der Geschichte, Theologie und Soziologie
der Quäker. Seit 2004 ist er Mitglied im Board of Trustees der englischen
Weiterbildungseinrichtung Woodbrooke in Birmingham.

\medskip

\begin{center}
\parbox{7,5cm}{
Tauroggener Str.2
\\10589 Berlin
\\\texttt{forschung@bernetc.com}
}
\end{center}

\end{footnotesize}
