\pagebreak

\section{Über Claus Bernet}

Claus Bernet studierte Geschichtswissenschaften, Kunstgeschichte und 
Stadtplanung in Berlin, Wien und Birmingham (UK). Die Dissertation im Bereich 
Frühe Neuzeit (magna cum laude) erfolgte 2005 an der Martin-Luther-Universität 
Halle. Parallel dazu studierte er Erziehungswissenschaften an der Freien 
Universität Berlin. 2007 promovierte er mit einer Arbeit zum Erziehungsgedanken 
in nichtstaatlichen Einrichtungen (summa cum laude).

\medskip

Lehrveranstaltungen zu Themen der Gewaltprävention (PAG) und zur neosokratischen 
Gesprächsführung; 2002/03 Mitarbeit beim Zentrum für Demokratische Kultur in 
Berlin. 2008/09 koordinierte Bernet das Interview-Projekt 
"`Erweckte Geschichte"', ebenfalls in Berlin, anschließend erarbeitete er am 
Literaturarchiv Marbach eine Studie zu deutsch-italienischen Intellektuellen um 1900 

\medskip

Tätig als Berater für die Lebenshilfe e.V. in Berlin (2008-2012), seit 2013 
fest angestellt bei Sehstern e.V. Berlin, psychosoziale Beratung und Betreuung.

\medskip

Die Forschungsgebiete von Claus Bernet sind historische Friedenskirchen, J
erusalemsvorstellungen, Genossenschaftswesen und Behindertenpädagogik.

\medskip

Folgende Veröffentlichungen zur Quäkerforschung sind bisher von Bernet erschienen:

\begin{itemize}
 \item Das Quäkertum in Pyrmont, Friedensthal und Minden. Von den Revolutionskriegen bis zum Kaiserreich, Hamburg 2016 (im Erscheinen). 

 \item Klaus Fuchs-Kittowski, Claus Bernet (Hrsg.): Emil Fuchs: Der Brief des Paulus an die Römer, Hamburg 2015 (Theos – Studienreihe Theologische Forschungsergebnisse, 121).

 \item Rufus Jones (1863-1948). Life and Bibliography of an American Scholar, Writer, and
Social Activist. With a Foreword by Douglas Gwyn, New York 2009.

 \item Quäker aus Politik, Kunst und Wissenschaft in Deutschland. 20. Jahrhundert. Ein
biographisches Lexikon, Nordhausen 2007. 2. Auflage Nordhausen 2008. 3. Auflage
in Planung.

 \item Between Quietism and Radical Pietism: The German Quaker Settlement Friedensthal.
1792-1814, Birmingham 2004 (Woodbrooke Journal Series, 14).
\end{itemize}

\medskip

Claus Bernet ist auch Herausgeber der Reihe "`Deutsche Quäkerschriften"'. 
Zahlreiche Beiträge von ihm sind in der Zeitschrift "`Quäker"' (dem Organ der 
Deutschen Jahresversammlung e.V.) erschienen. Er hält Vorträge in Deutschland, 
England und den USA zu Themen der Geschichte, Theologie und Soziologie der 
Quäker. 

\medskip

Kontakt: \\\texttt{office@bernetc.com}


