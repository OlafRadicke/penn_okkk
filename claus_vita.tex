\chapter{Über Claus Bernet}

Studium der Geschichtswissenschaften, Kunstgeschichte und Stadtplanung in Berlin, Wien und Birmingham (UK), Dissertation 2005 an der Martin-Luther-Universität Halle im Bereich Frühe Neuzeit (magna cum laude). Parallel dazu Studium der Erziehungswissenschaften an der Freien Universität Berlin, 2007 Promotionsarbeit zum Erziehungsgedanken in nichtstaatlichen Einrichtungen (summa cum laude). Lehrveranstaltungen zur Gewaltprävention (PAG) und zur neosokratischen Gesprächsführung; 2002/03 Mitarbeit beim Zentrum für Demokratische Kultur (Berlin), 2008/9 Koordinierung des Interview-Projektes „Erweckte Geschichte“. Derzeit Arbeit an einer Studie zu deutsch-italienischen Intellektuellen um 1900 (Stipendiat des Literaturarchivs Marbach 2008). Ansonsten tätig als Mitarbeiter für die Lebenshilfe e.V. in Berlin.
Forschungsgebiete: historische Friedenskirchen, Pietismus, Stadtgeschichte, Genossenschaftswesen.

Veröffentlichungen auf dem Gebiet der Quäkerforschung:
-Rufus Jones (1863-1948). Life and Bibliography of an American Scholar, Writer, and Social Activist. With a Foreword by Douglas Gwyn, New York 2009.
-Quäker aus Politik, Kunst und Wissenschaft in Deutschland. 20. Jahrhundert. Ein biographisches Lexikon, Nordhausen 2007. 2. Auflage Nordhausen 2008. 3. Auflage in Planung.
-Between Quietism and Radical Pietism: The German Quaker Settlement Friedensthal. 1792-1814, Birmingham 2004 (Woodbrooke Journal Series, 14).

Daneben Herausgeber der Serie „Deutsche Quäkerschriften“. Zahlreiche Beiträge in der Zeitschrift „Quäker“ und Vorträge in Deutschland, UK und USA zur Geschichte und Theologie der Quäker. Mitglied im Board of Trustees der englischen Weiterbildungseinrichtung Woodbrooke in Birmingham seit 2004.


Kontakt: Tauroggener Str.2, 10589 Berlin. bernet2@web.de
