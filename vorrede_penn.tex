\chapter{Vorrede}

Die größte Angelegenheit des Lebens eines jeden Menschen ist die, dass er dem
Zweck seines Daseins entspreche; und dieser ist: Gott zu verherrlichen und seine
eigene Seele zu retten; -- eine Verordnung des Himmels, die so alt wie als die
Welt ist. Gewöhnlich kümmert sich aber der Mensch am wenigsten um das, was seine
Hauptsorge und wichtigste Beschäftigung sein sollte. Er ist abgeneigt, sich
selbst kennen zu lernen, Untersuchungen über sein Dasein, über den Ursprung, die
Pflichten und das Ende seines Lebens anzustellen. Lieber wendet er seine Tage
-- welche eben so viele Schritte zu seiner ewigen Wohlfahrt sein sollten -- nur
dazu an, dass er seinen Stolz, seine Gier
\footnote{\texttt{Anmerkung des Herausgebers:
Im Original wird das Wort 'Geiz' verwendet. Im weiteren Verlauf des
Textes wird aber klar, das sich die Bedeutung des Wortes heute etwas geändert
hat. Heute würde man das Wort 'gier' verwenden. Im weiteren Text wird also das
Wort 'Geiz' durch 'gier' ausgetauscht. Zu der Veränderung der Bedeutung von
dem Wort 'Geiz' siehe auch: Seite 'Geiz', in Wikipedia, Die freie
Enzyklopädie. Bearbeitungsstand: 4. August 2009, 10:48 UTC. URL:
http://de.wikipedia.org/w/index.php?title=Geiz\&oldid=62956916 (Abgerufen: 16.
August 2009, 13:07 UTC)}} und die Lüsternheit seines Herzens zu befriedigen
sucht.
\label{ref:vorw_sinndes_lebens} 
\textbf{Als wenn er bloß um seiner selbst willen da sei, oder als ob er
sich selbst das Dasein gegeben habe, und daher keiner höheren Macht Rechenschaft
schuldig ist, und ihrem Urteilsspruch auch nicht unterworfen ist.}

\medskip

In diesen verwilderten, beklagenswerten Zustand stürzt der Mensch, durch seinen
Ungehorsam gegen das Gesetz Gottes in seinem Herzen, sich selbst; indem er das
tut, was er, wie er wohl weiß, nicht tun sollte, und das unterlässt, was er,
seiner Erkenntnis nach, tun müsste. So lange nun dieser Krankheitszustand bei
dem Menschen andauert, macht er sich seinen Gott zum Feind, und sich selbst der
Liebe und Seligkeit unfähig, welche Gott durch seinen Sohn Jesus Christus der
Welt geoffenbaret hat.

\medskip

Gehörst du, mein Leser, zu dieser Klasse, so gebe ich dir den Rat: Kehre in dich
selbst ein, und untersuche den Zustand deiner Seele. Christus hat dir Licht
verliehen, dass du es tun kannst. Forsche sorgfältig, untersuche gründlich. Dein
Leben hängt davon ab; es gilt das Heil deiner Seele, das du nicht wieder
erlangen kannst, wenn du es einmal verloren hast. Wenn du hierin dich selbst
betrügst, so ist der Verlust unersetzlich; du kannst um die ganze Welt dich
nicht wiedererkaufen. Willst du denn um dieser niedrigen Welt willen dich selbst
aufs Spiel setzen? Die Zeit deines Heils versäumen, und deine Seele verlieren?
-- Ich gebe zu, daß du es mit einem Gott von großer Geduld zu tun hast; aber
die Zeit, in der du Gottes Geduld auf die Probe setzt, muss doch auch ein Ende
nehmen. Reize daher den Gott, der dich erschaffen hat, nicht, dich endlich zu
verwerfen. Weißt du, was das sagen will? -- Es heißt, in den Abgrund, in die
Hölle, in die ewige Seelenangst der Verdammten gestürzt werden. -- Oh Leser! Ich
bitte dich, als einer, der den Schrecken der Gerichte des Herrn erfahren hat,
sei ernsthaft, sei fleißig und eifrig um dein Heil bemüht! Ja, auch als einer,
der den Trost, den Frieden, die Freude und Seligkeit der Wege der Gerechtigkeit
kennt, ermahne ich dich, und lade dich ein, die Bestrafungen und Überzeugungen
des Lichtes oder Geistes Christus' in deinem eigenen Gewissen anzunehmen, und
dich seinen Gerichten zu unterziehen; da du dich der Sünde schuldig gemacht
hast. Das Feuer verbrennt nur die Stoppeln -- der Wind wehet nur die Spreu
hinweg! Übergib dich mit Leib, Seele und Geist dem, der alles neu macht; der
einen neuen Himmel und eine neue Erde, neue Liebe, neue Freude, neuen Frieden,
neue Werke, ein neues Leben und einen neuen Wandel hervorbringt. Die Menschen
sind durch die Sünde verderbt und gleichsam schlackig geworden; durch Feuer,
(nämlich durch geistiges Feuer) müssen sie von ihren Schlacken gereinigt,
geläutert und zur Seligkeit fähig gemacht werden. Darum wird das Wort Gottes
einem Feuer verglichen; der Tag des Heils einem Ofen, und Christus selbst dem
Schmelzer, der das Silber läutert\footnote{\texttt{Das 'läutert' bedeutet
'reinigen'}}.

\medskip

Wohlan, Leser! Höre mich ein wenig an. Ich suche dein Heil, das ist meine
einzige Absicht, die wirst du mir verzeihen. -- Der Schmelzer\footnote{\texttt{...Also
Jesus Christus}} ist dir nahe, seine Gnade ist dir erschienen, sie zeigt dir die
Lüste der Welt und lehrt dich, sie zu überwinden
\footnote{\texttt{Im Original wird das
Word 'verleugnen' verwendet. Da es heute so nicht mehr verwendet wird, wurde
es -- auch im künftigen Text -- durch 'überwinden' ersetzt}}. Lass dieselbe,
als den geistigen Sauerteig des Himmelreichs, dein Herz durchdringen, und sie
wird dich gänzlich umwandeln. Christus ist der wahre Arzt für die Seele;
gebrauche seine Arznei, sie wird dich heilen. Er ist eben so unfehlbar als
freigebig, er heilt umsonst und mit Gewissheit. Eine Berührung seines Gewandes
war ehemals hinreichend\footnote{\texttt{Vgl. Matthäus 9,19-22}}, die Genesung zu
bewirken, und sie ist es noch heute. Seine Kraft ist noch dieselbe, und sie ist
unerschöpflich, weil "`in ihm die ganze Fülle der Gottheit wohnt."' Und gelobet
sei Gott für seine Allgenugsamkeit! "`dass er mächtig ist, allen zu helfen, und
alle selig zu machen, die durch ihn zu Gott kommen."' Komm denn nur zu ihm, so
wird er eine selige Veränderung in dir hervorbringen, ja er wird deinen
nichtigen Leib seinem verklärten Leibe ähnlich machen. Er ist in der Tat der
große Philosoph, die Weisheit Gottes, die Blei in Gold, nichtswürdige Dinge, in
köstliche verwandelt; denn er macht aus Sündern Heilige und aus Menschen fast
Götter. -- Was haben wir aber nun zu tun, um zu dieser Erfahrung zu gelangen,
damit wir von seiner Macht und Liebe zeugen können? \index{Krone!Bedeutung} \index{Kreuz!Bedeutung} \textbf{Dieses ist die
Krone; aber wo ist das Kreuz?} Der bittere Kelch, die Feuertaufe? -- Fass Mut,
Leser! Sei wie er! Erhebe, um der alles übersteigenden Freude willen, dein Haupt
über die Welt empor, und deine Erlösung \footnote{\texttt{Im Original wird das Wort
'Heil' statt 'Erlösung' verwendet. Ist heute ungebräuchlich und erweckt
vielleicht noch Assoziationen zum Nationalsozialismus.}} wird dir in der Tat
nahe sein.

\medskip

\textbf{Das Kreuz Christi ist das Mittel, zu der Krone Christi zu gelangen.}
Dieses ist der Gegenstand der folgenden Abhandlung, die ich zuerst im Jahre 1668
während meiner Gefangenschaft \index{Orte:!Gefängnis} im Tower (Turm) zu London \index{Orte:!Tower zu
London} schrieb, und sie ist später, mit vielen Zusätzen vermehrt, wieder
aufgelegt worden, damit du, mein Leser, für Christus gewonnen werden mögest,
oder wenn du schon gewonnen bist, ihm näher gebracht wirst. Es ist der Pfad, auf
welchen Gott in seiner unendlichen Güte meine Füße in der Blüte meiner Jugend
leitete, als ich ungefähr 22 Jahre alt war. Da nahm er mich bei der Hand, und
führte mich weg von den Vergnügungen, Eitelkeiten und Hoffnungen der Welt. Ich
habe sowohl die Gerichte Christi als auch seine Barmherzigkeit, und auch den
Hass und Tadel der Welt erlebt \footnote{\texttt{Im Origimal wird 'geschmeckt' statt 'erlebt' verwendet}}; und ich freue mich meiner Erfahrungen, die ich nun deinem
Dienste in Christus widme. Es ist eine Schuld, die schon eine geraume Zeit auf
mir lag, und deren Abzahlung \footnote{\texttt{Im Original 'Abtragung' statt
'Abzahlung'}} man längst von mir erwartete. Jetzt habe ich mich ihrer
entledigt, und meine Seele davon befreit. -- Ich hinterlasse dieser Werk meinem
Vaterland und der ganzen Christenheit. Möge es Gott auf alle, die es lesen,
einwirken lassen! Möge er ihre Herzen ablenken von allem Neid und Hass, und von
aller Bitterkeit, die sie gegeneinander, um vergänglicher Dinge willen, in
einem solchen Grade hegen, dass sie jedes Gefühl von Menschlichkeit und Mitleid
dem Ehrgeiz und der Habsucht zum Opfer bringen, und die Erde mit Unruhe und
Bedrückung erfüllen. Und mögen sie, indem sie den Geist Christi -- dessen
Früchte Liebe, Friede, Freude, Mäßigkeit, Geduld, Bruderliebe und allgemeine
Liebe sind -- in ihren Herzen aufnehmen, mit Leib, Seele und Geist einen
dreifachen Bund schließen gegen die Welt, das Fleisch und den Teufel, die
gemeinschaftlichen Feinde der Menschen, und wenn sie dieselben, während eines
Lebens der Selbstüberwindung \footnote{\texttt{Im Original wird das Word
'Selbstverleugnung' benutz. Im Weiteren wird es duch 'Selbstüberwindung'
ersetzt, da es dem heutigen Wortgebrauch mehr entspricht}}, durch die Kraft des
Kreuzes von Jesus überwunden haben, und dann zur ewigen Ruhe im Reiche Gottes
gelangen, und eine Krone der Gerechtigkeit empfangen!


\medskip

Dieses, freundlicher Leser, ist der Wunsch und das Gebet deines wahrhaft
christlichen Freundes --- \textbf{Wilhelm Penn}.
