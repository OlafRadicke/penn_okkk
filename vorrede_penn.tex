\chapter{Vorrede}

Die große Angelegenheit des Lebens eines jeden Menschen ist die, daß er dem Zwecke seines Daseyns entspreche; und dieser ist: Gott zu verherrlichen und seine eigene Seele zu retten; – eine Verordnung des Himmels, die so alt als die Welt ist. Gewöhnlich bekümmert sich aber der Mensch am wenigsten um Das, was seine Hauptsorge und wichtigste Beschäftigung seyn sollte. Er ist abgeneigt, sich selbst kennen zu lernen, Untersuchungen über sein Daseyn, über den Ursprung, die Pflichten und das Ende seines Lebens anzustellen. Lieber wendet er seine Tage, – welche eben so viele Schritte zu seiner ewigen Wohlfahrt seyn sollten, – nur dazu an, daß er seinen Stolz, seinen Geitz und die Lüsternheit seines Herzens zu befriedigen sucht. Als wenn er bloß um sein selbst willen da sei, oder als ob er sich selbst das Daseyn gegeben habe, und daher keiner höhern Macht Rechenschaft schuldig, und ihrem Urteilsspruch nicht unterworfen sei.

Zeilenumbruch für die Labeled Section Transclusion In diesen verwilderten, beklagenswerten Zustand, stürzt der Mensch, durch seinen Ungehorsam gegen das Gesetz Gottes in seinem Herzen, sich selbst; indem er das thut, was er, wie er wohl weiß, nicht thun sollte, und das unterläßt, was er, seiner Erkenntniß nach, thun müßte. So lange nun dieser Krankheitszustand bei dem Menschen dauert, macht er seinen Gott sich zum Feinde, und sich selbst der Liebe und Seligkeit unfähig, welche Gott durch seinen Sohn Jesum Christum der Welt geoffenbaret hat.

Gehörst du, mein Leser, zu dieser Klasse, so gebe ich dir den Rath: Kehre in dich selbst ein, und untersuche den Zustand deiner Seele. Christus hat dir Licht verliehen, daß du es thun kannst. Forsche sorgfältig; untersuche gründlich. Dein Leben hängt davon ab; es gilt das Heil deiner Seele, das du nicht wieder erlangen kannst, wenn du es einmal verloren hast. Wenn du hierin dich selbst betrügst, so ist der Verlust unersetzlich; du kannst um die ganze Welt dich nicht wiedererkaufen. Willst du denn um dieser niedrigen Welt willen dich selbst aufs Spiel setzen? die Zeit deines Heils versäumen, und deine Seele verlieren? – Ich gebe dir zu, daß du es mit einem Gott von großer Geduld zu thun hast; aber die Zeit, in der du Gottes Geduld auf die Probe setzest, muß doch auch ein Ende nehmen. Reize daher den Gott, [V] der dich erschaffen hat, nicht, dich endlich zu verwerfen. Weißt du, was das sagen will? – Es heißt, in den Abgrund, in die Hölle, in die ewige Seelenangst der Verdammten gestürzt werden. – O Leser! ich bitte dich, als Einer, der den Schrecken der Gerichte des Herrn erfahren hat, sei ernsthaft; sei fleißig und eifrig um dein Heil bemüht! Ja, auch als Einer, der den Trost, den Frieden, die Freude und Seligkeit der Wege der Gerechtigkeit kennt, ermahne ich dich, und lade dich ein, die Bestrafungen und Ueberzeugungen des Lichtes oder Geistes Christi in deinem eigenen Gewissen anzunehmen, und dich seinen Gerichten zu unterziehen; da du dich der Sünde schuldig gemacht hast. Das Feuer verbrennt nur die Stoppeln; – der Wind wehet nur die Spreu hinweg! Uebergieb dich mit Leib, Seele und Geist Dem, der Alles neu macht; der einen neuen Himmel und eine neue Erde, neue Liebe, neue Freude, neuen Frieden, neue Werke, ein neues Leben und einen neuen Wandel hervorbringt. Die Menschen sind durch die Sünde verderbt und gleichsam schlackig geworden; durch Feuer, (nämlich durch geistiges Feuer,) müssen sie von ihren Schlacken gereinigt, geläutert und zur Seligkeit fähig gemacht werden. Darum wird das Wort Gottes einem Feuer verglichen; der Tag des Heils einem Ofen, und Christus selbst dem Schmelzer, der das Silber läutert.

Wohlan, Leser! Höre mich ein wenig an. Ich suche dein Heil, das ist meine einzige Absicht; die wirst du mir verzeihen. – Der Schmelzer ist dir nahe; seine Gnade ist dir erschienen; sie zeigt dir die Lüste der Welt, und lehret dich, sie zu verleugnen. Laß dieselbe, als den geistigen Sauerteig des Himmelreichs, dein Herz durchdringen, und sie wird dich gänzlich umwandeln. Christus ist der wahre Arzt für die Seele; gebrauche seine Arznei, sie wird dich heilen; Er ist eben so unfehlbar als freigebig; er heilt umsonst, und mit Gewißheit. Eine Berührung seines Gewandes war ehemals hinreichend, die Genesung zu bewirken, und sie ist es noch. Seine Kraft ist noch dieselbe; und sie ist unerschöpflich, weil „in ihm die ganze Fülle der Gottheit wohnt.“ Und gelobet sei Gott für seine Allgenugsamkeit! „daß er mächtig ist, Allen zu helfen, und Alle selig zu machen, die durch ihn zu Gott kommen.“ Komm denn nur zu ihm, so wird er eine selige Veränderung in dir hervorbringen, ja er wird deinen nichtigen Leib seinem verklärten Leibe ähnlich machen. Er ist in der That der große Philosoph, die Weisheit Gottes, die Blei in Gold, nichtswürdige Dinge in köstliche verwandelt; denn er macht aus Sündern Heilige und aus Menschen fast Götter. – Was haben wir aber [VII] nun zu thun, um zu dieser Erfahrung zu gelangen, damit wir von seiner Macht und Liebe zeugen können? Dieses ist die Krone; aber wo ist das Kreuz? der bittere Kelch, die Feuertaufe? – Faß Muth, Leser! Sei wie Er! Erhebe, um der Alles übersteigenden Freude willen, dein Haupt über die Welt empor, und dein Heil wird dir in der That nahe seyn.

Das Kreuz Christi ist das Mittel, zu der Krone Christi zu gelangen. Dieses ist der Gegenstand der folgenden Abhandlung, die ich zuerst im Jahre 1668 während meiner Gefangenschaft im Tower (Thurm) zu London schrieb, und sie ist hernach, mit vielen Zusätzen vermehrt, wieder aufgelegt worden, damit du, mein Leser, für Christum gewonnen werden mögest, oder wenn du schon gewonnen bist, ihm naher gebracht werdest. Es ist der Pfad, auf welchen Gott in seiner unendlichen Güte meine Füße in der Blüthe meiner Jugend leitete, als ich ungefähr zwei und zwanzig Jahre alt war. Da nahm er mich bei der Hand, und führte mich hinweg von den Vergnügungen, Eitelkeiten und Hoffnungen der Welt. Ich habe sowohl die Gerichte Christi als auch seine Barmherzigkeit, und auch den Haß und Tadel der Welt geschmecket; und ich freue mich meiner Erfahrungen, die ich nun deinem Dienste in Christo widme. Es ist eine Schuld, die schon eine geraume Zeit auf mir lag, und deren Abtragung man längst von mir erwartete. Jetzt habe ich mich ihrer entledigt, und meine Seele davon befreiet. – Ich hinterlasse dieser Werk meinem Vaterlande und der ganzen Christenheit. Möge es Gott auf Alle, die es lesen, einwirken lassen! Möge er ihre Herzen ablenken von allem Neide und Hasse, und von aller Bitterkeit, die sie gegen einander, um vergänglicher Dinge willen, in einem solchen Grade hegen, daß sie jedes Gefühl von Menschlichkeit und Mitleiden dem Ehrgeitze und der Habsucht zum Opfer bringen, und die Erde mit Unruhe und Bedrückung erfüllen. Und mögen sie, indem sie den Geist Christi, – dessen Früchte Liebe, Friede, Freude, Mäßigkeit, Geduld, Bruderliebe und allgemeine Liebe sind, – in ihren Herzen aufnehmen, mit Leib, Seele und Geist einen dreifachen Bund schließen gegen die Welt, das Fleisch und den Teufel, die gemeinschaftlichen Feinde des Menschengeschlechts, und wenn sie dieselben, während eines Lebens der Selbstverleugnung, durch die Kraft des Kreuzes Jesu überwunden haben, endlich zur ewigen Ruhe im Reiche Gottes gelangen, und eine Krone der Gerechtigkeit empfangen!

Dieses, freundlicher Leser, ist der Wunsch und das Gebet deines wahrhaft christlichen Freundes -- Wilhelm Penn.