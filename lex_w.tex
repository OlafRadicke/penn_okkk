\section*{W}

\articlesize

\begin{description}
 \item[Weighty Friends]
    a Friend, respected for their experience and ability over their history of participation with Friends, whose opinion or ministry is especially valued.

 
 \item[Wehrdienst]

 \item[Wetherill, Samuel] $\to$\textit{Free Quakers}.

 \item[Whittier, John Greenleaf]

 \item[Wilbur, John] (July 17, 1774 – May 1, 1856) was a prominent American Quaker minister and religious thinker who was at the forefront of a controversy that led to "the second split" in the Religious Society of Friends in the United States.

 \item[Wilburite-Gurneyite-Schisma] $\to$\textit{Gurney-Wilbur-Schisma}.

 \item[Wilburite Quakers]

 \item[Wilburite(s)]

 \item[Winstanley, Gerrard] $\ast$1609 in Wigan, Lancashire, \dag10.9.1676) war
 ein protestantischer Reformer und politischer Aktivist in England. Er führte
 zur Zeit des Protektorats (1649–1658) von $\to$Oliver Cromwell die Gruppe der
 \textit{"`True Levellers"'} (englisch: wahre Gleichmacher) an, die von ihren
 Gegnern abfällig \textit{"`Diggers"'} (englisch: Buddler) genannt wurden.
 Winstanley propagierte
 Gütergemeinschaft. Er betrieb kurke Zeit ein Projekt wo er mit anderen Mitstreitern
ungenutztes Land besetzde um es zu kultivieren. Die Erträge wurden gemeinsam
genutzt und die Erträge die über den Eigenbedarf hienausgingen, wurden verschenkt.
Das Projekt wurde nach wenigen Monaten gewaltsam aufgelöst. In Winstanley sehen
vierschiedene Gruppen ihren Vordenker. So die Kommunisten, als auch die Quaker.
Spähter, bei einem seiner Gefängnisaufenthalte, trat er dann auch den Quakern
bei. 

\item[Wochentage] $\to$\textit{Datum-Format}

\item[Woolman, John] $\ast$19.10.1720 in Northampton, New Jersey; \dag7.10.1772 in York. Amerikanischer Quäker-Prediger und vehementer Gegner der Sklaverei. Bekannt durch seine \textit{"`Aufzeichnungen"'} (englisch: \textit{Journal}).\endnote{Die Aufzeichnungen von John Woolman. Aus der Zeit der Sklavenbefreiung. 2. Auflage, Lorber U. Turm Verlag, 1964, ISBN 3799901019}

\item[Worship Group] $\to$\textit{Andachsgruppe}

 \end{description}


\normalsize