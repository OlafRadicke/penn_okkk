
\chapter{Kapitel 15} \label{kap15}

\section{Zusammenfassung des 15. Kapitel}

\begin{description}
\item[1. Abschnitt] Ankündigungen der Gerichte Gottes über die Juden, wegen
ihrer Maßlosigkeit\footnote{\texttt{'Üppigkeit' erstzt durch 'Maßlosigkeit'}}, wovon
kein Stand ausgenommen ist
\dotfill \textit{Seite~\pageref{kap15_ab1}}\\
\item[2. Abschnitt] Christus warnt seine Jünger, sich des
maßlosen\footnote{\texttt{'üppigen' ersetzt durch 'maßlosen'}} Lebens nicht
schuldig zu machen. -- Eine Ermahnung an die Bewohner Englands.
\dotfill \textit{Seite~\pageref{kap15_ab2}}\\
\item[3. Abschnitt] Mäßigkeit wurde von den Aposteln den Gemeinden dringend
empfohlen.
\dotfill \textit{Seite~\pageref{kap15_ab3}}\\
\item[4. Abschnitt] Wohlgemeinter Rat an England, sich nach dieser Richtschnur
zu prüfen.
\dotfill \textit{Seite~\pageref{kap15_ab4}}\\
\item[5. Abschnitt] Worin die Erholung der Christen besteht.
\dotfill \textit{Seite~\pageref{kap15_ab5}}\\
\item[6. Abschnitt] Wer anderer Vergnügungen bedarf, um sich die Zeit zu
vertreiben, der ist für den Himmel und die Ewigkeit nicht geschickt.
\dotfill \textit{Seite~\pageref{kap15_ab6}}\\
\item[7. Abschnitt] Der Mensch hat nur wenige Tage zu leben; -- er könnte sie
besser anwenden. Diese Lehre kann keinem, dem es wirklich um sein Heil zu tun
ist, unangenem sein.
\dotfill \textit{Seite~\pageref{kap15_ab6}}\\
\item[8. Abschnitt] Das maßlose\footnote{\texttt{'üppige' ersetzt durch 'maßlose'}}
Leben verhindert nicht nur die Ausübung des
Guten, sondern befördert auch die Begehung des Bösen, es zerreißt das Band der
Ehe und Liebe, zerstört die Gesundheit, richtet das Vermögen zu Grund, u.f.m.
Schaubühnen und Spielhäuser sind die mächtigsten Beförderungtungsmittel dieser
Übel.
\dotfill \textit{Seite~\pageref{kap15_ab8}}\\
\item[9. Abschnitt] Wie sehr die Jugend dadurch zur Eitelkeit gereizt wird,
welchen Nachteile das Spiel und die rauschenden Vergnügungen bringen. -- Die
besseren Heiden verachteten ein solches Leben.
\dotfill \textit{Seite~\pageref{kap15_ab9}}\\
\item[10. Abschnitt] Die wahren Jünger Jesu überwinden das
eitle\footnote{\texttt{'sterben dem eitlen' ersetzt durch 'überwinden das eitle'}}
Wesen der Welt
durch Selbstüberwindung\footnote{\texttt{'Selbsverleugnung ab'  ersetzt durch
'Selbstüberwindung'}}. -- Vergnügen und Belohnung einer guten Anwendung der
Zeit.
\dotfill \textit{Seite~\pageref{kap15_ab10}}\\

\end{description}

\newpage

\section{1. Abschnitt} \label{kap15_ab1}

Die Ausschweifungen in Kleiderpracht und eitlen Vergnügungen, welche in der
heiligen Schrift vielfältig verboten\index{Gebote!Bibel!Verbote} sind, gaben auch
Veranlassung zu jener an
das Volk Israel\index{Personen:!Volk Israel} gerichteten traurigen Botschaft,
\footnote{
    Die ausschweifenden Moden, eitlen Gebräuche und Vergnügungenn des gegenwärtigen
    Zeitalters, sind,
    ebensowohl als jene der damaligen Zeit, dem Strafgericht Gottes ausgesetzt,
    welches über England und und Europa schwebt, und über ihre europäischen
    Bewohner hereinzubrechen droht.}
die der Prophet Jesaias in folgenden Worten erklärte:
\textit{
    "`Darum, spricht der Herr, dass die Töchter Zions\index{Personen:!Töchter
    Zions} stolz sind, und gehen mit anfgerichtetem Hals und mit geschminkten
    Gesichtern, (nach dem Englischen: mit lüsternen Blicken,) treten einher und
    schwänzen, (gehen geziert mit kurzen Schritten,) und haben köstliche Schuhe an
    ihren Füßen, (machen ein Geräusch mit ihren Füßen oder Schuhen,) darum wird der
    Herr den Scheitel der Töchter Zions kahl machen und ihre Geschmeide
    wegnehmen. Zu der Zeit wird der Herr den Schmuck an den köstlichen Schuhen
    wegnehmen, und die Hefte, die Spangen, die Kettlein, die Armspangen, die
    Hauben, die Flittereien, das Gebräme, die Schnürlein, die Bisamäpfel,
    \footnote{
        Mit wohlriechenden Sachen angefüllte und zur Zierrat
        behängte Kapseln.}
    die Ohrringe, die Ringe, die Haarbänder, die Feierkleider,
    die Mäntel, die Schleier, die Beutel, die Spiegel, die Koller,
    \footnote{Schürzen von feiner Leinwand}
    die Borten, die Kittel,
    \footnote{Feine dünne und leichte
        Kleider, mit weichen Staat gemacht wurde.} und es wird Gestank für guten
        Geruch sein, und ein loses Band für einen Gürtel, und eine Glatze für ein
        krauses Haar, und für einen weiten Mantel ein enger Sack. Solches alles
	statt deiner Schönheit. Deine Männer werden durchs Schwert fallen, und deine
	Krieger im Streit. Ihre Söhne werden trauern und klagen, und sie wird
	jämmerlich (verlassen) sitzen auf der Erde."'}
\footnote{Jesaja 3,16-23.}
\index{Bibelstellen:!Jesaja 3)}

\medskip

Seht da! Ihr eitlen und törichten Bewohner Englands\index{Orte:!England} und Europas\index{Orte:!Europa}, worin euere
Torheit besteht, und was für ein Los auch ihr zu erwarten habt! Doch lest
auch noch des Propheten Ezechiels Vision\footnote{\texttt{'Gesicht' ersetzt durch 'Vision'}} von der unglücklichen Stadt Tyrus\index{Orte:!Tyrus}, was
für Strafen ihr Stolz und ihre Maßlosigkeit\footnote{\texttt{'Üppigkeit' ersetzt durch 'Maßlosigkeit'}} ihr zuzogen. Unter vielen anderen
Umständen, die Ezechiel berührt, führt er die Dinge an, worin die Kaufleute mit
Tyrus ihren Handel trieben, nämlich mit allerlei kostbaren Sachen: In prächtigen
Seiden und gestickten Kleidern und Tüchern, in Purpur, in köstlichen Kästen von
Zedernholz, in feiner Leinwand, in Gewürzen\index{Gewürze}, Korallen, Smaragden, Agaten und
allerlei Edelsteinen, in Gold, Pferden, Wagen, u.s.w. Aber nun hört auch einen
Teil ihres Strafurteils!\index{Strafurteil} \textit{"`Ein Ostwind,"'} sagt der Prophet,
\textit{"`wird dich mitten
auf dem Meer zerbrechen, so dass alle deine Waaren, (dein Reichtum,) u.s.w.
und alles Volk in dir zur Zeit deines Untergangs mitten im Meere umkommen wird.
Und alle, die auf den Inseln wohnen, werden über dir erschrecken, und die
Kaufleute dich anpfeifen, dass du so plötzlich untergegangen bist und nicht mehr
herauskommen kannst."'}
\footnote{Ezechiel 27.}
\index{Bibelstellen:!Ezechiel 27)}
Auf diese Weise hat Gott sein Missfallen
an dem maßlosen\footnote{\texttt{'üppigen' ersetzt durch 'maßlosen'}} und zügellosen Leben der Welt durch Ezechiel deutlich zu erkennen
gegeben. Noch weiter geht aber der Prophet Zephanja, wenn er sagt:
\textit{"`Und am Tage
des Schlachtopfers des Herrn will ich  die Fürsten und des Königs
Kinder, und alle, die ein fremdes Kleid tragen, heimsuchen."'}
\footnote{Zefanja 1,8}
\index{Bibelstellen:!Zefanja 1)}
Solche
schlimme Folgen hatte es zu jener Zeit, wenn die Großen es sich erlaubten, die
eitlen Moden und Gebräuche fremder Völker nachzuahmen, oder bei ihrer Kleidung\index{Kleidung}
nicht auf den wahren Zweck derselben, sondern vielmehr auf die Befriedigung
ihrer Prachtliebe zu sehen.

\section{2. Abschnitt} \label{kap15_ab2}

Der Herr Jesus erteilte seinen Jüngern den ausdrücklichen Befehl, sich um
weltliche Dinge keine Sorgen zu machen, indem er ihnen zugleich deutlich zu
verstehen gab, dass diejenigen, die dieses täten, seine Jünger nicht sein
könnten, als er ihnen sagte:
\textit{"`Ihr sollt nicht sorgen und sagen: Was werden wir
essen? Was werden wir trinken? Womit werden wir uns kleiden? Nach allem solchen
trachten die Heiden. Denn euer himmlischer Vater weiß, dass ihr dieser alles
bedürft. Trachtet aber am ersten nach dem Reich Gottes und nach seiner
Gerechtigkeit, so wird euch alles gegeben werden."'}
\footnote{Matthäus 6,31-33.}
\index{Bibelstellen:!Matthäus 6)}
Es leuchtet von selbst ein, dass unser Herr hier unter Essen,
Trinken und Kleidern alle äußeren Dinge überhaupt versteht, da sie, als
sichtbare Dinge, den unsichtbaren und himmlischen, die dass Reich Gottes\index{Reich!Gottes} und
seine Gerechtigkeit betreffen, entgegengesetzt, und auch als diejenigen, um
welche die Jünger nicht besorgt sein sollen, doch an sich die unschuldigsten und
notwendigsten sind. Wenn nun aber die Gemüter der Nachfolger Jesu sogar in
solchen Fällen nicht besorgt sein sollen, wie viel weniger dürfen sie sich denn
um die törichten und eitlen Moden und Gebräuche der Welt kümmern, die keinen
anderen Zweck haben, als die sinnlichen Neigungen irdisch gesinnter Menschen zu
befriedigen. Dann folgt hieraus auch eben so klar, dass diejenigen, die in
solchen weltlichen und maßlosen\footnote{\texttt{'üppigen' ersetzt durch 'maßlosen'}}
Dingen leben, keine Nachfolger Jesu\index{Personen:!Nachfolger!Jesu}, sondern
\textit{"`den Heiden oder Völkern der Welt gleich sind, die Gott nicht
kennen."'}
\footnote{Lukas 12,22-30.}
\index{Bibelstellen:!Lukas 12)}
Und da also die wahren Junger Jesu und die
Anhänger an den vergnüglichen Dingen der Welt sich offenbar darin voneinander
unterscheiden, dass die erstern die himmlischen Dinge wahrnehmen, die das Reich
Gottes betreffen, dass in Gerechtigkeit, in Frieden und in Freude im heiligen
Geist besteht, und um äußere Gegenstände, selbst um die unschuldigsten und
notwendigsten, nicht ängstlich besorgt sind, und die anderen ihre vornehmste
Aufmerksamkeit auf Essen, Trinken und Kleider richten, und den Angelegenheiten,
den Freuden, Vergnügungen, Vorteilen und der Ehre der Welt nachstreben, so lasst
mich euch, ihr Bewohner Englands,\index{Orte:!England} um eures ewigen Heils\index{Heil} willen 
bitten, einmal
recht ernstlich über euch selbst nachzudenken. Erwägt, wie viel Sorgen, Kosten
und Zeit ihr auf törichte, ja oft lasterhafte Dinge verwendet, denn daran
könnt ihr erkennen, wie weit ihr von dem ursprünglich wahren Leben des
Christentums abgewichen seid. Betrachtet das Kaufen und Verkaufen, das Handeln
und Dingen \footnote{\texttt{Von 'sich verdingen', also 'betreiben' oder 'arbeiten'}}
, die viele Schreiberei, die Mühe und Arbeit, den Lärm, das Treiben,
die Eile und Verwirrung, das Nachsinnen, die Kunstgriffe, die
Übervorteilungen, die Vorbereitungen zum Essen und Trinken, zu Besuchen, zum
Putze, zu Vergnügungen und oft ganz lächerlichen Ergötzungen, kurz, \textbf{wie früh die
Menschen aufstehen müssen und wie spät sie sich erst wieder niederlegen können,
und wie viel kostbare Zeit sie anwenden, um alles das zu besorgen, weil die
Eitelkeit und Torheit der Welt es fordert.} Seht die Leute auf den Straßen, in den
Läden, an der Börse, im Schauspiel, auf den Wandelplätzen, in den
Kaffeehäusern u.s.w., und gesteht, ob ihr nicht das Gepränge der Unruhe und
Begierde des Geistes dieser vergänglichen Welt fast auf jedem Gesicht
ausgedrückt findet?
\label{ref:15_02_genug_fuer_alle}
\textbf{Sagt nicht: Wie sollen denn die Menschen sonst leben, und
wie anders könnte die Welt bestehen? Eine gewöhnliche, aber schwache Ausrede\footnote{\texttt{'Einwurf' ersetzt durch 'Ausrede'}}!
Denn es ist genug für alle da\index{es-ist-genug-für-alle-da}, wenn nur einige sich mit 
etwas weniger begnügen
wollten.} Für ein wahrhaft christliches Leben\index{Leben!christliches} sind wenige 
einfache und anständige
Dinge hinreichend, aber die Sinnlichkeit, der Stolz und der Geiz verleiten die
Menschen zu solchen Torheiten. Denn, wenn sie ihre Gemüter mehr mit dem Reich
Gottes beschäftigten, so würden sie wenig an vergängliche Vergnügungen denken,
und ihnen auch nur wenig Zeit widmen.

\section{3. Abschnitt} \label{kap15_ab3}

Diese Lehre von der Selbstüberwindung\footnote{\texttt{'Selbstverleugnung' ersetzt 
durch 'Selbstüberwindung'}} bestätigten und empfahlen die Apostel, wie
wir bereits bemerkt haben, sowohl durch ihr eigenes Beispiel, als auch durch
Vorschriften in ihren Briefen und Episteln, wie aus zwei sehr merkwürdigen
Stellen hervorgeht, wo Paulus und Petrus uns nicht allein melden, wie es in
dieser Hinsicht gehalten werden soll, sondern auch anzeigen, was überwunden\footnote{\texttt{'verleugnet"'ersetzt durch 'überwunden'}} und
vermieden werden müsse. \textit{"`So will ich nun,"'} sagt Paulus,\index{Kleidung!von Frauen}
\textit{"`dass die Frauen in
zierlicher (sauberer) Kleidung sich schmücken,"'} -- wie ist das zu verstehen? --
\textit{"`mit Schamhaftigkeit und Zucht, nicht mit Zöpfen, (aufgeputzten Haaren,) oder
Gold, Perlen oder köstlichen Gewande;"'} -- Dergleichen war also nicht
erlaubt! -- \textit{"`sondern, wie es Frauen, die Gottseligkeit vorgeben, geziemt: mit
guten Werken."'}
\footnote{1. Timotheus 2,9+10.}
\index{Bibelstellen:!1. Timotheus 2)}
Hieraus folgt unwiderruflich\footnote{\texttt{'unwidersprechlich' ersetzt durch 'unwiderruflich'}}, dass
diejenigen Frauen, die sich mit Zöpfen\index{Zöpfe} \index{Schmuck}oder Haarflechten, Perlen, Silber, Gold,
und Köstlichkeiten oder prächtigen Kleidern schücken, so lange sie das tun,
keine Anspruch auf Gottseligkeit\index{Gottseligkeit} machen können, indem der Apostel zu erkennen
gibt, dass solcher Schmuck\footnote{\texttt{'Putz' ersetzt durch 'Schmuck'}} mit 
der christlichen Erhabenheit und Tugend nicht
übereinstimmt, und folglich etwas Böses ist, das den Frauen, welche sich zur
Gottseligkeit bekennen, nicht geziemt. -- Der Apostel Petrus gibt eine
Vorschrift gleichen Inhalts in folgenden Worten: \textit{"`Deren"'} (nämlich der Frauen)
\textit{"`Schmuck soll nicht äußerlich sein, mit Haarflechten und Goldumhängen oder
Kleideranlegen;"'} -- Wie denn dann? --
\textit{"`sondern der verborgene Mensch des Herzens,
mit unverrücktem, (unverderblichen) sanften und stillem Geist, der vor Gott
köstlich ist."'} Und um sie dazu zu bewegen, fügt er noch hinzu:
\textit{"`Denn so haben
sich auch vor Zeiten die heiligen Frauen\footnote{Es ist bemerkenswert, dass der
Apostel hier nur von den Frauen redet, als ob eitler Putz den
Männern nicht ebenso eigen wäre. Möchten sie dieses doch beherzigen!} geschmückt,
die ihr Vertrauen auf Gott setzen."'}
\footnote{1. Petrus 3,3-5}
\index{Bibelstellen:!1. Petrus 3)}
Dieses beweist nicht allein, dass vor Zeiten die heiligen Frauen\index{Personen:!Frauen!heilige} so 
geschmückt waren,
und dass es allen, welche ein heiliges Leben führen und ihr Vertrauen auf den
heiligen Gott setzen wollen, gezieme, sich so zu schmücken, sondern es gibt uns
auch deutlich zu erkennen, dass zu allen Zeiten diejenigen, welche sich
unerlaubter Zierraten bedienten, aller ihre frommen Worte ungeachtet, keine von
denen waren, die wirklich heilig lebten und ihr Vertrauen auf Gott setzten.
Solche sind in der Tat von einem wahren Vertrauen auf Gott weit entfernt,
weshalb auch der Apostel Paulus ausdrücklich erklärt,
\textit{"`dass diejenige, welche
in der Maßlosigkeit\footnote{\texttt{'Üppigkeit' erstzt durch 'Maßlosigkeit'}} lebe,"'} in 
Ansehung des Lebens aus Gott,
\textit{"`lebendig tot sei."'}
\footnote{1. Timotheus 5,6.}
\index{Bibelstellen:!1. Timotheus 5)}
Derselbe Apostel ermahnt ferner die Christen,
\textit{"`dass sie ihren Wandel im Himmel haben, und nach dem trachten sollen, was oben
ist. Lasst uns ehrbar wandeln, als am Tage!"'} sagt er
\textit{"`Nicht in Fressen und
Saufen, nicht in Kümmern und Unzucht, nicht in Hader und Neid. Hurerei und
allerlei Unreinheit oder Geiz lasset nicht von euch gesagt werden, auch keine
schandbaren Worte und Narrenteidinge (Possen) oder Scherze, sondern vielmehr
Danksagung. Lasst kein faules Geschwätz aus euerem Mund gehen, sondern was zur
nötigen Besserung dient, und holdselig zu hören ist (oder den Hörenden
Erbauung gewährt). Zieht den Herrn Jesum Christum an, und pflegt den Körper\footnote{\texttt{'wartet des Leibes' 
ersetzt durch 'pflegt den Körper'}} 
nicht so, dass er geil werde. Und betrügt nicht den heiligen Geist Gottes,"'} --
welches durch ein maßloses\footnote{\texttt{'üppiges' ersetzt durch 'maßloses'}} Leben geschieht; --
\textit{"`sondern seid Gottes Nachfolger,
als geliebte Kinder. Und wandelt vorsichtig, nicht wie Unweise, sondern wie
Weise, und schicket euch in die Zeit, (erkauft die gelegene Zeit) denn es sind
böse Tage."'}
\footnote{Philipper 3,20)  Kolosser 3,2)  Römer 13,13+14) Epheser 4,29+30) Epheser 5,1+3+4+15+16}
\index{Bibelstellen:!Philipper 3)}
\index{Bibelstellen:!Kolosser 3)}
\index{Bibelstellen:!Römer 13)}
\index{Bibelstellen:!Epheser 4)}
\index{Bibelstellen:!Epheser 5)}

\section{4. Abschnitt} \label{kap15_ab4}

Hiernach könnt ihr euch nun prüfen, ihr Bewohner diesen Landes! Die ihr glaubt,
man tue euch Unrecht, wenn man euch nicht für wahre Christen hält. Untersucht
doch nur, in wiefern euer Leben und euere Gesinnungen mit diesen heiligen
Vorschriften und Beispielen der Selbstüberwindung\footnote{\texttt{'Selbstverleugnung' 
ersetzt durch 'Selbstüberwindung'}} übereinstimmen. O! Meine
Freunde! Meine Seele trauert über euch! \label{15_04_penn_alte_freunde} Ich lebte einst selbst mit euch und in
eurer Mitte. Euer Leben und die Dinge, womit ihr eure Zeit verbringt, sind
meiner Bemerkung nicht entgangen, und mit tiefem Bedauern, ja mit
unaussprechlichem Mitleid, beklage ich eure Torheit. O! Dass ihr weise würdet!
O! Dass ihr der
\label{ref:15_04_innere_stimme}
\textbf{innern Stimme eures Richters} Gehör geben, und der Ewigkeit Zeit
vergönnen wolltet, ein wenig mit euch zu richten! Warum müssen euere Kleider,
eure Putztische, Besuche, Schauspiele, Gastmähler und so viele andere armselige
vorrübergehende Freuden, euere ganzen Seelen besitzen, und sowohl über euere Zeit,
Sorgfalt und Aufmerksamkeit, als auch selbst über euer Vermögen gebieten? Lasset
euch, -- ich bitte euch im Namen den lebendigen Gottes! -- lasset euch den Rat
und die Ermahnung eines Mannes gefallen, der, wie einige unter euch wissen, an
allen solchen Dingen Teil genommen und daher Zeit und Gelegenheit gehabt hat,
aus eigener Erfahrung einzusehen, wie wenig dergleichen Eitelkeiten zu unserer
wahren und dauernden Zufriedenheit\index{Zufriedenheit} beitragen können. Nein! Meine Freunde! Gott,
der Allmächtige\index{Gott!Allmächtiger}, weiß es, -- und o! Möchtet ihr mir glauben, und meinen Rat
annehmen! -- Sie enden in Schande und Kummer.\textbf{Er, der Allerheiligste, der getreu
und wahrhaftig ist, hat es einmal so beschlossen, dass jeder Mensch, er sei Mann
oder Frau, das ernten soll, was er gesät hat, und wird es daher nicht eine
traurige und schreckliche Ernte sein, wenn ihr für alle eure Zeit und Ausgaben,
die ihr bloß aus Überflüssigkeiten und eitlen Vergnügungen so übel verwendet
habt, nur Unruhe, Angst und getäuschte Hoffnung ernten werdet? -- O! Darum
zieht euch in euch selbst zurück! Dämpfet nicht den heiligen Geist Gottes in
euerem Inneren! Bestrebt  euch, euere so sehr vertrödelte\footnote{\texttt{'gemissbrauchte' ersetzt 
durch 'vertrödelte'}} Zeit wieder zu
erkaufen, und sucht einen Umgang, der euch in der Überwindung eurer verderbten
Neigungen behilflich sein kann} Dann werdet ihr zu der Fähigkeit gelangen, dass
ihr die Gebote Jesu Christi halten und sowohl seinem als auch seiner Jünger
Beispiel folgen könnt. Denn aus dem, was bisher erklärt worden ist, geht
deutlich hervor, dass eine solche Lebensweise, als unter euch in diesem Land
geführt wird, nie wahrhaft christlich war, noch jemals sein wird.

\section{5. Abschnitt} \label{kap15_ab5}

\label{ref:15_05_freizeitbeschaeftigung}
\textbf{Die beste Erhohlung der Christen ist: Gutes tun. Alle ihre Gebräuche bezielen
Mäßigkeit und irgend einen wohltätigen Zweck, der sich mehr oder weniger mit
jeder Handlung verbinden lässt.}
\footnote{1. Petrus 1,15.) Hebräer 10,25) 1. Petrus 4,9-11)
Matthäus 25,36+37) Philipper 2,4) Epheser 4,8.) 1. Mose 5,24) Psalm 1,2) Psalm 119,15) Psalm 143,5}
\index{Bibelstellen:!1. Petrus 1)}
\index{Bibelstellen:!Hebräer 10)}
\index{Bibelstellen:!1. Petrus 4)}
\index{Bibelstellen:!Matthäus 2)}
\index{Bibelstellen:!Philipper 2)}
\index{Bibelstellen:!Epheser 4)}
\index{Bibelstellen:!1. Mose 5)}
\index{Bibelstellen:!Psalm 1)}
\index{Bibelstellen:!Psalm 119)}
\index{Bibelstellen:!Psalm 143)}
\index{Beschäftigung!christlicher}\textbf{Z. B. wenn Männer und Frauen ihre Berufsgeschäfte fleißig 
wahrnehmen,
religiösen Versammlungen beiwohnen, ihre gutgesinnten Nachbarn besuchen\index{Besuche}, um sich
mit ihnen zu erbauen, und die bösartigen, um sie zu bessern. Wenn sie Sorgfalt
auf die Erziehung\index{Erziehung} ihrer Kinder\index{Personen:!Kinder} verwenden, ihren 
Dienstboten\index{Personen:!Dienstboten} mit guten Beispiel
vorangehen, die Armen unterstützen, Kranke pflegen\index{Krankenpflege}; Gefangene 
besuchen\index{Orte:!Gefängnis!Besuche}, zur
Erleichterung ihrer unglücklichen Lage beitragen, und Ruhe und Frieden unter den
Nachbarn zu befördern streben. Auch ist das mäßige Studium\index{Erkenntnis!Studium} nützlicher und
empfehlenswerther Künste\index{Kunst} und Wissenschaften\index{Erkenntnis!Wissenschaft}, z. B. 
der Arithmetik\index{Erkenntnis!Arithmetik}, Geometrie\index{Erkenntnis!Geometrie},
Mechanik\index{Erkenntnis!Mechanik}, der Schifffahrtskunde\index{Erkenntnis!Schifffahrtskunde}, 
der Landwirtschaft\index{Erkenntnis!Landwirtschaft}, des Gartenbaus\index{Erkenntnis!Gartenbau}, der
Arzneiwissenschaft\index{Erkenntnis!Medizin}, Naturkunde und dergleichn ein angenehmes Vergnügen für 
das männliche Geschlecht, so wie für das weibliche 
Spinnen, Nähen, Stricken,
Weberei, Gartenkunst, Einmachen der Früchte und andere nützliche Hausarbeiten\index{Hausarbeiten},
sehr angenehme Beschäftigungen sind, welche sogar die vornehmsten und edelsten
Matronen und jungen Personen unter den Heiden trieben, die sich ein Vergnügen
daraus machen, anderen, die aus Armut keine Dienstboten halten können, in ihren
häuslichen Beschäftigungen zu helfen, um ihnen ihre notwendigen Arbeiten zu
erleichtern. Das größte und wichtigste Vergnügen wahrer Christen besteht aber
darin, dass sie sich oft von allen weltlichen Gegenständen zurückziehen, um, in
geheimen und stillen Betrachtungen\index{Stille!Andacht/Betrachtung} über das 
göttliche Leben und himmlische Erbe\index{Erbe!himmlisches},
daß Nahesein des Herrn zu geniessen.} -- Und dieses zu versäumen, um, unter dem
Vorwande der Erholung, anderen Vergnügungen nachzugehen, ist in der Tat höchst
strafbare Lüsternheit und verwerfliche Ungottseligkeit.

\medskip

Es ist eine sehr gewöhnliche, aber ebenso nichtige Einwendung, wenn einige
sagen, dass sie sich doch nicht beständig mit ernsthaften Gegenständen
beschäftigen könnten, und es ihnen daher nicht zu verargen sei, wenn sie sich
die allgemein üblichen Zerstreuungen erlauben. Denn ich frage, was sie denn
eigentlich begehren? Was sie tun und haben wollen? Wer ein ordentliches
Geschäft oder Gewerbe treibt, kann kaum Zeit finden, die Hälfte von dem, was
oben empfohlen ist, zu tun. Und was diejenigen betrifft, die nichts zu tun
haben, -- und daher auch gewöhnlich nichts tun, als, was das schlimmste von
allem ist, nämlich sündigen; -- \textit{so habe ich ihnen ja eine Menge angenehmer,
nützlicher und selbst ehrvoller Beschäftigungen und Vergnügungen
vorgeschlagen, unter welchen sie die eine oder die andere wählen könnten. Allein das
sind keine Erholungen für sie. Die Dinge, an denen sie das größte Behagen
finden, sind Schauspiele, Bälle, Karten, Würfel u.s.w., oder Trinken,
Schwärmen\index{Schwärmen}, Schwelgen und dergleichen, und zwar vom Morgen bis in die Nacht, ja,
sie verwandeln die Nächte in Tage und verkehren die Ordnung der Natur, um ihre
verderbten Neigungen zu befriedigen.
\footnote{Amos 6,3-8.}
\index{Bibelstellen:!Amos 6)}
Und müssten sie nicht
essen und schlafen, so würden sie ohne Zweifel nie Zeit finden, von ihren eitlen
Zeitvertreiben aufzuhören, bis der plötzliche Ruf des Todes sie aufforderte, in
einer andern Welt zu erscheinen. Aber sonderbar ist es, dass diese Leute, die von
ihrem Spieltisch nicht aufstehen können, es unerträglich finden und kaum für
möglich halten, dass jemand lange unter einer nützlichen Gemütsübung oder in
einer religiösen Versammlung aushalten könne.}

\section{6. Abschnitt} \label{kap15_ab6}

\label{ref:15_06_langeweile}
\textbf{Aber wie glauben sie denn die unendliche Ewigkeit zubringen zu wollen? Denn
"`wie der Baum fällt, so wird er liegen."'}
\footnote{Kohelet 11,3. }
\index{Bibelstellen:!Kohelet(Prediger) 11)}
O! Dass die Menschen sich doch nicht selbst
betrügen und ihre unsterblichen Seelen mit dem angenehmen, aber falschen und
verderblichen Traum täuschen wollten, dass eine gewaltsame und unwiderstehliche
Macht sie in dem Augenblick, wo Leib und Seele geschieden werden, verändern und
bekehren werde! Nein! Meine Freunde!
\textit{"`Was ihr sät, das werdet ihr auch ernten."'}
\footnote{Galater 6,4-9. Epheser 5,6.}
\index{Bibelstellen:!Galater 6)}
\index{Bibelstellen:!Epheser 5)}
Habt ihr Eitelkeit, Torheit,
sichtbare Freuden, vergänglichen Genuss ausgesät, so werdet ihr nichts besseres
als Verderben, Kummer und peinliche Angst ewiger Verzweiflung ernten können. --
Aber ach, was ist die Ursache der so allgemeinen Ausrede der Menschen, dass man
doch nicht immer über geistliche Dinge grübeln müsse? Gewiss, sie ist keine
andere, als weil sie die Freude und den Frieden nicht kennen, welche die Seele
desjenigen genießt, der beständig als in der Gegenwart Gottes\index{Gott!Gegenwart} redet und handelt.
Dieses Gefühl des göttlichen Friedens übersteigt die eitlen Begriffe solcher
Menschen, deren Verstand von der Herrlichkeit und den Freuden des Gottes dieser
Welt verblendet und verfinstert ist.
\footnote{Epheser 4,18-20.}
\index{Bibelstellen:!Epheser 4)}
Daher bestehen
auch ihre Religionsübungen bloß in einer leeren kraftlosen Wiederhohlung
auswendig gelernter Worte. Denn wenn sie einen Schatz im Himmel hätten, so würde
auch ihr Herz daselbst sein und sich mit dem Gegenstand ihrer größten Freude
und Anhänglichkeit beständig beschäftigen. Und ich nehme keinen Anstand, gerade
heraus zu erklären, dass diejenigen Menschen, denen dieses eine Last ist, und
welche daher Erhohlung in Schauspielen\index{Schauspiel}, Bällen, Spielpartien und anderen
Zeitvertreiben suchen, weder jemals die Freundlichkeit Gottes und seiner
heiligen Wahrheit geschmeckt haben, noch die Fähigkeit besitzen, dieselbe in
einer künftigen Welt zu genießen. \textbf{Denn, wie wäre es möglich, dass sie während
einer ganzen Ewigkeit wahren Genuss und Befriedigung für ihren Geist in einem
Gegenstand finden sollten, der ihnen in dem kurzen Zeitraum von dreißig oder
vierzig Jahren schon so langweilig und unerträglich geworden ist, dass sie, um
sich zu erhohlen, ihre Zuflucht zu den geringfügigen Tändeleien und
Zeitvertreiben dieser vergänglichen Welt nehmen müssen.} Wahrlich! Diejenigen,
welche glauben,
\textit{"`dass sie einst von jedem unnützen Wort Rechenschaft geben
müssen,"'}
\footnote{Matthäus 12,26.}
\index{Bibelstellen:!Matthäus 12)}
dürfen keine Belustigungen\index{Belustigung} aufsuchen, um
mutwillig eine Zeit zu \textbf{\textit{vertreiben}}, die sie mit allem Fleiß zu erkaufen
angewiesen sind. O wenn die Menschen bedächten, dass sie nichts Geringeres zu
tun haben, als
\textit{"`ihren Beruf und ihre Erwählung fest zu machen;"'}
\footnote{2. Petrus 1,20}
\index{Bibelstellen:!2. Petrus 1)}
so würden sie weniger bemüht sein, immer neue Erhohlungen für ihre
leeren Seelen zu erfinden, und ihnen Tage, Monate und Jahre zu widmen, während
sie nicht den vierten Teil ihrer kostbaren Zeit auf die wichtigste
Angelegenheit ihres Lebens, auf die Errettung ihrer unsterblichen Seelen,
verwenden, wozu dieselbe ihnen doch gegeben wurde.

\section{7. Abschnitt} \label{kap15_ab7}

Es ist gewiss nicht nötig, durch eitle Ergötzlichkeiten eine Zeit zu vertreiben,
die an sich schon so schnell verstreicht, und welche, wenn sie einmal verflossen\footnote{\texttt{'entflohen' ersetzt 
durch 'verflossen'}} 
ist, niemand zurückrufen kann. Schauspiele\index{Schauspiel}, 
Bälle\index{Bälle}, Konzerte\index{Konzerte}, Lustpartien, Romane\index{Romane}
und dergleichen nutzlose Unterhaltungen, womit so viele, die ihnen ergeben sind,
ihre edle Zeit verbringen, werden am Tage der Offenbarung des gerechten Gerichts
Gottes\index{Gericht!Jüngstes} ihnen nur zu ihrer Verdammnis\index{Verdammnis} gereichen. 
Diese Dinge, o meine Freunde,
sind Erfindungen eines Geistes, der gleich im Anfang den Geschmack an dem
entzückenden Genusse der heiligen Gegenwart Gottes verlor. Und wollten diesem
zufolge die Menschen nur die große Zufriedenheit und sichere Belohnung, welche
ein allgemein \index{Wohltätigkeit} wohltätiges und tugendhaftes Leben sowohl in dieser Welt
begleiten, als auch in der künftigen erwarten, in gehörige Erwägung ziehen, und
sich dadurch von ihren eitlen, fruchtlosen Erhohlungen und Zeitvertreiben
entwöhnen lassen, so würden sie finden, dass jene bereits erwähnten,
vortrefflichen und den Neigungen vernünftiger Wesen würdige Beschäftigungen
mehr als hinreichend wären, ihre Nebenstunden auf eine Art auszufüllen, die
nicht nur ihnen selbst, sondern auch anderen edles Vergnügen, angenehme
Unterhaltung und wahren Nutzen gewähren würde. Auch kann das hier Gesagte nur
solchen Menschen missfallen, die nicht wissen, was es heißt: mit Gott wandeln,
sich auf die ewige Wohnung\index{Wohnung!ewige} vorbereiten, das Gemüt mit wahrhaft guten,
himmlischen Gegenständen beschäftigen und den Beispielen der Gläubigen einer
glücklichen Vorzeit nachstreben. Es kann nur solchen zuwider sein, welche von
der Lehre, von dem Leben, dem Tod und der Auferstehung Christi\index{Auferstehung!Christi} keine 
wahre
Erkenntnis haben, und deren Gemüt nur auf sinnliche Gegenstände gerichtet
sind, von denen sie hingerissen, getäuscht und ins Verderben\index{Verderben} gezogen werden. Ja,
es kann endlich nur denen verächtlich und anstößig vorkommen, die den Himmel
verachten und die unsichtbaren, aber ewig dauernden Freuden des Geistes, dem
Genusse einiger armseligen, vorübergehenden Ergötzungen der Sinne aufopfern.
Lässt sich von solechen wohl sagen,
\textit{"`dass sie in Jesum Christum, in seinen
schmerzhaften Leiden, in seinen schmachvollen Tod getauft und mit ihm zu einem
Verlangen nach unsterblichen Genüssen, zu himmlischen Betrachtungen, zu einem
neuen göttlichen Leben erweckt und auferstanden sind? Dass sie in der
Erkenntnis der göttlichen Geheimnisse und in der Heiligkeit wachsen, die sie
zu der völligen Größe Christi herankommen, und ihm ähnlich werden, der das große
Muster Aaler ist?"'}
\footnote{Römer 6,3-8.) 1. Korinther 12,13) Galater 3,27) Kolosser 2,12+13)
Epheser 4,13.}
\index{Bibelstellen:!Römer 6)}
\index{Bibelstellen:!1. Korinther 12)}
\index{Bibelstellen:!Galater 3)}
\index{Bibelstellen:!Kolosser 2)}
\index{Bibelstellen:!Epheser 4)}
Ich sage, ob diese notwendigen Eigenschaften eines Christen bei
ihnen anzutreffen sind, und was sie davon erfahren haben, möge, nach ernster
Untersuchung und kalter Überlegung, ihr eigenes Gewissen ihnen beantworten.

\section{8. Abschnitt} \label{kap15_ab8}

Die große Liebe, mit welcher die Menschen an eitlen Moden und Zeitvertreiben
hängen, beweit aber nicht allein ihren Weltsinn und ihren Mangel an Kenntnis
göttlicher Freuden, ihre Nachahmung jener schädlichen Erfindungen und ihre
häufigen Besuche der Belustigungsörter hindert sie auch an vielem Guten, das
sie tun könnten, und öffnet ihnen die Tür zu vielem Bösen, wozu sie sich
verleiten lassen. Erstlich verlieren sie eine kostbare Zeit, die aus dem
Sterbebett eine ganze Welt aufwiegen würde, dann verschwenden sie das Geld,
welches zu besseren und gemeinnützigen Zwecken hätte verwendet werden können.
Sie lernen Vergnügen an Dingen zu finden, die ihnen zur Schande gereichen, sorgen
nur für die Befriedigung ihrer verderbten Neigungen, und verhärten ihre Gemüter
gegen die Eindrücke himmlischer Dinge, indem sie dieselben beständig mit
Gegenständen der Torheit beschäftigen. Vom Stolz und von der Prachtliebe
geblendet, legen sie einen so hohen Wert auf die Kleidung, -- die doch nur zur
anständigen Bedeckung der Blöße dienen sollte --, \textbf{dass sie oft das edelste
Geschöpf Gottes vernachlässigen und verachten, indem sie den Menschen bloß nach
der Pracht und dem modigen Zuschnitte seiner Kleider beurteilen und schätzen,
woraus dann ganz natürlich das Ansehen der Person entspringt. Dieses ist so
klar, dass jemand, der es leugnen wollte, auch eben so leicht die Abwesenheit der
Sonne am hellen Mittag behaupten könnte.} Denn was ist gewöhnlicher und
weltkundiger, als dass die Menschen mit ihren Komplimenten, Verbeugungen und
Anreden sich nach dem mehr oder minder prächtigen Anzug der Personen richten,
mit denen sie zusammentreffen, wiewohl dieses in den Augen Gottes höchst
misssfällig und in der heiligen Schrift so ausdrücklich verboten ist, dass
diejenigen, die es tun, als Übertreter des göttlichen Gesetzes betrachtet und
folglich auch als solche bestraft werden.
\footnote{Jakobus 2,1-10.}
\index{Bibelstellen:!Jakobus 2)}
Und was für nachteilige
Folgen haben nicht ferner noch diese Ausschweifungen in allerlei Zerstreuungen
und Belustigungen, sowohl hinsichtlich des Vermögens als auch der Moralität der
Menschen? Wie oft werden nicht bei solchen Gelegenheiten die Geschäfte
vernachlässigt, junge Mädchen verführt, die Bande der \index{Ehe!-bruch}Ehe verletzt? Was für Zank
und Misshelligkeiten richten sie nicht in Familien an? Und wie oft veranlassen
sie nicht Trennungen unter Eheleuten, Enterbungen, Dienstentlassungen u.s.w.?
Wie häufig werden nicht deshalb die Dienstboten\index{Personen:!Dienstboten} wie 
Sklaven\index{Personen:!Sklaven} behandelt, und die
Kinder\index{Personen:!Kinder} ganz vernachlässigt? Auch ist nicht selten die Ausschweifungen und
unmäßige Lebensart der Männer\index{Personen:!Männer} Schuld,
dass sie ihre \index{Personen:!Frauen!Misshandlung}Frauen gering achten und
schändlich misshandeln, welche dann oft deshalb zu gleichen Ausschweifungen
schreiten, oder eine so ungerechte Behandlung sich so sehr zu Herzen nehmen, dass
sie ihre Tage in Kummer und Elend verbringen.
\label{ref:15_08_schauspiel}
\index{Schauspiel}\textbf{Unter allen jenen unglücklichen
Erfindungen der Maßlosigkeit
\footnote{\texttt{'Üppigkeit' erstzt durch 'Maßlosigkeit'}} 
sind jedoch die Schauspiele, diese Schulen der
Verführung, die verderblichsten, da sie das Laster reitzend darstellen, und fast
immer, wo nicht von offenbar schmutzigen und schändlichen, jedoch von
zweideutigen, schlüpfrigen und läppischen Dingen handeln, die, wie allgemein
bekannt ist, einen sehr schädlichen Einflus auf die Zuschauer, und vornemlich
auf die so leicht empfänglichen Gemüter der Jugend haben.} Allein, so klar es
auch am Tag liegt, dass kaum feinere und hinreißendere Verführungsmittel, als
die Schauspiele sind, für den Leichtsinn der Menschen hätten erdacht werden
können; -- wie sich aus dem Folgenden auch noch deutlicher entwickeln wird; --
so werden, dessen ungeachtet, die der Tugend so gefährlichen Theater noch immer
unterstützt und unterhalten. Und gewiss, nur das ausserordendliche Vergnügen,
welches die Menschen darin finden, kann ihre Augen so sehr verblenden, dass sie
das Schädliche derselben nicht einsehen.

\section{9. Abschnitt} \label{kap15_ab9}

Auch erzeugen diese Zerstreuungsmittel eine gänzliche Abneigung gegen ernste und
gründliche Betrachtungen religiöser Gegenstände, indem sie die Gemüter der
Menschen mit Rückerinnerungen an die verschiedenen abenteuerlichen
Begebenheiten, mit denen sie unterhalten wurden, unaufhörlich erfüllen, und
dadurch, besonders bei jungen Leuten\index{Personen:!jungen Leuten}, die Einbildungskraft 
erhitzen und die
Leidenschaften\index{Leidenschaft} aufregen. Die anderen gewöhnlichen Belustigungen, nämlich die
Bälle, Gasterein, Spiele, und dergleichen haben sooft ähnliche nachteilige Folgen, wie
die vielen Zänkereien und Feindseligkeiten, Zweikämpfe, Zeit- und
Vermögensverschwendungen, u.s.w., die sie nicht selten veranlassen, leider
überflüssig beweisen. Kurz, es sind Gebräuche, die unter den Heiden\index{Personen:!Heiden}, die Gott
nicht kannten, nie aber unter den Gläubigen, die ihn fürchteten, üblich waren.
Selbst die Bessern unter den Heiden\index{Personen:!Heiden}, z. B. Anaragoras\index{Personen:
!Anaragoras}, Sokrates\index{Personen:!Sokrates}, Plato\index{Personen:!Plato},
Antisthenes\index{Personen:!Antisthenes}, Heraklit\index{Personen:!Heraklit}, 
Zenon\index{Personen:!Zenon}, Aristides\index{Personen:!Aristides}, Cato\index{Personen:!Cato}, 
Cicero\index{Personen:!Cicero}, Epikur\index{Personen:!Epikur}, Seneca\index{Personen:!Seneca}, und
andere, haben in ihren Schriften ihren Abscheu vor solchen Dingen an den Tag
gelegt, die ihnen hassenswürdig waren, und sowohl gegen die Ehre des
unsterblichen Gottes, als auch gegen alle gute Ordnung in den Regierungen\index{Regierung} zu
streiten schienen, weil sie zur Ungebundendeit, zum Müssiggange, zur Unwissenheit
und Weichlichkeit, den giftigen und ansteckenden Seuchen der Staaten\index{Staat} und 
Reiche\index{Reich},
führen. Aber so groß ist die Anmassung der leichtsinnigen Menschen unseren
Zeitalters, dass sie sich beinahe schon für Heilige halten, wenn sie sich von
solchen groben Lastern und Verbrechen frei wissen, die das Gesetz mit
Gefängnisstrafen belegt. Ihre Gemüter sind von der vermeinten Unschuld ihrer
gewöhnlichen Ausschweifungen so eingenommen, dass sie sich ihnen ganz hingeben
und um bessere Dinge sich wenig bekümmern. Dies macht sie so kühn in ihrer
Verteidigung derselben, und so entschlossen, keinen Gedanken dagegen
einzulassen. Und was hat dieses für einen Grund? -- Die Freiheit, in der sie
leben, behagt ihrer Sinnlichkeit, sie befriedigt das lüsternde Auge und den
verwöhnten Gaumen der verderbten Natur. Daher halten sie es schon für etwas
lobenswertes, wenn jemand in seinen Genüssen sich auf das Beispiel der Tiere
beschränkt, die nur das zu sich nehmen, was die Natur fordert, wiewohl selbst
die Anzahl derer, die so denken, nur gering ist. So sehr haben viele in unseren
Tagen sich der Unmässigkeit ergeben, dass sie keine andere Richtschnur für ihre
Handlungen, als ihren eigenen Willen anerkennen, und wenn sie es hoch dringen,
einen Rubin darin suchen, dass sie sich keiner der niedrigsten Laster schuldig
gemacht haben. Dieses kann man ihnen auch in der Tat zu einer Zeit nicht
absprechen, wo keine Handlung so abscheulich sein kann, dass nicht einige sie
dennoch für erlaubt hielten. Indessen ist es gewiss doch immer ein Beweis
allgemein herrschender Ruchlosigkeit unter den Einwohnern eines Landes, wenn
jemand bei ihnen schon für tugendhaft, ja, für einen Christen gilt, und sogar im
Ruf nicht geringer Frömmigkeit steht, sobald man ihm nur keine solche Laster
vorwerfen kann, als selbst die Heiden\index{Personen:!Heiden} verabscheuen. Welch' ein trauriges 
Zeichen
der Verderbtheit eines Landes! Aber wie groß muss 
nicht die Verblendnng sein,
wenn Leute, die sich für gläubige Christen halten, dieselben Gebräuche, welche
von denen, die sie Ungläubige nennen, als schändliche Dinge verworfen werden,
nicht für das erkennen können oder wollen, was sie wirklich sind, sondern ihnen
die schönen Namen Anstand, Artigkeit, guter Ton, Erhohlun und dergleichen beilegen.
Aber, meine Freunde, gesetzt auch, es gäbe keinen Gott und weder Himmel noch
Hölle, keine Beispiele heiliger Menschen, keinen Herrn Jesum Cuhristum, dessen
Kreuz wir uns unterwerfen\index{unterwerfen}, und nach dessen Lehre und Leben wir uns richten
müssten, würde es nicht dennoch immer eine edlere Beschäftigung und weit
würdigere Verwendung unserer Zeit und unseres Geldes sein, wenn wir
Mildtätigkeit\index{Mildtätigkeit} gegen Arme ausübten, Dürftige und Verlegene unterstüzten,
Frieden und Eintracht unter Nachbarn zu befördern strebten, Kranke besuchten,
der Witwen und Waisen uns annähmen, und andere schon gedachte gute Handlungen
verrichteten, statt eitlen Belustigungen und Zerstreuungen nachzugehen? Es lässt
sich in der Tat auch nicht denken, dass auf dem Weg zur Seligkeit eine solche
Verschiedenheit sinnlicher Genüsse sollte anzutreffen sein, als die meisten
Menschen sich erlauben, denn sonst müsste wirklich ein überzeugtes Gewissen, ein
geängstigter Geist, ein zerschlagenes Herz, eine wiedergeborene Seele, mit
einem Worte, die Unsterblichkeit selbst eine bloße Erdichtung sein, wie einige
behaupten wollen, und andere daher auch geneigt sind zu glauben.
\footnote{Weisheit Salomos 18,14)
Psalm 52,17)
Matthäus 5,4)
Lukas 6,25)
Römer 2,7)
Psalm 40,8)
Römer 7,22)
Hebräer 11,13-16)
Römer 1,25-30)}
\index{Bibelstellen:!Weisheit Salomos 18)}
\index{Bibelstellen:!Psalm 52)}
\index{Bibelstellen:!Matthäus 5)}
\index{Bibelstellen:!Lukas 6,)}
\index{Bibelstellen:!Römer 2)}
\index{Bibelstellen:!Psalm 40)}
\index{Bibelstellen:!Römer 7)}
\index{Bibelstellen:!Hebräer 11)}
\index{Bibelstellen:!Römer 1)}
Nein! Solche weltliche Freuden finden im Reich Gottes\index{Reich!Gottes}
nicht statt, und müssen daher aus dem Gebiet des Christentums auf immer
gänzlich verwiesen werden. Denn ich behaupte, dass sie für jeden, der Gott in
seinem Inneren kennt, und ein Gefühl von seiner beseligenden Gegenwart hat, ein
wahrer Tod sind. Ja, sie sind auch gefährlicher und mehr geeignet, das Gemüt
von den Beschäftigungen mit göttlichen Gegenständen abzuziehen, als die gröberen
Ausschweifungen, deren Abscheulichkeit leichter in die Augen fällt, und welche
daher auch den vielen schon durch Hilfe ihrer Erziehung und gewohnten Mäßigkeit,
oder aus natürlichem Widerwillen verabscheut werden, so wie sie auch, wenn sie
begangen sind, allezeit einen ihnen angemessenen hohen Grad der überzeugenden
Unruhe und Bestrafung mit sich führen. Die vorgeblich unschuldigen Vergnügungen,
diese vermeintlichen unschädlichen Erheiterungen hingegen, schleichen sich
unmerklich in das Gemüt ein, und können eben deshalb desto zerstörender um sich
greifen. Denn da sie die Sinne sanft ansprechen, so gewinnen sie auch leicht
Eingang, und jemehr sie den Anschein der Unschuld haben, desto stärker fesseln
sie die Gemüter durch Allgemeinheit des Gebrauchs, bis die Menschen, durch
öfteren Genuss, sich so sehe daran gewöhnen, dass sie gegen die üblen Folgen
derselben ganz unempfindlich werden, und, mit festem Vertrauen auf ihre
Unschädlichkeit, sie sogar in Schutz nehmen.

\section{10. Abschnitt} \label{kap15_ab10}

Da nun aber eine solche Lebensweise, die nach den Erfindungen der Sinnlichkeit
geführt wird, keinesweges mit der Selbstverleugnung übereinstimmt, sondern im
Gegenteile bloß die Befriedigung der Augenlust, der Fleischeslust und des
hochmütigen\footnote{\texttt{'hofartigen' ersetzt durch 'hochmütigen'}} Lebens zum 
Ziele hat, und die Gemüter der Menschen mit Dingen
beschäftigt, die sie für den Genuss wahrer himmlischer Freuden ganz untüchtig
machen, so sei es allen, die solchen Dingen ergeben sind, hierdurch kund gemacht, dass
das göttliche Leben und die Freuden des Christen von ganz anderer Art als die
Freuden der Welt sind, wie auch bereits oben erklärt worden ist, dass die wahren
Jünger Jesu allen Gegenständen und Beschäftigungen der Welt, die ihre Gemüter
von Gott abziehen, gekreuzigt werden\index{Kreuz!gekreuzigt werden}, und dass ihre Neigungen 
auf erhabenere,
geistige Gegenstände gerichtet sind, so, dass sie die Dinge dieser Welt, selbst
in ihren unschuldigsten Genüssen, gebrauchen, als gebrauchten sie dieselben
nicht. Wollen sie sich aber irgend ein Vergnügen machen, so finden sie dasselbe
in solchen guten Handlungen, wie wir oben bemerkt haben, die auf eine oder die
andere Art anderen zur Wohltat gereichen, wodurch Gott über alles Sichtbare
geehrt, dem Lande ein Dienst geleistet oder das gemeine Wohl befördert wird.
Aus diese Weise geben sie anderen ein guter Beispiel, leben selbst in dieser Welt
glücklich, und hinterlassen der Nachwelt\index{Nachwelt} eine angenehme Erinnerung, indem sie
das Sichtbare und Vergängliche mit einer gegründeten Hoffnung verlassen, dass
ihnen zur rechten Hand Gottes ewige Freude und Wonne zu Teil werden wird. Und
was kann wohl ehrvoller, sicherer und glücklicher sein, als ein solches Leben
und Ende?

