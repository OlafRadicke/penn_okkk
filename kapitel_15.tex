
\chapter{15. Kapitel} \label{kap15}

\section{Zusammenfassung des 15. Kapitel}
\footnotesize
\begin{description}
\item[1. Abschnitt] Ankündigungen der Gerichte Gottes über die Juden, wegen
ihrer Maßlosigkeit\footnote{\texttt{'Üppigkeit' erstzt durch 'Maßlosigkeit'}}, wovon
kein Stand ausgenommen ist
\\(\textit{Seite \pageref{kap15_ab1}})
\item[2. Abschnitt] Christus warnt seine Jünger, sich des
maßlosen\footnote{\texttt{'üppigen' ersetzt durch 'maßlosen'}} Lebens nicht
schuldig zu machen. -- Eine Ermahnung an die Bewohner Englands.
\\(\textit{Seite \pageref{kap15_ab2}})
\item[3. Abschnitt] Mäßigkeit ward von den Aposteln den Gemieinen dringend
empfohlen.
\\(\textit{Seite \pageref{kap15_ab3}})
\item[4. Abschnitt] Wohlgemeinter Rath an England, sich nach dieser Richtschnur
zu prüfen.
\\(\textit{Seite \pageref{kap15_ab4}})
\item[5. Abschnitt] Worin die Erholung der Christen bestehen.
\\(\textit{Seite \pageref{kap15_ab5}})
\item[6. Abschnitt] Wer anderer Vergnügungen bedarf, um sich die Zeit zu
vertreiben, der ist für den Himmel und die Ewigkeit nicht geschickt.
\\(\textit{Seite \pageref{kap15_ab6}})
\item[7. Abschnitt] Der Mensch hat nur wenige Tage zu leben; -- er könnte sie
besser anwenden. Diese Lehre kann keinem, dem es wirklich um sein heil zu thun
ist, unangenem sein.
\\(\textit{Seite \pageref{kap15_ab6}})
\item[8. Abschnitt] Das maßlose\footnote{\texttt{'üppige' ersetzt durch 'maßlose'}}
Leben verhindere nicht nur die Ausübung des
Guten, sondern befördert auch die Begehung des Bösen; es zerreißt das Band der
Ehe und Liebe, zerstört die Gesundheit, richtet das Vermögen zu Grund, u.f.m.
Schaubühnen und Spielhäuser sind die mächtigsten Beförderungtungsmittel dieser
Üebel.
\\(\textit{Seite \pageref{kap15_ab8}})
\item[9. Abschnitt] Wie sehr die Jugend dadurch zur Eitelkeit gereizt wird;
welchen Nachteile das Spiel und die rauschenden Vergnügungen bringen. -- Die
bessern heiden verachteten ein solches Leben.
\\(\textit{Seite \pageref{kap15_ab9}})
\item[10. Abschnitt] Die wahren Jünger Jesu überwinden das
eite\footnote{\texttt{'sterben dem eitlen' ersetzt durch 'überwinden das eite'}}
Wesen der Welt
durch Selbstüberwindung\footnote{\texttt{'Selbsverleugnung ab'  ersettzt durch
'Selbstüberwindung'}}. -- Vergnügen und Belohnung einer guten Anwendung der
Zeit.
\\(\textit{Seite \pageref{kap15_ab10}})

\end{description}
\normalsize

\section{1. Abschnitt} \label{kap15_ab1}

Die Ausschweifungen in Kleiderpracht und eitlen Vergnügungen, welche in der
heiligen Schrift vielfältig verboten\index{Gebote!Bibel!Verbote} sind, gaben auch
Veranlassung zu jener an
das Volk Israel\index{Personen:!Volk Israel} gerichteten traurigen Botschaft,
\footnote{
    Die ausschweifenden Moden, eitlen Gebräuche und Vergnügungenn des gegenwärtigen
    Zeitalters, sind,
    edensowohl als jene der damaligen Zeit, dem Strafgericht Gottes ausgesetzt,
    welches über Englandnnd und Europa schwebt, und über ihre europäischen
    Bewohner hereinzubrechen drohet.}
die der Prophet Jesaias in folgenden Worten erklärte:
\textit{
    "`Darum, spricht der Herr, daß die Töchter Zionz\index{Personen:!Töchter
    Zionz} stolz sind, und gehen mit  anfgerichteten Halse und mit geschtninkten
    Angesichtern, (nach dem Englischen: mit lüsternen Blicken,) treten einher und
    schwänzen, (gehen geziert mit kurzen Schritten,) und haben köstliche Schuhe an
    ihren Füßen; (machen ein Geräusch mit ihren Füßen oder Schuhen;) darum wird der
    Herr den Scheitel der Töchter Zions kahl machen machen und ihre Geschmeide
    wegnehmen. Zu der Zeit wird der Herr den Schmuck an den köstlichen Schuhen
    wegnehmen, und die Hefte, die Spangen, die Kettlein, die Armspangen, die
    Hauben, die Flittern, das Gebräme, die Schnürlein, die Bisamäpfel,
    \footnote{
        Mit wohlriechenden Sachen angefüllte und zur Zierrath
        enghängte Kapseln.}
    die Ohrringe, die Ringe, die Haarbänder, die Feierkleider,
    die Mäntel, die Schleier, die Beutel, die Spiegel, die Koller,
    \footnote{Schürzen von feiner Leinwand.}
    die Boten, die Kittel;
    \footnote{Feine dünne und leichte
        Kleider, mit weichen Staat gemacht wurde.} und es wird Gestank für guten
        Geruch feyn, und ein losee Band für einen Gürtel, und eine Glatze für ein
        kraufes Haar, und für einen weiten Mantel ein enger Sack. Solches alles
	statt deiner Schöne. Deine Männer werden durchs Schwert fallen, und deine
	Krieger im Streite. Ihre There werden trauern und klagen, und sie wird
	jämmerlich (verlassen) sitzen auf der Erde."'}
\footnote{Jesaja 3,16-23.}
\index{Bibelstellen:!Jesaja 3)}

\medskip

Sehet da! ihr eitlen und thörichten Bewohner Englands\index{Orte:!England} und Europas\index{Orte:!Europa}, worin eure
Thorheit bestehet, und was für ein Loos auch ihr zu erwarten habt! Doch leset
auch noch des Propheten Ezechiels Gesicht von der unglücklichen Stadt Tyrus\index{Orte:!Tyrus}; was
für Strafen ihr Stolz und ihre Maßlosigkeit\footnote{\texttt{'Üppigkeit' erstzt durch 'Maßlosigkeit'}} ihr zuzogen. Unter vielen andern
Umständen, die Ezechiel berührt, führt er die Dinge an, worin die Kaufleute mit
Tyruz ihren Handel trieben, nämlich in allerlei kostbaren Sachen: in prächtigen
seiden und gestickten Kleidern und Tüchern, in Purpur, in köstlichen Kasten von
Zedernholz, in feiner Leinwand, in Gewürzen\index{Gewürze}, Korallen, Smaragden, Agaten und
allerlei Edelsteinen, in Gold, Pferden, Wagen, u.s.w. Aber nun höret auch einen
Theil ihrer Strafurtheils!\index{Strafurtheil} \textit{"`Ein Ostwind,"'} sagt der Prophet,
\textit{"`wird dich mitten
auf dem Meere zerbrechen; so, daß alle deine Waaren, (dein Reichtum,) u.s.w.
und alles Volk in dir zur Zeit deines Unterganges mitten im Meere umkommen wird.
Und Alle, die auf den Inseln wohnen, werden über dir erschrecken, und die
Kaufleute dich anpfeifen, daß du so plötzlich untergegangen bist, und nicht mehr
auskommen kannst."'}
\footnote{Ezechiel 27.}
\index{Bibelstellen:!Ezechiel 27)}
Auf diese Weise hat Gott sein Mißfallen
an dem maßlosen\footnote{\texttt{'üppigen' ersetzt durch 'maßlosen'}} und zügellosen Leben der Welt durch Ezechiel deutlich zu erkennen
gegeben. Noch weiter geht aber der Prophet Zephanja, wenn er sagt:
\textit{"`Und am Tage
des Schlachtopfers des Herrn will ich heimsuchen die Fürsten und des Koniges
Kinder, und Alle, die ein fremdes Kleid tragen."'}
\footnote{Zefanja 1,8}
\index{Bibelstellen:!Zefanja 1)}
Solche
schlimme Folgen hatte es zu jener Zeit, wenn die Großen es sich erlaubten, die
eitlen Moden und Gebräuche fremder Völker nachzuahmen, oder bei ihrer Kleidung\index{Kleidung}
nicht auf den wahren Zweck derselben, sondern vielmehr auf die Befriedigung
ihrer Prachtliebe zu sehen.

\section{2. Abschnitt} \label{kap15_ab2}

Der Herr Jesue ertheilte seinen Jüngern den ausdrücklichen Befehl, sich um
weltliche Dinge keine Sorgen zu machen; indem er ihnen zugleich deutlich zu
verstehen gab, daß Diejenigen, die dieses thäten, seine Junger nicht sein
könnten, als er ihnen sagte:
\textit{"`Ihr sollt nicht sorgen und sagen: was werden wie
essen? Was werden wir trinken?. Womit werden wir uns kleiden? Nach allem Solchem
trachten die Heiden. Denn euer himmlischer Vater weiß, daß ihr dieser alles
bedürft. Trachtet aber am ersten nach dem Reiche Gottes und nach seiner
Gerechtigkeit, so wird euch solchen Alles zuzugegeben werden."'}
\footnote{Matthäus 6,31-33.}
\index{Bibelstellen:!Matthäus 6)}
Es leuchtet von selbst ein, daß unser Herr hier unter Essen,
Trinken und Kleidern alle äußern Dinge überhaupt verstehet; da sie, als
sichtbare Dinge, den unsichtbaren und himmlischen, die daß Reich Gottes\index{Reich!Gottes} und
seine Gerechtigkeit betreffen, entgegengesetzt, und auch als Diejenigen, um
welche die Jünger nicht besorgt sein sollen, doch an sich die unschuldigsten und
nothwendigsten sind. Wenn nun aber die Gemüther der Nachfolger Jesu sogar in
solchen Fällen nicht besorgt sein sollen, wie viel weniger dürfen sie sich denn
um die törichten und eitlen Moden und Gebräuche der Welt bekümmern, die keinen
andern Zweck haben, als die sinnlichen Neigungen irdischgesinnter Menschen zu
befriedigen. Dann folgt hieraus auch eben so klar, daß Diejenigen, die in
solchen weltlichen und maßlosen\footnote{\texttt{'üppigen' ersetzt durch 'maßlosen'}} Dingen leben, keine Nachfolger Jesu\index{Nachfolger Jesu}, sondern
\textit{"`den Heiden oder Völkern der Welt gleich sind, die Gott nicht
kennen."'}
\footnote{Lukas 12,22-30.}
\index{Bibelstellen:!Lukas 12)}
Und da also die wahren Junger Jesu und die
Anhänger an den vergnüglichen Dingen der Welt, sich offenbar darin von einander
unterscheiden, daß die Erstern die himmlischen Dinge wahrnehmen, die das Reich
Gottes betreffen, daß in Gerechtigkeit, in Frieden und in Freude im heiligen
Geiste bestehet, und um äußere Gegenstände, selbst um die unschuldigsten und
nothwendigsten, nicht ängstlich besorgt sind, und die Andern ihre vornehmste
Aufmerksamkeit auf Essen, Trinken und Kleider richten, und den Angelegenheiten,
den Freuden, Vergnügungen, Vortheilen und der Ehre der Welt nachstrebenz so laßt
mich euch, ihr Bewohner Englands!\index{Orte:!England} um eures ewigen Heils\index{Heil} willen bitten, einmal
recht ernstlich über euch selbst nachzudenken. Erwäget, wie viel Sorge, Kosten
und Zeit ihr auf thörichte, ja, oft lasterhafte Dinge verwendet; denn daran
könnt ihr erkennen, wie weit ihr von dem ursprünglich wahren Leben der
Christentums abgewichen seid. Betrachtet das Kaufen und Verkaufen, das Handeln
und Dingen, die viele Schreiberei, die Mühe und Arbeit, den Lärm, das Treiben,
die Eile und Verwirrung; das Nachsinnen, die Kunstgriffe, die
Uebervortheilungen; die Vorbereitungen zum Essen und Trinken, zu Besuchen, zum
Putze, zu Vergnügungen und oft ganz lächerlichen Ergötzungen; kurz, \textbf{wie früh die
Menschen aufstehen Müssen und wie spät sie sich erst wieder niederlegen können,
und wie viel kostbare Zeit sie anwenden, um alles das zu besorgen, weis die
Eitelkeit und Thorheit der Welt fordert.} Sehet die Leute auf den Straßen, in den
Läden, an der Börse, im Schauspiele, auf den Wandelplätzen, in den
Kaffeehäusern, u.s.w. und gestehet, ob ihr nicht das Gepränge der Unruhe und
Begierde des Geistes dieser vergänglichen Welt fast auf jedem Gesichte
ausgedrückt findet? \textbf{Sagt nicht: wie sollten denn die Menschen sonst leben, und
wie anders könnte die Welt bestehen? Ein gewöhnlicher, aber schwacher Einwurf!
Denn es ist genug für Alle da\index{es-ist-genug-für-Alle-da}, wenn nur Einige sich mit etwas Wenigerm begnügen
wollten.} Für ein wahrhaft christliches Leben\index{Leben!christliches} sind wenige einfache und anständige
Dinge hinreichend; aber die Sinnlichkeit, der Stolz und der Geiz verleiten die
Menschen zu solchen Thorheiten. Denn, wenn sie ihre Gemüther mehr mit dem Reiche
Gottes beschäftigten, so würden sie wenig an vergängliche Vergnügungen denken,
und ihnen auch nur wenig Zeit widmen.

\section{3. Abschnitt} \label{kap15_ab3}

Diese Lehre von der Selbstüberwindung\footnote{\texttt{'Selbstverleugnung' ersetzt durch 'Selbstüberwindung'}} bestätigten und empfahlen die Apostel, wie
wir bereite bemerkt haben, sowohl durch ihr eigenes Beispiel, als auch durch
Vorschriften in ihren Briefen und Episteln, wie aus zwei sehr merkwürdigen
Stellen hervorgehet, wo Paulus und Petrus und nicht allein melden, wie es in
dieser Hinsicht gehalten werden solle, sondern auch anzeigen, was überwunden\footnote{'verleugnet"'ersetzt durch 'überwunden'} und
vermieden werden müsse. \textit{"`So will ich nun,"'} sagt Paulus,\index{Kleidung!von Frauen}
\textit{"`daß die Weiber in
zierlicher (sauberer) Kleidung sich schmücken,"'} -- wie ist das zu verstehen? --
\textit{"`mit Schamhaftigkeit und Zucht; nicht mit Zöpfen, (aufgeputzten Haaren,) oder
Gold, oder Perlen, oder köstlichen Gewande;"'} -- Dergleichen war also nicht
erlaubt! -- \textit{"`sondern, wie es Weibern, die Gottseligkeit vorgeben, geziemet: mit
guten Werken."'}
\footnote{1. Timotheus 2,9+10.}
\index{Bibelstellen:!1. Timotheus 2)}
Hieraus folget unwidersprechlich, daß
diejenigen Weiber, die sich mit Zöpfen\index{Zöpfe} \index{Schmuck}oder Haarflechten, Perlen, Silber, Gold,
und köstlichkeiten oder prächtigen Kleidern schücken, so lange sie das thun,
keine Anspruch auf Gottseligkeit\index{Gottseligkeit} machen können; indem der Apostel zu erkennen
giebt, daß solcher Schmuck\footnote{\texttt{'Putz' ersetzt durch 'Schmuck'}} mit der christlichen Erhabenheit und Tugend nicht
übereinstimmt, und folglich etwas Böses ist, das den Weibern, welche sich zur
Gottseligkeit bekennen, nicht geziemet. -- Der Apostel Petrus giebt eine
Vorschrift gleichen Inhalts in folgenden Worten: \textit{"`Deren"'} (nämlich der Weiber)
\textit{"`Schmuck soll nicht äußerlich sein, mit Haarflechten und Goldumhängen, oder
Kleideranlegen;"'} -- Wie denn? --
\textit{"`sondern der verborgene Mensch des Herzens,
mit unverrücktem (unverderblichen) sanften, und stillem Geiste, der vor Gott
köstlich ist."'} Und um sie dazu zu bewegen, fügt er noch hinzu:
\textit{"`Denn so haben
sich auch vorzeiten die heiligen Weider\footnote{Es ist bemerkenswehrt, daß der
Apostel hier nur von dem weiblichen Geschlte redet, als ob eitler Putz dem
männlichen nicht eigen wäre. Möchten sie dieses doch behetzigen!} geschmücket,
die ihr Vertrauen auf Gott setzen."'}
\footnote{1. Petrus 3,3-5}
\index{Bibelstellen:!1. Petrus 3)}
Dieses
beweiset nicht allein, daß vorzeiten die heiligen Weiber\index{Personen:!Weiber, heiligen} so geschmückt waren,
und daß es Allen, welche ein heiliges Leben führen und ihr Vertrauen auf den
heiligen Gott setzen wollen, gezieme, sich so zu schmücken, sondern es giebt uns
auch deutlich zu erkennen, daß zu allen Zeiten Diejenigen, welche sich
unerlaubter Zierrathen bedienten, aller ihre frommen Worte ungeachtet, keine von
Denen waren, die wirklich heilig lebten und ihr Vertrauen auf Gott setzten.
Solche sind in der That von einem wahren Vertrauen auf Gott weit entfernt;
weshalb auch der Apostel Paulus ausdrücklich erklährt,
\textit{"`daß Diejenige, welche
in der Maßlosigkeit\footnote{\texttt{'Üppigkeit' erstzt durch 'Maßlosigkeit'}} lebe,"'} in Ansehung des Ledens aus Gott,
\textit{"`lebendig todt sei."'}
\footnote{1. Timotheus 5,6.}
\index{Bibelstellen:!1 Timotheus 5)}
Derselbe Apostel ermahnt ferner die Christen,
\textit{"`daß sie ihren Wandel im Himmel haben, und nach Dem trachten sollen, was droben
ist. Laßt uns ehrbar wandeln, als am Tage!"'} sagt er
\textit{"`Nicht in Fressen und
Saufen; nicht in Kammern und Unzucht; nicht in Hader und Neid. Hurerei und
allerlei Unreinheit, oder Geiz, lasset nicht von euch gesagt werden; auch keine
schandbare Worte und Narrentheidinge (Possen) oder Scherz: sondern vielmehr
Danksagung. Lasset kein faules Geschwätz aus euerem Munde gehen, sondern was zur
nöthigen Besserung dienet, und holdselig zu hören ist (oder den Hörenden
Erbauung gewährt). Ziehet den Herrn Jesum Christum an, und wartet des Leibes
nicht so, daß er geil werde. Und betrüget nicht den heiligen Geist Gottes,"'} --
welches durch ein maßloses\footnote{\texttt{'üppiges' ersetzt durch 'maßloses'}} Leben geschiehet; --
\textit{"`sondern seid Gottes Nachfolger,
als geliebte Kinder. Und wandelt vorsichtig, nicht wie Unweise, sondern wie
Weise; und schicket euch in die Zeit, (erkaufet die gelegene Zeit) denn es sind
böse Tage."'}
\footnote{Philipper 3,20)  Kolosser 3,2)  Römer 13,13+14) Epheser 4,29+30) Epheser 5,1+3+4+15+16}
\index{Bibelstellen:!Philipper 3)}
\index{Bibelstellen:!Kolosser 3)}
\index{Bibelstellen:!Römer 13)}
\index{Bibelstellen:!Epheser 4)}
\index{Bibelstellen:!Epheser 5)}

\section{4. Abschnitt} \label{kap15_ab4}

Hiernach könnt ihr euch nun prüfen, ihr Bewohner diesen Landes! die ihr glaubt,
man thue euch Unrecht, wenn man euch nicht für wahre Christen halt. Untersuchet
doch nur, in wiefern euer Leben und eure Gesinnungen mit diesen heiligen
Vorschriften und Beispielen der Selbstüberwindung\footnote{\texttt{'Selbstverleugnung' ersetzt durch 'Selbstüberwindung'}} übereinstimmen. O! meine
Freunde! Meine Seele trauert über euch! Ich lebte einst selbst mit euch und in
eurer Mitte. Euer Leben und die Dinge, womit ihr eure Zeit hinbringe, sind
meiner Bemerkung nicht entgangen, und mit tiefem Bedauern, ja mit
unaussprechlichem Mitleide, beklage ich eure Thorheit. O! daß ihr weise würdet!
O! das ihr der \textbf{innern Stimme eures Richters} Gehör geben, und der Ewigkeit Zeit
vergönnen wolltet, ein wenig mit euch zu rechten! Warum müssen eure Kleider,
eure Putztische, Besuche, Schauspiele, Gastmähle und so viele andere armselige,
vorrübergehende Freuden, eure ganzen Seelen besitzen, und sowohl über eure Zeit,
Sorgfalt und Aufmerksamkeit, als auch selbst über euer Vermögen gebieten? Lasset
euch, -- ich bitte euch im Namen den lebendigen Gottes! -- lasset euch den Rath
und die Ermahnung eines Mannes gefallen, der, wie Einige unter euch wissen, an
allen solchen Dingen Theil genommen und daher Zeit und Gelegenheit gehabt hat,
aus eigener Erfahrung einzusehen, wie wenig dergleichen Eitelkeiten zu unserer
wahren und dauernden Zufriedenheit\index{Zufriedenheit} beitragen können. Nein! meine Freunde! Gott,
der Allmächtige\index{Gott!Allmächtiger}, weiß es, -- und o! möchtet ihr mir glauben, und meinen Rath
annehmen! -- sie enden in Schande und Kummer..\textbf{Er, der Allerheiligste, der getreu
und wahrhaftig ist, hat es einmal so beschlossen , daß jeder Mensch, er sei Mann
oder Weib, das ernten soll, was er gesäet hat; und wird es daher nicht eine
traurige und schreckliche Ernte sein, wenn ihr für alle eure Zeit und Ausgaben,
die ihr bloß aus Ueberflüssigkeiten und eitlen Vergnügungen so übel verwendet
habt, nur Unruhe, Angst und getäuschte Hoffnung ernten werdet? -- O! darum
ziehet euch in euch selbst zurück! Dämpfet nicht den heiligen Geist Gottes in
eurem Innern! Bestrebt  euch, eure so sehr gemissbrauchte Zeit wieder zu
erkaufen, und suchet einen Umgang, der euch in der Ueberwindung eurer verderbten
Neigungen dehülflich sein kann} Dann werdet ihr zu der Fähigkeit gelangen, daß
ihr die Gebote Jesu Christi halten und sowohl seinem als auch seiner Jünger
Beispiele folgen könnt. Denn aus Dem, was bisher erkläret worden ist, gehet
deutlich hervor, daß eine solche Lebensweise, als unter euch in diesem Lande
geführet wird, nie wahrhaft christlich war, noch jemals sein wird.

\section{5. Abschnitt} \label{kap15_ab5}

\textbf{Die beste Erhohlung der Christen ist: Gutes thun. Alle ihre Gebräuche bezzielen
Maßigkeit und irgend einen wohlthätigen Zweck, der sich mehr oder weniger mit
jeder Handlung verbinden läßt.}
\footnote{1. Petrus 1,15.) Hebräer 10,25) 1. Petrus 4,9-11)
Matthäus 25,36+37) Philipper 2,4) Epheser 4,8.) 1. Mose 5,24) Psalm 1,2) Psalm 119,15) Psalm 143,5}
\index{Bibelstellen:!1. Petrus 1)}
\index{Bibelstellen:!Hebräer 10)}
\index{Bibelstellen:!1. Petrus 4)}
\index{Bibelstellen:!Matthäus 2)}
\index{Bibelstellen:!Philipper 2)}
\index{Bibelstellen:!Epheser 4)}
\index{Bibelstellen:!1. Mose 5)}
\index{Bibelstellen:!Psalm 1)}
\index{Bibelstellen:!Psalm 119)}
\index{Bibelstellen:!Psalm 143)}
\index{Beschäftigung!christlicher}\textbf{Z. B. wenn Männer und Frauen ihre Berufsgeschäfte fleißig wahrnehrnen;
religiösen Versammlungen beiwohnen; ihre gutgesinnten Nachbarn besuchen\index{Besuche}, um sich
mit ihnen zu erbauen, und die bösartigen, um sie zu bessern. Wenn sie Sorgfalt
auf die Erziehung\index{Erziehung} ihrer Kinder\index{Personen:!Kinder} verwenden; ihren Dienstboten\index{Personen:!Dienstboten} mit guten Beispielen
vorgehen; die Armen unterstützen; Kranke pflegen\index{Krankenpflege}; Gefangene besuchen\index{Orte:!Gefängnis!Besuche}, zur
Erleichterung ihrer unglücklichen Lage beitragen, und Ruhe und Frieden unter den
Nachbarn zu befördern streben. Auch ist das mäßige Studium\index{Erkenntnis!Studium} nützlicher und
empfehlenswerther Künste\index{Kunst} und Wissenschaften\index{Erkenntnis!Wissenschaft}, z. B. der Arithmetik\index{Erkenntnis!Arithmetik}, Geometrie\index{Erkenntnis!Geometrie},
Mechanik\index{Erkenntnis!Mechanik}, der Schiffahrtskunde\index{Erkenntnis!Schiffahrtskunde}, der Landwirtschaft\index{Erkenntnis!Landwirtschaft}, des Gartenbaues\index{Erkenntnis!Gartenbau}, der
Arzneiwissenschaft\index{Erkenntnis!Medizien}, Naturkunde, U. dgl. ein angenehmea Vergnügen für das
mannliche Geschlecht, so wie für das weibliche Spinnen, Nahen, Stricken,
Weberei, Gartenkunst, Einmachen der Früchte und andere nützliche Hausarbeiten\index{Hausarbeiten},
sehr angenehme Beschäftigungen sind, welche sogar die vornehmsten und edelsten
Matronen und jungen Personen unter den Heiden trieben, die sich ein Vergnügen
daraus machen, Andern, die aus Armuth keine Dienstboten halten könnten, in ihren
häuslichen Beschäftigungen zu helfen, um ihnen ihre notwendigen Arbeiten zu
erleichtern. Das größte und wichtigste Vergnügen wahrer Christen bestehet aber
darin, daß sie sich oft von allen weltlichen Gegenständen zurückziehen, um, in
geheimen und stillen Betrachtungen\index{Stille Andacht/Betrachtung} über das göttliche Leben und himmlische Erbe\index{Erbe!himmlisches},
daß Nahesein des Herrn zu genießen.} -- Und dieses zu versäumen, um, unter dem
Vorwende der Erholung, andern Vergnügungen nachzugehen, ist in der That höchst
strafbare Lüsternheit und verwerfliche Ungottseligkeit.

\medskip

Es ist eine sehr gewöhnliche, aber eben so nichtige Einwendung, wenn Einige
sagen, daß sie sich doch nicht beständig mit ernsthaften Gegenständen
beschäftigen könnten, und es ihnen daher nicht zu verargen sei, wenn sie sich
die allgemein üblichen Zerstreuungen erlaubten. Denn ich frage, was sie denn
eigentlich begehren? Was sie thun und haben wollen? Wer ein ordentliches
Geschäft oder Gewerbe treibt, kann kaum Zeit finden, die Hälfte von Dem, was
oben anempfohlen ist, zu thun. Und was Diejenigen betrifft, die nichts zu thun
haben, -- und daher auch gewöhnlich nichts thun, als, was das schlimmste von
Allem ist, nämlich sündigen; -- \textit{so habe ich innen ja eine Menge angenehmer,
nützlicher und selbst ehrenvoller Beschäftigungen und Vergnügungen
vorgeschlagen, unter welchen sie eine oder die andere wählen könnten. Allein das
sind keine Erholungen für sie. Die Dinge, an denen sie das größte Behagen
finden, sind: Schauspiele, Bälle, Karten, Würfel u.s.w.., oder: Trinken,
Schwärmen\index{Schwärmen}, Schwelgen und dergleichen, und zwar vom Morgen bis in die Nacht; ja,
sie verwandeln die Nächte in Tage und verkehren die Ordnung der Natur, um ihre
verderbten Neigungen zu befriedigen.
\footnote{Amos 6,3-8.}
\index{Bibelstellen:!Amos 6)}
Und müßten sie nicht
essen und schlafen, so würden sie ohne Zweifel nie Zeit finden, von ihren eitlen
Zeitvertreiben aufzuhören, bis der plötzliche Ruf des Todes sie aufforderte, in
einer andern Welt zu erscheinen. Aber sonderbar ist es, daß diese Leute, die von
ihrem Spieltische nicht aufstehen können, es unerträglich finden, und kaum für
möglich halten, daß Jemand lange unter einer nützlichen Gemüthsübung oder in
einer religiösen Versammlung aushalten könne.}

\section{6. Abschnitt} \label{kap15_ab6}

\textbf{Aber wie glauben sie denn die unendliche Ewigkeit zubringen zu wollen? Denn
"`wie der Baum fällt, so wird er liegen."'}
\footnote{Kohelet 11,3. }
\index{Bibelstellen:!Kohelet(Prediger) 11)}
O! daß die Menschen sich doch nicht selbst
betrügen, und ihre unsterblichen Seelen mit dem angenehmen, aber falschen und
verderblichen Träume täuschen wollten, daß eine gewaltsame und unwiderstehliche
Macht sie in dem Augenblicke, wo Leib und Seele geschieden werden, verändern und
bekehren werde! Nein! meine Freunde!
\textit{"`Was ihr säet, das werdet ihr auch ernten."'}
\footnote{Galater 6,4-9. Epheser 5,6.}
\index{Bibelstellen:!Galater 6)}
\index{Bibelstellen:!Epheser 5)}
Habt ihr Eitelkeit, Thorheit,
sichtbare Freuden, vergänglichen Genuß ausgesäet, so werdet ihr nichts Besseres
als Verderben, Kummer, und peinliche Angst ewiger Verzweiflung ernten können. --
Aber ach! was ist die Ursache der so allgemeinen Ausrede der Menschen, daß man
doch nicht immer über geistliche Dinge grübeln müsse? Gewiß, sie ist keine
andere, als weil sie die Freude und den Frieden nicht kennen, Welche die Seele
Desjenigen genießt, der beständig als in der Gegenwart Gottes\index{Gott!Gegenwart} redet und handelt.
Dieses Gefühl des göttlichen Friedens übersteigt die eitlen Begriffe solcher
Menschen, deren Verstand von der Herrlichkeit und den Freuden des Gottes dieser
Welt verblendet und verfinstert ist.
\footnote{Epheser 4,18-20.}
\index{Bibelstellen:!)}
Daher bestehen
auch ihre Religionsübungen bloß in einer leeren kraftlosen Wiederhohlung
auswendig gelernter Worte. Denn wenn sie einen Schatz im Himmel hätten, so würde
auch ihr Herz daselbst sein, und sich mit dem Gegenstande ihrer größten Freude
und Anhänglichkeit beständig beschäftigen. Und ich nehme keinen Anstand, gerade
heraus zu erklären, daß diejenigen Menschen, denen dieses eine Last ist, und
welche daher Erhohlungen in Schauspielen\index{Schauspiel}, Bällen, Spielpartien und andern
Zeitvertreiben suchen, weder jemals die Freundlichkeit Gottes und seiner
heiligen Wahrheit geschmecket haben, noch die Fähigkeit besitzen, dieselbe in
einer künftigen Welt zu genießen. \textbf{Denn, wie wäre es möglich, daß sie während
einer ganzen Ewigkeit wahren Genuß und Befriedigung für ihren Geist in einem
Gegenstande finden sollten, der ihnen in dem kurzen Zeitraume von dreißig oder
vierzig Jahren schon so langweilig und unerträglich geworden ist, daß sie, um
sich zu erhohlen, ihre Zuflucht zu den geringfügigen Tändeleien und
Zeitvertreiben dieser vergänglichen Welt nehmen müssen.} Wahrlich! Diejenigen,
welche glauben,
\textit{"`daß sie einst von jedem unnützen Worte Rechenschaft geben
müssen,"'}
\footnote{Matthäus 12,26.}
\index{Bibelstellen:!Matthäus 12)}
dürfen keine Belustigungen\index{Belustigung} aufsuchen, um
muthwillig eine Zeit zu \textbf{\textit{vertreiben}}, die sie mit allem Fleiße zu erkaufen
angewiesen sind. O! wenn die Menschen bedächten, daß sie nichts Geringeres zu
thun haben, als
\textit{"`ihren Beruf und ihre Erwählung fest zu machen;"'}
\footnote{2. Petrus 1,20}
\index{Bibelstellen:!2. Petrus 1)}
so würden sie weniger bemüht sein, immer neue Erhohlungen für ihre
leeren Seelen zu erfinden, und ihnen Tage, Monate und Jahre zu widmen, während
sie nicht den vierten Theil ihrer kostbaren Zeit auf die wichtigste
Angelegenheit ihres Lebens, auf die Errettung ihrer unsterblichen Seelen,
verwenden, wozu dieselbe ihnen doch gegeden wird.

\section{7. Abschnitt} \label{kap15_ab7}

Es ist gewiß nicht nöthig, durch eitle Ergötzlichkeiten eine Zeit zu vertreiben,
die an sich schon so schnell verstreicht, und welche, wenn sie einmal entflohen
ist, Niemand zurückrufen kann Schauspiele\index{Schauspiel}, Bälle\index{Bälle}, Concerte\index{Konzerte}, Lustpartien, Romane\index{Romane}
und dergleichen nutzlose Unterhaltungen, womit so Viele, die ihnen ergeben sind,
ihre edle Zeit hinbringen, werden am Tage der Offenbarung des gerechten Gerichts
Gottes\index{Gericht!Jüngstes} ihnen nur zu ihrer Verdammniß\index{Verdammnis} gereichen. Diese Dinge, o! meine Freunde!
sind Erfindungen eines Geistes, der gleich im Anfange den Geschmack an dem
entzückenden Genusse der heiligen Gegenwart Gottes verlor. Und wollten diesem
zufolge die Menschen nur die große Zufriedenheit und sichere Belohnung, weiche
ein allgemein \index{Wohltätigkeit} wohltätiges und tugendhaftes Leben sowohl in dieser Welt
begleiten, als auch in der künftigen erwarten, in gehörige Erwägung ziehen, und
sich dadurch von ihren eitlen, fruchtlosen Erhohlungen und Zeitvertreiben
entwöhnen lassen; so wurden sie finden, daß jene bereits erwähnten,
vortrefflichen und der Neigungen vernünftiger Wesen würdigen Beschäftigungen,
mehr als hinreichend waren, ihre Nebeastunden auf eine Art auszufüllen, die
nicht nur ihnen selbst, sondern auch Andern, edles Vergnügen, angenehme
Unterhaltung und wahren Nutzen gewähren würde. Auch kann das hier Gesagte nur
solchen Menschen mißfallen, die nicht wissen, was es heißt: mit Gott wandeln,
sich auf die ewige Wohnung\index{Wohnung!ewige} vorbereiten, das Gemüth mit wahrhaft guten,
himmlischen Gegenständen beschäftigen und den Beispielen der Gläubigen einer
glücklichen Vorzeir nachstreben. Es kann nur Solchen zu wider sein, weiche von
der Lehre, von dem Leben, dem Tode und der Auferstehung Christi\index{Auferstehung Christi} keine wahre
Erkenntnis: haben, und deren Gemüther nur auf sinnliche Gegenstände gerichtet
sind, von denen sie hingerissen, getäuscht und ins Verderben\index{Verderben} gezogen werden. Ja,
es kann endlich nur Denen veraähtlich und anstößig vorkommen, die den Himmel
verachten und die unsichtbaren, aber ewig dauernden Freuden des Geistes, dem
Genusse einiger armseligen, vorübergehenden Ergötzungen der Sinne aufopfern.
Läßt sich von Solechen wohl sagen,
\textit{"`daß sie in Jesum Christum, in sein
schmerzhaften Leiden, in seine schmachvollen Tod getauft, und mit ihm zu einem
Verlangen nach unsterblichen Genüssen, zu himlischen Betrachtungen, zu einem
neuen göttlichen Leben erwecket und auferstanden sind? daß sie in der
Erkenntnis; der göttlichen Geheimnisse und in der Heiligkeit wachsen, die fsie
zu der völligen Größe Christi hinankommen, und Ihm ähnlich werden, der das große
Muster Aller ist?"'}
\footnote{Römer 6,3-8.) 1. Korinther 12,13) Galater 3,27) Kolosser 2,12+13)
Epheser 4,13.}
\index{Bibelstellen:!Römer 6)}
\index{Bibelstellen:!1. Korinther 12)}
\index{Bibelstellen:!Galater 3)}
\index{Bibelstellen:!Kolosser 2)}
\index{Bibelstellen:!Epheser 4)}
Ich sage, ob diese nothwendigen Eigenschaften eines Christen bei
ihnen anzutreffen sind, und was sie davon erfahren haben, möge, nach ernster
Untersuchung und kalter Ueberlegung, ihr eigenes Gewissen ihnen beantworten.

\section{8. Abschnitt} \label{kap15_ab8}

Die große Liebe, mit welcher die Menschen an eitlen Moden und Zeitvertreiben
hängen, beweiset aber nicht allein ihren Weltsinn und ihren Mangel an Kenntniß
göttlicher Freuden; ihre Nachahmung jener schädlichen Erfindungen und ihr
häufige Besuchen der Belustigungsörter verhindert sie auch an vielem Guten, das
sie thun könnten, und öffnet ihnen die Thür zu vielem Bösen, wozu sie sich
verleiten lassen. Erstlich verlieren sie eine kostbare Zeit, die aus dem
Sterbebette eine ganze Welt aufwiegen würde; dann verschwenden sie das Geld,
welchers zu bessern und gemeinnützigen Zwecken hätte verwendet werden können.
Sie lernen Vergnügen an Dingen finden, die ihnen zur Schande gereichen; sorgen
nur für die Befriedigung ihrer verderbten Neigungen, und verhärten ihre Gemüther
gegen die Eindrücke himmlischer Dinge, indem sie dieselben beständig mit
Gegenständen der Thorheit beschäftigen. Vom Stolze und von der Prachtliebe
geblendet, legen sie einen so hohen Werth auf die Kleidung, -- die doch nur zur
anständigen Bedeckung der Blöße dienen sollte --, \textbf{daß sie oft das edelste
Geschöpf Gottes vernachläßigen und verachten, indem sie den Menschen bloß nach
der Pracht und dem modigen Zuschnitte seiner Kleider beurtheilen und schätzen,
woraus dann ganz natürlich das Ansehen der Person entspringt. Diesers ist so
klar, daß Jemand, der es leugnen wollte, auch eben so leicht die Abwesenheit der
Sonne am hellen Mittage behaupten könnte.} Denn was ist gewöhnlicher und
weltkundiger, als daß die Menschen mit ihren Komplimenten, Verbeugungen und
Anreden sich nach dem mehr oder minder prächtigen Anzuge der Personen richten,
mit denen sie zusammentreffen; wiewohl dieses in den Augen Gottes hochst
mißsfällig und in der heiligen Schrift so ausdrücklich verboten ist, daß
Diejenigen, die es thun, als Uebertreter des göttlichen Gesezess betrachtet und
folglich auch als Solche bestraft werden.
\footnote{Jakobus 2,1-10.}
\index{Bibelstellen:!Jakobus 2)}
Und was für nachtheilige
Folgen haben nicht ferner noch diese Ausschweifungen in allerlei Zerstreuungen
und Belustigungen, sowohl hinsichtlich des Vermögens als auch der Moralität der
Menschen? Wie oft werden nicht bei solchen Gelegenheiten die Geschäfte
vernachläßigt, junge Mädchen verführt, die Bande der \index{Eherbruch}Ehe verletzt? Was für Zank
und Mißhelligkeiten richten sie nicht in Familien an? und wie oft veranlassen
sie nicht Trennungen unter Eheleuten, Enterbungen, Dienstentlassungen u.s.w.
Wie häufig werden nicht deshalb die Dienstboten\index{Personen:!Dienstboten} wie Sklaven\index{Personen:!Sklaven} behandelt, und die
Kinder\index{Personen:!Kinder} ganz vernachläßigt? Auch ist nicht selten die Ausschweifungen und
unmäßige LebenPersonen:!sart der Männer Schuld, daß sie ihre \index{Personen:!Frauen!Misshandlung}Frauen geringachten und
schändlich mißhandeln, welche dann oft deshalb zu gleichen Ausschweifungen
schreiten, oder eine so ungerechte Behandlung sich so sehr zu Herzen nehmen, daß
sie ihre Tage in Kummer und Elend hinbringen. \index{Schauspiel}\textbf{Unter allen jenen unglücklichen
Erfindungen der Üppigkeit sind jedoch die Schauspiele, diesse Schulen der
Verführung, die verderblichfsten; da sie das Laster reißend darstellen, und fast
immer, wo nicht von offenbar schmutzigen und schändlichen, deoch von
zweideutigen, schlüpfrigen und lappischen, Dingen handeln, die, wie allgemein
bekannt ist, einen sehr schädlichen Einfluß auf die Zuschauer, und vornemlich
auf die so leicht einpfänglichen Gemüther der Jugend haben.} Allein, so klar es
auch am Tage liegt, daß kaum feinere und hinreißendere Verführungsmittel, als
die Schauspiele sind, für den Leichtsinn der Menschen hatten erdacht werden
können; -- wie sich aus dem Folgenden auch noch deutlicher entwickeln wird; --
so werden, dessenungeachtet, die der Tugend so gefährlichen Theater noch immer
unterstützt und unterhalten. Und gewiß, nur das außerordendliche Vergnügen,
welches die Menschen darin finden, kann ihre Augen so sehr verblenden, daß sie
das Schädliche derselben nicht einsehen.

\section{9. Abschnitt} \label{kap15_ab9}

Auch erzeugen diese Zerstreuungsmittel eine gänzliche Abneigung gegen ernste und
gründliche Betrachtungen religiöser Gegenstände; indem sie die Gemüther der
Menschen mit Rückerinnerungen an die verschiedenen abenteuerlichen
Begebenheiten, mit denen sie unterhalten wurden, unaufhörlich erfüllen, und
dadurch, besonders bei jungen Leuten\index{Personen:!jungen Leuten}, die Einbildungskraft erhitzen und die
Leidenschaften\index{Leidenschaft} aufregen. Die anderen gewöhnlichen Belustigungen, nämlich die
Bälle, Gasterein, Spiele, u. dgl. haben soft ähnliche nachtheilige Folgen, wie
die vielen Zänkereien und Feindseligkeiten, Zweikampfe, Zeit- und
Vermögensverschwendungen, u.s.w. die sie nicht selten veranlassen, leider
überflüssig beweisen. Kurz, es sind Gebräuche, die unter den Heiden\index{Personen:!Heiden}, die Gott
nicht kannten, nie aber unter den Gläubigen, die ihn fürchteten, üblich waren.
Selbst die Bessern unter den Heiden\index{Personen:!Heiden}, z. B. Anaragoras\index{Personen:!Anaragoras}, Sokrates\index{Personen:!Sokrates}, Plato\index{Personen:!Plato},
Antisthenes\index{Personen:!Antisthenes}, Heraklit\index{Personen:!Heraklit}, Zeno\index{Personen:!Zeno}, Aristides\index{Personen:!Aristides}, Kato\index{Personen:!Kato}, Cicero\index{Personen:!Cicero}, Epiktet\index{Personen:!Epiktet}, Seneka\index{Personen:!Seneka}, und
Andere, haben in ihren Schriften ihren Abscheu vor solchen Dingen an den Tag
gelegt, die ihnen hassenswürdig waren, und sowohl gegen die Ehre des
Unsterblichen Gottes, als auch gegen alle gute Ordnung in den Regierungen\index{Regierung} zu
streiten schienen; weil sie zur Ungebundendeit, zum Müßiggange, zur Unwissenheit
und Weichlichkeit, den giftigen und ansteckenden Seuchen der Staaten\index{Staat} und Reiche\index{Reich},
führen. Aber so groß ist die Anmaßung der leichtsinnigen Menschen unsern
Zeitalters, daß sie sich beinahe schon für Heilige halten, wenn sie sich von
solchen groben Lastern und Verbrechen frei wissen, die das Gesetz mit
Gefängnisstrafen belegt. Ihre Gemüther sind von der vermeinten Unschuld ihrer
gewöhnlichen Auschweifungen so eingenommen, daß sie sich ihnen ganz hingeben,
und um bessere Dinge sich wenig bekümmern. Dies macht sie so kühn in ihrer
Vertheidigung derselben, und so entschlossen, keinen Gedanken dagegen
einzulassen. Und was hat dieses für einen Grund? -- Die Freiheit, in der sie
leben, behagt ihrer Sinnlichkeit; sie befriedigt das lüsternde Auge und den
verwöhnten Gaumen der verderbten Natur. Daher halten sie es schon für etwas
Lobenswerthes, wenn Jemand in seinen Genüssen sich auf das Beispiel der Thiere
beschränkt, die nur das zu sich nehmen, was die Natur fordert; wiewohl selbst
die Anzahl Derer, die so denken, nur gering ist. So sehr haben Viele in unsern
Tagen sich der Unmäßigkeit ergeben, daß sie keine andere Richtschnur für ihre
Handlungen, als ihren eigenen Willen anerkennen, und wenn sie es hoch dringen,
einen Rubin darin suchen, daß sie sich keiner der niedrigsten Laster schuldig
gemacht haben. Diesen kann man ihnen auch in der That zu einer Zeit nicht
absprechen, wo keine Handlung so abscheulich sein kann, daß nicht Einige sie
dennoch für erlaubt hielten. Indessen ist es gewiß doch immer ein Beweis
allgemein herrschender Ruchlosigkeit unter den Einwohnern eines Landes, wenn
Jemand bei ihnen schon für tugendhaft, ja, für einen Christen gilt, und sogar im
Rufe nicht geringer Frömmigkeit stehet, sobald man ihn: nur keine solche Laster
vorwerfen kann, als selbst die Heiden\index{Personen:!Heiden} verabscheuen. Welch ein trauriges Zeichen
der Verderbtheit eines Landes! Aber wie groß muss nicht die Verblendnng sein,
wenn Leute, die sich für gläubige Christen halfen, dieselben Gebräuche, welche
von denen, die sie Ungläubige nennen, als schändliche Dinge verworfen werden,
nicht für das erkennen können oder wollen, was sie wirklich sind, sondern ihnen
die schönen Nennen: Anstand, Artigkeit, guter Ton, Erhohlung, u. dgl. beilegen.
Aber, meine Freunde! gesetzt auch, es gäbe keinen Gott, und weder Himmel noch
Hölle; keine Beispiele heiliger Menschen; keinen Herrn Jesurn Cuhristum, dessen
Kreuze wir uns unterwerfen\index{unterwerfen}, und nach dessen Lehre und Leben wir uns richten
müßten; würde es nicht dennoch immer eine edlere Beschäftigung und weit
würdigere Verwendung unserer Zeit und unseres Geldes sein, wenn wir
Mildthaätigkeit\index{Mildthaätigkeit} gegen Arme ausübten, Dürftige und Verlegene unterstüzten,
Frieden und Eintracht unter Nachbarn zu befördern strebten, Kranke besuchten,
der Wittwen und Waisen uns annähmen, und andere schon gedachte gute Handlungen
verrichteten, statt eitlen Belustigungen und Zerstreuungen nachzugehen? Es läßt
sich in der That auch nicht denken, daß auf dem Wege zur Seligkeit eine solche
Verschiedenheit sinnlicher Genüsse sollte anzutreffen sein, als die mehrsten
Menschen sich erlauben; denn sonst müßte wirklich ein überzeugtes Gewissen, ein
geängstigter Geist, ein zerschlagendes Herz, eine wiedergeborene Seele, mit
einem Worte, die Unsterblichkeit selbst eine bloße Erdichtung sein, wie Einige
behaupten wollen, und Andere daher auch geneigt sind zu glauben.
\footnote{Weisheit Salomos 18,14)
Psalm 52,17)
Matthäus 5,4)
Lukas 6,25)
Römer 2,7)
Psalm 40,8)
Römer 7,22)
Hebräer 11,13-16)
Römer 1,25-30)}
\index{Bibelstellen:!Weisheit Salomos 18)}
\index{Bibelstellen:!Psalm 52)}
\index{Bibelstellen:!Matthäus 5)}
\index{Bibelstellen:!Lukas 6,)}
\index{Bibelstellen:!Römer 2)}
\index{Bibelstellen:!Psalm 40)}
\index{Bibelstellen:!Römer 7)}
\index{Bibelstellen:!Hebräer 11)}
\index{Bibelstellen:!Römer 1)}
Nein! solche weltliche Freuden finden im Reiche Gottes\index{Reich!Gottes}
nicht Statt, und müssen daher aus dem Gebiete des Christenthumes uns immer
gänzlich verwiesen werden. Denn ich behaupte, daß sie für Jeden, der Gott in
seinem Innern kennt, und ein Gefühl von seiner beseligenden Gegenwart hat, ein
wahrer Tod sind. Ja, sie sind auch gefährlicher und mehr geeignet, das Gemüth
von den Beschäftigungen mit göttlichen Gegenständen abzuzieben, als die gröbern
Ausschweifungen, deren Abscheulichkeit leichter in die Augen fällt, und welche
daher auch den Vielen schon durch Hülfe ihrer Erziehung und gewohnten Mäßigkeit,
oder aus natürlichem Widerwillen verabscheuet werden; so wie sie auch, wenn sie
begangen sind, allezeit einen ihnen angemessenen hohen Grad der überzeugenden
Unruhe und Bestrafung mit sich führen. Die vorgeblich unschuldigen Vergnügungen,
diese vermeinten unschädlichen Erheiterungen hingegen, schleichen sich
unmerklich in das Gemüth ein, und können eben deshalb desto zerstörender um sich
greifen. Denn da sie die Sinne sanft ansprechen, so gewinnen sie auch leicht
Eingang, und jemehr sie den Anschein der Unschuld haben, desto stärker fesseln
sie die Gemüter durch Allgemeinheit des Gebrauchs, bis die Menschen, durch
oftern Genuß, sich so sehe daran gewöhnen, daß sie gegen die üblen Folgen
derselben ganz unempfindlich werden, und, mit festem Vertrauen auf ihre
Unschädlichkeit, sie sogar in Schutz nehmen.

\section{10. Abschnitt} \label{kap15_ab10}

Da nun aber eine solche Lebensweise, die nach den Erfindungen der Sinnlichkeit
geführt wird, keinesweges mit der Selbstverleugnung übereinstimmt, sondern im
Gegentheile bloß die Befriedigung der Augenlust, der Fleischeslust und des
hochmütigen\footnote{\texttt{'hofartigen' ersetzt durch 'hochmütigen'}} Lebens zum Ziele hat, und die Gemüther der Menschen mit Dingen
beschäftigt, die sie für den Genuß wahrer himmlischer Freuden ganz untüchtig
machen; so sei es Allen, die solchen Dingen ergeben sind, hierdurch kund, daß
das göttliche Leben und die Freuden des Christen von ganz anderer Art als die
Freuden der Welt sind, wie auch bereite oben erklärt worden ist; daß die wahren
Jünger Jesu allen Gegenständen und Beschäftigungen der Welt, die ihre Gemüter
von Gott abziehen, gekreuzigt werden\index{Kreuz!gekreuzigt werden}, und daß ihre Neigungen auf erhabenere,
geistige Gegenstände gerichtet sind; so, daß sie die Dinge dieser Welt, selbst
in ihren unschuldigsten Genüssen, gebrauchen, als gebrauchten sie dieselben
nicht. Wollen sie sich aber irgend ein Vergnügen machen, so finden sie dasselbe
in solchen guten Handlungen, wie wir oben bemerkt haben, die auf eine oder die
andere Art Andern zur Wohlthat gereichen, wodurch Gott über Alles Sichtbare
geehret, dem Lande ein Dienst geleistet, oder das gemeine Wohl befördert wird.
Aus diese Weise geben sie Andere ein guter Beispiel, leben selbst in dieser Welt
glücklich, und hinterlassen der Nachwelt\index{Nachwelt} eine angenehme Erinnerung; indem sie
das Sichtbare und Vergängliche mit einer gegründeten Hoffnung verlassen, daß
ihnen zur rechten Hand Gottes ewige Freude und Wonne zu Theil werden wird. Und
was kann wohl ehrenvoller, sicherer und glücklicher sein, als ein solches Leben
und Ende?
