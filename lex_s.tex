\section*{S}

\articlesize

\begin{description}
 \item[Schissma]

 \item[Schreiber] $\to$\textit{Clerk}

 \item[Seebohm, Ludwig] 

 \item[Seekers]

 \item[Sense of the Meeting]

 \item[Service Civil International (SCI)]

 \item[Shaker]

 \item["`Speaks to my condition"' or "`Friend speaks my mind"'] (Frei übersetzt: "`Er spricht mir aus der Seele"' oder "`er spricht in meinem Sinne"')
    Commonly used during meetings for business to express that another Friend has spoken what is in the mind of the speaker; used to help add weight to the ministries of others.

 \item[Stephen, Caroline Emelia] $\ast$1834 \dag1909. Quakerin und ab 1889
 $\to$Minister. Als sie starb, vermachte sie iherer Cousine Virginia Woolf,
 die spähter sehr bekannte Autorin, 2.500 Pfund. Diese Summe reichte damals um
 finanziell abgesichert zu sein. \endnote{Biographisch-Bibliographischen
 Kirchenlexikon, Band XXXII (2011) Autor: Claus Bernet, \\
 \texttt{http://www.kirchenlexikon.de/s/s4/stephen\_c\_o.shtml}}
 \endnote{"`Quäker"', ISSN 1619-0394, 5/2009, Seite254}

 \item[Suffering]

 \end{description}

\normalsize
