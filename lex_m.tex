\section*{M}

\articlesize

\begin{description}

 \item[Meeting for Sufferings]
 
 \item[meeting for worship]  $\to$\textit{Andacht}

 \item[Member by Convincement] $\to$\textit{Convinced Friend}
 
 \item[Ministry]
    the act of speaking during a meeting for worship. (Many Friends use the term more broadly to mean living their testimonies in everyday life). "`Vocal"' or "`proclamational"' refer to ministries that are verbal.
 
 \item[Monatsversammlung]
 
 \item[Musik] Von anfang an war Musik verpöhn im Quakertum. Sowohl in der Liturgie
 als auch in der weltlichen verwendung. In der Litugie, weil man der Auffassung war
 das Gott nicht mit Musik unterhalten werden muss oder will, sondern sich offenbaren
 will. Also sollte man sich durch Musik nicht vom $\to$\textit{Inneren Licht}
 ablenken lassen. Weltliche Musik wurde als dekadent und überflüssig abgelehnt.
 Bekannt ist der Fall von $\to$\textit{Solomon Eccles}, der seine Musikintromente
 verkaufte, nach dem er zum Quakertum konvertiert war. Nach einer Zeit hatte er
 solche Gewissennöte, das er die Musikinstromente wieder zurückkaufte und
 verbrante. Er konnte den Gedanken nicht ertragen, das er mit dem Verkauf der
 Instromente Andere zu einem dekadenten Leben verteitete. Diese Einstellung
 änderte sich in einigen Flügeln nach den Schismen des 19. Jahrhunderts.
 $\to$\textit{Evangelical Friends} begannen irgend wann wieder an im
 Gottesdienst zu singen und im $\to$\textit{Liberales Quakertum} wurde irgend
 wann das weltliche Musizeren (also auserhalb der Liturgie) erlaubt und
 betrieben. Heute dürfte wohl die grösste Mehrheit der Quaker nichts mehr
 gegen Musik ein zu wenden haben.

 \end{description}

\normalsize
