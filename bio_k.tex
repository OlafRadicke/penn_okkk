\section*{K}

\articlesize

\begin{description}

\item[Kappes, Heinz]

 \item[Keith, George] $\ast$1638 oder $\ast$1639, \dag27.3.1716. Schottischer Quaker-Missionar. Bekannt durch die \textit{"`Keith-Kontroverse"'}. Zusammen mit den damals angesehensten Quakern missionierte er in Deutschland und den Niederlanden. In der Zeit um 1691, als er in den Kolonien (den heutigen USA) lebte, begann eine Auseinandersetzung um seine Person. Er vertrat die Auffassung, die Gemeinschaft hätte sich zusehr von den Christlichen Wurzeln entfernt. Zuerst kam es zum Bruch mit dem Philadelphia Yearly Meeting und später mit dem London Yearly Meeting. Jahre später konvertierte er zur Anglikanische Kirche und wurde dort Priester.


 \item[Klassen, Hans] $\ast$30.9.1893 \dag18.9.1959 Fällt durch seinen sein bizarrer Lebensweg etwas aus dem Ramen. Klassen war Rußlandmennonit, Lebensreformer und Bürokrat, Nationalsozialist und Kommunist und Quäker.\endnote{Mennonitischen Geschichtsblätter 2009, 66. Jahrgang. von Claus Bernet.}

 \item[Kraus Hertha]



 \end{description}
\normalsize
