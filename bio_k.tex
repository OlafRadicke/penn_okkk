\section*{K}

\articlesize

\begin{description}

\item[Kappes, Heinz]

 \item[Keith, George] $\ast$1638 oder $\ast$1639, \dag27.3.1716. Schottischer Quaker-Missionar. Bekannt durch die \textit{"`Keith-Kontroverse"'}. Zusammen mit den damals angesehensten Quakern missionierte er in Deutschland und den Niederlanden. In der Zeit um 1691, als er in den Kolonien (den heutigen USA) lebte, begann eine Auseinandersetzung um seine Person. Er vertrat die Auffassung, die Gemeinschaft hätte sich zusehr von den Christlichen Wurzeln entfernt. Zuerst kam es zum Bruch mit dem Philadelphia Yearly Meeting und später mit dem London Yearly Meeting. Jahre später konvertierte er zur Anglikanische Kirche und wurde dort Priester.

 \item[Keith-Kontroverse] $\to$\textit{George Keith}.

 \item[Klassen, Hans] $\ast$30.9.1893 \dag18.9.1959 Fällt durch seinen sein bizarrer Lebensweg etwas aus dem Ramen. Klassen war Rußlandmennonit, Lebensreformer und Bürokrat, Nationalsozialist und Kommunist und Quäker.\endnote{Mennonitischen Geschichtsblätter 2009, 66. Jahrgang. von Claus Bernet.}

 \item[Kraus Hertha]

 \item[Kryptoquäker] Von griechisch "`krypto"': geheim, verborgen. Ein Quäker, der sich nicht öffentlich als Quäker zu erkennen gibt, also ein "`Geheimquäker"'. Das gab es gelegentlich im 17. Jahrhundert, zu Zeiten schwerer Verfolgung, als nicht alle Quäker den Mut hatten, zu ihrer Gemeinschaft zu stehen. Verdächtigen Personen, die irgendwie als Quäker erschienen, wurde auch vorgeworfen, "`Kryptoquäker"' zu sein.\endnote{Verwendungsbeispiel: Claus Bernet, 2007, "`Deutsche Quäkerschriften. Band 2: Deutsche Quäkerschriften des 18. Jahrhunderts"', ISBN 9783487134086, ISBN 348713408X, Seite 371}

 \item[Kulturquakertum] $\to$\textit{Wharton, Joseph}

 \end{description}
\normalsize
