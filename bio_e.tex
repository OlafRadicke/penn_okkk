\section*{E}

\articlesize

\begin{description}

 \item[Eccles, Solomon] $\ast$1618 \dag1683, Quaker,  bekannt duch exzentrisches
 Verhalten. Etwar da durch, das er nackt auftrat, um damit zu zeigen, das er
 sich wieder im paradiesisch sündenfreien Urzustand befände (als Strafe dafür
 wurde er ausgepeitscht). Von Eccles ist bekannt das er Musiker, Musiklehrer und
 Musikinstomentebauer war, bevor er 1660 zum Quakertum konvertierte. Zuvor
 war er auch schon mitglied der episkopalischen Kirche, dann bei den
 Presbyterianer, Independenter und den Baptist. Nach seiner konversation zum
 Quakertum, verkaufte er zunächst seine Instumente und Noten, um sie dann wieder
 von seinen Käufern zurück zu kaufen und zu verbrennen. Da er sich um das
 Sehlenheil seiner Käufer sorgen machte. Um sich künftig ernähren zu können,
 erlehnte er den Beruf des Schuhhandwerks, was in den Augen der Quaker eine
 (moralisch) besserer Lebenserweb war. Eccles viel immer wieder durch
 provokatives Verhalten (in Kirchen) auf, was zu einer Reie von Bestrafungen und
 Gefängnisaufenthalten führte.\endnote{Biographisch-Bibliographischen
 Kirchenlexikon, Bautz, Band XX (2002)Spalten 424-427, Claus Bernet, \\
 \texttt{http://www.kirchenlexikon.de/e/eccles\_s.shtml}}


 \item[Elisabeth von der Pfalz] $\ast$26.12.1618 in Heidelberg; \dag8.2.1680
 in Herford. Empfing unter anderem $\to$William Penn und korospondierte mit
 Quaker. Trat aber nie zum Quakertum über auch wenn sie mit William Pann eine
 Freundschaft bis zu ihrem Lebensende verband. Beim englischen König setzte sie
 sich für die Duldung der Quäker ein und in ihnem Haus wurden Andachten und
 Gottesdienste abgehalten.\endnote{ \\
 \texttt{http://www.kirchenlexikon.de/e/elisabeth\_v\_h.shtml }, \\
 Band I (1990)Spalten 1494-1495 Autor: Friedrich Wilhelm Bautz}


 \end{description}

\normalsize
