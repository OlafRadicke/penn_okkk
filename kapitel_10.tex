
\chapter{Kapitel 10} \label{kap10}
\section{Zusammenfassung des 10. Kapitel}


\begin{description}
\item[1. Abschnitt] Noch ein Stück unserer Nichtgleichstellung der Welt besteht
darin, dass wir uns der richtigen Sprache bedienen und zu einer einzelnen Person
\textit{Du} statt \textit{Ihr} oder Sie sagen.
\dotfill \textit{Seite~\pageref{kap10_ab1}}\\
\item[2. Abschnitt] Dieses Stück unserer einfachen Sitte wird durch den rechten
Gebrauch der Wörter und den richtigen Begriff von Singular oder Einheit und
Plural oder Mehrheit gerechtfertigt.
\dotfill \textit{Seite~\pageref{kap10_ab2}}\\
\item[3. Abschnitt] Der Gebrauch zu \textit{einer} Person \textit{Du} zu sagen,
ist bei
allen hebräischen, griechischen und lateinischen Schriftstellern, die auf
Schulen und Universitäten gelesen werden, üblich.
\dotfill \textit{Seite~\pageref{kap10_ab3}}\\
\item[4. Abschnitt] In den Sprachen aller Völker wird dadurch der Unterschied
zwischen einzelnen und mehreren Personen bezeichnet.
\dotfill \textit{Seite~\pageref{kap10_ab4}}\\
\item[5. Abschnitt] Der Ursprung der eingerissenen Missbräuche in den Anreden
der
Personen rechtfertigt unsere Unterlassung derselben.
\dotfill \textit{Seite~\pageref{kap10_ab5}}\\
\item[6. Abschnitt] Wenn jedoch ein alter sinnvoller Gebrauch etwas gelten
soll, so spricht derselbe für uns.
\dotfill \textit{Seite~\pageref{kap10_ab6}}\\
\item[7. Abschnitt] Zu einer Person \textit{Du} zu sagen, kann weder unhöflich
noch
unsitlich sein, da Gott selbst, die Erzväter, die Propheten, Christus und
seine Apostel sich diese Redensart verbieten
\dotfill \textit{Seite~\pageref{kap10_ab7}}\\
\item[8. Abschnitt] Petrus gibt ein Beispiel von einer ungewöhnlichen
Sprache im Palast des Hohenpriesters.
\dotfill \textit{Seite~\pageref{kap10_ab8}}\\
\item[9. Abschnitt] Der Mensch redet in seinem Gebete Gott mit \textit{Du} an;
sein
Stolz verlangt von anderen eine bessere Anrede, als er selbst gegen seinen
Schöpfer gebraucht.
\dotfill \textit{Seite~\pageref{kap10_ab9}}\\
\item[10. Abschnitt] Zeugnisse verschiedener Schriftsteller zu unserer
Rechtfertigung.\\
.\dotfill \textit{Seite~\pageref{kap10_ab10}}\\
\item[11. Abschnitt] Des Verfassers Überzeugung von der Notwendigkeit, sich
der richtigen Sprache zu bedienen. - Seine Ermahnung an den Leser.
\dotfill \textit{Seite~\pageref{kap10_ab11}}

\end{description}

\newpage

\section{1. Abschnitt} \label{kap10_ab1}

Es gibt noch ein Stück, worin wir uns der Welt nicht
gleichstellen\index{Welt!Absondern von} können, und
weshalb man uns schon oft als Leute von schlechter Erziehung betrachtet hat,
dieses besteht nämlich darin, \index{Anrede!Du/Sie}\textbf{dass wir einzelne
Personen, ohne Unterschied des
Standes, in der Rede nicht \textit{Sie} oder \textit{Ihr}, sondern \textit{Du}
nennen.} Eine
Sprache, die einigen so sehr aufzufallen oder so grob vorzukommmen scheint, dass
sie nicht selten ihren Spott oder Unwillen erregt. Da nun dieselbe Hauptursache,
die uns zur Unterlassung der vorhin gedachten weltlichen Gebräuche bewog, auch
in diesem Stück der Beweggrund zu unserer Abweichung von dem allgemeinen
Sprachgebrauc (oder vielmehr von dem allgemein eingerissenen Missbrauch der
Sprache) ist, so werde ich nur noch diejenigen Vernunftgründe anführen, die ich
zu unserer Rechtfertigung für nötig halte, wiewohl sehr wahrscheinlich die
hohen Gedanken, die einige unserer Tadler von sich selbst haben, ihnen
schwerlich erlauben werden, zu glauben, dass es für ein so albernes Benehmen,
als
dieses ihnen zu sein scheint, vernünftige Gründe geben könne.

\section{2. Abschnitt} \label{kap10_ab2}

\index{Sprache!Wesen der} Worte sind, an sich betrachtet, nur Zeichen oder
Mittel, deren die Menschen sich
bedienen, um einander ihre Gedanken und Begriffe mitzuteilen und sich mit
einander zu unterreden. Nun ist zwar die Welt in viele Völkerschaften
eingeteilt, von denen fast jede ihre besondere Sprache und verschiedene
Mundarten hat, allein in dem Begriff von Singular und Plural, oder
Einzahl und Mehrzahl der Dinge und Personen stimmen doch alle, als in
einer Grundregel der Sprache, überein. \textbf{So versteht zum Beispiel jedermann, wenn
man
sagt: \textit{ich liebe, du liebst, er liebt}, dass nur von \textit{einer}
ersten, zweiten
oder dritter Person die Rede ist, und dass hingegen die Worte: \textit{wir
lieben, ihr
liebt, sie lieben,} mehr als eine Person bezeichnen. Diese unleugbare und
unveränderliche Sprachregel sollte billig jeden, der die Anfangsgründe seiner
Sprachlehre nicht vergessen hat, klar überzeugen, dass wir nicht gegen die
Vernunft handeln, wenn wir zu einer Person \textit{Du} sagen.} Denn wenn
\textit{du
liebst} die Einzahl, und \textit{ihr liebt}, oder \textit{sie lieben} die
Mehrzahl
ausdrückt, oder wenn \textit{du liebst} nur eine Person, und \textit{ihr
liebt} oder
\textit{sie lieben} mehrere Personen bezeichnet, und wenn man doch zu einer
Person
\textit{ihr liebt} oder \textit{sie lieben} sagen kann, wird es dann nicht eben
so
richtig sein, zu zehn Menschen \textit{du liebst} zu sagen? Aber warum können
wir
denn nicht auch \textit{ich liebe} statt \textit{wir lieben}, und \textit{wir
lieben} statt
\textit{ich liebe} sagen? Dieses müsste ja ohne Zweifel, so sonderbar und
lächerlich
es auch in der Sprache herauskommen würde, einerlei sein.

\section{3. Abschnitt} \label{kap10_ab3}

Wenn man zweitens zu einer Person \textit{Du} zu sagen als unschicklich oder
unhöflich betrachtet, wie kommt es dann, dass man in Schulen und auf
Universitäten die hebräischen, griechischen und römischen Schriftsteller liest,
die doch keine andere Sprache führen? Oder warum dienen sie uns nicht auch
hierin, wie in anderen Dingen, zur Richtschnur? Und warum findet man es doch bei
uns so lächerlich, dass wir in unseren Ausdrücken uns nach den richtigen
Vorschriften der Sprachlehre richten? Man hält es ja für vernünftig, den
Schülern scharfe Verweise zu geben, wenn sie gegen die Sprachregeln fehlen, und
\textit{ihr} oder \textit{sie} statt \textit{du} setzen, und uns glaubt man mit
Tadel und Verspottung
uberhäufen zu müssen, weil wir eben diese Regeln beachten
\footnote{\texttt{'beobachten' ersetzt durch 'beachten'.}}.

\section{4. Abschnitt} \label{kap10_ab4}

Einzelne Personen \textit{Du} zu nennen, kann, drittens, auch aus dem Grunde
weder
unschicklich noch unhöflich sein, weil es immer in allen Sprachen und auch zu
allen Zeiten üblich war. Dieses liegt klar am Tag, denn als Gott selbst zuerst
mit Adam redete, geschah es in der hebräischen Sprache, in welcher einzelne
Personen \textit{Du} genannt werden. Dasselbe ist der Fall im
Assyrischen\index{Sprache!Assyrisch},
Chaldäischen\index{Sprache!Chaldäisch}, Griechischen\index{Sprache!Griechisch}
und Lateinischen\index{Sprache!Lateinisch}. Auch in unseren Zeiten hat sich
diese Unterscheidung der zweiten Person des Singulars bei den
Türken\index{Personen:!Türken}, Tataren\index{Personen:!Tataren},
Russen\index{Personen:!Russen}, Italienern\index{Personen:!Italiener},
Indianern\index{Personen:!Indianer}, Persern\index{Personen:!Perser},
Spaniern\index{Personen:!Spanier}, Franzosen\index{Personen:!Franzosen},
Holländern\index{Personen:!Holländer},
Deutschen\index{Personen:!Deutschen}, Schweden\index{Personen:!Schweden},
Dänen\index{Personen:!Dänen}, Irländern\index{Personen:!Irländer},
Schottländern\index{Personen:!Schotten}, Wallisern\index{Personen:!Walliser}
und Engländern\index{Personen:!Engländer} erhalten, und das Wort \textit{Du}
ist nicht verloren gegangen,
denn obgleich einige der neueren Sprachen in den meisten Fällen dafür das Wort
\textit{ihr}, oder statt dessen auch \textit{sie} und \textit{er} gebrauchen, so
unterliegen
doch alle dabei einem und demselben Irrtum. \textbf{Hieraus geht aber auch
klar
hervor, dass unser \textit{Du} weder eine Neuerung noch Unschicklichkeit,
sondern
vielmehr in allen Sprachen das einzige schickliche Wort ist, wodurch man
einzelne Personen von mehreren in der Anrede gehörig unterscheiden kann,} und
dass
ohne den Gebrauch dieses Wortes alle Aussprache, Reden und Abhandlungen immer
sehr zweideutig, unbestimmt und zweifelhaft sein würden. Wir wollen zum Beispiel
annehmen, es stünden drei verschiedene Personen vor Gericht, die verschiedener
Verbrechen angeklagt und auch alle drei schuldig wären, und der Richter, indem
er das Urteil spräche, würde sagen: \textit{Ihr seid schuldig und zum Tode
verurteilt},
oder: \textit{Ihr seid unschuldig und freigesprochen}. Wie könnte man wissen,
wer hier
schuldig oder unschuldig wäre? Ob der Richter nur einen, oder zwei, oder
vielleicht alle drei gemeint habe? Darum werden auch unsere (englischen)
Urteilssprüche mit Anwendung der bestimmenden einfachen Zahl abgefasst; zum Beispiel:
\textit{Halte deine Hand in die Höhe! Du bist bei deinem Namen N. N. angeklagt,
dass du
die Furcht Gottes nicht vor Augen gehabt} usw. Dieses Beispiel lässt sich nun
auf alle Unterredungen und Verhandlungen anwenden. Auch sieht man leicht ein,
dass die Missverständnisse, die durch Verwechslung der Einzahl mit der Mehrzahl
unvermeidlich entstehen müssen, nur durch weitläuftige Umschreibung vermieden
werden können. Und da ohne Zweifel die Vermeidung solcher Weitschweifigkeiten
und Dunkelheiten den Gebrauch der bestimmenden einfachen Zahl zuerst notwendig
machte, so kann derselbe auch mit Recht nicht aufgegeben werden, so lange noch
dieselbe Notwendigkeit dafür vorhanden ist, und diese wird immer fortdauern, so
lange es zwei Menschen in der Welt gibt.

\section{5. Abschnitt} \label{kap10_ab5}

\textbf{Dieses ist jedoch noch nicht alles, was ich gegen die Verwechslung der
einfachen Zahl mit der Mehrzahl einzuwenden habe.} Es geschah zuerst aus
Nachahmung der eitlen Huldigungen, welche die Heiden\index{Personen:!Heiden}
ihren Göttern\index{Heidnische!Götter} darbrachten,
dass man diese verkehrte Art zu reden einführte, um dem Stolz der
Päpste\index{Personen:!Päpste} und
Kaiser\index{Personen:!Kaiser, Die} dadurch zu schmeicheln\index{Schmeicheln}.
Man glaubte einzelnen Großen mehr Ehre zu
erweisen, wenn man sie in der Mehrheit anredete, als wenn ein Papst aus mehreren
Göttern und ein Kaiser aus mehreren Menschen bestände. Aus einem solchen Grund
wurden zuerst die Wörter \textit{ihr, euch, euer} oder \textit{eure}, (und
hernach bei
den Deutschen auch \textit{sie, ihnen}, und \textit{ihr} oder \textit{ihre})
welche man sonst
nur von mehreren Personen gebrauchte, auf \textit{einzelne} angewendet. Es
scheint,
als wenn die Wörter: \textit{du, dir, dich} es für diejenigen, die gern ein
größeres
Ansehn haben wollten, als ihnen eigentlich gebührte, zu einfach waren, und die
glänzende Achtung, die sie verlangten, auszudrücken, und dass man daher einen
Stil zu erfinden suchte, der ihrem Ehrgeiz entspräche. Dieses ist aber ein
Grund, auf den wir nicht bauen können, da nur noch solche, als ihn legten,
Vergnügen und Vorteil in seiner Erhaltung zu finden glauben. Gesetzt aber auch,
die Ausdrücke \textit{Ihr, Euer, Sie, Ihre} usw. schickten sich für einen
Fürsten\index{Personen:!Fürsten, Die},
so folgt daraus nicht, dass man sie auch auf gewöhnliche Menschen anwenden
könne.
Denn, wenn zum Beispiel das Edikt \index{Edikt},
eines Fürsten so lautet: \textit{Wir wollen und
befehlen}
etc., so können wir unter dem Worte "`Wir"' verstehen, dass er vielleicht in
Verbindung mit seinen Ministern und Räten so rede, wird hingegen von einer
Privatperson in der Mehrheit geredet, so ist dies unstreitig ein verkehrter
Gebrauch der Worte. So wie jedoch der Stolz ihm seine Entstehung gab, ist auch
die Schmeichelei allzeit geschäftig gewesen, ihn zu verbreiten. In England und
Frankreich pflegte man ehemals die Namen \textit{Monsieur}, \textit{mein
Herr}, und
\textit{Sir, Herr}, nur dem König und seinem Bruder, und die Benennung:
\textit{Madame},
nur ihren Gattinnen beizulegen. Jetzt wird in Frankreich der
Bauer\index{Anrede!des Bauern} hinter dem
Pflug \textit{Monsieur}, und seine Frau \textit{Madame} genannt, und in England
nennt man
jeden Handwerker\index{Anrede!des Handwerkers} \textit{Sir}. d.h. Herr, und
seine
Frau \textit{Mistress}, welches Wort
ebensoviel als das französische \textit{Madame} ausdrückt. So wirksam haben sich
Stolz und
Schmeichelei in allen Zeitaltern bewiesen, indem der erstere immer gern nahm,
was die letztere gern gab, nämlich: sogenannte Achtung und Ehre.

\section{6. Abschnitt} \label{kap10_ab6}

\index{Sitten!sich unterwerfen}Man wird jedoch vielleicht einwenden, es sei
nicht mehr als billig, dass man sich
einem üblichen Gebrauch unterwerfe, und wir handelten demselben gerade
entgegen. Hieraus lässt sich aber leicht und mit mehr Wahrheit erwidern, dass
wir freilich in vernünftigen oder in gleichgültigen und unschädlichen Dingen den
Gebrauche nachgeben können, allein in unvernünftigen und unerlaubten ihnen keine
Herrschaft über uns einräumen dürfen. Der Gebrauch kann weder Zahlen noch
Geschlechter verändern, und ist folglich eben so wenig im Stande, den Begriff
der Mehrheit mit einzelnen Wesen zu vereinigen, als er männliche in weibliche
oder einzelne in tausende verwandeln kann. Soll aber dennoch der Gebrauch über
unser Betragen entscheiden, so kann diese Entscheidung nur zu unseren Gunsten
ausfallen, denn da der Gebrauch selbst nichts anderes als ein altes Herkommen
oder eine alte übliche Sitte ist, so kann ich mich dreist auf den Gebrauch aller
Völker vom Anfang der Welt her berufen, und möge dann die Sitte der Vorzeit
diese Frage entscheiden, ob die gegenwärtige Verwirrung in der Welt, zu einer
einzelnen Person statt \textit{du, ihr} oder \textit{sie} zu sagen, nicht eine
Neuerung
ist. \index{Wörtern!Bedeutung von} Man verstehe mich recht! Ich weiß sehr wohl,
dass Worte nur insofern etwas
gelten, als der Gebrauch, den die Menschen davon machen, ihnen Wert oder Kraft
beilegt. Wenn dann aber die Wörter \textit{ihr} und \textit{sie}, den Gebrauch
des Wortes
\textit{du} verdrängen, oder an dessen Stelle treten sollen, so gebe man uns
doch
wenigstens andere Wörter, deren wir uns statt "`ihr"' und "`sie"' bedienen
können, um
mehrere Personen von einzelnen bestimmt zu unterscheiden. Denn, ein und dasselbe
Wort sowohl für einzelne als auch für mehrere Personen zugleich zu gebrauchen,
wenn zur richtigen Bezeichnung der Einzahl und der Mehrzahl verschiedene Wörter
da sind, und dieses bloß zur Befriedigung des Stolzes und Dünkels eitler
Gemüter zu tun, kann in der Tat nicht vernünftig sein, wenigstens nicht nach
unseren Begriffen, die, wenn sie gleich dem Geist der Mode nicht angemessen
sind, doch, wie wir hoffen, mit der christlichen Religion übereinstimmen.

\section{7. Abschnitt} \label{kap10_ab7}

Wollte man auch noch sagen, eine einzelene Person mit \textit{Du} anzureden, sei
unhöflich oder unschicklich, so hätten Gott selbst, alle
Erzväter\footnote{\texttt{Ein anders Wort für 'Erzväter' ist auch
'Patriarchen'}}
und Propheten,
Christus und seine Apostel, die Heiligen der ersten Jahrhunderte, alle Sprachen
in der ganzen Welt und unsere eigenen Gerichtsbehörden, sämtlich hierin
gefehlt, welches zu denken jedoch, mit Erlaubnis zu reden, große Vermessenheit
sein würde. Übrigens ist es ja auch bei unseren besten Schriftstellern ganz
gebräuchlich, in den Vorreden zu ihren Werken den Leser in der Einzahl mit
\textit{Du} anzureden; zum Beispiel \textit{"`Leser! du wirst ersucht,"'} usw. oder
\textit{"`Leser dieses
soll dir zur Nachricht dienen."'} usw. Ebenso ist es auch eine bekannte
Sache,
dass die berühmtesten Dichter in ihren Zueignungsschriften selbst an hohe
Personen sich dieses Stils bedienen, wie wir bei
Chaucer\index{Personen:!Chaucer}, Spencer\index{Personen:!Spencer},
Waller\index{Personen:!Waller},
Cowley\index{Personen:!Cowley}, Dryden\index{Personen:!Dryden}, u.a. nachlesen
können. Und warum will man denn nun ein solches
Benehmen bei uns für so unhöflich, ungegebildet und unerträglich halten? Ich bin
überzeugt, dass man diese Frage nie wird recht beantworten können.

\section{8. Abschnitt} \label{kap10_ab8}

\index{Sprache!Eigenart der Jünger}Ich zweifle gar nicht, dass die Sprache Jesu
und seiner Jünger gleichfalls etwas
Sonderbares und Auffallendes gehabt habe, denn man warf, wie bekannt ist, dem
Petrus\index{Verleumdung!des Petrus}, als er im Palast des Hohenpriesters
seinen Herrn verleugnete, seine
Sprache als einen Beweis vor, dass er Jesu angehören müsse.
\\textbf{\textit{"`Wahrlich,"'} sagten
die Umstehenden,}\textit{"`du bist auch einer von ihnen, denn deine Sprache
verrät
dich."'}\footnote{Matthäus 26,37}
\index{Bibelstellen:!Matthäus 26)}
Sie glaubten kurz zuvor schon aus seinem äußeren
zu erraten, dass er mit Jesu gewesen wäre; als sie ihn aber reden hörten, setzte
seine Sprache sie deshalb außer allen Zweifel. Jetzt wussten sie es gewiss, dass
er
einer von denen war, die mit Jesu gewandelt waren. Petrus musste allerdings in
dem Umgang mit Jesu etwas angenommen haben, das sonderbar und auffallend war,
und ganz gegen das Benehmen der Welt abstach. \label{ref:10_08_sprache}
\textbf{Ohne Zweifel unterschieden sich
die \index{Welt!Absonderung} Nachfolger Jesu von der Welt sowohl in ihrer
Tracht, Haltung und Sprache,
als in seiner Lehre, die sie zu dieser Unterscheidung anleitete und es lässt sich
leicht denken, dass sie \index{Zeugnis!Einfachheit}einfacher,
ernsthafter\index{Zeugnis!Wahrhaftigkeit} und mehr genaunehmend als andere
waren.} Dieses gewinnt auch sehr an Wahrscheinlichkeit, wenn wir erwägen,
\textbf{was für
ein Mittel der arme, von seinem Selbstvertrauen betrogene Petrus in seiner
Furcht ergriff, um den anderen ihre Gedanken von ihm zu benehmen, denn
\textit{"`er fing
an zu fluchen und zu schwören."'} Ein trauriger Behelf! Er glaubte aber
vielleicht, das sicherste Mittel, allen Verdacht zu entfernen, würde das sein,
wenn er etwas täte, was mit Christo und seiner Lehre am wenigsten
übereinstimmte.} Der Kunstgriff gelang auch, er brachte sie mit ihren
Einwendungen zum Schweigen und Petrus war nun für ebenso
rechtgläubig\index{rechtgläubig} als
einer von ihnen gehalten. Auf diese Weise entging Petrus nun zwar den
Nachforschungen der Menschen, aber doch nicht dem Hahnengeschrei, das in seine
Ohren drang, und ihn an die Worte seines geliebten leidenden Herrn erinnerte.
\textit{"`Er ging hinaus und weinte bitterlich,"'} dass er seinen Meister
verleugnet
hatte, der nun überliefert war, auch für ihn in den Tod zu gehen.

\section{9. Abschnitt} \label{kap10_ab9}

\label{ref:10_08_duzen} \textbf{Der letzte Grund, den ich zur Rechtfertigung
unserer Sitte, einzelne Personen
nur in der einfachen Zahl anzureden, noch anzuführen habe, hat, meiner Ansicht
nach, das größte Gewicht, und da er allen Menschen einleuchtet, so werden unsere
Tadler am wenigsten etwas dagegen vorbringen können. Er besteht darin, dass man
es uns nicht zumuten müsse, einem Gebrauch nachzugehen, der gerade den
höchsten Grad des Stolzes sterblicher Menschen darin beweist, dass sie ihren
Mitmenschen eine bessere, anständigere oder höflichere Sprache verlangen oder
erwarten, als sie selbst gegen den unsterblichen Gott, ihren großen Schöpfer,
gebrauchen, wenn sie ihm Verehrung und Anbetung leisten. \index{Gott!mit "`du"'
ansprechen} Bist du, o Mensch, denn
größer als er, der dich erschaffet hat? Kannst du den Gott, der dir den Atem
gab, den großen Richter aller Handlungen deines Lebens, mit \textit{du}
anreden, und
sobald du dich von deinen Knien erhoben hast, einen Mitchristen beleidigen, weil
er dich Erdenwurm\index{Erdenwurm} mit eben der Sprache anredet, worin du so
eben zu deinem Gott
geredet hast?} Ist dieses nicht eine Anmaßung ohne gleichen? -- Zu jemand "`du"'
zu
sagen, ist aber entweder ein Zeichen von zu viel oder zu wenig Achtung. Ist
es zu viel Achtung, die wir dadurch beweisen, so werde darüber nicht zornig und
mache uns deshalb keine Vorwürfe, sondern lehne es mit Ernst und Demut von
dir ab. Scheint es dir aber zu wenig auszudrücken, warum erzeigst du denn Gott
keine größere Achtung? O! Wohin hat doch der Mensch sich verstiegen! Zu welchem
Gipfel will sein Stolz sich erschwingen? Er verlangt größere Achtung von seinen
Mitmenschen, als er selbst Gott erweist. Heißt das nicht für mehr als einen
Gott gehalten sein wollen? Indessen dürfte es ihm unter und ebensosehr an
Anbetern fehlen, als es ihm an der Göttlichkeit mangelt, die der Anbetung würdig
ist. Wir sind völlig überzeugt, dass der Geist Gottes niemand anleitet, Ehre von
Menschen zu suchen, und noch weniger, dieselbe zu verteidigen, oder auf
diejenigen zu zürnen, die, aus Gewissenhaftigkeit gegen Gott, anderen keine
weltliche Ehrenbezeigungen erweisen dürfen. Und es liegt auch klar am Tage, dass
nur die eitlen Gemüter des gegenwärtigen Geschlechts um ihren Hochmut zu
befriedigen, des Gebrauchs derselben sich schuldig machen. Welche Verstellung,
was für ein Kriechen\index{Kriechen} und Schmiegen sieht man nicht täglich! Ja,
wie viele
unnüze nichtssagende Worte und höchst übertriebene Ausdrücke, leere Komplimente,
grobe Schmeicheleien und offenbare Lügen werden nicht beständig unter dem Namen
von Höflichkeitsbezeigungen, von männlichen und weiblichen Personen in ihrem
Umgang gebraucht! O! Meine Freunde! Woher nehmt ihr die Beispiele zu einem
solchen Betragen? Welche Stellen aus den Schriften der heiligen Männer Gottes
können solche Dinge rechtfertigen? Aber ich muss euch noch näher kommen, und
euch
euer eigenes Bekenntnis vorhalten. -- Dient euch Christus, zu dessen Namen ihr
euch bekennt, hierin zum Muster?\index{Jesus!als Vorbild} Oder richtet ihr euch
nach jenen Heiligen der
Vorzeit, die in Einöden wohnten, und
\textit{"`deren die Welt nicht wert war?"'}\footnote{Hebräer 11,38}
\index{Bibelstellen:!Hebräer 11)}
Oder glaubt ihr in den Fußstapfen jener Christen zu
wandeln, die, aus Gehorsam gegen die Lehre und nach dem Vorbild der Lebens
ihres Meisters, dem Ansehn der Person entsagten und die Moden und Gebräuche,
die Ehre und Herrlichkeit dieser vergänglichen Welt verließen? Der Christen, die
sich nicht durch äußere Gebärden, Höflichkeitsbezeigungen, Komplimente etc.,
sondern durch
\textit{"`einen stillen und sanften Geist"'}\footnote{1. Petrus 3,4}
\index{Bibelstellen:!Petrus 1 03@1. Petrus 3)}
auszeichneten, der mit\index{Tugend!der Anhänger} Mäßigkeit, Tugend,
Bescheidenheit, Ernst, Geduld und
brüderlicher Liebe geschmückt war? Denn darin bestanden in jenen Zeiten des
Christentumes die wahren Ehrenzeichen und einzigen Merkmale der Würde und des
Adels der Christen. Und sehen wir uns nicht eben darum, weil wir ihnen und nicht
der Welt in ihren Gebräuchen nachahmen, von anderen verachtet und verhöhnt? Sagt
uns doch aufrichtig, machen nicht Romane\index{Romane},
Schauspiele\index{Schauspiel}, Maskenbälle\index{Maskenball},
Spielpartien\index{Spiele}, Konzerte\index{Konzert} und dergleichen euere
Lieblingsunterhaltungen aus? \label{ref:10_08_zeitvertreib} \textbf{Hättet ihr
wirklich den Geist des wahren Christentums\index{Christentum!wahres}, wie
könnet ihr denn euere so
kostbare und kurze Zeit mit so vielen unnötigen Besuchen\index{Besuche},
Spielen und
Zeitvertreiben\index{Zeitvertreib}, mit Komplimentenmachens und Schmeicheleien
verbringen? Wie
könnet ihr euch mit Erzählungen erdichteter
Geschichten\index{Geschichten!erdichtete}, mit Herumtragen
nutzloser Neuigkeiten\index{Neuigkeiten, nutzlose} und noch vielen anderen
eitlen
Dingen beschäftigten, die
bloß dazu da sind, und deren ihr euch auch nur bedient, um euch zu
zerstreuen\index{Zerstreuung};
um nicht an eueren wahren Zustand zu denken, und euch in gänzlicher
Gottesvergessenheit\index{Gott!Gottesvergessenheit} zu
betäuben\index{betäuben}.} Solche Unterhaltungen und Ergötzlichkeiten
waren gewiss nie unter den wahren Christen\index{Personen:!Christen!wahre},
sondern nur unter den Heiden\index{Personen:!Heiden}, die
Gott nicht kannten, üblich. Ach! Hättet ihr doch ein wahres Gefühl von euerem
sündhaften Zustand\index{sündhafter Zustand}, und wärt ihr nur in einigem
Grade neu geboren\index{neugeboren}! Möchtet ihr
doch das Kreuz Christi aufnehmen\index{Kreuz!Kreuz Christi} und unter seiner
Herrschaft\index{Herrschaft!Christi} leben! Dann wurden
diese Dinge, die eurer verderbten\index{Natur!verderbte} und sinnlichen
Natur\index{Natur!sinnliche} so sehr schmeicheln,
keinen Raum in eueren Herzen mehr finden. Das heißt nicht:
\textit{"`suchen was droben ist,"'}\footnote{Kolosser 3,1}
\index{Bibelstellen:!Kolosser 03@Kolosser 3)}
wenn man mit seinem Herzen an den niedrigen Dingen der Welt hängt. Das
heißt nicht:
\textit{"`seine Seligkeit mit Furcht und Zittern schaffen,"'}\index{Furcht!und
Zittern} wenn man
seine kostbare Zeit mit eitlen Dingen vertrödelt. Dann kann man nicht mit
Elihu\index{Personen:!Elihu}
ausrufen:
\textit{"`Ich will niemands Person ansehen und keinen Menschen rühmen, (oder
ihm schmeichelhafte Titel geben,) denn ich weiß nicht, wenn ich es täte, ob
mein Schöpfer mich nicht bald hinwegnehmen würde."'}
Nein, das heißt nicht: sich
\textit{"`selbst überwinden\footnote{\texttt{'verleugnen' ersetzt durch
'überwinden'}}, unvergängliche
Schätze sammeln und nach einem
unverweltlichten Erdteil im Himmel trachten."'} -- Und nun, meine Freunde, was
ihr auch davon denken mögt, so muss ich euch sagen, dass die Entschuldigung:
\textbf{euch
auf den allgemeinen Gebrauch zu berufen, vor Gottes Richtstuhl nicht gelten
wird.} \textit{Das Licht Christi}, das in eueren eigenen Herzen
scheint\index{Licht!inneres}, wird Sie
verwerfen, und dann wird der Geist, gegen den wir zeugen, so erscheinen, wie er
ist, und wie wir ihn geschildert haben. Sagt nicht, dass ich um Kleinigkeiten
eifere, hütet euch lieber selbst vor Leichtsinn und Unbedachtsamkeit in
ernsthaften Dingen.

\section{10. Abschnitt} \label{kap10_ab10}

Ehe ich dieses Kapitel schließe, will ich noch einige Zeugnisse allgemein
geachteter Männer zu Gunsten unserer Nichtgleichstellung der Welt in ihren
verkehrten Gebrauche der Sprache hier beifügen.

\medskip

\textbf{Luther\index{Personen:!Luther}, dieser große Reformator, dessen
Aussprüche man in seinem Zeitalter
wie Orakelsprüche betrachtete, und der auch noch heutigen Tags bei vielen
unserer Gegner\index{Personen:!Quaker-Gegner} in großem Ansehen steht,} Luther
war so weit entfernt, unsere
einfache Sprechart zu tadeln, dass er vielmehr in einem seiner Werke,
Ludus\index{Ludus (Buch)}, (das
Spiel,) betitelt, über den Gebrauch, einzelne Personen in der Mehrzahl
anzureden, als über eine unschickliche und lächerliche Sache sich lustig macht,
wo er nämlich sagt: \texttt{Magister! vos estis iratus}; \textit{"`Magister! Ihr
seid
unwillig"';} welches im Lateinischen\index{Lateinisch} ebenso abgeschmackt
herauskommt, als es in
jeder anderen Sprache lauten würde, wenn man sagte:
\textit{"`Meine Herren! Du bist unwillig."'} --
Erasmus\index{Personen:!Erasmus}, ein großer Gelehrter und ein so tiefer
Sprachforscher,
dass ich keinen wüsste, auf den man sich, hinsichtlich der Sprachrichtigkeit
eines
Ausdrucks, mit mehr Befugnis berufen könnte, stellte nicht nur jenen Gebrauch
von einzelnen Personen in der Mehrzahl zu reden, in ein lächerliches Licht,
sondern schrieb auch eine eigene Abhandlung über die Ungereimtheit desselben,
worin er deutlich zeigt, dass es unmöglich sei, den Unterschied zwischen der
Einheit und Mehrheit gehörig in Acht zu nehmen, wenn man ein Wort, das bloß dazu
da ist, die Mehrheit zu beizeichnen, auf einzelne Gegenstände anwendet. Auch
sagt er noch, dass diese Sprachverwirrung aus der Schmeichelei der Menschen
entsprungen sein. -- Lipsius\index{Personen:!Lipsius} versichert von den alten
Römern, dass die jetzige Art
der Begrüßung bei ihnen nicht üblich war. Und endlich gibt uns
Howel\index{Personen:!Howel} in seiner
Geschichte Frankreichs eine treffende Erörterung von dem Ursprunge des
Gebrauchs, einzelne Personen in der Mehrzahl anzureden, indem er uns versichert,
dass vor alten Zeiten die Bauern\index{Personen:!Bauern} ihre
Könige\index{Personen:!Könige} büßten, Stolz und Schmeichelei aber
zuerst die Untergeordneten bewogen, den einzelnen Personen ihrer Oberen
Ehrenbezeigungen in der mehrfachen Zahl zu erweisen, und die Oberen geneigt
machten, diese anzunehmen? Könnten wir nun auch, zur Rechtfertigung unseres
Gebrauchs der einfachen und richtigen Sprache, uns nicht auf die unverwerflichen
Beispiele Gottes und guter Menschen berufen, so würden wir dennoch, da wir
überzeugt sind, dass Stolz und Schmeichelei den jezt üblichen Missbrauch
derselben
einführten, schon aus Gewissenhaftigkeit uns nicht darein fügen können. Und so
sehr uns auch die Ungezügelten und Leichtsinnigen unsers Zeitalters, deren
Gemüter so unaufhörlich von der Liebe zu weltlichen Vergnügungen umgetrieben
werden, dass sie den wahren Ursprung und Zweck der Worte und Dinge zu prüfen und
zu unterscheiden nicht im Stande sind, -- so sehr diese uns auch als sonderbare
Menschen tadeln mögen; so können wir dennoch, die wir durch Gottes Licht und
Geist von der Torheit und schädlichen Wirkung solcher weltlichen Gebräuche
überzeugt, und zu einer klaren Einsicht und geistlichen Unterscheidung ihres
Ursprungs und ihrer Eigenschaften gelangt sind, diese Dinge nicht anders als
Früchte des Stolzes und der Schmeichelei erkennen, und deswegen und auch darin
nach dem Verlangen irdisch gesinnter Gemüter nicht mehr bequemen. Wir möchten
sonst unseren Gott beleidigen und unsere Gewissen mit Schuld beladen. Denn da
wir
durch die innern Züchtigungen der göttlichen Gnade\index{Gott!göttliche Gnade}
aufrichtig gerührt und zur
aufmerksamen Unterwerfung unter das heilige Gesetz Jesu in unseren
Herzen\index{Gebote!Gesetz, im Herzen}
gebracht worden sind, so dass wir
\textit{"`unsere Werke an das Licht bringen, um zu
sehen, ob sie in Gott getan sind oder nicht,"'}\footnote{Johannes 3,13 und
20+21}
\index{Bibelstellen:!Johannes 03@Johannes 3)}
so können und dürfen wir uns der Welt, die mit ihrer Lust vergeht, in ihrem
eitlen
Wesen nicht mehr gleichstellen: indem wir gewiss wissen,
\textit{"`dass die Menschen am
Tage des Gerichts von jedem unnützen Wort, das sie geredet haben, Rechenschaft
geben müssen."'}\footnote{Matthäus 12,36}
\index{Bibelstellen:!Matthäus 12)}

\section{11. Abschnitt} \label{kap10_ab11}

Darum, o Leser, du magst nun ein in der Nacht sich zu Jesu schleichender
Nicodemus\index{Personen:!Nicodemus}, oder ein ihn verhöhnender Schriftgelehrter
sein, nämlich einer, der
den glorreichen Messias auch gern besuchte, aber doch lieber von den finstern
Gebräuchen der Welt bedeckt zu ihm käme, damit du unerkannt durchgehen und der
Schmach seines Kreuzes ausweichen könntest oder ein Begünstiger und Verteidiger
des Hamanschen \index{Personen:!Haman}
Stolzes, und hältst vielleicht diese hier abgelegten
Zeugnisse\index{Zeugnis}
nur für alberne Sonderbarkeiten: \textbf{so muss ich dir sagen, dass göttliche
Liebe mich
verpflichtet, dir die Wahrheit zu verkündigen und ein getreues
Zeugnis\index{Zeugnis} gegen das
ungöttliche Wesen der entarteten Welt,} so wie in anderen, auch in diesen
Stücken
\textbf{bei dir abzulegen,} in welchen der Geist der Eitelkeit und sinnlichen
Begierden
eine so große Macht gewonnen und so lange unbeschränkt geherrscht hat, dass er
Unverschämtheit genug besitzt, seine Finsternis Licht zu nennen, und den
Früchten seines verderbten Baumes Namen beizulegen, die nur Erzeugnissen von
einer edleren Art gebühren, um dadurch die Menschen desto leichter zu täuschen
und für den Gebrauch derselben zu gewinnen. Und wahrlich, die meisten sind,
leider, so verblendet, und so kalt geworden, dass sie gar nicht wissen,
welches Geistes sie sind, und haben so niedrige oder so irrige Begriffe von dem
demütigen Leben der Selbstüberwindung\footnote{\texttt{'Selbstverleugnung'
ersetzt durch 'Selbstüberwindung'}} und
von der Verbindlichkeit der Lehre des
heiligen Jesu, dass sie einander \textit{Rabbi}, das ist so viel als
\textit{Meister,
Herr, gnädiger Herr, Eure Gnaden,} etc. nennen, dass sie Verbeugungen vor
einander machen, die ich als Anbetung ihrer Person betrachte, dass sie aus
Schmeichelei einander schöne Titel beilegen, um ihren Ehrgeiz zu befriedigen und
zu nähren, dass sie sich zu wenig geehrt oder beleidigt finden, wenn sie in
Ausdrücken angeredet werden, deren sie sich selbst gegen ihren Schöpfer
bedienen, und dass sie endlich ihre Zeit und ihr Vermögen verschwenden, um ihre
sinnlichen Begierden zu befriedigen, indem sie sich den Sitten und Gebräuchen
der Heiden ergeben, die Gott nicht kannten, und ihre Torheiten für
Höflichkeiten, Erziehung, Anstand, Bildung usw. halten. O! Möchtest
du doch, da es nur einen guten und einen bösen Geist gibt, ernstlich erwägen,
welcher von beiden es ist, der die Welt zu solchen Dingen anleitet, ob Nicodemus
oder Mardochai\index{Personen:!Mardochai} dich gegen die verachteten Christen in
deinem Innern geneigt
macht, und welcher von ihnen die Furcht und Scham einflößt, demjenigen in deinem
Umgang mit der Welt öffentlich zu entsagen, was das wahre Licht dir als
Eitelkeit und Sünde im Verborgenen deines Herzens\index{Licht!inneres} zu
erkennen gegeben hat. Oder
wenn du zu unseren Verächtern gehörst, so sage mir, ich bitte dich, wem glaubst
du mit deinen Spöttereien, mit deinem Unwillen und mit deiner Verachtung am
ähnlichsten zu sein: dem stolzen Haman \index{Personen:!Haman}
 oder dem guten Mardochai? Wisse, mein
Freund, dass vielleicht kein Mensch diese Eitelkeiten, die man Höflichkeiten
nennt, mehr geliebt und verschwenderischer angewendet hat, als ich, und hätte
ich mein Gewissen unter den Zeremonien der Welt verbergen können, so wäre ich
gewiss manchen Stürmen von Vorwürfen entgangen, die, meines offenen
Bekenntnisses\index{Bekenntnis}
wegen, oft heftig über mich ausbrechen. Aber dann würde ich auch, wenn ich, nach
den weltlichen Sitten\index{Sitten!weltlichen} und Gebräuchen mich bequemt
hätte, wider meinen Gott
gesündigt und den Frieden meiner Seele\index{Seele!frieden der} verloren haben.

\medskip

Glaube jedoch nicht, Freund, dass wir um der bloßen Titel oder um des nackten
Wortes \textit{Du} Willen solche Schwierigkeiten machen, oder die Absicht haben,
neue
Formen einzuführen, die mit der Aufrichtigkeit und wahren Höflichkeit
unverträglich sind, denn es gibt ja deren, leider, auch schon zu viele. Nein!
Der Wert, die eitle Gemüter auf jene weltliche Zeremonien legen, die hohe
Meinung, die sie davon haben, und die Notwendigkeit, dass ihnen diese
schädlichen Dinge zu Gesicht gebracht werden, damit sie davon gereinigt und
befreit werden können, dieses sind die Beweggründe, die uns zwingen, ein
standhaftes Zeugnis\index{Zeugnis} dagegen an den Tag zu legen. Und wir können
dir aus der vom
heiligen Geist Gottes uns verliehenen Erkenntnis bezeugen, dass dasjenige im
Menschen, welches die Beachtung\footnote{\texttt{'Beobachtung'ersetzt durch
'Beachtung'}} solcher weltlichen
Gebräuche verlangt, das
eine Furcht in ihm erzeugt, wenn er sich davon losreißen will, oder sie zu
verteidigen und zu rechtfertigen sucht, und unzufrieden ist, wenn sie nicht
beobachtet werden, dass alles dieses im Grunde nichts anderes als Wirkungen des
Geistes der Schmeichelei und des Stolzes sind, obgleich bei einigen der öftere
Gebrauch oder die Gewohnheit eine gewisse Gleichgültigkeit dagegen erzeugt, und
bei anderen der Edelmut die Triebfedern des Stolzes geschwächt haben mag. Und
da dieses in dem himmlischen Licht erkannt wird, welches jetzt in den Herzen
der verachteten Christen, in deren Gemeinschaft ich lebe, mit Klarheit scheint,
so finden sie sich oft dadurch bewogen, gegen die \textit{"`unfruchtbaren Werke
der
Finsternis"'}\index{Finsternis} öffentlich zu zeugen, so wie auch ich, als einer
von den ihrigen,
und in ihrem Namen, dieses Zeugnis hier ablege, um vornehmlich die Treulosen und
Wankelmütigen, die, wiewohl sie eines Besseren überzeugt sind, doch gern noch
unbemerkt fortwandeln möchten, zu bestrafen und anzuspornen, und die Heftigkeit
unserer Tadler, die uns als ein affektiertes, sonderbares Volk geringachten,
einigermaßen zu mildern. Denn der ewige Gott, der sich unter uns mächtig
erwiesen hat, und ausgegangen ist, den Bewohnern der Erde seine Macht kund zu
tun,
\textit{"`wird jede Pflanze ausrotten, die nicht von seiner Hand gepflanzt
ist."'}\footnote{Matthäus 15,13}
\index{Bibelstellen:!Matthäus 15)}

\medskip

Darum, mein Leser, lass mich dich bitten, die dir hier vorgelegten Gründe wohl
zu
erwägen: Sie wurden nur größtenteils von dem Herrn zu der Zeit gegeben, da man
meine Einwilligung, in den Sitten und Gebräuchen der Welt zu bleiben, fast um
jeden Preis gern erkauft haben würde. Allein die gewisse Überzeugung, die ich
hatte, dass sie mit dem demütigen Leben der Selbstverleugnung des heiligen Jesu
im Widerspruche stehen, gebot mir, ihnen gänzlich zu entsagen, und ein getreues
Zeugnis dagegen abzulegen\index{Zeugnis!ablegen}. Ich rede die Wahrheit in
Christo, und lüge nicht! Ich
würde mich dem Tadel und der Verachtung anderer nicht ausgesetzt haben, wenn ich
mit Frieden des Gewissens unter einem weltlichen Betragen meinen Glauben hätte
bewahren können. Es fiel mir in der Tat schwer, mich auf solche Weise
auszuzeichnen und so sonderbar zu erscheinen, allein meine Gewissen, mir
wiederholt gegebene Überzeugung, dass Stolz, Eigenliebe und Schmeichelei die
Grundursachen dieser eitlen Gebräuche sind, erlaubte mir nicht, so böse
Eigenschaften noch länger in mir selbst und in anderen zu nähren. Aus diesem
Grund bin ich so ernstlich bemüht, meinen Lesern in Ansehung ihrer
Beurteilung unseres Betragens Vorsichtigkeit zu empfehlen, und ich wiederhole
daher meine Bitte, dass sie bei sich selbst ernstlich erwägen wollen, ob es der
Geist der Welt oder der Geist unseres himmlischen Vaters ist, der über unser
ehrliches, gerades und harmloses \textit{Du} sich entrüstet, damit so jede
Pflanze,
die Gott nicht in die Herzen der Söhne und Töchter der Menschen gepflanzt hat,
ausgerottet werden möge.

\label{kap10_ende}


