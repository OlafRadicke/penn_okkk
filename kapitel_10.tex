% STATUS=Unfertig
% FUSSNOTEN=2.Durchlauf
% ANFÜHRUNGSZEICHEN=ok
% KORSIVSCHRIFT=ok
% FETTSCHRIFT=ok
% RANDBEMERKUNGEN=weg
% MODERNE_RECHTSCHREIBUNG=noch nicht begonnen
% WORTERSÄTZUNG=ok
% VERSCHLAGWORTUNG(INDEX)=ok
% REFERENZIERUNG=abgeschlossen
% BIBELSTELLEN=ungeprüft


\chapter{10. Kapitel} \label{kap10}


\section{Zusammenfassung des 10. Kapitel}

\footnotesize
\begin{description}
\item[1. Abschnitt] Noch ein Stück unserer Nichtgleichstellung der Welt Bestehet
darin, daß wir uns der richtigen Sprache bedienen, und zu einer einzelnen Person
\textit{Du} Statt \textit{Ihr} oder Sie sagen. (Seite: \pageref{kap10_ab1})
\item[2. Abschnitt] Dieses Stück unserer einfachen Sitte wird durch den rechten
Gebrauch der Wörter und den richtigen Begriff von Singular oder Einheit und
Plural oder Mehrheit gerechtfertigt. (Seite: \pageref{kap10_ab2})
\item[3. Abschnitt] Der Gebrauch: zu \textit{einer} Person \textit{Du} zu sagen, ist bei
allen ebräischen, griechischen und lateinischen Schriftstellern, die auf
Schulen und Universitäten gelesen werden, üblich. (Seite: \pageref{kap10_ab3})
\item[4. Abschnitt] In den Sprachen aller Volker wird dadurch der unterschied
zwischen einzelnen und mehreren Personen bezeichnet. (Seite: \pageref{kap10_ab4})
\item[5. Abschnitt] Der Ursprung der eingerissenen Mißbräuche in den Anreden der
Personen rechtfertigt unsere Unterlassung derselben. (Seite: \pageref{kap10_ab5})
\item[6. Abschnitt] Wenn jedoch ein alter bekömmlicher Gebrauch etwas gelten
soll, so spricht derselbe für uns. (Seite: \pageref{kap10_ab6})
\item[7. Abschnitt] Zu einer Person \textit{Du} zu sagen, kann weder unhöflich noch
unsitlich sein; da Gott selbst, die Erzväter, die Prophelen, Christus und
seine Apostel sich dieser Redensart verbieten (Seite: \pageref{kap10_ab7})
\item[8. Abschnitt] Petrus giebt ein Beispiel von einer ungewöhnlichen
Sprache im Palaste des Hohenpristers. (Seite: \pageref{kap10_ab8})
\item[9. Abschnitt] Der Mensch redet in seinem Gebete Gott mit \textit{Du} an; sein
Stotz verlangt von Anders: eine bessere Anrede, als er selbst gegen seinen
Schöpfer gebraucht. (Seite: \pageref{kap10_ab9})
\item[10. Abschnitt] Zeugnisse verschiedener Schriftsteller zu unserer
Rechtfertigung. (Seite: \pageref{kap10_ab10})
\item[11. Abschnitt] Des Verfassers Überzeugung von der Nothwengigkeit, sich
der richtigen Sprache zu bedienen. - Seine Ermahnung an den Leser. (Seite: \pageref{kap10_ab11})

\end{description}
\normalsize

\section{1. Abschnitt} \label{kap10_ab1}

Es giebt noch ein Stück, worin wir uns der Welt nicht gleichstellen\index{Welt, Absondern von} können, und
weshalb man uns schon oft als Leute von schlechter Erziehung betrachtet hat;
dieses bestehet nämlich darin, \index{Anrede Du/Sie}\textbf{daß wir einzelne Personen, ohne Unterschied des
Standes, in der Rede nicht \textit{Sie} oder \textit{Ihr}, sondern \textit{Du} nennen.} Eine
Sprache, die Einigen so sehr aufzufallen oder so grob vorzukommmen scheint, daß
sie nicht selten ihren Spott oder Unwillen erregt. Da nun dieselbe Hauptursache,
die uns zur Unterlassung der vorhin gedachten weltlichen Gedrauche bewog, auch
in diesem Stücke der Beweggrund zu unserer Abweichung von dem allgemeinen
Sprachgedrauche (oder vielmehr von dem allgemein eingerissenen Mißbrauch der
Sprache) ist, so werde ich nur noch diejenigen Vernunftgründe anführen, die ich
zu unserer Rechtfertigung für nöthig halte; wiewohl sehr wahrscheinlich die
hohen Gedanken, die einige unserer Tadler von sich selbst haben, ihnen
schwerlich erlauben werden, zu glauben, das es für ein so albernes Benehmen, als
dieser ihnen zu sein scheint, vernünftige Gründe geben könne.

\section{2. Abschnitt} \label{kap10_ab2}

\index{Sprache, Wesen der} Worte sind, an sich betrachtet, nur Zeichen oder Mittel, deren die Menschen sich
bedienen, um einander ihre Gedanken und Begriffe mitzuteilen, und sich mit
einander zu unterreden. Nun ist zwar die Welt in viele Völkerschaften
eingetheilt, von denen fast jede ihre besondere Sprache und von verschiedene
Mundart hat; allein in dem Begriffe von Singular und Plural, oder
Einzahl und Mehrzahl der Dinge und Personen stimmen doch Alle, als in
einer Grundregel der Sprache, überein. \textbf{So verstehet z.B. Jedermann, wenn man
sagt: \textit{ich liebe, du liebst, er liebt}, daß nur von \textit{einer} ersten, zweiten
oder dritter Person die Rede ist; und daß hingegen die Worte: \textit{wir lieben, ihr
liebet, sie lieben,} mehr als eine Person bezeichnen. Diese unleugbare und
unveränderliche Sprachregel sollte billig Jeden, der die Anfangsgründe seiner
Sprachlehre nicht vergessen hat, klar überzeugen, daß wir nicht gegen die
Vernunft handeln, wenn wir zu einer Person \textit{Du} sagen.} Denn wenn: \textit{du
liebst} die Einzahl, und: \textit{ihr liebet}, oder \textit{sie lieben} die Mehrzahl
ausdrückt, oder wenn: \textit{du liebst} nur eine Person, und: \textit{ihr liebet} oder
\textit{sie lieben} mehrere Personen bezeichnet, und wenn man doch zu einer Person:
\textit{ihr liebet} oder \textit{sie lieben} sagen kann, wird es denn nicht eben so
richtig sein, zu zehn Menschen: \textit{du liebst} zu sagen? Aber warum können wir
denn nicht auch: \textit{ich liebe} statt \textit{wir lieben}, und: \textit{wir lieben} statt
\textit{ich liebe} sagen? Dieses müßte ja ohne Zweifel, so sonderbar und lächerlich
es auch in der Sprache herauskommen würde, einerlei sein.

\section{3. Abschnitt} \label{kap10_ab3}

Wenn man zweitens, zu einer Person \textit{Du} zu sagen, als unschicklich oder
unhöflich betrachtet, wie kommt es denn, daß man in Schulen und aus
Universitäten die ebräischen, griechischen und römischen Schriftsteller lieset,
die doch keine andere Sprache führen? Oder warum dienen sie uns nicht auch
hierin, wie in andern Dingen, zur Richtschnur? Und warum findet man es doch bei
uns so lächerlich, daß wir in unsern Ausdrücken und nach den richtigen
Vorschriften der Sprachlehre richten? Man halt es ja für vernünftig, den
Schülern scharfe Verweise zu geben, wenn sie gegen die Sprachregeln fehlen, und
\textit{ihr} oder \textit{sie} statt \textit{du} setzen, und uns glaubt man mit Tadel und Verspottung
uberhäufen zu müssen, weil wir eben diese Regeln bes\footnote{beobachten}.

\section{4. Abschnitt} \label{kap10_ab4}

Einzelne Personen \textit{Du} zu nennen, kann, drittens, auch aus dem Grunde weder
unschicklich noch unhöflich sein, weil es immer in allen Sprachen und auch zu
allen Zeiten üblich war. Dieses liegt klar am Tage; denn als Gott selbst zuerst
mit Adam redete, geschah es in der ebräischen Sprache, in welcher einzelne
Personen \textit{Du} genannt werden. Dasselbe ist der Fall im Assyrischen\index{Assyrisch, Sprache},
Chaldaischen\index{Chaldaisch, Sprache}, Griechischen\index{Griechisch, Sprache} und Lateinischen\index{Lateinisch, Sprache}. Auch in unsern Zeiten hat sich
diese Unterscheidung der zweiten Person des Singulars bei den Türken\index{Personen:!Türken}, Tatarn\index{Personen:!Tatarn},
Russen\index{Personen:!Russen}, Italienern\index{Personen:!Italiener}, Indianern\index{Personen:!Indianer}, Persern\index{Personen:!Perser}, Spaniern\index{Personen:!Spanier}, Franzosen\index{Personen:!Franzosen}, Holländern\index{Personen:!Holländer},
Deutschen\index{Personen:!Deutschen}, Schweden\index{Personen:!Schweden}, Dänen\index{Personen:!Dänen}, Irländern\index{Personen:!Irländer}, Schottländern\index{Personen:!Schottländer}, Wallisern\index{Personen:!Walliser}
 und Engländern\index{Personen:!Engländer} erhalten, und das Wort \textit{Du} ist nicht verloren gegangen;
denn obgleich einige der neuern Sprachen in den meisten Fällen dafür das Wert
\textit{ihr}, oder statt dessen auch \textit{sie} und \textit{er} gebrauchen, so unterliegen
doch alle dabei einem und demselben Irrthume. \textbf{Hieraus gehet aber auch klar
hervor, daß unser \textit{Du} weder eine Neuerung noch Unschieklichkeit, sondern
vielmehr in allen Sprachen das einzige schickliche Wart ist, wodurch man
einzelne Personen von mehrern in der Anrede gehörig unterscheiden kann,} und daß
ohne den Gebrauch dieses Wortes alle Aussprache, Reden und Abhandlungen immer
sehr zweideutig, unbestimmt und zweifelhaft sein würden. Wir wollen z.B.
annehmen, es stünden drei verschiedene Personen vor Gericht, die verschiedener
Verbrechen angeklagt und auch alle drei schuldig wären, und der Richter, indem
er das Urtheil spräche, würde sagen: \textit{Ihr seid schuldig und zum Tode verurtheilt},
oder: \textit{Ihr seid unschuldig und freigesprochen}; wie könnte man wissen, wer hier
schuldig oder unschuldig wäre? ob der Richter nur einen, oder zwei, oder
vielleicht alle drei gemeint habe. Darum werden auch unsere (englischen)
Urtheilsprüche mit Anwendung der destimmenden einfachen Zahl abgefaßt; z.B.:
\textit{Halte deine Hand in die Höhe! Du bist bei deinem Namen N. N. angeklagt, daß du
die Furcht Gottes nicht vor Augen gehabt} u.s.w. Dieses Beispiel läßt sich nun
aus alle Unterredungen und Verhandlungen anwenden. Auch siehet man leicht ein,
daß die Mißverständnisse, die durch Verwechselung der Einzahl mit der Mehrzahl
unvermeidlich entstehen müssen, nur durch weitläuftige Umschreibung vermieden
werden können. Und da ohne Zweifel die Vermeidung solcher Weitschweifigkeiten
und Dunkelheiten den Gebrauch der bestimmenden einfachen Zahl zuerst nothwendig
machte, so kann derselbe auch mit Recht nicht aufgegeben werden, so lange noch
dieselbe Nothwendigkeit dafür vorhanden ist, und diese wird immer fortdauern, so
lange es zwei Menschen in der Welt giebt.

\section{5. Abschnitt} \label{kap10_ab5}

\textbf{Dieses ist jedoch noch nicht Alles, was ich gegen die Verwechselung der
einfachen Zahl mit der Mehrzahl einzuwenden habe.} Es geschah zuerst aus
Nachahmung der eitlen Huldigungen, welche die Heiden\index{Personen:!Heiden} ihren Göttern\index{Heidnische Götter} darbrachten,
daß man diese verkehrte Art zu reden einführte, um dem Stolze der Päpste\index{Personen:!Päpste} und
Kaiser\index{Personen:!Kaiser, Die} dadurch zu schmeicheln\index{Schmeicheln}. Man glaubte einzelnen Großen mehr Ehre zu
erweisen, wenn man sie in der Mehrheit anredete; als wenn ein Papst aus mehrern
Göttern und ein Kaiser aus mehrern Menschen bestände. Aus einem solchen Grunde
wurden zuerst die Wörter: \textit{ihr, euch, euer} oder \textit{eure}, (und hernach bei
den Deutschen auch: \textit{sie, ihnen}, und \textit{ihr} oder \textit{ihre}) welche man sonst
nur von mehrern Personen gebrauchte, auf \textit{einzelne} angewendet. Es scheint,
als wenn die Wörter: \textit{du, dir, dich} es für Diejenigen, die gern ein größeres
Ansehn haben wollten, als ihnen eigentlich gebührte, zu einfach waren, und die
glänzende Achtung, die sie verlangten, auszudrücken, und daß man daher einen
Styl zu erfinden suchte: der ihrem Ehrgeize entspräche. Dieses ist aber ein
Grund, auf den wir nicht bauen können; da nur noch Solche, als ihn legten,
Vergnügen und Vortheil in seiner Erhaltung zu finden glauben. Gesetzt aber auch,
die Ausdrücke: \textit{Ihr, Euer, Sie, Ihre} usw. schickten sich für einen Fürsten\index{Personen:!Fürsten, Die},
so folgt daraus nicht, daß man sie auch auf gewöhnliche Menschen anwenden könne.
Denn, wenn z. B. das Edict eines Fürsten so lautet: \textit{Wir wollen und befehlen}
etc., so können wir unter dem Worte: Wir verstehen, daß er vielleicht in
Verbindung mit seinen Ministern und Räthen so rede; wird hingegen von einer
Privatperson in der Mehrheit geredet, so ist dies unstreitig ein verkehrter
Gebrauch der Worte. So wie jedoch der Stolz ihm seine Entstehung gab, ist auch
die Schmeichelei allezeit geschäftig gewesen, ihn zu verbreiten. In England und
Frankreich pflegte man ehemals die Nennen: \textit{Monsieur}, \textit{mein Herr}, und
\textit{Sir, Herr}, nur dem Könige und seinem Bruder, und die Benennung: \textit{Madame},
nur ihren Gattinnen beizulegen. Jetzt wird in Frankreich der Bauer\index{Bauer, Anrede des} hinter dem
Pfluge \textit{Monsieur}, und seine Frau \textit{Madame} genannt, und in England nennt man
jeden Handwerker\index{Handwerker, Anrede des} \textit{Sir}. d.h. Herr, und seine Frau \textit{Mistress} welches Wort
ebensoviel als daß französische \textit{Madame} ausdrükt So wirksam haben sich Stolz und
Schmeichelei in allen Zeitaltern bewiesen, indem der erstere immer gern nahm,
was die letztere gern gab; nämlich: sogenannte Achtung und Ehre.

\section{6. Abschnitt} \label{kap10_ab6}

\index{Sitten unterwerffen, sich}Man wird jedoch vielleicht einwenden, es sei nicht mehr als billig, daß man sich
einem üblichen Gebrauche unterwerfe, und wir handelten demselben gerade
entgegen. Hieraus läßt sich aber leicht, und mit mehr Wahrheit, erwiedern, daß
wir freilich in vernünftigen oder in gleichgültigen und unschädlichen Dingen den
Gebrauche nachgeben können; allein in unvernünftigen und unerlaubten ihn keine
Herrschaft über uns einräumen dürfen. Der Gebrauch kann weder Zahlen noch
Geschlechter verändern, und ist folglich eben so wenig im Stande, den Begriff
der Mehrheit mit einzelnen Wesen zu vereinigen, als er männliche in weibliche
oder einzelne in tausende verwandeln kann. Soll aber dennoch der Gebrauch über
unser Beträgen entscheiden, so kann diese Entscheidung nur zu unsern Gunsten
ausfallen; denn da der Gebrauch selbst nichts anders als ein altes Herkommen
oder eine alte übliche Sitte ist, so kann ich mich dreist auf den Gebrauch aller
Völker vom Anfang der Welt her berufen; und möge dann die Sitte der Vorzeit
diese Frage entscheiden, ob die gegenwärtige Verwirrung in der Welt: zu einer
einzelnen Person statt \textit{du, ihr} oder \textit{sie} zu sagen, nicht eine Neuerung
ist. \index{Worte, Bedeutung von} Man verstehe mich recht! Ich weiß sehr wohl, daß Worte nur in so fern etwas
gelten, als der Gebrauch, den die Menschen davon machen, ihnen Werth oder Kraft
beilegt. Wenn dann aber die Wörter \textit{ihr} und \textit{sie}, den Gebrauch des Wortes
\textit{du} verdrängen, oder an dessen Stelle treten sollen, so gebe man uns doch
wenigstens andere Wörter deren wir uns statt ihr und sie bedienen können, um
mehrere Personen von einzelnen bestimmt zu unterscheiden. Denn, ein und dasselbe
Wort sowohl für einzelne als auch für mehrere Personen zugleich zu gebrauchen,
wenn zur richtigen Bezeichnung der Einzahl und der Mehrzahl verschiedene Wörter
da sind, und dieses bloß zur Befriedigung des Stolzes und Dünkels eitler
Gemüther zu thun, kann in der Tat nicht vernünftig sein; wenigstens nicht nach
unsern Begriffen, die, wenn sie gleich dem Geiste der Mode nicht angemessen
sind, doch, wie wir hoffen, mit der christlichen Religion übereinstimmen.

\section{7. Abschnitt} \label{kap10_ab7}

Wollte man auch noch sagen, eine einzelne Person mit \textit{Du} anzureden, sei
unhöflich oder unschicklich, so hätten Gott selbst, alle Erzväter und Propheten,
Christus und seine Apostel, die Heiligen der ersten Jahrhunderte, alle Sprachen
in der ganzen Welt und unsere eigenen Gerichtsbehörden, sämmtlich hierin
gefehlt. welches zu denken jedoch, mit Erlaubnis zu reden, große Vermessenheit
sein würde. Übrigens ist es ja auch bei unsern besten Schriftstellern ganz
gebräuchlich, in den Vorreden zu ihren Werken den Leser in der Einzahl mit
\textit{Du} anzureden; z.B. \textit{"`Leser! du wirst ersucht,"'} u.s.w. oder \textit{"`Leser dieses
soll dir zur Nachricht dienen."'} u.s.w. Eben so ist es auch eine bekannte Sache,
daß die berühmtesten Dichter in ihren Zueignungsschriften selbst an hohe
Personen sich dieses Styles bedienen, wie wir bei Chaucer\index{Personen:!Chaucer}, Spencer\index{Personen:!Spencer}, Waller\index{Personen:!Waller},
Cowley\index{Personen:!Cowley}, Dryden\index{Personen:!Dryden}, u.a. nachlesen können. Und warum will man denn nun ein solches
Benehmen bei uns für so unhöflich, ungegebildet und unerträglich halten? Ich bin
überzeugt, daß man diese Frage nie wird recht beantworten können.

\section{8. Abschnitt} \label{kap10_ab8}

\index{Sprache, Eigenart der Jünger}Ich zweifle gar nicht, daß die Sprache Jesu und seiner Jünger gleichfalls etwas
Sonderbares und Auffallendes gehabt habe; denn man warf, wie bekannt ist, dem
Petrus\index{Verleumdung, des Petrus}, als er im Palaste des Hohenpriesters seinen Herrn verleugnete, seine
Sprache als einen Beweis vor, daß er Jesu. angehören müsse.
\\textbf{\textit{"`Wahrlich,"'} sagten
die Umstehenden,}\textit{"`du bist auch Einer von ihnen; denn deine Sprache verräth
dich."'}
\footnote{Matthäus 26,37}
\index{Bibelstellen:!Matthäus 26)}
Sie glaubten kurz zuvor schon aus seinem Äußern
zu errathen, daß er mit Jesu gewesen wäre; als sie ihn aber reden hörten, setzte
seine Sprache sie deshalb außer allen Zweifel. Jetzt wußten sie es gewiß, daß er
Einer von Denen war, die mit Jesu gewandelt hatten. Petrus mußte allerdings in
dem Umgange mit Jesu Etwas angenommen haben, das sonderbar und auffallend war,
und ganz gegen das Benehmen der Welt abstach. \textbf{Ohne Zweifel unterschieden sich
die \index{Absonderung von der Welt} Nachfolger Jesu von der Welt sowohl in ihrer Tracht, Haltung und Sprache,
als in seiner Lehre, die sie zu dieser Unterscheidung anleitete und es laßt sich
leicht denken, daß sie \index{Zeugnis, Einfachheit}einfacher, ernsthafter\index{Zeugnis, Wahrhaftigkeit} und mehr genaunehmend als Andere
waren.} Dieses gewinnt auch sehr an Wahrscheinlichkeit, wenn wir erwägen, \textbf{was für
ein Mittel der arme, von seinem Selbstvertrauen betrogene Petrus in seiner
Furcht ergriff, um den Andern ihre Gedanken von ihm zu benehmen; denn \textit{"`er fing
an zu fluchen und zu schwören."'} Ein trauriger Behelf! Er glaubte aber
vielleicht, daß sicherste Mittel, allen Verdacht zu entfernen, würde das sein,
wenn er etwas thäte, was mit Christo und seiner Lehre am wenigstens
übereinstimmte.} Der Kunstgriff gelang auch; er brachte sie mit ihren
Einwendungen zum Schweigen und Petrus ward nun für eben so rechtgläubig\index{rechtgläubig} als
Einer von ihnen gehalten. Aus diese Weise entging Petrus nun zwar den
Nachforschungen der Menschen, aber doch nicht dem Hahnengeschrei, das in seine
Ohren drang, und ihn an die Worte seinen geliebten leidenden Herrn erinnerte.
\textit{"`Er ging hinaus und beweinte bitterlich,"'} daß er seinen Meister verleugnet
hatte, der nun überliefert war, auch für ihn in den Tod zu gehen.

\section{9. Abschnitt} \label{kap10_ab9}

\textbf{Der letzte Grund, den ich zur Rechtfertigung unserer Sitte: einzelne Personen
nur in der einfachen Zahl anzureden, noch anzuführen habe, hat, meiner Ansicht
nach, das größte Gewicht; und da er allen Menschen einleuchtet, so werden unsere
Tadler am wenigsten Etwas dagegen Vorbringen können. Er besteht darin, daß man
es uns nicht zumuthen müsse, einem Gebrauche nachzugehen, der gerade den
höchsten Grad des Stolzes sterblicher Menschen darin beweiset, daß sie den ihren
Mitmenschen eine bessere, anständigere oder höflichere Sprache verlangen oder
erwarten, als sie selbst gegen den unsterblichen Gott, ihren großen Schöpfer,
gebrauchen, wenn sie ihn Verehrung und Anbetung leisten. \index{Gott mit "`du"' ansprechen} Bist du, o Mensch! denn
größer als Er, der dich erschaffet hat? Kannst du den Gott, der dir den Athem
gab, den großen Richter aller Handlunngen deines Lebens, mit \textit{du} anreden, und
sobald du dich von deinen Knien erhoben hast, einen Mitchristen beleidigen, weil
er dich Erdenwurm\index{Erdenwurm} mit eben der Sprache anredet, worin du so eben zu deinem Gott
geredet hast?} Ist dieses nicht eine Anmaßung ohne Gleichen? -- Zu Jemand du zu
sagen, ist aber entweder ein Zeichen von zu vieler oder zu weniger Achtung. Ist
es zu viel Achtung, die wir dadurch beweisen, so werde darüber nicht zornig und
mache uns deshalb keine Vorwürfe, sondern lehne es mit mit Ernst und Demuth von
dir ab. Scheint es die aber zu wenig auszudrücken, warum erzeigst du denn Gott
keine größere Achtung? O! Wohin hat doch der Mensch sich verstiegen! Zu welchem
Gipfel will sein Stolz sich erschwingen! Er verlangt größere Achtung von seinen
Nebenmenschen, als er selbst Gott erweiset. Heißt das nicht für mehr als einen
Gott gehalten sein wollen? Indessen dürfte es ihm unter und eben so sehr an
Anbetern fehlen, als es ihm an der Göttlichkeit mangelt, die der Anbetung würdig
ist. Wir sind völlig überzeugt, daß der Geist Gottes Niemand anleitet, Ehre von
Menschen zu suchen, und noch weniger, dieselbe zu vertheidigen, oder auf
Diejenigen zu zürnen, die, aus Gewissenhaftigkeit gegen Gott, Andern keine
weltliche Ehrenbezeigungen erweisen dürfen. Und es liegt auch klar am Tage, daß
nur die eitlen Gemüther des gegenwärtigen Geschlechts um ihren Hochmuth zu
befriedigen, des Gebrauchs derselben sich schuldig machen. Welche Verstellung,
was für ein Kriechen\index{Kriechen} und Schmiegen siehet man nicht täglich! Ja, wie viele
unnüze nichtssagende Worte und höchst übertriebene Ausdrücke, leere Komplimente,
grobe Schmeicheleien und offenbare Lügen werden nicht beständig, unter dem Namen
von Höflichkeitsbezeigungen, von männlichen und weiblichen Personen in ihrem
Umgange gebraucht! O! meine Freunde! Woher nehmet ihr die Beispiele zu einem
solchen Beträgen? Welche Stellen aus den Schriften der heiligen Männer Gottes
können solche Dinge rechtfertigen? Aber ich muß euch noch naher kommen, und euch
euer eigenes Bekenntniß vorhalten. -- Dienet euch Christus, zu dessen Namen ihr
euch bekennet, hierin zum Muster?\index{Jesus als Vorbild} Oder richtet ihr euch nach jenen Heiligen der
Vorzeit, die in Einöden wohnten, und
\textit{"`deren die Welt nicht Werth war?"'}
\footnote{Hebräer 11,38}
\index{Bibelstellen:!Hebräer 11)}
Oder glaubt ihr in den Fußstapfen jener Christen zu
wandeln, die, aus Gehorsam gegen die Lehre und nach dem Vorbilde der Lebens
ihres Meisters, dem Ansehen der Person entsagten und die Moden und Gebräuche,
die Ehre und Herrlichkeit dieser vergänglichen Welt verließen? Der Christen, die
sich nicht durch äußere Geberden, Höflichkeitezbezeigungen, Komplimente etc.,
sondern durch
\textit{"`einen stillen und sanften Geist"'}
\footnote{1. Petrus 3,4}
\index{Bibelstellen:!1. Petrus 3)}
auszeichneten, der mit\index{Tugend der Anhänger} Mäßigkeit, Tugend, Bescheidenheit, Ernst, Geduld und
brüderlicher Liebe geschmucket war? Denn darin bestanden in jenen Zeiten der
Christenthumes die wahren Ehrenzeichen und einzigen Merkmaale der Würde und der
Adels der Christen. Und sehen wir uns nicht eben darum, weil wir ihnen und nicht
der Welt in ihren Gebräuchen nachahmen, von Andern verachtet und verhöhnet? Sagt
uns doch aufrichtig, machen nicht Romane\index{Romane}, Schauspiele\index{Schauspie}, Maskenbälle\index{Maskenbälle},
Spielpartien\index{Spiele}, Concerte\index{Konzerte} u. dgl. eure Lieblingsunterhaltungen aus? \textbf{Hätter ihr
wirklich den Geist den wahren Christenthumes\index{Christenthum, das wahre}, wie könntet ihr denn eure so
kostbare und kurze Zeit mit so vielen unnöthigen Besuchen\index{Besuche}, Spielen und
Zeitvertreiben\index{Zeitvertreib}, mit Komplilnentenmachens und Schmeicheleieir hinbringen? Wie
könntet ihr euch mit Erzählungen erdichteter Geschichten\index{Geschichten, erdichtete}, mit Herumtragen
nutzloser Neuigkeiten\index{Neuigkeiten, nutzlose} und noch vielen andern eitlen Dingen beschäftigten, die
bloß dazu da sind, und deren ihr euch auch nur bedienet, um euch zu zerstreuen\index{Zerstreuung};
um nicht an euren wahren Zustand zu denken, und euch in gänzlicher
Gottesvergessenheit\index{Gottesvergessenheit} zu betäuben\index{betäuben}.} Solche Unterhaltungen und Ergötzlichkeiten
waren gewiß nie unter den wahren Christen\index{Christen, wahre}, sondern nur unter den Heiden\index{Heiden}, die
Gott nicht kannten, üblich. Ach! hättet ihr doch ein wahres Gefühl von eurem
sündhaften Zustande\index{sündhafter Zustand}, und wäret ihr nur in einigem Grade neu geboren\index{geboren, neu}! Möchtet ihr
doch das Kreuz Christi aufnehmen\index{Kreuz Christi} und unter seiner Herrschaft\index{Herrschaft Christi} leben! Dann wurden
diese Dinge, die eurer verderbten\index{Natur, verderbte} und sinnlichen Natur\index{Natur, sinnliche} so sehr schmeicheln,
keinen Raum in euren Herzen mehr finden. Das heißt nicht:
\textit{"`suchen was droben ist,"'}
\footnote{Kolosser 3,1}
\index{Bibelstellen:!Kolosser 3)}
wenn man mit seinem Herzen an den niedrigen Dingen der Welt hängt. Das
heisit nicht:
\textit{"`seine Seligkeit mit Furcht und Zittern schaffen,"'}\index{Furcht und Zittern} wenn man
seine kostbare Zeit mit eitlen Dingen vertändelt. Dann kann man nicht mit Elihu\index{Personen:!Elihu}
ausrufen:
\textit{"`Ich will Niemand's Person ansehen, und keinen Menschen rühmen; (oder
ihm schmeichelhafte Titel geben;) denn ich weiß nicht, wenn ich es thäte, ob
mein Schöpfer mich nicht bald hinwegnehmen würde."'}
Nein, das heißt nicht: sich
\textit{"`selbst überwinden\footnote{"'verleugnen"` ersetzt durch "'überwinden"`}, unvergängliehe Schätze sammelte und noch einem
unverweltlichten Erdtheile im Himmel trachten."'} -- Und nun, meine Freunde! was
ihr auch davon denken möget, so muß ich euch sagen, daß die Entschuldigung: \textbf{euch
auf den allgemeinen Gebrauch zu berufen, vor Gottes Richtstuhle nicht gelten
wird.} \textit{Das Lich Christi}, das in euren eigenen Herzen scheint\index{Innere Licht}, wird sie
verwerfen, und dann wird der Geist, gegen den wir zeugen, so erscheinen, wie er
ist, und wie wir ihn geschildert haben. Saget nicht, daß ich um Kleinigkeiten
eifere; hütet euch lieber selbst vor Leichtsinn und Unbedachtsamkeit in
ernsthaften Dingen.

\section{10. Abschnitt} \label{kap10_ab10}

Ehe ich dieses Kapitel schließe, will ich noch einige Zeugnisse allgemein
geachteter Männer zu Gunsten unserer Nichtgleichstellung der Welt in ihren:
verkehrten Gebrauche der Sprache hier beifügen.

\medskip

\textbf{Luther\index{Personen:!Luther, Martin}, dieser große Reformator, dessen Ausprüche man in seinem Zeitalter
wie Orakeisprüche betrachtete, und der auch noch heutiges Tages bei vielen
unserer Gegner\index{Quaker-Gegner} in großem Ansehen stehet,} Luther war so weit entfernt, unsere
einfache Sprechart zu tadeln, daß er vielmehr in einem seiner Werke, Ludus\index{Ludus (Buch)}, (das
Spiel,) betitelt, über den Gebrauch: einzelne Personen in der Mehrzahl
anzureden, als über eine unschickliche und lächerliche Suche sich lustig macht,
wo er nämlich sagt: \texttt{Magister! vos estis iratus}; \textit{"`Magister! ihr seid
unwillig"';} welches im Lateinischen\index{Lateinisch} eben so abgeschnrackt herauskommt, als es in
jeder andern Sprache lauten würde, wenn man sagte:
\textit{"`Meine Herren! du bist unwillig."'} -- Erasmus\index{Personen:!Erasmus}, ein großer Gelehrter und ein so tiefer Sprachforscher,
daß ich keinen wüßte, auf den man sich, hinsichtlich der Sprachrichtigkeit eines
Ausdruckes, mit mehr Befugniß berufen könnte, stellte nicht nur jenen Gebrauch:
von einzelnen Personen in der Mehrzahl zu reden, in ein lächerliches Licht,
sondern schrieb auch eine eigene Abhandlung über die Ungereimtheit desselben,
worin er deutlich zeigt, daß es unmöglich sei, den Unterschied zwischen der
Einheit und Mehrheit gehörig in Acht zu nehmen, wenn man ein Wort, das bloß dazu
da ist, die Mehrheit zu beizeichnen, auf einzelne Gegenstände anwendet. Auch
sagt er noch, daß diese Sprachverwirrung aus der Schmeichelei der Menschen
entsprungen sein. -- Lipsiuz\index{Personen:!} versichert von den alten Römern, daß die jetzige Art
der Begrüßung bei ihnen nicht üblich war. Und endlich giebt uns Howel\index{Personen:!Howel} in seiner
Geschichte Frankreichs eine treffende Erörterung von dem Ursprunge des
Gebrauchs: einzelne Personen in der Mehrzahl anzureden, indem er uns versichert,
daß vor alten Zeiten die Bauern\index{Bauern} ihre Könige\index{Könige} bußten, Stolz und Schmeichelei aber
zuerst die Untergeordneten bewogen, den einzelnen Personen ihrer Obern
Ehrenbezeigungen in der mehrfachen Zahl zu erweisen, und die Obern geneigt
machten, diese anzunehmen? Könnten wir nun auch, zur Rechtfertigung unsers
Gebrauchs der einfachen und richtigen Sprache, uns nicht auf die unverwerflichen
Beispiele Gottes und guter Menschen berufen, so würden wir dennoch, da wir
überzeugt sind, daß Stolz und Schmeichelei den jezt üblichen Mißbrauch derselben
einführten, schon aus Gewissenhaftigkeit und nicht darein fügen können. Und so
sehr uns auch die Ungezügelten und leichtsinnigen unsers Zeitalters, deren
Gemüther so unaufhörlich von der Liebe zu weltlichen Vergnögnngen umgetrieben
werden, daß sie den wahren Ursprung und Zweck der Worte und Dinge zu prüfen und
zu unterscheiden nicht im Stande sind, -- so sehr diese uns auch als sonderbare
Menschen tadeln mögen; so können dennoch wir, die wir durch Gottes Licht und
Geist von der Thorheit und schädlichen Wirkung solcher weltlichen Gedrauche
überzeugt, und zu einer klaren Einsicht und geistlichen Unterscheidung ihres
Ursprunges und ihrer Eigenschaften gelangt find, diese Dinge nicht anders als
Früchte des Stolzes und der Schmeichelei erkennen, und deswegen und auch darin
nach dem Verlangen irdischgesinnter Gemüther nicht mehr bequemen. Wir möchten
sonst unsern Gott beleidigen und unsere Gewissen mit Schuld beladen. Denn da wir
durch die innern Züchtigungen der göttlichen Gnade\index{göttlichen Gnade} aufrichtig gerührt und zur
aufmerksamen Unterwerfung unter das heilige Gesetz Jesu in unsern Herzen\index{Gesetz, im Herzen}
gebracht worden sind, so daß wir
\textit{"`unsere Werke an das Licht bringen, um zu
sehen, od sie in Gott gethan sind oder nicht;"'}
\footnote{Johannes 3,13 und 20+21}
\index{Bibelstellen:!Johannes 3)}
so können und dürfen wir uns der Welt, die mit ihrer Lust vergehet, in ihrem eitlen
Wesen nicht mehr gleiehstellen: indem wir gewiß wissen,
\textit{"`daß die Menschen am
Tage des Gerichte von jedem unnützen Worte, das sie geredet haben, Rechenschaft
geben müssen."'}
\footnote{Matthäus 12,36}
\index{Bibelstellen:!Matthäus 12)}

\section{11. Abschnitt} \label{kap10_ab11}

Darum, o Leser! du magst nun ein in der Nacht sich zu Jesu schleichender
Nicodemus\index{Personen:!Nicodemus}, oder ein ihn verhöhnender Schriftgelehrter sein; nämlich Einer, der
den glorreichen Messias auch gern besuchte, aber doch lieber von den finstern
Gebräuchen der Welt bedekket zu ihm käme, damit du unerkannt durchgehen und der
Schwach seines Kreuzes ausweichen köntest oder ein Begünstiger und Vertheidiger
des Hamanschen Stolzes, und hältst vielleicht diese hier abgelegten Zeugnisse\index{Zeugnisse}
nur für alberne Sonderbarkeiten: \textbf{so muß ich dir sagen, daß göttliche Liebe mich
verpflichtet, dir die Wahrheit zu verkündigen und ein getreues Zeugniß\index{Zeugniß} gegen das
ungöttliche Wesen der entarteten Welt,} so wie in andern, auch in diesen Stücken
\textbf{bei dir abzulegen,} in welchen der Geist der Eitelkeit und sinnlichen Begierden
eine so große Macht gewonnen und so lange unbeschränkt geherrscht hat, daß er
Unverschämtheit genug besitzt, seine Finsterniß Licht zu nennen, und den
Früchten seines verderbten Baumes Namen beizulegen, die nur Erzeugnissen von
einer edlern Art gebühren, um dadurch die Menschen desto leichter zu täuschen
und für den Gebrauch derselben zu gewinnen. Und wahrlich, die Mehrsten sind,
leider! so verblendet, und so fühllos geworden, daß sie gar nicht wissen,
welches Geistes sie sind, und haben so niedrige oder so irrige Begriffe von dem
demüthigen Leben der Selbstüberwindung\footnote{"`Selbstverleugnung"' ersetzt durch "`Selbstüberwindung"'} und von der Verbindlichkeit der Lehre des
heiliger: Jesu, daß sie einander \textit{Rabbi}, das ist so viel als \textit{Meister,
Herr, gnädiger Herr, Eure Gnaden,} etc. nennen; daß sie Verbeugungen vor
einander machen, die ich als Anbetung ihrer Person betrachte; daß sie aus
Schmeichelei einander schöne Titel beilegen, um ihren Ehrgeiz zu befriedigen und
zu nähren; dass sie sich zu wenig geehrt oder beleidigt finden, wenn sie in
Ausdrücken angeredet werden, deren sie sich selbst gegen ihren Schöpfer
bedienen, und daß sie endlich ihre Zeit und ihr Vermögen verschwenden, um ihre
sinnlichen Begierden zu befriedigen, indem sie sich den Sitten und Gebräuchen
der Heiden ergeben, die Gott nicht kannten, und ihre Thorheiten für
Höflichkeiten, Erziehung, Anstand, keine Bildung, u.s.w. halten. O! möchtest
du doch, da es nur einen guten und einen bösen Geist giebt, ernstlich erwägen,
welcher von beiden es ist, der die Welt zu solchen Dingen anleitet; ob Nicodemue
oder Mardochai\index{Personen:!Mardochai} dich gegen die verachteten Christen in deinem Innern geneigt
macht, und welcher von ihnen die Furcht und Scham einflößt, Demjenigen in deinem
Umgange mit der Welt öffentlich zu entsagen, was das wahre Licht dir als
Eitelkeit und Sünde im Verborgenen deines Herzens\index{Innere Licht} zu erkennen gegeben hat. Oder
wenn du zu unsern Verächtern gehörst, so sage mir, ich bitte dich, wem glaubst
du mit deinen Spöttereien, mit deinem Unwillen und mit deiner Verachtung am
ähnlichsten zu sein, dem stolzen Haman oder dem guten Mardochai? Wisse, mein
Freund! daß vielleicht kein Mensch diese Eitelkeiten, die man Höflichkeiten
nennt, mehr geliebt und verschwenderischer angewendet hat, als ich; und hätte
ich mein Gewissen unter den Zeremonien der Welt verbergen können, so wäre ich
gewiß manchen Stürmen von Vorwürfen entgangen, die, meines offenen Bekenntnisses\index{Bekenntnis}
wegen, oft heftig über mich ausbrechen. Aber dann würde ich auch, wenn ich, nach
den weltlichen Sitten\index{weltlichen Sitten} und Gebräurhen mich bequemt hatte, wider meinen Gott
gesündigt und den Frieden meiner Seele\index{Seele, frieden der} verloren haben.

\medskip

Glaube jedoch nicht, Freund! daß wir um der bloßen Titel oder um des nackten
Wortes \textit{Du} willen solche Schwierigkeiten machen, oder die Absicht haben, neue
Formen einzuführen, die mit der Aufrichtigkeit und wahren Höflichkeit
unverträglich sind; denn es giebt ja deren, leider! auch schon zu viele. Nein!
der Werth, die eitle Gemüther aus jene weltliche Zeremonien legen, die hohe
Meinung, die sie davon haben, und die Nothwendigkeit, daß ihnen diese
schädlichen Dinge zu Gesichte gebracht werden, damit sie davon gereinigt und
befreiet werden  können, dieses sind die Beweggründe, die uns zwingen, ein
standhaftes Zeugnis\index{Zeugnis} dagegen an den Tag zu legen. Und wir können dir aus der vom
heiligen Geiste Gottes und verliehenen Erkenntniß bezeugen, daß Dasjenige im
Menschen, welches die Beachtung\footnote{"`Beobachtung"' ersetzt durch "`Beachtung"'} solcher weltlichen Gebräuche verlangt, das
eine Furcht in ihm erzeugt, wenn er sich davon losreißen will, oder sie zu
vertheidigen und zu rechtfertigen sucht, und unzufrieden ist, wenn sie nicht
beobachtet werden, daß alles Dieses im Grunde nichts anders als Wirkungen des
Geistes der Schmeichelei und des Stolzes sind; obgleich bei Einigen der öftere
Gebrauch oder die Gewohnheit eine gewisse Gleichgültigkeit dagen erzeugt, und
bei Andern der Edelmuth die Triebfedern des Stolzes geschwächet haben mag. Und
da dieses in dem himmlischen Lichte erkannt wird, welches jetzt in den Herzen
der verachteten Christen, in deren Gemeinschaft ich lebe, mit Klarheit scheinet,
so finden sie sich oft dadurch bewogen, gegen die \textit{"`unfruchtbaren Werke der
Finsterniß"'}\index{Finsterniß, Die} öffentlich zu zeugen, so wie auch ich, als einer von den Ihrigen,
und in ihren Namen, dieses Zeugniß hier ablege, um vornehmlich die Treulosen und
Wankelmüthigen, die, wiewohl sie eines Bessern überzeugt sind, doch gern noch
unbemerkt fortwandeln möchten, zu bestrafen und anzuspornen, und die Heftigkeit
unserer Tadler, die uns als ein affektirtes, sonderbares Volk geringachten,
einigermaßen zu mildern. Denn der ewige Gott, der sich unter uns mächtig
erwiesen hat, und ausgegangen ist, den Bewohnern der Erde seine Macht kund zu
thun,
\textit{"`wird jede Pflanze ausrotten, die nicht von seiner Hand gepflanzt
ist."'}
\footnote{Matthäus 15,13}
\index{Bibelstellen:!Matthäus 15)}

\medskip

Darum, mein Leser! laß mich dich bitten, die dir hier vorgelegten Gründe wohl zu
erwägen: Sie wurden nur größtentheils von dem Herrn zu der Zeit gegeben, da man
meine Einwilligung, in den Sitten und Gebrauchen der Welt zu bleiben, fast um
jeden Preis gern erkauft haben würde. Allein die gewisse Überzeugung, die ich
hatte, daß sie mit dem demüthigen Leben der Selbstverleugnung des heiligen Jesu
im Widerspruche stehen, gebot mir, ihnen gänzlich zu entsagen, und ein getreues
Zeugniß dagegen abzulegen\index{Zeugniß ablegen}. Ich rede die Wahrheit in Christo, und lüge nicht! Ich
würde mich dem Tadel und der Verachtung Anderer nicht ausgesetzt haben, wenn ich
mit Frieden des Gewissens unter einem weltlichen Betragen meinen Glauben hätte
bewahren können. Es fiel mir in der Tat schwer, mich auf solche Weise
auszuzeichnen und so sonderbar zu erscheinen, allein meine gewisse, mir
wiederholt gegebene Überzeugung, daß Stolz, Eigenliebe und Schmeichelei die
Grundursachen dieser eitlen Gebräuche sind, erlaubte mir nicht, so böse
Eigenschaften noch länger in mir selbst und in Andern zu nähren. Aus diesem
Grunde bin ich so ernstlich bemühet, meinen Lesern in Ansehung ihrer
Beurtheilung unsere Betragens Vorsichtigkeit zu empfehlen; und ich wiederhohle
daher meine Bitte, daß sie bei sich selbst ernstlich erwägen wollen, ob es der
Geist der Welt oder der Geist unsere himmlischen Vaters ist, der über unser
ehrliches, gerades und harmloses \textit{Du} sich entrüstet; damit so jede Pflanze,
die Gott nicht in die Herzen der Söhne und Töchter der Menschen gepflanzet hat,
möge ausgerottet werden.


