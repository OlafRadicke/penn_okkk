\section*{N}

\articlesize

\begin{description}
 \item[Nayler, James] $\ast$1618, \dag1660. Englischer Quäker. Ist der Nachwelt in erster Linie durch den Konflikt mit $\to$Geroge Fox in Erinnereung geblieben.

 \item[New Foundation Fellowship]
% <li><b><i>New Foundation Fellowship</i></b> <KBD>( http://www.nffellowship.org/index.html )</KBD> Das ist kein Schismar oder Flügel, als mehr eine "Erneuerungsbewegung". Einige Freunde, vorallem wohl in den Liberalen Flügeln, versuchen die Erinnerung an die Frühen Freunde wach zu halten und sich auf alte Traditionen zu besinnen. New Foundation Fellowship ist auch in der Deutschen Jahresversammlung vertreten. Ich glaube Annette Fricke\endnote{Nontheist Friend. (2009, May 18). In Wikipedia, The Free Encyclopedia. Retrieved 19:35, May 18, 2009, from http://en.wikipedia.org/w/index.php?title=Nontheist_Friend&oldid=290733148} kann man dazu zahlen. Sie bemühte sich seinerzeit vergeblich das Tagebuch von G. Fox neu zu übersetzen und über das GYM verlegen zu lassen. Es gibt zeit geraumerzeit im "Quäker" eine Reihe wo Rex Amblers Geogre-Fox-Verkexelungen übersetzt werden. Ob sie daran beteiligt ist, weiß ich nicht. Bei aller Begeisterung für G. Fox, aber mit dem Rex Amblers Zeugs kann ich überhaupt nichts anfangen.</li>

 \item[Nontheist Friends]
%  \endnote{\texttt{http://www.nontheistfriends.org/}} \endnote{Nontheist Friend. (2009, May 18). In Wikipedia, The Free Encyclopedia. Retrieved 19:35, May 18, 2009, from \texttt{http://en.wikipedia.org/w/index.php?title=Nontheist_Friend\&oldid=290733148}}
%  Ob man zu \textit{Nontheist Friends} auch Athistisch Quäker sagen kann ist für mich offen geblieben.
%  \endnote{Im "`Quäker"' 2/2008 Seite 88 veröffentlichte H. Konopatzky das Gedicht
%  "`Bekenntnis"', worin es Heißt:
%  \textit{"`Ist kein Jude oder Christ, weder Moslem noch Buddhist,
%  meint, er sei ein Atheist. - Sollt ich selbst mich bekennen, müsst ich mich ungläubig
%  nennen im Sinne aller Religionen, in deren Templn Herrscher thronen."'}}

 \end{description}

\normalsize
