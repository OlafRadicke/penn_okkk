\section*{G}

\articlesize

\begin{description}

 \item[Graeff, Abraham Isacks op den]  $\ast$1649 in Krefeld; \dag25. März 1731
 in Perkiomen (heute Montgomery). Quaker. Entstammte einer mennonitischen Familie.
Er gehörte der ersten geschlossenen Gruppe deutscher Auswanderer nach Amerika
an, der sogenannten "`Original 13"'. Er war einer der ersten Unterzeichner einer
Petition gegen die Sklaverei. Im Jahre 1692 wurde er zum Bürgermeister von
Germantown ernannt.\endnote{Seite "`Abraham Isacks op den Graeff"'. In:
Wikipedia, Die freie Enzyklopädie. Bearbeitungsstand: 31. Dezember 2009,
07:28 UTC. URL: \\ \texttt{http://de.wikipedia.org/w/index.php?title=\\
$\hookrightarrow$~Abraham\_Isacks\_op\_den\_Graeff\&oldid=68641530} }

\item[Gurneys] Einflussreiche alte Quakerfamilie. Bekannteste Mitglieder sind:
$\to$\textit{Joseph John Gurney}, $\to$\textit{Elizabeth (Gurney) Fry},
$\to$\textit{Samuel Gurney}

\item[Gurney, Joseph John] $\ast$2.8.1788 \dag4.1.1847; Quaker, englischer Banker und
und evangelikaler Prediger. Seine Ansichten und Predigttätikeiten fürten zu
dem zweiten großen Schisma im Nordamerikanischen Quakertum. Dem
$\to$\textit{Gurney-Wilbur-Schisma}, bei dem sich die Orthodox-Linie noch ein
mal auf spaltete.

\item[Gurney, Samuel] $\ast$18.10.1786 \dag5.6.1856; Quaker, englischer Banker und
Philanthrop.


 \end{description}

\normalsize