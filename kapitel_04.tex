

\chapter{Kapitel 4} \label{kap4}

\section{Zusammenfassung des 4. Kapitel}

\begin{description}
\item[1. Abschnitt] Worin entsteht die große Wirkung des Kreuzes? Die
Beantwortung dieser Frage ist von hoher Wichtigkeit.
\dotfill \textit{Seite~\pageref{kap4_ab1}}\\
\item[2. Abschnitt] Die Wirkung des Kreuzes ist Selbstüberwindung
\footnote{\texttt{'Selbstverleugnung' ersetzt durch 'Selbstüberwindung'}}.
\dotfill \textit{Seite~\pageref{kap4_ab2}}\\
\item[3. Abschnitt] Worin bestand der Kelch, den Christus trank, und das Kreuz,
das er trug?
\dotfill \textit{Seite~\pageref{kap4_ab3}}\\
\item[4. Abschnitt]  Was ist unser Kelch und unser Kreuz?
\dotfill \textit{Seite~\pageref{kap4_ab4}}\\
\item[5. Abschnitt] Es ist unsere Pflicht, Christus, als dem Herzog unserer
Seligkeit, zu folgen.
\dotfill \textit{Seite~\pageref{kap4_ab5}}\\
\item[6. Abschnitt] Von dem Unterschiede zwischen der erlaubten und unerlaubten
Eigenliebe.
\dotfill \textit{Seite~\pageref{kap4_ab6}}\\
\item[7. Abschnitt] Erklärung der erlaubten Eigenliebe.
\dotfill \textit{Seite~\pageref{kap4_ab7}}\\
\item[8. Abschnitt] Sie muss, nach Christi Lehre und Beispiel, in einigen Fällen
überwunden\footnote{\texttt{'verleugnet' durch 'überwunden' ersetzt}} werden.
\dotfill \textit{Seite~\pageref{kap4_ab8}}\\
\item[9. Abschnitt] Auch nach dem Vorbild der Apostel.
\dotfill \textit{Seite~\pageref{kap4_ab9}}\\
\item[10. Abschnitt] Von der Gefahr, worin diejenigen sich befinden, die ihre
Eigenliebe ihren Pflichten gegen Gott vorziehen.
\dotfill \textit{Seite~\pageref{kap4_ab10}}\\
\item[11. Abschnitt] Belohnung der Selbstverleugnung, Aufmunterung derselben.
\dotfill \textit{Seite~\pageref{kap4_ab11}}\\
\item[12. Abschnitt] Diese Lehre ist so alt als Abrahams Zeitalter.
\dotfill \textit{Seite~\pageref{kap4_ab12}}\\
\item[13. Abschnitt] Abrahams Gehorsam des Glaubens ist sehr merkwürdig.
\dotfill \textit{Seite~\pageref{kap4_ab13}}\\
\item[14. Abschnitt] Hiob gab ein großes Beispiel der Selbstüberwindung
\footnote{\texttt{'Selbstverleugnung'duch 'Selbstüberwindung' ersetzt}}; - von
seiner Zufriedenheit.
\dotfill \textit{Seite~\pageref{kap4_ab14}}\\
\item[15. Abschnitt] Auch Moses gab ein
bemerkenswertes\footnote{\texttt{'merkwürdiges' duch 'bemerkenswertes' ersetzt}}
Beispiel der
Selbstüberwindung\footnote{\texttt{'Selbstverleugnung' duch 'Selbstüberwindung'
ersetzt}}; -- von seiner Verachtung des Hofes des Pharaos.
\dotfill \textit{Seite~\pageref{kap4_ab15}}\\
\item[16. Abschnitt] Seine Wahl.
\dotfill \textit{Seite~\pageref{kap4_ab16}}\\
\item[17. Abschnitt] Sein Beweggrund dazu, nämlich die Hoffnung der Belohnung.
\dotfill \textit{Seite~\pageref{kap4_ab17}}\\
\item[18. Abschnitt] Jesaja gab kein geringeres Beispiel, da er aus einem
Hofmanne ein Prophet des Herrn wurde.
\dotfill \textit{Seite~\pageref{kap4_ab18}}\\
\item[19. Abschnitt] Diese Beispiel schließen mit dem, welches Daniel gab,
seine Geduld und Aufrichtigkeit und was für einen Eindruck sein Betragen auf
den König machte.
\dotfill \textit{Seite~\pageref{kap4_ab19}}\\
\item[20. Abschnitt] Es könnten noch viele andere Beispiele zur Bestätigung
dieser vortrefflichen Lehre angeführt werden.
\dotfill \textit{Seite~\pageref{kap4_ab20}}\\
\item[21. Abschnitt] Man muss um Christi Willen alles verlassen, wenn man selig
werden will.
\dotfill \textit{Seite~\pageref{kap4_ab21}}\\
\item[22. Abschnitt] Gottes Weg ist ein Weg des Glaubens und der
Selbstüberwindung\footnote{\texttt{'Selbstverleugnung' duch 'Selbstüberwindung'
ersetzt}}.
\dotfill \textit{Seite~\pageref{kap4_ab22}}\\
\item[23. Abschnitt] Ernste Bitte und Ermahnung an alle, dass sie diese Dinge
beherzigen mögen.
\dotfill \textit{Seite~\pageref{kap4_ab23}}\\
\end{description}

\newpage

Die vierte Frage ist nun: Worin bestehet die große Wirkung, die das Kreuz in dem
Menschen hervorbringt?


\section{1. Abschnitt} \label{kap4_ab1}

Diese Frage, der Wahrheit gemäß, deutlich und vollständig zu beantworten, ist in
der Tat von so hoher Wichtigkeit, dass alles Vorhergegangene nur als Einleitung
dazu zu dienen scheint, indem eine unrichtige Beantwortung derselben nichts
Geringeres als ein Irreleiten der Seele auf ihrem Wege zur Seligkeit sein würde.
Ich werde daher diese Frage unter dem Beistand Gottes, nach der besten
Erkenntnis, die er mir während meiner mehrjährigen Nachfolge Christi durch
Erfahrung davon gegeben hat, gründlich erörtern.

\section{2. Abschnitt} \label{kap4_ab2}

Das große Werk, welches das Kreuz Christi im Menschen hervorbringt, ist
Selbstüberwindung\footnote{\texttt{'Selbstverleugnung' duch 'Selbstüberwindung'
ersetzt}}. Ein Ausdruck, der an sich selbst von eben so tiefer Bedeutung
ist, als die Sache, die er bezeichnet, im empfindlichsten Widerspruche mit der
Welt steht; eine Sache, die so wenig von der Welt verstanden und noch weniger
von ihr angenommen wird, und der man sich dessen ungeachtet dennoch unterwerfen
muss. Der Sohn Gottes ist uns darin vorangegangen und hat uns durch den bittern
Kelch, den er trank, und durch die Taufe \index{Taufe}, die er
erduldete, ein
Vorbild
hinterlassen, wie wir seinen Fußstapfen nachfolgen sollen. Als einst des
Zebedäus Ehefrau \index{Personen:!Zebedäus Ehefrau} ihn bat, dass in seinem
Reiche einer von ihren Söhnen zu
seiner Rechten und der andere zu seiner Linken sitzen möchte, legte er ihr und
ihren beiden Söhnen die schwere Frage vor:
\textit{"`Könnet ihr den Kelch trinken, den
ich trinken werde, und euch mit der Taufe taufen lassen, mit welcher ich
getauft werde?"'}\footnote{Matthäus  20,21-23}
\index{Bibelstellen:!Matthäus  20)}
Ihr Glaube schien groß zu sein, denn
sie antworteten:\textit{ "`Ja! Wir können es!"'} worauf er erwiderte:
\index{Taufe} \index{Kelch!bitterer} \textit{"`Meinen Kelch
sollt ihr zwar trinken, und mit der Taufe, mit welcher ich getauft werde, sollt
auch ihr getauft werden;"'} ihre Belohnung aber stellte er seinem himmlischen
Vater anheim.

\section{3. Abschnitt} \label{kap4_ab3}

\index{Taufe} \index{Kelch!bitterer}
\textbf{Was war aber der Kelch, den er trank, und die Taufe die er erduldete?}
Ich
antworte: Sie bestanden darin, dass er durch die Kraft des ewigen Geistes sich
selbst überwand\footnote{\texttt{'verleugnete' durch 'überwand' ersetzt}} und
dem
Willen Gottes aufopferte, indem er sich allen
Trübsalen seines Lebens und den Schmerzen seines Kreuzestodes für das Heil der
Menschen willig unterwarf.

\section{4. Abschnitt} \label{kap4_ab4}

\label{ref:04_04_kelch_taufe_kreuz}
\index{Taufe} \index{Kelch!bitterer}
\textbf{Worin bestehet nun unser Kelch, den wir trinken müssen, die Taufe und
der
Kreuzestod, die wir zu erdulden haben? Darin: Dass wir durch die Kraft desselben
Geistes uns selbst überwinden\footnote{\texttt{'verleugnen' durch 'überwinden'
ersetzt}} und uns gänzlich hingeben, den Willen Gottes zu
seinem Dienste und zu seiner Verherrlichung zu tun, oder zu leiden. Dieses ist
das wahre Leben des Kreuzes Jesu, das im Gehorsam gegen dasselbe bestehet.
Freilich ist dies immer ein schmaler Weg, der aber zuvor noch ungebahnt war.
Denn, als
niemand da war, der helfen konnte, als keiner die Siegel \index{Siegel, Das} zu
öffnen und die wahre
Erkenntnis mitzuteilen verstand, um die Schritte zur Rettung des armen
verlornen Menschen zu leiten, -- da kam er mit der ganzen Fülle seiner Liebe und
Kraft, und, mit allen Schwachheiten sterblicher Menschen bekleidet, -- wiewohl
innerlich durch die Allmacht \index{Gott!Allmacht Gottes} des unsterblichen
Gottes
gestärkt, -- durchwanderte
er alle Beschwerden und Bedürfnisse der menschlichen Natur,
ebnete\footnote{\texttt{'brach' durch 'ebnete' ersetzt}} vor allen
andern zuerst den Weg\footnote{\texttt{'die Bahn' durch 'den Weg' ersetzt}} und
betrat den noch ungebahnten Pfad zur Herrlichkeit.}

\section{5. Abschnitt} \label{kap4_ab5}

\label{ref:04_05_besigen}
O kommt! Lasst uns ihm folgen, dem unermüdetsten, dem siegreichsten Heerführer
unsers Heils, gegen welchen alle großen Alexander \index{Personen:!Alexander}
und mächtigen Cäsarn der Welt \index{Personen:!Cäsarn}
weit geringer erscheinen, als die ärmsten Soldaten ihrer Heere
\index{Personen:!Soldaten} gegen sie sein
konnten. Jene waren freilich große Fürsten und auch Eroberer
\index{Personen:!Eroberer} ihrer Art, allein
nach ganz andern Grundsätzen. \textbf{Denn Christus erniedrigte sich selbst, um
das
Menschengeschlecht zu retten, jene hingegen richteten eine Menge Völker zu
Grunde, um ihr Ansehn zu erhöhen und ihren Ruhm zu vergrößern. Jene überwunden
andere, nur nicht sich selbst; Christus überwand die Selbstsucht, die sie
beständig besiegte. So war Christus, dem Verdienst nach, gewiss der erhabenste
Fürst und Eroberer. Auch vergrößerten jene ihre Reiche durch Raub und
Blutvergießen, Christus hingegen erweiterte das seinige durch Leiden und sanfte
Überredung. Er suchte nie seinen Zweck durch Zwang zu erreichen, sie verfolgten
allezeit ihre Absichten mit Gewalt. -- Elend und Sklaverei begleiteten alle ihre
Siege, die seinigen brachten seinen Überwundenen nur desto größere Freiheit und
Glückseligkeit.} In allen ihren Taten suchten sie nur sich selbst zu gefallen
in allem, was er tat, bezweckte\footnote{\texttt{'bezielte' ersetzt durch
'bezweckte'}}
er den Wohlgefallen seines himmlischen Vaters
der ein Gott über alle Götter, ein König über alle Könige, und ein Herr aller
Herrn ist.

\medskip

Dieses ist das vollkommenste Muster der
Selbstüberwindung\footnote{\texttt{'Selbstverleugnung'duch 'Selbstüberwindung'
ersetzt}}, dem wir
folgen
müssen, wenn wir je zur Herrlichkeit gelangen wollen. Und um dieses Ziel zu
erreichen, lasst uns die Eigenliebe ihrer wahren Bedeutung und ihrem ganzen
Umfange nach betrachten.

\section{6. Abschnitt} \label{kap4_ab6}

\index{Eigenliebe!unerlaubten}
\index{Eigenliebe!erlaubte}
\textbf{Es gibt eine erlaubte und eine unerlaubte Eigenliebe}, und die erstere
muss in
gewissen Fällen eben sowohl als die letztere um \textit{seinetwillen}
überwunden\footnote{\texttt{'verleugnet' durch 'überwunden' ersetzt}}
werden, der in der Unterwerfung unter den Willen Gottes nichts zu wert und zu
teuer schätzte, um uns zu erretten.

Es mögen sich freilich wohl nicht viele Menschen in der Welt finden, für welche
die Lehre von der Überwindung\footnote{\texttt{'Verleugnung' durch 'Überwindung'
ersetzt}} der erlaubten Eigenliebe geeignet ist, da der
größte Teil derselben sich Tag für Tag den Freuden der unerlaubten
Eigenliebe überlässt. Um aber das Ganze gehörigzu behandeln, und weil die
gegenwärtige Abhandlung vielleicht einigen in die Hände fallen mag, die in dem
geistlichen Kampfe so weit vorgedrungen sind, dass ihnen einiger Nutzen daraus
zuwachsen könnte, so will ich diesen Gegenstand doch wenigstens berühren.

\section{7. Abschnitt} \label{kap4_ab7}

\label{ref:04_07_vorteile}
\index{Eigenliebe!erlaubte}
\textbf{Die erlaubte Eigenliebe,} die wir zuweilen
überwinden\footnote{\texttt{'verleugnen' durch 'überwinden' ersetzt}} müssen,
besteht in dem
Genusse jener Bequemlichkeiten, Annehmlichkeiten und reichlichen Vorräten des
Lebens, die an sich keinesweges etwas Böses, sondern vielmehr \textbf{Wohltaten
und
Segnungen Gottes sind, z.B. Gatte und Gattin, Kinder, Häuser, Land, Ruf,
Freiheit und das Leben selbst. Dieses sind Gaben Gottes, die wir mit erlaubtem
Vergnügen genießen, veredeln, verbessern und rechtmäßig besitzen können. Wenn
aber Gott, der sie uns verlieh, sie zu irgend einer Zeit von uns fordert, oder
als der Darleiher das Seinige zurück verlangt, oder wenn es ihm gefällt,
dadurch, dass wir uns von ihnen trennen müssen, unsere Anhänglichkeit an ihn und
an jene Gegenstände zu prüfen, -- alsdann, sage ich, wenn sie mit ihm in
Konkurrenz\footnote{\texttt{'Vergleichung' durch 'Konkurrenz' ersetzt}} treten,
müssen
sie ihm nicht vorgezogen sondern überwunden\footnote{\texttt{'verleugnet' durch
'überwunden' ersetzt}} werden.}
Christus selbst verließ die Herrlichkeit seines Vaters und war freiwillig der
Verachtetste unter den Menschen, damit er uns Ehre bei Gott erwerbe und
\textit{"`wiewohl er in Gottes Gestalt war, und es nicht für einen Raub hielt,
Gott
gleich zu sein, so erniedrigte er sich doch selbst bis zur Knechtsgestalt
herab, ja bis zum schmachvollen Tode am Kreuz;"'}\footnote{Philipper 2,5-8}
\index{Bibelstellen:!Philipper 2)}
damit er
uns ein Beispiel der reinsten Demut und der vollkommensten Unterwerfung unter
den Willen unsere himmlischen Vaters gibt.

\section{8. Abschnitt} \label{kap4_ab8}

Dieses ist die Lehre, die er uns mit den Worten gibt:
\textbf{
\textit{"`Wer Vater oder Mutter mehr liebt, als mich, der ist meiner nicht
wert."'}\footnote{Matthäus  10,37}
\index{Bibelstellen:!Matthäus  10)}
Und bei einer anderen Gelegenheit:
\textit{"`Wer unter euch nicht allem absagt, was er hat, der kann mein Jünger
nicht sein."'}\footnote{Lukas 14,33} Auch dem reichen
\index{Bibelstellen:!Lukas 14)}
Jünglinge sagte er gerade heraus, wie er, wenn er das ewige Leben erlangen
wollte,
\textit{"`Alles verkaufen und ihm nachfolgen müsse."'}
\footnote{Markus 10,21-22}
\index{Bibelstellen:!Markus 10)}
}
Eine harte Lehre für ihn, und für alle, die, wie er, ungeachtet ihrer hohen
Ansprüche auf Religiosität, dennoch in der Wahrheit das, was sie besitzen, mehr
als Christum lieben. So stellt also die Lehre von der Selbstverleugnung die
Bedingungen fest, ohne deren Erfüllung keine ewige Glückseligkeit zu erwarten
ist, indem Christus ausdrücklich sagt:
\textbf{
\textit{"`Will mir jemand nachfolgen, der überwinde\footnote{\texttt{'verleugne'
durch 'überwinde' ersetzt}} sich selbst, nehme sein Kreuz auf und folge
mir."'}\footnote{Matthäus 16,24}}
\index{Bibelstellen:!Matthäus 16)}
Der tue, was ich tue, oder, der muss eben so handeln wie ich, sonst kann
er mir, dem Sohne Gottes, nicht ähnlich sein.

\section{9. Abschnitt} \label{kap4_ab9}

\label{ref:04_09_besitz}
Dieses bewog jene ehrlichen Fischer, ihr erlaubtes Gewerbe zu verlassen, und ihm
zu folgen, als er sie dazu aufforderte. Andere, welche auf die Tröstung Israels
\index{Orte:!Israel}
warteten, wurden dadurch bewogen, ihr Vermögen, ihren Ruf, ihre Freiheit und
selbst ihr Leben dem Unwillen und der Wut ihrer Verwandten und der
Regierungen, unter welchen sie lebten, preis zu geben, um die geistlichen
Vorteile zu genießen, die ihnen ihre getreue Anhänglichkeit an seine heilige
Lehre gewährten. \textbf{Es gab freilich viele, die, wie in dem Gleichnis vom
Abendmale, sich entschuldigen, dass sie ihm nicht nachfolgen könnten;} wo
nämlich
einige Land angekauft, einige geheiratet\footnote{\texttt{'Weiber genommen'
ersetzt
duch 'geheiratet'}}, andere Ochsen gekauft hatten und
deswegen nicht kommen konnten.\footnote{Lukas  14,18-20}
\index{Bibelstellen:!Lukas  14)}
Es ist klar, dass ihre
unmäßige Liebe zur Welt sie davon abhielt. \textbf{Ihre erlaubten Besitzungen,
die ihre
Diener hatten sein sollen, waren ihre Götzen \index{Götzen} geworden,} denen sie
mehr Verehrung
als Gott erwiesen, und welche sie nicht verlassen wollten, um zu Gott zu kommen.
Dieses ist ihnen zur Schande und zum Vorwurfe in den Urkunden der heiligen
Schrift ausgezeichnet worden, und wir können hieraus sehen, was für eine Macht
die Selbstliebe über den Weltmenschen hat, und welcher Gefahr er sich durch den
Missbrauch erlaubter Dinge aussetzt. -- \textbf{Wie? Ist deine Frau dir lieber
als dein
Heiland? Und ziehest du dein Land und deine Ochsen dem Heile deiner Seele vor?}
O
nimm dich in Acht, dass deine Güter dir nicht zuerst zu Schlingen dienen, und
hernach zum Fluch werden. Sie zu hoch zu schätzen, heißt den, der sie uns gab,
reizen und auffordern, sie uns wieder zu nehmen. Komm', folge ihm, der der Seele
ewiges Leben gibt.

\section{10. Abschnitt} \label{kap4_ab10}

\label{ref:04_10_pilger}
\index{Materialismus} Wehe denen, die ihre Herzen an irdische Güter hängen, denn
wenn diese
verschwinden, so verschwindet mit ihnen ihr Himmel. \index{Orte:!Himmel} Es ist
aber nur zu sehr die
Sünde des großen Haufens, dass er den Annehmlichkeiten der Welt anklebt, und es
ist wirklich bedauernswürdig anzusehen, wie sehr die Neigungen der Menschen von
der Sorge für ihre Bequemlichkeit eingenommen und ihre Gedanken mit der
Einrichtung derselben beschäftigt sind. \textbf{Wer sich wahrhaft selbst
überwindet\footnote{\texttt{'verleugnet' durch 'überwindet' ersetzt}}, ist
ein Pilger \index{Personen:!Pilger}, der Selbstsüchtige hingegen ein Ansiedler
in dieser
Welt. Jener
bedient sich ihrer, wie man Schiffe gebraucht, zur Überfahrt, um nach Hause zu
kommen. Der andere, -- was er auch immer schwatzen mag, -- sieht oder sorgt
nicht weiter, als wie er sich hier am besten im Überflusse niederlassen und in
Gemächlichkeit festsetzen möge und dieses liebt er so sehr, dass er, wenn es in
seiner Macht stünde, seinen Zustand nie vertauschen möchte. Er mag sich auch mit
den Gedanken an eine andere Welt nicht beunruhigen, bis er endlich überzeugt
wird, dass er nicht länger in dieser Welt leben kann.} Aber ach! Dann ist es zu
spät, wenn er nicht zu Abraham, sondern zu dem reichen Manne gewiesen wird,
dessen Geschichte eben so wahr wie traurig ist.

\section{11. Abschnitt} \label{kap4_ab11}


Von der anderen Seite betrachtet, ist es aber auch nicht umsonst, dass die
Jünger
Jesu sich selbst überwinden\footnote{\texttt{'verleugnen' durch 'überwinden'
ersetzt}}; Christus selbst hatte die ewige Freude im Auge, da er,
wie der Verfasser der Epistel an die Hebräer uns sagt, um der ihm vorgestellten
Freude willen
\textit{"`dass Kreuz erduldete;"'}\footnote{Hebräer 12,2}
\index{Bibelstellen:!Hebräer 12)}
das heißt, sich
selbst überwinden\footnote{\texttt{'verleugnete' durch 'überwinden' ersetzt}},
den
Spott der Gottlosen ertrug, von ihren grausamen Händen
den Tod erlitt, und die Schande, nämlich die Beschimpfung und Verhöhnung der
Welt, verachtete. Dieses alles machte ihn nicht furchtsam und schreckte ihn
nicht zurück, denn er achtete es nicht und sitzt nun aber auch zur rechten Hand
Gottes.

\medskip

\index{Gericht!Jüngstes}
Als Petrus zu ihm sagte:
\textit{"`Siehe, wir haben alles verlassen, und sind dir
nachgefolgt, was wird uns dafür?"'}
%sollte man nicht sagen: "was bekommen wir dafür?
% @Claus: In der Schlachter-Übersetzung heißt es:
% "`Da antwortete Petrus und sprach zu ihm: Siehe,
% wir haben alles verlassen und sind dir nachgefolgt;
% was wird uns dafür zuteil?"'
gab zur Aufmunterung seiner Jünger die tröstliche Antwort:
\index{Personen:!zwölf Geschlechter Israel}
\index{Personen:!zwölf Stämme Israel}
\textit{"`Wahrlich ich sage euch, dass ihr, die ihr mir in der
Wiedergeburt nachgefolgt seid, wenn des Menschen Sohn auf dem Throne seiner
Herrlichkeit sitzen wird, auch auf zwölf Thronen sitzen und die zwölf
Geschlechter Israel richten werdet,"'}\footnote{Matthäus  19,27-29}
\index{Bibelstellen:!Matthäus  19)}
die nämlich
damals sich in einem Zustande des Abfalls von dem Leben und der Kraft der
Gottseligkeit befanden. Dieses war das Los seiner Jünger, welche die
unmittelbaren Gefährten seiner Trübsale und die ersten Gesandten seines
Reiches waren. Was er gleich darauf sagt, geht aber alle an:
\textbf{ "`Und ein
jeder, der da verlässt Häuser, oder Brüder, oder Schwestern, oder Vater, oder
Mutter, oder Weib, oder Kinder, oder Äcker um meines Namens willen, der
soll es hundertfach empfangen und das ewige Leben erben."'\footnote{Matthäus
19,27-29}
\index{Bibelstellen:!Matthäus  19)}
Diese sichere Belohnung, diese
\textit{Krone der Gerechtigkeit} \index{Krone!der Gerechtigkeit}
war es, die in jedem Zeitalter in den Seelen der Gerechten
eine heilige Vernachlässigung, ja, eine gewisse Verachtung der Welt erzeugt
hat.} Dieser verdanken wir die Standhaftigkeit der Märtyrer,
\index{Personen:!Märtyrer}
so wie ihrem
Blute den Triumph der Wahrheit.
\label{ref:04_10_pilger_ende}

\section{12. Abschnitt} \label{kap4_ab12}

Auch ist dies keineswegs eine neue Lehre, sie schreibt sich schon von
Abraham \index{Personen:!Abraham} her. Sein Leben war ein Leben der
Selbstverleugnung, wie aus vielen
höchst merkwürdigen Beispielen hervorgehet. Erstlich, in dem er sein Vaterland
verließ, wo er, -- wie sich leicht vermuten läßt -- alles im Überfluss oder
doch wenigstens zur Genüge besaß. Und warum verließ er es? -- Weil Gott ihn
rief. Dieses sollte freilich immer ein hinreichender Grund in ähnlichen Fällen
sein, allein die Welt ist so sehr entartet, dass es sich in der Wirklichkeit
nicht so verhält. Denn, wenn jemand heutzutage dieselbe Handlung aus denselben
Beweggründen vollzöge, so würde man ihn deshalb verlachen, obgleich man sie bei
Abraham lobt und preist.
\label{ref:04_12_tugend}
\textbf{So sehr sind die Menschen schon gewohnt, nichts von
dem zu verstehen, was sie loben und empfehlen, ja, sogar dieselben Handlungen
bei ihren eigenen Zeitgenossen zu verachten, welche sie bei ihren Vorfahren zu
bewundern vorgeben.}

\section{13. Abschnitt} \label{kap4_ab13}

Abraham \index{Personen:!Abraham} war indessen dem göttlichen Befehle gehorsam,
und die Folge davon war,
dass ihm Gott ein großes Land gab. Dieses war die erste Belohnung seines
Gehorsams. Die nächste war ein Sohn in seinem hohen Alter, und zwar -- was die
Segnung noch erhöhte, -- als seine Frau, der Naturordnung gemäß, über die Zeit
des Kindergebärens hinans war.\footnote{1. Mose 22}
\index{Bibelstellen:!1. Mose 22)}
Dennoch verlangte Gott diesen
Liebling, ihr einziges Kind, die Freude ihres Alters, den Wundersohn, von dem
die Erfüllung der Verheißung abhing, welche Abraham gegeben war, diesen Sohn,
sage ich, verlangte Gott! -- Wahrlich eine große Prüfung, von der man wohl hätte
denken mögen, dass sie leicht im Stande gewesen wäre, den Glauben des Erzvaters
zu erschüttern und seine Standhaftigkeit wankend zu machen. Wenigstens hätte er
dabei in Zweifel geraten und ganz natürlich so schließen können: Dieser Befehl
ist unvernünftig und grausam, er kommt vom Versucher, nicht von Gott. Denn wie
läßt es sich denken, daß mir Gott einen Sohn geben würde, um ihn zum Opfer
darzubringen, -- dass der Vater der Mörder seines einzigen Kindes werden soll?
-- Und wie könnte Gott von mir verlangen, den Sohn seiner eigenen Verheißung,
durch welchen sein Bund errichtet werden soll, zu opfern? Das ist unglaublich!
So, sage ich, hatte Abraham ganz natürlich vernünfteln und schließen können, um
der Stimme Gottes zu widerstehen und seiner großen Zuneigung zu seinem geliebten
Isaak \index{Personen:!Isaak} nach zugeben. Aber der gute alte Abraham, der die
Stimme, die ihm einen
Sohn verheißen hatte, wohl kannte, hatte nicht vergessen, dieselbe auch dann zu
kennen, als sie diesen Sohn zuruckforderte. Er ging daher nicht mit Fleisch und
Blut zu Rate und zweifelte nicht, obgleich es sonderbar schien, und ihn, als
Mensch, vielleicht in einiges Erstaunen und Schrecken setzte. Er hatte gelernt
zu glauben, dass Gott, der ihm durch ein Wunder ein Kind gegeben habe, auch
wieder ein Wunder tun könne, um dasselbe zu erhalten oder wiederherzustellen.
Seine zärtlichen Neigungen konnten daher seine Pflicht nicht überwiegen, und
noch weniger seinen Glauben überwinden, denn er hatte seinen Sohn aus eine Art
empfangen, die ihn an nichts von dem zweifeln ließ, was Gott ihm von demselben
verheißen hatte.

\medskip

Darum beugte sich Ahraham in Unterwerfung unter den Befehl der Allmacht,
\index{Gott!Allmacht Gottes} er
bauete einen Altar, band seinen einzigen Sohn darauf, zündete das Feuer an, und
ergriff schon das Messer, um ihn zu schlachten, als plötzlich der Engel des
Herrn dem Todesstreich Einhalt tat: \textit{Halt! Abraham! Deine Aufrichtigkeit
ist
geprüft und bewährt erfunden!}-- Und was war nun die Folge? Ein Widder diente
dem Zweck, und Isaak war wieder sein. Dies zeigt uns, wie wenig da
hinreicht, wo alles dargebracht wird, und was für ein geringes Opfer den
Allmächtigen befriedigt, wenn die Gesinnungen des Herzens rechtschaffen erfunden
werden. \label{ref:04_13_opfer} \textbf{Es ist also nicht das Opfer, welches
das Herz Gott wohlgefällig macht,
sondern das Herz, das dem Opfer Annahme verschafft.} \index{Opfer}

\medskip

Gott rührt oft unsere besten Güter an und fordert das, was wir am meisten
lieben, und am wenigsten geneigt sind, zu verlassen. Nicht, dass er jedesmal es
uns gänzlich nehmen wolle, sondern um die Aufrichtigkeit unserer Herzen zu
prüfen, uns vor Übertreibungen zu warnen und uns zu erinnern, dass er der
Urheber aller unserer Segnungen ist und wir an dem, was wir besitzen, nicht mit
unsern Herzen hängen müssen. Ich spreche aus eigener Erfahrung. Das Mittel, die
Annehmlichkeiten unsers Lebens zu behalten, ist: Verzicht darauf zu tun. Dies
ist freilich schwer, aber es ist auch lieblich und angenehm, wenn sie uns
hernach, wie Isaak seinem Vater Abraham, noch liebevoller und
segensreicher wiedergegeben werden. -- \label{ref:04_13_weltliche_christen}
\textbf{O unsinnige Welt! O weltliche Christen!
Ihr seid nicht nur mit diesem vortrefflichen Glauben unbekannt, ihr seid sogar
seine Feinde! -- Und so lange es so mit euch steht, werdet ihr die Belohnung
desselben nie erfahren.}

\section{14. Abschnitt} \label{kap4_ab14}

Zunächst auf Abraham folgt Hiob \index{Personen:!Hiob}, dessen
Selbstüberwindung\footnote{\texttt{'Selbstverleugnung' durch 'Selbstüberwindung'
ersetzt}} gleichfalls
ausgezeichnet war. Denn als die Boten seines Trübsals eilig aufeinander
folgten und eine schmerzhafte Nachricht von seinen Verlusten nach der andern
einlief, bis er fast so nackt und bloß wie bei seiner Geburt war, da war das
erste, was er tat, dieses, dass er nieder fiel und die Macht anbetete und die
Hand küsste, die ihm alles genommen hatte. Ja, er war so weit entfernt, wider
Gott zu murren, dass er bei dem Verluste seiner Güter und aller seiner Kinder
ausrief:
\textit{"`Ich bin nackt von meiner Mutter Leibe gekommen, nackt werde ich
wieder dahinfahren. Der Herr hat es gegeben, der Herr hat es genommen; der Name
des Herrn sei gelobet."'}\footnote{Hiob 1, 21.}
\index{Bibelstellen:!Hiob 1)}
Welch' ein fester Glaube, welche
Geduld und Zufriedenheit dieses vortrefflichen Mannes! Man hätte glauben sollen,
die wiederholten Nachrichten von seinem Untergange wären hinreichend gewesen,
sein Vertrauen auf Gott umzustoßen; aber das geschah nicht, es blieb seine
Stütze. Auch erklärt er uns, warum? Sein Erlöser lebte.
\textit{"`Ich weiß,"'} sagt er,
\textit{"`dass mein Erlöser lebt!"'}\footnote{Hiob 19,25.}
\index{Bibelstellen:!Hiob 19)}
Und es zeigte sich
klar, dass er lebte, denn er hatte ihn von der Welt erlößt. Sein Herz hing
nicht an seinen irdischen Besitzungen, seine Hoffnung war sowohl über die
Freuden der Zeit als über die Leiden der Sterblichkeit erhaben, denn er war fest
überzeugt, dass er,
\textit{"`wenn gleich die Würmer seinen Leib verzehren würden
dennoch mit seinen Augen Gott schauen werde."'}\footnote{Hiob 19,26.}
\index{Bibelstellen:!Hiob 19)}
So war Hiobs
Herz nicht allein dem göttlichen Willen unterworfen, – sondern fand auch darin
seinen Trost. \index{Trost}

\section{15. Abschnitt} \label{kap4_ab15}

Das nächste große Beispiel einer merkwürdigen
Selbstverleugnung, welches wir in
der heiligen Geschichte vor den Zeiten der äußern Erscheinung Christi
aufgezeichnet finden, gibt uns Moses\index{Personen:!Mose}. Er war, als Kind,
durch eine
außerordentliche Bewahrung der göttlichen Vorsehung und wie aus den folgenden
Ereignissen hervorgehet, zu einem großen Zwecke erhalten worden. Die Tochter
Pharaos, \index{Personen:!Tochter
Pharao} deren Mitleid das Mittel zu seiner Erhaltung wurde, als der König die
Ermordung aller hebräischen Knaben befohlen hatte, nahm ihn als ihren Sohn zu
sich und gab ihm die Erziehung, welche an dem Hofe ihres Vaters üblich war.
Seine einnehmende Person und seine außerordentlichen Fähigkeiten, verbunden mit
der Liebe, die Pharaos Tochter für ihn hegte, und mit dem Einflusse, den sie,
hinsichtlich seiner Beförderung, auf ihren Vater hatte, würden ihn
wahrscheinlich in den Stand gesetzt haben, wo nicht Nachfolger, doch wenigstens
erster Staatsdiener dieses mächtigen und großen Fürsten zu werden, da
Ägypten \index{Orte:!Ägypten} damals das war, was später Athen
\index{Orte:!Athen} und Rom \index{Orte:!Rom} wurden: Ein Land, das
durch Gelehrsamkeit, Künste und Glanz sich vor allen andern Ländern am meisten
auszeichnete.

\section{16. Abschnitt} \label{kap4_ab16}

Allein Moses, der für ein anderes Werk bestimmt war und durch einen bessern
Stern, durch ein höheres Prinzip geleitet wurde, gelangte nicht sobald zu den
Jahren reifer Beurteilung, als die Gottlosigkeit Ägyptens und die in demselben
herrschende Unterdrückung seiner Brüder ihm zu einer Last wurde, die ihm zu
schwer zu ertragen fiel, und obgleich es einem so weisen und guten Manne, wie
Moses, nicht an jener edelmütigen und dankbaren Erkenntlichkeit mangeln
konnte, welche der ihm erwiesenen Güte einer Königstochter gebührte, so hatte er
doch auch den unsichtbaren Gott erkannt und wagte es daher nicht, in
Gemächlichkeit und im Überflusse an Pharaos Hofe zu leben, während seine armen
Brüder gezwungen werden sollten, Ziegel zu brennen, ohne dass man ihnen Stroh
dazu gäbe.\footnote{Hebräer 11,24) 2. Mose 5,7-16}
\index{Bibelstellen:!Hebräer 11)}
\index{Bibelstellen:!2. Mose 5)}

\medskip

Da also die Furcht des Allmächtigen sein Herz tief durchdrungen hatte, schlug er
es großmütig aus, ein Sohn der Tochter des Pharao genannt zu werden, und wählte
lieber ein Leben voller Trübsale mit den so sehr verachteten und unterdrückten
Israeliten und ein Gefährte aller ihrer Leiden und Gefahren zu sein, als eine
zeitlang die Ergötzlichkeiten der Sünde zu genießen, indem er die Schmach
Christi, die er wegen dieser unweltlichen Wahl erdulden musste, für größern
Reichtum als alle Schätze jenes Reichs hielt.

\section{17. Abschnitt} \label{kap4_ab17}

\label{ref:04_17_mose}
Moses handelte hierin auch nicht so töricht, als man meinte. Er hatte einen
guten vernünftigen Grund für sein Benehmen, denn es heißt von ihm:
\textit{"`Er sah auf die Belohnung hin."'}\footnote{Hebräer 11,26}
\index{Bibelstellen:!Hebräer 11)}
Er schlug nur einen kleinen
Vorteil aus, um einen größern zu erlangen. \textbf{In dieser Wahl übertraf
gewiss seine
Weisheit die der Ägypter, denn jene erwählten die gegenwärtige Welt, die doch so
ungewiss als das Wetter in ihr ist, und verloren dadurch jene, die ewig währet.}
Mose blickte tiefer und weiter, er wog die Genüsse dieses Lebens in der Wage der
Ewigkeit, \index{Ewigkeit, Die} und fand, dass sie da kein Gewicht hatten. Er
ließ sich nicht durch
einen augenblicklichen Besitz, sondern durch die Beschaffenheit und Dauer der
Belohnung bestimmen. \textbf{Sein Glaube hielt seine Neigungen im Zügel, und
lehrte ihn,
die Freuden der Selbstliebe der Hoffnung einer künftigen besseren Belohnung
aufzuopfern.}

\section{18. Abschnitt} \label{kap4_ab18}

\index{Personen:!Jesaja} Jesaias gibt uns gleichfalls ein nicht unbedeutendes
Beispiel dieser
segensvollen Selbstüberwindung\footnote{\texttt{'Selbstverleugnung' durch
'Selbstüberwindung' ersetzt}}. Aus einem Hofmanne wurde er ein Prophet, und
verleugnete die weltlichen Vorteile seines erstern Standes, um an dem Glauben,
der Geduld und den Leiden des leztern Teil zu haben. Durch diese Wahl verlor er
nicht allein die Gunst der Menschen, sondern ihre Gottlosigkeit, die durch seine
in scharfem und kühnem Tadel derselben sich laut aussprechende getreue
Anhänglichkeit an Gott aufs höchste in Wut gebracht war, machte ihn auch
endlich zum Märtyrer \index{Personen:!Märtyrer}. Er wurde unter der Regierung
des Könige
Manasse \index{Personen:!Manasse} auf eine
grausame Art mitten auseinander gesägt. So starb dieser große Mann, der auch
gewöhnlich der evangelische Prophet oder der Evangelist des alten
Bundes genannt wird.

\section{19. Abschnitt} \label{kap4_ab19}

Von vielen anderen Beispiele will ich nur noch eins anführen, nämlich von der
Treue Daniels \index{Personen:!Daniel}, dieses heiligen
\index{Personen:!Heilige} und
weisen Jünglings, der, sobald seine äußeren
Vorteile mit seinen Pflichten gegen den allmächtigen Gott in Widerspruch
gerieten, sie alle überwand\footnote{\texttt{'verleugnete' durch 'überwand'
ersetzt}} und aufgab, und statt auf seine eigene
Sicherheit bedacht zu sein, vielmehr gar nicht darauf achtete, sondern, ohne die
ihm drohende Gefahr zu scheuen, nur die Ehre Gottes durch getreue Erfüllung
seines Willens zu befördern strebte. Dieses setzte ihn freilich zuerst dem
Untergange aus, allein zuletzt erhob es ihn, -- als ein Beispiel zur großen
Aufmunterung für alle, die, wie er, in bösen Zeiten ein gutes Gewissen zu
bewahren trachten, -- zu hohem Ansehen in der Welt, so, dass durch seine
ausharrende Treue der Gott Daniels groß und furchtbar in den Augen der
heidnischen Könige wurde.

\section{20. Abschnitt} \label{kap4_ab20}

Was soll ich noch von allen übrigen sagen, die nichts für wert achteten, um den
Willen Gottes zu tun, die, so oft eine himmlische Erscheinung sie rief, aller
weltlichen Behaglichkeit entsagten, und ihre Ruhe und Sicherheit der Wut und
Bosheit entarteter Fürsten und einer abgefallenen Kirche
\index{Kirche!abgefallenen}
preisgaben. Unter
diesen befinden sich vornehmlich Jeremia \index{Personen:!Jeremia}, Ezechiel
\index{Personen:!Ezechiel} und Micha \index{Personen:!Micha}, welche, nach dem
sie im gegen die göttliche Stimme sich selbst
überwunden\footnote{\texttt{'verleugnet' durch 'überwunden' ersetzt, auch wenn
es in diesem Satz
etwas widersprüchlich klingt.}} hatten, ihre Zeugnisse
mit ihrem Blut versiegelten.

\medskip

\label{ref:04_20_opfertod}
\index{Opfertod} \index{Sühnetod} \textbf{Auf diese Weise war
Selbstüberwindung\footnote{\texttt{'Selbstverleugnug' durch 'Selbstüberwindung'
ersetzt}}
die beständige Übung und der Ruhm
unserer alten Vorfahren, welche Vorgänger der äußeren Erscheinung Christi waren.
Und wie können wir hoffen, jetzt ohne dieselbe in den Himmel
\index{Orte:!Himmel} zu kommen? Da
unser Heiland selbst das erhabenste Muster der
Selbstüberwindung\footnote{\texttt{'Selbstverleugnug'
durch 'Selbstüberwindung' ersetzt}} geworden ist;
und zwar nicht, -- wie einige es gern haben möchten, -- \textit{für uns, oder
statt
unserer}, so dass wir derselben nicht bedürften, sondern \textit{so für uns},
dass
wir uns eben so verleugnen und auf diese Art wahre Nachfolger seines heiligen
Vorbildes werden sollen?}

\section{21. Abschnitt} \label{kap4_ab21}

Wer du daher auch sein magst, der du den Willen Gottes gern tun wolltest, aber
durch weltliche Rücksichten in deinen Entschlüssen wankend geworden bist,
erinnere dich -- ich bitte dich im Namen Christi -- dass derjenige, der Vater
oder
Mutter, Schwester oder Bruder, Frau oder Kind, Haus oder Land, Ruf, Ehre, Amt,
Freiheit oder selbst das Leben dem Zeugnisse des Lichtes Jesu \index{Licht!Jesu}
in seinem
Gewissen vorziehet, an dem schauerlichen und allgemeinen Gerichtstage
\index{Gerichtstag} der Welt \label{ref:04_21_gericht}
von ihm verworfen werden wird, \textbf{wenn alle werden gerichtet werden, und
ein jeder
nach dem, wie er in diesem Leben gehandelt, nicht nach dem, was er hier mit dem
Munde bekannt \index{Bekenntnis} hat, die Vergeltung empfangen wird.} Jesus hat
gelehrt:
\textit{"`Wenn
dein rechtes Auge dich ärgert, so musst du es ausreißen, und wenn deine rechte
Hand dich ärgert, so musst du sie abhauen."'}\footnote{Matthäus  5,29-30}
\index{Bibelstellen:!Matthäus  5)}
Das heißt:
Wenn das Teuerste, das Nützlichste und was du am Zärtlichsten auf der Welt
liebst, dem Heile deiner Seele im Wege steht, deinen Gehorsam gegen die Stimme
Gottes unterbricht, und dich an der Gleichförmigkeit mit seinem in deinem Herzen
dir geoffenbarten Willen hindert, so bist du bei Strafe der Verdammnis
verpflichtet, diesen Dingen zu entsagen und dich von ihnen los zu machen.

\section{22. Abschnitt} \label{kap4_ab22}

\index{Eigenliebe} \label{ref:04_22_vernunft} \index{Gott!Weg Gottes}
\textbf{Der Weg Gottes ist ein Weg des Glaubens, eben so dunkel für die
Vernunft, \index{Vernunft} als
tödlich für die Eigenliebe.} Nur die Kinder!des Gehorsams,
\index{Personen:!Kinder!des Gehorsams} die mit dem heiligen
Paulus alles für Schaden und Unrat halten, damit sie Christum gewinnen mögen,
nur diese sind es, die den schmalen Weg kennen und auf demselben wandeln.
\textbf{Bloßes
beschauen und Grübeln kann die Sache nicht ausrichten, auch können verfeinerte
Begriffe den Eingang zu diesem Wege nicht öffnen.
\textit{"`Nur die Gehorsamen sollen das Gute des Landes
genießen."'}\footnote{Jesaja 1,19}}
\index{Bibelstellen:!Jesaja 1)}
Von denen, die bereit sind,
den Willen Gottes zu tun, sagt Jesus, dass sie seine Lehre erkennen
werden,\footnote{Johannes 7,17}
\index{Bibelstellen:!Johannes 7)}
diese will er unterrichten. Wo aber die Eigenliebe,
-- auch die erlaubte -- die Herrschaft hat und nicht untergeordnet ist, da kann
sein Unterricht nicht stattfinden. Der Eigenliebige kann ihn nicht annehmen und
das, was im Menschen unterrichtet werden soll, wird von der Eigenliebe
unterdrückt und furchtsam gemacht, und wagt es also nicht, zum Gehorsam zu
schreiten. -- O was wollte mein Vater oder meine Mutter sagen? Wie würde mein
Mann mich behandeln? Oder endlich: Wie würde die Obrigkeit mit mir verfahren?
Denn wenn ich auch von diesem und jenem eine ganz klare und kräftige
Überzeugung und völlige Gewissheit in meinem Herzen habe, und doch wieder
bedenke, wie ungebräuchlich es ist, was für Feinde die Sache hat, und was für
ein seltenes und sonderbares Ansehen ich mir dadurch geben würde, so hoffe ich,
Gott wird Mitleid mit meiner Schwachheit haben. Unterliege ich, so bin ich ja
nur Fleisch und Blut. Vielleicht wird Gott mich später besser in Stand
setzen und ich habe ja auch noch Zeit. So vernünftelt, so schließt der
eigenliebige, furchtsame Mensch.

\medskip

\label{ref:04_22_vernunft_und_ego}
\textbf{Nichts ist gefährlicher als ein solches Beratschlagen mit seiner
Selbstliebe.
Die Seele ist in solchen Unterhandlungen immer der verlierende Teil,
denn die
nötige Kraft, welche die Offenbarung des göttlichen Willens mit sich führt,
wird nur im Gehorsam gegen denselben gefunden.} Auch hat Gott nie jemand von
etwas überzeugt, ohne ihn nicht mit Kraft dazu auszurüsten, sobald er sich
seinem Willen unterwarf. \textbf{Er verlangt nichts, wozu er nicht auch die
Fähigkeit
verleiht, es tun.} Das hieße ja sonst die Menschen zu
veralbern\footnote{\texttt{'zum Besten haben' durch 'zu veralbern' ersetzt}} und
nicht sie
selig zu machen. Es ist aber genug, wenn du im Stande bist, deine Pflicht zu
tun, die Gott dir als solche anzeigt, und dieses wirst du können, insofern du
dich zu seinem Licht und Geist hältst, wodurch er dir jene Erkenntnis
erteilt. Diejenigen, denen es an Kraft mangelt, sind solche, die Christum nicht
durch Gehorsam gegen seine Überzeugungen in ihrem Herzen aufnehmen und diesen
wird es immer an Kraft fehlen. Diejenigen aber, die ihn so aufnehmen, empfangen
auch eben sowohl (als jene, in den ersten Zeiten) Macht, durch reinen Gehorsam
des Glaubens Gottes Kinder zu werden.

\section{23. Abschnitt} \label{kap4_ab23}

\label{ref:04_23_innere_stimme}
Darum bitte ich euch bei der Liebe und Barmherzigkeit Gottes, bei dem Leben und
Tode Christi, bei der Kraft seines Geistes und der Hoffnung der Unsterblichkeit,
dass ihr, die ihr mit euern Herzen an zeitlichen Genüssen hängt, und folglich
euch selbst mehr als die himmlischen Güter liebt: Laßt die bisher so
verbrachte Zeit nun genug sein. Haltet es nicht für hinreichend, dass ihr frei
von solchen Gottlosigkeiten seid, deren, leider! so viele andere sich schuldig
machen, so lange eure übermäßige Liebe zu erlaubten Dingen euern Genuss
derselben
befehlt, und eure Herzen von der Furcht und Liebe, von dem Gehorsam und der
Selbstüberwindung\footnote{\texttt{'Selbstverleugnung' durch 'Selbstüberwindung'
ersetzt}} der wahren Jünger Jesu abzieht. -- \textbf{Lenke du nun in den
rechten Weg ein, und achte auch auf die leise Stimme, die in deinem Gewissen
redet.
Diese sagt dir, worin deine Sünden bestehen und was für Elend sie zur Folge
haben. Sie gibt dir eine klare Einsicht in das überaus eitle Wesen der Welt,
sie eröffnet deiner Seele einen Blick in die Ewigkeit und in die Glückseligkeit
der Gerechten, die zu ihrer Ruhe eingegangen sind. Wenn du dich hieran hältst,
so wird es dich von der Sünde und sinnlichen Eigenliebe scheiden, dann wirst du
bald finden, dass die Macht der Reihe dieser Vorstellung diejenige des
Reichtums, der Ehre und der Pracht der Welt weit übertrifft und dass dein
Gehorsam gegen diese innere Stimme \index{Stimme!innere}\footnote{\texttt{Diese
'innere Stimme' wurde von anderen Quakern unter anderen
auch 'innerer Christus' genannt. W. Penn verstand sie auch als die Offenbarung
Gottes. Also höchste Autorität. Noch über der Bibel stehent. Die Bibel
ist zwar Zeugnis des Heiligen Geistes, aber diese 'innere Stimme' ist
der Geist Gottes selbst (in 'Personalunion' mit Jesus und Gott)}}
%wichtiges Wort, "innere Stimme" sollte erklärt werden.
% @Claus: Ist der Kommentar in deinem Sinne?
dir endlich eine Gemütsruhe gewährt, welche
die Stürme der Zeit nicht erschüttern, nie zerstören können.}Dann werden alle
deine Genüsse dir gesegnet sein, die, so gering sie auch sein mögen, durch die
Gegenwart dessen, der in und mit ihnen ist, dennoch groß sein werden.
\label{ref:04_23_innere_stimme_ende}

\medskip

\label{ref:04_23_dinge_der_welt}
\textbf{Selbst in dieser Zeit haben die Gerechten viel voraus, indem sie die
Güter der
Welt gebrauchen, ohne dabei Gewissensvorwürfe zu empfinden, weil sie dieselben
nicht missbrauchen. Sie sehen und preisen die Hand, die sie ernährt, kleidet und
erhält, und da sie den Geber in allen seinen Gaben erkennen, so beten sie nicht
diese, sondern ihn an.} Auf diese Art ist der angenehme Genuss seiner Segnungen
ein Vorzug, den sie jenen voraus haben, die ihn in seinen Gaben nicht
erkennen. Überdies können sie weder in ihrem Wohlstande übermütig, noch im
Missgeschick niedergeschlagen sein, und die Ursache davon ist: Weil seine
göttliche Gegenwart bei dem Genuss des ersteren sie in Schranken hält und in
dem lezteren sie tröstet.

\medskip

\textbf{Kurz, der Himmel ist der Thron, und die Erde nur der Fußschemel auch
desjenigen
Menschen, der seine Eigenliebe unter die Füße gebracht hat.} Die, welche diesen
Standpunkt erreicht haben, lassen sich nicht leicht das Ziel verrücken, diese
lernen ihre Tage zählen, damit die Stunde ihrer Auslösung sie nicht überrasche,
sie erkaufen ihre Zeit, weil die Tage böse sind,\footnote{Epheser 5,15-16}
\index{Bibelstellen:!Epheser 5)}
indem
sie bedenken, dass sie bloß Haushalter sind, und einem unparteiischen Richter
Rechenschaft zu geben haben. Darum leben sie nicht sich selbst, sondern ihm und
in ihm sterben sie, selig mit denen, die in dem Herrn sterben. Hiermit schließe
ich nun diese Abhandlung über den rechten Gebrauch der erlaubten Eigenliebe.







