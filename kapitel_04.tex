
\chapter{4. Kapitel}
\section{Zusammenfassung des 4. Kapitel}
\footnotesize
\begin{description}
\item[1. Abschnitt] Worin destehet die große Wirkung des Kreuzes? Die
Beantwortung dieser Frage ist von hoher Wichtigkeit.
\item[2. Abschnitt] Die Wirkung des Kreuzes ist Selbstverleugnung.
\item[3. Abschnitt] Worin bestand der Kelch, den Christus trank, und das Kreuz,
das er trug?
\item[4. Abschnitt]  Was ist unser Kelch und unser Kreuz.
\item[5. Abschnitt] Es ist unsere Pflicht, Christo, als dem Herzoge unserer
Seligkeit, zu folgen.
\item[6. Abschnitt] Von dem Unterschiede zwischen der erlaubten und unerlaubten
Eigenliebe.
\item[7. Abschnitt] Erklärung der erlaubten Eigenliebe.
\item[8. Abschnitt] Sie muß, nach Christi Lehre und Beispiele, in einigen Fällen
verleugnet werden.
\item[9. Abschnitt] Auch nach dem Vorbilde der Apostel.
\item[10. Abschnitt] Von der Gefahr, worin Diejenigen sich besinden, die ihre
Eigenliebe ihren Pflichten gegen Gott vorziehen.
\item[11. Abschnitt] Belohnung der Selbstverleugnung; Aufmunterung derselben.
\item[12. Abschnitt] Diese Lehre ist so alt als Abrahams Zeitalter.
\item[13. Abschnitt] Abrahams Gehorsam des Glaubens ist sehr merkwürdig.
\item[14. Abschnitt] Hiob gab ein großes Beispiel der Selbstverleugnung; - son
seiner Zufriedenheit.
\item[15. Abschnitt] Auch Moses gab ein merkwürdiges Beispiel der
Selbstverleugnung; -- von feiner Verachtung des Hofes Pharao's.
\item[16. Abschnitt] Seine Wahl.
\item[17. Abschnitt] Sein Beweggrund dazu; nämlich die Hoffnung der Belohnung.
\item[18. Abschnitt] Jesaias gab kein geringeres Beispiel; da er aus einem
Hofmanne ein Prophet des Herrn ward.
\item[19. Abschnitt] Diese Beispiele schließen mit dem, welches Daniel gab;
seine Geduld und Aufrichtigkeit; und was für einen Eindruck sein Betragen auf
den König machte.
\item[20. Abschnitt] Es könnten noch viele andere Beispiele zur Bestätigung
dieser vortrefflichen Lehre angeführt werden.
\item[21. Abschnitt] Man muß um Christi willen Alles verlassen, wenn man selig
werden will.
\item[22. Abschnitt] Gottes Weg ist ein Weg des Glaubens und der
Selbstverleugnung.
\item[23. Abschnitt] Ernste Bitte und Ermahnung an Alle, daß sie diese Dinge
beherzigen mögen.
\end{description}
\normalsize


Die vierte Frage ist nun: worin bestehet die große Wirkung, die das Kreuz in dem
Menschen hervorbringt?

\section{1. Abschnitt}

Diese Frage, der Wahrheit gemäß, deutlich und vollständig zu beantworten, ist in
der That von so hoher Wichtigkeit, daß alles Vorhergegangene nur als Einleitung
dazu zu dienen scheint; indem eine unrichtige Beantwortung derselben nichts
Geringeres als ein Irreleiten der Seele auf ihrem Wege zur Seligkeit seyn würde.
Jch werde daher diese Frage unter dem Beistande Gottes, nach der besten
Erkenntniß, die er mir während meiner mehrjährigen Nachfolge Christi durch
Erfahrung davon gegeben hat, gründlich erörtern.

\section{2. Abschnitt}

Das große Werk, welches das Kreuz Christi im Menschen hervorbringt, ist
Selbstverleugnung. Ein Ausdruck, der an sich selbst von eben so tiefer Bedeutung
ist, als die Sache, die er bezeichnet, im empfindlichsten Widerspruche mit der
Welt stehet; eine Sache, die so wenig von der Welt verstanden, und noch weniger
von ihr angenommen wird, und der man sich dessen ungeachtet dennoch unterwerfen
muß. Der Sohn Gottes ist uns darin vorangegangen, und hat uns durch den bittern
Kelch, den er trank, und durch die Taufe, die er erduldete, ein Vorbild
hinterlassen, wie wir seinen Fußstapfen nachfolgen sollen. Als einst des
''Zebedäus'' Ehefrau ihn bat, daß in seinem Reiche einer von ihren Söhnen zu
seiner Rechten und der andere zu seiner Linken sitzen möchte, legte er ihr und
ihren beiden Söhnen die schwere Frage vor: "`Könnet ihr den Kelch trinken, den
ich trinken werde, und tcuch mit der Taufe taufen lassen, mit welcher ich
getauft werde?"'\footnote{Matth. 20,21-23} Ihr Glaube schien groß zu seyn; denn
sie antworteten: "`Ja! wir können es!"' worauf er erwiederte: "`Meinen Kelch
sollt ihr zwar trinken, und mit der Taufe, mit welcher ich getauft werde, sollt
auch ihr getauft werden;"' ihre Belohnung aber stellte er seinem himmlischen
Vater anheim.

\section{3. Abschnitt}

Was war aber der Kelch, den er trank, und die Taufe die er erduldete? Ich
antworte: Sie bestanden darin, daß er durch die Kraft des ewigen Geistes sich
selbst verleugnete und dem Willen Gottes aufopferte; indem er sich allen
Trübsalen seines Lebens und den Schmerzen feines Kreuzestodes für das Heil der
Menschen willig unterwarf.

\section{4. Abschnitt}

Worin bestehet nun unser Kelch, den wir trinken müssen, die Taufe und der
Kreuzestod, die wir zu erdulden haben? Darin: daß wir durch die Kraft desselben
Geistes uns selbst verleugnen und uns gänzlich hingeben, den Willen Gottes zu
seinem Dienste und zu seiner Verherrlichung zu thun, oder zu leiden. Dieses ist
das wahre Leben des Kreuzes Jesu, das im Gehorsame gegen dasselbe bestehet.
Freilich immer ein schmaler Weg, der aber zuvor noch ungebahnet war. Denn, als
Niemand da war, der helfen konnte; als Keiner die Siegel zu öffnen und die wahre
Erkenntniß mitzutheilen verstand, um die Schritte zur Rettung des armen
verlornen Menschen zu leiten; -- da kam er mit der ganzen Fülle seiner Liebe und
Kraft; und, mit allen Schwachheiten sterblicher Menschen bekleidet, -- wiewohl
innerlich durch die Allmacht des unsterblichen Gottes gestärkt, -- durchwanderte
er alle Beschwerden und Bedürfnisse der menschlichen Natur, brach vor allen
Andern zuerst die Bahn, und betrat den ungebahnten Pfad zur Herrlichkeit.

\section{5. Abschnitt}

O kommt! Laßt uns ihm folgen, dem unermüdetsten, dem siegreichsten Heerführer
unsers Heils, gegen welchen alle großen Alexander und mächtigen Cäsarn der Welt
weit geringer erscheinen, als die ärmsten Soldaten ihrer Heere gegen sie seyn
konnten. Jene waren freilich große Fürsten und auch Eroberer ihrer Art, allein
nach ganz andern Grundsätzen. Denn Christus erniedrigte sich selbst, um das
Menschengeschlecht zu retten; jene hingegen richteten eine Menge Völker zu
Grunde, um ihr Ansehn zu erhöhen und ihren Ruhm zu vergrößern. Jene überwunden
Andere, nur nicht sich selbst; Christus überwand die Selbstsucht, die sie
beständig besiegte. So war Christus, dem Verdienste nach, gewiß der erhabenste
Fürst und Eroberer. Auch vergrößerten jene ihre Reiche durch Raub und
Blutvergießen; Christus hingegen erweiterte das seinige durch Leiden und sanfte
Ueberredung. Er suchte nie seinen Zweck durch Zwang zu erreichen; sie verfolgten
allezeit ihre Absichten mit Gewalt. -- Elend und Sklaverei begleiteten alle ihre
Siege; die seinigen brachten seinen Ueberwundenen nur desto größere Freiheit und
Glückseligkeit. In allen ihren Thaten suchten sie nur sich selbst zu gefallen;
in Allem, was er that, bezielte er den Wohlgefallen seines himmlischen Vaters
der ein Gott über alle Götter, ein König über alle Könige, und ein Herr aller
Herrn ist.

Dieses ist das vollkommenste Muster der Selbstverleugnung, dem wir folgen
müssen, wenn wir je zur Herrlichkeit gelangen wollen; und um dieses Ziel zu
erreichen, laßt uns die ''Eigenliebe'' ihrer wahren Bedeutung und ihrem ganzen
Umfange nach betrachten.

\section{6. Abschnitt}

Es giebt eine erlaubte und eine unerlaubte Eigenliebe, und die erstere muß in
gewissen Fällen eben sowohl als die letztere um ''Seinetwillen'' verleugnet
werden, der in der Unterwerfung unter den Willen Gottes Nichts zu werth und zu
theuer schätzte, um uns zu erretten.

Es mögen sich freilich wohl nicht viele Menschen in der Welt finden , für welche
die Lehre von der Verleugnung der ''erlaubten Eigenliebe'' geeignet ist; da der
größte Theil derselben sich Tag für Tag den Freuden der ''unerlaubten
Eigenliebe'' überläßt. Um aber das Ganze gehörigzu behandeln, und weil die
gegenwärtige Abhandlung vielleicht Einigen in die Hände fallen mag, die in dem
geistlichen Kampfe so weit vorgedrungen sind, daß ihnen einiger Nutzen daraus
zuwachsen könnte; so will ich diesen Gegenstand doch wenigstens berühren.

\section{7. Abschnitt}

Die ''erlaubte Eigenliebe'', die wir zuweilen verleugnen müssen, bestehet in dem
Genusse jener Bequemlichkeiten, Annehmlichkeiten und reichlichen Vorräthe des
Lebens, die an sich keinesweges etwas Böses, sondern vielmehr Wohlthaten und
Segnungen Gottes sind, z.B. Gatte und Gattin, Kinder, Häuser, Land, Ruf,
Freiheit und das Leben selbst. Dieses sind Gaben Gottes, die wir mit erlaubtem
Vergnügen genießen, veredeln, verbessern und rechtmößig besitzen können. Wenn
aber Gott, der sie uns verlieh, zu irgend einer Zeit sie. von uns fordert, oder
als der Darleiher das Seinige zurück verlangt, oder wenn es ihm gefällt,
dadurch, daß wir uns von ihnen trennen müssen, unsere Anhänglichkeit an ihn und
an jene Gegenstände zu prüfen, -- alsdann, sage ich, wenn sie mit ihm in
Vergleichung treten, müssen sie ihm nicht vorgezogen sondern verleugnet werden.
Christus selbst verließ die Herrlichkeit seines Vaters und ward freiwillig der
Verachtetste unter den Menschen; damit er uns Ehre bei Gott erwürbe; und
"`wiewohl er in Gottes Gestalt war, und es nicht für einen Raub hielt, Gott
gleich"' "`zu seyn, so erniedrigte er sich doch selbst bis zur Knechtsgestalt
herab, ja bis zum schmachvollen Tode am Kreuze;"'\footnote{Phil. 2,5-8} damit er
uns ein Beispiel der reinsten Demuth und der vollkommensten Unterwerfung unter
den Willene unsere himmlischen Vaters gäbe.

\section{8. Abschnitt}

Dieses ist die Lehre, die er und in den Worten giebt: "`Wer Vater oder Mutter
mehr liebt, als mich, der ist meiner nicht werth."'\footnote{Matth. 10,37} Und
bei einer andern Gelegenheit: "`Wer unter euch nicht Allem absagt, was er hat,
der kann mein Jünger nicht seyn."'\footnote{Luk. 14,33} Auch dem reichen
Jünglinge sagte er gerade heraus, wie er, wenn er das ewige Leben erlangen
wollte, "`Alles verkaufen und ihm nachfolgen müßte."'\footnote{Mark. 10,21-22}
Eine harte Lehre für ihn, und für Alle, die, wie er, ungeachtet ihrer hohen
Ansprüche aus Religiosität, dennoch in der Wahrheit das, was sie besitzen, mehr
als Christum lieben. So stellt also die Lehre von der Selbstverleugnung die
Bedingungen fest, ohne deren Erfüllung keine ewige Glückseligkeit zu erwarten
ist; indem Christus ausdrücklich sagt: ''"`Will mir Jemand nachfolgen, der
verleugne sich selbst, nehme sein Kreuz auf, und folge mir."'''\footnote{Matth.
16,24} Der thue, was ich thue; oder, der muß eben so handeln wie ich, sonst kann
er mir, dem Sohne Gottes, nicht ähnlich seyn.

\section{9. Abschnitt}

Dieses bewog jene ehrlichen Fischer, ihr erlaubtes Gewerbe zu verlassen, und ihm
zu folgen, als er sie dazu aufforderte. Andere, welche auf die Tröstung Israels
warteten, wurden dadurch bewogen, ihr Vermögen, ihren Ruf, ihre Freiheit und
selbst ihr Leben dem Unwillen und der. Wuth ihrer Verwandten und der
Regierungen, unter welchen sie lebten, preis zu geben, um die geistlichen
Vortheile zu genießen, die ihnen ihre getreue Anhänglichkeit an seine heilige
Lehre gewährten. Es gab freilich Viele, die, wie in dem Gleichnisse vom
Abendmahle, sich entschuldigen, daß sie ihm nicht nachfolgen könnten; wo nämlich
Einige Land angekauft, Einige Weiber genommen, Andere Ochsen gekauft hatten, und
deswegen nicht kommen konnten.\footnote{Luk. 14,18-20} Es ist klar, daß ihre
unmäßige Liebe zur Welt sie davon abhielt. Ihre erlaubten Besitzungen, die ihre
Diener hatten seyn sollen, waren ihre Götzen geworden, denen sie mehr Verehrung
als Gott erwiesen, und welche sie nicht verlassen wollten, um zu Gott zu kommen.
Dieses ist ihnen zur Schande und zum Vorwurfe in den Urkunden der heiligen
Schrift ausgezeichnet worden, und wir können hieraus sehen, was für eine Macht
die Selbstliebe über den Weltmenschen hat, und welcher Gefahr er sich durch den
Mißbrauch erlaubter Dinge aussetzt. -- Wie? ist deine Fran dir lieber als dein
Heiland? und ziehest du dein Land und deine Ochsen dem Heile deiner Seele vor? O
nimm dich in Acht, daß deine Güter dir nicht zuerst zu Schlingen dienen, und
hernach zum Fluche werden. Sie zu hoch zu schätzen, heißt den, der sie und gab,
reizen und auffordern, sie uns wieder zu nehmen. Komm, folge ihm, der der Seele
ewiges Leben giebt.

\section{10. Abschnitt}

Wehe denen, die ihre Herzen an irdische Güter hängen; denn wenn diese
verschwinden, so verschwindet mit ihnen ihr Himmel. Es ist aber nur zu sehr die
Sünde des großen Haufens, daß er den Annehmlichkeiten der Welt anklebt; und es
ist wirklich bedauernswürdig anzusehen, wie sehr die Neigungen der Menschen von
der Sorge für ihre Bequemlichkeit eingenommen und ihre Gedanken mit der
Einrichtung derselben beschäftigt sind. Wer sich wahrhaft selbst verleugnet, ist
ein Pilger, der Selbstsüchtige hingegen, ein Ansiedler in dieser Welt. Jener
bedient sich ihrer, wie man Schiffe gebraucht, zur Ueberfahrt, um nach Hause zu
kommen. Der Andere, -- was er auch immer schwatzen mag, -- sieht oder sorgt
nicht weiter, als wie er sich hier am besten im Ueberflusse niederlassen und in
Gemächlichkeit festsetzen möge; und dieses liebt er so sehr, daß er, wenn es in
seiner Macht stände, seinen Zustand nie vertauschen möchte. Er mag sich auch mit
den Gedanken an eine andere Welt nicht beunruhigen, bis er endlich überzeugt
wird, daß er nicht länger in dieser Welt leben kann. Aber ach! dann ist es zu
spät, wenn er nicht zu Abraham, sondern zu dem reichen Manne gewiesen wird,
dessen Geschichte eben so wahr als traurig ist.

\section{11. Abschnitt}

Von der andern Seite betrachtet, ist es aber auch nicht umsonst, daß die Jünger
Jesu sich verleugnen; Christus; selbst hatte die ewige Freude im Auge, da er,
wie der Verfasser der Epistel an die Hebräer uns sagt, um der ihm vorgestellten
Freude willen "`daß Kreuz erduldete;"'\footnote{Hebr 12,2} das heißt, sich
selbst verleugnete, den Spott der Gottlosen ertrug, von ihren grausamen Händen
den Tod erlitt, und die Schande, nämlich die Beschimpfung und Verhöhnung der
Welt, verachtete. Dieses Alles machte ihn nicht furchtsam und schreckte ihn
nicht zurück, denn er achtete es nicht; und sitzt nun aber auch zur rechten Hand
Gottes.

Als Petrus zu ihm sagte: "`Siehe, wir haben Alles verlassen, und sind dir
nachgefolgt; was wird uns dafür?"' gab zur Aufmunterung seiner Jünger die
trostliche Antwort: "`Wahrlich ich sage euch, daß ihr, die ihr mir in der
Wiedergeburt nachgefolgt seid, wenn des Menschen Sohn auf dem Throne seiner
Herrlichkeit sitzen wird, auch auf zwölf Thronen sitzen und die zwölf
Geschlechter Israel richten werdet,"'\footnote{Matth. 19,27-29} die nämlich
damals sich in einem Zustande des Abfalles von dem Leben und der Kraft der
Gottseligkeit befanden. Dieses war das Loos seiner Jünger, welche die
unmittelbaren Gefährten seiner Trübsale und die ersten Gesandten seines
Reiches waren. Was er gleich darauf sagt, gehet aber alle an: "`Und ein
Jeder, der da verläßt Häuser, oder Brüder, oder Schwestern, oder Vater, oder
Mutter, oder Weib, oder Kinder, oder Aecker um meines Namens willen, der
soll es hundertfältig empfangen und das ewige Leben ererben."'
\footnote{Matth. 19,27-29} Diese sichere Belohnung, diese ''Krone der
Gerechtigkeit'' war es, die in jedem Zeitalter in den Seelen der Gerechten
eine heilige Vernachlässigung, ja, eine gewisse Verachtung der Welt erzeugt
hat. Dieser verdanken wir die Standhaftigkeit der Märtyrer, so wie ihrem
Blute den Triumph der Wahrheit.

\section{12. Abschnitt}

Auch ist dies keineswegs eine neue Lehre; sie schreibt sich schon von
''Abraham'' her. Sein Leben war ein Leben der Selbstverleugnung, wie aus vielen
höchst merkwürdigen Beispielen hervorgehet. Erstlich, in dem er sein Vaterland
verließ, wo.er, -- wie sich leicht vermuthen läßt, -- Allee im Ueberflusse, oder
doch wenigstens zur Genüge besaß. Und warum verließ er es? -- Weil Gott ihn
rief. Dieses sollte freilich immer ein hinreichender Grund in ähnlichen Fällen
seyn; allein die Welt ist so sehr entartet, daß es sich in der Wirklichkeit
nicht so verhält. Denn, wenn Jemand heut zu Tage dieselbe Handlung aus denselben
Beweggründen vollzöge, so würde man ihn deßhalb verlachen, obgleich man sie bei
Abraham lobt und preiset. So sehr sind die Menschen schon gewohnt, nichts von
dem zu verstehen, was sie loben und empfehlen, ja, sogar dieselben Handlungen
bei ihren eigenen Zeitgenossen zu verachten, welche sie bei ihren Vorfahren zu
bewundern vorgeben.

\section{13. Abschnitt}

Abraham war indessen dem göttlichen Befehle gehorsam, und die Folge davon war,
daß ihm Gott ein großes Land gab. Dieses war die erste Belohnung seines
Gehorsams. Die nächste war ein Sohn in seinem hohen Alter; und zwar -- was die
Segnung noch. erhöhete, -- als seine Frau, der Naturordnung gemäß, über die Zeit
des Kindergebärenö hinans war.\footnote{1 Mos. 22} Dennoch verlangte Gott diesen
Liebling, ihr einziges Kind, die , Freude ihres Alters, den Wundersohn, von dem
die Erfüllung der Verheißung abhing, welche Abraham gegeben war; diesen Sohn,
sage ich, verlangte Gott! -- Wahrlich eine große Prüfung, von der man wohl hätte
denken mögen, daß sie leicht im Stande gewesen wäre, den Glauben des Erzvaters
zu erschüttern und seine Standhaftigkeit wankend zu machen. Wenigstens hätte er
dabei in Zweifel gerathen und ganz natürlich so schließen können: dieser Befehl
ist unvernünftig und grausam; er kommt vom Versucher, nicht von Gott. Denn wie
läßt es sich denken, daß mir Gott einen Sohn geben würde, um ihn zum Opfer
darzubringen, -- daß der Vater der Mörder seines: einzigen Kindes; werden solle?
-- Und wie könnte Gott von mir verlangen, den Sohn seiner eigenen Verheißung,
durch welchen sein Bund errichtet werden soll, zu opfern? das ist unglaublich!
So, sage ich, hatte Abraham ganz natürlich overnünfteln und schließen können, um
der Stimme Gottes zu widerstehen und seiner großen Zuneigung zu seinem geliebten
Isaak nach zugeben. Aber der gute alte ''Abrah, der die Stimme, die ihm einen
Sohn verheißen hatte, wohl kannte, hatte nicht vergessen, dieselbe auch dann zu
kennen, als sie diesen Sohn zuruckforderte. Er ging daher nicht mit Fleisch und
Blut zu Rathe, und zweifelte nicht, obgleich es sonderbar schien, und ihn, als
Mensch, vielleicht in einiges Erstaunen und Schrecken setzte. Er hatte gelernt
zu glauben, daß Gott, der ihm durch ein Wunder ein Kind gegeben habe, auch
wieder ein Wunder thun könne, um dasselbe zu erhalten oder wiederherzustellen.
Seine zärtlichen Neigungen konnten daher seine Pflicht nicht überwiegen, und
noch weniger seinen Glauben überwinden; denn er hatte seinen Sohn aus eine Art
empfangen, die ihn an nichts von dem zweifeln ließ, was Gott ihm von demselben
verheißen hatte.

Darum beugte sich Ahraham in Unterwerfung unter den Befehl der Allmacht; er
bauete einen Altar, band seinen einzigen Sohn darauf, zündete das Feuer an, und
ergriff, schon das Messer, um ihn zu schlachten, als plötzlich der Engel des
Herrn dem Todesstreicht Einhalt that: Halt! Abraham! deine Aufrichtigkeit ist
geprüft und bewährt erfunden! -- Und was war nun die Folge? Ein Widder diente
dem Zwecke, und ''Isaak'' war wieder sein. Dieß zeigt uns, wie wenig da
hinreicht, wo Alles dargebracht wird, und was für ein geringes Opfer den
Allmächtigen befriedigt, wenn die Gesinnungen des Herzens rechtschaffen erfunden
werden. Es ist also nicht das Opfer, welches das Herz Gott wohlgefällig macht;
sondern das Herz, das dem Opfer Annahme verschafft.

Gott rührt oft unsere besten Gitter an, und fordert das, was wir am meisten
lieben, und am wenigsten geneigt sind, zu verlassen. Nicht daß er jedesmal es
uns gänzlich nehmen wolle; sondern um die Aufrichtigkeit unserer Herzen zu
prüfen, uns vor Uebertreibungen zu warnen, und uns zu erinnern, daß Er der
Urheber aller unserer Segnungen ist, und wir an dem, was wir besitzen, nicht mit
unsern Herzen hängen müssen. Ich spreche aus eigener Erfahrung. Das Mittel, die
Annehmlichkeiten unsers Lebens zu behalten, ist: Verzicht darauf zu thun. Dieß
ist freilich schwer, aber es ist auch lieblich und angenehm, wenn sie uns
hernach, wie ''Isaak'' feinem Vater ''Abraham'', noch liebevoller und
segensreicher wiedergegeben werden. -- O! unsinnige Welt! O! weltliche Christen!
Ihr seid nicht nur mit diesem vortrefflichen Glauben unbekannt; ihr seid sogar
feine Feinde! -- Und so lange es so mit euch stehet, werdet ihr die Belohnung
desselben nie erfahren.

\section{14. Abschnitt}

Zunächst auf ''Abraham'' folgt Hiob, dessen Selbstverleugnung gleichfalls sehr
ausgezeichnet war. Denn als die Boten seiner Trübsales eilig auf einander
folgten, und eine schmerzhafte Nachricht von seinen Verlusten nach der andern
einlief, bis er fast so nackend und bloß als bei seiner Geburt war, da war das
erste, was er that, dieses, daß er nieder fiel und die Macht anbetete, und die
Hand küßte, die ihm Alles genommen hatte. Ja, er war so weit entfernt, wider
Gott zu murren, daß er bei dem Verluste seiner Güter und aller seiner Kinder
ausrief: "`Ich bin nackend von meiner Mutter Leibe gekommen, nackend werde ich
wieder dahinfahren. Der Herr hat es gegeben, der Herr hat es genommen; der Name
des Herrn sei gelobet."'\footnote{Hiob 1, 21.} Welch ein fester Glaube, welche
Geduld und Zufriedenheit dieses vortrefflichen Mannes! Man hätte glauben sollen,
die wiederholten Nachrichten von seinem Untergange wären hinreichend gewesen,
sein Vertrauen auf Gott umzustoßen; aber das geschah nicht; es blieb seine
Stütze. Auch erklärt er uns, warum? Sein Erlöser lebte. "`''Ich weiß'',"' sagt
er, "`''daß mein Erlöser lebt!''"'\footnote{Kap 19, 25.} Und es zeigte sich
klar, daß er lebte; denn er hatte ihn von der Welt erlösset. Sein Herz hing
nicht an seinen irdischen Besitzungen; seine Hoffnung war sowohl über die
Freuden der Zeit als über die Leiden der Sterblichkeit erhaben; denn er war fest
überzeugt, daß er, "`wenn gleich die Würmer seinen Leib verzehren würden;
dennoch mit seinen Augen Gott schauen werde."'\footnote{V. 26.} So war ''Hiobs''
Herz nicht allein dem göttlichen Willen unterworfen, – sondern fand auch darin
seinen Trost.

\section{15. Abschnitt}

Das nächste große Beispiel einer merkwürdigen Selbstverleugnung, welches wir in
der heiligen Geschichte vor den Zeiten der äußern Erscheinung Christi
aufgezeichnet finden, giebt uns Moses. Er war, als Kind, durch eine
saußerordentliche Bewahrung der göttlichen Vorsehung, und wie. aus den folgenden
Ereignissen hervorgehet, zu einem großen Zwecke erhalten worden. Die ''Tochter
Pharao’s'', deren Mitleid das Mittel zu seiner Erhaltung ward, als der König die
Ermordung aller hebräischen Knaben befohlen hatte, nahm ihn als ihren Sohn zu
sich, und gab ihm die Erziehung, welche an dem Hofe ihres Vaters üblich war.
Seine einnehmende Person und seine außerordentlichen Fähigkeiten, verbunden mit
der Liebe, die Pharao’s Tochter für ihn hegte, und mit dem Einflusse, den sie,
hinsichtlich seiner Beförderung, auf ihren Vater hatte, würden ihn
wahrscheinlich in den Stand gesetzt haben, wo nicht Nachfolger, doch wenigstenis
erster Staatsdiener dieses mächtigen und großen Fürsten zu werden; da
''Egypten'' damals das war, was nachmalß Athen und Rom wurden: ein Land, das
durch Gelehrsamkeit, Künste und Glanz sich vor allen andern Ländern am meisten
auszeichnete.

\section{16. Abschnitt}

Allein Mosed, der für ein anderes Werk bestimmt war, und durch einen bessern
Stern, durch ein höheres Prinzip geleitet wurde, gelangte nicht sobald zu den
Jahren reifer Beurtheilung, als die Gottlosigkeit Egyptens und die in demselben
herrschende Unterdrückung seiner Brüder ihm zu einer Last wurde, die ihm zu
schwer zu ertragen fiel; und obgleich es einem so weisen und guten Manne, wie
''Moses'', nicht an jener edelmüthigen und dankbaren Erkenntlichkeit mangeln
konnte, welche der ihm erwiesenen Güte einer Königstochter gebührte; so hatte er
doch auch den unsichtbaren Gott erkannt, und wagte es daher nicht, in
Gemächlichkeit und im Ueberflusse an Pharao's Hofe zu leben, während seine armen
Brüder gezwungen werden sollten, Ziegel zu brennen, ohne daß man ihnen Stroh
dazu gäbe.\footnote{Hebr. 11,24 2 Mose 5,7-16}

Da also die Furcht des Allmächtigen sein Herz tief durchdrungen hatte, schlug er
es großmüthig aus, ein Sohn der Tochter Pharaoe genannt zu werden, und wählte
lieber ein Leben voller Trübsale mit den so sehr verachteten und unterdrückten
Israeliten, und ein Gefährte aller ihrer Leiden und Gefahren zu seyn, als eine
zeitlang die Ergötzlichkeiten der Sünde zu genießen; indem er die Schmach
Christi, die er wegen dieser unweltlichen Wahl erdulden mußte, für größern
Reichthum als alle Schätze jenes Reichs hielt.

\section{17. Abschnitt}

Mofes handelte hierin auch nicht so thöricht, als man wähnte. Er hatte einen
guten vernünftigen - Grund für sein Benehmen, denn es heißt von ihm: "`Er sah
aus die Belohnung hin."'\footnote{Hebr. 11,26} Er schlug nur einen kleinen
Vortheil aus, um einen größern zu erlangen. In dieser Wahl übertraf gewiß seine
Weisheit die der Egypter; denn jene erwählten die gegenwärtige Welt, die doch so
ungewiß als das Wetter in ihr ist, und verloren dadurch jene, die ewig währet.
Mose blickte tiefer und weiter; er wog die Genüsse dieses Lebens in der Wage der
Ewigkeit, und fand, daß sie da kein Gewicht hatten. Er ließ sich nicht durch
einen augenblicklichen Besitz, sondern durch die Beschaffenheit und Dauer der
Belohnung bestimmen. Sein Glaube hielt seine Neigungen im Zügel, und lehrte ihn,
die Freuden der Selbstliebe der Hoffnung einer künftigem bessern Belohnung
aufzuopfern.

\section{18. Abschnitt}

Jesaias giebt uns gleichfalls ein nicht unbedeutendes Beispiel dieser
segensvollen Selbstverleugnung. Aus einem Hofmanne ward er ein Prophet, und
verleugnete die weltlichen Vortheile seines erstern Standes, um an dem Glauben,
der Geduld und den Leiden des leztern Theil zu haben. Durch diese Wahl verlor er
nicht allein die Gunst der Menschen, sondern ihre Gottlosigkeit, die durch seine
in scharfem und kühnem Tadel derselben sich laut aussprechende getreue
Anhänglichkeit an Gott aufs höchste in Wuth gebracht war, machte ihn auch
endlich zum Märtyrer. Er ward unter der Regierung, des Könige Manasse auf eine
grausame Art mitten von einander gesägt. So starb dieser große Mann, der auch
gewöhnlich der ''evangelische Prophet'', oder der ''Evangelist des alten
Bundes'' genannt wird.

\section{19. Abschnitt}

Von Vielen andern Beispielen will ich nur noch eins anführen, nämlich von der
Treue Daniels, dieses heiligen und weisen Jünglings, der, sobald seine äußern
Vortheile mit seinen Pflichten gegen den allmächtigen Gott in Widerspruch
geriethen, sie alle verleugnete und aufgab, und statt auf seine eigene
Sicherheit bedacht zu seyn, vielmehr gar nicht darauf achtete, sondern, ohne die
ihm drohende Gefahr zu scheuen, nur die Ehre Gottes durch getreue Erfüllung
seines Willens zu befördern strebte. Dieses setzte ihn freilich zuerst dem
Untergange aus; allein zuletzt erhob es ihn, -- als ein Beispiel zur großen
Aufmunterung für Alle, die, wie er, in bösen Zeiten ein gutes Gewissen zu
bewahren trachten, -- zu hohem Ansehen in der Welt; so, daß durch seine
ausharrende Treue der Gott Daniels groß und furchtbar in den Augen der
heidnischen Könige ward.

\section{20. Abschnitt}

Was soll ich noch von allen Uebrigen sagen, die nichts zu werth achteten, um den
Willen Gottes zu thun; die, so oft eine himmlische Erscheinung sie rief, aller
weltlichen Behaglichkeit entsagten, und ihre Ruhe und Sicherheit der Wuth und
Bosheit entarteter Fürsten und einer abgefallenen Kirche preisgaben. Unter
diesen befinden sich vornehmlich Jeremias, Ezechiel und Micha, welche, nach dem
sie im gegen die göttliche Stimme sich selbst verleugnet hatten, ihre Zeugnisse
mit ihrem Blute versiegelten.

Auf diese Weise war ''Selbstverleugnug'' die beständige Uebung und der Ruhm
unserer alten Vorfahren, welche Vorgänger der äußern Erscheinung Christi waren.
Und wie können wir hoffen, jetzt ''ohne die selbe'' in den Himmel zu kommen? da
unser Heiland selbst das erhabenste Muster der Selbstverleugnung geworden ist;
und zwar nicht, -- wie Einige es gern haben möchten, -- ''für uns, oder statt
unserer'', so daß wir derselben nicht bedürften; sondern ''so für uns'', daß wir
uns eben so verleugnen und auf diese Art wahre Nachfolger seiner heiligen
Vorbildes werden sollen?

\section{21. Abschnitt}

Wer du daher auch seyn magst, der du den Willen Gottes gern thun wolltest, aber
durch weltliche Rücksichten in deinen Entschlüssen wankend geworden bist,
erinnere dich, ich bitte dich im Namen Christi, daß derjenige, der Vater oder
Mutter, Schwester oder Bruder, Weib oder Kind, Haus oder Land, Ruf, Ehre, Amt,
Freiheit oder selbst das Leben dem Zeugnisse des Lichtes - Jesu in seinem
Gewissen vorziehet, an dem schauerlichen und allgemeinen Gerichtstage der Welt
von ihm verworfen werden wird, wenn Alle werden gerichtet werden, und ein Jeder
nach dem, wie er in diesem Leben gehandelt, nicht nach dem, was er hier mit dem
Munde bekannt hat, die Vergeltung empfangen wird. Jesus hat gelehret: "`Wenn
dein rechtes Auge dich ärgert, so mußt du es ausreißen, und wenn deines rechte
Hand dich ärgert, so mußt du sie abhauen."'\footnote{Matth. 5,29-30} Das heißt:
Wenn das Theuerste, das Nützlichste, und was du am zärtlichsten auf der Welt
liebst, dem Heile deiner Seele im Wege stehet, deinen Gehorsam gegen die Stimme
Gottes unterbricht, und dich an der Gleichförmigkeit mit seinem in deinem Herzen
dir geoffenbarten Willen hindert, so bist du bei Strafe der Verdammniß
verpflichtet, diesen Dingen zu entsagen und dich von ihnen los zu machen.

\section{22. Abschnitt}

Der Weg Gottes ist ein Weg des Glaubens; eben so dunkel für die Vernunft, als
tödtlich für die Eigenliebe. Nur die Kinder des Gehorsams, die mit dem heiligen
Paulus Alles für Schaden und Unrath halten, damit sie Christum gewinnen mögen,
nur diese sind es, die den schmalen Weg kennen und auf demselben wandeln. Bloßes
beschauen und Grübeln kann die Sache nicht ausrichten; auch können verfeinerte
Begriffe den Eingang zu diesem Wege nicht öffnen. "`Nur die Gehorsamen sollen
das Gute des Landes genießen."'\footnote{Jes. 1,19} Von denen, die bereit sind,
den Willen Gottes zu thun, sagt Jesus, daß sie seine Lehre erkennen
werden;\footnote{Joh. 7,17} diese will er unterrichten. Wo aber die Eigenliebe,
-- auch die erlaubte, - die Herrschaft hat, und nicht untergeordnet ist, da kann
sein Unterricht nicht stattfinden. Der Eigenliebige kann ihn nicht annehmen; und
das, was im Menschen unterrichtet werden soll, wird von der Eigenliebe
unterdrückt und furchtsam gemacht, und wagt es also nicht, zum Gehorsam zu
schreiten. -- O! was wollte mein Vater oder meine Mutter sagen? Wie würde mein
Mann mich behandeln? oder endlich: wie würde die Obrigkeit mit mir verfahren?
Denn wenn ich auch von Diesem und Jenem eine ganz klare und kräftige
Ueberzeugung und völlige Gewißheit in meinem Herzen habe, und doch wieder
bedenke, wie ungebräuchlich es ist, was für Feinde die Sache hat, und was für
ein seltenes und sonderbares Ansehen ich mir dadurch geben würde; so hoffe ich,
Gott wird Mitleid mit meiner Schwachheit haben. Unterliege ich, so bin ich ja
nur Fleisch und Blut. Vielleicht wird Gott mich späterhin besser in Stand
setzen; und ich habe ja auch noch Zeit. So vernünftelt, so schließt der
eigenliebige, furchtsame Mensch.

Nichts ist gefährlicher als ein solches Berathschlagen mit seiner Selbstliebe.
Die Seele ist in solchen Unterhandlungen immer der verlierende Theil; denn die
nöthige Kraft, welche die Offenbarung des göttlichen Willens mit sich führt,
wird nur im Gehorsame gegen den selben gefunden. Auch hat Gott nie Jemand von
Etwas überzeugt, ohne ihn nicht mit Kraft dazu auszurüsten, sobald er sich
seinem Willen unterwarf. Er verlangt Nichts, wozu er nicht auch die Fähigkeit
verleihet, es thun. Das hieße ja sonst die Menschen. zum Besten haben nicht, sie
selig zu machen. Es ist aber genug, wenn du im Stande bist, deine Pflicht zu
thun, die Gott dir als solche anzeigt; und dieses wirst du können, insofern du
dich zu seinem Lichte und Geiste hältst, wodurch er dir jene Erkenntniß
ertheilt. Diejenigen, denen es an Kraft mangelt, sind Solche, die Christum nicht
durch Gehorsam gegen seine Ueberzeugungen in ihrem Herzen aufnehmen, und diesen
wird es immer an Kraft fehlen. Diejenigen aber, die ihn so aufnehmen, empfangen
auch, eben sowohl als Jene in den ersten Zeiten, Macht, durch reinen Gehorsam
des Glaubens Gottes Kinder zu werden.

\section{23. Abschnitt}

Darum bitte ich euch bei der Liebe und Barmherzigkeit Gottes, bei dem Leben und
Tode Christi, bei der Kraft seines Geistes und der Hoffnung der Unsterblichkeit,
daß ihr, die ihr mit euern Herzen an zeitlichen Genüssen hängt, und folglich
euch selbst mehr als die himmlischen Güter liebt, -- laßt die bisher so
verbrachte Zeit nun genug seyn. Haltet es nicht für hinreichend, daß ihr frei
von solchen Gottlosigkeiten seid, deren, leider! so viele Andere sich schuldig
machen, so lange eure übermäßige Liebe zu erlaubten Dingen euern Genuß derselben
befledt, und eure Herzen von der Furcht und Liebe, von dem Gehorsame und der
Selbstverleugnung derb wahren Jünger Jesu abziehet. -- Lenke du nun in den
rechten Weg ein, und achte auch die leise Stimme, die in deinem Gewissen redet.
Diese sagt dir, worin deine Sünden bestehen, und was für Elend sie zur Folge
haben. Sie giebt dir eine klare Einsicht in das überaus eitle Wesen der Welt;
sie eröffnet deiner Seele einen Blick in die Ewigkeit und in die Glückseligkeit
der Gerechten, die zu ihrer Ruhe eingegangen sind. Wenn du dich hieran hältst,
so wird es dich von der Sünde und sinnlichen Eigenliebe scheiden; dann wirst du
bald finden, daß die Macht der Reihe dieser Vorstellung diejenige des
Reichthums, der Ehre und der Pracht der Welt weit übertrifft, und daß. dein
Gehorsam gegen diese innere Stimme dir endlich eine Gemüthzruhe gewährt, welche
die Stürme der Zeit nicht erschüttern , nie zerstören können. Dann werden alle
deine Genüsse dir gesegnet seyn, die, so gering sie auch seyn mögen, durch die
Gegenwart Dessen, der in und mit ihnen ist, dennoch groß seyn werden.

Selbst in dieser Zeit haben die Gerechten viel voraus, indem sie die Güter der
Welt gebrauchen, ohne dabei Gewissensvorwürfe zu empfinden, weil sie dieselben
nicht mißbrauchen. Sie sehen und preisen die Hand, die sie ernähret, kleidet und
erhält; und da sie den Geber in allen seinen Gaben erkennen, so beten sie nicht
diese, sondern Ihn an. Auf diese Art ist der angenehme Genuß seiner Segnungen
ein Vorzug, den sie vor Jenen voraus haben, die ihn in seinen Gaben nicht
erkennen. Ueberdieß können sie weder in ihrem Wohlstande übermüthig, noch im
Mißgeschicke niedergeschlagen seyn, und die Ursache davon ist: weil seine
göttliche Gegenwart bei dem Genusse des erstern sie in Schranken halt, und in,
dem leztern sie tröstet.

Kurz, der Himmel ist der Thron, und die Erde nur der Fußschcmer auch desjenigen
Menschen, der seine Eigenliebe unter die Füße gebracht hat. Die, welche diesen
Standpunkt erreicht haben, lassen sich nicht leicht das Ziel verrücken; diese
lernen ihre Tage zählen, damit die Stunde ihrer Auslösung sie nicht überrasche;
sie erkaufen ihre Zeit, weil die Tage böse sind,\footnote{Eph. 5,15-16} indem
sie bedenken, daß sie bloß Haushalter sind, und einem unpartheiischen Richter
Rechenschaft zu geben haben. Darum leben sie nicht sich selbst, sondern Ihm und
in Ihm sterben sie, selig mit denen, die in dem Herrn sterben. Hiermit schließe
ich nun diese Abhandlung über den rechten Gebrauch der erlaubten Eigenliebe.






