\section*{F}

\articlesize

\begin{description}

\item[Facing Benches]
    Older meetinghouses often have benches on a raised platform which face the rest of the congregation where Weighty Friends (see below) who might be expected to speak would sit. Historically (and in some meetings still) these would be the recorded ministers and elders
    
 \item[Fell, Margaret] $\ast$1614 in Marsh Grange, \dag23.4.1702 in Swarthmore. Gelegentlich auch als \textit{"`Mother of Quakerism"'} -- zu Deutsch: \textit{"`Mutter des Quäkertums"'} bezeichnet. 1632 heiratete sie Thomas Fell ($\ast$1598 \dag1658). Dieser war Richter und gehörte dem \textit{"`Long Parliament"'} unter $\to$Oliver Cromwell an. Im Juni 1652 traf sie zum ersten mal $\to$George Fox. Nach dem Tod ihres ersten Mannes heiratete sie am 27.10.1669 George Fox in Bristol. Margaret Fell war Autorin des $\to$\textit{"`histrischen Friedenszeugnisses"'} und maßgeblich mit der Organisation der Missionsreisen beschäftigt.


 \item[Fell, Margaret] $\ast$1614 in Marsh Grange, \dag23.4.1702 in Swarthmore. Gelegentlich auch als \textit{"`Mother of Quakerism"'} -- zu Deutsch: \textit{"`Mutter des Quäkertums"'} bezeichnet. 1632 heiratete sie Thomas Fell ($\ast$1598 \dag1658). Dieser war Richter und gehörte dem \textit{"`Long Parliament"'} unter $\to$Oliver Cromwell an. Im Juni 1652 traf sie zum ersten mal $\to$George Fox. Nach dem Tod ihres ersten Mannes heiratete sie am 27.10.1669 George Fox in Bristol. Margaret Fell war Autorin des $\to$\textit{"`histrischen Friedenszeugnisses"'} und maßgeblich mit der Organisation der Missionsreisen beschäftigt.


 \item[Fell, Margaret] $\ast$1614 in Marsh Grange, \dag23.4.1702 in Swarthmore. Gelegentlich auch als \textit{"`Mother of Quakerism"'} -- zu Deutsch: \textit{"`Mutter des Quäkertums"'} bezeichnet. 1632 heiratete sie Thomas Fell ($\ast$1598 \dag1658). Dieser war Richter und gehörte dem \textit{"`Long Parliament"'} unter $\to$Oliver Cromwell an. Im Juni 1652 traf sie zum ersten mal $\to$George Fox. Nach dem Tod ihres ersten Mannes heiratete sie am 27.10.1669 George Fox in Bristol. Margaret Fell war Autorin des $\to$\textit{"`histrischen Friedenszeugnisses"'} und maßgeblich mit der Organisation der Missionsreisen beschäftigt.

 \item[Finlayson, James]

\item[Finsternis] $\to$\textit{dark countries}

 \item[Fisher, Mary] $\ast$1623 \dag1698. Quakerin. Wollte 1657 zusammen mit fünf anderen Quakern den Sultan von Konstantinopel bekehren.

 \item[Fox, Geroge] $\ast$Juli 1624 in Drayton-in-the-Clay, Leicestershire, heute Fenny Drayton, \dag13.1.1691. War einer der Gründerväter der Quaker und erfolgreicher Missionar.

 \item[Freund]

 \item[Free Quakers]
% <li><b><i></i></b> <KBD>( http://www.qhpress.org/quakerpages/qwhp/freequakertoc.htm )</KBD> Das wiederum ist ein insich abgeschlossenes Schismar. Entstanden während des Amerikanischen Bürgerkriegs. Nämlich eine Abspaltung von Quäkern, die im Bürgerkrieg nicht nur Verbal sondern auch mit der Waffe in der Hand Partei ergriffen. Sich somit vom "Friedenszeugnis" los sagten. Heute gibt es sie nicht mehr. Ausgestorben.</li>
 
 \item[Freunde der Freunde]

 \item[Friedensthal]

 \item[Friends General Conference]

 \item[Friends World Committee for Consultation]

 \item[Friends United Meeting (FUM)]

 \item[Frühe Freund]

 \item[Fry, Elizabeth] $\ast$21. 5.1780 in Cartham Hall bei Norwich, \dag12.10.1845 in Ramsgate. Quäkerin. War britische Reformerin des Gefängniswesens.

 \item[Fuchs Emil]


 \end{description}

\normalsize