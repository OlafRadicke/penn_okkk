\section*{F}

\articlesize

\begin{description}

\item[Facing Benches] In älteren Versammlungshäusern (englisch: meeting haus)
gibt es eine spezelle anordnung der Sitzbänge. Dabe gibt es eine einzelne Bank
die etwas erhoben ist und den anderen Bänken gegenüber steht. Dort sitzen die
so genannten \textit{"`gewichtien Freunde"'} ($\to$\textit{Weighty Friends}),
von denen zu erwarten ist, das sie in der Andacht das Wort ergreifen und zur
Versammlung sprechen werden. Das sind oft und traditonell die Ältestent
($\to$\textit{Ältester}) und die $\to$\textit{recorded ministers}. Heute ist
diese Form der Bestuhlung unüblich geworden. Der liberale Flügel
($\to$\textit{Liberal Friends}) verzichtet meist völlig auf
\textit{recorded ministers} und der \textit{evangelikale} beziehungsweise
\textit{pastoralen Fügel} ($\to$\textit{Evangelical Friends},
$\to$\textit{Pastoral Friends}) hat im Grunde jetzt seine Pastoren.
\endnote{Artikel "`Religious Society of Friends"'. Abgerufen 2010, January 12.
In Wikipedia, Die freie Enzyklopädie. Bearbeitungsstand: 16:41 Uhr 2010-01-13, \\ \texttt{http://en.wikipedia.org/w/index.php?title=Religious\_Society\_of\_Friends\&oldid=337403234}}


 \item[Fell, Margaret] $\ast$1614 in Marsh Grange, \dag23.4.1702 in Swarthmore.
 Gelegentlich auch als \textit{"`Mother of Quakerism"'} (\textit{"`Mutter des
 Quäkertums"'}) bezeichnet. 1632 heiratete sie Thomas Fell ($\ast$1598
 \dag1658). Dieser war Richter und gehörte dem \textit{"`Long Parliament"'}
 unter $\to$Oliver Cromwell an. Im Juni 1652 traf sie zum ersten mal
 $\to$George Fox. Nach dem Tod ihres ersten Mannes heiratete sie am 27.10.1669
 George Fox in Bristol. Margaret Fell war Autorin unter anderem des
 $\to$\textit{"`histrischen Friedenszeugnisses"'} und maßgeblich mit der
 Organisation der Missionsreisen beschäftigt.
\medskip 
 Sie war unermüdlich damit
 beschäftigt Quaker aus der Gefangenschaft zu befreiehen und Rechtsbeistand vor
 gericht zu organisieren. Sie sammelte und verteilte Gelder um in Notgeratene
 und verfolgte Quakern zu helfen. Sie selbst war mehr mal im Gefängnis. Das
 erste mal 1663 für die Verweigerung des Eids vor Gericht. 1670
 sorgte sogar ihr eigener Sohn, wegen Erbstreitigkeiten, für ihre Inhaftierung.
 Bis 1684 kam sie immer wieder in gefängnis dafür, weil sie sich weigerte die
 örtlicher Gottesdienste zu besuchen (heute unvorstellbar, weil dann Heute ~80\%
 der Bundesbürger im Gefängnis sitzen würde). Die Bedeutung ihrer Person kann
 man daran sehen, das noch heute 500 an sie adressierte Briefe, aus den Jahren
 der schlimmsten Verfolgiung(1654 bis 1670) erhalten sind.

 \item[Finlayson, James] $\ast$vermutlich am 29. August 1772 in Penicuik,
 Schottland; \dag vermutlich am 18. August 1852 in Edinburgh. Schottischer
 Quäker, der wesentlich zum Aufbau der finnischen Textilindustrie beitrug. In
 inischen Tampere des Jahres 1820, beaufsichtigte er den Aufbau eines
 gewaltigen Industriekomplexes, dessen Werke von der Wasserkraft des gestauten
 Tammerkoski angetrieben wurden. Zunächst wurde eine Wollweberei errichtet,
 1828 dann eine Baumwollmühle und zwei Baumwollspinnereien. Aber auch
 karitative Projekte wie die Einrichtung eines Waisenhauses witmete er sich.
 Obwohl Finlayson schon früh kontakte zu Quäkern hatte, trat er selbst erst
 sehr späht der $\to$\textit{Gesellschaft} bei.

\item[Finsternis] $\to$\textit{dark countries}

 \item[Fisher, Mary] $\ast$1623 \dag1698. Quakerin. Wollte 1657 zusammen mit
 fünf anderen Quakern den Sultan von Konstantinopel bekehren. Selbiger empfing
 sie auch freundlich, hörte sich an was sie zu sagen hatte und liess sie dann
 wieder nach hause fahren.

 \item[Five Years Meeting] Vorläuferorganisation von $\to$\textit{Friends
 United Meeting}.

 \item[Fox, Geroge] $\ast$Juli 1624 in Drayton-in-the-Clay, Leicestershire,
 heute Fenny Drayton, \dag13.1.1691. War einer der Gründerväter der Quaker und
 erfolgreicher Missionar. Wie fiele Andere, sass fiele mal im Gefängnis. Sein
 wichtigstes Verdienst ist wohl, die administrative Grundlage zu schaffen, nach
 der sich bis Heute die Versammlungen organisieren. Sehr bekannt ist er auch
 als Autor der des "`The Journal of George Fox"', in den er seine Erlebnisse
 während seiner Missions-Jahre schildert.

 \item[Free Quakers] Eine Abspaltung von Quäkern, die von ihren Versammlungen
 ausgeschlossen wurden, weil sie sich an den amerikanischen Unabhängigkeiskrig
 beteiligten. Als Gründer ist Samuel Wetherill bekannt. Das einzigste
 Versammlungshaus von ihnen steht noch heute in der 5th und Arch Street in
 Philadelphia. Es wurde mit prominenter Unterstützung wie Washington und
 Benjamin Franklin 1783 errichtet. Es war ein Prizip der \textit{Free Quakers}
 niemanden aus zu schiessen. weder aus theologischen noch aus moralischen
 Gründen. Nachdem einige Jahre niemand mehr die Andachten besucht hatte, wurde
 das Versammlungshaus 1834 geschlossen.
\endnote{\texttt{http://www.qhpress.org/quakerpages/qwhp/freequakertoc.htm}}
\endnote{Claus Bernet: "`Samuel Wetherill"' In: Biographisch-Bibliographisches
Kirchenlexikon (BBKL). Band 30, Nordhausen 2009, ISBN 978-3-88309-478-6,
Spalte~1555–1557.}


 \item[Freund(e)] Zum Teil bezeichnen sich Quaker untereinander als "`Freund"'. Streckenweise wurde der Begriff inflationär oder floskelhaft benutzt, was
 als unaufrichtig empfunden wurde, so das es einige ablehnen, grundsätzlich
 aller Quaker als "`Freunde"' zu bezeichnen. Abgeleitet wird das von der
 Bezeichnung "`Religiöse Gesellschaft der Freunde"'. Und so benutzen viele
 Quaker unreflektiert "`Freund"' als stehenden Begriff für "`Mitglied"'.
 
 \item[Freunde der Freunde] Sind Menschen die mit dem Quakertum symparisieren
 und engen kontakt zu Quakern pflegen, aber selber keine Quaker sind.

 \item[Friedensthal] Eine ehemalige deutsche Quäkerkolonie, die von 1792 bis
 1870 im heutigen Bad Pyrmonter Ortsteil Löwensen existierte. Begründer und
 wichtigste Figur war $\to$\textit{Ludwig Seebohm}.

 \item[Friends] $\to$\textit{Freund}

 \item[Friends House] $\to$\textit{Meeting House}

 \item[Friends General Conference (FGC)] Gehört neben der
 $\to$\textit{Friends United Meeting} und $\to$\textit{Evangelical Friends
 International} zu einer der drei größten Quäkerdachorganisation. Die FGC
 vereint unter sich Quäkergruppen mit unprogrammierter Andacht. Die Mitglieder
 der FGC vertreten mehr zu liberalen theologischen Ansichten
 ($\to$\\textit{Liberal Friends}). \endnote{Artikel "`Friends General
 Conference"'. In: Wikipedia, Die freie Enzyklopädie. Bearbeitungsstand:
 13.6.2009, 12:54 UTC. URL: \\ \texttt{http://de.wikipedia.org/w/index.php?title=Friends\_General\_Conference\&oldid=62161112}}



 \item[Friends World Committee for Consultation (FWCC)] Eine weltweite
 Dachorganisation der Quäker. Im Gegensatz zu Dachorganisationen wie
 $\to$\textit{Friends General Conference}, hat das FWCC keine thologischa
 Austrichtung. Die Aufgabe von FWCC ist der rein infomelle Kontakt von Quakern.
 Gegründet wurde die Organisation 1937 in Swarthmore, Pennsylvania.
 Grundsätzlich sind nur Quaker-Organisationen Mitglied in der FWCC, es seiden
 es handelt sich um allein lebende Quaker, die kein Anschlus an einer
 Jahresversammlung haben. Die ca. 70 Jahresversammlungen, die dem FWCC
 angeschlossen sind, representieren etwar 265.000 der insgesammt 368.000
 Quäkern welt weit.\endnote{"`an introduction to quakerism"', Pnk Danelion,
 2007, ISBN 052160088-x}\endnote{Sie Artikel "`Friends World Committee for
 Consultation"', in der freie Enzyklopädie Wikipedia. Bearbeitungsstand:
 18. November 2009, 13:29 UTC. URL: \\ \texttt{http://de.wikipedia.org/w/index.php?title=Friends\_World\_Committee\_for\_Consultation\&oldid=66966169} }



 \item[Frühe Freund] (engl. \textit{early friends}) Damit ist die erste
 Quakergeneration gemein. Namentlich, zum Beispiel $\to$\textit{Geroge Fox}
 und $\to$\textit{Willam Penn}.

 \item[Fry, Elizabeth] $\ast$21.5.1780 in Cartham Hall bei Norwich,
 \dag12.10.1845 in Ramsgate. Quäkerin. War britische Reformerin des
 Gefängniswesens. Sie war die Tochter von $\to$\textit{John Gurney}

 \item[Fuchs, Emil] $\ast$13.5.1874 in Beerfelden/Odenwaldkreis; \dag13.2.1971
 in Ost-Berlin. Deutscher Theologe, Religiösen Sozialist,
 evangelisch-lutherischen Pfarrer und Quaker. 1933 verlor Fuchs seine
 Anstellung als Pfarrer und trat im selben Jahr der $\to$\textit{Deutschen
 Jahresversammlung} bei. In den 30ern war Fuchs in der Berliner Gruppe eine
 prägende Persönlichkeit gewesen, in seinen Predigten fanden viele Besucher
 geistigen Halt.


 \end{description}

\normalsize