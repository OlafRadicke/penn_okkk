\section*{F}

\articlesize

\begin{description}

\item[Facing Benches] In älteren Versammlungshäusern (englisch: meeting haus)
gibt es eine spezelle anordnung der Sitzbänge. Dabe gibt es eine einzelne Bank
die etwas erhoben ist und den anderen Bänken gegenüber steht. Dort sitzen die
so genannten \textit{"`gewichtien Freunde"'} ($\to$\textit{Weighty Friends}),
von denen zu erwarten ist, das sie in der Andacht das Wort ergreifen und zur
Versammlung sprechen werden. Das sind oft und traditonell die Ältestent
($\to$\textit{Ältester}) und die $\to$\textit{recorded ministers}. Heute ist
diese Form der Bestuhlung unüblich geworden. Der liberale Flügel
($\to$\textit{Liberal Friends}) verzichtet meist völlig auf
\textit{recorded ministers} und der \textit{evangelikale} beziehungsweise
\textit{pastoralen Fügel} ($\to$\textit{Evangelical Friends},
$\to$\textit{Pastoral Friends}) hat im Grunde jetzt seine Pastoren.
\endnote{Artikel "`Religious Society of Friends"'. Abgerufen 2010, January 12.
In Wikipedia, Die freie Enzyklopädie. Bearbeitungsstand: 16:41 Uhr 2010-01-13, \\ \texttt{http://en.wikipedia.org/w/index.php?title=\\
$\hookrightarrow$~Religious\_Society\_of\_Friends\&oldid=337403234}}


\item[Finsternis] $\to$\textit{dark countries}


 \item[Five Years Meeting] Vorläuferorganisation von $\to$\textit{Friends
 United Meeting}.


 \item[Free Quakers] Eine Abspaltung von Quäkern, die von ihren Versammlungen
 ausgeschlossen wurden, weil sie sich an den amerikanischen Unabhängigkeiskrig
 beteiligten. Als Gründer ist Samuel Wetherill bekannt. Das einzigste
 Versammlungshaus von ihnen steht noch heute in der 5th und Arch Street in
 Philadelphia. Es wurde mit prominenter Unterstützung wie Washington und
 Benjamin Franklin 1783 errichtet. Es war ein Prizip der \textit{Free Quakers}
 niemanden aus zu schiessen. weder aus theologischen noch aus moralischen
 Gründen. Nachdem einige Jahre niemand mehr die Andachten besucht hatte, wurde
 das Versammlungshaus 1834 geschlossen.
\endnote{\texttt{http://www.qhpress.org/quakerpages/qwhp/freequakertoc.htm}}
\endnote{Claus Bernet: "`Samuel Wetherill"' In: Biographisch-Bibliographisches
Kirchenlexikon (BBKL). Band 30, Nordhausen 2009, ISBN 978-3-88309-478-6,
Spalte~1555–1557.}

 \item[Friedenszeugnis] Das Friedenszeugnis teilt sich in ein so genanntes
 $\to$\textit{"`historisches Friedenszeugnis"'} und ein \textit{generelles}
 oder \textit{praktisches} Friedenszeugnis, wo z.B. die Verweigerung des
 Wehrdienstes dazu gehört. Für dieses Friedenszeugnis wird das Quakertum der
 so genannten \textit{Friedenskirche} zugerechnet.\endnote{Seite "`Friedenskirche
 (Konfession)"'. In: Wikipedia, Die freie Enzyklopädie. Bearbeitungsstand:
 12.~Februar 2010, 08:16 UTC. URL: \\
 \texttt{http://de.wikipedia.org/w/index.php?title=Friedenskirche\_(Konfession)\&oldid=70569238}
 (Abgerufen: 7. März 2010, 12:28 UTC) }

 \item[Freund(e)] Zum Teil bezeichnen sich Quaker untereinander als "`Freund"'.
 Streckenweise wurde der Begriff inflationär oder floskelhaft benutzt, was
 als unaufrichtig empfunden wurde, so das es einige ablehnen, grundsätzlich
 aller Quaker als "`Freunde"' zu bezeichnen. Abgeleitet wird das von der
 Bezeichnung "`Religiöse Gesellschaft der Freunde"'. Und so benutzen viele
 Quaker unreflektiert "`Freund"' als stehenden Begriff für "`Mitglied"'. Die
 ursprüngliche Bedeutung für die frühen Freunde ist aber der Bezug auf das
 Johannes-Evangelium 15,11-15. Wenn die (frühen) Quaker also jemanden als
 "`Freunde"' an sprachen, wollten sie damit zum Ausdruck bringen, das sie in dem
 Gegenüber jemand zu erkennen glaubten, der in der Nachfolge Jesu lebte. Ein
 "`Freunde"' musste also nicht unbedingt ein Mitglied der eigenen
 Glaubensgemeinschaft sein, sondern ein "`Freund Jesu"' sein und ein Leben führen,
 was tadellos ist (nicht trinken, morden, stehlen, etc....).
 
 \item[Freunde der Freunde] Sind Menschen die mit dem Quakertum symparisieren
 und engen kontakt zu Quakern pflegen, aber selber keine Quaker sind.

 \item[Friedensthal] Eine ehemalige deutsche Quäkerkolonie, die von 1792 bis
 1870 im heutigen Bad Pyrmonter Ortsteil Löwensen existierte. Begründer und
 wichtigste Figur war $\to$\textit{Ludwig Seebohm}.

 \item[Friends] $\to$\textit{Freund}

 \item[Friends House] $\to$\textit{Meeting House}

 \item[Friends General Conference (FGC)] Gehört neben der
 $\to$\textit{Friends United Meeting} und $\to$\textit{Evangelical Friends
 International} zu einer der drei größten Quäkerdachorganisation. Die FGC
 vereint unter sich Quäkergruppen mit unprogrammierter Andacht. Die Mitglieder
 der FGC vertreten mehr zu liberalen theologischen Ansichten
 ($\to$\\textit{Liberal Friends}). \endnote{Artikel "`Friends General
 Conference"'. In: Wikipedia, Die freie Enzyklopädie. Bearbeitungsstand:
 13.6.2009, 12:54 UTC. URL: \\
 \texttt{http://de.wikipedia.org/w/index.php?title=\\
 $\hookrightarrow$~Friends\_General\_Conference\&oldid=62161112}}



 \item[Friends World Committee for Consultation (FWCC)] Eine weltweite
 Dachorganisation der Quäker. Im Gegensatz zu Dachorganisationen wie
 $\to$\textit{Friends General Conference}, hat das FWCC keine thologischa
 Austrichtung. Die Aufgabe von FWCC ist der rein infomelle Kontakt von Quakern.
 Gegründet wurde die Organisation 1937 in Swarthmore, Pennsylvania.
 Grundsätzlich sind nur Quaker-Organisationen Mitglied in der FWCC, es seiden
 es handelt sich um allein lebende Quaker, die kein Anschlus an einer
 Jahresversammlung haben. Die ca. 70 Jahresversammlungen, die dem FWCC
 angeschlossen sind, representieren etwar 265.000 der insgesammt 368.000
 Quäkern welt weit.\endnote{"`an introduction to quakerism"', Pnk Danelion,
 2007, ISBN 052160088-x}\endnote{Sie Artikel "`Friends World Committee for
 Consultation"', in der freie Enzyklopädie Wikipedia. Bearbeitungsstand:
 18. November 2009, 13:29 UTC. URL: \\
 \texttt{http://de.wikipedia.org/w/index.php?title=\\
 $\hookrightarrow$~Friends\_World\_Committee\_for\_Consultation\&oldid=66966169} }



 \item[Frühe Freund] (engl. \textit{early friends}) Damit ist die erste
 Quakergeneration gemein. Namentlich, zum Beispiel $\to$\textit{Geroge Fox}
 und $\to$\textit{Willam Penn}.


 \item[Fry, Elizabeth] $\ast$21.5.1780 in Cartham Hall bei Norwich,
 \dag12.10.1845 in Ramsgate. Quäkerin. War britische Reformerin des
 Gefängniswesens. Sie war die Tochter von $\to$\textit{John Gurney}

 \end{description}

\normalsize