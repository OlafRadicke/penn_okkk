\section*{E}

\articlesize

\begin{description}

 \item[early friends] $\to$\textit{Frühe Freund}.

 \item[Eccles, Solomon] $\ast$1618 \dag1683, Quaker,  bekannt duch exzentrisches
 Verhalten. Etwar da durch, das er nackt auftrat, um damit zu zeigen, das er
 sich wieder im paradiesisch sündenfreien Urzustand befände (als Strafe dafür
 wurde er ausgepeitscht). Von Eccles ist bekannt das er Musiker, Musiklehrer und
 Musikinstomentebauer war, bevor er 1660 zum Quakertum konvertierte. Zuvor
 war er auch schon mitglied der episkopalischen Kirche, dann bei den
 Presbyterianer, Independenter und den Baptist. Nach seiner konversation zum
 Quakertum, verkaufte er zunächst seine Instumente und Noten, um sie dann wieder
 von seinen Käufern zurück zu kaufen und zu verbrennen. Da er sich um das
 Sehlenheil seiner Käufer sorgen machte. Um sich künftig ernähren zu können,
 erlehnte er den Beruf des Schuhhandwerks, was in den Augen der Quaker eine
 (moralisch) besserer Lebenserweb war. Eccles viel immer wieder durch
 provokatives Verhalten (in Kirchen) auf, was zu einer Reie von Bestrafungen und
 Gefängnisaufenthalten führte.\endnote{Biographisch-Bibliographischen
 Kirchenlexikon, Bautz, Band XX (2002)Spalten 424-427, Claus Bernet, \\
 \texttt{http://www.kirchenlexikon.de/e/eccles\_s.shtml}}

 \item[Ekklesiologie] In den Anfängen des Quäkertums wurde nicht unterschieden zwischen "`Kirche"' als Gebeute, "`Kirche"' als Institution oder "`Kirche"' in Theologischen sinne. Demnach war die Kirche die Gemeinschaft Derer, die nach Gottes Willen lebten und nicht in der Sünde verharrten. Perfektionismus - also Seligkeit - war das erstrebte Ziel und Zeichen der Mitgliedschaft der sonst nicht Äußerlich sichtbaren Kirche Gottes. Sakramente und Liturgie wurden als "`weltlich"' abgelehnt. Diese Auffassung wird aber Heute im evangelikalen Quäkertum und im liberalen Quäkertum weitestgehend nicht mehr vertreten. Im Konservativen Quakertum, sind aber noch Elemente davon zu finden.


 \item[Elisabeth von der Pfalz] $\ast$26.12.1618 in Heidelberg; \dag8.2.1680
 in Herford. Empfing unter anderem $\to$William Penn und korospondierte mit
 Quaker. Trat aber nie zum Quakertum über auch wenn sie mit William Pann eine
 Freundschaft bis zu ihrem Lebensende verband. Beim englischen König setzte sie
 sich für die Duldung der Quäker ein und in ihnem Haus wurden Andachten und
 Gottesdienste abgehalten.\endnote{ \\
 \texttt{http://www.kirchenlexikon.de/e/elisabeth\_v\_h.shtml }, \\
 Band I (1990)Spalten 1494-1495 Autor: Friedrich Wilhelm Bautz}

  \item[Elders] $\to$\textit{Ältester}

  \item[Esoterische Quäker] Esoterische Strömungen gab es schon seid den Anfängen
  des Quakertums. Durch Gerorge Fox wurden diese aber bekämpft und zurückgedrängt.
  Eine überaus interessante Person und Vertrete diese Strömung in den Anfängen
  des Quäkertums war $\to$\textit{Franciscus Mercurius van Helmont}. Aber auch
  heute noch ist diese Strömung gerade im $\to$\textit{Liberales Quakertum} und
  $\to$\textit{Universalismus} des Quakertums überaus pressend. Und auch in der
  Deutschen Jahresversammlung.
 \endnote{Seite "`Deutsche Jahresversammlung"'. In: Wikipedia, Die freie
 Enzyklopädie. Bearbeitungsstand: 24. Dezember 2009, 10:46 UTC. URL: \\
 \texttt{http://de.wikipedia.org/w/index.php?title=\\
 $\hookrightarrow$~Deutsche\_Jahresversammlung\&oldid=68383041} \\
 (Abgerufen: 5. Januar 2010, 22:26 UTC)  }

 \item[Evangelical Friends]
  Die \textit{Evangelical Friends} sind von den heute exsistirenden Flügeln -
  rein Äusserlich - am weitesten vo \textit{"`Urquakertum"'} entfehrnt. Sie haben
  viele Dinge wieder eingeführt, gegen die $\to$\textit{George Fox} und seine
  Freund gekämpft und wo für sie Gesundheit und Leben aufs Spiel gesetzt haben.
  Die \textit{Evangelical Friends} nennen ihrer Versammlungshäuser wieder
  "`Kirchen"', wo gegen auch William Penn in "`Ohne Kreuz keine Krohne"'
  ausfühlich stellung genommen hatt (Siehe hier zu Kapitel 5, 6.Absatz, Seite
  \pageref{kap5_ab6}). Bei den \textit{Evangelical Friends} gibt es auch wieder
  Pastoren, liturgische Gesänge und zum Teil sogar Taufen mit Wasser. Ein
  weiteres Merkmal ist die Betonung der Missionstätikeit und die Autorität der
  Bibel und der Heilsrelevanz von Jesus als der Christus.
\medskip 
  Die Geschichte der \textit{Evangelical Friends} ist etwas koplizierter. In
  den 1870er und 1880er Jahren spalteten sich eine Reihe von Jahresversammlungen
  (Western 1877, Kansas 1880, Canada 1881, Iowa 1883) über die Frage ab, ob
  Pastoren eingeführt werden sollten. Die meisten Versammlungen die sich zu dem
  Pastoralen Zeig zusammenschlossen, kam aus dem Gurneyite-Zweig der eine
  Abspaltung von den $\to$\textit{Orthodox Friends} waren. Sie schlossen sich
  zu dem $\to$\textit{Five Years Meeting} (dem spähteren $\to$\textit{Friends
  United Meeting}) zusammen. Das war ein Zusammenschluss von Yahresversammlungen
  die für sich die $\to$\textit{Richmond Declaration} von 1877 übernahmen.
\medskip 
  1947 kam es zu einer weiteren Spaltung und der Gründung der
  \textit{Association of Evangelical Friends}. Diese bestad bis 1970. 1965
  formte sich die \textit{Evangelical Friends Association} die dann 1989 zu
  \textit{Evangelical Friends Church International} wurde.

  Heute ist der Flügel der \textit{Evangelical Friends} deutlich der grösste und
  es sieht nichts danach aus, als wenn sich daran in zukunft etwas ändern wird.
  Der wachstum der \textit{Evangelical Friends} ist ungebrochen. Im gegensatz
  zum den $\to$\textit{Liberal Friends} die stagnieren und zum Teil auch
  schrumpfen.
\endnote{\texttt{http://www.evangelicalfriends.org/}}\endnote{Wikipedia
contributors, "`Evangelical Friends International,"' Wikipedia, The Free
Encyclopedia, \\
\texttt{http://en.wikipedia.org/w/index.php?title=\\
$\hookrightarrow$~Evangelical\_Friends\_International\&oldid=329570717}
(accessed January 9, 2010).}


 \item[Evangelical Friends International (EFI)] beziehungsweise
 \textit{Evangelical Friends Church International (EFCI)}
 $\to$\textit{Evangelical Friends}


 \end{description}

\normalsize
