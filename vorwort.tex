\part{Vorwort von Olaf Radicke}


Ich habe eine sehr emotionale Beziehung zu dem Text. Das Buch habe ich zufällig
entdeckt, als ich für einige Monate in Nord-Thailand war, um in
buddhistischen
Klöstern zu meditieren. Zu diesem Zeitpunkt hatte ich mich zehn Jahre intensiv
mit
Buddhismus beschäftigt und stellte fest, dass ich mit den buddhistischen Übungen
nicht weiter kam. Zu diesem Zeitpunkt liefen mir ein paar enthusiastische
Mitglieder der charismatischen Gemeinde \textit{"`Hope of Chiang Mai"'} über
den Weg. Sie bemühten sich redlich um meine Missionierung und luden mich zu
ihrem \textit{Bibelkongress} ein. Da ich die Leute ganz kauzig fand, ging ich
mit und fand dort auf dem (obligatorischen) Büchertisch eine Ausgabe
von \textit{"`No cross no crown"'}.
Ich konnte von dem Englisch kaum etwas verstehen, aber es sprach mich dermaßen
an, dass ich es kaufte. Die enttäuschte Reaktion meiner charismatischen Freunde,
als ich ihnen stolz meine Neuerwerbung präsentierte, bestärkte mich in meinem
Interesse an dem Buch\footnote{Ich vermute mal, das Buch gehörte nicht zum
Standard Repertoire und war er ein kleines Versehen. Aber die jungen Missionarin
aus USA, wusste offensichtlich -- der Reaktion nach zu urteilen -- das Werk
sofort ein zu ordnen.}. Der entscheidende Satz daraus, den ich gelesen und
verstanden hatte, war:

\begin{center}
\parbox{7,5cm}{
\textit{"`And as long as this disease continueth upon man he will make his God
his enemy, and himself incapable of the love and salvation that He hath
manifested, by his Son Jesus Christ, to the world."'}

\medskip

\textit{"`Als wenn er bloss um seiner selbst willen da sei, oder als ob er
sich selbst das Dasein gegeben habe, und daher keiner höheren Macht Rechenschaft
schuldig ist, und ihrem Urteilsspruch auch nicht unterworfen ist."'}
}
\end{center}

\medskip

Es dauerte noch eine Weile, bis ich dann auf Quaker stieß und die deutsche
Übersetzung von \textit{"`No cross no crown"'} fand. Aber im Grunde war es ein
Bekehrungserlebnis. Ziemlich unspektakulär und nicht wirklich ein Bruch zum
Buddhismus. Es gibt viele Überschneidungen zwischen Buddhismus und Quakertum.
Ich denke, dass sich das Quakertum, besser als Buddhismus, in die westliche Welt
(und das westliche
Denken) passt. Buddhismus vs. Quakertum ist aber noch mal
ein anderes Thema, was bestimmt auch sehr interessant ist, aber hier den Rahmen
sprengen würde.

\medskip

Claus Bernet hatte ich immer wieder von dem Buch erzählt. Irgendwann hat er mir
bei seinen Archiv-Recherchen das erste Kapitel der deutschen Übersetzung als
Fotokopie geschickt. Ich war aber sofort \textit{Feuer und
Flamme}. Ich ließ mir das gesamte Buch von ihm als Kopie schicken. Nach Lesen
von einem Drittel war
ich davon überzeugt, dass der Text unbedingt wieder allgemein zugänglich und
bekannt gemacht werden sollte.

\medskip

Dann begann ich in dem Internet-Projekt \textit{de.WikiSource.org} von meinen
Plänen
zu schreiben. Ich ließ für knapp hundert Euro vom Göttinger
Digitalisierungszentrum das Werk hochwertig einscannen. Das Ergebnis ist hier zu
finden:

\begin{center}
\texttt{http://gdz.sub.uni-goettingen.de/dms/load/img/?IDDOC=402779}
\end{center}

Danach wurde mir von Mitarbeitern des WikiSource-Projekts geholfen,
Texterkennungssoftware über die Scans laufen zu lassen. Da es sich um
Altdeutsche Schrift handelt, wo das "`f"' und das "`s"' fast gleich aussehen,
und das "`b"' fast wie das "`d"', war das Ergebnis sehr fehlerhaft. In
zweimonatiger Arbeit habe ich dann die Texte korrigiert und die Fußnoten
übertragen.

\medskip

Im nächsten Schritt habe ich dann den Text zu \LaTeX{} konvertiert und den Index
aufgebaut. In der Zwischenzeit konnte ich Claus Bernet von meinem Vorhaben
überzeugen, eine
Neuauflage herauszubringen. So stieg er dann mit ein und begann
den Text der heutigen Rechtschreibung anzupassen und einige der Texte
hinzuzufügen, die die 
Erschließung des Werkes von William Penn erleichtern sollen.
Das Ergebnis ist nun das, welches jetzt vorliegt. Ich habe mir bewusst die
Option für weiter Auflagen offen gehalten, indem ich der Auflage die
\textit{Versionsnummer 1.0} gab. Ich würde mich über Rückmeldungen des Lesers
sehr freuen. Am besten bin ich unter der folgenden E-Mail-Adresse zu erreichen:

\begin{center}
\texttt{briefkasten@olaf-radicke.de}
\end{center}

Vielleicht noch kurz zu meiner Person: Ich bin 1971 in West-Berlin
\textit{Insulaner} geboren und protestantisch erzogen worden. Zunächst lernte
ich den Beruf des \textit{Schauwerbegestalters}, indem ich aber nie gearbeitet
habe. Darauf folgten verschiedenste Jobs, immer wieder unterbrochen durch Zeiten
der Arbeitslosigkeit. In den letzten zehn Jahren habe ich zumeist in meinem
zweiten Beruf als Heilerziehungspfleger in der Behinderten-Assistenz gearbeitet.
Ich wohne in München und gehöre dort einer Gruppe Quaker an, die nicht zum
Verein \textit{Deutsche Jahresversammlung e.V}. gehört. Zu finden unter:

\begin{center}
\texttt{www.WasKannstDuSelbstSagen.de}
\end{center}

Und meinen privaten Weblog findet man unter:

\begin{center}
\texttt{www.Olaf-Radicke.de}
\end{center}

Zum Schluss bleibt mir nur zu wünschen, dass die hoffentlich zahlreichen Leser
den gleichen Gewinn aus dem Text ziehen, wie ich.



