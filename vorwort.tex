\part{Editorvorwort}

\chapter{Über die Herrausgeber}
Vita....

\chapter{Bemerkungen zur Edition}

Ziel dieser Ausgabe ist nicht eine genaue wiedergabe oder neuauflage eine
Historischen Dokuments, sondern eine überarbeitete Version, die möglichst
leicht zu lesen sein soll.

Scann, Texterkennung, 2 Manate Korekturlesen auf wikisource.org, konvertierung
zu LaTeX, Nachbearbeitung, Druck



Der Text basiert auf eine Übersetzung von 1825
\footnote{
Titel: "`Ohne Kreuz keine Krone"' \\
Entstehungsdatum: 1825, \\
Erscheinungsdatum: 1826, \\
Verlag: Georg Uslar, \\
Drucker: Heinrich Gelpke, \\
Erscheinungsort: Pyrmont (Bad-Pyrmont),\\
Übersetzer: ?, \\
Originaltitel: No Kross No Krown, \\
Quelle: \textbf{http://de.wikisource.org/wiki/Ohne\_Kreuz\_keine\_Krone}}. Der
Text wurde eingescannt, mit einer Texterkennungssoftware wieder zu einem
bearbeitbaren Text umgewandelt. Dieser Text wurde an in ein Projekt auf
wikisource.org übertragen und korektur gelesen. Das korekturlesen von mir
dauerte zwei Monate. Danach wurde der Text von mir in \LaTeX konvertiert.

Ich habe dann den Text weiter verarbeitet. Wer den historischen Urtext lesen
will (oder auch darauf angewiesen ist) dem steht nun die Version auf
wikisource.org zur verfühgung. Dort sind auch die Scanns des Orginal
Dokuments zu finden.

Diese Ausgabe soll den möglichst leichten Zugang zu dem Text ermöglichen. Da
zu habe ich einige Bearbeitungen bemacht. Es ist ein Index erstellt worden
um leichtet Themen, Personen, Orte oder Bibelstellen zu finden. Desweiteren
habe ich an einigen Stellen Wörter, die heute ungebräuchlich oder un- oder miss-
verständlich sind, ersetzt durch heute gebräuchliche. Das möglichst ohne den
Sinn zu verfälschen. Diese Stellen sind mit Fussnoten gekennzeichnet. Dadurch
klingt der Text an einigen Stellen stielistisch holprig. Die Verständlichkeit
war mir aber wichtiger. Wer die Ursprüngliche Stielistk beforzugt, der sei
wieder auf das Projelt bei wikisource.org verwiesen.

\medskip

Der Urtext neigt stark zu kettensätzen, die wahrscheinlich eine Qual jeden
ungeübten Lesers sind (den \textit{Geüben Lesern} vielleicht auch). Um
Das lesen dieser Satzmonster zu erleichten ohne ganze Absätze einfach neu
zu schreiben, wurden verschiedene Hilfen eingefügt:

\begin{itemize}
 \item Nebensätze mit nachrangiger Bedeutung in runden Klammern gesetzt.
 \item Kernaussagen Fett herrausgestellt.
 \item An einigen stellen wurde auch die Formulierung oder der Satzaufbau
 ein klein wenig geändert, wenn die Verständlichkeit dadurch verbessert wurde.
 -- Zu Not auch auf kosten der Stielistik.
 \item Kommentare in Fussnoten der Editoren werden mit
 \texttt{Mschienschrift kenntlich gemacht}
 \footnote{\texttt{Ein Beispiel für ein Editoren
 Kommentar}}
\end{itemize}

\begin{table} \centering
\caption{Beispiele für Ersetzungen im Text:}
\label{ref:tab_wortersetzungen}
\begin{tabular}{|l|l|} \hline
\textbf{Original} 	& \textbf{Ersetzung} 		\\ \hline \hline
allezeit 		& immer  			\\ \hline
arge 			& böse  			\\ \hline
einerlei 		& gleichen 	 		\\ \hline
figürlicher 		& sinnbildlicher  		\\ \hline
fleischlichen  		& materiellen / weltliche  	\\ \hline
Geize			& Gier\\ \hline
gehörig 		& richtig			\\ \hline
hofartig 		& hochmütig			\\ \hline
lautere 		& erliche   			\\ \hline
reiflich erwägest 	& genau prüfst 			\\ \hline
unbillig       		& in Irrtum  			\\ \hline
uneingedenk 		& ungeachtet			\\ \hline
Üppigkeit 		& Verschwendung / Maßlosigkeit		\\ \hline
Vergleichung 		& Konkurenz  			\\ \hline
vertilgen 		& vernichten			\\ \hline
Weiber genommen 	& geheiratet  			\\ \hline
 
\end{tabular} 

\end{table}

In Tabelle \ref{ref:tab_wortersetzungen} sind einige Beispiele für Wörtet die wir
ersetzt haben. Sohat 'Geize' in der Vergangenheit einen starken Bedeutungswandel erfahren. Früher war in erster liene das damit gemeint, was wir heute 'gier' nennen würden.
\\ \textbf{Sie zur Bedeutungswandes auch:}
\\ \textit{Seite 'Geiz'. In: Wikipedia, Die freie Enzyklopädie. Bearbeitungsstand: 4. August 2009, 10:48 UTC. URL:
\\ http://de.wikipedia.org/w/index.php?title=Geiz\&oldid=62956916
\\ (Abgerufen: 1. September 2009, 11:50 UTC) }
\\ \\ Desweiteren ersetze ich jetzt 'Geiz' durch 'gier' ohne explizit darauf hinzuweisen, da der Begriff zu oft verwendet wird!
	


\chapter{Danksagung}

\begin{itemize}
 \item WikiSource
 \item Heiko Lorenz
 \item Claus Bernet
 \item "`Hope of Chiang Main"'(Thailand)
\end{itemize}


\chapter{(Nutzungs-)Rechte}
http://de.wikisource.org/wiki/Index:Penn\_Ohne\_Kreuz\_keine\_Krone
