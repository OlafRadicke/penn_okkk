\part{Vorwort von Olaf Radicke}

\chapter{Vorwort zur Version 1.0}
Ich habe eine sehr emotionale Beziehung zu dem Text. Das Buch habe ich zufällig
entdeckt, als ich für einige Monate in Nord-Thailand war, um in
buddhistischen Klöstern zu meditieren. Zu diesem Zeitpunkt habe
ich mich zehn Jahre intensiv mit Buddhismus beschäftigt und stellte fest, dass ich mit den buddhistischen Übungen
nicht weiter kam. Zu diesem Zeitpunkt liefen mir ein paar enthusiastische
Mitglieder der charismatischen Gemeinde \textit{"`Hope of Chiang Mai"'} über
den Weg. Sie bemühten sich redlich um meine Missionierung und luden mich zu
ihrem \textit{Bibelkongress} ein. Da ich die Leute ganz kauzig fand, ging ich
mit und fand dort auf dem (obligatorischen) Büchertisch eine Ausgabe
von \textit{"`No cross no crown"'}.
Ich konnte von dem Englisch kaum etwas verstehen, aber es sprach mich dermaßen
an, dass ich es kaufte. Die enttäuschte Reaktion meiner charismatischen Freunde,
als ich ihnen stolz meine Neuerwerbung präsentierte, bestärkte mich in meinem
Interesse an dem Buch\footnote{Ich vermute mal, das Buch gehörte nicht zum
Standardrepertoire und es war ein kleines Versehen. Aber die junge Missionarin
aus USA wusste offensichtlich -- der Reaktion nach zu urteilen -- das Werk
sofort einzuordnen.}. Der entscheidende Satz daraus, den ich gelesen und
verstanden hatte, war (in diesem Buch auf
Seite \pageref{ref:vorw_sinndes_lebens} zu finden.):

\begin{center}
\parbox{7,5cm}{
\textit{"`[...] as if he had been born for himself, or rather given himself being,
and so not subject to the reckoning and judgment of a superior power."'}

\medskip

\textit{"`[...] Als wenn er bloß um seiner selbst willen da sei, oder als ob er
sich selbst das Dasein gegeben habe, und daher keiner höheren Macht Rechenschaft
schuldig ist, und ihrem Urteilsspruch auch nicht unterworfen ist."'}
}
\end{center}

\medskip

Es dauerte noch eine Weile, bis ich dann auf Quaker stieß und die deutsche
Übersetzung von \textit{"`No cross no crown"'} fand. Aber im Grunde war es ein
Bekehrungserlebnis. Ziemlich unspektakulär und nicht wirklich ein Bruch zum
Buddhismus. Es gibt viele Überschneidungen zwischen Buddhismus und Quakertum.
Ich denke, dass das Quakertum, besser als der Buddhismus, in die westliche Welt
(und das westliche
Denken) passt. Buddhismus vs. Quakertum ist aber noch mal
ein anderes Thema, was bestimmt auch sehr interessant ist, aber hier den Rahmen
sprengen würde.

\medskip

Claus Bernet hatte ich immer wieder von dem Buch erzählt. Irgendwann hat er mir
bei seinen Archiv-Recherchen das erste Kapitel der deutschen Übersetzung als
Fotokopie geschickt. Ich war sofort \textit{Feuer und
Flamme}. Ich ließ mir das gesamte Buch von ihm als Kopie schicken. Nach Lesen
von einem Drittel war
ich davon überzeugt, dass der Text unbedingt wieder allgemein zugänglich und
bekannt gemacht werden sollte.

\medskip

Dann begann ich in dem Internet-Projekt \textit{de.WikiSource.org} von meinen
Plänen
zu schreiben. Ich ließ für knapp hundert Euro vom Göttinger
Digitalisierungszentrum das Werk hochwertig einscannen. Das Ergebnis ist hier zu
finden:

\begin{center}
\texttt{http://gdz.sub.uni-goettingen.de/dms/load/img/?IDDOC=402779}
\end{center}

Danach wurde mir von Mitarbeitern des WikiSource-Projekts geholfen,
Texterkennungssoftware über die Scans laufen zu lassen. Da es sich um
Altdeutsche Schrift handelt, wo das "`f"' und das "`s"' fast gleich aussehen,
und das "`b"' fast wie das "`d"', war das Ergebnis sehr fehlerhaft. In
zweimonatiger Arbeit habe ich dann die Texte korrigiert und die Fußnoten
übertragen.

\medskip

Im nächsten Schritt habe ich dann den Text zu \LaTeX{} konvertiert und den Index
aufgebaut. In der Zwischenzeit konnte ich Claus Bernet von meinem Vorhaben
überzeugen, eine Neuauflage herauszubringen. So stieg er dann mit ein und begann
den Text der heutigen Rechtschreibung anzupassen und einige der Texte
hinzuzufügen, die die Erschließung des Werkes von William Penn erleichtern sollen.
Das Ergebnis ist nun das, welches jetzt vorliegt. Ich habe mir bewusst die
Option für weiter Auflagen offen gehalten, indem ich der Auflage die
\textit{Versionsnummer 1.0} gab. Ich würde mich über Rückmeldungen des Lesers
sehr freuen. Am besten bin ich unter der folgenden E-Mail-Adresse zu erreichen:

\begin{center}
\texttt{briefkasten@olaf-radicke.de}
\end{center}

Vielleicht noch kurz zu meiner Person: Ich bin 1971 in West-Berlin
als \textit{Insulaner} geboren und protestantisch aufgewachsen. Zunächst lernte
ich den Beruf des \textit{Schauwerbegestalters}, in dem ich jedoch nicht gearbeitet
habe. Darauf folgten verschiedenste Jobs, immer wieder unterbrochen durch Zeiten
der Arbeitslosigkeit. Es folgten zehn Jahre in meinem zweiten Beruf
als Heilerziehungspfleger in der Assistenz von Menschen mit Behinderungen.
Heute wohne ich in meiner Wahlheimat München und arbeite im IT-Umfeld mit dem
Open-Source-Betriebsystem Linux, einer langjährigen Leidenschaft, die ich
glücklicherweise zu meinem Beruf machen konnte.

Meinen privaten Weblog findet man unter:

\begin{center}
\texttt{www.Olaf-Radicke.de}
\end{center}

Zum Schluss bleibt mir nur zu wünschen, dass die hoffentlich zahlreichen Leser
den gleichen Gewinn aus dem Text ziehen werden, wie ich.

Olaf Radicke, 2009


\chapter{Vorwort zur Version 2.0}

Zirka ein Jahr nach der Veröffentlichung der Version 1.0 entschließe ich mich
nun eine überarbeitete Version herauszugeben. Die Änderungen betreffen einige
Korrekturen in den Fußnoten und am Satz. Die wichtigste Korrektur betrifft ein
falsches Zitat in meinem Vorwort. Hier hat sich beim Kopieren ein falscher Satz
eingeschmuggelt. Die nächste größere Veränderung betrifft den Umgang mit
Bibelzitaten. Hierfür wurden bisher Fußnoten verwendet. Nun werden sie als
Randnotizen dargestellt. Dies sorgt für mehr Übersichtlichkeit
und dient der leichteren Auffindbarkeit. Neu ist auch die Aufteilung des Index in
drei separate Indexe, jeweils für Personen, für Bibelstellen und für Schlagworte.

Olaf Radicke, 2011


\chapter{Vorwort zur Version 3.0}

Die dritte Auflage (Version 3.0) beinhaltet zahlreiche Fehlerkorrekturen.
Form und Inhalt sind weitestgehend unverändert geblieben. Mein besonderer Dank
richtet sich an Dorothea aus Eisfeld, die mir den maßgeblichen Anstoß zu dieser
Auflage gab, indem sie mir über 400(!) Fehlerkorrekturen und Verbesserungsvorschläge
zugeschickt hat. Hätte es sich nur um ein paar Dutzend Korrekturen gehandelt,
hätte ich die Versionsnummer 2.1 gewählt. Bei der Menge an Änderungen halte
ich die Versionsnummer 3.0 für gerechtfertigt.

Olaf Radicke, 2014


\chapter{Vorwort zur Version 4.0}

Die vierte Auflage (Version 4.0) beinhaltet ca. weitere 300 Fehlerkorrekturen,
kleinere Glättungen und erklärenden Fußnoten, sowie Ergänzungen bei den Quellenangaben
der Bibelzitate. Form und Inhalt sind unverändert geblieben. durch die kontinuierlichen
Verbesserungen seid der zweiten Auflage wurden insgesamt über 5600 Stellen überarbeitet.
Ein erheblichen Anteil an der Arbeit der letzten zwei Ausgaben hatte, Dorothea Meißner.
Deshalb wird Sie nun als dritte Herausgeberin genannt.


Diese Version
erscheint zum ersten mal auch als eBook.

Olaf Radicke, 2015