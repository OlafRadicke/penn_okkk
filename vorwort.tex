\part{Vorwort der Herausgeber}

\chapter{Für wen ist dieses Buch}

Diese Ausgabe hat das Ziel, es dem Leser so leicht wie möglich zum machen, sich
dem über 300 Jahre alten Text zugänglich und verständlich zu machen. Grundlage
des Textes ist eine Übersetzung von 1825 \footnote{
Titel: "`Ohne Kreuz keine Krone"' \\
Entstehungsdatum: 1825, \\
Erscheinungsdatum: 1826, \\
Verlag: Georg Uslar, \\
Drucker: Heinrich Gelpke, \\
Erscheinungsort: Pyrmont (Bad-Pyrmont),\\
Übersetzer: ?, \\
Originaltitel: No Kross No Krown, \\
Quelle: \textbf{http://de.wikisource.org/wiki/Ohne\_Kreuz\_keine\_Krone}}. Es
ist nicht Ziel dieser Ausgabe, diesen Text in seiner historischen original
Zustand wiederzugeben, um z.B. daraus als Quelle zu Zitieren. Fehler die im
Urtext enthalten sind -- z.B. fehlerhafte Nummerierungen -- werden hier nicht
übertragen. Wer mit dem Quelltext zu wissenschaftlichen Zwecken arbeiten will,
sei hier auf \texttt{http://de.WikiSource.org} verwiesen, wo er den
Originalitext als Scann findet.

\medskip

Um der Zielstellung (den Text möglichst vielen Lesern zugänglich zumachen)
gerecht zu werden, haben wir eine Reihe von Änderungen und Ergänzungen gemacht.
Zum einen haben wir den Text der modernen Rechtschreibung angepasst. Die
Interpunktion wurde der heutig Verwendung angepasst. An einigen wenigen Stellen,
wurden die Setze, zur besseren Verständlichkeit, ein wenig umgestellt. An vielen
Stellen wurden Wörter ersetzt, wenn diese heute ungebräuchlich und deshalb
unverständlich sind. Oder wenn sich die Bedeutung von Wörtern geändert hat. So
hat das Wort \textit{Geize} in der Vergangenheit einen starken Bedeutungswandel
erfahren. Früher war in erster Linie das damit gemeint, was wir heute als
\textit{Gier} bezeichnen würden. So haben wir also im kompletten Text
\textit{Geize} durch \textit{Gier} ersetzt \footnote{Wer zu diesem konkreten
Beispiel mehr wissen will, der sei auf den Artikel "`Geiz"' in de.Wikipedia.org
verwiesen}. In folgenden Tabelle \ref{ref:tab_wortersetzungen} sind einige
weitere Beispiele aufgeführt, für Wörter die wir ersetzt haben. An allen
stellen, wo wir Wörter ersetzt haben, haben wir das mit einer Fussnote kenntlich
gemacht.

\medskip

\begin{center}
\label{ref:tab_wortersetzungen}
\begin{tabular}{|l|l|} \hline
\textbf{Original}       & \textbf{Ersetzung}            \\ \hline \hline
allezeit                & immer                         \\ \hline
arge                    & böse                          \\ \hline
einerlei                & gleichen                      \\ \hline
figürlicher             & sinnbildlicher                \\ \hline
fleischlichen           & materiellen / weltliche       \\ \hline
Geize                   & Gier\\ \hline
gehörig                 & richtig                       \\ \hline
hofartig                & hochmütig                     \\ \hline
lautere                 & erliche                       \\ \hline
reiflich erwägest       & genau prüfst                  \\ \hline
unbillig                & in Irrtum                     \\ \hline
uneingedenk             & ungeachtet                    \\ \hline
Üppigkeit               & Verschwendung / Masslosigkeit          \\ \hline
Vergleichung            & Konkurrenz                     \\ \hline
vertilgen               & vernichten                    \\ \hline
Weiber genommen         & geheiratet                    \\ \hline
\end{tabular}
\end{center}
\medskip

\medskip

Als weitere Hilfestellung für den Leser, haben wir noch einige erklärende
Begleittexte verfasst. Im hinteren Teil, gehen wir noch mal in separaten
Abschnitten darauf ein, was es interessantes zur entstehungsgeschichte des
Textes zu sagen gibt; Wer der Autor war und was für ein Leben er führte; Und die
sich das Quakertum seid der Entstehung des Textes weiter entwickelt hat.

\chapter{Typographische Konventionen}

In dem Buch wird der Begriff "`Quaker"' \index{Quaker!Schreibweise} in seiner
englischen original Schreibweise verwendet. Mir scheint es nicht sinnvoll
"`Quäker"' zu schreiben. Englische Begriffe wie "`Safe"' werden im Deutschen
auch nicht "`Säfe"' geschrieben. Ich vermag also kein Sinn darin zu sehen
"`Quäker"' zu schreiben.

\medskip

Es gibt zwei verschiedene Fussnoten in dem Abschnitt des Textes von "`Ohne Kreuz
keine Krone"' Die Fussnoten in normaler Schrift, sind die des Autors, des
original Textes. Hingegen werden Fussnoten der Herausgeber
\texttt{Maschinenschrift kenntlich gemacht}. Das sind meist Anmerkungen zu
Veränderungen, oder Erklärungen zum Verständnis.

\medskip

Einige Stellen sind mit \textbf{Fettschrift herausgestellt}. Das soll dem Leser
Helfen auf wichtige Kernaussagen aufmerksam zu werden. Es gibt einige langatmige
Passagen, in denen sich W. Penn in Widerlegungen seiner Gegner aus lässt, die
heute von ihrer Bedeutung kaum noch von Belangen sind. So z.B. der Abschnitt
"`10. Kapitel"' (\textit{Seite \pageref{kap10} - \pageref{kap10_ende}}) über
Ehrentitel, die Heute ohne hin nicht mehr gebräuchlich sind. Trotzdem gibt es
einige prägnanten Sätze, die Dinge sehr prägnant darstellen. Diese haben wir mit
Fettschrift herausgestellt.

\chapter{(Nutzungs-)Rechte}
Der Urtext im Original von W. Penn ist \textit{Gemeinfrei}. Er ist im Internet
unter dieser Adresse zu finden:

\begin{center}
\texttt{http://de.wikisource.org/wiki/Index:Penn\_Ohne\_Kreuz\_keine\_Krone}
\end{center}

Die Bearbeitungen, Fussnoten, Einleitungen, Anmerkungen, Erläuterungen und
Extra-Kapitel unterstehen dem Urheberrecht von Claus Bernet und Olaf Radicke.


\chapter{Vorwort von Herrausgeber}

\begin{flushright}
\begin{footnotesize}
Olaf Radicke
\end{footnotesize}
\end{flushright}
\smallskip

Ich habe ein sehr emotionale Beziehung zu dem Text. Ich hatte das Bucht zufällig
entdeckt. Ich war für einige Monate in Nord-Thailand um in Buddhistischen
Klöstern zu meditieren. Zu diesen Zeitpunkt hatte ich mich 10 Jahre intensiv mit
Buddhismus beschäftigt, und stellte fest das ich mit den buddhistischen Übungen
nicht weiter kam. Zu diesem Zeitpunkt liefen mir ein paar enthusiastische
Mitglieder der Charismatischen Gemeinde \textit{"`Hope of Chiang Main"'} über
den Weg. Sie bemühten sich redlich um meine Missionierung, und luden mich auf
ihren \textit{Bibelkonkres} ein. Da ich Leute ganz kauzig fand, ging ich mit und
dort fand ich auf dem (obligatorischen) Büchertisch \textit{"`no cross no crown"'}.
Ich konnte von dem Englisch kaum was verstehen, aber es sprach mich dermassen
an, das ich es kaufte. Die enttäuschte Reaktion meiner Charismatischen Freunde,
als ich ihnen stolz meine Neuerwerbung presentierte, bestärkte mich in meinen Interesse an dem Buch. Den entscheidenenen Satz daraus den ich gelesen und verstanden hatte war:

\begin{center}
\parbox{7,5cm}{
\textit{"`And as long as this disease continueth upon man he will make his God his enemy, and himself incapable of the love and salvation that He hath manifested, by his Son Jesus Christ, to the world."'}

\medskip

\textit{"`Als wenn er bloß um seiner selbst willen da sei, oder als ob er
sich selbst das Dasein gegeben habe, und daher keiner höheren Macht Rechenschaft
schuldig ist, und ihrem Urteilsspruch auch nicht unterworfen ist."'}
}
\end{center}

\medskip

Es dauerte dann noch eine weile bis ich dann auf Quäker stiess und die Deutisch
Übersetzung von \textit{"`no cross no crown"'}. Aber im Grunde wahr es ein
Bekehrungserlebnis. Ziemlich unspektakulär und nicht wirklich ein Bruch zum
Buddhismus. Es gibt viele Überschneidungen zwischen Buddhismus und Quakertum.
Ich denke das Quakertum sich besser in die westliche Welt (und das westliche
Denken) ein passt, als Buddhismus. Buddhismus Vs. Quakertum ist aber noch mal
ein anderes Thema, was bestimmt auch sehr interessant ist, aber hier den Rahmen
sprengt.

\medskip

Claus Bernet hatte ich immer wieder von dem Buch erzählt. Irgend wann hat er mir
bei seinen Archiv-Recherchen das erste Kapitel der deutschen Übersetzung als
Fotokopie geschickt. Ich war aber sofort \textit{Feuer und
Flamme}. Ich lies mir das gesamte Buch von im als Kopie schicken. Nach 1/3 war
ich davon überzeugt, das der Text unbedingt wieder allgemein zugänglich und
bekannt gemacht werden sollte.

\medskip

Ich begann in dem Internet-Projekt \textit{de.WikiSource.org} von meinen Plänen
zu schreiben. Ich lies dann für knapp 100 Euro von Göttinger
Digtalisierungszentrum das Werk hochwertig einscannen. Das Ergebnis ist hier zu
finden:

\begin{center}
\texttt{http://gdz.sub.uni-goettingen.de/dms/load/img/?IDDOC=402779}
\end{center}

Danach wurde mir von Mitarbeitern des WikiSource-Projekts geholfen,
Texterkennungssoftware über die Scanns laufen zu lassen. Das es sich um
Altdeutsche Schrift handelt, wo das "`f"' und das "`s"' fast gleich aussieht,
und das "`b"' fast wie das "`d"', war das Ergebnis sehr fehlerhaft. In
zweimonatiger Arbeit, habe ich dann die Texte korrigiert und die Fussnoten
übertragen.

\medskip

Im nächsten Schritt habe ich dann den Text zu \LaTeX{} konvertiert und das Index
aufgebaut. In der Zwischenzeit, konnte ich Claus Bernet von meinen Vorhaben eine
Neuauflage erraus zu bringen, überzeugen. So stieg er dann mit ein, und begann
den Text auf heutige Rechtschreibung an zu passen und einige der Texte zu zu
steuern, die die Erschlissung des Werkes von W. Penn erleichten sollen.
Das Ergebnis ist nun das was jetzt vorliegt. Ich habe mir bewusst die
Option für weiter Auflagen offen gehalten, in dem ich der Auflage die
\textit{Versionsnummer 1.0} gab. Ich würde mich über Rückmeldungen des Lesers
sehr Freuen am besten bin ich unter der folgende E-Mail-Adresse zu erreichen:

\begin{center}
\texttt{briefkasten@olaf-radicke.de}
\end{center}

Vielleicht noch kurz zu meiner Person: Ich bin 1971 in West-Berlin
\textit{Insulaner} geboren und Protestantisch erzogen worden. Zunächst lernte
ich den Beruf des \textit{Schauwerbegestalters} in dem ich aber nie gearbeitet
habe. Darauf folgten verschiedenste Jobs, immer wieder unterbrochen durch Zeiten
der Arbeitslosigkeit. In den letzten zehn Jahren habe ich zumeist in meinem
zweiten Beruf als Heilerziehungspfleger in der Behinderten-Assistenz gearbeitet.
Ich Wohne in München und gehöre dort einer Gruppe Quaker an, die nicht zum
Verein \textit{Deutsche Jahresversammlung e.V}. gehört. Zu finden unter:

\begin{center}
\texttt{www.WasKannstDuSelbstSagen.de}
\end{center}

Und meinen privaten Weblog findet man unter:

\begin{center}
\texttt{www.Olaf-Radicke.de}
\end{center}

Zum Schluss bleibt mir nur zu wünschen, das die hoffentlich zahlreichen Leser
den gleichen Gewinn aus dem Text ziehen, wie ich.



