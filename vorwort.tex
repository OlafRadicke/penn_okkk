\part{Vorwort der Herrausgeber}

\chapter{Für wehn ist dieses Buch}

Diese Ausgabe hat das Ziel, es dem Leser so leicht wie möglich zum machen, sich dem über 300 Jahre alten Text zugänglich und verständlich zu machen. Grundlage des Textes ist eine Übersetzung von 1825 \footnote{
Titel: "`Ohne Kreuz keine Krone"' \\
Entstehungsdatum: 1825, \\
Erscheinungsdatum: 1826, \\
Verlag: Georg Uslar, \\
Drucker: Heinrich Gelpke, \\
Erscheinungsort: Pyrmont (Bad-Pyrmont),\\
Übersetzer: ?, \\
Originaltitel: No Kross No Krown, \\
Quelle: \textbf{http://de.wikisource.org/wiki/Ohne\_Kreuz\_keine\_Krone}}. Es ist nicht Ziehl dieser Ausgabe, diesen Text in seiner historischen original Zustand wiederzugeben, um z.B. daraus als Quelle zu Zitiren. Fehler die im Urtext enthalten sind -- z.B. fehlerhafte Nummerierungen -- werden hier nicht übertragen. Wer mit dem Quelltext zu wissenschaftlichen Zwecken arbeinen will, sei hier auf \texttt{http://de.WikiSource.org} verwiesen, wo er den Originaltext als Scann findet.

\medskip

Um der Zielstellung (den Text möglichst vielen Lesern zugänglich zumachen) gerecht zu werden, haben wir eine Reie von Änderungen und Ergänzungen gemacht. Zum einen haben wir den Text der modernen Rechtschreibung angepasst. Die Interpunktion wurde der heutig verwendung angepasst. An einigen wenigen Stellen, wurden die Setze, zur besseren verständlichkeit, ein wenig umgestellt. An vielen Stellen wurden Wörtet ersetzt, wenn diese heute ungebräuchlich und deshalbt unverständlich sind. Oder wenn sich die Bedeuting von Wörtern geändert hat. So hat das Wort \textit{Geize} in der Vergangenheit einen starken Bedeutungswandel erfahren. Früher war in erster liene das damit gemeint, was wir heute als \textit{Gier} bezeichnen würden. So haben wir also im kompletten Text \textit{Geize} durch \textit{Gier} ersetzt \footnote{Wer zu diesem kontreten Beispiel mehr wissen will, der sei auf den Artikel "`Geiz"' in de.Wikipedia.org verwiesen}. In folgenden Tabelle \ref{ref:tab_wortersetzungen} sind einige weitere Beispiele aufgeführt, für Wörtet die wir ersetzt haben. An allen stellen, wo wir Wörter ersetzt haben, haben wir das mit einer Fussnote kenntlich gemacht.

\medskip

\begin{center}
\label{ref:tab_wortersetzungen}
\begin{tabular}{|l|l|} \hline
\textbf{Original}       & \textbf{Ersetzung}            \\ \hline \hline
allezeit                & immer                         \\ \hline
arge                    & böse                          \\ \hline
einerlei                & gleichen                      \\ \hline
figürlicher             & sinnbildlicher                \\ \hline
fleischlichen           & materiellen / weltliche       \\ \hline
Geize                   & Gier\\ \hline
gehörig                 & richtig                       \\ \hline
hofartig                & hochmütig                     \\ \hline
lautere                 & erliche                       \\ \hline
reiflich erwägest       & genau prüfst                  \\ \hline
unbillig                & in Irrtum                     \\ \hline
uneingedenk             & ungeachtet                    \\ \hline
Üppigkeit               & Verschwendung / Maßlosigkeit          \\ \hline
Vergleichung            & Konkurenz                     \\ \hline
vertilgen               & vernichten                    \\ \hline
Weiber genommen         & geheiratet                    \\ \hline
\end{tabular}
\end{center}
\medskip

\medskip

Als weitere Hilfestellung für den Leser, haben wir noch einige erklährende Begleittexte verfasst. Im hinteren Teil, gehen wir noch mal in seperaten Abschitten darauf ein, was es interessantes zur entstehungsgeschichte des Textes zu sagen gibt; Wer der Autor war und was für ein Leben er führte; Und die sich das Quakertum seid der entstehung des Textetes weiter entwikelt hat.

\chapter{Typographische Konventionen}

In dem Buch wird der Begriff "`Quaker"' \index{Quaker!Schreibweise} in seiner englischen original Schreibweise verwendet. Mir scheint es nicht sinnvoll "`Quäker"' zu schreiben. Englische Begriffe wie "`Safe"' werden im Deutschen auch nicht "`Säfe"' geschrieben. Ich vermag also kein Sinn darin zu sehen "`Quäker"' zu schreiben.

\medskip

Es gibt zwei verschiedene Fussnoten in dem Abschnitt des Textes von "`Ohne Kreuz keine Krone"' Die Fussnoten in normaler Schrift, sind die des Autors, des original Textes. Hingegen werden Fussnoten der Herrausgeber \texttt{Maschienschrift kenntlich gemacht}. Das sind meist Anmerkungen zu Veränderungen, oder erklährungen zum Verständniss.

\medskip

Einige Stellen sind mit \textbf{Fettschrift herrausgestellt}. Das soll dem Leser Helfen auf wichtige Kernaussagen aufmerksam zu werden. Es gibt einige langatmige Pasagen, in denen sich W. Penn in wiederlegungen seiner Gegner ausläst, die heute von ihrer Bedeutung kaum noch von Belangen sind. So z.B. der Abschnitt über Ehrentitel, die Heute ohne hin nicht mehr gebrächlich sind. Trotzdem gibt es einige pregnaten Sätze, die Dinge serhr pregnant darstellen. Diese haben wir mit Fettschrift herrausgestellt.

\chapter{Vorwort von Claus Bernet}


\chapter{Vorwort von Olaf Radicke}


Scann, Texterkennung, 2 Manate Korekturlesen auf wikisource.org, konvertierung
zu LaTeX, Nachbearbeitung, Druck



Der Text basiert auf eine Übersetzung von 1825
Der
Text wurde eingescannt, mit einer Texterkennungssoftware wieder zu einem
bearbeitbaren Text umgewandelt. Dieser Text wurde an in ein Projekt auf
wikisource.org übertragen und korektur gelesen. Das korekturlesen von mir
dauerte zwei Monate. Danach wurde der Text von mir in \LaTeX konvertiert.


\medskip


\begin{itemize}
 \item WikiSource
 \item Heiko Lorenz
 \item Claus Bernet
 \item "`Hope of Chiang Main"'(Thailand)
\end{itemize}


\chapter{(Nutzungs-)Rechte}
http://de.wikisource.org/wiki/Index:Penn\_Ohne\_Kreuz\_keine\_Krone
