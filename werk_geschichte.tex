
\begin{flushright}
\begin{footnotesize}
\textit{Von Claus Bernet}
\end{footnotesize}
\end{flushright}
\smallskip

Das \textit{"`Journal"'}\index{Journal} (spirituelle Tagebuch) von George Fox\index{Personen:!Fox,
George}, das Buch \textit{"`Früchte der Einsamkeit"'}\index{Früchte der Einsamkeit} von William
Penn und das Tagebuch von John
Woolman\index{Personen:!Woolman, John} waren (und sind) nicht nur im Quäkertum
äußerst beliebt, sondern zählen sogar zur Weltliteratur\index{Weltliteratur}.
Sie wurden vielfach aufgelegt und in viele Sprachen übersetzt. Unmittelbar
danach, gewissermaßen auf dem undankbaren vierten Platz, folgt schon\textit{"`Ohne
Kreuz keine Krone"'}. Ich will im Folgenden der Frage nachgehen, wer diese
Schrift
aus welchen Gründen übersetzt hat, und anschließend (ab
Seite~\pageref{ref:rezeptionsgeschichte}) davon berichten, wie \textit{"`Ohne Kreuz
keine Krone"'}aufgenommen wurde.

\section{Zum Übersetzer Ludwig Seebohm} \label{ref:l_seebohm}


Mit großer Sicherheit ist Ludwig Seebohm
Übersetzer der Fassung von 1825.\footnote{Als Übersetzer wird im an und für sich
zuverlässigen \textit{"`Nineteenth century
short title catalogue"'}(Series I~\&~II London 1801-1870, S. 127),\textit{"`Lewis
Seebohm"'} also Ludwig Seebohm, als Übersetzer angegeben. Auch das Exemplar der
Oxforder Universitätsbibliothek (Sig. Bodleian Library 1419 e.5328,
\texttt{http://www.ouls.ox.ac.uk}) nennt Seebohm als Übersetzer.}
Die deutschen Quäker haben diesem Ludwig Seebohm viel zu
verdanken. Seebohm ist gewissermaßen eine kleine Ausgabe des großen Penn. Es
gibt unzählige Gemeinsamkeiten: Beide waren Kolonisten (Penn gründete einen
Staat, Seebohm eine Kolonie), beide waren überzeugte Quäker, beide hielten
Mission\index{Mission} als Grundvoraussetzung für das Überleben, beide haben
vielerlei Schriften und Traktate für das Quäkertum verfasst, und beide waren
auch leidenschaftliche Prediger\index{Predigen}\index{Prediger}.
Bei den deutschen Quäkern war Seebohm nicht beliebt -- für Beliebtheiten war
Seebohm zu direkt, zu geradlinig, zu unbeugsam. Unter den vielen Illiteraten,
selbstgenügsamen Bauern\index{Bauer} und
Handwerkern\index{Handwerker} unter den deutschen Quäkern war Seebohm
eine einsame Ausnahmegestalt. Unter den Lutheranern\index{Personen:!Lutheraner}
oder Katholiken\index{Katholik} hätte er es als Stubengelehrter weit
gebracht -- aber bei den Quäkern war nichts dergleichen zu erwarten -- das war
sein Kreuz.

\medskip

Seebohm stammt aus Pyrmont\index{Orte:!Pyrmont}, der Stätte seines lebenslangen
Wirkens, wo er 1757 geboren wurde.\footnote{Alle Angaben, soweit nicht anders
angegeben, sind folgender (wesentlich
ausführlicheren) Biographie Seebohms entnommen: Claus Bernet: Ludwig Seebohm
(1757-1835). Founder of Friedensthal, in: The Friends Quarterly, 34, 1, 2004,
S.~20-30.} Er wuchs ihn behüteten Verhältnissen auf, lernte Französisch,
Englisch und sogar Latein. Durch seine Heirat mit Juliane von
Borries\index{Personen:!Borries, Juliane von} hatte er sogar Zugang zum niederen
Adel\index{Adel!niederer} -- das hatte den Effekt, dass später
hochrangige Besucher wie Katharina Pawlowna\index{Personen:!Pawlowna, Katharina}
(1788-1819, die Schwester des russischen Zaren) oder Königin
Luise\index{Personen:!Luise, Königin} (1776-1810) die Familie Seebohm besuchten,
die damit Kontakte bis an den preußischen Königshof\index{Preußischer
Königshof} in Berlin\index{Orte:!Berlin} hatte. Immer wieder korrespondierte
Seebohm mit der Regierung in Berlin und verhandelte über das Recht auf freie
Religionsausübung der Quäker im preußischen Minden\index{Orte:!Minden}, über das
Recht der Kriegsdienstverweigerung\index{Krieg!Kriegsdienstverweigerung} und
über das Problem der Anerkennung von Quäkerehen.

\medskip

Seebohm führte ein relativ sorgloses Leben, bis 1790 sein Geschäft für Mode- und
Luxuswaren\index{Luxusware} pleite ging -- sein Grossist aus
Frankreich\index{Orte:!Frankreich} konnte wegen der französischen
Revolution\index{Revolution!Französische} nicht mehr liefern. Ähnlich wie Penn
durchlitt auch Seebohm eine religiöse Sinnkrise\index{Sinnkrise} und erlebte
anschließend eine geistige Wiedergeburt\index{Wiedergeburt!geistige}. Er
versuchte nun, in England\index{Orte:!England} neue
Handelskontakte\index{Handelskontakt} zu knüpfen, lernte dabei die Quäker,
darunter den Philanthropen George Dillwyn\index{Personen:!Dillwyn, George}
(1738-1820), kennen.\footnote{Library of the Society of Friends (London)
(zukünftig LSF), Meeting for
Sufferings, XXXVIII, 1788-1791, S.~94-95.} Seebohm erkannte sofort die
Notwendigkeit der Werbung\index{Werbung} und organisierte sogleich Buchsendungen
nach Deutschland, darunter Barclays\index{Personen:!Barclay, Robert}
\textit{"`Apologie"'}\index{Apologie}, Sewels "`Geschichte der Quäker"' und eben auch Penns\textit{"`Call to
Christendom"'}, das am besten dasjenige ausdrückte, was Seebohm sich erhoffte:
eine Metanoia hin zum reinen
Frühchristentum\index{Frühchristentum}\index{Urchristen} und eine echte
Herzensbekehrung aller
\textit{Taufscheinchristen}\index{Taufe!Taufscheinchristen}.

\medskip

Seebohm verweigerte nun gemeinsam mit seiner Frau gegen Jahresende 1790 die
Zahlung der kirchlichen Taufgebühren\index{Taufe!Taufgebühr}: Das war der erste
zivile Ungehorsam\index{Ungehorsam!ziviler} eines Quäkers auf deutschen Boden
seit langer Zeit. Ein Versuch des Pyrmonter Konsistorialrats Johann
Steinmetz\index{Personen:!Steinmetz, Johann} (1739-1814), der das Kind in einer
Nacht-und-Nebel-Aktion zwangstaufen wollte, schlug fehl, die kleine
Johanne\index{Personen:!Seebohm, Johanne} blieb ungetauft. Wegen weiterer
sozialer Auffälligkeiten und Protestaktionen\index{Protestaktion} wurde
Seebohm, wie einst Penn, schließlich
inhaftiert\index{Inhaftierung}\index{Gefängnis}, wenn auch nur für kurze
Zeit.\footnote{Zudem ist eine weitere Inhaftierung für 1798 bezeugt, als Seebohm
sich vor
dem Magistrat zu Minden weigerte, seinen Hut abzulegen. Ein drittes Mal wurde er
wenige Monate darauf im Oktober 1798 in Friedberg von französischen Soldaten
inhaftiert, die ihn irrtümlich für einen Spion hielten.} Ende Januar 1791 kam es
in Pyrmont zu einer kleinen Sensation: Friedrich von
Waldeck\index{Personen:!Waldeck, Friedrich von} (1743-1812), ein
leidenschaftlicher Freimaurer\index{Freimaurer}, Freigeist und vor
allem weitsichtiger und toleranter Landesherr, stellte den Quäkern eine
Duldungsakte\index{Duldungsakte} aus -- hier konnte sich Seebohm wie sein
Vorbild Penn vor König Charles II.\index{Personen:!Charles II., König} fühlen. 
Zwar bekam er keine Kolonie vermacht, wie Penns Pennsylvania, aber doch
immerhin einen beträchtlichen Streifen Wald, der nun urbar gemacht werden
musste. Seebohm baute dort zunächst sein eigenes Wohnhaus, dann Manufakturen,
Mühlen und weitere Wohnbauten. Die Kolonie,
Friedensthal\index{Orte:!Friedensthal} genannt, wuchs, immer mehr Quäker fanden
sich ein.

\medskip

Von Anfang an war Seebohm der unbestrittene Anführer, der intellektuelle Kopf
und der charismatische Leiter seiner eigenen kleinen Gemeinde. Anerkennung und
Legitimation von außen ließen nicht lange auf sich warten: 1798 war Seebohm von
den amerikanischen Quäkerinnen Mary Swett\index{Personen:!Swett, Mary} (ca.
1739-1821), Sarah Harrison (1746-1812) und Charity Cook\index{Personen:!Cook,
Charity} (1745-1822) unter Handauflegung und den Worten \textit{"`Ich salbe dich mit dem
Freuden-Öl, mehr als deine Genossen"'} zum \textit{"`recorded Minister"'}\index{Minister!recorded}\index{Ministry} ernannt worden.\footnote{"`Recorded Minister"' hat nichts mit einem
heutigen Minister im politischen
Sinne zu tun, sondern meinte einen begabten Prediger, der die Zustimmung der
Gemeinde zum unbezahlten Predigtamt hatte.} Diese sogenannte
\textit{"`Liebestaufe"'}\index{Taufe!Liebestaufe} oder Salbung\index{Salbung}, die nach
1. Johannes 2,27 als Verschmelzung mit dem Heiligen Geist\index{Geist!Heiliger}
gedeutet wurde, stärkte seine Führungsposition und Autorität erheblich.

\medskip

Im Gegensatz zu manchen Idealisten war Seebohm klar, dass der Mensch nicht von
Brot alleine lebt, sondern dass die neuen Quäker nach geistiger Nahrung
verlangten. Hier half ihm sein Sprachtalent, und er fand immer wieder die Zeit
und Ruhe, sich an eigene Texte zu machen -- am bekanntesten sind hier vielleicht
seine "`\textit{Bemerkungen über verschiedene Gegenstände des Christentums}"',
die er bereits 1794 veröffentlichte. In den ersten Jahren trat Seebohm auch als
Verfasser strenger Mahn- und Trostbriefe hervor, die an Einzelpersonen wie an
die ganze Gemeinde gerichtet sein konnten. Darunter finden sich Warnungen vor
dem "`\textit{falschen Licht}"', vor dem Alkohol\index{Alkohol} oder vor dem
Hochmut\index{Hochmut} -- alles Themen, die Seebohm, wie er betont, aus eigener
Erfahrung allzu bekannt waren, die aber auch der deutschen Quäkergemeinde zu
schaffen machten.\footnote{Kommunalarchiv Minden, WR 1, Nr. 15: Korrespondenz
Ludwig Seebohm,
1797-1802.}

\medskip

Vor allem Schriften von Robert Barclay\index{Personen:!Barclay, Robert} und
William Penn brachte er ins Deutsche und sorgte
für ihre Verbreitung. Beispielsweise überreichte er 1830 persönlich dem
Philosoph Karl Christian Friedrich Krause\index{Personen:!Krause, Karl Christian
Friedrich} (geb. 1781) eine Ausgabe seiner Übersetzung Penns "`\textit{Früchte
der Einsamkeit}"'.\footnote{Karl Christian Friedrich Krause: Anschauungen oder
Lehren und Entwürfe zur
Höherbildung des Menschheitlebens, 3, Leipzig 1892, S. 204. Krause hat jedoch
von den Quäkern nicht viel verstanden und verwechselt sie mit den Herrnhutern.}
Das Werk von Krause wird diese Gabe kaum beeinflusst haben, denn Krause verstarb
schon 1832. Ein weiteres Interessensgebiet waren Schulbücher. Seebohm verfasste
eine englische Grammatik sowie ein "`\textit{Buchstabier-Lesebuch zum
Unterrichten}"' -- das erste Quäkerbuch für Kinder\index{Kinder} in
deutscher Sprache. Den Druck seiner Schulbücher\index{Schulbuch}, immerhin 300
Exemplare insgesamt, übernahm er in eigener Verantwortung. Der Verkauf ging
jedoch derart schlecht, dass Seebohm schließlich die überschüssigen Exemplare
den deutschen Quäkern schenkte.\footnote{LSF, MS 127, Pyrmont, Nr. 150: Ludwig
Seebohm, Friedensthal, 15.4.1815.
Freilich war es ein Problem, dass die Konfessionsschulen um keinen Preis ein
Schulbuch angeschafft hätten, das von einem Quäker geschrieben wurde -- völlig
ungeachtet der Quälität.} Freilich wurden die Bücher auch in seiner
eigenen Schule\index{Schule} verwendet, denn Seebohm gab bereits 1796
fünfundzwanzig Kindern Unterricht. Seine pädagogischen Fähigkeiten waren
geschätzt, sogar aus England\index{Orte:!England} wurden Kinder nach
Friedensthal gebracht, wo sie für meist ein Jahr Deutsch lernten.\footnote{Das
ganze war eine Art frühes internationales Austauschprogramm, denn
umgekehrt wurden Kinder aus Friedensthal für ein Jahr nach England geschickt.}

\medskip

Nachweislich beabsichtigte Seebohm um das Jahr 1791, das gesamte Alte Testament
neu zu übersetzen. Grundlage sollte unter anderem die englische Übersetzung der
Berleburger Bibel\index{Bibel}\index{Berleburger Bibel} sein.\footnote{Archiv
der Christentums-Gesellschaft, DV 13, 17.} Wahrscheinlich ist er von dem Pastor
Johann Friedrich Delkeskamp\index{Personen:!Delkeskamp, Johann Friedrich}
(1731-1805), mit dem er sich darüber beriet, abgehalten worden, da dieser zu
einer Bibelübersetzung\index{Bibel!Übersetzung} ohne Griechisch- und
Hebräischkenntnisse abriet. So kam es leider nicht zu dem bemerkenswerten
Vorhaben einer Quäkerbibelübersetzung, die die erste weltweit gewesen wäre.

\medskip

Gelegentlich ergaben sich für Seebohm auch auswärtige Übersetzungsaufträge. 1795
erhielt er vom Londoner "`\textit{Meeting for Suffering}"'\index{Meeting for
Sufferings}, einer wichtigen Quäkerbehörde in London\index{Orte:!London}\index{Quaker!-behörde in London}, 2,2
Pfund für die Übersetzung der Schrift "`\textit{Serious Call}"' von Benjamin
Holme\index{Personen:!Holme, Benjamin} (1683-1749). 1802 erhielt er für
Übersetzungsarbeiten 47 Pfund vom Meeting for Sufferings überwiesen, und 24
Pfund für die Übersetzung von William Tukes \textit{"`Principles"'} 1816.\footnote{LSF,
Meeting for Sufferings, XLII, 1815-1823; LSF, MS 127, Pyrmont, S. 153:
Ludwig Seebohm an John Eliot, Hamburg, 19.3.1816. Während seines Aufenthaltes in
Hamburg wohnte er im Petri-Kirchhof.}

\medskip

Nun war es mit dem Übersetzen allein nicht getan, die Bücher mussten auch
gedruckt werden. Hier machten die Druckereien\index{Druckerei}, die ja stets in
konfessioneller Hand waren, immer wieder Schwierigkeiten. So wurde in
Friedensthal eine eigene Buchdruckerei, die erste deutsche
Quäkerpresse\index{Quaker!-druckerei (deutsche)}, eingerichtet, und, um gleich
richtig loslegen zu können, auch eine Papierfabrik\index{Quaker!-Papierfabrik}
errichtet. Alle in Friedensthal entstandenen Drucke stammen aus diesem Betrieb.
Die Schriften wurden zu Werbezwecken unter Pietisten\index{Pietist}
verteilt -- so erhielt beispielsweise David Francks\index{Personen:!Francks,
David} bis 1795 einhundert Drucke von Seebohm nach Exten\index{Orte:!Exten} (bei
Porta Westfalica)\index{Orte:!Porta Westfalica} zugeschickt, um sie dort zu
verteilen. Friedensthaler Druckerzeugnisse fanden sogar den Weg über den
Atlantik in amerikanische Haushalte in Germantown\index{Orte:!Germantown},
Philadelphia\index{Orte:!Philadelphia} und Baltimore\index{Orte:!Baltimore}.
Einige der Bücher wurden auch in der Helwingschen Hofbuchhandlung zu Pyrmont und
Hannover\index{Orte:!Hannover} vertrieben -- offensichtlich hatte man in
Hannover Dank der Personalunion des Welfenhauses
(der Kurfürst von Hannover war gleichzeitig englischer König) Quäker kennen
gelernt oder hatte zumindest eine gute Meinung von diesen gewonnen.

\medskip

Die Französische Revolution\index{Revolution!Französische}, die
Koalitionskriege\index{Krieg!Koalitionskrieg} und dann die
Freiheitskriege\index{Krieg!Freiheitskrieg} setzten der vielversprechenden
Kolonie ein abruptes Ende. Auch Seebohm geriet in eine Finanzkrise und musste
sogar Schulden\index{Schulden} machen: In Bremen\index{Orte:!Bremen} lieh er
sich von dem Hamburger Kaufmann Johann Christoph Saphir\index{Personen:!Saphir,
Johann Christoph} und dem Bremer Senator Johann Vollmer\index{Personen:!Vollmer,
Johann} (1753-1818) je 1.500 Louis d’or (7.500 Reichstaler).\footnote{Hessisches
Staatsarchiv Marburg, 118a Waldeckisches Kabinett Nr. 3429: Vier
Schreiben zur Schuldforderung des Johann Christoph Saphir an Ludwig Seebohm,
1811.} Das Leihen von
Geldern widersprach damals den Grundsätzen\index{Quaker!Grundsätze} der Quäker,
und Verstöße wurden hier sehr ernst genommen. Seebohms Ansehen wurde zunächst
von den Londoner Quäkern\index{Quaker!Londoner} wiederhergestellt: Im
Mai 1799 sammelten Londoner Quäker Gelder mit dem Ziel, Seebohm schuldenfrei zu
machen. Dabei wurde festgelegt, dass ein etwaiger Überschuss für den Bau eines
Schulraumes und eines Versammlungshauses\index{Versammlungshaus} Verwendung
finden sollte. In der erhaltenen Liste der edlen Spender finden sich Namen der
bedeutendsten Quäker um 1800: der bereits erwähnte George
Dillwyn\index{Personen:!Dillwyn, George}, der Meteorologe Luke
Howard\index{Personen:!Howard, Luke} (1772-1864) oder der
Sozialreformer William Tuke\index{Personen:!Tuke, William} (1732-1822). Ende
1799 bis 1800 kam es dann tatsächlich zu dem Bau dieses Versammlungshauses,
wofür Seebohm den Grundriss zeichnete. Das noch heute vorhandene Pyrmonter
Quäkerhauses\index{Quaker!-haus} ist also eigentlich den Schulden Seebohms zu
verdanken.

\medskip

Im Winter 1806/07 ereignete sich eine neue Katastrophe: das Friedensthaler Haus
der Familie Seebohm wurde von marodierenden französischen Truppen
geplündert\index{Plünderung}, während sich Julia
Seebohm\index{Personen:!Seebohm, Julia} mit ihren Kinder auf dem Dachboden
versteckt hielt, den Überfall unbeschadet überstand, aber kurz darauf entkräftet
verstarb. Die finanzielle Lage Seebohms hatte sich auch nach der Entschuldung
nicht grundlegend verbessert und er musste sich nach neuen Einnahmequellen
umsehen, um zu überleben. Die Quäkerpublikationen warfen nichts ab, der Handel
lag danieder, die Schule brachte viel Mühsal und Ärgernis, aber kaum Einnahmen.
Da zeichnete sich ein Silberstreifen am Horizont ab: 1805 trat der neue Fürst
Georg\index{Personen:!Georg, Fürst} (1747-1813) die Herrschaft von Pyrmont an.
Unter diesem war Seebohm ab 1807 als fürstlicher Brunnendirektor und als
Baudirektor tätig.

\medskip

Im August 1813 musste Seebohm jedoch eine erhebliche Gehaltskürzung hinnehmen
und verschuldete sich erneut. Bei einer stattlichen Anzahl von Personen --
darunter selbst der Gastwirt Franz Reesen\index{Personen:!Reesen, Franz} im nahegelegenen Aerzen\index{Orte:!Aerzen} -- stand Seebohm in der
Kreide.\footnote{Hessisches Staatsarchiv Marburg, 125 Waldeckische Kammer Nr.
3897, Brunnen-
und Baudirektor Seebohm 1810-1817.} Diesmal halfen die Quäker nicht. Um seine
Einnahmen zu steigern, vermietete Seebohm schließlich sein Friedensthaler Haus
und zog mit seinen Kindern als Witwer für einige Monate nach Pyrmont. In seiner
Verzweiflung entwickelt Seebohm ganz unterschiedliche Projekte. Zunächst will er
mit seinem Sohn Benjamin\index{Personen:!Seebohm, Benjamin}, der schon längst
nach England\index{Orte:!England} emigriert\index{Emigration} war und dort ein
besseres Leben führte, in den Export\index{Export} von Besteck und in den
Wollhandel einsteigen.\footnote{Hertfordshire Archives and Local Studies,
DE/Se/C24/1, 27 12m 1814;
Hertfordshire Archives and Local Studies, DE/Se/C24/8, 1817-1819: 5 letters from
Johann, Samuel and William Seebohm, Pyrmont; Hertfordshire Archives and Local
Studies, DE/Se/C24/17: Letter from Benjamin Seebohm to J. Hustler, Bradford,
1819.}
Dann wieder empfiehlt er, in Pyrmont eine
Brauerei\index{Alkohol!Brauerei}\index{Brauerei} einzurichten, denn\textit{"`good beer
brings people in from drinking drams (Branntwein), and is healthy and
nourishing"'}\footnote{Hertfordshire Archives and Local Studies, DE/Se/C24/1 18
October 1818.}. 1819 beschäftigte er sich mit dem Vorhaben, Pyrmont zu einem
internationalen Messeplatz neben Leipzig, Braunschweig und Hannover zu
etablieren. Dazu sollten in Pyrmont neue Gewölbe zur Warenlagerung gebaut
werden. Sein umfangreiches Gutachten, das erhalten ist, wurde jedoch vom Fürsten
nicht umgesetzt,\footnote{Stadtarchiv Bad Pyrmont, 1668 bis 1848, A I 66.} und
auch aus den anderen Vorhaben wurde nichts.

\medskip

Welche Stellung hatte Seebohm zu dieser Zeit unter den Quäkern? Ab 1807 nahm er
nicht mehr an den Geschäftsversammlungen\index{Quaker!Geschäftsversammlung} und
den Andachten\index{Quaker!Andacht} der Quäker teil. 1808 wurde von den
deutschen Quäkern nach London\index{Quaker!London} die Nachricht
gegeben, die gesamte Korrespondenz ausschließlich an Ludwig
Heydorn\index{Personen:!Heydorn, Ludwig}, Diedrich
Seebohm\index{Personen:!Seebohm, Diedrich} oder Heinrich
Meyer\index{Personen:!Meyer, Ludwig} zu richten, jedoch keinesfalls länger an
Ludwig Seebohm -- diesem könne man nicht trauen.\footnote{LSF, MS 127, Pyrmont,
Nr. 78: Ludwig Heydorn et al. an das Meeting for
Sufferings, Pyrmont, 3.8.1808.} Die Quäker hatten seiner Tätigkeiten bei Hofe
gegenüber Vorbehalte, da sie ihn aus Friedensthal hinausführte und von der
einfachen Quäkergemeinschaft zu entfremden drohte. Die Tätigkeit beim Fürsten
wurde als ein gefährliches Einlassen mit der Obrigkeit angesehen. Die Gunst des
neuen Fürsten habe sich Seebohm erschlichen, indem er sich als zuverlässiger und
vertrauenswürdiger Quäker für dessen Handelsgeschäfte
prostituiert\index{Prostitution} habe.\footnote{LSF, MS 127, Pyrmont, Nr. 105:
Schreiben von Ludwig Heydorn an das Meeting
for Sufferings, 7.1.1810; siehe auch Archiv Quäkerhaus Bad Pyrmont, Mappe V,
Reich-Collection, III, 3-4.}
Dann gab es immer wieder finanzielle Ungereimtheiten: 1807 hatte Seebohm einen
Wechsel über 48 Pfund eingelöst -- eine stolze Summe in Krisenzeiten. Seebohm
hatte sich dabei nicht mit dem Schulausschuss\index{Quaker!Schulausschuss}
verständigt, sondern eigenmächtig gehandelt, und die deutschen Quäker
beschwerten sich nun zu Recht in London, dass der Wechsel -- entgegen der
Handelssitte -- akzeptierte worden war, obwohl er allein die Unterschrift
Seebohms trug.\footnote{LSF, MS 127, Pyrmont, Nr. 105: Schreiben von Ludwig
Heydorn an das Meeting
for Sufferings, 7.1.1810.}

\medskip

Wie häufig bei solchen Konflikten\index{Quaker!-Konflikte} ging es jedoch um
etwas ganz anderes. Schon seit langem war das Verhältnis zwischen Ludwig Seebohm
und Heinrich Meyer angespannt.\footnote{Zu den Einzelheiten siehe: Claus Bernet
(Hrsg.): Deutsche Quäkerschriften,
2, Hildesheim 2007, S. 458-462.} Heinrich Meyer war ebenfalls bei den deutschen
Quäkern von Anfang an dabei, arbeite in vielen Ausschüssen mit und verfasste
sogar Schriften der deutschen Quäker. Er blieb jedoch hinter Seebohm als ewiger
Zweiter zurück und konnte das offensichtlich nicht ertragen. Als Mitglied im
Schulausschuss war er 1807 von dem eigenmächtigen Verhalten Seebohms unmittelbar
betroffen und betrieb nun offen die
Exkommunikation\index{Quaker!Exkommunikation} Seebohms.

\medskip

1810 ist Seebohm noch formal Mitglied der Quäkergemeinde,\footnote{LSF
Continental Committee Minutes, 66.} kurz darauf muss er entweder die
Gesellschaft von sich aus verlassen haben oder wurde, was zu vermuten ist, von
ihr ausgeschlossen. Seebohm akzeptiert jedoch diesen
Ausschluss\index{Quaker!Ausschluss} keineswegs, sondern bestand auf seiner
weiteren Teilnahme am Sozialleben der Quäkergemeinde. In den Andachten, die er
seit 1811 wieder kontinuierlich besuchte, riefen seine Predigten nur
Beklommenheit und Beschämung hervor. Ein Ausschluss von den öffentlichen
Andachten war jedoch kaum möglich -- anders als von den internen
Geschäftsversammlungen\index{Quaker!Geschäftsversammlung}, zu deren Teilnahme
Seebohm kein Recht mehr hatte. Dennoch versuchte er drei Mal, sich gewaltsam
Zutritt zu verschaffen (so stellten es die Quäker jedenfalls dar).\footnote{LSF,
MS 127, Pyrmont, Nr. 110: Ludwig Heydorn an London Yearly Meeting,
Pyrmont, 1. bis 7.4.1811.}

\medskip

Das Leben Seebohms als bekanntester deutscher Quäker ging auch ohne formale
Mitgliedschaft weiter. Zunächst fand er Trost in weiblichen Armen: Im Sommer
1815 heiratete er Louise Henriette Eisel\index{Personen:!Eisel, Louise
Henriette} aus Rinteln\index{Orte:!Rinteln} (1790-1870). Obwohl Eisel
reformierten Glaubens war und ihr zukünftiger Mann ein ehemaliger Quäker, fand
die Hochzeitsfeier in der lutherischen Oesdorfer Kirche (gehört heute zu Bad
Pyrmont) statt \index{Quaker!Mischehen}-- eine eigenartige ökumenische
Veranstaltung. Noch wenige Jahre zuvor war Seebohm in Ausschüssen tätig gewesen,
die Mitglieder wegen des Eingehens von Ehen mit
"`\textit{Andersgläubigen}"'\index{Andersgläubige}, zumeist
Lutheranern\index{Personen:!Lutheraner}, aus der Gemeinschaft
verstoßen\index{Quaker!Ausschluss} hatten, nun sah er sich selbst nicht mehr an diesen
Grundsatz gebunden.

\medskip

1819 verstarb der Opponent Meyer, und Ludwig Seebohm konnte das Vertrauen der
Quäker in Friedensthal zurückgewinnen. Nun jedoch gab es Quäker in
Minden\index{Orte:!Minden}, die ihn nicht als Mitglied sehen wollten: Ihm wurde
vorgeworfen, lediglich äußerlich fromm zu erscheinen. Die Gemeinde war
gespalten, es gab leidenschaftliche Befürworter und ebenso leidenschaftliche
Gegner einer erneuten Aufnahme. 1826 war es soweit, Seebohm stellte formal
seinen Wiederaufnahmeantrag\index{Quaker!Wiederaufnahmeantrag}. Über Jahre wurde
die Sache nun durch Verzögerungstaktiken, Vertagungen und langatmigen
Schriftverkehr hinausgezögert. Seebohm, der immer älter wurde, verpflichtete
sich in seiner Not sogar schriftlich, die Ermahnungen der Quäker in Zukunft
anzunehmen -- eine Erniedrigung, die ihm sicher nicht leicht gefallen ist. Doch
der Wunsch auf Zugehörigkeit war stärker. Selbst eine kleine internationale
Sonderkonferenz 1831 in der Angelegenheit Seebohm mit anwesenden Engländern und
Amerikanern schien zu scheitern; als dann doch die Mindener Quäker -- vermutlich
unter Druck der anwesenden Ausländer -- überraschend ihre Blockadehaltung
aufgaben. Bereits in der Anwesenheitsliste der Geschäftsversammlung vom 1.
Februar 1832 wird Ludwig Seebohm wieder unter den Anwesenden aufgeführt, gleich,
wie in alten Tagen, oben an zweiter Stelle. Er arbeitete sofort wieder in den
Angelegenheiten der Gemeinde mit. Viel Zeit war ihm jedoch nicht mehr vergönnt.
Seebohm besuchte noch im Winter 1832/33 eine neue Quäkergemeinde, die in
Barmen\index{Orte:!Barmen} entstanden war,\footnote{Claus Bernet: Wilde Ehen,
andere Provokationen und eine Zichorienfabrik.
Die erste Quäkergemeinde Barmens zur Zeit der Protoindustrialisierung um 1830,
in: Zeitschrift des Bergischen Geschichtsvereins, 101, 2008, S. 95-108.} und
machte 1833 seinen Traum wahr, die Jahresversammlung der Quäker in London zu
besuchen. Dort wurde der bestens bekannte Seebohm wie ein
Prophet\index{Prophet} gefeiert, und er erhielt für die deutschen
Quäker 50 Pfund, um endlich wieder Quäkerbücher in Deutschland zu
drucken.\footnote{LSF, Meeting for Sufferings, XLIV, 1831-1839, S. 145.} Kurz
darauf, am 22. März 1835, ist Ludwig Seebohm in Friedensthal verstorben.

\section{Zur Rezeptionsgeschichte}\label{ref:rezeptionsgeschichte}

Blicken wir noch einmal auf die frühen 1820er Jahre, in denen Seebohm \textit{"`No
cross, no crown"'} übersetzte. Aus dem oben Gesagten wird nun klar, warum sich Seebohm
mit dieser Schrift besonders identifizieren konnte. Generell muss man sagen,
dass wir aus dieser Zeit nicht so viel wissen, weil die Quäkerprotokolle\index{Quaker!-protokoll} meist
nur Informationen zu Mitgliedern beinhalten. Dennoch sind ein paar Umstände
bekannt: Die frühen 1820er Jahre waren für Seebohm eine besondere Krisenzeit.
Erstmals hat er seinen Wohnsitz nicht mehr in Friedensthal, sondern ab 1821 bis
mindestens Ende 1822 in Bielefeld, wo er am dortigen Gymnasium\index{Gymnasium} die englische
Sprache unterrichtete. Er ist verarmt und pflegt enge Kontakte mit der
evangelischen Kirche\index{Kirche!evangelische},\footnote{Landeskirchliche Archiv der Evangelischen Kirche
von Westfalen, Archiv des
Kirchenkreises Herford, Nr. 28: "'Was ich von den Quäkern weiß..."', Brinkdöpke,
Herford, 16.2.1827.} wohl auch, um seine Stelle nicht zu verlieren. Ab
1824, als er wieder in Friedensthal wohnt, gab er auch den Kindern aus
Quäkerfamilien wieder Unterricht. In dieses Jahr fiel noch ein ganz besonderes
Ereignis:\footnote{Siehe dazu jetzt: Claus Bernet: 300 Jahre angloamerikanische
Beziehungen in
Berlin: Die Quäkerpräsenz vom 17. Jahrhundert bis heute, in: Jahrbuch
Berlin-Brandenburgische Kirchengeschichte, 67, 2009.} 1824 traf Seebohm, in
Begleitung der englischen Quäker Thomas Shillitoe\index{Personen:!Shillitoe, Thomas} (1754-1836) und Thomas Christy
(1776-1846) in Berlin\index{Orte:!Berlin} auf den preußischen König und den Kronprinzen\index{Ort:!Preußen}, wobei die
Quäker für Gewissensfreiheit und Duldung der kleinen Gemeinde im preußischen
Minden\index{Orte:!Minden} eintraten. Eine geplante Weiterreise Seebohms nach St.~Petersburg\index{Orte:!St. Petersburg} wurde
aus nicht bekannten Gründen abgebrochen.

\medskip

In diesen Jahren muss Seebohm die Übersetzung\index{Übersetzung} von \textit{"`No cross no crown"'}
angefertigt haben. Von seinen anderen Projekten wissen wir, dass er nicht
besonders schnell vorging, sondern solche Arbeiten vor sich herschob. Die Pläne
einer ersten Übersetzung können durchaus weit zurückliegen. Schon im Jahre 1804
berichteten einige deutsche Quäker -- Seebohm befindet sich nicht darunter -- von
dem Plan, \textit{"`No cross no crown"'} zu übersetzen. Die Engländer\index{Quaker!englische} haben dieses
Vorhaben jedoch nicht gefördert, und so ist erst einmal nichts daraus
geworden.\footnote{LSF, MS 127, Pyrmont, S. 84.} Seebohms Übersetzung von 1825
diente natürlich auch dazu, die deutschen Quäker zu beeindrucken. Ob sie davon
zuvor wussten, ist eher unwahrscheinlich, da das Projekt in den Protokollen der
deutschen Quäker nicht erwähnt ist, da Seebohm ja formal zu diesem Zeitpunkt
kein Mitglied war. Andererseits hat er in den 1820er Jahren durchaus andere
Bücher im Auftrag der deutschen Quäker übersetzt, die letztlich froh waren,
einen unterbezahlten Übersetzer bei der Hand zu haben. Eineinhalb Reichstaler
hat das Werk zunächst gekostet -- das sind, bei aller Problematik von
Umrechnungs- und Vergleichswerten, in etwa 15 Euro gewesen.\footnote{Thesaurus
librorum rei Catholicae, Würzburg 1848, S. 626.} Gedruckt wurde das Buch im
Verlag von Heinrich Gelpke (geb. 1775) in Pyrmont, und von der Hofbuchhandlung\index{Hofbuchhandlung}
des Georg Uslar\index{Personen:!Uslar, Georg} wurde es vertrieben, die beide den Quäkern nahe standen und bei denen die
Quäker auch andere Werke in den Druck gaben.\footnote{Etwa die Schrift \textit{"`Ernste
Untersuchung des Gebrauches, Krieg zu führen"'},
Pyrmont 1819.} Anders als in Preußen\index{Orte:!Preußen} war es in
Pyrmont, das zum liberalen Fürstentum Waldeck-Pyr\-mont\index{Orte:!Fürstentum Waldeck-Pyrmont} gehörte, stets möglich,
Quäkerschriften\index{Quaker!-schriften} zu drucken. Die Hofbuchhandlung Uslar erwarb merkwürdigerweise
1825 die Messerfabrik\index{Messerfabrik} zu Friedensthal, die von Quäkern gegründet worden
war.\footnote{Margarethe Tinnappel-Becker: Chronik von Löwensen, Bad Pyrmont
1988, S. 64.} Insofern ist es verständlich, dass die Buchhandlung nicht zögerte,
ein Werk der Quäker zu diesem Zeitpunkt herauszubringen.

\medskip


Leider hatten es schon die Quäker damals verpasst, für ihre Publikationen auch
angemessen zu werben\index{Werbung}. Folgerichtig spielt sich die Rezeptionsgeschichte\index{Rezeption} von
\textit{"`Ohne Kreuz keine Krone"'} außerhalb des engen Kreises der deutschen Quäker ab.
Die Zeit für dieses Buch war günstig gewählt, der Neupietismus\index{Neupietismus} \index{Pietismus} stand vor der
Tür. Schon 1847 erschien eine weitere Fassung unter dem gleichen Titel, die von
dem Londoner Verlag "`Wertheimer~\&~Company"'\index{Verlag!Wertheimer~\&~Company} gedruckt wurde. Ob hinter dem
Druck englische oder deutsche Quäker standen ist ebenso unbekannt, wie oder ob
diese Drucke nach Deutschland gelangten, oder ob sie für den nordamerikanischen
Markt\index{Markt!nordamerikanischer} vorgesehen waren.

\medskip

Vor diesen Übersetzungen spielte \textit{"`No cross no crown"'} bei deutschsprachigen
Autoren keine nennenswerte Rolle, aus drei Gründen: Zunächst gab es in Deutschland
einfach zu wenige Quäker als potentielle Leser. Dann bekämpfte und ignorierte das Luthertum alles,
was nicht der eigenen Linie entsprach. Und schließlich wurde diese Arbeit von Penn irgendwie übersehen,
da die Polemiken von Fox oder Nayler viel mehr Angriffsfläche boten. In den hunderten von
deutschsprachigen Streitschriften\index{Streit!-schrift} des 17.
und frühen 18. Jahrhunderts wird aus dem Werk selten zitiert. Das berühmte
Universal-Lexikon\index{Universal-Lexikon} führt in dem Eintrag zu William Penn das Buch nicht einmal
an.\footnote{Penn (Wilhelm), in: Großes vollständiges Universal-Lexicon, Aller
Wissenschafften und Künste, Welche bißhero durch menschlichen Verstand und Witz
erfunden und verbessert worden, (...), XXVII, Halle 1741, Sp. 259-262.}
Allerdings gibt es ein kurzes Gedicht des niederrheinischen Pietisten Gerhard
Tersteegen (1697-1769), das den Titel \textit{"`Ohne Kreuz keine Krone"'} trägt:

\begin{verse}\index{Gedicht}
\textit{
\underline{Ohne Kreuz keine Krone}\\
% \smallskip
Wer außer dir die wahre Ruh,\\
O teures Kreuz, vermeint zu finden,\\
Der ist betrogen, und dazu\\
Wird seine Kron’ der Glorie ganz verschwinden.\footnote{Gerhard Tersteegen:
Geistliches Blumen-Gärtlein Inniger Seelen, Franckfurt
1729, S. 671.}}
\end{verse}

Nach den zwei Drucken stieg selbstverständlich die Kenntnis von dem Text und man
kann, wenn auch nur ansatzweise, weitere Reaktionen entdecken. 1825 erschien das
\textit{"`Schatzkästchen"'}\index{Schatzkästchen}, eine fromme Liedersammlung\index{Liedersammlung}, die massenhaft vertrieben und
vielfach aufgelegt wurde. Darin findet sich folgende Gegenüberstellung:\textit{"`Ohne
große Trübsal keine große Freude; ohne Kreuz keine Krone; ohne Kampf kein Sieg.
Ohne Wehen keine Geburt"'}.\footnote{Johannes Gossner: Schatzkästchen enthaltend
biblische Betrachtungen mit
erbaulichen Liedern auf alle Tage im Jahre zur Beförderung häuslicher Andacht
und Gottseligkeit, Berlin 1860, S. 316 (Erstauflage Leipzig 1825, also im Jahr
des Erscheinens von \textit{"`Ohne Kreuz keine Krone"'}).} Der Autor ist ein gewisser
Johannes Goßner\index{Personen:!Goßner, Johannes} (1773-1858), ein Berliner\index{Orte:!Berlin} pietistischer Prediger\index{Prediger!pietistisch}\index{Pietismus}. Wir wissen von
seinen Quäkerkontakten\index{Quaker!-kontakte}, und 1832 wurde er von William Allen\index{Personen:!Allen, William} (1770-1843) aus
England und Stephen Grellet\index{Personen:!Grellet, Stephen} (1773-1855) aus den USA\index{Orte:!USA} besucht.\footnote{Siehe dazu
jetzt: Claus Bernet: 300 Jahre angloamerikanische Beziehungen in
Berlin: Die Quäkerpräsenz vom 17. Jahrhundert bis heute, in: Jahrbuch
Berlin-Brandenburgische Kirchengeschichte, 67, 2009.} Vermutlich hat diese
Liedsammlung den Spruch erst so richtig bekannt gemacht. Gegenüber deutschen
Lesern wurde 1847 in einem Schulbuch\index{Schulbuch} sogar behauptet, dieses sei der Wahlspruch\index{Wahlspruch}
von Penn gewesen.\footnote{Julius Kell: Biblische Lehrstoffe für den gesamten
religiösen Unterricht
in allen Classen evangelischer Volksschulen so wie für den
Confirmandenunterricht, Leipzig 1843, S. 537.}

\medskip

1878 wurde der Grundstein der Großen Kreuzkirche\index{Kirche!Große Kreuzkirche} in Hermannsburg\index{Orte:!Hermannsburg (bei Celle)} (bei Celle)
gelegt. Es ist ein Bau der Selbständigen Evangelisch-Lutherischen Kirche (SELK)\index{Kirche!Selbständigen Ev.-Luth. Kirche (SELK)},
eine Freikirche\index{Kirche!Freikirche} altkonfessioneller\index{Kirche!altkonfessionell} Prägung, die sich selbst als lutherische
Bekenntniskirche\index{Kirche!Bekenntniskirche} bezeichnet. \textit{"`Ohne Kreuz keine Krone"'} wurde zum Wahlspruch
dieser Gemeinde; er findet sich sowohl auf dem Grundstein gemeißelt als auch
über dem Hauptportal der Kirche. In einer Selbstdarstellung der Gemeinde kann
man dazu lesen: \textit{"`Der Spruch 'Ohne Kreuz keine Krone' über dem Hauptportal der
Kirche will sagen, dass es nur eine Brücke gibt, die über den Graben des Todes
hinweg zu Gott führt, nämlich das Kreuz Jesu
Christi"'}.\footnote{
http://cms.heidekirchen.de/pages/suedheide/uebersicht/groDFe-kreuzkirche-hermann
sburg.php} Ob die Gemeinde 1878
wusste, dass dieser Spruch, den der erste Pastor Theodor Harms auswählte, mit Penn zusammenhing,
ist nicht bekannt. Da die Kreuzkirche entstand, weil dieser Pastor amtsenthoben wurde, als er sich
weigerte, eine neue Trauagende zu benutzen und die junge Gemeinde viel Widerstand und Anfeindung
auf sich zog, hat sie ihren Weg auch als Kreuzweg gesehen, aber im Sinne lutherischer Kreuzestheologie.

\medskip

Zwar ist es richtig, dass der Spruch \textit{"`Ohne Kreuz keine Krone"'} heute eng mit
William Penn verbunden ist, doch offensichtlich ist er wesentlich älter. Es soll
der Wahlspruch von Florian Geyer\index{Personen:!Geyer, Florian} (1490-1525) gewesen sein, ein fränkischer
Ritter und Diplomat. Obwohl er adeliger Herkunft war, schloss er sich den Bauern
an und kämpfte für eine Gesellschaft auf Grundlage des Evangeliums, wie es Geyer
bei Luther\index{Personen:!Luther, Martin}, den er persönlich kannte, verstanden hatte. Der Spruch \textit{"`Ohne Kreuz,
keine Krone"'} oder seine lateinische Variante \textit{"`Nulla crux, nulla corona"'} soll
auf seinem Schwert\index{Schwert} eingraviert gewesen sein. Die Frage ist nur, ob es sich um
eine historische Tatsache oder um eine Legende handelt. Bekannt wurde der Spruch
durch seine Verwendung in Gerhart Hauptmanns\index{Personen:!Hauptmann, Gerhart} historischem Revolutionsdrama
\textit{"`Florian Geyer"'} aus dem Jahr 1896.\footnote{Gerhart Hauptmann: Florian Geyer,
Berlin 1896, S. 302.} Eine Generation später wurde der Spruch im
Nationalsozialismus\index{Nationalsozialismus} neu gedeutet. Egon Harnapp\index{Personen:!Harnapp, Egon} sieht Geyer als \textit{"`das Idealbild
des deutschen Menschen, über dessen Leben gleichsam der auf seinem Schwert
stehende Spruch zu stehen scheint: Ohne Kreuz, keine Krone (‚Nulla crux, nulla
corona)"'}.\footnote{Egon Harnapp: Masse und Persönlichkeit, Chemnitz 1933, S. 57.}
Robert Bauer\index{Personen:!Bauer, Robert} (1898-1965), ein Politiker der NSDAP\index{NSDAP}, setzt \textit{"`Nulla crux, nulla
corona"'} als Wahlspruch über eines seiner Kapitel in einem Buch zu England.\footnote{Robert Bauer:
Irland, die Insel der Heiligen und Rebellen, Leipzig 1938, S.
131.}
Dagobert von Mikusch\index{Personen:!Mikusch, Dagobert von}, ein Verfasser scheinbar harmloser, im Grunde aber
völkischer Abenteuer- und Reisegeschichten, gibt 1941 irrtümlich an, der Spruch
sei dem Wappen von Geyer entnommen.\footnote{Dagobert von Mikusch: Florian Geyer
und der Kampf um das Reich, Berlin
1941, S. 230.} Auch nach 1945 wird \textit{"`Ohne Kreuz keine
Krone"'} mit Geyer verbunden. Dieser Hinweise findet sich am Ende von Ernst
Blochs\index{Personen:!Bloch, Ernst} \textit{"`Atheismus im Christentum"'}\footnote{Ernst Bloch: Atheismus im
Christentum. Zur Religion des Exodus und des
Reichs, Frankfurt a. M. 1968, S. 353 (Gesamtausgabe, 14).} ebenso wie in
wissenschaftlichen Fachjournalen.\footnote{Heinz-Horst Schrey: Atheismus im
Christentum? in: Frankfurter Hefte.
Zeitschrift für Kultur und Politik, 24, 6, 1969, S. 418-428, hier S. 423.}
Letztlich ist es nicht erwiesen, ob Geyer diesen Spruch überhaupt
kannte\footnote{In seinem auch heute noch maßgeblichen Werk zu Geyer erwähnt
Barge den
Spruch jedenfalls nicht; Hermann Barge: Florian Geyer. Eine biographische
Studie, Berlin 1920, Reprint Hildesheim 1972. Viel wahrscheinlicher ist, dass
Penn mit dem Titel an die Abschiedsworte eines gleichnamigen Vaters von 1670
erinnern wollte, die in Deutsch in etwa lauten:\textit{"`Trage dein Kreuz, Lieber. Steh
fest zu Gott, leg Zeugnis ab für deine Zeit und deine Generation. Und Gott wird
dir die Ruhmeskrone der Ewigkeit geben, die niemand von dir nehmen kann"'}; 
Emilia Fogelklou: William Penn: Quäker und Staatengründer, Leipzig 1963, S. 50.}
und falls ja, ob Penn davon wusste. Dies ist aber eher unwahrscheinlich, da Penn
sein Buch im Tower zu London (ein Gefängnis)\index{Orte:!Tower zu London} schon 1668 verfasste, seine Deutschlandreisen, auf
denen er davon gehört haben könnte, aber erst 1671 oder 1677 unternahm.

\medskip

In der Nachkriegszeit war von \textit{"`Ohne Kreuz, keine Krone"'} nicht viel zu hören.
Ein Abschnitt \textit{"`Kein Kreuz -- keine Krone"'} in einem Buch von Egon Larsen\index{Personen:!Larsen, Egon} aus
dem Jahre 1963 stellte sich als Jugendbuch heraus.\footnote{Egon Larsen: Kein
Kreuz -- keine Krone. William Penn, der Gründer
Pennsylvaniens, in: Ders.: Rebellen für die Freiheit, Berlin 1963, S. 13-36. Schon vor Gründung der DJV war in Deutschland ein Jugendbuch zu Penn erschienen: Ilse Lange: Aus dem Leben
des Quäkers William Penn, Berlin 1924. Inzwischen ist übrigens ein weiteres Jugendbuch zu Penn erschienen, nämlich von
Kurt Rose: Der Sohn des Admirals. William Penn -- Aufbruch in die Neue Welt,
Basel 1991.} Die Deutsche
Jahresversammlung\index{Deutsche Jahresversammlung} der Quäker in Deutschland\index{Orte:!Deutschland} und Österreich\index{Orte:!Österreich} hat sich weder in
ihren Publikationen noch in ihrer Zeitschrift \textit{"`Quäker"'} mit dem Werk beschäftigt,
die einzige Ausnahme ist Emilia Fogelklou-Norlind\index{Personen:!Fogelklou-Norlind, Emilia} (1878-1972), eine angesehene,
wenngleich auch autoritäre schwedische Quäkerin\index{Quaker!Schwedische}. Fogelklou-Norlind schätzte die
Schrift Penns nicht besonders und urteilt lapidar:.\textit{"`Für William Penn ist alles
Theologische etwas Sekundäres"'}\index{Theologie}\footnote{Emilia Fogelklou: Die Stellung von
James Naylor und William Penn zur Bibel,
in: Der Quäker, 23, 11, 1949, S. 173-175, hier S. 175.} Das zeigt, dass
Fogelklou-Norlind das Buch, über das sie spricht, wohl nicht einmal angesehen
hat. Auch um in Zukunft solche Fehlurteile zu vermeiden ist diese Studienausgabe
erarbeitet worden.




