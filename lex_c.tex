\section*{C}

\articlesize

\begin{description}

 \item[Cadbury, John] $\ast$12.8.1801 \dag11.5.1889, Quaker, Schokoladen-Fabrikant aus Birmingham (England).

 \item[Campaign for Nuclear Disarmament (CND)]

 \item[Ceresole, Pierre]

 \item[Chrisozentrische Quäker]
% Das ist ein Oberbegriff der nicht genau definiert ist und bezeichnet nur vage eine Tendenz, die Rolle und Bedeutung der Person Jesus zu betonen. Das das Beispiel oben von den \textit{Independent Quaker} zeigt das der Gruppe die Ausrichtung des Britain Yearly Meeting zu liberal war in Bezug auf die Christlichen Wurzeln. Es gibt aber keine "`Chrisozentrische Ordnung des Zusammenlebens"'. Unter diesen Begriff könnte man z.B. auch die Evangelical Friends sehen. Das Ohio Yearly Meeting wird sich theologisch aber überhaupt nicht in der Nähe von den Evangelicals sehen.

 \item[clearness committees]

 \item[Conservative Friends]
%  Das schon genannte Ohio Yearly Meeting wird z.B. in der Regel diesem Flügel zugeordnet. Conservative ist hier nicht im politischen Sinne gemeint, sondern in theologischen Sinne. Dieser Flügel orientiert sich also eng an den <i>Frühen Freunden</i>. Also "Old School Quaker" währe für den Deutschsprachigen Raum weniger missverständlich. Ich persönlich tendiere hier zu. Mein Meeting ( http://WasKannstDuSelbstSagen.de/ ) ist aber er als Liberal ein zu ordnen, und wir kratzen uns nicht gegenseitig die Augen aus!</li>

 \item[Cookworthy, William]

 \item[Cromwell, Oliver] $\ast$25.4.1599 in Huntingdon, \dag3. 9.1658 in Westminster, Gründer der englischen Republik, regierte als Lordprotektor England, Schottland und Irland während der kurzen republikanischen Periode der britischen Geschichte. Die Quaker genossen zum Teil seinen Schutz.

 \end{description}

\normalsize