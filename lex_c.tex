\section*{C}

\articlesize

\begin{description}


 \item[Campaign for Nuclear Disarmament (CND)] Zu Deutsch ``Kampagne für
nukleare Abrüstung''. Der Quaker $\to$\textit{Eric Baker} war Mitbegründer der
Organisation. Das Logo der Organisation wurde dann zum Friedenszeichen schleht
weg. Der erste der Märsche die von CND organisiert wurde, der
\textit{Aldermaston March} zu Ostern (4. bis 7. April 1958), wurde Vorbild für
so genannten ``Ostermärsche'' in verschiedenen westeuropäischen Ländern. Nach
Deutschland wurden die Ostermärsche unter anderem von dem Quakerehepaar Tempel
getragen.

 \item[Christ-centered theologically] $\to$\textit{Chrisozentrische Quäker}.

 \item[Chrisozentrische Quäker/Theologie] Das ist ein Oberbegriff der nicht genau
definiert ist und bezeichnet nur vage eine Tendenz, die Rolle und Bedeutung der
Person Jesus zu betonen. Unter $\to$\textit{Evangelical Friends} spielt der
Begriff keine rolle, da selbige der Rolle Jesus Christus ohne hin sehr betonen.
Er wird er in leberalen Quakertum benutz, um sich innerhalb des Quakerzweigs von
universalistischen Strömungen ($\to$\textit{Quaker Universalist Fellowship})zu
distanzieren oder auch von den $\to$\textit{Nontheist Friends}. Inhaltlich gibt
es überschneidungen zu dem $\to$\textit{New Foundation Fellowship}, da die
Frühen Freunde klar in der christlichen Tradition standen.

 \item[Clarks] Schuhfirma, gegründet 1825 von den Quakerbrüdern Cyrus and James
Clark.
 
 \item[Clearness (Committees)] Ein Vorgehen bei schwirigen Problemen, Fragen und
Kriesen. Es soll dabei gemeinsam in der Gruppe nach dem Willen Gottes gesucht
werden. Eine geplante Vermählung kann zum Beispiel ein solcher Anlass sein.

 \item[Clerk] Im Deutschem: "`Schreiber"'. In ermangelung eines Klerus (Pastor,
Prister oder Pfarrer) gibt es eine Person in der Versammlung, die Protokoll
fürt. Selbige Person hat keinerlei Auorität. Ist ist lediglich ihre Aufgabe,
beschlüsse und Feststellungen schriftlich fest zu haten. Hier zu muss sie in
einer dienenden Grundhaltung versuchen, zu erfassen, worin die Versammlung sich
einig ist. Dazu werden vom Schreiber der Versammlug Formierungsvorschläge
gemacht. Gibt es keine weiteren Wortmeldungen, signalisiert die Versammlung
durch schweigen ihrer Zustimmung und der Schreiber kann den Beschluss
niederschreiben. Bei Wortmeldungen obliegt es dem Schreiber (und seinem Gefühl)
Wem und in welcher Reihenfolge er zum Sprechen aufruft. Der Wunsch zum Sprechen,
signalisieren die Anwesenden dadurch, das sie aufstehen und warten.

\item[Concern] Im deutschen wird das Wort ``Anliegen'' verwendet. In anderen
Konfessionen würde man vielleicht ``Berufung'' sagen. Ein zentrales Konzept des
Quakertum, was davon aus geht, das Menschen - oder auch Gruppen -  von Gott für
bestimmte Aufgaben berufen werden. Wenn ein Mitglied zu einer Aufgabe berufen
fühlt, wendet er sich mit sein Anliegen an seine Gruppe
($\to$\textit{Monatsversammlung}), die das Anliegen prüft und versuch Gotes
Willen zu erkennen. Wenn das \textit{Anliegen} von der Gruppe als \textit{Willen
Gottes} erkannt werden konnte, wird das Anliegen ``unterstützt''. Das bedeutet,
das die Monatsversammlung das Anliegen zu ihrem eigenen Anliegen gemacht hat. Je
nach dem um was es sich handelt, wird ein Mitglid der Monatsversammlung (oder
eine Gruppe von Mitgliedern) ernannt, die das Anliegen in der nächst höheren
Orgnisationseinhei vertritt. So wird das Anliegen von einer Monatsversammlung
zum Beispiel in eine $\to$\textit{Virteljahresversammlung} getragen. Dort wird
das Anliegen erneut geprüft. Auf diese Weise wird das Anliegen einer immer
grösseren Gruppe bekanntgemacht (und geprüft). Durch die Unterstützung eines
Anliegens ("`acting under concern"') wurden viele bekannte Organisationen
gegründet. So zum Beispiel das \textit{American Friends Service Committee},
\textit{``Don't Make a Wave Committee''} (eine Vorläuferorgansation von
\textit{Greenpeace}), \textit{Oxfam} und \textit{Amnesty International}.
\endnote{``Don't Make a Wave Committee'' (2009, December 22). In Wikipedia, The
Free Encyclopedia. Retrieved 08:17, January 5, 2010, from \\
 \texttt{
http://en.wikipedia.org/w/index.php?title=\\
$\hookrightarrow$~Don\%27t\_Make\_a\_Wave\_Committee\&oldid=333258233}}


 \item[Conservative Friends] Conservative (Deutsch: \textit{Konsevativ}) ist
hier nicht im politischen Sinne gemeint, sondern in theologischen Sinne. Dieser
Flügel orientiert sich also eng an den Überzeugungen der $\to$\textit{Frühen
Freunden}. Also \textit{"`Old School Quaker"'} wenn man so will. Das
$\to$\textit{Ohio Yearly Meeting} wird zum Beispiel diesem
Conservative-Flügel zugeordnet. Sie haben eine $\to$\textit{"`unprogrammed"'}
Versammlung (also ohne Liturgie), sind $\to$\textit{"`non-pastoral"'} (haben
also keine Pastoren) und haben eine "`Christ-centered"' (also Christozentrische)
Theologie.

 \item[Convinced Friend] Gemeint sind Quaker die einmal bewust aus überzeugung
zum Quakertum übergetreten (Konvertiert) sind. Das Pendant ist quasie das
$\to$\textit{Birthright Membership}. Im gegensatz zum \textit{Birthright
Membership} wird \textit{Member by Convincement} von allen Versammlungen
anerkannt. Strittig ist aber wieder auf jemand "`Quaker"' ist, wenn er sich zum
Quakertum bekennt und danach lebt; oder erst, wenn er in einer
$\to$\textit{Monatsversammlung} als Mitglied aufgenommen wurde. Es gibt einige
Beispiele in der Quakergeschichte, wie zum Beispiel bei $\to$\textit{Benjamin
Lay}, wo keine formale Mitgliedschaft gestand (oder verlohren ging) und die
trozdem unbestritten als Quaker gelten.


 \end{description}

\normalsize