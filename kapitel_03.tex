
\chapter{3. Kapitel} \label{kap3}

\section{Zusammenfassung des 3. Kapitel}

\footnotesize
\begin{description}
\item[1. Abschnitt] Was das Kreuz Christi ist. -- Das Wort \textit{Kreuz} ist ein
sinnbildlicher \footnote{"`figürlicher"' durch "`sinnbildlicher"' ersetzt} 
Ausdruck der göttlichen Kraft, welche die Welt kreuzigt.
(Seite: \pageref{kap3_ab1})
\item[2. Abschnitt] So versteht es Paulus in seiner Epistel an die
Korinther. (Seite: \pageref{kap3_ab2})
\item[3. Abschnitt] Wo das Kreuz erscheint, und wo es getragen werden muß. Es
erscheint im Innern des Herzens; da wo die bösen Leidenschaften ihren Siz haben,
müssen sie auch gekreuzingt werden. (Seite: \pageref{kap3_ab3})
\item[4. Abschnitt] Dieser lehrt eine Jeden die Erfahrung, und Christus
behauptet es in den Worten: "`\textit{Aus dem Herzen kommen böse \footnote{
"'arge"` duch "'böse"` ersett} Gedanken, Mord,
Ehebruch1e}"' Dieses ist das Haus, wo der Sträkere den Starken binden mus.
(Seite: \pageref{kap3_ab4})
\item[5. Abschnitt] Wie das Kreuz ertragen werden müsse! Es geschiehet auf eine
geistliche Weise, wenn man sich selbst und die Vergnügungen der Sünde überwindet
\footnote{''verleugnet`` ersetzt duch ''überwindet``}, Gott zu gefallen und 
seinem Willen zu gehorchen bestrebt ist, wie
derselbe durch das Licht, das er Verleiht, der Seele geoffendaret wird.
\item[6. Abschnitt] Dieses zeigt, wie schwer, aber auch, wie nothwendig es ist.
das Kreuz zu trauen. (Seite: \pageref{kap3_ab5})
\end{description}
\normalsize

Da nun in den Zeiten des ersten Christenthums das Kreuz Christi täglich zu
tragen das Mittel zur Herrlichkeit zu gelangen war, (und auch noch jetzt ist)
so ist es nothwendig für dich, -- o Christenheit -- damit der Inhalt der folgenden
Kapitel, der sich gänzlich aus diesen wichtigen Gegenstand beziehet, (mit desto
klarerer Überzeugung und größerm Nutzen auf dein Gewissen wirken möge,) 
folgende Punkte aufs ernstlichste zu betrachten:
\begin{description}
\item[Erstens:] ...was das Kreuz Christi sei?
\item[Zweitens:] ...wo das Kreuz Christi müsse aufgenommen werden?
\item[Drittens:] ...wie, oder auf welche Art dasselbe getragen werden müsse? und...
\item[Viertens:] ...worin die großen Wirkungen des Kreuzes bestehen?
\end{description}

...Also \textbf{zuerst: Was ist das Kreuz Christi?}

\section{1. Abschnitt} \label{kap3_ab1} 
\textbf{Der Ausdruck: \textit{Kreuz Christi}, ist eine bildliche \footnote{
''figürliche`` durch ''bildlichr`` ersetzt} Redensart, die von dem äußern
\textit{hölzernen Kreuze} \marginpar{hölzernen Kreuze} entlehnt ist, an welchem 
\index{hölzernen Kreuze, Das} Christus sich dem Willen Gottes
unterwarf, der es zuließ, daß er durch die Hände böser Menschen den Tod erlitt.}
Die geheime und geistliche Bedeutung des Kreuzes bezeichnet aber jene göttliche
Gnade und Kraft, welche den fleischlichen Willen der Menschen kreuzigt, indem
sie mit den verderbten Neigungen derselben im Widerspruch steht, und dem
unordentlichen und materiellen \footnote{''fleischlichen`` duch ''materiellen`` 
ersetzt} Verlangen ihrer Gemüther. beständig
entgegenwirkt. Diese göttliche Kraft und Gnade kann also auch mit Recht das
Mittel genannt werden, wodurch der Mensch der Welt gekreuzigt und dem Willen
Gottes unterwürfig gemacht wird. Denn nichts anders ist vermögend, die
sündlichen Neigungen in uns zu tödten, und es uns leicht zu machen, in den
Dingen, die von Natur unserm eigenen Willen zuwider sind, uns. dem göttlichen
Willen zu unterwerfen.

\section{2. Abschnitt} \label{kap3_ab2} 

Das Wort vom Kreuze, (oder die Verkündigung desselben,) ward daher auch in den
ersten Zeiten von Paulo, jenem berühmten und in geistlichen Dingen wohl
bewanderten Apostel, eine Kraft Gottes genannt; wiewohl es damals denen, die
verloren gingen, Thorheit war, so wie es Allen, die jetzt verloren gehen, noch
Thorheit ist. Das heißt: für die müden und beladenen Seelen, die eines Erlösers
bedurften, denen die Sünde lästig und verhaßt geworden war, für diese war die
Verkündigung des Kreuzes, durch welches die sinnlichen Neigungen in ihnen
getödtet werden konnten, eine Kraft Gottes oder eine Verkündigung der göttlichen
Kraft, durch welche sie zu Jüngern Christi und zu Kindern Gottes gemacht wurden.
Auch hatte diese göttliche Kraft des Kreuzes einen so mächtigen Einfluß auf ihre
Gemüther, daß kein stolzer oder leichtsinniger Spötter im Stande war, sie von
der Liebe zu derselben abwendig zumachen. Aber Solchen, die aus dem breiten Wege
wandelten, die sich ganz zügellos ihren Lüsten hingeben, und ihre Zeit und
Sorgfalt dem Genusse ihrer Vergnügungen, der Befriedigung ihrer verderbten
Begierden widmeten, denen jedes Joch, jeder Zügel für ihre Leidenschaften
unerträglich war, Solchen war das Wort vom Kreuze eine Thorheit, und ist es
allen Solchen auch noch; ja, man kann von nur zu Vielen unserer jetzigen
Zeitgenossen noch hinzufügen, daß ihnen die Lehre vom Kreuze lächerlich ist, da
sie, ihrer Meinung nach, nur von haldklugen Leuten angenommen wird, die von
beschränktem Verstande und sonderbarer Gemüthsart, mit der Milzsucht behaftet,
oder vom Trübsinne niedergedrückt sind. Denn alles dieses, und noch viel mehr,
haben selbst schon Bekenner und vorgebliche Verehrer der Religion Jesu von den
Wirkungen seines heiligen Kreuzes, oder vielmehr von denen, die dasselbe
wirklich tragen, behaupten wollen.

\section{3. Abschnitt} \label{kap3_ab3} 

Du wirst nun aber fragen: wo erscheint denn dieses Kreuz! oder wo offenbart es
sich? und wo muß, man es aufnehmen?

Ich antworte: im Innern des Herzens; in der Seele. Wo die Sünde ist, da zeigt
sich auch das Kreuz. Denn alles Böse kommt aus dem Innern des Menschen, wie
Christus selbst gelehret hat. "`Von innen, aus den; Herzen des Menschen,"' \textit{sagt
er, "`gehen hervor: böse Gedanken, Ehebruch, Mord, Dieberei, Geiz, Schalkheit,
List, Unzucht, ein neidisches Auge, Gotteslästerung, Hoffahrt, Thorheit. Alle
diese bösen Dinge gehen von innen hervor, und verunreinigen den
Menschen.}"'\footnote{Markus 7,21-23} Das Herz des Menschen ist also der Sitz der
Sünde, und wo nun der Mensch verunreinigt ist, da muß er auch gereinigt und
geheiligt werden; wo die Sünde lebt, da muß sie gekreuzigt werden; da muß. sie
sterben. Gewohnheit im Bösen hat es dem Menschen zur Natur gemacht, Böses zu
thun; und wie die Seele den Körper regiert, so beherrscht die Verderbte Natur
den ganzen Menschen; Alles kommt aber dennoch aus seinem Innern.

\section{4. Abschnitt} \label{kap3_ab4} 

Dieser Behauptung müssen alle Söhne und Töchter Adams aus eigener Erfahrung
beistimmen. Denn die Versuchungen des Feindes sind beständig auf das Gemüth des
Menschen, aufs Innere seiner Seele gerichtet. Finden sie hier keinen Eingang,
sondern werden verleugnet und abgewiesen, so sündigt die Seele nicht. Läßt man
sich aber mit der Versuchung ein, so empfängt die Lust sogleich , das heißt, es
entstehen unerlaubte Begierden. "`\textit{Wenn aber die Lust empfangen hat, so erzeugt
sie die Sünde; und wenn die Sünde vollendet ist, d. h. wenn sie ausgeübt wird,
so gebiert sie den Tod ,}"'\footnote{Jakobus 1,15} oder versetzt das Gemüth in einen
Zustand des geistlichen Todes. Hier haben wir die Ursache und die Wirkung, die
wahre Geschlechtskunde, den Ursprung und das Ende der Sünde.

\medskip

In allem diesem ist das Herz des bösen Menschen die Kunstkammer des Feindes;
seine Werkstatt und sein Wohnsitz, wo er seine Kunst treibt, und seine Macht
ausübt. Daher wird die Erlösung der Seele durch Christum sehr passend "`\textit{die
Zerstörung der, Werke des Teufels, und die Herrschaft der Gnade durch die
Gerechtigkeit zum ewigen Leben,}"'\footnote{Johannes 3,8; Römer 5,21} genannt. -- Als
die Juden die Wunder Christi, die er durch Austreibung des Teufels bewies,
dadurch zu verrufen suchten, daß sie dieselben aus eine gotteslästerliche Weise
der Macht des Beelzebub zuschrieben, sagte Christus zu ihnen: "`\textit{es kann Niemand
einem Starken in sein Haus fallen, und seinen Hausrath rauben, es sey denn, daß
er zuvor den Starken binde.}"'\footnote{Markus 3,27} Diese Worte, indem sie die
große Verschiedenheit zwischen der Macht des Beelzebub, und der Kraft, durch
welche Christus ihn austrieb, deutlich zeigen, geben uns zugleich zu erkennen,
daß die Herzen der Gottlosen die Werkstätte des Feindes der menschlichen
Glückseligkeit sind, und daß seine Hausgeräthe, nämlich die bösen Werke, die er
in dem Menschen wirket, nicht eher zerstöret werden können, bis er selbst, der
sie hervorgebracht hat, gebunden werde. Aus allem diesem ist nun leicht
einzusehen, wo das Kreuz müsse aufgenommen werden, durch dessen Kraft allein der
Starke gebunden, seiner Güter beraubt, und mit seinen Versuchungen
zurückgewiesen werden kann, nämlich: im Innern der Seele.

\section{5. Abschnitt} \label{kap3_ab5} 

Nun schreite ich zur Beantwortung der Frage: Wie, oder aus welche Art das Kreuz
täglich getragen werden müsse?

\medskip

Die Art und Weise, das Kreuz zu tragen, ist, wie das Kreuz selbst, geistlich.
Sie bestehet in einer innern Unterwerfung der Seele unter den Willen Gottes , so
wie dieser ihr durch das Licht Christi in ihrem Innern geoffenbaret wird;
wenngleich derselbe ihren eigenen Neigungen zuwider ist. Wenn, zum Beispiele,
etwas Böses sich dem Menschen zeigt, und ihn anlockt, so giebt Dasjenige,
welches ihm das Böse entdeckt, ihm auch zu verstehen, daß er in dasselbe nicht
einwilligen solle. Folgt er nun dieser Warnung, und nimmt aus diese Weise das
Kreuz gegen seine verderbten Neigungen auf, so giebt dasselbe ihm auch Kraft,
der Versuchung zu entgehen. Diejenigen hingegen, welche die Versuchungen ansehen
und betrachten, und sich dabei aushalten, fallen endlich in dieselben hinein,
und werden von ihnen überwunden, wovon die Folge alsdann Schuld, Gewissensangst
und Gericht ist

\medskip

So wie nun das Kreuz Christi nichts anders als derjenige Geist und diejenige
Kraft im Menschen -- jedoch nicht vom Menschen selbst, sondern allein aus Gott --
ist, wodurch den fleischlichen Neigungen und Begierden in ihm beständig
entgegengewirkt, gleichsam ein Kreuz- oder Querstrich gemacht, und mit Zucht und
Bestrafung begegnet wird; so bestehet auch die wahre Aufnahme des Kreuzes in
nichts anders, als in einer gänzlichen Hingebung der Seele zum Gehorsame gegen
die Offenbarungen und Forderungen dieses göttlichen Einflusses; wobei der Mensch
weder seine weltlichen Vergnügungen, noch seine fleischliche Gemächlichkeit,
noch seine irrdischen Vortheile zu Rathe ziehen darf, -- denn sonst ist er in
einem Augenblicke gefangen, -- sondern beständig gegen jede Erscheinung des
Bösen. auf der Wache stehen, und durch den Gehorsam des Glaubens, nämlich der
wahren Liebe und des festen Vertrauens zu Gott, jene böse Selbstsucht freudig
dem Kreuzestode übergeben muß, die, weil sie die Hitze der Belagerung des
Seelenfeindes nicht aushalten und in der Stunde der Versuchung die Geduld nicht
bewahren kann, vermöge ihrer nahen Verwandtschaft mit dem Versucher, wie ein
innerer Judas die Seele verrathen und in seine Hände überliefern würde.



\section{6. Abschnitt} \label{kap3_ab6} 

O dieses zeigt einem Jeden durch eigene Erfahrung, wie schwer es ist, ein wahrer
Jünger Jesu zu seyn! Der Weg ist in der That schmal und die Pforte sehr enge, an
welcher nicht ein Wort, nein, auch kein Gedanke der, Wache entschlüpfen, oder
dem Gerichte entrinnen darf;\footnote{Matthäus 24,42; Matthäus 25,13; Matthäus 26,38-42;
Philipper 2,12; 1. Korinther 15,50} solche Umsicht, solche Vorsichtigkeit, solche Geduld,
solche Standhaftigkeit, so viel heilige Furcht und Zittern ist dabei nothwendig.
Dieses giebt uns eine leichte Auslegung jener harten Rede: "`\textit{Fleisch und Blut
können das Reich Gottes nicht ererben!}"' da Diejenigen, die in fleischlichen
Lüsten und Begierden gefangen liegen, das Kreuz nicht erdulden können; und
Solche, die das Kreuz nicht tragen, niemals eine Krone erlangen werden. Wollen
wir mit herrschen, so müssen wir auch erst mit leiden!

