

\chapter{Kapitel 3} \label{kap3}

\section{Zusammenfassung des 3. Kapitel}


\begin{description}
\item[1. Abschnitt] Was das Kreuz Christi ist. -- Das Wort \textit{Kreuz} ist
ein
sinnbildlicher
\footnote{\texttt{'figürlicher' durch 'sinnbildlicher' ersetzt.}}
Ausdruck der göttlichen Kraft, welche die Welt kreuzigt.
\dotfill \textit{Seite~\pageref{kap3_ab1}}\\
\item[2. Abschnitt] So versteht es Paulus in seiner Epistel an die
Korinther.
\dotfill \textit{Seite~\pageref{kap3_ab2}}\\
\item[3. Abschnitt] Wo das Kreuz erscheint, und wo es getragen werden muss. Es
erscheint im Innern des Herzens, da wo die bösen Leidenschaften ihren Sitz
haben,
müssen sie auch gekreuzigt werden.
\dotfill \textit{Seite~\pageref{kap3_ab3}}\\
\item[4. Abschnitt] Dieses lehrt einen jeden
die Erfahrung, und Christus
behauptet es mit den Worten: "`\textit{Aus dem Herzen kommen
böse\footnote{\texttt{'arge' durch 'böse' ersetzt.}} Gedanken, Mord,
Ehebruch}"' Dieses ist das Haus, wo der Stärkere den Starken binden muss.
\dotfill \textit{Seite~\pageref{kap3_ab4}}\\
\item[5. Abschnitt] Wie das Kreuz ertragen werden müsse! Es geschieht auf eine
geistliche Weise, wenn man sich selbst und die Vergnügungen der Sünde
überwindet\footnote{\texttt{'verleugnet' ersetzt durch 'überwindet'.}}, Gott zu
gefallen und
seinem Willen zu gehorchen bestrebt ist, wie
derselbe durch das Licht, das er verleiht, der Seele geoffenbart wird.
\dotfill \textit{Seite~\pageref{kap3_ab5}}\\
\item[6. Abschnitt] Dieses zeigt, wie schwer, aber auch, wie notwendig es ist
das Kreuz zu tragen.
\dotfill \textit{Seite~\pageref{kap3_ab6}}\\
\end{description}

\newpage

Da nun in den Zeiten des ersten Christentums das Kreuz Christi täglich zu
tragen das Mittel zur Herrlichkeit zu gelangen war, (und auch noch jetzt ist)
so ist es notwendig für dich, -- o Christenheit -- damit der Inhalt der
folgenden
Kapitel, der sich gänzlich auf diesen wichtigen Gegenstand bezieht, (mit desto
klarerer Überzeugung und größerem Nutzen auf dein Gewissen wirken möge,)
folgende Punkte aufs ernstlichste zu betrachten:
\begin{description}
\item[Erstens:] ...was das Kreuz Christi sei?
\item[Zweitens:] ...wo das Kreuz Christi müsse aufgenommen werden?
\item[Drittens:] ...wie, oder auf welche Art dasselbe getragen werden müsse?
Und...
\item[Viertens:] ...worin die großen Wirkungen des Kreuzes bestehen?
\end{description}

...Also \textbf{zuerst: Was ist das Kreuz Christi?}


\section{1. Abschnitt} \label{kap3_ab1}

\label{ref:03_01_das_kreuz}
\textbf{Der Ausdruck: \textit{Kreuz Christi}, ist eine
bildliche\footnote{\texttt{'figürliche' durch 'bildliche' ersetzt.}} Redensart,
die von dem äußeren
\textit{hölzerne Kreuz} entlehnt ist, an welchem
\index{Kreuz!hölzernen Kreuze, Das} Christus sich dem Willen Gottes
unterwarf, der es zuließ, dass er durch die Hände böser Menschen den Tod
erlitt.}
Die geheime und geistliche Bedeutung des Kreuzes bezeichnet aber jene göttliche
Gnade und Kraft, welche den fleischlichen Willen der Menschen kreuzigt, indem
sie mit den verderbten Neigungen derselben im Widerspruch steht, und dem
unordentlichen und materiellen\footnote{\texttt{'fleischlichen' durch
'materiellen' ersetzt.}} Verlangen ihrer Gemüter beständig
entgegenwirkt. Diese göttliche Kraft und Gnade kann also auch mit Recht das
Mittel genannt werden, wodurch der Mensch der Welt gekreuzigt und dem Willen
Gottes unterwürfig gemacht wird. Denn nichts anderes ist vermögend, die
sündlichen Neigungen in uns zu töten, und es uns leicht zu machen, in den
Dingen, die von Natur unserem eigenen Willen zuwider sind, uns dem göttlichen
Willen zu unterwerfen.

\section{2. Abschnitt} \label{kap3_ab2}

Das Wort vom Kreuze, (oder die Verkündigung desselben,) ward daher auch in den
ersten Zeiten von Paulus, jenem berühmten und in geistlichen Dingen wohl
bewanderten Apostel, eine Kraft Gottes genannt, wiewohl es damals denen, die
verloren gingen, Torheit war, so wie es allen, die jetzt verloren gehen, noch
Torheit ist. Das heißt: Für die müden und beladenen Seelen, die eines Erlösers
bedurften, denen die Sünde lästig und verhasst geworden war, für diese war die
Verkündigung des Kreuzes, durch welches die sinnlichen Neigungen in ihnen
getötet werden konnten, eine Kraft Gottes oder eine Verkündigung der göttlichen
Kraft, durch welche sie zu Jüngern Christi und zu Kindern Gottes gemacht wurden.
Auch hatte diese göttliche Kraft des Kreuzes einen so mächtigen Einfluss auf
ihre
Gemüter, dass kein stolzer oder leichtsinniger Spötter im Stande war, sie von
der Liebe zu derselben abwendig zumachen. Aber solchen, die auf dem breiten Wege
wandelten, die sich ganz zügellos ihren Lüsten hingeben, und ihre Zeit und
Sorgfalt dem Genusse ihrer Vergnügungen, der Befriedigung ihrer verderbten
Begierden widmeten, denen jedes Joch, \label{ref:03_02_leidenschaft}
\textbf{jeder Zügel für ihre Leidenschaften
unerträglich war, solchen war das Wort vom Kreuz eine Torheit, und ist es
allen solchen auch noch;} ja, man kann von nur zu vielen unserer jetzigen
Zeitgenossen noch hinzufügen, dass ihnen die Lehre vom Kreuz lächerlich ist, da
sie, ihrer Meinung nach, nur von halbklugen Leuten angenommen wird, die von
beschränktem Verstande und sonderbarer Gemütsart, mit der
Milzsucht\footnote{\texttt{Im 17. Jahrhundert verstand man unter Milzsucht die
Hypochondrie, also eine psychische Störung, bei der der Patient glaubt, an den
verschiedensten Krankheiten zu leiden. Einst sah man die Milz als Sitz dieser
Krankheit an.}} behaftet
oder vom Trübsinn niedergedrückt sind. Denn alles dieses, und noch viel mehr,
haben selbst schon Bekenner und vorgebliche Verehrer der Religion Jesu von den
Wirkungen seines heiligen Kreuzes, oder vielmehr von denen, die dasselbe
wirklich tragen, behaupten wollen.

\section{3. Abschnitt} \label{kap3_ab3}

Du wirst nun aber fragen: Wo erscheint denn dieses Kreuz? Oder wo offenbart es
sich? Und wo muss man es aufnehmen?

\medskip

\label{ref:03_03_ort_des_kreuzes}
Ich antworte: Im Inneren des Herzens, in der Seele. \textbf{Wo die Sünde ist,
da zeigt
sich auch das Kreuz.}Denn alles Böse kommt aus dem Innern des Menschen, wie
Christus selbst gelehrt hat.
\textit{"`Von innen, aus dem Herzen des Menschen,"'} sagt er,
\textit{"`gehen hervor: böse Gedanken, Ehebruch, Mord, Diebstahl, Geiz,
Schalkheit,
List, Unzucht, ein neidisches Auge, Gotteslästerung,
Hochmütigkeit\footnote{\texttt{'Hoffart' ersetzt durch 'Hochmütigkeit'.}},
Torheit. Alle
diese bösen Dinge gehen von innen hervor und verunreinigen den
Menschen."'}\footnote{Markus 7,21-23.}
\index{Bibelstellen:!Markus 07@Markus 7)}
\textbf{Das Herz des Menschen ist also der Sitz der Sünde},
\index{Sünde!Sitz der} und wo nun der Mensch
verunreinigt ist, da muss er auch gereinigt und
geheiligt werden; wo die Sünde lebt, da muss sie gekreuzigt werden, da muss sie
sterben. Gewohnheit im Bösen hat es dem Menschen zur Natur gemacht, Böses zu
tun, und wie die Seele den Körper regiert, so beherrscht die verderbte Natur
den ganzen Menschen. Alles kommt aber dennoch aus seinem Innern.

\section{4. Abschnitt} \label{kap3_ab4}

Dieser Behauptung müssen alle Söhne und Töchter Adams aus eigener Erfahrung
beistimmen. Denn die Versuchungen des Feindes sind beständig auf das Gemüt des
Menschen, aufs Innere seiner Seele gerichtet. Finden sie hier keinen Eingang,
sondern werden verleugnet und abgewiesen, so sündigt die Seele nicht. Lässt man
sich aber mit der Versuchung ein, so empfängt die Lust sogleich, das heißt, es
entstehen unerlaubte Begierden.
\textit{"`Wenn aber die Lust empfangen wurde, so erzeugt
sie die Sünde, und wenn die Sünde vollendet ist, d. h. wenn sie ausgeübt wird,
so gebärt sie den Tod,"'}\footnote{Jakobus 1,15.}
\index{Bibelstellen:!Jakobus 01@Jakobus 1)}
oder versetzt das Gemüt in einen
Zustand des geistlichen Todes\index{Tod!geistlicher}. Hier haben wir die
Ursache und die Wirkung, die
wahre Geschlechtskunde, den Ursprung und das Ende der Sünde.

\medskip

\label{ref:03_04_reich_des_boesen}
\textbf{In allem diesem ist das Herz des bösen Menschen die Kunstkammer des
Feindes, seine Werkstatt und sein Wohnsitz, wo er seine Kunst treibt und seine
Macht ausübt.} Daher wird die Erlösung der Seele durch Christum sehr passend
\textit{"`die Zerstörung der Werke des Teufels und die Herrschaft der Gnade
durch die Gerechtigkeit zum ewigen Leben,"'}\footnote{Johannes 3,8) Römer 5,21.}
\index{Bibelstellen:!Johannes 03@Johannes 3)}
\index{Bibelstellen:!Römer 05@Römer 5)}
genannt. -- Als
die Juden die Wunder Christi, die er durch Austreibung des Teufels bewies,
dadurch zu verrufen suchten, dass sie dieselben aus eine gotteslästerliche Weise
der Macht des Beelzebub zuschrieben, sagte Christus zu ihnen:
\textit{"`Es kann niemand einem Starken in sein Haus fallen, und seinen
Hausrat rauben, es sei denn, dass er zuvor den Starken binde."'}\footnote{Markus
3,27.}
\index{Bibelstellen:!Markus 03@Markus 3)}
Diese Worte, indem sie die
große Verschiedenheit zwischen der Macht des Beelzebub und der Kraft, durch
welche Christus ihn austrieb, deutlich zeigen, geben uns zugleich zu erkennen,
dass die Herzen der Gottlosen die Werkstätte des Feindes der menschlichen
Glückseligkeit sind, und dass seine Hausgeräte, nämlich die bösen Werke, die er
in dem Menschen wirkt, nicht eher zerstört werden können, bis er selbst, der
sie hervorgebracht hat, gebunden werde. Aus allem diesem ist nun leicht
einzusehen, wo das Kreuz müsse aufgenommen werden, durch dessen Kraft allein der
Starke gebunden, seiner Güter beraubt, und mit seinen Versuchungen
zurückgewiesen werden kann, nämlich im Inneren der Seele.

\section{5. Abschnitt} \label{kap3_ab5}

Nun schreite ich zur Beantwortung der Frage: Wie, oder aus welche Art das Kreuz
täglich getragen werden müsse?

\medskip

\label{ref:03_05_kreuz_auf_sich_nehmen}
\textbf{Die Art und Weise, das Kreuz zu tragen, ist, wie das Kreuz selbst,
geistlich.
Sie besteht in einer inneren Unterwerfung der Seele unter den Willen Gottes,} so
wie dieser ihr durch das Licht Christi in ihrem Inneren geoffenbart wird;
wenngleich derselbe ihren eigenen Neigungen zuwider ist. Wenn, zum Beispiel,
etwas Böses sich dem Menschen zeigt, und ihn anlockt, so gibt dasjenige,
welches ihm das Böse entdeckt, ihm auch zu verstehen, dass er in dasselbe nicht
einwilligen solle. Folgt er nun dieser Warnung, und nimmt auf diese Weise das
Kreuz gegen seine verderbten Neigungen auf, so gibt dasselbe ihm auch Kraft,
der Versuchung zu entgehen. Diejenigen hingegen, welche die Versuchungen ansehen
und betrachten, und sich dabei aushalten, fallen endlich in dieselben hinein,
und werden von ihnen überwunden, wovon die Folge alsdann Schuld, Gewissensangst
und Gericht ist.

\medskip

So wie nun das Kreuz Christi nichts anders als derjenige Geist und diejenige
Kraft im Menschen -- jedoch nicht vom Menschen selbst, sondern allein aus Gott
--
ist, wodurch den fleischlichen Neigungen und Begierden in ihm beständig
entgegengewirkt, gleichsam ein Kreuz- oder Querstrich gemacht, und mit Zucht und
Bestrafung begegnet wird; so besteht auch die wahre Aufnahme des Kreuzes in
nichts anderem, als in einer gänzlichen Hingebung der Seele zum Gehorsam gegen
die Offenbarungen und Forderungen dieses göttlichen Einflusses, wobei der Mensch
weder seine weltlichen Vergnügungen, noch seine
körperlichen\footnote{\texttt{'fleischliche' durch 'körperlichen' ersetzt.}}
Gemächlichkeit,
noch seine irdischen Vorteile zu Rate ziehen darf, -- denn sonst ist er in
einem Augenblicke gefangen, -- sondern beständig gegen jede Erscheinung des
Bösen auf der Wache stehen, und durch den Gehorsam des Glaubens, nämlich der
wahren Liebe und des festen Vertrauens zu Gott, jene böse Selbstsucht freudig
dem Kreuzestod übergeben muss, die, weil sie die Hitze der Belagerung des
Seelenfeindes nicht aushalten und in der Stunde der Versuchung die Geduld nicht
bewahren kann, vermöge ihrer nahen Verwandtschaft mit dem Versucher, wie ein
\textbf{innerer Judas}
\footnote{\texttt{Der 'innerer Judas' ist quasi der Gegenspieler von 'inneren
Christus', also der inneren Stimme die einen über seine Missetaten Anklagt.}}
\index{Judas!Innerer}
%hier auf den Index verweisen, wo Judas erklärt wird!
die Seele verraten und in seine Hände überliefern würde.



\section{6. Abschnitt} \label{kap3_ab6}

O dieses zeigt einem jeden durch eigene Erfahrung, wie schwer es ist, ein wahrer
Jünger Jesu zu sein! Der Weg ist in der Tat schmal und die Pforte sehr enge, an
welcher nicht ein Wort -- nein! -- auch kein Gedanke der Wache entschlüpfen,
oder
dem Gerichte entrinnen darf;\footnote{Matthäus 24,42) Matthäus 25,13.) Matthäus
26,38-42)
Philipper 2,12) 1. Korinther 15,50}
\index{Bibelstellen:!Matthäus 24)}
\index{Bibelstellen:!Matthäus 25)}
\index{Bibelstellen:!Matthäus 26)}
\index{Bibelstellen:!Philipper 02@Philipper 2)}
\index{Bibelstellen:!Korinther 1 15@1. Korinther 15)}
solche Umsicht, solche Vorsichtigkeit, solche Geduld,
solche Standhaftigkeit, so viel heilige Furcht und Zittern ist dabei notwendig.
Dieses gibt uns eine leichte Auslegung jener harten Rede:
\textit{"`Fleisch und Blut können das Reich Gottes nicht erben!"'}
da diejenigen, die in fleischlichen
Lüsten und Begierden gefangen liegen, das Kreuz nicht erdulden können, und
solche, die das Kreuz nicht tragen, niemals eine Krone erlangen werden. Wollen
wir mit herrschen, so müssen wir auch erst mit leiden!




