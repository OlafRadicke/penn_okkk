
\chapter{2. Kapitel}  \label{kap2}


\section{Zusammenfassung des 2. Kapitel}
\footnotesize
\begin{description}
\item[1. Abschnitt] Aus dem, was bisher gesagt ist, kann die Christenheit ihren
Verfall und ihre große Verderbtheit erkennen. -- Ihr Zustand ist wegen ihrer
Ansprüche aus Christenthum nur desto schlimmer. (Seite \pageref{kap2_ab1})
\item[2. Abschnitt] Bei Gott ist aber Barmherzigkeit und Versöhnung durch das
Blut Jesu, wenn sie ihre Sünden bereuet und ihr Leben ändert. (Seite \pageref{kap2_ab2})
\item[3. Abschnitt] Christus ist das Licht der Welt, welchen die Finsternis,
nämlich das Böse in der Welt, bestraft; er wird im Innern der Seele erkannt. (Seite \pageref{kap2_ab3})
\item[4. Abschnitt] Die Christenheit ist. wie die Herberge dor Zeiten, in
welcher kein Raum für ihn war, voll anderer Gäste. -- Sie wird angewiesen, an
Christum zu glauben, ihn aufzunehmen und sich an ihn zu wenden. (Seite \pageref{kap2_ab4})
\item[5. Abschnitt] Von der Eigenschaft des wahren Glaubens; er giebt Kraft,
jede Erscheinung des Bösen zu überwinden. Dieses leiter zur Betrachtung des
Kreuzes Christi, woran es bisher so sehe gemangelt hat. (Seite \pageref{kap2_ab5})
\item[6. Abschnitt] Vom apostolischen Amte; Zweck und gesegnete Wirkungen
desselben. -- Charakter der apostolischen Zeiten. (Seite \pageref{kap2_ab6})
\item[7. Abschnitt] Vortrefflichkeit des Kreuzes Christi, und sein Triumph über
die heidnische Welt; ein Spiegel für die Christen, worin sie sehen können, was
sie nicht sind und was sie sein sollten. (Seite \pageref{kap2_ab7})
\item[8. Abschnitt] Die Ursachen ihres Verfalles. (Seite \pageref{kap2_ab8})
\item[9. Abschnitt] Die traurige Wirkungen , die daraus erfolgt sind. (Seite \pageref{kap2_ab9})
\item[10. Abschnitt] Aus der Erwägung der Ursache dieser Verfalles, kann das
Mittel zu ihrer Wiederherstellung leicht erkannt werden; oder, da die tägliche
Vernachlässigung des getreuen Aufnehmens des Kreuzes die Ursache desselben ist,
so muß auch das- tägliche getreue Tragen des Kreuzer das Mittel zu ihrer
Wiederherstellung sein. (Seite \pageref{kap2_ab10})
\end{description}
\normalsize

\section{1. Abschnitt}  \label{kap2_ab1}

Aus Allem, was dir, o Christenheit, bisher gesagt ist, und vermöge jener bessern
Hilfe, -- wenn du dich derselben nur bedienen wolltest! -- nämlich dem Lichts,
das Gott in dir angezündet hat, und welches noch nicht ganz erloschen ist,
kannst du nun erkennen, wie groß und entsetzlich dein Fall ist. Dann wirst du
auch einsehen, wie du ungeachtet deines offenbaren Verfalles, nichts
destoweniger mit deinen: leeren Bekenntnisse des Christentum deines verderbten
Egoismus \footnote{``Selbstliebe'' duch ``Egoismus'' ersetzt} geschmeichelt, 
und auf eine schreckliche Art dich selbst mit
falschen Hoffnungen der Seligkeit getäuscht hast. Das Erstere macht deine
Krankheit gefährlich, durch das Letztere wird sie aber fast unheilbar.

\section{2. Abschnitt}  \label{kap2_ab2}

Jedoch, da beidem Gott des Mitleids Barmherzigkeit ist, damit man ihn
\marginpar{Gott will erretten und nicht verdammen.}
fürchte, und da er keinen Gefallen an dem ewigen Tode armer Sünder hat, wenn
sie auch Abtrünnige sind; sondern vielmehr will, daß Alle zur Erkenntnis der
Wahrheit und zum Gehorsame gegen dieselbe gebracht, und errettet werden
sollen: \footnote{Ezechiel (Hesekiel) 18,20+23+24.}
\index{Bibelstellen:!Ezechiel (Hesekiel) 18)} so hat er seinen Sohn zur Versöhnung
dargestellt, und zum Heilande gegeben, um die Sünden der ganzen Welt
hinwegzunehmen, damit Diejenigen, die an ihn glauben und ihm folgen in der
Vergebung ihrer Sünden und ewigen Tilgung ihrer Uebertretungen die
Gerechtigkeit Gottes erkennen mögen.\footnote{Matthäus 1,21. Lukas 1,77. Hebräer
9,24-28. Johannes 2, 1+2.}
\index{Bibelstellen:!Matthäus 1)} \index{Bibelstellen:!Lukas 1)}
\index{Bibelstellen:!Johannes 2)}
Siehe! hier ist das unfehlbare Heilmittel für dein
Übel, das Gott selbst verordnet hat; in der That eine köstliche Arznei, die
niemals fehlschlägt; das große Universalmittel, dem alle Krankheiten weichen
müssen.

\section{3. Abschnitt}  \label{kap2_ab3}

Du wirst aber vielleicht fragen: Was ist denn Christus? und wo ist er zu finden?
Wie kann man diese Arznei für die Seele erhalten? und wie muß man sie anwenden,
um ihre mächtige Heilkraft zu erfahren? Ich sage dir daher \textbf{erstens:}
Christus ist das große geistige Licht der Welt, welches alle Menschen die in
diese Welt kommen erleuchtet, ihnen ihre Werke der Finsterniß und Gottlosigkeit
zeigtund offenbar macht und sie über die Ausübung derselben bestraft.\footnote{Johannes 1,9. und 8,12. Epheser 5,13.}
\index{Bibelstellen:!Johannes 1)} \index{Bibelstellen:!Johannes 8)}
\index{Bibelstellen:!Epheser 5)}
\textbf{Zweitens:} Er ist nicht fern von dir, wie der
Apostel Paulus den Atheniensern von Gott erklärte;
\footnote{Apostelgeschichte 17,27.} \index{Bibelstellen:!Apostelgeschichte 17)}
und Christus selbst sagt: "`\textit{Siehe! ich stehe vor der Thür und klopfe an;
wenn Jemand meine Stimme höret, und die Thür aufthut, zu dem werde ich eingehen
und Abendmahl mit ihm halten, und er mit mir.}"'\footnote{Offenbarung 3,20.}
Kann \index{Bibelstellen:!Offenbarung 3)}
nun diese Thür, wovon Christus hier redet, wohl eine andere, als die des
unsterblichen Herzens sein?

\section{4. Abschnitt}  \label{kap2_ab4}

Aber dein Herz war bisher, wie die Herberge vor Zeiten, so sehr von andern
Gästen angefüllt. Deine Neigungen waren so eifrig auf andere Gegenstände deiner
Liebe gerichtet, daß für deinen Heiland kein Raum in dir übrig war. Darum ist
dass Heil noch nicht in deinem Hause eingekehrt, wiewohl es bis zu deiner Tür
gekommen ist, und sich dir oft angeboten hat. Auch hast du es schon lange
versäumt -- obgleich du es zu besitzen vorgegeben hast. Doch, wenn dein Heiland
dich noch ruft, wenn er noch bei dir anklopft, das heißt: wenn sein Licht dir
noch scheint, dich noch bestraft, so ist noch Hoffnung da, daß der Tag deines
Heils noch nicht vorüber, noch nicht vor deinen Augen verborgen, -- daß noch Reue
möglich ist; da seine Liebe dir noch nachgehet, und seine heiligen Einladungen
zu deiner Errettung noch fortdauern.

Darum, o Christenheit, glaube an ihn, nimm ihn auf, und mache die rechte
Anwendung von ihm. Dieses ist von unumgänglicher Nothwendigkeit, wenn deine
Seele ewig mit ihm leben soll.
Er sagte zu den Juden: "`\textit{Wenn ihr nicht glaubet,
daß ich es bin, so werdet ihr in eurer Sünden sterben, und Wohin ich gehe, dahin
könnt ihr nicht kommen}"' \footnote{Johannes 8,21+24.}.
\index{Bibelstellen:!Johannes 8)} Weil sie nicht an ihn 
glaubten, nahmen sie ihn auch nicht an, und verloren also die Wohlthat seiner
Erscheinung. Diejenigen aber, die an ihn glaubten, nahmen ihn an, und "`Allen,
die ihn annahmen,"' sagt uns sein eigener geliebter Jünger, "`\textit{gab er Macht,
Kinder Gottes zu werden, die nicht von dem Geblüte, noch von dem Willen des
Fleisches, noch von dem Willen eines Mannes, sondern von Gott geboren
sind}"'\footnote{Johannes 1, 12. 13.} \index{Bibelstellen:!Johannes 1)}
Das heißt: die nicht Kinder Gottes nach den
Moden, Gebräuchen, Vorschriften. und überlieferten Sagen der Menschen sind, die
sich selbst den Namen der Kirche und des Volks Gottes beilegen; denn die wahren
Mitglieder der Kirche Christi werden keinesweges. nach dem Willen von Fleisch
und Blut, oder nach es der Erfindung fleischlichgesinnter, mit der Wiedergeburt
und Kraft des heiligen Geistes unbekannter Menschen hervorgebracht; sondern
wirklich von Gott, nämlich, nach seinem Willen und  durch die in ihren Herzen
wirkende, heiligende Kraft seines Geistes und Wortes des Lebens, wiedergeboren.
Diese haben immer die rechte Anwendung Christi wohl verstanden; ihnen ist er in
der That zur Versöhnung, Weisheit, Heiligung, Gerechtigkeit, Erlösung und
Rechtfertigung gemacht.

Nun sage ich dir: Wenn du nicht glaubst, daß Er, der vor der Thür deines Herzens
stehet und anklopft, der dir deine Sünden der Reihe nach verhält, und dich zur
Buße (d. i. zur Reue und Sinnesänderung) ruft, der Heiland der Welt sey, so
wirst du in deinen Sünden sterben, und wohin er gegangen ist, dahin wirst du nie
kommen. Denn, wenn du nicht an ihn glaubst, so ist es ganz unmöglich, daß er dir
helfen und deine Seligkeit bewirken könne. Er konnte ehemals, wie wir lesen, an
einigen, Orten nicht viele mächtige Werke verrichten, weil die Menschen nicht an
ihn glaubten.\footnote{Matthäus 13,58; Markus 6,5-6}
\index{Bibelstellen:!Matthäus 13)} \index{Bibelstellen:!Markus 6)}
Wenn du aber wahrhaft an ihn
glaubst, so wird dein inneres, Ohr aus feine Stimme in dir aufmerksam sein, und
dann wirst du die Thür deines Herzens seinem Anklopfen öffnen. Du wirst den
Offenbarungen seines, Lichts nachgeben, und die Belehrungen seiner Gnade werden
dir sehr scheinbar sein.

\section{5. Abschnitt}  \label{kap2_ab5}

Es liegt in der Natur des wahren Glaubens, daß er eine heilige Furcht, Gott zu
beleidigen, eine tiefe Ehrfurcht vor seinen Geboten, und eine sehr zarte
Aufmerksamkeit auf das innere Zeugniß seines Geistes in uns erzeugt. Dadurch
sind zu allen Zeiten die Kinder Gottes sicher zur Herrlichkeit geführet worden.
Denn, so wie Diejenigen, die wahrhaft glauben, Christum mit allen seinen Gaben
in ihen Herzen aufnehmen, so ist es auch gewiß, daß Diejenigen, die ihn auf
diese Weise aufnehmen, durch ihn Macht empfangen, Gottes Kinder zu werden. Sie
empfangen nämlich innere Kraft und Fähigkeit, Allee zu tun, was er von ihnen
fordert; Kraft, ihre Lüste zu bekämpfen; ihre Leidenschaften zu beherrschen; den
bösen Regungen der verderbten Natur zu widerstehen; sich selbst zu verleugnen
und die Welt in allen ihren schmeichelhaften und verführerischen Lockungen zu
überwinden. Dieses ist das Leben des heiligen und gesegneten Kreuzes Christi,
wovon in dieser Abhandlung noch nähere Erklärung gegeben wird, und welches du, o
Mensch! aufnehmen mußt, wenn du je ein wahrer Jünger Jesu werden willst. Wie
könnte auch sonst von dir gesagt werden, daß du Christume ausgenommen habest,
oder daß du an ihn glaubest, wenn du noch immer sein Kreuz verwirfst. Denn, da
Christum aufzunehmen das von Gott verordnete Mittel zur Seligkeit ist, so ist
auch das tägliche Tragen seines Kreuzes der einzige wahre Beweis, daß man ihn
wirklich aufgenommen habe, und darum hat er es auch Allen als das große
Kennzeichen seiner Nachfolge mit den Worten auferlegt: "`\textit{Wenn Jemand mit
nachfolgen will, der verleugne sich selbst, nehme sein Kreuz auf, und folge
mir.}"'\footnote{Matthäus 16, 24.}
\index{Bibelstellen:!Matthäus 16)}
Hieran hat es dir -- o Christenheit! -- bisher so sehr gefehlt, und dieser
Mangel ist
die einzige Ursache deines traurigen Abfalles vom wahren Christenthume. Dieses
nun wohl zu erwägen, ist eben sowohl deine Pflicht, als es dir zu deiner
Wiederherstellung gewiß sehr behilflich sein wird. Denn, so wie der Arzt durch
die Kenntniß der Ursache einer Krankheit in den Stand gesetzt wird, ein
richtiges und sicheres Urtheil über die anzuwendenden Heilmittel zu fällen, eben
so wird es auch dir auf dem Wege deiner Genesung Licht und Aufschlüsse geben,
wenn du die erste Ursache dieser geistlichen Krankheit und Schwäche, die dich
befallen hat, einsiehest und genau prüfst \footnote{"`reiflich erwägest"' durch
"`genau prüfst"' ersetzt}. Um aber dieses Absicht zu
erreichen, wird es nöthig sein, aus deinen ursprünglichen Zustand, und folglich
auch auf die Arbeiten Derer, die zuerst in dem christlichen Weinberge
arbeiteten, einen allgemeinen Ueberblick zu tun. Sollten dabei auch einige
Wiederholungen vorkommen, so wird die Würde und Wichtigkeit der Sache sie auch
ohne Entschuldigung schon erlauben.

\section{6. Abschnitt}  \label{kap2_ab6}

\marginpar{Das Licht Christi offenbart die Finsteren Seiten im Menschen.}
Das Amt der Apostel - wie einer der Ersten, die dasselbe bekleideten, uns sagt -
bestand darin: "`\textit{daß sie den Menschen die Augen öffnen sollten, damit sie
sich von der Finsterniß zum Lichte, und von der Gewalt des Satans zu Gott bekehren
möchten;}"' \footnote{Apostelgeschichte 26,18}
\index{Bibelstellen:!Apostelgeschichte 26)}.
Das heißt: damit die Menschen, statt den 
Versuchungen und Eingebungen des Satans, \index{Satan} der Fürsten der
Finsterniß oder der Ungerechtigkeit und Bosheit  wovon eine Benennung immer nur
ein figürlicher Ausdruck der andern ist) statt diesem mächtigen Einflusse des
Bösen, wodurch ihre Verstandeskräfte verfinstert und ihre Seelen in der
Knechtschaft der Sünde gehalten würden, nachzugeben, ihre Gemüther auf die
Erscheinung Christi, das Lichtes und Heilandes der Welt, richten möchten, der
mit seinem göttlichen Lichte ihre Seelen erleuchtet, ihnen dadurch Erkenntniß
ihrer Sünden giebt, die Versuchungen und Bewegungen zum Bösen in ihnen entdeckt,
und sie innerlich bestraft, wenn sie denselben nachgeben und in das Böse
einwilligen; damit sie aus diese Weise Kinder des Lichte würden, und auf dem
Pfade der Gerechtigkeit wandeln möchten.

\medskip 

Zu diesem gesegneten Werke der Verbesserung begabte Christus seine Apostel mit
Geist und Kraft, damit die Menschen nicht langer in der Sünde, und in der
Unwissenheit von Gott und göttlichen Dingen sicher hinschlummern möchten,
sondern zur Gerechtigkeit erwecket würden, und der Herr Jesus ihnen Leben geben
könnte; damit sie mit Sündigen aufhören, dem Vergnügen eines ungöttlichen Lebens
entsagen, mit wahrhaft reuevollem Herzen sich zu Gott wenden, und Gutes tun
möchten, worin Friede geschmeckt wird. Und wahrlich, Gott segnete die getreuen
Arbeiten (jener armen Handwerker, welche dennoch nichtsdestoweniger seine
großen Gesandten an das Menschengeschlecht waren) in solchem Maße, daß in
wenig Jahren viele Tausende, (die wie ohne Gott, ohne Gefühl von ihm und ohne ihn
zu fürchten, gesetzlos und und gänzlich unbekannt mit den Wirkungen seines
Geistes in ihrem Herzen, in fleischlichen Lüsten gefangen, in der Welt gelebt
hatten) durch das Wort des Lebens in ihrem Innern getroffen und zu einem
lebendigen Gefühle erweckt wurden, (so daß sie die Erscheinung und Kraft des
Herrn Jesu Christi, als eines Richters und Gesetzgebers, in ihren Herzen
erkannten. indem das Licht seines heiligen Geistes die verborgenen Dinge der
Finsterniß in ihnen offenbar machte und bestrafte, und aufrichtige Reue über
jene todten Werke mit dem festen Vorsatze erzeugte) hinfort dem lebendigen Gott
in einem neuen Leben des Geistes zu dienen. Diese lebten nun künftig nicht mehr
sich selbst, und ließen sich auch nicht länger von den mannichfaltigen Lüsten
hinreißen, durch welche sie von der wahren Furcht Gottes waren abgezogen und
abgeleitet worden; denn das Gesetz des lebendigmachenden Geistes, wodurch sie
"`\textit{das Gesetz der Sünde und des Todes}"' überwanden,
\footnote{Römer 8,2} \index{Bibelstellen:!Römer 8)} war nun
ihre wahre Lust, und dieses betrachteten sie Tag und Nacht. Die Ehrfurcht vor
Gott durfte ihnen nun nicht mehr durch menschliche Lehren beigebracht werden;
\footnote{Jesaja 29,13.} \index{Bibelstellen:!Jesaja 29)} sie entsprang
natürlich aus der Erkenntniß, die sie von
ihm durch seine eigenen Wirkung und Eindrück in ihren Herzen empfangen
hatten. Sie hatten ihre alten Herren: den \textit{Geist der Welt}, die
fleischlichen
Leidenschaften, und den Einfluß des Feindes ihrer Seelen verlassen, und sich
ganz der heiligen Leitung der Gnade Jesu Christi ergeben, welche sie lehrte:
"`\textit{das ungöttliche Wesen und die weltlichen Lüste zu verleugnen, und
züchtig, gerecht und gottselig in dieser Welt zu leben.}"'
\footnote{Titus 2,11+12.} \index{Bibelstellen:!Titus 2)}
Dieses ist das wahre \textbf{Kreuz Christi}, und hierin bestehet der Sieg, den
es Allen giebt, die es wirklich aufnehmen und getreulich tragen. Durch dieses
Kreuz starben sie dem alten Leben, das sie zuvor geführt hatten, täglich ab, und
durch heilige Wachsamkeit gegen die geheimen Regungen des Bösen in ihren Herzen
erstickten sie die Sünde in ihrer Geburt, im Augenblicke der Versuchung; so daß
sie, wie der Apostel Johannes anräth, "`\textit{sich bewahrten, und der Böse sie
nicht antasten konnte.}"' \footnote{Johannes 5, 18.}
\index{Bibelstellen:!Johannes 5)}
Denn das Licht, mit welchem Christus sie erleuchtet hatte, und welches der Böse
\footnote{"`der Arge"' durch "`der Böse"' ersezt}
nicht ertragen kann, entdeckte ihn in allen seinen Annährungen und Angriffen auf
ihre Seelen, und die Kraft, die sie durch ihren Gehorsam gegen die innern
Offenbarungen dieses heiligen Lichtes empfingen, machte sie vermögend, ihm in
allen seinen Kunstgriffen zu widerstehen, und ihn zu besiegen. Auf diese Art
konnte nun nichts mehr ununtersucht durchgehen, Was sonst gar nicht untersucht
wurde. Jeder Gedanke mußte geprüft werden, und ehe nicht Ursprung und Zweck
desselben völlig gebilligt werden konnten, gestatteten sie ihm keinen Raum in
ihren Seelen. Während sie also auf diese Weise jeden Zugang in ihre Herzen genau
bewachten, war nicht zu befürchten, daß sie Feinde für Freunde aufnehmen würden.
Schnell verschwanden nun auch der alte Himmel und die alte Erde, nämlich: der
alte fleischliche oder jüdische, in Schatten und Bilder gehüllte Gottesdienst,
\index{Jüdischer Gottesdienst} 
und der alte irdische Sinn und Wandel, und Alles wurde täglich neu.
"`\textit{Man hielt
den nicht mehr für einen Juden, der bloß äußerlich ein Jude war; auch war das
keine Beschneidung, die äußerlich am Fleische geschah; sondern der war ein Jude,
der es innerlich im Verborgenen war, und das war eine Beschneidung, die am
Herzen, im Geiste und nicht im Buchstaben geschah, deren Lob, nicht von
Menschen, sondern von Gott war.}"'\footnote{Römer. 2,28-29}

\section{7. Abschnitt}  \label{kap2_ab7}

In der That, die Herrlichkeit des Kreuzes leuchtete aus dem Leben der
Selbstüberwindung \footnote{"`Selbstverleugnung"' durch "`Selbstüberwindung"'
ersezt} Derer, die es trugen, so augenscheinlich hervor, daß es die
Heiden mit Staunen erfüllte, und in sehr kurzer Zeit ihre Altäre so
erschütterte, ihre Orakel so sehr um ihren Ruf brachte, die Menge so ergriff,
daß es sogar bis an die Höfe drang, ihre Armeen überwand, und Priester,
Obrigkeiten und Feldherren, als Trophäen seiner Macht, und seines Sieges, im
Triumphe nach sich zog.

\medskip

So lange nun jener erliche\footnote{"`lautere"' durch "`erliche"' ersezt}
Sinn unter den Christen herrschte, war auch die
göttliche Gegenwart bei ihnen groß, und die Kraft, die sie begleitete,
unüberwindlich. "`\textit{Sie löschte des Feuers Gewalt, bändigte Löwen,
wandte die Schärfe des Schwertes ab, trotzte den Werkzeugen der Grausamkeit,
überzeugte Richter, und bekehrte Henkersknechte.}"'
\footnote{Hebräer 11,32 bis zu Ende) Jesaja 43,2) Daniel 3,12 bis zu Ende}
\index{Bibelstellen:!Hebräer 11)} \index{Bibelstellen:!Jesaja 43)}
\index{Bibelstellen:!Daniel 3)}
Kurz, die Mittel, die ihre Feinde anwendeten, sie
zu vernichten \footnote{"`vertilgen"' durch "`vernichten"' ersetzt}, dienten
nur, ihre Anzahl zu vermehren; und durch die tiefe
Weisheit. Gottes wurden selbst Diejenigen zu Beförderern der Wahrheit gemacht,
welche mit allen ihren Anschlägen ihr entgegen zu wirken versuchten. Damals war
bei den Christen kein eitler Gedanke, kein unnützes Wort, keine ungeziemende
Handlung -- nein -- nicht einmal ein unbescheidener Blick gestattet. Putz und
Kleiderpracht, Verbeugungen oder körperliche Ehrenbezeugungen waren keinesweges
bei ihnen erlaubt; noch weniger aber fand man unter ihnen weder Beispiel noch
Nachsicht für solche niedrige Unsittlichkeiten und schändliche Laster, als unter
den jetzigen Bekennern des cihristenthumes norml sind
\footnote{"`im Schwange gehen"' ersetzt durch "`normal sind"'}. Jene waren nicht
besorgt, wie sie ihre kostbare Zeit vertreiben und verschwenden sollten. Nein,
sie suchten dieselbe vielmehr sorgfältig zu erkaufen\footnote{Epheser 6,15-16}
\index{Bibelstellen:!Epheser 6)},
damit ihnen genug davon übrig bliebe, das wichtige Heil ihrer Seelen zu
bewirken, welches sie denn auch mit Furcht und Zittern sorgsam thaten. Daher
hatten sie auch keine Bälle und Maskeraden, keine Schauspiele, keine
Tanzparthieen, Gastereien und Spielgesellschaften. Nein! Nein!
"`\textit{Ihren himmlischen Beruf und ihre Erwählung sicher zu stellen}"' \footnote{2. Petrus 2,10} \index{Bibelstellen:!2. Petrus 2)}
war ihnen weit wichtiger und teurer, als der Genuß armseliger, geringfügiger
Freuden der Vergänglichkeit. Denn, da sie -- wie Moses -- den unsichtbaren
gesehen und erkannt hatten, daß seine liebende Güte besser als das Leben,
der Friede seines Geistes besser als Fürstengunst ist; und daher den Zorn eines
Cäsars \index{Cäsar} nicht fürchteten; so wahlten sie viel lieber die Trübsale
der wahren Pilger \index{Pilger} Christi zu erdulden, als die Vergnügungen der
Sünde zu genießen, — die doch auch nur von kurzer Dauer sind, — und schätzten
die Schmach Christi unendlich höher, als die vergänglichen Schätze der Erde.
Diese konnten endlich aber auch für Diejenigen, welche überzeugt waren, daß
die mit dem wahren Christenthume verbundenen Prüfungen und Trübsale den
Ergötzlichkeiten der Welt, und die Schmach Christi aller weltlichen Ehre weit
vorzuziehen sei, gewiß keine Versuchung enthalten, die stark genug gewesen wäre,
ihre aufrichtige Anhänglichkeit an die Religion Christi zu erschüttern.

\section{8. Abschnitt}  \label{kap2_ab8}

\textbf{Aus diesem kurzen Abrisse von dem, was das Christenthum ehemals war, kannst
du -- o Christenheit! -- nun sehen, was du nicht bist, und was du folglich sein
solltest. Wie gehet es aber zu, daß wir statt eines so sanften, barmherzigen,
sich selbst überwinden \footnote{"`verleugnenden"' durch "`überwinden"' ersetzt},
duldenden, mäßigen, heiligen, gerechten und guten Christenthums, das Christo,
dessen Namen es führet, so ähnlich war; jetzt ein abergläubisches, abgöttisches,
verfolgendes, stolzes, leidenschaftliches, neidisches, boshaftes, selbstsüchtiges,
trunkenes, wollüstiges, unreines, lügenhaftes, fluchendes, habsüchtiges,
bedrückendes, betrügerisches Christenthum vorfinden? ein Christenthum, das aller
Abscheulichkeiten, die man nur auf der Erde kennt, voll ist, und diese noch dazu
in einem so hohen Uebermaße ausübt, daß es den schlimmsten der heidnischen
Zeitalter zur Schande gereichen würde; indem es jene früheren Jahrhunderte noch
mehr im Bösen selbst, als in der Zeitdauer desselben übertriftt.} Ich frage,
woher kommt dieser beklagenswerthe Verfall?

\medskip

\textbf{Die unzweifelbare Ursache dieser Entartung ist} (wie ich behaupte)
\textbf{keine andere,
als die innere Unachtsamkeit deines Gernüths auf das in dir scheinende Licht
Christi, das dir zuerst deine Sünden zeigte, dich über dieselben destrafte, und
dich lehrte und fähig machte, sie zu überwinden \footnote{"`verleugnen"'
ersetzt durch "`überwinden"'} und ihnen zu widerstehen.} Denn,
so wie deine Gottesfurcht und heilige Enthaltung von allem Bösen dir zuerst
nicht durch Menschengebote, sondern durch das göttliche Licht gelehret wurde,
welches dir die geheimsten Gedanken und Absichten deines Herzens offenbarte und,
dein Innerstes erforschte, indem es dir deine Sünden der Reihe nach vor Augen
stellte, dich über dieselben innerlich bestrafte, und keine unfruchtbare
Gedanken, Worte oder Werke der Finsternißi ungerichtet durchgehen ließ; so
geschah es auch hernach, als du anfingest, dieses göttliche Licht, diese innere
Gnade aus der Acht zu lassen, die vorhin in deinem Herzen unterhaltene heilige
Wachsamkeit zu vernachlässigen, und nicht, wie zuvor, gegen Alles, was der Ehre
Gottes und deinem eigenen Frieden nachtheilig sein konnte, auf deiner Hut zu
stehen; daß der rastlose Feind der menschlichen Glückseligkeit diese
Erschlaffung und Unwachsamkeit schnell benutzte, und dich oft mit Versuchungen
überraschte, die jedesmal deinen Neigungen angemessen waren, und ihm daher den
Sieg über dich leicht machten. -- Kurz -- du unterließest, das heilige Joch
Christi aufzunehmen, und dein tägliches Kreuz zu tragen; du gabest deinen
Neigungen zu viel nach, und führtest kein Gegenregister über deine Handlungen;
denn du warest abgeneigt, Christo, deinem Lichte, dem großen Bischofe deiner
Seele und Richter deiner Werke, in deinem Gewissen Rechenschaft zu geben. So
geschah es denn, daß die heilige Furcht Gottes sich verminderte, die Liebe
erkaltete, die Eitelkeit überhand nahm, und die Erfüllung deiner Pflichten dir
lästig wurde. Nun trat leere Formalität an die Stelle der Kraft der
Gottseligkeit; Aberglaube an die der Anordnung Christi; und \textbf{obgleich Christus
beständig den Zweck hatte, die Gemüther seiner Jünger vom äußern Tempel und von
fleischlichen Gebräuchen und Zeremonien abzuziehen, und zur innern, geistlichen,
der Natur und dem Wesen der Gottheit angemessenen Gottesverehrung anzuleiten, so
wurden dennoch wieder ein weltlicher, von Menschen erfundener, prachtvoller
Gottesdienst \index{Gottesdienst} und ein weltliches Priesterthum
\index{Priesterthum} eingeführt, und auch wieder Tempel 
und Altäre errichtet.}
"`\textit{Da sahen die Kinder Gottes wieder nach den Töchtern der
Menschen, wie sie schön waren.}"'\footnote{1. Mose 6,2}
\index{Bibelstellen:!1. Mose 6)}Es ward nämlich das durch
Reue in dir geöffnete, reine Auge, das zuvor außer Christo keine Schönheit
erblickte, wieder verdunkelt, und das Auge der Weltlust von dem Gotte dieser
Welt von neuem geöffnet; die weltlichen Ergötzungen, welche Diejenigen, die sie
lieb gewinnen, von Gott abziehen, erlangten nun, -- wiewohl sie einst um Christi
willen waren verleugnet worden -- ihren alten Reitz und ihre vorige Herrschaft
über deine Neigungen wieder, und so wurden sie dann, -- da du ihnen nachhingest
-- auch wieder die Gegenstände der Betrachtung, der Sorge und der Freude deines
Lebens.

\medskip 

Es blieb freilich immer noch eine äußere Form des Gottesdienstes und eine
scheinbare, mündliche Verehrung Gottes und Christi übrig; das war aber auch
Alles. Das Ärgerniß und die Schmach des heiligen Kreuzes hörten auf, die Kraft
der Gottseligkeit verschwand, an Selbstüberwindung
\footnote{"`Selbstverleugnung"' ersetzt duch "`Selbstüberwindung"'}
ward nicht mehr gedacht. \textbf{Man
war allerdings sehr fruchtbar in Erfindung neuer Zeremonien und Verzierungen,
allein desto unfruchtbarer im Guten; die köstlichen Früchte des Geistes blieben
zurück. Denn Tausende von Schalen können nicht einen einzigen Kern, und viele
todte Körper nicht einen lebendigen Menschen ersetzen.}

\section{9. Abschnitt}  \label{kap2_ab9}

\textbf{So sank die Religion von der lebendigen Erfahrung zu überlieferten
Meinungen,
die wahre Gottesverehrung von der Kraft zu der leeren Form, von dem Leben des
Geistes zu dem todten Buchstaben herab.} Statt lebendiger und kräftiger Gebete,
die aus einem tiefen Gefühle unserer Bedürfnisse entspringen, und unter dem
Beistande des heiligen Geistes hervorgebracht werden. Statt solche Gebete,
in welchen die Alten mit Gott rangen und ihn übermochten,
hört und bemerkte man ein
eingeübtes Gemurmel, eine schale, bloß aus körperlichen Beugungen und
Schmiegungen, Gewänden und Geräten, Wohlgerüchen, Gesängen und musikalischen
Tönen zusammengesetzte, kraftlose Formalität, die eher zum Empfange eines
irrdischen Fürsten, als zur himmlischen Verehrung und Anbetung des allein wahren
und Unsterblichen Gottes, eines ewigen und unsichtbaren Geistes, geeignet ist.

\medskip 

Da aber einmal dein Herz fleischlich geworden war, so ward es nun auch deine
Religion, die du, weil sie dir so, wie sie war, nicht gefiel, nach deinem
eigenen Gefallen umbildetest; ungeachtet \footnote{"`uneingedenk"' durch
"`ungeachtet"' ersetzt} der Worte des heiligen Propheten:
"`\textit{Der Gottlosen Opfer ist dem Herrn ein Greuel;}"'
\footnote{Sprüche 15,8} \index{Bibelstellen:!Sprüche 15)} und
dessen, was Jacobus sagt: "`\textit{Ihr bittet, und empfanget nicht;}"' und warum
nicht? "`\textit{weil ihr übel bittet;}"' \footnote{Jakobus 4,3}
\index{Bibelstellen:!Jakobus 4)} nämlich mit einem
Herzen, das nicht gerade, sondern unaufrichtig ist, das sich nicht sich
selbst überwindet \footnote{"`selbst verleugnet"' ersetzt durch "`sich
selbst überwindet"'}, das nicht in dem Glauben stehet, der daß Herz reinigt,
und daher
auch die Dinge nicht erlangen kann, um welche er bittet; so daß man mit
Wahrheit von dir sagen kann: \textbf{dein Zustand ist durch deine Religion nur noch
schlimmer geworden, weil du der Versuchung nachgiebst, dich der Ausübung
deiner religiösen Gebräuche und Zeremonien wegen für besser zu halten, als du
bist.}

\section{10. Abschnitt}  \label{kap2_ab10}

\textbf{Aus dieser Einsicht nun}, die dir in deinen traurigen Abfall vom ursprünglichen
Christenthume, und in die wahre Ursache desselben, -- die in nichts anders, als
in deiner Vernachlässigung des täglichen Kreuzes Christi bestehet, -- hier
gegeben ist, \textbf{wird es dir leicht sein, dir selbst von dem Mittel zu deiner
Wiederherstellung einen klaren Begriff zu machen. Denn, zu derselben Thür, aus
welcher du hinausgegangen bist, mußt du auch wieder hereingehen; oder: so wie du
dadurch, daß du das tägliche Kreuz sinken ließest, und dich demselben entzogest,
dich um deinen glücklichen Zustand gebracht hast, eben so muß auch dein
Wiederaufnehmen und dein tägliches ausdauerndes Tragen desselben dich wieder
herstellen.} Es ist ein und dasselbe Mittel, durch welches Sünder und Abtrünnige
zu Jüngern Jesu gemacht werden; wie er selbst gesagt hat: "` \textbf{Wer mir
folgen will, der überwinde \footnote{"'verleugne"` durch "'überwinde"` ersetzt}
sich selbst, nehme täglich sein Kreuz auf, und folge
mir nach; und wer nicht sein Kreuz trägt, und mir nachfolgt, der kann nicht mein
Jünger sein.} "' \footnote{Matthäus 16,24) Markus 8,34) Lukas 14,17}
\index{Bibelstellen:!Matthäus 16)} \index{Bibelstellen:!Markus 8)}
\index{Bibelstellen:!Lukas 14)} Nichte
Geringeres als dieses wird hinreichen; daß merke dir. So wie es aber hinreichend
ist, so ist es auch unerläßlich. \textbf{Keine Krone, als nur durchs Kreuz;
kein ewiges Leben, als nur durch den Tod!
Und es ist auch nicht mehr als gerecht, daß jene
bösen und grausamen Leidenschaften, die Christum von neuem gekreuzigt haben,
durch sein täglicher Kreuz wieder gekreuzigt werden sollten. Sein Kreuz ist der
Tod der Sünde, die seinen Tod verursachte.} Er selbst aber ist
"`\textit{der Tod des Todes,}"' nach der Schriftstelle beim Hosea:
"`\textit{O Tod! ich will dir ein Gift sein!}"' \footnote{Hosea 13,14}
\index{Bibelstellen:!Hosea 13)} (nach der englischen Bibel: "`O Tod! ich will dein
Tod sein."')

