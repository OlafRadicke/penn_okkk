\chapter{Zu dem Autor William Penn} \label{ref:zum_autor_penn}
% Vielleicht eine bessere Überschrift??

\begin{flushright}
\begin{footnotesize}
\textit{Von Claus Bernet}
\end{footnotesize}
\end{flushright}
\smallskip 

William Penn ist der in der Öffentlichkeit bekannteste Quäker, was nicht zuletzt ein umfangreiches Schrifttum über sein Leben und sein Wirken belegt.  Er wurde am 14. Oktober 1644 in London geboren, als ältester Sohn von Sir William Penn (1621-1670), einem angesehenen Admiral, und seiner Frau Margaret Jasper (um 1610-1682), die Familienbezüge nach Irland hatte. Nach einem häuslichen Schulunterricht und dem Besucht der Chigwell Academy bei Wanstead (Essex) studierte der junge Penn in Oxford am Christ Church College unter John Owen (1616-1683), Robert Spencer (1640-1702) und möglicher Weise auch unter John Locke (1632-1704). Anschließend begab er sich auf eine Kavalierstour nach Oberitalien und studierte in Paris protestantische Theologie, wo er durch Protektion von Moise Amyraut (1596-1664) am Hof des Sonnenkönigs zur Audienz zugelassen wurde. Zurück in London studierte er erneut einige Monate – diesmal Jura – an der Lincoln’s Inn Stiftung und begann dann, das Familienvermögen der Penns, Grundbesitz in England und Irland, zu verwalten.
William Penn kam frühzeitig mit Denominationen in Kontakt. Schon während seiner Zeit in Oxford stand er mit dem Prediger Thomas Loe (gest. 1668) in Verbindung, der der Bewegung „Kinder des Lichts“ angehörte und Quäker wurde. 1668 erlebte Penn ein Bekehrungserlebnis, ein Jahr später besuchte er in Cork erstmals eine Quäkerversammlung. Darüber kam es zu einem Zerwürfnis mit seiner Familie: Penn wurde von seinem Vater verstoßen und wegen seines Glaubens mehrfach inhaftiert. Während einer Haft in dem berüchtigten Gefängnis von Newgate schrieb Penn sein bekanntes Buch „No Cross No Crown“ (1669), das neben seinen „Fruits of Solitude“ selbst heute noch von einer breiteren Leserschaft herangezogen wird und eine erste Zusammenfassung der quäkerischen Verhaltensweisen darstellt.
1672 heiratete Penn die Quäkerin Gulielma Maria Springett (gest. 1694). Aus dieser Ehe sind sieben Kinder hervorgegangen, darunter William (gest. 1720) und Lätitia (genannt Letty). Nach dem tragischen Tod seiner Frau durch eine langwierige Erkrankung heiratete Penn 1696 erneut. Mit seiner zweiten Ehefrau Hannah Callowhill (1670-1726) hatte er sechs Kinder, darunter Margaret, John, Richard, Dennis und Thomas (1702-1775).

\medskip 

Penn war vor allem ein Vorkämpfer der religiösen Toleranz, fast alle seiner etwa fünfzig Schriften behandeln in der ein oder anderen Form dieses Thema. Pazifistischen Überlegungen zu einem europäischen Frieden in Wohlstand und Gerechtigkeit, entstanden zwischen 1691 bis 1693, sind in „An Essay Toward the Present and Future Peace of Europe“ (1693) niedergelegt. Maßgebliches Instrument zur Erreichung dieser Ziele war die Einrichtung eines Europäischen Parlaments. Penn reiste 1671 und erneut 1677 nach Norddeutschland und gelangte bis an den Hof der Äbtissin Elisabeth (1618-1680). Vor allem in Frankfurt warb Penn später unter den dortigen Pietisten Kolonisten für seine nordamerikanischen Besitzungen. 1681 beglich König Charles II. seine Schulden an die Familie Penn, indem er William Penn mit einem Gebiet in Nordamerika (heute die Bundesstaaten Pennsylvania und Delaware) belehnte und ihn zu dessen Gouverneur ernannte. Noch im gleichen Jahr wurde „Philadelphia“ gegründet, bald darauf Germantown. Penn konnte in diesem „holy experiment“ einen Teil seiner religiösen und humanistischen Ideale verwirklichen: religiöse Toleranz für alle christlichen Bekenntnisse (außer dem Katholizismus), persönliche Freiheit für die Kolonisten, Amtszugang und Gerichtsverfahren ohne Eidesleistung, weitgehende Einschränkung der Todesstrafe, Respektierung der Rechte von Ureinwohnern. Penn erwarb sich sogar Sprachkenntnisse der Lenni Lenape, nicht um diese zu Bekehren, sondern um mit ihnen in direkte Verhandlungen treten zu können und um ihre Lebenswelt besser verstehen zu können. Finanziell jedoch wurde die Kolonie zu einem Misserfolg, was ihm 1707 sogar eine Inhaftierung wegen Überschuldung einbrachte.

\medskip 

Seinen politischen Höhepunkt erreichte Penn unter James II. in der Zeitspanne von 1685 bis 1688. Für den katholischen König war er am oranischen Hause in den Niederlanden diplomatisch tätig, um Zustimmung zur Abschaffung der Test Act zu gewinnen. Penns größter Erfolg war die Formulierung der „Declaration of Indulgence“ 1687, mit der nicht allein Quäker staatlicherseits toleriert waren, sondern der Theorie nach sogar alle Religionsangehörige, nicht jedoch Atheisten. Als hingegen 1688 Wilhelm von Oranien (1650-1702) intervenierte, wurde Penn als angeblicher Jesuit inhaftiert. In den folgenden Jahren konnte er seinen ehemaligen politischen Einfluss am Königshofe nicht wiedererlangen.

\medskip 

1699 bis 1701 hielt sich Penn zum zweiten Mal in Pennsylvania auf. Seine letzten Lebensjahre verbrachte er dann in England, geplagt von den verschiedensten Krankheiten und körperlichen Verfallserscheinungen. Er erlitt drei Schlaganfälle, nach denen seine Ehefrau die Geschäfte führte. Verbittert und vom Leben enttäuscht verstarb er am 30. Juli 1718 in Ruscombe, Berkshire und wurde in Jordans, Buckinghamshire, begraben.




