% Bemerkung zu Steuerzeichen:
% Kommentare ohne Auswirkung des Ausgabeformats werden mit "%" (Prozentzeichen)
% am Anfang der Zeile markiert. Alles hinter diesem Zeichen kann korrigiert
% werden. Es können auch Editor-Kommentare hinterlassen werden, um auf
% Korrekturprobleme hinzuweisen
%
% Steuerbefehle beginnen fast immer mit "\" und sind so zu belassen Z.B.:
% \begin{x}
% \end{x}
% \smallskip
% \medskip
% \label{x}
% \pageref{x}
%
% Ausnahme ist:
% \footnote{Y}
% \textit{Y}
% \textbf{Y}
% Hier ist jeweils der Text (Y) in den geschweiften Klammern "{}" zu
% korrigieren.
%
% Bei dem Steuerbefehl...
% \index{X!Y}
% Ist X und Y zu korrigieren aber nicht das "!". Bei "!" handelt es sich
% um ein Steuerzeichen, was unverändert bleiben muss.



\section{Zur Person und Autor William Penn} \label{ref:zum_autor_penn}
% Vielleicht eine bessere Überschrift??

\begin{flushright}
\begin{footnotesize}
\textit{Von Claus Bernet \& Olaf Radicke}
\end{footnotesize}
\end{flushright}
\smallskip

Die frühen Quäker waren wirklich ein bunter Haufen. Es war so ziemlich alles
unter ihnen zu finden ehemalige Soldaten der New Model Army\index{New Model
Army}, Musikanten, Handwerker, Hausfrauen, Bauern, Händler... einfach alles.
Die Mennoiten  waren --
z.B. als Vergleich -- wesentlich homogener in ihrer sozialen Zusammensetzung.
Aber
William Penn war selbst unter Quäkern noch mal etwas besonderes. Er stammt aus
dem Adel und hatte einflussreiche Kontakte in die höchsten Kreise. An einigen
Stellen lässt er das im Text auch durchscheinen. So beschreibt er auf
Seite~\pageref{kap9_ab2_duell_penn} das Erlebnis eines Duells
mit einem anderen Mann (ebenfalls aus dem Adel), der mit einem Degen auf ihn los
geht\footnote{Der Besitz eines
Degens war dem Adel vorbehalten.}. An einer anderen Stelle im Text auf
Seite~\pageref{15_04_penn_alte_freunde} will er den Adel als Adressat seiner
Botschaft ansprechen. Das tut er, in dem er sich als seines Gleichen zu erkennen
gibt.
\textit{"`Untersucht doch nur, inwiefern euer Leben und euere Gesinnungen mit
diesen heiligen Vorschriften und Beispielen der Selbstüberwindung
übereinstimmen. O! Meine Freunde! Meine Seele trauert über euch! Ich lebte einst
selbst mit euch und in eurer Mitte."'}

\medskip

Penn wurde am 14. Oktober 1644 in London
\index{Orte:!London} geboren, als ältester Sohn von Sir William Penn
(1621-1670)\index{Personen:!Penn, Sir William}, einem angesehenen Admiral, und
seiner Frau Margaret Jasper (um 1610-1682) \index{Personen:!Jasper, Margaret},
die familiere Verbindungen nach Irland \index{Orte:!Irland} hatte.
\medskip

Als Penn sich zum Quäkertum bekannte, wurde er von seinem Vater verstoßen.
Sein Vater ließ sich jedoch trozdem nicht nehmen, sich immer wieder für seinen
Sohn einzusetzen, wenn dieser mal wieder
wegen seiner quakerischen Überzeugungen im Gefängnis saß. Aber noch
kurz vor dem Tod des Vaters, gab es eine Versöhnung zwischen den beiden. Während
einer seiner Gefängnissaufenthalte in Newgate \index{Orte:!Gefängnis!von
Newgate} oder den Tower zu London (von dem er selber schreibt in Vorwort), entstand
das Buch "`Ohne Kreuz keine Krone"'.

\medskip

Penn selbst war zweimal verheiratet. Das erste Mal heiratete er 1672 die
Quäkerin Gulielma Maria Springett (gest. 1694)
\index{Personen:!Springett, Gulielma Maria}. Aus dieser Ehe sind sieben Kinder
hervorgegangen, darunter William (gest. 1720) und Lätitia (genannt Letty)
\index{Personen:!Penn, Lätitia 'Letty'}. Nach dem tragischen Tod seiner Frau
durch eine langwierige Erkrankung heiratete Penn 1696 erneut. Mit seiner zweiten
Ehefrau Hannah Callowhill (1670-1726) \index{Personen:!Callowhill, Hannah} hatte
er sechs Kinder, darunter Margaret, John, Richard, Dennis und Thomas (1702-1775)
\index{Personen:!Penn, Thomas}.

\medskip

Im Gegensatz zu anderen prominenten Quäkern wie Nayler\index{Personen:!Nayler,
James}  Fox\index{Personen:!Fox, George} und Fell\index{Personen:!Fell,
Margaret} war er nicht nur sehr produktiv bei Korrespondenz und
Flugblättern\index{Flugblättern}, sondern schrieb er auch eine Reihe von
Büchern\footnote{Insgesammt verfasste Penn etwa fünfzig Schriften}.
Das wohl bekannteste -- noch vor \textit{"`Ohne Kreuz keine Krone"'} -- ist
\textit{"`Früchte der Einsamkeit"'}. Das Buch \textit{"`Ohne Kreuz keine
Krone"'} steht in der Gattung zwischen den Anekdotenhaften Journals
(Tagebücher) von Fox
oder auch Woolman\index{Personen:!Woolman, john} und theologisch trockenen
Abhandlungen eines Robert Barclay\index{Personen:!Barclay, Robert}. Das Buch
bildet so auch sehr gut den Charakter von Penn im realen Leben ab. Einerseits
handfester Pragmatiker, aber andererseits auch Visionär. Literarisch und im Leben,
eine mehr als glückliche Mischung, kann man sagen. Auch wenn er selbst sich
nicht als besonders erfolgreich erlebt haben mag. Den er starb später, tief
enttäuscht und verbittert vom Leben, am 30. Juli 1718 in Ruscombe, Berkshire
\index{Orte:!Berkshire}\footnote{Begraben wurde er in
Jordans, Buckinghamshire \index{Orte:!Buckinghamshire}}.

\medskip

Doch nüchtern betrachtet, hat Penn die Geschicke des jungen Quäkertums
beeinflusst wie kaum  ein anderer. Das vorrangige Bild, welches nach wie vor weit
verbreitet ist unter den Quäkern, ist das der Quäker in Nordamerika. Daß sie
dort so tiefe kulturelle und politische Spuren hinterlassen haben, ist in erster
Linie Penn zu verdanken. Als sein Vater starb, stand der König in seiner Schuld.
1681 beglich König Charles II \index{Personen:!Charles II, König}. seine
Schulden \index{Schulden} an die Familie Penn, indem er William Penn mit einem
Gebiet in Nordamerika (heute die Bundesstaaten Pennsylvania
\index{Orte:!Pennsylvania} und Delaware \index{Orte:!Delaware}) belehnte und ihn
zu dessen Gouverneur ernannte.

\medskip

Das war die Grundlage für ein etwa 80 Jahre währendes, in der Quäkergeschichte
-- bisher -- einmaliges Phänomen: Ein eigener Staat, mit eigener Verfassung und
Gesetzgebung. Von Penn "`Heiliges Experiment"' genannt. In diesem Abschnitt
stellten die Quäker ihre hohen Ideale in der rauen Realität der Politik auf die
Probe. Viele Glaubensgemeinschaften vor ihnen, die genauso unter Verfolgung
litten wie die Quäker, wurden später selber zu Tyrannen, die andere
Glaubensgemeinschaften bis auf das Blut verfolgten. So z.B. die
Calvinisten\footnote{Personen:!Calvinisten}, die zunächst von der katholischen
Kirche verfolgt wurden, und dann in der Neuen Welt (in den Kolonien
Nordamerikas) zu erbarmungslosen Verfolgern wurden. Was würde wohl aus den
schwärmerischen Quäkern werden, die andere immer so eindringlich für ihre
mangelnde Moral anklagten? Würden sie auch zu Tyrannen werden?

\medskip

Die Antwort lautet glücklicher Weise: Nein! Nach wie vor wurde der Kriegsdienst
abgelehnt. Es war üblich, daß die Kolonien der Krone\footnote{Vereinigtes
Königreich Grossbritannien} Soldaten ausheben und für militärische
Aktivitäten zu Verfügung stellen mussten. Das taten die Quäker auch nicht. Statt
dessen hatte Pennsylvania eine Strafzahlung zu leisten. Die Verfassung die sich
die Quäker in Pennsylvania gaben, war zu damaliger Zeit bahnbrechend: Persönliche
Freiheit für die Kolonisten, Amtszugang und Gerichtsverfahren ohne
Eidesleistung \index{Eid}, weitgehende Einschränkung der Todesstrafe
\index{Todesstrafe}, Respektierung der Rechte von Ureinwohnern
\index{Personen:!Ureinwohnern}. Penn erwarb sich sogar Sprachkenntnisse
\index{Sprachkenntnisse} der Lenni Lenape \index{Personen:!Lenni Lenape}, nicht
um diese zu bekehren, sondern um mit ihnen in direkte Verhandlungen treten zu
können und um ihre Lebenswelt besser verstehen zu können. Der faire Umgang und
Handel mit den Ureinwohnern führte immer wieder zu Konflikten mit den
benachtbarten Kolonien, da diese die Indianer regelmässig übervorteilten und
die Quäkern deshalb als Handelspartner von den Ureinwohnern bevorzugt wurden.
Auch unter den ständigen Indianerüberfällen und Entführungen
hatten Quäker nicht zu leiden, was auch zu Missgunst führte. Für Penn jedoch
selbst wurde die Kolonie finanziell zu einem Misserfolg, was ihm 1707 sogar eine
Inhaftierung \index{Inhaftierung} wegen Überschuldung einbrachte.

\medskip

Was die Religionsfreiheit betrifft, wurde das Ideal der religiösen Toleranz für
alle
christlichen Bekenntnisse tatsächlich umgesetzt. Es gab nur eine Ausnahme: dem
Katholizismus \index{Katholizismus}! Das ist in soweit erstaunlich, als das Penn
seinen politischen Höhepunkt unter James II.,
\index{Personen:!James II.} in der Zeitspanne von 1685 bis 1688  erreichte. Einem
katholischen König ,für den er am oranischen Hause\index{Personen:!Oranien} in
den Niederlanden
\index{Orte:!Niederlanden} diplomatisch tätig war, um die Zustimmung zur
Abschaffung des \textit{Test Act}\index{Test Act} zu gewinnen. Penns grösster
Erfolg war die Formulierung
der \textit{"`Declaration of Indulgence"'} 1687, mit der nicht allein Quäker
staatlicherseits toleriert wurden, sondern der Theorie nach sogar, alle
Religionsangehörigen, nicht jedoch Atheisten \index{Personen:!Atheisten}. Als
hingegen 1688 Wilhelm von Oranien (1650-1702) \index{Personen:!von Oranien,
Wilhelm} intervenierte, wurde Penn als angeblicher
Jesuit\index{Personen:!Jesuiten} inhaftiert. In den folgenden Jahren konnte er
seinen vormaligen politischen Einfluss am Königshofe \index{Königshof} nicht
wieder herstellen.

% was ist denn ein "Test Act"?

\medskip

Nun, wer sich mit der Quäkergeschichte intensiver beschäftigt, wird feststellen,
daß die Kontakte zwischen katholischer Kirche und Quäkern schwierig und
bisweilen tödlich war (für Quäker). Doch noch kurz zurück zum \textit{Heiligen
Experiment}: Penn drückte der Kolonie auch seinen Stempel als Städtegründer
b.z.w. Planer auf. Er gründete die Stadt "`Philadelphia"' mit strengen geraden
Strassennetz. Philadelphia City ist eine ältesten Städte der heutigen USA. Es gab
bis 1986 das ungeschriebene Gesetz, daß kein Gebäude höher gebaut werden sollte
als die \textit{Philadelphia City Hall} \index{Orte:!Philadelphia City Hall},
auf dessen Spitze eine Statue von Penn thront. Es gibt wohl nicht wenige in
Philadelphia City, die den Misserfolg ihres Sportclubs in Zusammenhang bringen,
mit dem Bruch dieses ungeschrieben Gesetzes und daher einen Fluch vermuten.
Fluch hin oder her, es zeigt das Penn einem, in der USA, immer noch auf
Schritt und Tritt
begegnet. Nach Philadelphia wurde kurz darauf  \textit{Germantown}
\index{Orte:!Germantown} gegründet. 1854 wurde sie in die Stadt Philadelphia
eingemeindet. Nach anderen Quellen wird auch Franz Daniel Pastorius als Gründer
von Germantown genannt. Da Pastorius im Auftrag von Penn in Deutschland für das
\textit{Heiliges Experiment} Kolonisten warb, ist die Frage wohl nebensächlich.

\medskip

Penn war auch selbst in Deutschland missionarisch tätig. Er reiste 1671 und 1677
nach Norddeutschland \index{Orte:!Deutschland!Nord-} und gelangte bis an den Hof
der Äbtissin Elisabeth (1618-1680) \index{Personen:!Äbtissin Elisabeth}. Vor
allem in Frankfurt \index{Orte:!Frankfurt} warb Penn später, unter den dortigen
Pietisten,
Kolonisten \index{Personen:!Pietisten} für seine nordamerikanischen
\index{Orte:!Nordamerika} Besitztümer. Das Interesse von Penn an Europa, war
aber nicht nur missionarischer Natur. Die zahllosen Kriege und Konflikte brachten
ihn zu pazifistischen \index{Pazifismus}
Überlegungen; hin zu einem europäischen Frieden \index{Frieden!europäischer} in
Wohlstand \index{Wohlstand} und Gerechtigkeit \index{Gerechtigkeit}, entstanden
zwischen 1691 bis 1693, sind in \textit{"`An Essay Toward the Present and Future
Peace of Europe"'}(In Deutsch erschienen unter dem Titel "`William Penns
Friedensplan für Europa"' ) niedergelegt. Massgebliches Instrument zur
Erreichung dieser
Ziele sah er in der Einrichtung eines Europäischen Parlaments
\index{Europäisches!Parlament}.
