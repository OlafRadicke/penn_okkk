\chapter{Zu dem Autor William Penn} \label{ref:zum_autor_penn}
% Vielleicht eine bessere Überschrift??

\begin{flushright}
\begin{footnotesize}
\textit{Von Claus Bernet}
\end{footnotesize}
\end{flushright}
\smallskip

William Penn ist der in der Öffentlichkeit bekannteste
Quäker\index{Quäker!Bekannteste}, was nicht zuletzt ein umfangreiches Schrifttum
über sein Leben und sein Wirken belegt.  Er wurde am 14. Oktober 1644 in London
\index{Orte:!London} geboren, als ältester Sohn von Sir William Penn
(1621-1670)\index{Personen:!Penn, Sir William}, einem angesehenen Admiral, und
seiner Frau Margaret Jasper (um 1610-1682) \index{Personen:!Jasper, Margaret},
die Familienbezüge nach Irland \index{Orte:!Irland} hatte.

\medskip

Nach einem häuslichen Schulunterricht und dem Besucht der Chigwell Academy
\index{Orte:!Chigwell Academy} bei Wanstead (Essex) \index{Orte:!Wanstead
(Essex)} studierte der junge Penn in Oxford \index{Orte:!Oxford} am Christ
Church College \index{Orte:!Christ Church College} unter John Owen (1616-1683)
\index{Personen:!Owen, John}, Robert Spencer (1640-1702)
\index{Personen:!Spencer, Robert} und möglicher Weise auch unter John Locke
(1632-1704) \index{Personen:!Locke, John}. Anschließend begab er sich auf eine
Kavalierstour nach Oberitalien \index{Orte:!Italien} und studierte
\index{Studium} in Paris \index{Orte:!Paris} protestantische Theologie, wo er
durch Protektion von Moise Amyraut (1596-1664) \index{Personen:!Amyraut, Moise}
am Hof des Sonnenkönigs \index{Personen:!Sonnenkönig} zur Audienz
\index{Audienz} zugelassen wurde. Zurück in London studierte er erneut einige
Monate -- diesmal Jura \index{Erkenntnis!Jura} -- an der Lincoln’s Inn Stiftung
\index{Orte:!Lincoln’s Inn Stiftung} und begann dann, das Familienvermögen der
Penns, Grundbesitz in England \index{Orte:!England} und Irland, zu verwalten.

\medskip

William Penn kam frühzeitig mit Denominationen \index{Denominationen} in
Kontakt. Schon während seiner Zeit in Oxford \index{Orte:!Oxford} stand er mit
dem Prediger Thomas Loe (gest. 1668) \index{Personen:!Loe, Thomas} in
Verbindung, der der Bewegung "`Kinder des Lichts"'
\index{Personen:!Kinder!des Lichts} angehörte und Quäker wurde. 1668 erlebte
Penn ein Bekehrungserlebnis \index{Bekehrung!-erlebnis}, ein Jahr später besuchte
er in Cork \index{Orte:!Cork(Irland)} erstmals eine Quäkerversammlung
\index{Quäker!-versammlung}.

\medskip

Darüber kam es zu einem Zerwürfnis \index{Zerwürfnis!mit Familie} mit seiner
Familie: Penn wurde von seinem Vater verstoßen und wegen seines Glaubens
mehrfach inhaftiert. Während einer Haft in dem berüchtigten Gefängnis von
Newgate \index{Orte:!Gefängnis!von Newgate} schrieb Penn sein bekanntes Buch
"`No Cross No Crown"' (1669), das neben seinen "`Fruits of Solitude"'
\index{Fruits-of-Solitude} selbst heute noch von einer breiteren Leserschaft
herangezogen wird und eine erste Zusammenfassung der quäkerischen
Verhaltensweisen darstellt.

\medskip


1672 heiratete Penn die Quäkerin Gulielma Maria Springett (gest. 1694)
\index{Personen:!Springett, Gulielma Maria}. Aus dieser Ehe sind sieben Kinder
hervorgegangen, darunter William (gest. 1720) und Lätitia (genannt Letty)
\index{Personen:!Penn, Lätitia 'Letty'}. Nach dem tragischen Tod seiner Frau
durch eine langwierige Erkrankung heiratete Penn 1696 erneut. Mit seiner zweiten
Ehefrau Hannah Callowhill (1670-1726) \index{Personen:!Callowhill, Hannah} hatte
er sechs Kinder, darunter Margaret, John, Richard, Dennis und Thomas (1702-1775)
\index{Personen:!Penn, Thomas}.

\medskip

Penn war vor allem ein Vorkämpfer der religiösen Toleranz
\index{Toleranz!religiöse}, fast alle seiner etwa fünfzig Schriften behandeln in
der ein oder anderen Form dieses Thema. Pazifistischen \index{Pazifismus}
Überlegungen zu einem europäischen Frieden \index{Frieden!europäischer} in
Wohlstand \index{Wohlstand} und Gerechtigkeit \index{Gerechtigkeit}, entstanden
zwischen 1691 bis 1693, sind in "`An Essay Toward the Present and Future Peace
of Europe"' (1693) niedergelegt. Maßgebliches Instrument zur Erreichung dieser
Ziele war die Einrichtung eines Europäischen Parlaments
\index{Europäisches!Parlament}.

\medskip

Penn reiste 1671 und erneut 1677 nach Norddeutschland
\index{Orte:!Deutschland!Nord-} und gelangte bis an den Hof der Äbtissin
Elisabeth (1618-1680) \index{Personen:!Äbtissin Elisabeth}. Vor allem in
Frankfurt \index{Orte:!Frankfurt} warb Penn später unter den dortigen Pietisten
Kolonisten \index{Personen:!Pietisten} für seine nordamerikanischen
\index{Orte:!Nordamerika} Besitzungen.

\medskip

1681 beglich König Charles II \index{Personen:!Charles II, König}. seine
Schulden \index{Schulden} an die Familie Penn, indem er William Penn mit einem
Gebiet in Nordamerika (heute die Bundesstaaten Pennsylvania
\index{Orte:!Pennsylvania} und Delaware \index{Orte:!Delaware}) belehnte und ihn
zu dessen Gouverneur ernannte. Noch im gleichen Jahr wurde "`Philadelphia"'
gegründet, bald darauf Germantown \index{Orte:!Germantown}. Penn konnte in
diesem "`holy experiment"' \index{holy experiment} \index{Heiliges!Experiment}
einen Teil seiner religiösen und humanistischen Ideale
\index{humanistischen!Ideale} verwirklichen: religiöse Toleranz für alle
christlichen Bekenntnisse (außer dem Katholizismus \index{Katholizismus}),
persönliche Freiheit für die Kolonisten, Amtszugang und Gerichtsverfahren ohne
Eidesleistung \index{Eid}, weitgehende Einschränkung der Todesstrafe
\index{Todesstrafe}, Respektierung der Rechte von Ureinwohnern
\index{Personen:!Ureinwohnern}. Penn erwarb sich sogar Sprachkenntnisse
\index{Sprachkenntnisse} der Lenni Lenape \index{Personen:!Lenni Lenape}, nicht
um diese zu Bekehren, sondern um mit ihnen in direkte Verhandlungen treten zu
können und um ihre Lebenswelt besser verstehen zu können. Finanziell jedoch
wurde die Kolonie zu einem Misserfolg, was ihm 1707 sogar eine Inhaftierung
\index{Inhaftierung} wegen Überschuldung einbrachte.

\medskip

Seinen politischen Höhepunkt erreichte Penn unter James II.
\index{Personen:!James II.} in der Zeitspanne von 1685 bis 1688. Für den
katholischen König war er am oranischen Hause in den Niederlanden
\index{Orte:!Niederlanden} diplomatisch tätig, um Zustimmung zur Abschaffung der
Test Act \index{Test Act} zu gewinnen. Penns größter Erfolg war die Formulierung
der "`Declaration of Indulgence"' 1687, mit der nicht allein Quäker
staatlicherseits toleriert waren, sondern der Theorie nach sogar alle
Religionsangehörige, nicht jedoch Atheisten \index{Personen:!Atheisten}. Als
hingegen 1688 Wilhelm von Oranien (1650-1702) \index{Personen:!von Oranien,
Wilhelm} intervenierte, wurde Penn als angeblicher
Jesuit\index{Personen:!Jesuiten} inhaftiert. In den folgenden Jahren konnte er
seinen ehemaligen politischen Einfluss am Königshofe \index{Königshof} nicht
wiedererlangen.

\medskip

1699 bis 1701 hielt sich Penn zum zweiten Mal in Pennsylvania auf. Seine letzten
Lebensjahre verbrachte er dann in England, geplagt von den verschiedensten
Krankheiten \index{Krankheit} und körperlichen Verfallserscheinungen. Er erlitt
drei Schlaganfälle \index{Schlaganfall}, nach denen seine Ehefrau die Geschäfte
führte. Verbittert \index{Verbitterung} und vom Leben enttäuscht verstarb er am
30. Juli 1718 in Ruscombe, Berkshire \index{Orte:!Berkshire} und wurde in
Jordans, Buckinghamshire \index{Orte:!Buckinghamshire}, begraben.




