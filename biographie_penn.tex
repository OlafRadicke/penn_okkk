\chapter{Zu dem Autor William Penn} \label{ref:zum_autor_penn}
% Vielleicht eine bessere Überschrift??

\begin{center}
\begin{footnotesize}
\begin{minipage}[c]{7.5cm}
Der folgende Abschitt berut auf den Artikel "`\textit{William Penn}"' des freien
Online-Lexikon \textit{Wikipedia.org}. in der Version vom  19. Aug. 2009
(08:56Uhr). Dieser Abschitt steht unter der Creative Commons
Attribution/Share-Alike Lizenz 3.0 Details:
\\ \texttt{http://creativecommons.org/licenses/by-sa/3.0/deed.de}
\end{minipage}
\end{footnotesize}
\end{center}


\smallskip

Penn wurde als Sohn eines gleichnamigen Admirals geboren, der zu den
reichsten und einflussreichsten Männern Englands gehörte. Nach seiner
Ausbildung in Oxford\index{Orte:!Oxford} studierte er zunächst in Frankreich
\index{Orte:!Frankreich} protestantische Theologie
\index{Theologie!protestantische} und wurde unter anderem am Hof von Ludwig XIV.
\index{Personen:!Ludwig XIV} empfangen. Anschließend studierte
\index{Erkenntnis!Studieren} er in London \index{Orte:!London} Jura\index{Jura}
und begann die umfangreichen Besitzungen seiner Familie in England
\index{Orte:!England} und Irland \index{Orte:!Irland} zu verwalten. In dieser
Zeit kam er in Kontakt mit der Bewegung der Quaker \index{Quaker}, zu deren
bekanntestem und respektiertestem Sprecher er bald wurde. Wegen seines Glaubens
wurde er von seinem Vater \index{Personen:!Vater} verstoßen und in England
mehrmals gefangengenommen \index{Gefangenschaft}. Trotz des Zerwürfnisses mit
dem Vater machte dieser zweimal seinen Einfluss geltend, um die Freilassung
seines Sohnes zu erwirken. Am 16. September 1670 starb der Vater, kurz nachdem
Penn ein zweites Mal freigelassen worden war. Vor seinem Tod versöhnte sich
Vater Penn zu guter Letzt noch mit seinem Sohn. Dieser heiratete \index{Heirat}
1672 die Quäkerin \textit{Guglielma Springett} \index{Personen:!Springett,
Guglielma}, mit der er drei Kinder hatte: Springett,
\textit{William}\index{Personen:!William} und
\textit{Lätitia}\index{Personen:!Lätitia} (auch \textit{Letty} genannt).

\medskip

Der eingeschränkten Religionsfreiheit und mehrmaliger Inhaftierungen zum Trotz
setzte Penn sein Auftreten als Quäker fort und predigte \index{predigen}
religiöse Toleranz \index{Toleranz} und politischen Liberalismus
\index{Liberalismus!politischer}. In den 1670er Jahren entwickelte er ein Modell
für eine neue Siedlung in Nordamerika \index{Orte:!Nordamerika}.

\medskip

Hierzu reiste er 1671 und 1677 unter anderem auch nach Deutschland
\index{Orte:!Deutschland} und warb für die Ansiedlung \index{Siedlung} deutscher
Kolonisten \index{Personen:!Kolonisten} in Nordamerika. Nach dem Tod von Penns
Vater beglich König Karl II.\index{Personen:!König Karl II.} im Jahre 1681 eine
größere Geldschuld, indem er Penn ein riesiges Gebiet in der nordamerikanischen
Wildnis vermachte und ihn zum dortigen Gouverneur ernannte. William Penn wollte
die Kolonie New Wales nennen. Als das von der britischen Krone abgelehnt wurde,
schlug er als Name Sylvania vor. Karl II. verfügte dann den endgültigen Namen
Pennsylvania \index{Orte:!Pennsylvania} als Ehrung für Admiral Sir
\index{Personen:!Penn, Admiral Sir} William Penn. Es umfasste die beiden
heutigen Bundesstaaten Pennsylvania und Delaware \index{Orte:!Delaware}. Noch im
nämlichen Jahr gründete Penn die Hauptstadt Philadelphia
\index{Orte:!Philadelphia}, die damit zu den ältesten Städten der USA
\index{Orte:!USA} zählt.

\medskip

Penn wagte das \textit{"`heilige Experiment"'}, wie er es nannte, und setzte ein
Regierungssystem \index{Regierung!system} in Kraft, das auf Brüderlichkeit und
persönlicher Freiheit für Siedler und Indianer \index{Personen:!Indianer}
beruhte. Unter seinem Einfluss begann 1683 die Ansiedlung deutscher
Siedler\index{Personen:!Siedler!deutsche} in Germantown
\index{Personen:!Germantown}, Pennsylvania, durch seinen Freund Franz Daniel
Pastorius \index{Personen:!Pastorius, Franz Daniel}.

\medskip

Mit seinem ungewöhnlich liberalen Wahlrecht \index{Wahlrecht} und der vollen
Religionsfreiheit \index{Religionsfreiheit} war Penns System seiner Zeit weit
voraus. Aufgrund der Tatsache, dass Penn die Indianer vor Alkohol
\index{Alkohol} und ausbeuterischen Weißen \index{Ausbeutung} schützte und sich
strikt an die Landabtretungsverträge \index{Landabtretungsverträge} hielt, blieb
Pennsylvania von indianischen Überfällen verschont. Penn hatte intensiven
Kontakt mit den benachbarten indianischen Ethnien, wie den \textit{Lenni Lenape}
\index{Personen:!Lenni Lenape} oder den \textit{Irokesen}
\index{Personen:!Irokesen}, und sprach sogar ihre Sprachen.

\medskip

Insgesamt verbrachte Penn nur einige wenige Jahre in Pennsylvania. Die meiste
Zeit setzte er sich in England für sein \textit{heiliges Experiment} ein, das
immer wieder stark gefährdet war. Zudem musste er sich in England um seine
Familie kümmern. Seine Frau Guglielma starb 1693 nach langer Krankheit. Sein
ältester Sohn Springett folgte ihr am 10. Februar 1696 ebenfalls nach
langer Krankheit in den Tod. Noch im gleichen Jahr heiratete Penn ein zweites
Mal. Seine neue Frau, Hannah Callowhill \index{Personen:!Callowhill, Hannah},
schenkte ihm in der Folge sechs Kinder.

\medskip

Obwohl Pennsylvania rasch zu einer reichen Kolonie wurde, verdiente Penn kaum
etwas daran. Von seinem finanziellen Verwalter Philip Ford
\index{Personen:!Ford, Philip} betrogen, verschuldete er sich so stark, dass er
einige Zeit ins Schuldgefängnis \index{Schuldgefängnis} musste. Der nachfolgende
Prozess \index{Gerichtsprozess} gegen Ford kostete ihn viel Kraft. Er erlitt
drei Schlaganfälle \index{Schlaganfall}, die seine Gesundheit sehr schwächten.
Enttäuscht und verbittert \index{Entteuschung}\index{Verbitterung} starb William
Penn im Jahre 1718.

Penns Einfluss in Pennsylvanien war noch über Generationen sehr deutlich
spürbar. Am 28. November 1984 erklärte der US-amerikanische Präsident
Ronald Reagan \index{Personen:!Reagan, Ronald} William Penn und seine zweite
Frau, Hannah Callowhill, postum zu Ehrenbürgern \index{Ehrenbürgern} der USA.

\medskip

Ein von Freundschaft und Tod
handelndes Zitat aus seinem Werk \textit{"`More Fruits of Solitude"'} (englisch:
\textit{"`Weitere Früchte der Einsamkeit"'}) ist eines der beiden Mottos des
siebten Bandes der \textit{Harry-Potter}-Reihe.

\medskip

\section{Literarisches Wirken}

Zahlreiche Schriften Penns wurden ins Deutsche \index{Sprache!Deutsch} übersetzt
und erzielten hohe Auflagen. Penn war damit in Deutschland populärer als George
Fox \index{Personen:!Fox, George}, der weithin als der Gründer des Quakertums
gilt \footnote{Claus Bernet (Hrsg.): Deutsche \textit{Quäkerschriften}. Bd. 2
(18. Jahrhundert), Georg Olms Verlag, 2007, ISBN 978-3-48713408-6, S. 393.}.
Auch fand Penns Werk mehr Anklang als die sehr bekannte \textit{"`Apologie"'}
\index{Apologie} von Robert Barclay \index{Personen:!Barclay, Robert}.

\medskip

Zu den schon früh übersetzten Werken gehörten:
\begin{itemize}
 \item \textit{Forderung der Christenheit vors Gericht}
 \item \textit{Eine Freundliche Heimsuchung in der Liebe Gottes, welche die Welt
überwindet}
\item \textit{Kurze Nachricht von der Entstehung und dem Fortgang der
christlichen
Gesellschaft der Freunde die man Quäker nennt}
\item \textit{Zärtlicher Rath und Zärtlicher Besuch in der Liebe Gottes}
\item \textit{Wiederherstellung des ersten Christentums}
\item \textit{Schlüssel zu den Grundsätzen}
\item \textit{Früchte der Einsamkeit}
\end{itemize}

\medskip

Das letzt genanntes Werk war, mit knapp 240 Seiten, lange Zeit das
umfangreichste . Die "`Kurze Nachricht"' fand sogar eine öffentliche Rezension,
die aber vernichtend ausfiel \footnote{Claus Bernet (Hrsg.): Deutsche
Quäkerschriften. Bd. 2 (18. Jahrhundert), Georg Olms Verlag, 2007, ISBN
978-3-48713408-6, S. 393.}.

\medskip

Während des Aufenthalts 1668 im \textit{Tower zu London} \index{Orte:!Tower zu
London} verfasst er sein Werk \textit{"`Ohne Kreuz keine Krone"'} (engl.
\textit{No Cross No Crown}). Dieses sprachgewaltige Werk wurde schon früh in
verschiedene Sprachen übersetzt. Aber zunächst noch nicht ins Deutsche, trotz
großen Interesses aus Deutschland.

\section{Essay towards the Present and Future Peace of Europe}


William Penns Plan einer europäischen Einigung entstand in London
\index{Orte:!London} während der politisch unsicheren und gespannten Phase der
Jahre 1691 bis 1693. Penns Essay erreichte eine viermalige Auflage (1693, 1693,
1696, 1702).

\medskip

Penn schilderte im Essay, warum er sich entschlossen hatte, für die
Etablierung und Sicherung des Friedens \index{Frieden} in Europa
\index{Orte:!Europa} einzutreten. Friede bedeutete in den Augen Penns Sicherung
des Besitzstandes, freier Handel, Ansiedlung von Industrie, allgemeiner
wirtschaftlicher Aufschwung durch eine verbesserte Auftragslage sowie eine
Förderung der allgemeinen Wohlfahrt \index{Wohlfahrt} und Gastlichkeit während
der Krieg \index{Krieg} neben Tod, Gräuel und Verelendung vor allem die Gier
\index{Gier} und Hamsterei der Wohlhabenden verstärkt, den Armen
\index{Personen:!Arme} ein Leben als Soldat \index{Personen:!Soldaten} oder Dieb
\index{Personen:!Diebe} aufzwingt und keinerlei volkswirtschaftlichen
\index{Volkswirtschaft} Nutzen nach sich zieht. Die Gerechtigkeit
\index{Gerechtigkeit} sah er als Weggefährtin des Friedens an, die zwischen den
Parteien zur Vermittlung benötigt wurde und in Form der Gesetzgebung
\index{Gesetzgebung} schließlich Rechte und Pflichten definierte und bewahrte.
Die Regierung \index{Regierung} stellte darüber hinaus ein notwendiges Mittel
gegen die Verworrenheit dar und ging aus einem allgemeinen gesellschaftlichen
Konsens hervor.

\medskip

Des Weiteren beschäftigte Penn sich mit der institutionellen Einrichtung eines
Parlaments \index{Parlament}, Staatenbundes \index{Staatenbund} oder
-versammlung auf europäischer Ebene, wobei die verwendeten Termini
(\textit{Parliament of Europe}, \textit{States of Europe}, \textit{European
Confederacy}, \textit{European League}, \textit{General Diet Estates},
\textit{Imperial Parliament}) oft einen recht beliebigen Austausch fanden. Sie
thematisieren primär die möglichen Mitglieder und deren Aufnahmeberechtigung,
die Natur von Rechtstiteln und erörterten Fragen der Präsidentschaft und der
Repräsentation, des Stimm- und Wahlverhaltens, der Anwesenheits- und
Vertretungspflicht sowie mögliche Straf- und Sanktionsmaßnahmen. Als
Diskussions-, Verhandlungs- und Vertragssprache schlug Penn entweder Latein
\index{Sprache!Latein} oder Französisch \index{Sprache!Französisch} vor, die
beide Vorteile besäßen und gleichermaßen akzeptiert werden könnten.

\medskip

William Penn verwies zur Rechtfertigung seines Friedensplanes
\index{Friedensplan} explizit auf die blutigen Geschehnisse, die sich seit 1688
beispielsweise in Ungarn \index{Orte:!Ungarn}, Flandern \index{Orte:!Flandern},
England \index{Orte:!England}, Irland \index{Orte:!Irland} oder zur See
abgespielt hätten und von denen kein menschliches Wesen ungerührt sein könne.
Zur künftigen Verhinderung solcher Tragödien sei ein europaweiter Frieden
vonnöten, doch lieferte Penn keine generelle Definition dessen, was
\textit{Europa} für ihn darstellte.

\medskip

Penn entschied sich für einen politisch-geographischen Begriff Europas, der sich
besonders aus der Zusammensetzung einer General- oder Staatenversammlung sowie
der Stimmverteilung der anwesenden Delegierten ablesen ließ. Demzufolge sollten
die Stimmen wie folgt verteilt werden...

\begin{footnotesize}
\begin{center}
\begin{tabular}{|l|r|} \hline
\textbf{Land }                      & \textbf{Stimmen} \\ \hline \hline
Heilige Römische Reich              & 12               \\ \hline
Frankreich                          & 10               \\ \hline
Spanien                             & 10               \\ \hline
Italien (ohne die Republik Venedig) & 8                \\ \hline
England                             & 6                \\ \hline
Schweden                            & 4                \\ \hline
Polen                               & 4                \\ \hline
Generalstaaten                      & 4                \\ \hline
Schweizer Kantone                   & 1                \\ \hline
Herzogtum Holstein                  & 1                \\ \hline
Kurland                             & 1                \\ \hline
kleinere Fürstentümer               & 1                \\ \hline
\end{tabular}
\end{center}
\end{footnotesize}

\index{Orte:!Heilige Römische Reich} \index{Orte:!Frankreich}
\index{Orte:!Spanien} \index{Orte:!Italien} \index{Orte:!England}
\index{Orte:!Schweden} \index{Orte:!Polen} \index{Orte:!Generalstaaten}
\index{Orte:!Schweiz} \index{Orte:!Holstein} \index{Orte:!Türkei}
\index{Orte:!Rusland} Die Gesamtsumme der Stimmen sollten sich auf 70 belaufen.
Für den (gerechten) Fall, dass die Türken und Russen (\textit{"`Muscovites"'})
ebenfalls als Mitglieder Aufnahme gefunden hätten, hätten sie jeweils über 10
Delegierte verfügt, was die Stimmenanzahl auf 90 erhöht hätte. Eine solche
Zusammenkunft hätte nach seiner Ansicht \textit{"`Europa"'} als den Teil der
Welt hervorgehoben, der als \textit{"`the Best and wealthyest part of the known
World"'} bekannt wäre und wo \textit{"`Religion and Learning, Civility and Arts
have
their Seat and Empire."'}

\medskip

Für den Fall, dass veränderte Rechtstitel eine Überarbeitung der Zusammensetzung
erforderlich gemacht hätten, schlug Penn eine genaue Prüfung aller vorgebrachten
Ansprüche vor. Rechtstitel hätten vor der General- oder Staatenversammlung nur
dann Bestand gehabt, wenn sie durch eine unzweifelhafte Erbfolge (England,
Frankreich, Spanien), eine Wahl (Kaiserwahl im Reich, polnisches Wahlkönigtum),
eine Heirat (Stuartdynastie in England, Erwerb des Herzogtums
Kleve durch den brandenburgischen Kurfürsten) oder durch einen rechtmäßigen
Erwerb (Beispiele aus dem Reich und Italien) legitimiert gewesen seien.
Dahingegen hätte die zu schaffende europäische Staatenversammlung solche
Ansprüche und Titel zurückweisen sollen, die durch Schwert und Blut (türkische
Eroberungen christlicher Gebiete, Eroberung Flanderns \index{Orte:!Flandern}
durch Spanien, Eroberung Burgunds \index{Orte:!Burgund}, der Normandie und
Lothringens \index{Orte:!Lothring} durch Frankreich) erreicht worden waren, da
sie moralisch fragwürdig seien, wenn sie nicht nachträglich durch einen Vertrag
eine Bestätigung gefunden hätten. In diesem Punkt wurde der Wunsch nach einer
friedlichen beziehungsweise an christlichen Maßstäben orientierten Rechts- und
Wertegemeinschaft zwar besonders gut sichtbar, doch zog er sich durch das ganze
Essay. Die Staatengemeinschaft führte somit \textit{"`to the benefits of an
Universal Monarchy, without the Inconveniences that attend it"'}.

\medskip

In den letzten Abschnitten wurden mögliche Einwände vorweggenommen, entkräftet,
und die positiven Auswirkungen des Planes wie etwa ein gottgefälliges Leben in
christlicher Nächstenliebe und Frieden (\textit{"`Pacifick means"'}), das
verbesserte Ansehen des Christentums \index{Christentum} bei den Ungläubigen
\index{Personen:!Ungläubige}, die Vermeidung der oft enormen
Kriegsausgaben\index{Krieg!Ausgaben} und ihrer friedlichen Verwendung, die
Erleichterung von Handel \index{Handel} und Reisen \index{Reisen}, die
verstärkte Abwehrkraft gegen türkische Expansionsbestrebungen
\index{Expansionsbestrebungen!türkische} und die intensivere Freundschaft
zwischen allen europäischen Fürsten, Staaten und Völkern, aufgezählt.

\medskip

Abschließend wies Penn darauf hin, dass die Schaffung einer solchen -- sehr
eingeschränkt supranational agierenden -- Institution durchaus nicht unnatürlich
sei, weil \textit{"`the same Rules of Justice and Prudence, by which Parents and
Masters Govern their Families, and Magistrates their Cities, and Estates their
Republicks, and Princes and Kings their Principalities and Kingdoms, Europe may
Obtain and Preserve peace among the Soveraignties."'} Die besten Anregungen für
seinen Friedensplan fand er nach eigenen Angaben bei seinen Recherchen über den
französischen König Heinrich IV. \index{Personen:!Heinrich IV}, der bereits
viele Gedankengänge rund 100 Jahre vor ihm angestellt hätte. Penn bezeichnet
seine Ausführungen deshalb als ein Experiment, das nötig sei, um den Frieden in
Europa und der ganzen Welt dauerhaft einzurichten. Ideengeschichtlich markierte
William Penns Essay einen Meilenstein in der Geschichte der
\textit{"`Europapläne"'}, der -- auch nach eigenen Angaben -- eine enge
Verknüpfung zu den Plänen von Heinrich IV \index{Personen:!Heinrich IV} und
dessen Weggefährten Maximilien de Béthune, duc de Sully \index{Personen:!de
Béthune, Maximilien} aufweist und erstaunlich detailliert zahlreiche
Errungenschaften des Nachkriegseuropas des 20. Jahrhunderts vorwegnahm.


