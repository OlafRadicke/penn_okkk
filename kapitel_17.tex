% STATUS=Unfertig
% FUSSNOTEN=2.Durchlauf
% ANFÜHRUNGSZEICHEN=ok
% KORSIVSCHRIFT=ok
% FETTSCHRIFT=ok
% RANDBEMERKUNGEN=weg
% MODERNE_RECHTSCHREIBUNG=noch nicht begonnen
% WORTERSÄTZUNG=ok
% VERSCHLAGWORTUNG(INDEX)=ok
% REFERENZIERUNG=abgeschlossen
% BIBELSTELLEN=ungeprüft
% EDITORFUSSNOTEN=ok

\chapter{17. Kapitel} \label{kap17}
\section{Zusammenfassung des 17. Kapitel}
\small
\begin{description}
\item[1. Abschnitt] Die Gebräuche, Moden, Unterhaltungen, etc. welche die
anziehende Pracht und das Vergnügen der jetzigen Welt ausmacht, verhindern die
Menschen, in sich selbst einzukehren. (Seite: \pageref{kap17_ab1})
\item[2. Abschnitt] Ihr Zweck ist bloß Befriedigung der Sinnlichkeit. (Seite:
\pageref{kap17_ab2})
\item[3. Abschnitt] Gewährten sie wahre Freuden, so währe Adam und Eva, die sie
nicht kannten, nicht vollkommen glücklich gewesen. (Seite: \pageref{kap17_ab3})
\item[4. Abschnitt] Die Dreistigkeit und Vermessenheit, womit sogenannte
Christen sich ihnen hingeben, ist scheußlich. (Seite: \pageref{kap17_ab4})
\item[5. Abschnitt] Zu wissen, das sie eine Erfindung leichtsinniger und eitler
Menschen sind, ist schon ein hinreichender Grund, sie zu verwerfen. (Seite:
\pageref{kap17_ab5})
\item[6. Abschnitt] Sie sind größtentheils von den Heiden entlehnt, die Gott
nicht kannten. (Seite: \pageref{kap17_ab6})
\item[7. Abschnitt] Eine Einwendung die man hinsichtlich ihrer Nuzbarkeit
vorbringt, wird beleuchtet und beantwortet, und ihre Vertheidiger werden
bestraft. (Seite: \pageref{kap17_ab7})
\item[8. Abschnitt] Die aufgeklärten Heiden verabscheueten, was die vorgeblichen
Christen vertheidigen. (Seite: \pageref{kap17_ab8})
\item[9. Abschnitt] Die Aufnahme, welche dergleichen Thorheiten finden, dient
den Erfindern derselben zur Aufmunterung, damit fortzufahren (Seite:
\pageref{kap17_ab9})
\item[10. Abschnitt] Beantwortung der Einwendungen, daß viele Familien durch
solche Dinge ihre Nahrung finden. -- Man mus nicht Böses thun, damit Guten
daraus entstehe. Es lassen sich bessere Beschäftigungen auffinden, die der
Menschheit mehr Nutzen bringen können. (Seite: \pageref{kap17_ab10})
\item[11. Abschnitt] Beantwortung einer andern Einwendung. Gott ist nicht der
Urheber solcher eitlen Erfindungen; daher lassen sie sich auch nicht als von
inne herrührende Einrichtungen entschuldigen. (Seite: \pageref{kap17_ab11})
\item[12. Abschnitt] Wer solche Eitelkeiten in Schulz nimmt, zeigt, was er ist.
-- Ermahnung an alle Vernünftigdenkende und Ernsthaftgesinnte. -- Es ist ein
nothwendiger Schritt, zu der wahren Nachfolge Jesu zu gelangen, daß man die
Schule der Maßlosigkeit
\footnote{\texttt{'Üppigkeit' erstzt durch 'Maßlosigkeit'}}
verlasse. (Seite: \pageref{kap17_ab12})

\end{description}
\normalsize

\section{1. Abschnitt} \label{kap17_ab1}

Endlich sind jene \footnote{\texttt{'üppigen' ersetzt durch 'maßlosen'}}
maßlose Moden, Gebräuche und Belustigungen, welche die
anziehende Pracht und beständige Unterhaltung des gegenwärtigen Zeitalters
ausmachen, ein mächtiges Hindernis; der innern stillen Sammlung\index{stillen
Sammlung}\index{stiller Andacht} oder Einkehr des
Gemüths, wodurch der Mensch zum Anschauen der Herrlichkeit ewiger
Unsterblichkeit\index{Unsterblichkeit} gelangen kann.
\textbf{\index{Erziehung}Denn, anstatt daß die Menschen angewiesen werden
sollten,
\textit{"`in ihrer Jugend an ihren Schöpfer zu denken, und zuerst nach dem
Reiche Gottes\index{Reiche Gottes} zu trachten, alles übrige zur Nothdurft
Gehürige aber, nach dem
Befehle Gottes und unsers Herrn Jesu Christi, als eine Zugabe zu
erwarten;"'}
\footnote{Weisheit Salomos 12,1) Lukas  12,29-31}
\index{Bibelstellen:!Weisheit Salomos 12)}
\index{Bibelstellen:!Lukas  12)}
werden sie, sobald sie nur
etwas thun können, von diesen Erfindungen des Stolzes und der Maßlosigkeit
\footnote{\texttt{'Üppigkeit' erstzt durch 'Maßlosigkeit'}}
angezogen, und zu solchen Unterhaltungen angeleitet, die der Sinnlichkeit am
mehrsten behagen, und welche dann hernach die Gegenstände ihres höchsten
Vergnügens werden. Auf diese Weise werden offenbar unerlaubte Begierden erzeugt
und maßlose
\footnote{\texttt{'üppige' ersetzt durch 'maßlose'}} Gedanken erweckt, die zu
leichtsinnigen Gesprächen, unzüchtigen
Spielen und ausgelassenen, rauschenden Freuden, wo nicht endlich zu gottlosen
Handlungen führen.} Für die Menschen, welche von einer so ausschweifenden
Lebensweise hingerissen sind, ist es langweilig und anstößig, vom Himmel oder
von einem künftigen Leben zuhören. Sage ihnen, daß sie über ihre Handlungen
nachdenken, den heiligen Geist\index{heiliger Geist} nicht betrüben, ihr ewiges
Loos beherzigen oder
sich auf den Tag des gerechten Gerichtes Gottes\index{Gericht Gottes}
vorbereiten möchten; so bestehet
ihre Antwort, -- wenn sie nicht grob ausfahren -- gewöhnlich in spöttischen
Scherzreden und leichtsinnigen Bemerkungen. Es sind ganz andere Gegenstände die
ihre Gedanken besckäftigen. Ihre Morgenstunden reichen kaum zu, ihren Anzug zu
besorgen, ihren Schmuck anzulegen, und ihren ganzen Puz in Ordnung zu bringen;
und ihre Nachmittage sind gewöhnlich Besuchen vorbehalten und
\index{Schauspiel}\index{Theater}\textbf{dem Schauspiele
gewidmet, wo einige aus den beliebtesten Romanen entlehnte Scenen zu ihrer
Unterhaltung dienen. Da giebt es dann eine Menge Abwechselungen: seltsame
Abenteuere; merkwürdige Liebesgeschichten; grausame Weigerungen,
unübersteigliche Hindernisse; zudringliche Besuche; unglückliche Täuschungen;
wunderbare Ueberraschungen; unerwartetes Zusammentreffen; überfallene Schlösser;
gefangene und befreiete Liebende; plötzliche Erscheinungen todt geglaubter
Personen; blutige Zweikämpfe; schmachtende Stimmen, die aus einsamen Grotten
hervorhallen; leise vernommene Klagen; tiefe Seufzer, die sich aus wüsten
Oertern vernehmen lassen; heimliche Ranke, die mit unerhörter Feinheit
geschmiedet werden; und endlich, wenn Alles verloren zu gehn scheint, läßt man
Todte wieder lebendig, Feinde wieder Freunde werden. Die Verzweiflung verwandelt
sich in Entzücken, und alle Unmöglichkeiten gleichen sich aus. Da tragen sich
dann Dinge zu, die nie gewesen sind, die auch jetzt nicht sind, und nie sein
werden, noch jemals sein können.} Und als wenn die Menschen zu langsam, oder
nicht bereitwillig genug wären, den zügellosen Neigungen ihrer verderbten
Naturen zu folgen, oder, als wenn Gefahr da wäre, daß ihre Gemüther sich zu sehr
mit gottseligen Betrachtungen und himmlischen Gegenständen beschäftigen möchten,
werden alle Kräfte des Witzes und der Ersindsamkeit ausgeboten, und nicht allein
offenbare Lügen, sondern in der Natur ganz unmögliche Dinge vorgestellt, um
schädliche Leidenschaften in ihren Herzen aufzuregen und ihre schwindelnde
Einbildung mit schwellenden Bildern der Sinnlichkeit zu erhitzen. \textbf{So
wird dann
nicht allein ihr Zeit verschwendet, ihre Natur verweichlicht und ihre Vernunft
entehrt, sondern auch nicht selten bei ihnen der Gedanke erzeugt, solche Dinge
in der Wirklichkeit auszuführen und daß eine oder andere Abenteuer nachzuahmen.
Will dieses aber nicht gelingen, -- wie sich denn von bloßen Hirngespinsten
nicht anders erwarten läßt, so scheint dem Abenteurer kein anderes Mittel übrig
zu bleiben, als sich den lasterhaftesten Ausschweifungen in die Arme zu werfen.}
Dieses sind einige ihrer unschuldigsten Erhohlungen, die jedoch jeder wahre
Christ nicht anders als gefährliche Schlingen des Feindes der menschlichen
Glückseligkeit betrachtet, in welchen der Versucher die Gemüther der Unwachsamen
um so leichter fangen und zu sich ziehen kann, da sie den natürlichen
Schwachheiten der Menschen so angemessen sind, und sich ganz unmerklich ihrer
Neigungen durch solche Dinge bemeistern, die den stärksten Eindruck aus ihre
Sinnlichkeit machen und sie mit einer unwiderstehlich scheinenden Gewalt
hinreißen. \textbf{Bei solchen Gelegenheiten geschiehet es, daß ihre Seelen
Eitelkeit
brüten, ihre Augen die Dolmetscher ihrer Gedanken werden, und ihre Blicke die
geheimen Flammen ihrer ausschweifenden Seelen verrathen, die so lange
umherirren, bis die Nacht ihre Uebelthaten bedeckt, welche ihr Gewissen mit
Schuld und ihren Ruf mit Schande beflecken.
\footnote{Sprüche 7,16-31}}
\index{Bibelstellen:!Sprüche 7)}

\section{2. Abschnitt} \label{kap17_ab2}

Hier sehen wir, daß der Zweck aller weltlichen Moden und Erhohlungen kein
anderer ist, als die sinnlichen Begierden des Menschen:
\textit{"`die Augenlust, die
Fleischeslust, und den Hochmuth des Lebens zu befriedigen"'}
\footnote{1 Johannes 2,15+16}
\index{Bibelstellen:!1 Johannes 2)}
Und darin hat man es so weit getrieben, daß in der That die Kleidung, die
einst bloß zur Bedeckung der Blöße eingeführt ward, jetzt wohl noch einer
Bedekung bedürfte, um ihre schamlose Pracht zu verhüllen, indem der Mensch Das,
was ihn an den Verlust seiner Unschuld erinnern sollte, als ein Mittel zur
Befriedigung seines Stolzes und seiner Prachtliede gebraucht. \textbf{Gewiß
schon der
hundertste Theil der Dinge, weiche jetzt das angenehmste Vergnügen der Menschen
ausmachen, ja, in unsern Tagen durchaus zum guten Tone gehören, würden unsern
ersten Eltern den Verlust des Paradieses gekostet haben. Denn so wie
Adams\index{Personen:!Adam}
Uebertretung darin bestand, daß er einen andern Genuß suchte, als Gott ihm
angewiesen hatte, so ist es auch der Fehler der jetzt lebenden Menschen, daß sie
unerlaubten Vergnügungen nachgehen und den größten Theil ihrer Zeit mit eitlen
Dingen zubringen, die so weit entfernt sind, dem Zwecke ihres Daseins, einem
Gott wohlgefälligen Leben, zu entsprechen, daß sie vielmehr sehr nachtheilig und
zerstörend auf dasselbe wirken.}

\section{3. Abschnitt} \label{kap17_ab3}

Wären die Freuden der Menschen unserer Zeit ächtet und wahrer Art, so würden
Adam\index{Personen:!Adam} und Eva\index{Personen:!Eva} in dem Stande ihrer
Unschuld nicht glücklich gewesen sern; da sie
dieselben nicht kannten. Allein es machte vielmehr einen großen Theil ihrer
Glückseligkeit aus, sie nicht zu kennen; und eben so bestehet auch ein hoher
Grad der Seligkeit Derer, die Christum wirklich kennen, darin, daß sie durch
seine ewige Kraft von jenen Thorheiten erlöset und zur Liebe eines
unvergänglichen Lebens erwecket sind. Dieses ist noch Denen ein Geheimniß, deren
Lebensgenuß und höchstes Vergnügen in ausgesuchten prächtigen Kleidern; in
Schmuck\index{Schmuck} und Putz; in Erfindungen\index{Erfindungen} und
Nachahmungen neuer Moden, gezierten
Stellungen, Biegungen, Haltungen und Bewegungen des Körpers und der lüsternen
Blicke und der Wendungen der Augen; in Romanlesen\index{Romanlesen}, Besuchen,
Spielen\index{Spiele},
Lustpartien, Schauspielen\index{Schauspiele}, Bällen\index{Bällen},
Festen\index{Feste}, Gastmählern und andern so genannten
Erhohlungen bestehet. Denn, da alle dergleichen Dinge nie erstanden sein würden,
wenn der Mensch seinen Schöpfer nicht verlassen und sein Gemüth nur mit den
edlen Zwecken seiner Erschaffung beschäftigt hätte; so leuchtet es klar ein, daß
Diejenigen, welche solchen Eitelkeiten ergeben sind, die höheren himmlischen
Freuden, den Genuß des göttlichen Friedens und der wahren Ruhe der Seele, noch
nicht kennen, weil eben diese Dinge sie von der stillen Einkehr in ihr Inneres
und vom ernsten Trachten nach ewigen Gütern beständig abhalten. Ach! mit was für
Mühe, Anstrengung und Unruhe, mit welchem Erfindungsgeiste und Kunstfleiße, und
mit wie vielem Zeit- und Kosienaufwande ist der Mensch so beständig bestrebt,
für die vorübergehenden Genüsse seiner Sinne zu sorgen, und wie wenig ziehet er
seine unsterbliche Seele, das Bild der Gottheit, in Betrachtung! Gewiß, es
bedarf keiner stärkerer Beweise, keiner deutlichern Merkmale, um die Menschen zu
überführen, daß es nur der sinnliche Leib, ein mit Fleisch und Haut überzogenes
Knochengerippe ist, wofür sie so viel Tand\index{Tand} und
Flitterstaat\index{Flitterstaat} anschaffen; und daß
es die nichtigsten Thorheiten und Eitelkeiten sind, die ihre Gemüther so sehr
eingenommen genommen haben, daß sie nicht Liebe, Zeit und Geld genug darauf
verwenden zu können glauben.

\section{4. Abschnitt} \label{kap17_ab4}

Auf diese Weise sind ihre Gemüther unaufhörlich beschäftigt; und dabei sind sie
so sehr von sich selbst eingenommen, oder vielmehr so verfinstert in ihrem
Verstande, daß sie nicht allein alle jene eitlen Thorheiten für ganz unschuldig
halten, sondern sich auch einbilden, sie könnten bei dem Gebrauche der selben
nichtsdestoweniger gute Christen\index{Christen, gute} sein. Ihnen Vorwürfe
darüber zu machen, wurde
die ärgste Ketzerei\index{Ketzerei} sein. \textbf{So entfernt sind sie von dem
innern göttlichen Leben;
-- da ihre immerwährenden Zerstreuungen ihnen keine ernste
Selbstprüfung\index{Selbstprüfung}
erlauben, -- daß sie bei ihrem Gottesdienste\index{Gottesdienst}\index{Gebet}
sich damit begnügen, mit einem
erzwungenen Eifer eine halbe Stunde lang die Worte eines Andern herzusagen,
womit sie auch weiter keinen Sinn verbinden. Denn das, was sie sagen, hat so
wenig Beziehung auf ihren eigenen Zustand, oder sie haben, -- wie ihre
Handlungen beweisen, -- eben so wenig die Absicht darnach zu thun, als jener
Jüngling im Evangelio, weicher sagte,
\textit{"`er wolle gehen, und doch nicht ging."'}}
Aber ach! warum thun sie es nicht? -- Sie sind mit andern Gästen beschäftigt.
Und wer sind diese? Pharamund\index{Personen:!Pharamund},
Cleopatra\index{Personen:!Cleopatra}, Cassandra\index{Personen:!Cassandra},
Clelia\index{Personen:!Clelia}, ein Schauspiel, ein
Ball, ein Lustgarten, ein Park, ein Verehrer, die Börse\index{Börse}, mit einem
Worte, \textbf{die
Welt.}\index{Welt (Befrifflichkeit)}Diese hält sie ab, diese erwartet, ruft,
sucht, quält sie; und da müssen
sie ihr ja Gehör geben, und könnten sich ihrer Gesellschaft entziehen. So werden
ihre Herzen gefangen genommen und von der Betrachtung göttlicher Dinge, ja, oft
selbst von der Wahrnehmung solcher äußern Angelegenheiten abgehalten, die mit
ihrem eigenen Vortheile oder mit der Wohlfahrt ihres Nächsten in unmittelbarer
Verbindung stehen. \textbf{Sie finden nur Geschmack an den Vorstellungen, die
jene
thörichten Tändeleien ihren Gemüthern einflößen, und wenn sie diese auch nicht
alle mitmachen, weil es ihnen an den Mitteln dazu fehlt, so sind sie doch so
sehr davon eingenommen, daß sie mit Vergnügen ihren Gedanken freien Spielraum
lassen, ihnen beständig nachzuhängen, Dadurch werden dann natürlich ihre
Gemüther gänzlich abgeneigt gemacht, das göttliche Leben und die heiligen Lehren
Jesu zu betrachten.} Vornehmlich ist dieses ader, -- wie schon öfter bemerkt
ist,
-- der Fall mit jungen Gemüthern, denen solche Vergnügungen, die ihre Sinne mit
neuen, ihren Neigungen angemessenen Reizen ansprechen, und nie zuvor gekannte
Gefühle der Eitelkeit in ihnen erwecken, viel lieber und schätzbarer sind, als
Alles, was ihnen von der Furcht Gottes, von einem eingekehrten gottseligen
Leben, von ewigen Belohnungen und unaussprechlich herrlichen Freuden gesagt
werden kann. So sehr kann die Eitelkeit die Menschen verblenden, und so
unempfindlich macht dieselbe sie gegen Alles was zur wahren Nachfolge Christi
gehört! O! Möchten sie es doch ernstlich erwägen! Möchten sie,
\textit{"`um der Zukunft
den Herrn"'} willen, gegen die eitlen Thorheiten der Welt auf ihrer Hut stehen
und ihnen gänzlich entsagen! damit sie nicht, unvorbereitet und mit fremden
Gästen beschäftigt, von seiner ewigen Ruhe ausgeschlossen würden.

\section{5. Abschnitt} \label{kap17_ab5}

War ferner noch die Unerlaubtheit der zahlreichen Moden, Gebräuche und
Ergötzungen der Welt klar beweiset, ist ihr Ursprung und ihr Zweck; da sie
entweder von eitlen, müßigen und ausschweifenden Menschen erfunden werden, die
mit ihrer Einführung keinen andern Zweck Verbinden, als ihre eigene Sinnlichkeit
zu befriedigen und Andere zur Nachahmung solcher strafbaren Neuheiten zu reizen,
die zur Beförderung der Maßlosigkeit
\footnote{\texttt{'Üppigkeit' erstzt durch 'Maßlosigkeit'}} und Thorheit dienen,
oder bloß erzwungene
Produkte verarmter Witzlinge sind, die zu solchen Erfindungen und Erdichtungen
ihre Zuflucht nehmen, um sich ihren Unterhalt zu verschaffen. In beiden Fällen
aber verdienen sie Verabscheuung. Denn in der ersten Hinsicht bahnen sie den Weg
zu offenbaren Lastern, und in der andern unterstützen sie einen schändlichen
Broderwerb\index{Beruf, schändlicher}, und halten fähige Menschen von erlaubten,
nützlichen und
nothwendigen Beschäftigungen zurück.

\medskip

Daß die jezt in der Welt üblichen Moden, Gebrauche und Ergötzungen im Anfange
nicht bekannt waren, sondern eine Erfindung eitler Menschen neuerer Zeiten sind,
wird uns leicht von selbst einleuchten, wenn wir erwägen, wie
Adam\index{Personen:!Adam} und Eva\index{Personen:!Eva}
gekleidet waren, denen, wie wir lesen, Gott selbst Kleider von Thierfellen
machte, und wenn wir in der Schrift nachsuchen, was von allen diesen Eitelkeiten
und Thorheiten unter den heiligen Männern\index{Personen:!Männer, heiligen} und
Weibern\index{Personen:!Weiber, heilige} der Vorzeit anzutreffen
war. Ich möchte wohl fragen, wieviel Band, was für Federn, Spitzen und andere
Zierrathen Adam und Eva im Paradiese, und auch nach ihrer Vertreibung aus
demselben, an sich trugen? Und was für reiche, prachtvoll gestickte und besetzte
Kleider hatten Abel\index{Personen:!Abel}, Enoch\index{Personen:!Enoch},
Noah\index{Personen:!Noah} und der gute alte Abraham\index{Personen:!Abraham}?
Pflegten Eva, Sara\index{Personen:!Sara},
Susanna\index{Personen:!Susanna}, Elisabeth\index{Personen:!Elisabeth} und die
Jungfrau Maria\index{Personen:!Jungfrau Maria} sich zu frisiren, zu pudern, zu
schminken? Trugen sie schönfarbige falsche Locken, kostbare Spitzen, künstlich
gezierte Hauben, gestickte Kleider\index{Kleider, gestickte} mit langen
Schleppen? Schmückten sie ihren
Leib mit vielen Ellen oder ganzen Stücken Band, und ihre Schuh mit goldenen und
silbernen Flittern, Schleifen u.s.w.? Bei welchen Schauspielen oder andern
Ergötzungen waren Jesus und seine Jünger zugegen, um Erhohlun zu suchen? Was für
Mährchen\index{Mährchen}, Romane\index{Romane}, Komödien\index{Komödien}, und
ähnliche Werke haben die Apostel\index{Personen:!Apostel} und Heiligen
geschrieben, oder gelesen, um sich die Zeit zu vertreiben? -- So viel ich weiß,
ermahnten sie Alle und Jeden,
\textit{"`die Zeit zu erkaufen, und schändliche Worte,
Possen, Scherz, leeres Geschwätz, erdichtete Erzählungen, u. dgl. zu meiden,"'}
weil diese Dinge zu einem ungottseligen Leben führen. Sie empfohlen vielmehr
allen Menschen,
\textit{"`zu wachen; ihre Seligkeit mit Furcht und Zittern zu schaffen;
die thörichten Lüste der Jugend zu fliehen; der Gerechtigkeit, dem Frieden, der
Sampftmuth der Liebe und Barmherzigkeit nachzustreben, und nach Dem, was droben
im Himmel ist, zu trachten, wenn sie Ehre, Ruhm, Unsterblichen Wesen und ewiges
Leben erlangen wollten."'}
\footnote{Epheser 5,1-5+15+16)
2. Timotheus 2,16+22)
Matthäus 25,13)
Philipper 2,12+13)
Kolosser 3,1+2+5.
Römer 2,6+7.}
\index{Bibelstellen:!Epheser 5)}
\index{Bibelstellen:!2. Timotheus 2)}
\index{Bibelstellen:!Matthäus 25)}
\index{Bibelstellen:!Philipper 2)}
\index{Bibelstellen:!Philipper 2)}
\index{Bibelstellen:!Kolosser 3)}
\index{Bibelstellen:!Römer 2)}

\section{6. Abschnitt} \label{kap17_ab6}

Fragt man mich, woher alle jene Thorheiten zuerst kamen? so bin ich bereit zu
antworten: Sie entsprangen zuerst unter solchen Heiden, die Gott gar nicht
kannten; denn Einige derselben verabscheueten sie, wie wir hernach hören werden.
Sie gehörten zu den Vergnügungen eines wollüstigen
Sardanapals\index{Personen:!}, eines
phantastischen Mirakles\index{Personen:!Mirakles}, eines komischen
Aristophanes\index{Personen:!Aristophanes}, eines verschwenderischen
Chararus\index{Personen:!Chararus}, eines maßlosen
\footnote{\texttt{'üppigen' ersetzt durch 'maßlosen'}}
Aristippus\index{Personen:!Aristippus} und zu den Gebräuchen solcher Weiber, wie
die
schändliche Clytemnestra\index{Personen:!Clytemnestra}, die geschminkte
Isabel\index{Personen:!Isabel}, die unzüchtige
Campaspe\index{Personen:!Campaspe}, die
freche Posthumia\index{Personen:!}, die berüchtigte Lais von
Korinth\index{Personen:!}, die unverschämte Flora\index{Personen:!}, die
prachtliebende ägyptische Kleopatra\index{Personen:!}, und die schamlose
Messalina\index{Personen:!}. Männer und
Weiber wie diese, die, mit unauslöschlicher Schande gebrandmarkt, einen bösen
Geruch durch alle Zeitalter verbreiret haben; aber nicht die heiligen, sich
selbst überwindenen
\footnote{\texttt{'verleugnenen' ersetzt durch 'überwindenen'}} Männer und
Weiber der Vorzeit, waren solchen eitlen
Belustigungen ergeben. Ja, selbst die aufgeklärten
Heiden\index{Personen:!Heiden} verabscheueten sie,
und zwar -- wie allgemein zugestanden wird, -- aus sehr edlen, moralischen
Beweggründen. Mir sinden keine Begünstigung derselben in
Plato’s\index{Personen:!Plato} oder in
Seneka's\index{Personen:!Seneka} Werken. Phthagoras\index{Personen:!Phthagoras},
Sokrates\index{Personen:!Sokrates}, Phocion\index{Personen:!Phocion},
Zeno\index{Personen:!Zeno} und Andere, hatten sich an
solche Erhohlungen nicht gewöhnt. Die tugendhafte
Penelope\index{Personen:!Penelope}, die keusche
Lucretia\index{Personen:!Lucretia}, die ernste
Kornelia\index{Personen:!Kornelia} und mehrere ihres Gleichen, fanden angenehme
Beschäftigungen genug unter ihren Kindern, Dienstboten und Nachbarn; und wiewohl
sie von hoher Abkunft waren, so bestand dennoch, nächst ihren Andachtsübungen,
ihr großtes Vergnügen in Spinnen, Weben, Nahen, oder andern hauslichen Arbeiten
und lobentiwerthen Verrichtungsen. Diese, welche Heiden genannt werden, zeigten
in ihren Handlungen mehr wahres Christenthum, als die jetzigen Verehrer und
Vertheidiger der eitlen Moden, üppigen Gebränche und sinnlichen Ergötzungen,
die, aller ihrern Thoheiten ungeachtet, sich dennoch Christen nennen.

\medskip

Aber vor Allen, ihr Theaterliebhaber\index{Theaterliebhaber}! Woher, glaubt ihr,
daß die Lustsplele, die
euch so sehr ergölzen, entstanden sind? da sie doch, -- obgleich sie unter allen
erdachten Erfindungen die verderblichsten sind, von euch so sehr geschätzt und
so gern und häufig besucht werden! Ich will es euch sagen: ihr Stammvater war
ein Heide\index{Personen:!Heiden}; und zwar keiner der bessern Klasse. Er hieß
Epicharmus\index{Personen:!Epicharmus}. Man nennt ihn
freilich einen Philosophen\index{Philosophen}, oder einen Verehrer der Weisheit;
er war es aber nur
dem Namen nach, und eben so wenig in der Wirklichkeit, als die heutigen
Schauspieler wahre Christen sind. Suidas\index{Personen:!Suidas}, ein
griechischer Geschichtschreiber,
erzählt von ihm, er sei der erste gewesen, der Komödien erfand, und habe auch
mit Hülfe eines gewissen Phormus\index{Personen:!Phormus} funfzig Fabeln
gemacht. Wollt ihr auch sein
Vaterland und die Veranlassung zu seiner Erfindung kennen? -- Seine Vaterstadt
war Syrakus\index{Orte:!Syrakus}, die (ehemalige) Hauptstadt
Siciliens\index{Orte:!Siciliens}, welche wegen der schandlichen
Handlungen vieler ihrer Tyrannen merkwürdig war. Diesen durch Ergötzung ihrer
Sinne zu gefallen, strengte Epicharmus\index{Personen:!Epicharmus} seinen Witz
an, und erfand die Komödien.
Ist dieses nun aber nicht ein böser und verwerflicher Ursprung derselben? Und
ist es daher nicht eben so tadelnswerth, wenn Jemand solche Dinge nachahmt, oder
zu rechtfertigen sucht, welche schon die Tugendhaften unter den Heiden
verwarfen? Ja, ist es nicht entsetzlich, wenn Diejenigen, die sich Christen
nennen, an solchen Erfindungen Vergnügen finden, sie unterstützen und in Schulz
nehmen? -- Auch können die traurigen Trauerspiele sich keiner bessern Abkunst
rühmen. Ihr Erfinder war ein gewisser atheniensischer Dichter, Namens
Thespis\index{Personen:!Thespis}, dem man auch die Einführung des unverschämten
Gebrauches, das Gesicht anzumalen,
und andere Personen, mittelst Nachahmung ihrer Kleidung, Gebehrden, Sprache etc.
vorzustellen, zuschreibt. Diesen beiden Erfindern muß ich aber noch einen
verliebten Dichter an die Seite setzen, den seine unbändige Leidenschaft so
hingerissen hatte, daß er sie in Versen laut werden ließ, welche die
schmuzigsten und niedrigsten Gesinnungen, wo nicht gar Abgötterei verrichthen.
Er hieß Aleman\index{Personen:!Aleman} oder Aleina\index{Personen:!Aleina}, und
war ein Lydier\index{Personen:!Lydier}. Von ihm sagt man, daß er der
erste gewesen sei, der die Welt mit der Thorheit: Liebesgeschichten und
Liebesgedanken in Liedern vorzutragen, beschenkte, welche hernach alle Nationen
in ihren Romanzen so eifrig nachgetan haben.

\section{7. Abschnitt} \label{kap17_ab7}

\textbf{Einige werden vermuthlich hiergegen einwenden, daß es doch auch viele
Lustspiele, Trauerspiele, Sonneiten, Lieder etc. gäbe, welche durchaus eine
Tendenz haben, das Laster zu bestrafen, und auß denen man viel Gutes lernen
können.} So nichtig dieser Einwurf auch ist, so haben ihn Einige, entweder aus
Mangel an Einsicht oder aus Unverschämtheit, mir doch schon gemacht. Ich gebe
auch zu, daß dergleichen Vorstellungen unter den Heiden, nächst dem Lesen ihrer
ernsten Moral-Philosophen, dienliche Heilmittel gegen die im Schwange gehenden
Laster waren. Unter diesen will ich zwei Beispiele ausheben:
Euripides\index{Personen:!Euripides}, dem
Suidas die Benennung eines tragischen Dichters beilegt, und
Eupolis\index{Personen:!Eupolis}, den
derselbe Schriftsteller einen komischen Poeten nennt. Der Erster führte ein so
keusches Leben, -- und war also von der größern Anzahl unserer jetzigen Männer
so sehr verschieden, -- daß man ihn einen Weiberfeind\index{Weiberfeind} nannte;

\medskip

wiewohl er nur die auzschweifenden Weiber haßte; denn er war zweimal
verheirathet. Den Andern beschreibt Suidas\index{Personen:!Suidas} als einen
scharfen Tadler der Laster.
Hieraus schließe ich, daß er nicht die Absicht dieser Männer war, die Eitelkeit
und Thorheit der Menschen zu nähren, oder mit ihren Arbeiten sich Geld zu
verdienen; sondern daß sie vielmehr, da daß Volk durch die frechen
Ausschweifungen seiner Vorgänger verführt worden war, dasselbe von seinen
Thorheiten wirklich dadurch wieder zurückzuführen suchten, daß sie daß Laster
lächerlich machten und die Macht ihres Witzes gegen die Zügellosigkeit der
Menschen spielen ließen. Dieses wird auch aus der Beschreibung, die wir von
ihnen haben, um so wahrscheinlicher, da man von
Euripides\index{Personen:!Euripides\} erzählt, er sei von
unzüchtigen Weibern in Stücken zerrissen worden; ohne Zweifel darum, weil er
gegen ihre Frechheit und Schamlosigkeit eiferte. Und von
Eupolis\index{Personen:!Eupolis} sagt man, er
sei, als er in einer Schlacht zwischen den
Atheniensern\index{Personen:!Athenern} und Lacedämoniern
umgekommen war, so sehr bedauert worden, daß man ein Gesetz gemacht habe,
welches solchen Dichtern nicht mehr erlaubte, die Waffen zu ergreifen; ohne
Zweifel aus keiner andern Ursache, als weil sonst mit ihnen die Bestrafer des
Lasters verloren gingen. Demnach hatten also die bessern komischen und
tragischen Dichter jener Zeit keinen andern Zweck, als das Volk dadurch zu
bessern, daß sie das Laster verhaßt machten; und zwar nicht so sehr durch
vernünftige Beweisgründe, wie es bei ihren Philosophen üblich war, sondern durch
beißenden Spott\index{Spott} und empfindliche Rügen, oder durch bildliche
Vorstellungen, die
ihre lasterhaften Handlungen in ein lächerliches und abscheuliches Licht
stellten; damit sie dadurch gezwungen würden, um ihres guten Rufes willen
sich derselben, nicht mehr schuldig zu machen. \textbf{Indessen sind alle diese
Mittel,
meiner Ansicht nach, doch nicht viel gelinder als die Geißel oder das
Zuchthaus\index{Zuchthaus}.}
-- Wollt ihr jedoch, die ihr dieselben vertheidigt, es euch gefallen lassen, daß
man euch als Heiden\index{Personen:!Heiden}, und zwar als Solche betrachte und
behandle, die eher durch
die Geißel des Spottes, als durch überredende Vernunftgründe, von bösen Dingen
abzubringen sind, dann wollen wir euch einräumen, daß eure Lust- und
Trauerspiele einigen Nutzen für euch haben können. Aber dann auch, wenn eure
verderbten Neigungen so stark sind, daß ihr zu den elenden Hülfsmitteln der
Heiden eure Zuflucht nehmen müsset, um sie zu zügeln und einzuschränken; wenn
ihr eure bösen Wege nicht aus Liebe zur Tugend verlassen, dem Laster nur aus
Furcht vor der Schande oder des schlechten Rufes wegen entsagen wollet; o! dann
schämt euch doch auch, den Namen Jesu Christi so offenbar zu mißbrauchen, daß
ihr euch Christen nennt. -- Stehet es so mit eurer Liebe zu Jesu?
\index{Gottesdienst, unterhaltsamer}\textbf{Beweiset dieses
eure Hochachtung vor der heiligen Schrift\index{heiligen Schrift}, die den
Menschen Gottes durch den
Glauben an Christuin vollkommen machen kann? Ist es mit allen euren schönen
Reden von göttlichen Anordnungen, vom Gebete\index{Gebete}, von den
Sakramenten\index{Sakrament}, vom
Christenthume, u.s.w. dahin gekommen, daß ihr euch solcher Belehrungsmittel
bedienen müsset, welche die tugendhaften Heiden zuließen, damit die
Verworfensten unter ihnen von ihren Lastern abgeschreckt werden möchten? ja,
solcher Mittel, die nicht viel mehr als körperliche Züchtigungen ausrichten
können?}

\section{8. Abschnitt} \label{kap17_ab8}

Diese Dinge stimmen in der That nicht mit einem wahrhaft christlichen Charakter
überein; da es unter den edlern Heiden\index{Personen:!Heiden} schon Männer und
Frauen gegeben hat, die
besser unterrichtet waren und erhabenere Gesinnungen hegten. Diese kannten
Gegenstände von höherer Art und ewig dauernder Beschaffenheit, denen sie ihre
Betrachtungen und ihr Nachdenken widmeten. \index{Sittenwächter}\textbf{Ja,
Viele von ihnen übertrafen die
Christen unserer Zeit, indem sie sich durch ein ernstes, gesetztes Betragen
auszeichneten. Die Athenienser ernannten die Gynaecosmi, oder zwanzig Männer,
die auf die Kleidung und das Betragen des Volks ein wachsames Auge haben mußten,
und daß Recht hatten, Jeden, der sich unbscheiden oder unanständig aufführte,
darüber zu bestrafen. Jetzt steht die Sache ganz andere. Wer solchen Leuten
Vorwürfe macht, wird strafbar, oder seht sich wenigstens dem bittersten Tadel
und der größten Verhöhnung aus.} Einige sind so ruchlos und treiben ihre
Unverschämtheit so weit, daß sie nicht allein religiöse Personen, sondern sogar
die heiligsten Sachen zu Gegenständen ihres elenden Gespöttes machen, und, nicht
damit zufrieden, ihre Geringachtung aller religiösen Grundsätze durch die
Ungebundenheit ihres eigenen Lebens an den Tag zu legen, ihre gänzliche
Verachtung der Religion auch dadurch beweisen, daß sie dieselbe durch komische
und niedrige Späße auf der öffentlichen Bühne zu entwürdigen und lächerlich zu
machen suchen. Wie gefährlich dieses ist, und wie leicht es die Wirkung hat, die
Religion in den Augen des Volkes herabzusetzen, beweiset das Beispiel des
Aristophanes\index{Personen:!Aristophanes}, der kein geschickteres Mittel wußte,
den Ruf des Sokrates\index{Personen:!Sokrates} bei dem
Volke, das ihn seiner ernsten Lehren und seines tugendhaften Lebens wegen
hochschätzte, verdächtig zu machen, als daß er ihn in einem Schauspiele von
einer lächerlichen Seite darstellte; welches denn auch die Folge hatte, daß der
leichtsinnige, ausgelassene und unbändige Pödel\index{Personen:!Pödel} lieber
dem ernsten, als dem
lächerlich gemachten Sokrates den Rücken zukehrte. Auch kann man leicht
einsehen, daß die wahre Ursache, warum die sogenannten
\textit{Quaker}\footnote{\texttt{...man beachte die Schreibweise
'Quaker'!}}\index{Quaker} von
leichtsinnigen und ausschweifenden Menschen so sehr verspottet werden, bloß
darin liegt, daß sie daß sündliche und eitle Leben der Menschen so ernstlich
rügen, und ihrer Unmäßigkeit in allen Arten weltlicher Vergnügungen durch ein
enthaltsames Leben der Selbstverleugnung beständig widersprechen. Denn alle jene
Freigeister\index{Personen:!Freigeister} wollen, ihres wüsten Lebens;
ungeachtet, für gute Christen gelten,
während man uns für eigensinnige, eingebildete, tiefsinnige und finstere
Sonderlinge, ja, für Ketzer\index{Personen:!Ketzer}, Betrüger, und wer weiß, für
was, noch mehr hält. O!
der großen Verblendung und pharisäischen Heuchelei\index{Heuchelei,
pharisäische}! Als wenn solche Menschen im
Stande wären, religiöse Gegenstände zu beurtheilen, oder als wenn es möglich
wäre, daß sie einen richtigen Begriff und ein inneres Gefühl von wahrer Religion
haben könnten, während ihr Verstand von dem Gotte der Weltfreuden verfinstert
und ihre Seele in äußern Genüssen ganz versunken ist. Nein! ich sage euch im
Namen des ewigen Gottes: Ihr spottet seiner und betrüget eure eigenen Seelen!
Denn der Zorn des Allmächtigen ist gegen euch Alle gekehret, so lange ihr in
einem solchen Geiste und Zustande verharret! Umsonst sind alle eure leeren Worte
und eitlen Beobachtungen! Gott lacht und spottet eurer! Sein Zorn ist über eure
Untugend entbraunt! Darum laßt euch warnen und zur Mäßigkeit ermahnen; bereuet
eure Abweichungen und bessert euren Lebenswandel!

\section{9. Abschnitt} \label{kap17_ab9}

\index{Volkswirtschaft}\textbf{Ueberdies werden die leichtsinnigen und
ausschweifenden Erfinder der weltlichen
Thorheiten durch den Beifall, womit ihre Erfindungen und Darstellungen
aufgenommen werden, in ihren Unternehmungen sehr bestärkt und aufgemuntert, und
folglich von ehrenvollen und nutzlichen Beschäftigungen zurückgehalten.} Auch
sind aus keiner andern Ursache viele nothwendige Lebensbedürfnisse in so hohem
Preise, als weil die Arbeit so theuer ist; und dieses ist offenbar nur darum der
Fall, weil so viele Hände mit den Anschaffungen und Besorgungen eitler
Ueberflüssigkeiten beschäftigt werden. Ja, wie häufig geschiehet es nicht, daß
jene Erfinder und Besorger der menschlichen Thorheiten, wenn sie um Geld
verlegen sind, dem Publikum eine neue Mode anheften, die, ihrem Vorgehen nach,
mehr als eine andere auf Bequen1lichkeit und Putz berechnet ist, und welche oft
schon dann eingeführt werden muß, wenn die vorigen Sachen kaum zur Hälfte
verbraucht sind, die hernach weggegeben oder mit neuen Kosten nach der neuesten
Mode zugestutzt werden müssen. O! der verschwenderischen, und doch so allgemein
üblichen Thorheit!

\section{10. Abschnitt} \label{kap17_ab10}

\index{Volkswirtschaft} Hier erwarte ich einer der scheinbarsten Einwendungen zu
begegnen, welche die
Vertheidiger dieser Dinge gewöhnlich Vorbringen, wenn sie sich in die Enge
getrieben sehen. \index{Arbeit, ehrliche}\textbf{Sie sagen nämlich: Wovon sollen
denn die vielen Familien leben,
die ihren Unterhalt in der Besorgung der Moden, Gebräuche und Ergötzungen der
Welt finden, gegen welche ihr so ernstlich eure Stimme erhebt? Ich erwiedere: es
ist ein schlechter Behelf, wenn man etwas Böses, sei es auch noch so gering,
darum vertheidigt, weil ein guter Zweck dadurch erreicht wird. Findet ihr an
sündlichen Eitelkeiten ein Vergnügen, und ziehen Jene ans der Besorgung
derselben einen Vortheil, so müßt ihr euch auch gefallen lassen, daß sie euch
beiden zur Pein und Strafe dienen, bis der Eine ohne solche Thorheiten zu leben
gelernt, und der Andere eine ehrlichere Beschäftigung gefunden hat.} Es ist die
Eitelkeit der wenigen Großen, welche den vielen Kleinen so viel zu schaffen
macht; denn wenn jene nicht alle Schranken überschritten, so würden diese nicht
nothig haben, so hart für sie zu arbeiten. Wollten daher nur die Menschen mit
Wenigem, oder mit dem unentbehrlichen sich begnügen, wie die ersten Christen
thaten, so würde Manchen bei weitem nicht so theuer und das Leben überhaupt viel
leichter zu erhalten sein. Hatten die Gutsbesitzer nicht so Viel zur
Befriedigung ihrer Leidenschaften nöthig, so brauchten ihre Pächter keine so
hohe Pacht zu bezahlen, könnten aus einem unbemittelten Stande sich zur
Wohlhabenheit hinauf arbeiten, und ihren Kindern ehrliche, häusliche
Beschäftigungen geben; wohingegen diese oft genöthigt sind, sich auf Nebenwegen
in der Welt durchzuschlagen, und daher nicht selten zu unerlaubten oder
lasterhaften Erwerbsmitteln greifen.

\medskip

\index{Landwirtschaft}\index{Volkswirtschaft}Wenn wir einsichtsvollen
Landwirthen Glauben beimessen wollen, so ließe sich der
Ertrag sehr vieler Ländereien noch bis aufs Doppelte vermehren, wenn es nicht an
thätigen Menschen dazu fehlte. Eben so konnten auch noch mehrere Hände bei dem
Betriebe erlaubter und nützlicher Manufakturen angebracht werden. Das würde die
Fabrikate billiger im Preise machen, ihren Absatz vermehren und der ganzen Welt
Vortheil bringen. Dadurch aber, daß die Unterhaltung der städtischen Eitelkeit
dem Ackerbaue und andern nützlichen Gewerben so viele Hände entziehet, fällt die
Last um so viel schwerer auf den Landmann und arbeitsamen Fabrikanten. Wenn die
Menschen sich nie für reich genug halten, so fehlt es ihnen auch niemals an
Sorgen und Mühe. Wer aber den ursprünglichen Zustand der Schöpfung Gottes zu
seiner Richtschnur nimmt, der lernt sich mit Wenigem begnügen, indem er
einsiehet, daß der Durst nach Reichthum\index{Reichthum} nicht allein den wahren
Glauben
untergräbt und zerstöret, sondern auch Denen, die zu seinem Besitze gelangt
sind, zum Fallstricke und zu einer Quelle vieler Unruhe dient. Es ist nicht
unrecht, Unrecht zu bereuen; doch kann es nicht eher dahin kommen, bis die
Menschen aufhören, Das, was sie bereuen sollten, zu rechtfertigen. Auch ist es
in der That ein schlechter Beweggrund, wenn Jemand darum keine Maßigkeit übt,
und die Unmaßigkeit in Schutz nimmt, weil dadurch Viele sich ihren Unterhalt
erwerben und die Erfinder und Verbreiter der maßlosen
\footnote{\texttt{'üppigen' ersetzt durch 'maßlosen'}} Moden und Gebrauthe sonst
keine Nahrung haben würden. Menschen aus diese Weise erhalten, heißt, das Laster
füttern und nähren, statt ihm seine Nahrung zu entziehen. Wurde es nicht
wohlgethan sein, wenn die reichen Besorger und Beförderer der Eitelkeit, die
sich schon viel mehr erworben haben, als sie brauchen, aus ihren Geschäften sich
zurückzogen, und anfingen, ihr Vermögen besser anzuwenden, als sie es erwarben;
indem sie wirkliche Arme damit unterstützen und ihnen zu bessern Beschäftigungen
behülflich wären? Gewiß dieses wäre klüger, edler und auch christlicher, als die
Menschen zu Ausgaben für Tand und Thorheiten zu verleiten. Oeffentliche
Arbeitshäuser\index{Sozialer Wohnungsbau} würden treffliche Heilmittel gegen die
Ausbrüche dieses so
ansteckenden Uebels der Ueppigkeit sein, wobei auch Jeder sowohl in seiner Kasse
als in seinem Gewissen sich besser stehen würde.

\medskip

Nach solchen Ansichten und Grundsätzen können und dürfen wir in unserm
Lebenswandel unter den Menschen uns nicht nach den herrschenden Gebräuchen der
Welt bequemen. Wir müssen vielmehr durch unsere Einfachheit und Maßigkeit gegen
ihre eitle Verschwendung zeugen\index{Zeugnis}, und durch unser ernsten und
bestimmtes Betragen
zur Ehre Gottes an den Tag legen, wie sehr wir die verschwenderische Prachtliebe
und daß zügellose Leben der Menschen mißbilligen; ja, wir müssen Manches, was
wir sonst wohl als erlaubt betrachten würden, und mit völliger Gleichgültigkeit,
oft selbst mit Vergnügen gebrauchen könnten, bloß nur der Mißbrauchs willen uns
versorgen, der so allgemein damit getrieben wird.

\section{11. Abschnitt} \label{kap17_ab11}

\textbf{Einige sind ferner mit einem andern Einwurfe bereit, indem sie sagen:
Hat denn
Gott uns diese Lebensgenüsse bloß gegeben, um uns zu verdammen, wenn wir sie
gebrauchen?} Solchen armen, unwissenden Menschen, die lieber dem allerhöchsten
und heiligsten Gott die Erfindung oder Erschaffung ihrer thörichten Eitelkeiten
zur Last legen, als daß es ihnen an Entschuldigungsgründen für den Gebrauch
derselben mangeln sollte, die aus Furcht oder Scham, oder aus Anhänglichkeit an
dieselben, nicht wissen, wie sie ihnen entsagen und sich davon losmachen sollen,
-- \textbf{solchen Unglücklichen antworte ich: Alles, was Gott zum Gebrauche des
Menschen schuf, war gut, und was unser Heiland Jesus Christus erlaubt,
verordnet, oder durch sein eigenes Beispiel anempfohlen hat, muß beobachtet,
geglaubt und geübt werden. \index{Bibel, verbindlichkeit}
\footnote{Lukas  8,14. Lukas 12,28-31.}
\index{Bibelstellen:!Lukas  8)}
\index{Bibelstellen:!Lukas 12)}
Allein ich
finde in dem ganzen Verzeichnisse seiner Lehren und Gebote, welches die heilige
Schrift uns vorhält, keine solche Kleiderpracht, noch solche Ergötzungen, oder
eine solche Lebensweise, als heutiges Tages unter den mehrsten Bekennern des
Christenthumes üblich ist, weder geboten noch empfohlen.} Nein, wahrlich! Gott
schuf den Menschen zu einem heiligen, weisen, mäßigen und ernsten Wesen, und
begabte ihn mit Vernunft und Fähigkeit, sich selbst und die Welt zu beherrschen.
Damals war die Erkenntniß Gottes der große Gegenstand seiner Betrachtung, seines
Nachdenkenis und seiner Freuden; alle seine äußern Genüsse, die Gott ihm gab,
waren der Nothwendigkeit, der Bequemlichkeit und der Erlaubtheit des Vergnügens
mit dem Vorbehalte untergeordnet, daß er in Allem den Allmächtigen sehen,
fühlen, genießen und verehren sollte. Aber ach! wie weit sind die mehrsten
Christen von dieser ursprünglichen Verordnung abgewichen, die dessen ungeachtet
so hohe Ansprüche auf das Leben, das Verdienst und den Tod eines heiligern Jesus
machen, der nicht allein durch seine Erscheinung der Welt einen sichern Beweis
von der Möglichkeit einer glücklichen Wiederherstellung des Menschen gegeben
hat, sondern auch Allen seinen gnädigen Beistand verheißt, die ihm auf dem Wege
des heiligen Kreuzen und der Selbstüberwindung
\footnote{\texttt{'Selbstverleugnung' ersetzt durch 'Selbstüberwindung'}}
nachfolgen wollen, den er ihnen
als den einzigen Pfad zu ihrer ewigen Seligkeit vorgezeichnet hat. \textbf{Ob
nun aber
die Gemüther jener Menschen, beides Geschlechts, nicht so tief in Thorheiten und
Eitelkeiten versunken sind, daß sie den Herrn des Lebens nicht weiter als vom
Hörensagen kennen; und ab ihr begieriges Trachten nach niedrigere Dingen dieser
Welt nicht die Ursache ist, daß sie, des Genusses der Gegenwart Gottes beraubt,
allen Gegenwart an göttlichen Freuden verloren haben, und daher sich
eingebildete Vergnügungen und immer neue Zerstreuungen ersinnen, um die
anklagende Stimme in ihrem Innern nicht zu vernehmen, oder dieselbe zu
übertäuben, und so ihre Tage und Nächte ohne jene störenden Gefühle der Angst
und Unruhe, welche die unvermeidlichen Folgen ihrer Uebertretungen sind,
gemächlicher und sicherer in dieser Welt zubringen zu können, -- ob dieses nicht
mit ihnen der Fall ist, möge ihr eigenes Gewissen ihnen
beantworten.}
\footnote{Römer 2,8+9}
\index{Bibelstellen:!Römer 2)}

\medskip

Die Versuchung Adams wird dadurch vorgestellt, daß er gereitzt ward, von der
Frucht eines Baumes zu essen.
\footnote{1 Mose 3,6.}
\index{Bibelstellen:!1 Mose 3)}
Dieses zeigt uns, was für
einen mächtigen Einfluß äußere schöne oder reihende Gegenstände auf Unsere Sinne
haben. Ja, die Macht der sichtbaren Dinge ist in der That so hinreißend, daß
Jeder, der nicht beständig in seinem Gemüthe dagegen macht, sehr leicht von
ihnen gefangen genommen wird, Und ist er erst einmal von ihnen überwunden und
zum Sklaven\index{Personen:!Sklaven} gemacht, so verbreiten sie einen so dichten
und finstern Schleier
über seine Seele, daß er sich selbst nicht mehr erkennet, und nicht allein die
Fesseln des maßlosen
\footnote{\texttt{'üppigen' ersetzt durch 'maßlosen'}} und eitlen Lebens mit
Vergnügen tragt, sondern sogar aus
seinen Sklavenstand\index{Sklavenstand} so stolz ist, daß er Andere, die sich
demselben entziehen,
mit seinem Tadel überhäuft, indem er ihn als einen nützlichen und angenehmen
Stand vertheidigt. Eine so sonderbare Leidenschaft erzeuget die Liebe zu den
vergänglichen Gegenständen des Vergangene der Welt in den Herzen Derer, die ihr
Eingang verstatten und Nahrung gewähren!
\textit{"`Wir wissen aber, daß Jesus Christus,
der Sohn Gottes, gekommen und in uns geoffenbaret ist, und er hat uns einen Sinn
gegeben, und unsern Verstand erleuchtet, daß wir ihn, den Wahrhaftigen,
erkennen."'}
\footnote{1. Johannes 5,20}
\index{Bibelstellen:!1. Johannes 5)}
\textbf{Und Er hat Allen ein hinreichendes Maß seines
guten Geistes verliehen, welches, wenn sie ihm nur Gehorsam leisten wollen,
vermögend ist, ihre Seelen aus der Sklaverei der Eitelkeit zu erretten, und von
der Herrschaft aller sinnlichen Gegenstände, welche die Augenlusl, die
Fleischeslust und das hofartige Leben nähren, gänzlich zu befreien;} so daß ihre
Herzen erneuert und wiedergeboren\index{wiedergeboren} werden, ihre Neigungen
eine andere Richtung
bekommen und ihre ganze Seele sich mit den Dingen beschäftigt, die drohen sind,
wo weder Rost noch Motten eindringen, und ihre Schätze nicht zerstören können.

\section{12. Abschnitt} \label{kap17_ab12}

Es ist leicht einzusehen, was für Menschen es sein müssen, welche sich den
irdischen Freuden hingeben, und jene Ueberbleibsel der ägyptischen Zierrathen in
Schutz nehmen. Sie müssenn das demütige, sanfte, einfache, heilige und
Musterhafte Leben der Selbstüberwindung
\footnote{\texttt{'Selbstverleugnung' ersetzt durch
'Selbstüberwindung'}},\footnote{Galater 5,22+25) Epheser 5,8-11)
Epheser 15,16}
\index{Bibelstellen:!Galater 5)}
\index{Bibelstellen:!Epheser 5)}
\index{Bibelstellen:!Epheser 15)}
wozu der heilige Geist alle ihm gehorsame Herzen anleiten und fähig
macht, entweder nie gekannt oder sich von demselben wieder entfernt haben. Ja,
es leidet keinen Zweifel, daß alle solche Menschen diesen gute Land, dieses
himmlische Vaterland und selige Erbteil niemals recht sahen, oder, wenn sie auch
einen entfernten schwachen Blick davon hatten, doch wieder ganz aus dem Gesichte
verloren haben. O! \textbf{möchten sie sich doch einmal ruhig niedersetzen sich
selbst
einkehren und ernstlich erwägen, wo sie sind, und wessen Werk und Willen sie
thun!} Möchten sie doch einsehen, daß unter allen listigen Kunstgriffen des
Feindes ihrer wahren Glückseligkeit keiner für ihre Unsterblichen Seelen so
gefährlich ist, als dieser, daß er ihre Sinne und Gedanken unaufhörlich mit den
thörichten Moden und üppigen Vergnügungen der eitlen Welt beschäftigt! Grobe, in
die Augen fallende Laster, erregen gewöhnlich bei Denen, die eine gute Erziehung
genossen haben, und welche auf einen guten Ruf etwas halten, den größten
Abscheu. Daher greift der schlaue Feind, da er wohl einsiehet, daß er mit seinen
Versuchungen zu denselben bei Vielen nichts ausrichten kann, zu seinem und
verfänglichern Mitteln, indem er die Gemüther der Menschen zu Zerstreuungen und
Erholungen verleitet, die beim ersten Anblicke nicht so schädlich erscheinen,
weil sie weniger mit Schande verbunden sind, und indem sie erlaubte Vergnügungen
versprechen, auch stärker anziehen Und desto sicherer zu seinem Zwecke führen,
der kein anderer ist, als die Menschen durch immerwährende Beschäftigung mit
sinnlichen Gegenständen von einem ernsten Forschen und Trachten nach Dem, was zu
ihrem ewigen Frieden dienet, abzuhalten.
\footnote{Epheser 6,12-19}
\index{Bibelstellen:!Epheser 6)}
Anf solche
Weise sucht der arge Feind die Menschen in Ansehung der himmlischen Dinge in
beständiger Unwissenheit zu erhalten, damit sie das ewige Leben nicht kennen
lernen, und folglich auch nicht darnach ringen, sondern sich mit den
Beobachtungen leerer, von Menschen erfundenen und vorgeschriebenen
Religionsgebräuchen begnügen, wobei sie ihren gewöhnlichen Vergnügungen
ungestört nachgehen können, indem ihre Religion sind ihr Lebenswandel
größtenteils mit einander übereinstimmen. Daher wissen sie denn auch nicht, was
es heißt.
\textit{"`in der Erkenntniß Gottes wachsen, Gnade um Gnade empfangen, und zu
dem vollkommnen Mannsalter Christi gelangen."'}
\footnote{Epheser 1,16-23. Epheser 4,12+13}
\index{Bibelstellen:!Epheser 1)}
\index{Bibelstellen:!Epheser 4)}
Was daher Viele in ihrem siebenten Jahre waren, das sind sie noch in
ihrem siebzigsten; nur nicht mehr so unschuldig; es sei denn daß das alte
Sprichwort eintreffe, daß Greise wieder Kinder werden.

\medskip

Wahrlich! das Geheimniß der Gottseligkeit\index{Gottseligkeit}, das göttliche
Leben, daß wahre
Christenthum, sind Gegenstände von der höchsten Wichtigkeit! Und da wir sehen,
daß der Feind, wenn er die Menschen nicht zu groben Lastern verleiten kann,
allezeit geschäsftig ist, ihre Gemüther mit anscheinend unschuldigen
Unterhaltungen zu beschäftigen und an sich zu ziehen, damit er sie um so
leichter an der Wahrnehmung ihrer Pflichten und an ihrem Wachsthume in der
Ertenntniß des allein wahren Göttes, welche ewiges Leben ist,
\footnote{Johannes 17,3}
\index{Bibelstellen:!Johannes 17)}
verhindern, und ihre Gemüther von der Betrachtung himmlischer und
unverganglicher Dinge gänzlich abziehen könne;\\textbf{so müssen Alle, welche
seinen
Schlingen entgehen wollen, ihre Aufmerksamkeit auf die Erscheinung der
göttlichen Gnade\index{Gnade} in ihrem Innern richten, die sie lehret und fähig
macht, das
ungöttliche Wesen zu überwinden
\footnote{\texttt{'verleugnen' ersetzt durch 'überwinden'}}, und allen eitlen
und bösen Dingen auf immer zu
entsagen.}
\footnote{Titus 2,11-15}
\index{Bibelstellen:!)}
Dann wird ihr verbesserter Lebenswandel gegen
das unmäßige und ausschweifende Leben der Welt zeugen; dann werden sie zu der
Zahl der wahren sich selbst überwindenden
\footnote{\texttt{'verleundenden' ersetzt durch 'überwindenden'}} Junger Jesu
gehören. Thun sie dieses
aber nicht, so werden sie die schrecklichen und verderblichen Folgen davon
erfahren.

\medskip

Dadurch, daß die Menschen ein Verlangen nach immer neuen Unterhaltungen äußern,
und so viel Zeit und Geld darauf verwenden, geben sie nicht allein den Erfindern
und Verbreitern solcher Thorheiten, die bloß zur Befriedigung eitler und üppiger
Neigungen dienen, große Aufmunterung, mit ihren Bemühungen fortzufahren, sondern
machen sich auch des Bösen dieser Beförderer der Maßlosigkeit
\footnote{\texttt{'Üppigkeit' erstzt durch 'Maßlosigkeit'}} theilhaft, die aus
eine elende Art ihren Witz anstrengen und ihre eigene Zeit verschwenden, um neue
Zeitvertreibe für Andere zu erfinden. Fänden hingegen solche Menschen nicht die
Unterstützung und Aufmunterung, die ihnen so allgemein gewähret wird, so könnten
Mangel und Dürftigkeit die Mittel werden, sie, wie den verlornen Sohn, an ihres
Vaters Haus zu erinnern. Denn, was auch Einige davon halten mögen, so hatte doch
der Feind der wahren menschlichen Glückseligkeit\index{Glückseligkeit} unter
allen seinen
Verführungsmitteln nie angenehmere Lockspeisen, anziehendere Gegenstände,
gefälligere Unterhaltungen, listigere Abgesandte, einnehmendere
Prediger\index{Prediger},
dezauberndere Vorträge und hinreißendere Redner, durch welche er die Menschen
anlocken und in sein Netz ziehen, und so ganz von aller Betrachtung göttlicher
Dinge entfernen konnte, als der Putz, die Ergözungen, Schauspiele und üblichen
Zeitvertreibe unsers verwilderten Zeitalters sind, welche, als wahre Schulen der
Maßlosigkeit
\footnote{\texttt{'Üppigkeit' erstzt durch 'Maßlosigkeit'}} und Werkstätte des
Verderbens, diese gerechte Rüge verdienen.




