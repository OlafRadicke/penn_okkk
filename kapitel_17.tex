
\chapter{17. Kapitel} \label{kap17}
\section{Zusammenfassung des 17. Kapitel}

\begin{description}
\item[1. Abschnitt] Die Gebräuche, Moden, Unterhaltungen, etc., welche die
anziehende Pracht und das Vergnügen der jetzigen Welt ausmacht, verhindern die
Menschen, in sich selbst einzukehren.
\dotfill \textit{Seite~\pageref{kap17_ab1}}\\
\item[2. Abschnitt] Ihr Zweck ist bloß die Befriedigung der Sinnlichkeit.
\dotfill \textit{Seite~\pageref{kap17_ab2}}\\
\item[3. Abschnitt] Gewährten sie wahre Freuden, so wäre Adam und Eva, die sie
nicht kannten, nicht vollkommen glücklich gewesen.
\dotfill \textit{Seite~\pageref{kap17_ab3}}\\
\item[4. Abschnitt] Die Dreistigkeit und Vermessenheit, womit sogenannte
Christen sich ihnen hingeben, ist scheußlich.
\dotfill \textit{Seite~\pageref{kap17_ab4}}\\
\item[5. Abschnitt] Zu wissen, dass sie eine Erfindung leichtsinniger und eitler
Menschen sind, ist schon ein hinreichender Grund, sie zu verwerfen.
\dotfill \textit{Seite~\pageref{kap17_ab5}}\\
\item[6. Abschnitt] Sie sind größtenteils von den Heiden entlehnt, die Gott
nicht kannten.
\dotfill \textit{Seite~\pageref{kap17_ab6}}\\
\item[7. Abschnitt] Eine Einwendung die man hinsichtlich ihrer Nutzbarkeit
vorbringt, wird beleuchtet und beantwortet, und ihre Verteidiger werden
bestraft.
\dotfill \textit{Seite~\pageref{kap17_ab7}}\\
\item[8. Abschnitt] Die aufgeklärten Heiden verabscheuten, was die vorgeblichen
Christen verteidigen.
\dotfill \textit{Seite~\pageref{kap17_ab8}}\\
\item[9. Abschnitt] Die Aufnahme, welche dergleichen Torheiten finden, dient
den Erfindern derselben zur Aufmunterung, damit fortzufahren
\dotfill \textit{Seite~\pageref{kap17_ab9}}\\
\item[10. Abschnitt] Beantwortung der Einwendungen, dass viele Familien durch
solche Dinge ihre Nahrung finden. -- Man mus nicht Böses tun, damit Gutes
daraus entstehe. Es lassen sich bessere Beschäftigungen auffinden, die der
Menschheit mehr Nutzen bringen können.
\dotfill \textit{Seite~\pageref{kap17_ab10}}\\
\item[11. Abschnitt] Beantwortung einer anderen Einwendung. Gott ist nicht der
Urheber solcher eitlen Erfindungen, daher lassen sie sich auch nicht als von
ihm herrührende Einrichtungen entschuldigen.
\dotfill \textit{Seite~\pageref{kap17_ab11}}\\
\item[12. Abschnitt] Wer solche Eitelkeiten in Schutz nimmt, zeigt, was er ist.
-- Ermahnung an alle vernünftig Denkende und ernsthaft Gesinnte. -- Es ist ein
notwendiger Schritt, zu der wahren Nachfolge Jesu zu gelangen, dass man die
Schule der Maßlosigkeit
\footnote{\texttt{'Üppigkeit' ersetzt durch 'Maßlosigkeit'}}
verlasse.
\dotfill \textit{Seite~\pageref{kap17_ab12}}\\

\end{description}

\newpage

\section{1. Abschnitt} \label{kap17_ab1}

Endlich sind jene \footnote{\texttt{'üppigen' ersetzt durch 'maßlosen'}}
maßlose Moden, Gebräuche und Belustigungen, welche die
anziehende Pracht und beständige Unterhaltung des gegenwärtigen Zeitalters
ausmachen, ein mächtiges Hindernis, der inneren stillen 
Sammlung\index{Stille!Sammlung}\index{stiller Andacht} oder Einkehr des
Gemüts, wodurch der Mensch zum Anschauen der Herrlichkeit ewiger
Unsterblichkeit\index{Unsterblichkeit} gelangen kann.
\label{ref:17_01_erziehung}
\textbf{\index{Erziehung}Denn, anstatt dass die Menschen angewiesen werden
sollten,
\textit{"`in ihrer Jugend an ihren Schöpfer zu denken, und zuerst nach dem
Reiche Gottes\index{Reich!Gottes} zu trachten, alles übrige zur Nothdurft
%wie könnte man "Nothdurft" hier ersetzen? Vielleicht "zum Leben"?
Gehörige aber, nach dem
Befehl Gottes und unseres Herrn Jesu Christi, als eine Zugabe zu
erwarten,"'}
\footnote{Weisheit Salomos 12,1) Lukas  12,29-31}
\index{Bibelstellen:!Weisheit Salomos 12)}
\index{Bibelstellen:!Lukas  12)}
werden sie, sobald sie nur
etwas tun können, von diesen Erfindungen des Stolzes und der Maßlosigkeit
\footnote{\texttt{'Üppigkeit' ersetzt durch 'Maßlosigkeit'}}
angezogen, und zu solchen Unterhaltungen angeleitet, die der Sinnlichkeit am
meisten behagen, und welche dann danach die Gegenstände ihres höchsten
Vergnügens werden. Auf diese Weise werden offenbar unerlaubte Begierden erzeugt
und maßlose
\footnote{\texttt{'üppige' ersetzt durch 'maßlose'}} Gedanken erweckt, die zu
leichtsinnigen Gesprächen, unzüchtigen
Spielen und ausgelassenen, rauschenden Freuden, wo nicht endlich zu gottlosen
Handlungen führen.} Für die Menschen, welche von einer so ausschweifenden
Lebensweise hingerissen sind, ist es langweilig und anstößig, vom Himmel oder
von einem künftigen Leben zu hören. Sage ihnen, dass sie über ihre Handlungen
nachdenken, den heiligen Geist\index{heiliger Geist} nicht betrüben, ihr ewiges
Los beherzigen oder
sich auf den Tag des gerechten Gerichts Gottes\index{Gott!Gericht}
vorbereiten möchten, so besteht
ihre Antwort, -- wenn sie nicht grob ausfahren -- gewöhnlich in spöttischen
Scherzreden und leichtsinnigen Bemerkungen. Es sind ganz andere Gegenstände, die
ihre Gedanken beschäftigen. Ihre Morgenstunden reichen kaum aus, ihren Anzug zu
besorgen, ihren Schmuck anzulegen, und ihren ganzen Putz in Ordnung zu bringen,
und ihre Nachmittage sind gewöhnlich Besuchen vorbehalten und
\label{ref:17_01_schauspiel}
\index{Schauspiel}\index{Theater}\textbf{dem Schauspiele
gewidmet, wo einige aus den beliebtesten Romanen entnommene\footnote{\texttt{'entlehnte' ersetzt durch 'entnommene'}} Szenen zu ihrer
Unterhaltung dienen. Da gibt es dann eine Menge Abwechselungen: Seltsame
Abenteuer, merkwürdige Liebesgeschichten, grausame Weigerungen,
unübersteigliche Hindernisse, zudringliche Besuche, unglückliche Täuschungen,
wunderbare Überraschungen, unerwartetes Zusammentreffen, überfallene Schlösser,
gefangene und befreite Liebende, plötzliche Erscheinungen tot geglaubter
Personen, blutige Zweikämpfe, schmachtende Stimmen, die aus einsamen Grotten
hervorhallen, leise vernommene Klagen, tiefe Seufzer, die sich aus wüsten
Orten vernehmen lassen, heimliche Ränke, die mit unerhörter Feinheit
geschmiedet werden, und endlich, wenn alles verloren zu gehn scheint, lässt man
Tote wieder lebendig, Feinde wieder Freunde werden. Die Verzweiflung verwandelt
sich in Entzücken, und alle Unmöglichkeiten gleichen sich aus. Da tragen sich
dann Dinge zu, die nie gewesen sind, die auch jetzt nicht sind, und nie sein
werden, noch jemals sein können.} Und als wenn die Menschen zu langsam, oder
nicht bereitwillig genug wären, den zügellosen Neigungen ihrer verderbten
Naturen zu folgen, oder, als wenn Gefahr da wäre, dass ihre Gemüter sich zu sehr
mit gottseligen Betrachtungen und himmlischen Gegenständen beschäftigen möchten,
werden alle Kräfte des Witzes und der Empfindsamkeit aufgeboten, und nicht allein
offenbare Lügen, sondern in der Natur ganz unmögliche Dinge vorgestellt, um
schädliche Leidenschaften in ihren Herzen aufzuregen und ihre schwindelnde
Einbildung mit schwellenden Bildern der Sinnlichkeit zu erhitzen.
\label{ref:17_01_schauspiel_2}
\textbf{So wird dann
nicht allein ihre Zeit verschwendet, ihre Natur verweichlicht und ihre Vernunft
entehrt, sondern auch nicht selten bei ihnen der Gedanke erzeugt, solche Dinge
in der Wirklichkeit auszuführen und dass eine oder andere Abenteuer nachzuahmen.
Will dieses aber nicht gelingen, -- wie sich denn von bloßen Hirngespinsten
nichts anderes erwarten lässt, so scheint dem Abenteurer kein anderes Mittel übrig
zu bleiben, als sich den lasterhaftesten Ausschweifungen in die Arme zu werfen.}
Dieses sind einige ihrer unschuldigsten Erholungen, die jedoch jeder wahre
Christ nicht anders als gefährliche Schlingen des Feindes der menschlichen
Glückseligkeit betrachtet, in welchen der Versucher die Gemüter der Unwachsamen
um so leichter fangen und zu sich ziehen kann, da sie den natürlichen
Schwachheiten der Menschen so angemessen sind und sich ganz unmerklich ihrer
Neigungen durch solche Dinge bemeistern, die den stärksten Eindruck auf ihre
Sinnlichkeit machen und sie mit einer unwiderstehlich scheinenden Gewalt
hinreißen. \textbf{Bei solchen Gelegenheiten geschieht es, dass ihre Seelen
Eitelkeit
brüten, ihre Augen die Dolmetscher ihrer Gedanken werden, und ihre Blicke die
geheimen Flammen ihrer ausschweifenden Seelen verraten, die so lange
umherirren, bis die Nacht ihre Übeltaten bedeckt, welche ihr Gewissen mit
Schuld und ihren Ruf mit Schande beflecken.
\footnote{Sprüche 7,16-31}}
\index{Bibelstellen:!Sprüche 7)}

\section{2. Abschnitt} \label{kap17_ab2}

Hier sehen wir, dass der Zweck aller weltlichen Moden und Erhohlungen kein
anderer ist, als die sinnlichen Begierden des Menschen:
\textit{"`Die Augenlust, die
Fleischeslust und den Hochmut des Lebens zu befriedigen"'}
\footnote{1. Johannes 2,15+16}
\index{Bibelstellen:!1. Johannes 2)}
Und darin hat man es so weit getrieben, dass in der Tat die Kleidung, die
einst bloß zur Bedeckung der Blöße eingeführt war, jetzt wohl noch einer
Bedekung bedürfte, um ihre schamlose Pracht zu verhüllen, indem der Mensch das,
was ihn an den Verlust seiner Unschuld erinnern sollte, als ein Mittel zur
Befriedigung seines Stolzes und seiner Prachtliebe gebraucht.
\label{ref:17_02_adam_und_eva}
\textbf{Gewiss schon der
hundertste Teil der Dinge, welche jetzt das angenehmste Vergnügen der Menschen
ausmachen, ja, in unseren Tagen durchaus zum guten Tone gehören, würden unseren
ersten Eltern den Verlust des Paradieses gekostet haben. Denn so wie
Adams\index{Personen:!Adam}
Übertretung darin bestand, dass er einen andern Genuss suchte, als Gott ihm
angewiesen hatte, so ist es auch der Fehler der jetzt lebenden Menschen, dass sie
unerlaubten Vergnügungen nachgehen und den größten Teil ihrer Zeit mit eitlen
Dingen zubringen, die so weit entfernt sind, dem Zwecke ihres Daseins, einem
Gott wohlgefälligen Leben, zu entsprechen, dass sie vielmehr sehr nachteilig und
zerstörend auf dasselbe wirken.}

\section{3. Abschnitt} \label{kap17_ab3}

Wären die Freuden der Menschen unserer Zeit echter und wahrer Art, so würden
Adam\index{Personen:!Adam} und Eva\index{Personen:!Eva} in dem Stande ihrer
Unschuld nicht glücklich gewesen sein, da sie
dieselben nicht kannten. Allein es machte vielmehr einen großen Teil ihrer
Glückseligkeit aus, sie nicht zu kennen, und eben so besteht auch ein hoher
Grad der Seligkeit derer, die Christus wirklich kennen, darin, dass sie durch
seine ewige Kraft von jenen Torheiten erlöset und zur Liebe eines
unvergänglichen Lebens erweckt sind. Dieses ist noch denen ein Geheimnis, deren
Lebensgenuss und höchstes Vergnügen in ausgesuchten prächtigen Kleidern, in
Schmuck\index{Schmuck} und Putz, in Erfindungen\index{Erfindungen} und
Nachahmungen neuer Moden, gezierten
Stellungen, Biegungen, Haltungen und Bewegungen des Körpers und der lüsternen
Blicke und der Wendungen der Augen, im Romanlesen\index{Romanlesen}, Besuchen,
Spielen\index{Spiele},
Lustpartien, Schauspielen\index{Schauspiele}, Bällen\index{Bälle},
Festen\index{Feste}, Gastmählern und andern sogenannten
Erholungen besteht. Denn, da alle dergleichen Dinge nie entstanden sein würden,
wenn der Mensch seinen Schöpfer nicht verlassen und sein Gemüt nur mit den
edlen Zwecken seiner Erschaffung beschäftigt hätte, so leuchtet es klar ein, dass
diejenigen, welche solchen Eitelkeiten ergeben sind, die höheren himmlischen
Freuden, den Genuss des göttlichen Friedens und der wahren Ruhe der Seele noch
nicht kennen, weil eben diese Dinge sie von der stillen Einkehr in ihr Inneres
und vom ernsten Trachten nach ewigen Gütern beständig abhalten. Ach! Mit was für
Mühe, Anstrengung und Unruhe, mit welchem Erfindungsgeist und Kunstfleiß, und
mit wie vielem Zeit- und Kostenaufwande ist der Mensch so beständig bestrebt,
für die vorübergehenden Genüsse seiner Sinne zu sorgen, und wie wenig zieht er
seine unsterbliche Seele, das Bild der Gottheit, in Betrachtung! Gewiss, es
bedarf keiner stärkerer Beweise, keiner deutlicheren Merkmale, um die Menschen zu
überführen, dass es nur der sinnliche Leib, ein mit Fleisch und Haut überzogenes
Knochengerippe ist, wofür sie so viel Tand\index{Tand} und
Flitterstaat\index{Flitterstaat} anschaffen, und dass
es die nichtigsten Torheiten und Eitelkeiten sind, die ihre Gemüter so sehr
eingenommen haben, dass sie nicht Liebe, Zeit und Geld genug darauf
verwenden zu können glauben.

\section{4. Abschnitt} \label{kap17_ab4}

Auf diese Weise sind ihre Gemüter unaufhörlich beschäftigt, und dabei sind sie
so sehr von sich selbst eingenommen, oder vielmehr so verfinstert in ihrem
Verstand, dass sie nicht allein alle jene eitlen Torheiten für ganz unschuldig
halten, sondern sich auch einbilden, sie könnten bei dem Gebrauch der selben
nichtsdestoweniger gute Christen\index{Personen:!Christen!gute} sein. Ihnen Vorwürfe
darüber zu machen, würde
die ärgste Ketzerei\index{Personen:!Ketzer} sein.
\label{ref:17_04_gottesdienst}
\textbf{So entfernt sind sie von dem inneren göttlichen Leben,
-- da ihre immerwährenden Zerstreuungen ihnen keine ernste
Selbstprüfung\index{Selbst!-prüfung}
erlauben, -- dass sie bei ihrem Gottesdienst\index{Gottesdienst}\index{Gebet}
sich damit begnügen, mit einem
erzwungenen Eifer eine halbe Stunde lang die Worte eines anderen herzusagen,
womit sie auch weiter keinen Sinn verbinden. Denn das, was sie sagen, hat so
wenig Beziehung auf ihren eigenen Zustand, oder sie haben, -- wie ihre
Handlungen beweisen, -- ebenso wenig die Absicht danach zu tun, als jener
Jüngling im Evangelium, welcher sagte,
\textit{"`er wolle gehen, und doch nicht ging."'}}
Aber ach! Warum tun sie es nicht? -- Sie sind mit anderen Gästen beschäftigt.
Und wer sind diese? Pharamund\index{Personen:!Pharamund},
Cleopatra\index{Personen:!Cleopatra}, Cassandra\index{Personen:!Cassandra},
Clelia\index{Personen:!Clelia}, ein Schauspiel, ein
Ball, ein Lustgarten, ein Park, ein Verehrer, die Börse\index{Börse}, mit einem
Worte, \textbf{die
Welt.}\index{Welt!Befrifflichkeit}Diese hält sie ab, diese erwartet, ruft,
sucht, quält sie, und da müssen
sie ihr ja Gehör geben, und könnten sich ihrer Gesellschaft entziehen. So werden
ihre Herzen gefangen genommen und von der Betrachtung göttlicher Dinge, ja, oft
selbst von der Wahrnehmung solcher äußeren Angelegenheiten abgehalten, die mit
ihrem eigenen Vorteile oder mit der Wohlfahrt ihres Nächsten in unmittelbarer
Verbindung stehen. \textbf{Sie finden nur Geschmack an den Vorstellungen, die
jene
törichten Tändeleien ihren Gemütern einflößen, und wenn sie diese auch nicht
alle mitmachen, weil es ihnen an den Mitteln dazu fehlt, so sind sie doch so
sehr davon eingenommen, dass sie mit Vergnügen ihren Gedanken freien Spielraum
lassen, ihnen beständig nachzuhängen. Dadurch werden dann natürlich ihre
Gemüter gänzlich abgeneigt gemacht, das göttliche Leben und die heiligen Lehren
Jesu zu betrachten.} Vornehmlich ist dieses aber, -- wie schon öfter bemerkt
ist,
-- der Fall mit jungen Gemütern, denen solche Vergnügungen, die ihre Sinne mit
neuen, ihren Neigungen angemessenen Reizen ansprechen, und nie zuvor gekannte
Gefühle der Eitelkeit in ihnen erwecken, viel lieber und schätzbarer, als
alles, was ihnen von der Furcht Gottes, von einem eingekehrten gottseligen
Leben, von ewigen Belohnungen und unaussprechlich herrlichen Freuden gesagt
werden kann. So sehr kann die Eitelkeit die Menschen verblenden, und so
unempfindlich macht dieselbe sie gegen alles was zur wahren Nachfolge Christi
gehört! O! Möchten sie es doch ernstlich erwägen! Möchten sie,
\textit{"`um der Zukunft
den Herrn"'} willen, gegen die eitlen Torheiten der Welt auf ihrer Hut stehen
und ihnen gänzlich entsagen, damit sie nicht, unvorbereitet und mit fremden
Gästen beschäftigt, von seiner ewigen Ruhe ausgeschlossen würden.

\section{5. Abschnitt} \label{kap17_ab5}

Was ferner noch die Unerlaubtheit der zahlreichen Moden, Gebräuche und
Ergötzungen der Welt klar beweist, ist ihr Ursprung und ihr Zweck, da sie
entweder von eitlen, müßigen und ausschweifenden Menschen erfunden werden, die
mit ihrer Einführung keinen anderen Zweck verbinden, als ihre eigene Sinnlichkeit
zu befriedigen und andere zur Nachahmung solcher strafbaren Neuheiten zu reizen,
die zur Beförderung der Maßlosigkeit
\footnote{\texttt{'Üppigkeit' ersetzt durch 'Maßlosigkeit'}} und Thorheit dienen,
oder bloß erzwungene
Produkte verarmter Witzlinge sind, die zu solchen Erfindungen und Erdichtungen
ihre Zuflucht nehmen, um sich ihren Unterhalt zu verschaffen. In beiden Fällen
aber verdienen sie Verabscheuung. Denn in der ersten Hinsicht bahnen sie den Weg
zu offenbaren Lastern, und in der anderen unterstützen sie einen schändlichen
Broterwerb\index{Beruf!schändlicher} und halten fähige Menschen von erlaubten,
nützlichen und
notwendigen Beschäftigungen ab.

\medskip

Dass die jetzt in der Welt üblichen Moden, Gebräuche und Ergötzungen am Anfang
nicht bekannt waren, sondern eine Erfindung eitler Menschen neuerer Zeiten sind,
wird uns leicht von selbst einleuchten, wenn wir erwägen, wie
Adam\index{Personen:!Adam} und Eva\index{Personen:!Eva}
gekleidet waren, denen, wie wir lesen, Gott selbst Kleider von Tierfellen
machte, und wenn wir in der Schrift nachsuchen, was von allen diesen Eitelkeiten
und Torheiten unter den heiligen Männern\index{Personen:!Männer, heiligen} und
Frauen\index{Personen:!Frauen!heilige} der Vorzeit anzutreffen
war. Ich möchte wohl fragen, wieviel Band, was für Federn, Spitzen und andere
Zierrat Adam und Eva im Paradies, und auch nach ihrer Vertreibung aus
demselben, an sich trugen? Und was für reiche, prachtvoll gestickte und besetzte
Kleider hatten Abel\index{Personen:!Abel}, Enoch\index{Personen:!Enoch},
Noah\index{Personen:!Noah} und der gute alte Abraham\index{Personen:!Abraham}?
Pflegten Eva, Sarah\index{Personen:!Sarah},
Susanna\index{Personen:!Susanna}, Elisabeth\index{Personen:!Elisabeth} und die
Jungfrau Maria\index{Personen:!Jungfrau Maria} sich zu frisieren, zu pudern, zu
schminken? Trugen sie schönfarbige falsche Locken, kostbare Spitzen, künstlich
gezierte Hauben, gestickte Kleider\index{Kleider, gestickte} mit langen
Schleppen? Schmückten sie ihren
Leib mit vielen Ellen oder ganzen Stücken Band, und ihre Schuhe mit goldenen und
silbernen Flittern, Schleifen u.s.w.? Bei welchen Schauspielen oder anderen
Ergötzungen waren Jesus und seine Jünger zugegen, um Erholung zu suchen? Was für
Märchen\index{Märchen}, Romane\index{Romane}, Komödien\index{Komödien} und
ähnliche Werke haben die Apostel\index{Personen:!Apostel} und Heiligen
geschrieben oder gelesen, um sich die Zeit zu vertreiben? -- So viel ich weiß,
ermahnten sie alle und jeden,
\textit{"`die Zeit zu erkaufen und schändliche Worte,
Possen, Scherz, leeres Geschwätz, erdichtete Erzählungen und dregleichen zu meiden,"'}
weil diese Dinge zu einem ungottseligen Leben führen. Sie empfohlen vielmehr
allen Menschen,
\textit{"`zu wachen, ihre Seligkeit mit Furcht und Zittern zu schaffen,
die törichten Lüste der Jugend zu fliehen, der Gerechtigkeit, dem Frieden, der
Sanftmut der Liebe und Barmherzigkeit nachzustreben und nach dem, was oben
im Himmel ist, zu trachten, wenn sie Ehre, Ruhm, unsterbliches Wesen und ewiges
Leben erlangen wollten."'}
\footnote{Epheser 5,1-5+15+16)
2. Timotheus 2,16+22)
Matthäus 25,13)
Philipper 2,12+13)
Kolosser 3,1+2+5.
Römer 2,6+7.}
\index{Bibelstellen:!Epheser 5)}
\index{Bibelstellen:!2. Timotheus 2)}
\index{Bibelstellen:!Matthäus 25)}
\index{Bibelstellen:!Philipper 2)}
\index{Bibelstellen:!Philipper 2)}
\index{Bibelstellen:!Kolosser 3)}
\index{Bibelstellen:!Römer 2)}

\section{6. Abschnitt} \label{kap17_ab6}

Fragt man mich, woher alle jene Torheiten zuerst kamen, so bin ich bereit zu
antworten: Sie entsprangen zuerst unter solchen Heiden, die Gott gar nicht
kannten, denn einige derselben verabscheueten sie, wie wir hernach hören werden.
Sie gehörten zu den Vergnügungen eines wollüstigen
Sardanapals\index{Personen:!}, eines
phantastischen Mirakles\index{Personen:!Mirakles}, eines komischen
Aristophanes\index{Personen:!Aristophanes}, eines verschwenderischen
Chararus\index{Personen:!Chararus}, eines maßlosen
\footnote{\texttt{'üppigen' ersetzt durch 'maßlosen'}}
Aristippus\index{Personen:!Aristippus} und zu den Gebräuchen solcher Frauen, wie
die
schändliche Clytemnestra\index{Personen:!Clytemnestra}, die geschminkte
Isabel\index{Personen:!Isabel}, die unzüchtige
Campaspe\index{Personen:!Campaspe}, die
freche Posthumia\index{Personen:!}, die berüchtigte Lais von
Korinth\index{Personen:!}, die unverschämte Flora\index{Personen:!}, die
prachtliebende ägyptische Kleopatra\index{Personen:!} und die schamlose
Messalina\index{Personen:!}. Männer und
Frauen wie diese, die, mit unauslöschlicher Schande gebrandmarkt, einen bösen
Geruch durch alle Zeitalter verbreitet haben, aber nicht die heiligen, sich
selbst überwindenden
\footnote{\texttt{'verleugnenden' ersetzt durch 'überwindenden'}} Männer und
Frauen der Vorzeit, waren solchen eitlen
Belustigungen ergeben. Ja, selbst die aufgeklärten
Heiden\index{Personen:!Heiden} verabscheuten sie,
und zwar -- wie allgemein zugestanden wird, -- aus sehr edlen, moralischen
Beweggründen. Wir finden keine Begünstigung derselben in
Platos\index{Personen:!Plato} oder in
Senecas\index{Personen:!Seneca} Werken. Pythagoras\index{Personen:!Pythagoras},
Sokrates\index{Personen:!Sokrates}, Phocion\index{Personen:!Phocion},
Zeno\index{Personen:!Zeno} und andere hatten sich an
solche Erholungen nicht gewöhnt. Die tugendhafte
Penelope\index{Personen:!Penelope}, die keusche
Lucretia\index{Personen:!Lucretia}, die ernste
Kornelia\index{Personen:!Kornelia} und mehrere ihresgleichen, fanden angenehme
Beschäftigungen genug unter ihren Kindern, Dienstboten und Nachbarn; und wiewohl
sie von hoher Abkunft waren, so bestand dennoch, nächst ihren Andachtsübungen,
ihr größtes Vergnügen in Spinnen, Weben, Nähen oder anderen häuslichen Arbeiten
und lobenswerten Verrichtungen. Diese, welche Heiden genannt werden, zeigten
in ihren Handlungen mehr wahres Christentum, als die jetzigen Verehrer und
Verteidiger der eitlen Moden, üppigen Gebräuche und sinnlichen Ergötzungen,
die, aller ihrern Toheiten ungeachtet, sich dennoch Christen nennen.

\medskip

Aber vor allen ihr Theaterliebhaber\index{Theaterliebhaber}! Woher, glaubt ihr,
dass die Lustspiele, die
euch so sehr ergötzen, entstanden sind? Da sie doch, -- obgleich sie unter allen
erdachten Erfindungen die verderblichsten sind, von euch so sehr geschätzt und
so gern und häufig besucht werden! Ich will es euch sagen: Ihr Stammvater war
ein Heide\index{Personen:!Heiden}, und zwar keiner der bessern Klasse. Er hieß
Epicharmus\index{Personen:!Epicharmus}. Man nennt ihn
freilich einen Philosophen\index{Philosophen} oder einen Verehrer der Weisheit,
er war es aber nur
dem Namen nach, und eben so wenig in der Wirklichkeit, als die heutigen
Schauspieler wahre Christen sind. Suidas\index{Personen:!Suidas}, ein
griechischer Geschichtschreiber,
erzählt von ihm, er sei der erste gewesen, der Komödien erfand, und habe auch
mit Hilfe eines gewissen Phormus\index{Personen:!Phormus} fünfzig Fabeln
gemacht. Wollt ihr auch sein
Vaterland und die Veranlassung zu seiner Erfindung kennen? -- Seine Vaterstadt
war Syrakus\index{Orte:!Syrakus}, die (ehemalige) Hauptstadt
Siziliens\index{Orte:!Sizilien}, welche wegen der schändlichen
Handlungen vieler ihrer Tyrannen merkwürdig war. Diesen durch Ergötzung ihrer
Sinne zu gefallen, strengte Epicharmus\index{Personen:!Epicharmus} seinen Witz
an, und erfand die Komödien.
Ist dieses nun aber nicht ein böser und verwerflicher Ursprung derselben? Und
ist es daher nicht eben so tadelnswert, wenn jemand solche Dinge nachahmt oder
zu rechtfertigen sucht, welche schon die Tugendhaften unter den Heiden
verwarfen? Ja, ist es nicht entsetzlich, wenn diejenigen, die sich Christen
nennen, an solchen Erfindungen Vergnügen finden, sie unterstützen und in Schutz
nehmen? -- Auch können die traurigen Trauerspiele sich keiner besseren Herkunft
rühmen. Ihr Erfinder war ein gewisser atheniensischer Dichter, namens
Thespis\index{Personen:!Thespis}, dem man auch die Einführung des unverschämten
Gebrauches, das Gesicht anzumalen,
und andere Personen, mittels Nachahmung ihrer Kleidung, Gebärden, Sprache etc.
vorzustellen, zuschreibt. Diesen beiden Erfindern muss ich aber noch einen
verliebten Dichter an die Seite setzen, den seine unbändige Leidenschaft so
hingerissen hatte, dass er sie in Versen laut werden ließ, welche die
schmutzigsten und niedrigsten Gesinnungen, wenn nicht gar Abgötterei, verrichten.
Er hieß Aleman\index{Personen:!Aleman} oder Aleina\index{Personen:!Aleina}, und
war ein Lydier\index{Personen:!Lydier}. Von ihm sagt man, dass er der
erste gewesen sei, der die Welt mit der Torheit Liebesgeschichten und
Liebesgedanken in Liedern vorzutragen, beschenkte, welche hernach alle Nationen
in ihren Romanzen so eifrig nachgeeifert haben.

\section{7. Abschnitt} \label{kap17_ab7}

\label{ref:17_07_einwand}
\textbf{Einige werden vermutlich hiergegen einwenden, dass es doch auch viele
Lustspiele, Trauerspiele, Sonetten, Lieder etc. gäbe, welche durchaus eine
Tendenz haben, das Laster zu bestrafen, und aus denen man viel Gutes lernen
können.} So nichtig dieser Einwurf auch ist, so haben mir ihn einige, entweder aus
Mangel an Einsicht oder aus Unverschämtheit, doch schon gemacht. Ich gebe
auch zu, dass dergleichen Vorstellungen unter den Heiden, nächst dem Lesen ihrer
ernsten Moral-Philosophen, dienliche Heilmittel gegen die im Schwange gehenden
Laster waren. Unter diesen will ich zwei Beispiele anführen:
Euripides\index{Personen:!Euripides}, dem
Suidas die Benennung eines tragischen Dichters beilegt, und
Eupolis\index{Personen:!Eupolis}, den
derselbe Schriftsteller einen komischen Poeten nennt. Der erste führte ein so
keusches Leben, -- und war also von der größern Anzahl unserer jetzigen Männer
so sehr verschieden, -- dass man ihn einen Frauenfeind\index{Personen:!Frauen!-feind} nannte,

\medskip

wiewohl er nur die ausschweifenden Frauen hasste, denn er war zweimal
verheiratet. Den anderen beschreibt Suidas\index{Personen:!Suidas} als einen
scharfen Tadler der Laster.
Hieraus schließe ich, dass es nicht die Absicht dieser Männer war, die Eitelkeit
und Torheit der Menschen zu nähren, oder mit ihren Arbeiten sich Geld zu
verdienen, sondern dass sie vielmehr, da das Volk durch die frechen
Ausschweifungen seiner Vorgänger verführt worden war, dasselbe von seinen
Torheiten wirklich dadurch wieder zurückzuführen suchten, dass sie das Laster
lächerlich machten und die Macht ihres Witzes gegen die Zügellosigkeit der
Menschen spielen ließen. Dieses wird auch aus der Beschreibung, die wir von
ihnen haben, um so wahrscheinlicher, da man von
Euripides\index{Personen:!Euripides\} erzählt, er sei von
unzüchtigen Weibern in Stücken zerrissen worden, ohne Zweifel darum, weil er
gegen ihre Frechheit und Schamlosigkeit eiferte. Und von
Eupolis\index{Personen:!Eupolis} sagt man, er
sei, als er in einer Schlacht zwischen den
Atheniensern\index{Personen:!Athenern} und Lacedämoniern
umgekommen war, so sehr bedauert worden, dass man ein Gesetz gemacht habe,
welches solchen Dichtern nicht mehr erlaubte, die Waffen zu ergreifen, ohne
Zweifel aus keiner anderen Ursache, als weil sonst mit ihnen die Bestrafer des
Lasters verloren gingen. Demnach hatten also die besseren komischen und
tragischen Dichter jener Zeit keinen anderen Zweck, als das Volk dadurch zu
bessern, dass sie das Laster verhaßt machten, und zwar nicht so sehr durch
vernünftige Beweisgründe, wie es bei ihren Philosophen üblich war, sondern durch
beißenden Spott\index{Spott} und empfindliche Rügen, oder durch bildliche
Vorstellungen, die
ihre lasterhaften Handlungen in ein lächerliches und abscheuliches Licht
stellten, damit sie dadurch gezwungen würden, um ihres guten Rufes willen
sich derselben nicht mehr schuldig zu machen. \textbf{Indessen sind alle diese
Mittel,
meiner Ansicht nach, doch nicht viel gelinder als die Geißel oder das
Zuchthaus\index{Zuchthaus}.}
-- Wollt ihr jedoch, die ihr dieselben verteidigt, es euch gefallen lassen, dass
man euch als Heiden\index{Personen:!Heiden}, und zwar als solche betrachte und
behandle, die eher durch
die Geißel des Spotts, als durch überredende Vernunftgründe von bösen Dingen
abzubringen sind, dann wollen wir euch einräumen, dass euere Lust- und
Trauerspiele einigen Nutzen für euch haben können. Aber dann auch, wenn euere
verderbten Neigungen so stark sind, dass ihr zu den elenden Hilfsmitteln der
Heiden euere Zuflucht nehmen müsset, um sie zu zügeln und einzuschränken, wenn
ihr euere bösen Wege nicht aus Liebe zur Tugend verlassen, dem Laster nur aus
Furcht vor der Schande oder des schlechten Rufes wegen entsagen wollt, o, dann
schämt euch doch auch, den Namen Jesu Christi so offenbar zu missbrauchen, dass
ihr euch Christen nennt. -- Steht es so mit euerer Liebe zu Jesu?
\index{Gottesdienst!unterhaltsamer}\textbf{Beweist dieses
euere Hochachtung vor der heiligen Schrift\index{Heilige!Schrift}, die den
Menschen Gottes durch den
Glauben an Christus vollkommen machen kann? Ist es mit allen eueren schönen
Reden von göttlichen Anordnungen, vom Gebet\index{Gebet}, von den
Sakramenten\index{Sakrament}, vom
Christentume, u.s.w. dahin gekommen, dass ihr euch solcher Belehrungsmittel
bedienen müsst, welche die tugendhaften Heiden zuließen, damit die
Verworfensten unter ihnen von ihren Lastern abgeschreckt werden möchten? Ja,
solcher Mittel, die nicht viel mehr als körperliche Züchtigungen ausrichten
können?}
\label{ref:17_07_einwand_ende}

\section{8. Abschnitt} \label{kap17_ab8}

Diese Dinge stimmen in der Tat nicht mit einem wahrhaft christlichen Charakter
überein, da es unter den edleren Heiden\index{Personen:!Heiden} schon Männer und
Frauen gegeben hat, die
besser unterrichtet waren und erhabenere Gesinnungen hegten. Diese kannten
Gegenstände von höherer Art und ewig dauernder Beschaffenheit, denen sie ihre
Betrachtungen und ihr Nachdenken widmeten. \index{Sitten!-wächter}
\label{ref:17_08_sittenwaechter}
\textbf{Ja, viele von ihnen übertrafen die
Christen unserer Zeit, indem sie sich durch ein ernstes, gesetztes Betragen
auszeichneten. Die Athenienser ernannten die Gynaecosmi, oder zwanzig Männer,
die auf die Kleidung und das Betragen des Volks ein wachsames Auge haben mussten,
und das Recht hatten, jeden, der sich unbscheiden oder unanständig aufführte,
zu bestrafen. Jetzt steht die Sache ganz anders. Wer solchen Leuten
Vorwürfe macht, wird strafbar, oder setzt sich wenigstens dem bittersten Tadel
und der größten Verhöhnung aus.} Einige sind so ruchlos und treiben ihre
Unverschämtheit so weit, dass sie nicht allein religiöse Personen, sondern sogar
die heiligsten Sachen zu Gegenständen ihres elenden Gespötts machen, und, nicht
damit zufrieden, ihre Geringachtung aller religiösen Grundsätze durch die
Ungebundenheit ihres eigenen Lebens an den Tag zu legen, ihre gänzliche
Verachtung der Religion auch dadurch beweisen, dass sie dieselbe durch komische
und niedrige Späße auf der öffentlichen Bühne zu entwürdigen und lächerlich zu
machen suchen. Wie gefährlich dieses ist, und wie leicht es die Wirkung hat, die
Religion in den Augen des Volkes herabzusetzen, beweist das Beispiel des
Aristophanes\index{Personen:!Aristophanes}, der kein geschickteres Mittel wußte,
den Ruf des Sokrates\index{Personen:!Sokrates} bei dem
Volk, das ihn seiner ernsten Lehren und seines tugendhaften Lebens wegen
hochschätzte, verdächtig zu machen, als dass er ihn in einem Schauspiel von
einer lächerlichen Seite darstellte, welches denn auch die Folge hatte, dass der
leichtsinnige, ausgelassene und unbändige Pöbel\index{Personen:!Pöbel} lieber
dem ernsten, als dem
lächerlich gemachten Sokrates den Rücken zukehrte. Auch kann man leicht
einsehen, dass die wahre Ursache, warum die sogenannten
\textit{Quaker}\footnote{\texttt{...man beachte die Schreibweise
'Quaker'! Heute wird im Deutschen zumeist 'Quäker' geschrieben. Wenn man schon ein eigenes 
Wort im Deutschen wollte, müsste man eigentlich 'Schüttler' benutzen. Es käme ja auch 
keiner auf die Idee, 'Iventmänäschement' statt 'Eventmanagement' zu schreiben.}}\index{Quaker} von
leichtsinnigen und ausschweifenden Menschen so sehr verspottet werden, bloß
darin liegt, dass sie das sündliche und eitle Leben der Menschen so ernstlich
rügen, und ihrer Unmäßigkeit in allen Arten weltlicher Vergnügungen durch ein
enthaltsames Leben der Selbstverleugnung beständig widersprechen. Denn alle jene
Freigeister\index{Personen:!Freigeister} wollen, ihres wüsten Lebens
ungeachtet, für gute Christen gelten,
während man uns für eigensinnige, eingebildete, tiefsinnige und finstere
Sonderlinge, ja, für Ketzer\index{Personen:!Ketzer}, Betrüger, und wer weiß für
was noch mehr hält. O der großen Verblendung und pharisäischen Heuchelei\index{Heuchelei!pharisäische}! 
Als wenn solche Menschen im
Stande wären, religiöse Gegenstände zu beurteilen, oder als wenn es möglich
wäre, dass sie einen richtigen Begriff und ein inneres Gefühl von wahrer Religion
haben könnten, während ihr Verstand von dem Gott der Weltfreuden verfinstert
und ihre Seele in äußeren Genüssen ganz versunken ist. Nein! ch sage euch im
Namen des ewigen Gottes: Ihr spottet seiner und betrügt eure eigenen Seelen!
Denn der Zorn des Allmächtigen ist gegen euch alle gekehrt, so lange ihr in
einem solchen Geist und Zustand verharrt! Umsonst sind alle euere leeren Worte
und eitlen Beobachtungen! Gott lacht und spottet eurer! Sein Zorn ist über euere
Untugend entbrannt! Darum lässt euch warnen und zur Mäßigkeit ermahnen, bereut
eure Abweichungen und bessert eueren Lebenswandel!

\section{9. Abschnitt} \label{kap17_ab9}

\label{ref:17_09_bedarf_wecken}
\index{Volkswirtschaft}\textbf{Überdies werden die leichtsinnigen und
ausschweifenden Erfinder der weltlichen
Torheiten durch den Beifall, womit ihre Erfindungen und Darstellungen
aufgenommen werden, in ihren Unternehmungen sehr bestärkt und aufgemuntert, und
folglich von ehrenvollen und nützlichen Beschäftigungen zurückgehalten.} Auch
sind aus keiner anderen Ursache viele notwendige Lebensbedürfnisse in so hohem
Preis, als weil die Arbeit so teuer ist, und dieses ist offenbar nur darum der
Fall, weil so viele Hände mit den Anschaffungen und Besorgungen eitler
Überflüssigkeiten beschäftigt werden. Ja, wie häufig geschieht es nicht, dass
jene Erfinder und Besorger der menschlichen Torheiten, wenn sie um Geld
verlegen sind, dem Publikum eine neue Mode anheften, die, ihrem Vorgehen nach,
mehr als eine andere auf Bequemlichkeit und Putz berechnet ist und welche oft
schon dann eingeführt werden muss, wenn die vorigen Sachen kaum zur Hälfte
verbraucht sind, die hernach weggegeben oder mit neuen Kosten nach der neuesten
Mode zugestutzt werden müssen. O welch' verschwenderische und doch so allgemein
übliche Torheit!

\section{10. Abschnitt} \label{kap17_ab10}

\index{Volkswirtschaft} Hier erwarte ich einer der scheinbarsten Einwendungen zu
begegnen, welche die
Verteidiger dieser Dinge gewöhnlich Vorbringen, wenn sie sich in die Enge
getrieben sehen. \label{ref:17_10_einwand} \index{Arbeit!ehrliche}\textbf{Sie sagen nämlich: Wovon sollen
denn die vielen Familien leben,
die ihren Unterhalt in der Besorgung der Moden, Gebräuche und Ergötzungen der
Welt finden, gegen welche ihr so ernstlich euere Stimme erhebt? Ich erwiedere: Es
ist ein schlechter Behelf, wenn man etwas Böses, sei es auch noch so gering,
darum verteidigt, weil ein guter Zweck dadurch erreicht wird. Findet ihr an
sündlichen Eitelkeiten ein Vergnügen, und ziehen jene aus der Besorgung
derselben einen Vorteil, so müsst ihr euch auch gefallen lassen, dass sie euch
beiden zur Pein und Strafe dienen, bis der eine ohne solche Torheiten zu leben
gelernt, und der andere eine ehrlichere Beschäftigung gefunden hat.} Es ist die
Eitelkeit der wenigen Großen, welche den vielen Kleinen so viel zu schaffen
macht, denn wenn jene nicht alle Schranken überschritten, so würden diese nicht
nötig haben, so hart für sie zu arbeiten. Wollten daher nur die Menschen mit
Wenigem, oder mit dem unentbehrlichen sich begnügen, wie die ersten Christen
taten, so würde manches bei weitem nicht so teuer und das Leben überhaupt viel
leichter zu erhalten sein. Hätten die Gutsbesitzer nicht so viel zur
Befriedigung ihrer Leidenschaften nötig, so brauchten ihre Pächter keine so
hohe Pacht zu bezahlen, könnten aus einem unbemittelten Stande sich zur
Wohlhabenheit hinaufarbeiten, und ihren Kindern ehrliche, häusliche
Beschäftigungen geben, wohingegen diese oft genötigt sind, sich auf Nebenwegen
in der Welt durchzuschlagen, und daher nicht selten zu unerlaubten oder
lasterhaften Erwerbsmitteln greifen.

\medskip

\index{Landwirtschaft}\index{Volkswirtschaft}Wenn wir einsichtsvollen
Landwirten Glauben beimessen wollen, so ließe sich der
Ertrag sehr vieler Ländereien noch bis aufs Doppelte vermehren, wenn es nicht an
tätigen Menschen dazu fehlte. Ebenso könnten auch noch mehrere Hände bei dem
Betriebe erlaubter und nützlicher Manufakturen angebracht werden. Das würde die
Fabrikate im Preis billiger machen, ihren Absatz vermehren und der ganzen Welt
Vorteil bringen. Dadurch aber, dass die Unterhaltung der städtischen Eitelkeit
dem Ackerbau und anderen nützlichen Gewerben so viele Hände entzieht, fällt die
Last um so viel schwerer auf den Landmann und arbeitsamen Fabrikanten. Wenn die
Menschen sich nie für reich genug halten, so fehlt es ihnen auch niemals an
Sorgen und Mühe. Wer aber den ursprünglichen Zustand der Schöpfung Gottes zu
seiner Richtschnur nimmt, der lernt sich mit Wenigem begnügen, indem er
einsieht, dass der Durst nach Reichtum\index{Reichtum} nicht allein den wahren
Glauben
untergräbt und zerstört, sondern auch denen, die zu seinem Besitz gelangt
sind, zum Fallstrick und zu einer Quelle vieler Unruhe dient. Es ist nicht
unrecht, Unrecht zu bereuen, doch kann es nicht eher dahin kommen, bis die
Menschen aufhören, das, was sie bereuen sollten, zu rechtfertigen. Auch ist es
in der Tat ein schlechter Beweggrund, wenn jemand darum keine Mäßigkeit übt,
und die Unmäßigkeit in Schutz nimmt, weil dadurch viele sich ihren Unterhalt
erwerben und die Erfinder und Verbreiter der maßlosen
\footnote{\texttt{'üppigen' ersetzt durch 'maßlosen'}} Moden und Gebräuche sonst
keine Nahrung haben würden. Menschen auf diese Weise erhalten, heißt, das Laster
füttern und nähren, statt ihm seine Nahrung zu entziehen. Würde es nicht
wohlgetan sein, wenn die reichen Besorger und Beförderer der Eitelkeit, die
sich schon viel mehr erworben haben, als sie brauchen, sich aus ihren Geschäften 
zurückzogen und anfingen, ihr Vermögen besser anzuwenden, als sie es erwarben,
indem sie wirkliche Arme damit unterstützen und ihnen zu besseren Beschäftigungen
behilflich wären? Gewiss, dieses wäre klüger, edler und auch christlicher, als die
Menschen zu Ausgaben für Tand und Torheiten zu verleiten. Öffentliche
Arbeitshäuser\index{Sozialer Wohnungsbau} würden treffliche Heilmittel gegen die
Ausbrüche dieses so
ansteckenden Übels der Üppigkeit sein, wobei auch jeder sowohl in seiner Kasse
als auch 
%wenn irgendwo "sowohl" steht, muss IMMER ein "auch" kommen. Kann man net einfach weglassen.
in seinem Gewissen sich besser stehen würde.

\medskip

Nach solchen Ansichten und Grundsätzen können und dürfen wir in unserem
Lebenswandel unter den Menschen uns nicht nach den herrschenden Gebräuchen der
Welt bequemen. Wir müssen vielmehr durch unsere Einfachheit und Mäßigkeit gegen
ihre eitle Verschwendung zeugen\index{Zeugnis}, und durch unser ernstes und
bestimmtes Betragen
zur Ehre Gottes an den Tag legen, wie sehr wir die verschwenderische Prachtliebe
und das zügellose Leben der Menschen missbilligen, ja, wir müssen manches, was
wir sonst wohl als erlaubt betrachten würden, und mit völliger Gleichgültigkeit,
oft selbst mit Vergnügen gebrauchen könnten, bloß nur des Missbrauchs willen uns
versorgen, der so allgemein damit getrieben wird.

\section{11. Abschnitt} \label{kap17_ab11}

\label{ref:17_11_einwand_2} \textbf{Einige sind ferner mit einem anderen Einwurf 
bereit, indem sie sagen:
Hat denn
Gott uns diese Lebensgenüsse bloß gegeben, um uns zu verdammen, wenn wir sie
gebrauchen?} Solchen armen, unwissenden Menschen, die lieber dem allerhöchsten
und heiligsten Gott die Erfindung oder Erschaffung ihrer törichten Eitelkeiten
zur Last legen, als dass es ihnen an Entschuldigungsgründen für den Gebrauch
derselben mangeln sollte, die aus Furcht oder Scham, oder aus Anhänglichkeit an
dieselben, nicht wissen, wie sie ihnen entsagen und sich davon losmachen sollen,
-- \label{ref:17_11_beteubung}\textbf{solchen Unglücklichen antworte ich: Alles, was Gott zum Gebrauche 
des
Menschen schuf, war gut, und was unser Heiland Jesus Christus erlaubt,
verordnet oder durch sein eigenes Beispiel empfohlen hat, muss beobachtet,
geglaubt und geübt werden. \index{Bibel!verbindlichkeit}
\footnote{Lukas  8,14) Lukas 12,28-31}
\index{Bibelstellen:!Lukas  8)}
\index{Bibelstellen:!Lukas 12)}
Allein ich
finde in dem ganzen Verzeichnis seiner Lehren und Gebote, welches die heilige
Schrift uns vorhält, keine solche Kleiderpracht, noch solche Ergötzungen oder
eine solche Lebensweise, als heutigen Tags unter den meisten Bekennern des
Christentumes üblich ist, weder geboten noch empfohlen.} Nein, wahrlich! Gott
schuf den Menschen zu einem heiligen, weisen, mäßigen und ernsten Wesen, und
begabte ihn mit Vernunft und der Fähigkeit, sich selbst und die Welt zu beherrschen.
Damals war die Erkenntnis Gottes der große Gegenstand seiner Betrachtung, seines
Nachdenkens und seiner Freuden; alle seine äußern Genüsse, die Gott ihm gab,
waren der Notwendigkeit, der Bequemlichkeit und der Erlaubtheit des Vergnügens
mit dem Vorbehalt untergeordnet, dass er in allem den Allmächtigen sehen,
fühlen, genießen und verehren sollte. Aber ach, wie weit sind die meisten
Christen von dieser ursprünglichen Verordnung abgewichen, die dessen ungeachtet
so hohe Ansprüche auf das Leben, das Verdienst und den Tod eines heiligern Jesus
machen, der nicht allein durch seine Erscheinung der Welt einen sicheren Beweis
von der Möglichkeit einer glücklichen Wiederherstellung des Menschen gegeben
hat, sondern auch allen seinen gnädigen Beistand verheißt, die ihm auf dem Weg
des heiligen Kreuzes und der Selbstüberwindung
\footnote{\texttt{'Selbstverleugnung' ersetzt durch 'Selbstüberwindung'}}
nachfolgen wollen, den er ihnen
als den einzigen Pfad zu ihrer ewigen Seligkeit vorgezeichnet hat. \textbf{Ob
nun aber
die Gemüter jener Menschen, beiderlei Geschlechts, nicht so tief in Torheiten und
Eitelkeiten versunken sind, dass sie den Herrn des Lebens nicht weiter als vom
Hörensagen kennen, und ob ihr begieriges Trachten nach niedrigere Dingen dieser
Welt nicht die Ursache ist, dass sie, des Genusses der der Gegenwart Gottes beraubt,
alle Gegenwart an göttlichen Freuden verloren haben, und daher sich
eingebildete Vergnügungen und immer neue Zerstreuungen ersinnen, um die
anklagende Stimme in ihrem Innern nicht zu vernehmen, oder dieselbe zu
übertäuben und so ihre Tage und Nächte ohne jene störenden Gefühle der Angst
und Unruhe, welche die unvermeidlichen Folgen ihrer Übertretungen sind,
gemächlicher und sicherer in dieser Welt zubringen zu können, -- ob dieses nicht
mit ihnen der Fall ist, möge ihr eigenes Gewissen ihnen
beantworten.}
\footnote{Römer 2,8+9}
\index{Bibelstellen:!Römer 2)}

\medskip

Die Versuchung Adams wird dadurch vorgestellt, dass er gereizt wurde, von der
Frucht eines Baums zu essen.
\footnote{1. Mose 3,6}
\index{Bibelstellen:!1. Mose 3)}
Dieses zeigt uns, was für
einen mächtigen Einflus äußere schöne oder reine Gegenstände auf unsere Sinne
haben. Ja, die Macht der sichtbaren Dinge ist in der Tat so hinreißend, dass
jeder, der nicht beständig in seinem Gemüth dagegen wacht, sehr leicht von
ihnen gefangen genommen wird; und ist er erst einmal von ihnen überwunden und
zum Sklaven\index{Personen:!Sklaven} gemacht, so verbreiten sie einen so dichten
und finsteren Schleier
über seine Seele, dass er sich selbst nicht mehr erkennt, und nicht allein die
Fesseln des maßlosen
\footnote{\texttt{'üppigen' ersetzt durch 'maßlosen'}} und eitlen Lebens mit
Vergnügen trägt, sondern sogar aus
seinen Sklavenstand\index{Sklavenstand} so stolz ist, dass er andere, die sich
demselben entziehen,
mit seinem Tadel überhäuft, indem er ihn als einen nützlichen und angenehmen
Stand verteidigt. Eine so sonderbare Leidenschaft erzeugt die Liebe zu den
vergänglichen Gegenständen des Vergangene der Welt in den Herzen derer, die ihr
Eingang erlauben\footnote{\texttt{'verstatten' ersetzt durch 'erlauben'}} und Nahrung gewähren!
\textit{"`Wir wissen aber, dass Jesus Christus,
der Sohn Gottes, gekommen und in uns geoffenbart ist, und er hat uns einen Sinn
gegeben und unseren Verstand erleuchtet, dass wir ihn, den Wahrhaftigen,
erkennen."'}
\footnote{1. Johannes 5,20}
\index{Bibelstellen:!1. Johannes 5)}
\textbf{Und er hat allen ein hinreichendes Maß seines
guten Geistes verliehen, welches, wenn sie ihm nur Gehorsam leisten wollen,
vermögend ist, ihre Seelen aus der Sklaverei der Eitelkeit zu erretten und von
der Herrschaft aller sinnlichen Gegenstände, welche die Augenlust, die
Fleischeslust und das hoffartige Leben nähren, gänzlich zu befreien,} so dass ihre
Herzen erneuert und wiedergeboren\index{Wiedergeboren} werden, ihre Neigungen
eine andere Richtung
bekommen und ihre ganze Seele sich mit den Dingen beschäftigt, die oben sind,
wo weder Rost noch Motten eindringen und ihre Schätze nicht zerstören können.

\section{12. Abschnitt} \label{kap17_ab12}

Es ist leicht einzusehen, was für Menschen es sein müssen, welche sich den
irdischen Freuden hingeben, und jene Überbleibsel der ägyptischen Zierraten in
Schutz nehmen. Sie müssenn das demütige, sanfte, einfache, heilige und
musterhafte Leben der Selbstüberwindung
\footnote{\texttt{'Selbstverleugnung' ersetzt durch
'Selbstüberwindung'}},\footnote{Galater 5,22+25) Epheser 5,8-11)
Epheser 15,16}
\index{Bibelstellen:!Galater 5)}
\index{Bibelstellen:!Epheser 5)}
\index{Bibelstellen:!Epheser 15)}
wozu der heilige Geist alle ihm gehorsame Herzen anleiten und fähig
macht, entweder nie gekannt oder sich von demselben wieder entfernt haben. Ja,
es leidet keinen Zweifel, dass alle solche Menschen dieses gute Land, dieses
himmlische Vaterland und selige Erbteil niemals recht sahen, oder, wenn sie auch
einen entfernten schwachen Blick davon hatten, doch wieder ganz aus dem Gesicht
verloren haben. O! \label{ref:17_12_einkehr}\textbf{Möchten sie sich doch einmal ruhig niedersetzen, bei sich
selbst
einkehren und ernstlich erwägen, wo sie sind und wessen Werk und Willen sie
tun!} Möchten sie doch einsehen, dass unter allen listigen Kunstgriffen des
Feindes ihrer wahren Glückseligkeit keiner für ihre Unsterblichen Seelen so
gefährlich ist als dieser, dass er ihre Sinne und Gedanken unaufhörlich mit den
törichten Moden und üppigen Vergnügungen der eitlen Welt beschäftigt! Grobe, in
die Augen fallende Laster, erregen gewöhnlich bei denen, die eine gute Erziehung
genossen haben und welche auf einen guten Ruf etwas halten, den größten
Abscheu. Daher greift der schlaue Feind, da er wohl einsiehet, dass er mit seinen
Versuchungen zu denselben bei vielen nichts ausrichten kann, zu seinem uns
verfänglichern Mitteln, indem er die Gemüter der Menschen zu Zerstreuungen und
Erholungen verleitet, die beim ersten Anblicke nicht so schädlich erscheinen,
weil sie weniger mit Schande verbunden sind, und indem sie erlaubte Vergnügungen
versprechen, auch stärker anziehen und desto sicherer zu seinem Zweck führen,
der kein anderer ist, als die Menschen durch immerwährende Beschäftigung mit
sinnlichen Gegenständen von einem ernsten Forschen und Trachten nach dem, was zu
ihrem ewigen Frieden dient, abzuhalten.
\footnote{Epheser 6,12-19}
\index{Bibelstellen:!Epheser 6)}
Auf solche
Weise sucht der arge Feind die Menschen in Ansehung der himmlischen Dinge in
beständiger Unwissenheit zu halten, damit sie das ewige Leben nicht kennen
lernen und folglich auch nicht danach ringen, sondern sich mit den
Beobachtungen leerer, von Menschen erfundenen und vorgeschriebenen
Religionsgebräuchen begnügen, wobei sie ihren gewöhnlichen Vergnügungen
ungestört nachgehen können, indem ihre Religion und ihr Lebenswandel
größtenteils miteinander übereinstimmen. Daher wissen sie denn auch nicht, was
es heißt
\textit{"`in der Erkenntnis Gottes wachsen, Gnade um Gnade empfangen und zu
dem vollkommnen Mannesalter Christi gelangen."'}
\footnote{Epheser 1,16-23) Epheser 4,12+13}
\index{Bibelstellen:!Epheser 1)}
\index{Bibelstellen:!Epheser 4)}
Was daher viele in ihrem siebten Jahre waren, das sind sie noch in
ihrem siebzigsten, nur nicht mehr so unschuldig, es sei denn, dass das alte
Sprichwort zutreffe, dass Greise wieder Kinder werden.

\medskip

Wahrlich! Das Geheimnis der Gottseligkeit\index{Gottseligkeit}, das göttliche
Leben, das wahre
Christentum sind Gegenstände von der höchsten Wichtigkeit! Und da wir sehen,
dass der Feind, wenn er die Menschen nicht zu groben Lastern verleiten kann,
allezeit geschäftig ist, ihre Gemüter mit anscheinend unschuldigen
Unterhaltungen zu beschäftigen und an sich zu ziehen, damit er sie um so
leichter an der Wahrnehmung ihrer Pflichten und an ihrem Wachstum in der
Erkenntnis des allein wahren Gottes, welche ewiges Leben ist,
\footnote{Johannes 17,3}
\index{Bibelstellen:!Johannes 17)}
verhindern, und ihre Gemüter von der Betrachtung himmlischer und
unvergänglicher Dinge gänzlich abziehen könne,\\textbf{so müssen alle, welche
seinen
Schlingen entgehen wollen, ihre Aufmerksamkeit auf die Erscheinung der
göttlichen Gnade\index{Gnade} in ihrem Innern richten, die sie lehrt und fähig
macht, das
ungöttliche Wesen zu überwinden
\footnote{\texttt{'verleugnen' ersetzt durch 'überwinden'}}, und allen eitlen
und bösen Dingen auf immer zu
entsagen.}
\footnote{Titus 2,11-15}
\index{Bibelstellen:!Titus 2)}
Dann wird ihr verbesserter Lebenswandel gegen
das unmäßige und ausschweifende Leben der Welt zeugen, dann werden sie zu der
Zahl der wahren sich selbst überwindenden
\footnote{\texttt{'verleundenden' ersetzt durch 'überwindenden'}} Jünger Jesu
gehören. Tun sie dieses
aber nicht, so werden sie die schrecklichen und verderblichen Folgen davon
erfahren.

\medskip

Dadurch, dass die Menschen ein Verlangen nach immer neuen Unterhaltungen äußern
und so viel Zeit und Geld darauf verwenden, geben sie nicht allein den Erfindern
und Verbreitern solcher Torheiten, die bloß zur Befriedigung eitler und üppiger
Neigungen dienen, große Aufmunterung, mit ihren Bemühungen fortzufahren, sondern
machen sich auch des Bösen dieser Beförderer der Maßlosigkeit
\footnote{\texttt{'Üppigkeit' ersetzt durch 'Maßlosigkeit'}} teilhaft, die auf
eine elende Art ihren Witz anstrengen und ihre eigene Zeit verschwenden, um neue
Zeitvertreibe für andere zu erfinden. Fänden hingegen solche Menschen nicht die
Unterstützung und Aufmunterung, die ihnen so allgemein gewähret wird, so könnten
Mangel und Dürftigkeit die Mittel werden, sie, wie den verlornen Sohn, an ihres
Vaters Haus zu erinnern. Denn, was auch einige davon halten mögen, so hatte doch
der Feind der wahren menschlichen Glückseligkeit\index{Glückseligkeit} unter
allen seinen
Verführungsmitteln nie angenehmere Lockspeisen, anziehendere Gegenstände,
gefälligere Unterhaltungen, listigere Abgesandte, einnehmendere
Prediger\index{Personen:!Prediger},
bezauberndere Vorträge und hinreißendere Redner, durch welche er die Menschen
anlocken und in sein Netz ziehen, und so ganz von aller Betrachtung göttlicher
Dinge entfernen konnte, als den Putz, die Ergözungen, Schauspiele und üblichen
Zeitvertreibe unseres verwilderten Zeitalters, welche als wahre Schulen der
Maßlosigkeit
\footnote{\texttt{'Üppigkeit' ersetzt durch 'Maßlosigkeit'}} und Werkstätte des
Verderbens, diese gerechte Rüge verdienen.





