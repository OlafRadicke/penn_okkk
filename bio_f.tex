\section*{F}

\articlesize

\begin{description}


 \item[Fell, Margaret] $\ast$1614 in Marsh Grange, \dag23.4.1702 in Swarthmore.
 Gelegentlich auch als \textit{"`Mother of Quakerism"'} (\textit{"`Mutter des
 Quäkertums"'}) bezeichnet. 1632 heiratete sie Thomas Fell ($\ast$1598
 \dag1658). Dieser war Richter und gehörte dem \textit{"`Long Parliament"'}
 unter $\to$Oliver Cromwell an. Im Juni 1652 traf sie zum ersten mal
 $\to$George Fox. Nach dem Tod ihres ersten Mannes heiratete sie am 27.10.1669
 George Fox in Bristol. Margaret Fell war Autorin unter anderem des
 $\to$\textit{"`histrischen Friedenszeugnisses"'} und maßgeblich mit der
 Organisation der Missionsreisen beschäftigt.
\medskip 
 Sie war unermüdlich damit
 beschäftigt Quaker aus der Gefangenschaft zu befreiehen und Rechtsbeistand vor
 gericht zu organisieren. Sie sammelte und verteilte Gelder um in Notgeratene
 und verfolgte Quakern zu helfen. Sie selbst war mehr mal im Gefängnis. Das
 erste mal 1663 für die Verweigerung des Eids vor Gericht. 1670
 sorgte sogar ihr eigener Sohn, wegen Erbstreitigkeiten, für ihre Inhaftierung.
 Bis 1684 kam sie immer wieder in gefängnis dafür, weil sie sich weigerte die
 örtlicher Gottesdienste zu besuchen (heute unvorstellbar, weil dann Heute ~80\%
 der Bundesbürger im Gefängnis sitzen würde). Die Bedeutung ihrer Person kann
 man daran sehen, das noch heute 500 an sie adressierte Briefe, aus den Jahren
 der schlimmsten Verfolgiung(1654 bis 1670) erhalten sind.

 \item[Finlayson, James] $\ast$vermutlich am 29. August 1772 in Penicuik,
 Schottland; \dag vermutlich am 18. August 1852 in Edinburgh. Schottischer
 Quäker, der wesentlich zum Aufbau der finnischen Textilindustrie beitrug. In
 inischen Tampere des Jahres 1820, beaufsichtigte er den Aufbau eines
 gewaltigen Industriekomplexes, dessen Werke von der Wasserkraft des gestauten
 Tammerkoski angetrieben wurden. Zunächst wurde eine Wollweberei errichtet,
 1828 dann eine Baumwollmühle und zwei Baumwollspinnereien. Aber auch
 karitative Projekte wie die Einrichtung eines Waisenhauses witmete er sich.
 Obwohl Finlayson schon früh kontakte zu Quäkern hatte, trat er selbst erst
 sehr späht der $\to$\textit{Gesellschaft} bei.


 \item[Fisher, Mary] $\ast$1623 \dag1698. Quakerin. Wollte 1657 zusammen mit
 fünf anderen Quakern den Sultan von Konstantinopel bekehren. Selbiger empfing
 sie auch freundlich, hörte sich an was sie zu sagen hatte und liess sie dann
 wieder nach hause fahren.


 \item[Fox, Geroge] $\ast$Juli 1624 in Drayton-in-the-Clay, Leicestershire,
 heute Fenny Drayton, \dag13.1.1691. War einer der Gründerväter der Quaker und
 erfolgreicher Missionar. Wie fiele Andere, sass fiele mal im Gefängnis. Sein
 wichtigstes Verdienst ist wohl, die administrative Grundlage zu schaffen, nach
 der sich bis Heute die Versammlungen organisieren. Sehr bekannt ist er auch
 als Autor der des "`The Journal of George Fox"', in den er seine Erlebnisse
 während seiner Missions-Jahre schildert.

  \item[Friedrich, Leonhard] $\ast$1889 \dag1979 $\to$\textit{Freund der
  Freunde}. Gelegendlich
  wird behaubtet Leonhard Friedrich sei Quaker gewesen, was aber nicht der
  Fall war. Während der der Jahre des Nationalsozialismus, bemühte er sich
  um den Erhalt des $\to$\textit{Quakerhauses} in Bad Pyarmond.\endnote{In
  dem Austellungskatalog der Austellung "`Stille Helfer - Die Quäkerhilfe
  im Nachkriegsdeutschland"' (12. Januar bis 12. März 1996, Zeughaus des
  Deutschen Historischen Museum), ist ein Dokument abgebildet, woraus
  die Religionszugehörigkeit von Leonhard Friedrich herforgeht.}

 \item[Fry, Elizabeth] $\ast$21.5.1780 in Cartham Hall bei Norwich,
 \dag12.10.1845 in Ramsgate. Quäkerin. War britische Reformerin des
 Gefängniswesens. Sie war die Tochter von $\to$\textit{John Gurney}

 \item[Fuchs, Emil] $\ast$13.5.1874 in Beerfelden/Odenwaldkreis; \dag13.2.1971
 in Ost-Berlin. Deutscher Theologe, Religiösen Sozialist,
 evangelisch-lutherischen Pfarrer und Quaker. 1933 verlor Fuchs seine
 Anstellung als Pfarrer und trat im selben Jahr der $\to$\textit{Deutschen
 Jahresversammlung} bei. In den 30ern war Fuchs in der Berliner Gruppe eine
 prägende Persönlichkeit gewesen, in seinen Predigten fanden viele Besucher
 geistigen Halt.


 \end{description}

\normalsize