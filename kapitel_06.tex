
\chapter{6. Kapitel}  \label{kap6}

\section{Zusammenfassung des 6. Kapitel}
\footnotesize
\begin{description}
\item[1. Abschnitt] Auch Menschen,. die in ihrem Glauben und in der Ausübung
ihrer religiösen Grundsätze mehr verfeinert sind, machen sich dennoch
hinsichtlich der Religion, der unerlaubten Eigenliebe schuldig. (Seite: \pageref{kap6_ab1})
\item[2. Abschnitt] Gott sicher beider Verehrung und Anbetung, die wir ihm
leisten, auf die Beweggründe, die wir dazu haben. (Seite: \pageref{kap6_ab2})
\item[3. Abschnitt] Wahre Gottesverehrung kann nur mit einem Herzen vollzogen
werden, das durch den Geist Gottes dazu vorbereitet ist. (Seite: \pageref{kap6_ab3})
\item[4. Abschnitt] Ohne den göttlicher: Athem des Lebens ist die Seele des
Menschen geistlich todt, und daher nicht im Stande, den lebendigen Gott gehörig
anzubeten. (Seite: \pageref{kap6_ab4})
\item[5. Abschnitt] Wir sollten keine Gebetsformeln ersinnen, nach denen wir
beten wollen. -- Wie die Christen beten sollen; und von dem Beistande, den ihnen
der Geist Gottes dabei leistet. (Seite: \pageref{kap6_ab5})
\item[6. Abschnitt] Wie die Vorbereitung des Herzens zur wahren Gottesverehrung
erlangt wird. Das Mittel dazu ist, dass man, wie David und Andere der Alten
gethan haben, im heiligen Schweigen auf Gott harre. Dadurch können wir sowohl
unsere Mängel, als auch unsere Hülfmittel, am besten erkennen. (Seite: \pageref{kap6_ab6})
\item[7. Abschnitt] Diejenigen, welche sich für gesund halten, oder von sich
selbst angefüllt und satt sind, glauben dieses Harrens nicht zu bedürfen, und
wenden es daher auch nicht an. Aber die Armen am Geiste denken anders; darum
erhöret sie auch der Herr, und füllet sie mit seinen Gütern. (Seite: \pageref{kap6_ab7})
\item[8. Abschnitt] Wäre eine solche
Vorbereitung des Herzens nicht notwendig würden die jüdischen Zeiten heiliger
und geistlicher gewesen seyn, als die Zeit des Evangeliums ist; sie war aber
schon damals erforderlich, und wie viel mehr muss sie es nicht jetzt sehn? (Seite: \pageref{kap6_ab8})
\item[9. Abschnitt] So wie die Sünde Gott nicht ehret, so kann es auch keine
leere Formalität; das sagt David, Jesaias, u. a. (Seite: \pageref{kap6_ab9})
\item[10. Abschnitt] Selbst die Formen und Einrichtungen, die Gott verordnet
hat, sind ihm missfällig, wenn sein eigener Geist sie nicht belebt: wievielmehr
werden es denn nicht die menschlichen Erfindungen seyn? (Seite: \pageref{kap6_ab10})
\item[11. Abschnitt] Gottes Kinder fanden Gott zu allen Zeiten auf seinem Wege,
nicht auf den ihrigen; und aus dem seinigen fanden sie auch allezeit Hülfe und
Trost. So war es zu den Zeiten Jeremia's; die Güte des Herrn offenbarte sich
seinen Kindern, die wahrhaft auf ihn harreten, und ein dürstendes Verlangen nach
einem innern Gefühle und Genusse seines Naheseyns hatten. Auch Christus befahl
seinen Jüngern, dass sie auf den heiligen Geist warten sollten. (Seite: \pageref{kap6_ab11})
\item[12. Abschnitt] Fernere Erklärung dieser Lehre vom stillen Harren auf den
Einfluss Gottes. Sie endigt mit einer Anspielung auf den Teich zu Bethesda,
einem treffenden Bilde des innern Harrene oder Wartens und der segenereichen
Wirkungen desselben. (Seite: \pageref{kap6_ab12})
\item[13. Abschnitt] Zur wahren Anbetung Gottes sind vier Stücke nothwendig: die
Heiligung des Anbeters; die Weihe seines Opfers; die Sache, nur welche er bittet
und endlich der Glaube, in dem er bittet. Alle diese Stücke müssen recht und gut
seyn, d. h. sie müssen aus dem Einflusse des Geistes Gottes entspringen. (Seite: \pageref{kap6_ab13})
\item[14. Abschnitt] Von der grossen Kraft des Glaubene im Gebete, wovon die
Anhaltsamkeit des Weibee, Matth. 15,28., zum Beweise dienet. -- Die Gottlosen
und Formalisien bitten, und empfangen nicht; -- Erklärung der Ursache, warum sie
vergeblich bitten; -- Jakob aber und seine Wahren Nachkommen, die Nachfolger
seiner Glaubens, erringen
den Seelen. (Seite: \pageref{kap6_ab14})
\item[15. Abschnitt] Dieses zeigt, warum Christus seinen Jüngern ihre
Kleingläubigkeit vorwarf. -- Nothwendigkeit des Glaubens; da Christus ohne
denselben kein Gutes in dem Menschen wirken kann. (Seite: \pageref{kap6_ab15})
\item[16. Abschnitt] Ein solcher Glaube ist auch jetzt nicht allein möglich,
sondern auch durchaus nothwendig. (Seite: \pageref{kap6_ab16})
\item[17. Abschnitt] Fernere Erklärung des wahren Glaubens. (Seite: \pageref{kap6_ab17})
\item[18. Abschnitt] Welches die rechtten Nachfolger dieses Glaubens sind, und
was für edle Werke derselbe unter den vorigen Geschlechtern der Gerechten
hervorgebracht hat. (Seite: \pageref{kap6_ab18})

\end{description}
\normalsize


\section{1. Abschnitt} \label{kap6_ab1}

Nun giebt es noch Andere, deren Begriffe mehr verfeinert sind, und welche ihre
Religionsübungen so weit
gereinigt und verbessert haben, dass sie dabei sich keiner Bilder von Holz,
Steinen, Silber oder Gold bedienen,
und noch weniger dieselben anbeten dürfen; die auch bei ihrer Gottesverehrung
sich den jüdischen oder vielmehr heidnischen Pomp nicht erlauben, wovon Andere
Gebrauch
machen. -- als wenn die Anbetung des Vaters, die Christus gelehret hat, von
dieser Welt seyn könnte, obgleich sein Reich von einer andern ist; -- ja, die
sogar in ihrer Lehre solchen abergläubischen Gebräuchen widersprechen, und
dennoch nicht einsehen, dass sie bei dem Allen sich vor ihrer eigenen hohen
Meinung beugen, die sie von ihren religiösen Uebungen haben, indem sie die
Vollziehung einiger Stücke ihres Gottesdienstes, die sie in ihrer fleischlichen
Ruhe und Bequemlichkeit störet, und die Pünktlichkeit, mit welcher sie dieselbe
beobachten, als
kein kleines Kreuz für sie betrachten. Solche stehen gewöhnlich in dem Wahne,
dass sie in dem Bezirke. der Nachfolge Jesu und im Schoosse der christlichen
Kirche sicher genug sind, wenn sie sich nur von groben und schändlichen Sünden
enthalten, oder das Böse nicht wirklich und mit der That ausüben, obgleich sie
den Gedanken an dasselbe Raum geben und in ihren Gemüthern freien Lauf lassen;
Allein auch dieses ist ein viel zu niedriger Begriff von dem Charakter und der
Wirkung des Kreuzes Christi, und es werden gewiss Alle, die sich mit dem
Gedanken schmeicheln, dass ein solches Aufnehmen und Tragen desselben
hinreichend sei, am Ende beim Mitternachtsgeschrei als Solche, die auf den Sand
gebauet haben, sich schrecklich betrogen finden; denn Christus erklärt: "`Ich
sage euch aber, dass die Menschen am Tage des Gerichts von jedem unnützen Worte,
das sie geredet haben, Rechenschaft geben müssen."'\footnote{Matth. 12,36}

\section{2. Abschnitt} \label{kap6_ab2}

Gott siehet bei den religiösen Handlungen der Menschen nicht auf die äussere
Vorrichtung derselben, sondern vielmehr auf die Triebfedern, die sie dazu
bewegen. Die Menschen können oft -- und Viele thun es -- ihren eigenen Willen
eigenwillig kreuzigen, indem sie willkührlich, oder nach eigener Wahl, Etwas
thun oder
unterlassen. Darum sagte einst der Herr zu den Juden, als sie ihm mit vielem
Eifer zu dienen schienen:
"`Wer fordert solches von euern Händen?"' \footnote{Jes. 1,2} Ihre
gottesdienstlichen Handlungen bestanden in Werken ihrer eigenen Erfindung und
Einrichtung, die sie nach ihrem eigenen Willen und sin ihrer selbst gewählten
Zeit, aber nicht mit einem von der Kraft Gottes wahrhaft gerührten und zu ihrem
Gottesdienste vorbereiteten Herzen vollzogen. Es waren bloss leibliche Uebungen,
von denen
Paulus uns sagt, "`dass sie wenig nützen."' Auch ist es offenbar eine
Hauptursache des Aberglaubens, der noch
immer die Welt beunruhigt, dass die Menschen sowohl in ihrer Gottesverehrung,
als auch in andern Dingen, das wahre Kreuz zu tragen aufgehört haben. Sie sind
eben, so abgeneigt, die Beschaffenheit ihres Gottesdienstes, als er den
sündlichen Zustand ihrer Herzen zu untersuchen. Ja,
die Mehrsten bekümmern sich um die Untersuchung ihres Gottesdienstes am
wenigsten; da sie aus Unwissenheit in
dem Wahne stehen, dass die Vollziehung desselben als eine Art von Genugthuung
oder Entschuldigung für ihre Uebertretungen diene, und glauben nicht, dass auch
die religiösen Handlungen des Menschen eines Kreuzes für seine eigene
Wirksamkeit bedürfen.

\section{3. Abschnitt} \label{kap6_ab3}

Wahre Gottesverehrung kann nur mit einem Herzen vollzogen werden, das vom Herrn
dazu vorbereitet ist. Diese Vorbereitung des Herzens kann aber nur
durch den Einfluss des göttlichen Geistes bewirkt werden, dessen Leitung, wenn,
nach Pauli Lehre, die Kinder Gottes dieselbe im gewöhnlichen Laufe ihres Lebens
bedürfen, ihnen gewiss bei der Verehrung und Anbetung ihres Schöpfers und
Erlösers ganz unentbehrlich ist; da kein Beten oder Lehren, das nicht aus dem
Einflusse des
heiligen Geistes entspringt, und unter der Leitung des selben hervorgebracht
wird, bei Gott Annahme findet. Auch kann eine solche Gottesverehrung, die ohne
gehörige Vorbereitung des Herzens vollzogen wird, nicht die Wahre evangelische
Anbetung seyn, welche nur im Geiste und in der Wahrheit, nämlich nur unter dem
Einflusse und durch Hülfe des göttlichen Geistes geleistet werden
kann. Denn was kann eine Menge der erhabensten und rührendsten Worte bei dem
allmächtigen Gott ausrichten, oder was sind geweihete Oerter und Zeiten vor
Ihm, der ein Geist ist, und mit dessen Wesen, -- wenn wir es genau betrachten,
-- Worte, Oerter und Zeiten in gar keinem Verhältnisse stehen. Wir bedienen uns
ihrer freilich als Hülfsmittel bei unserer öffentlichen Gottesverehruug; allein
sie sind doch nur körperliche und sichtbare Mittel, die keinesweges dem Anliegen
unserer
Seelen etwas beifügen, und dasselbe befördern oder den unsichtbaren Gott
empfehlen können. Sie sind auch nur
zum Dienste der versammelten Gemeine erforderlich; denn Gott hört die Sprache
der Seele an, und diese kann
nur auf eine geistige Art reden, nur durch Hülfe des heiligen Geistes aus die
rechte Weise zudem Allmächtigen seufzen.

\section{4. Abschnitt} \label{kap6_ab4}

So lebendig die Seele des Menschen auch in andern Dingen seyn mag, so befindet
sie sich doch in Ansehung des Lebens, das aus Gott ist, in einem Zustande des
Todes, bis der Allmächtige ihr den Geist des Lebens einathmet, ohne welchen sie
vor ihm nicht leben und noch weniger ihn recht anbeten kann. Dieses erklärt uns
Gott durch ''Hesekiel'', als derselbe ein Gesicht von der Wiederherstellung des
Menschengeschlechts hatte, und,
auf eine unter den Propheten übliche, aber oft auch missverstandene, Art zu
reden, unter der Benennung von Israel zu dem ganzen Volke des Herrn sagte: "`So
spricht der Herr: Ich will eure Gräber aufthun, ... und will meinen Geist in
euch geben, dass ihr wieder leben sollt."'\footnote{Hes. 37,12-15} Obgleich nun
auch Christus seines
Jünger lehrete, wie sie beten sollten, so waren sie doch, ehe er ihnen diesen
Unterricht ertheilte, in gewissem Betracht schon wirkliche Jünger, und keine
weltlichgesinnte Menschen, deren Gebet dem Herrn ein Greuel ist. Auch
lässt sich daraus, dass Christus seinen Jüngern die Gegenstände anzeigte, um
welche sie bitten sollten, nicht
vernünftig schliessen, dass jeder Mensch das Gebet, welches er seinen Jüngern zu
beten empfahl, nachbeten müsse, er möge es mit demselben Herzen oder in
derselben Gemüthsverfassung jener armen Nachfolger thun oder nicht, wie heutiges
Tages aus eine nur zu abergläubische und anmassende Weise geschiehet. Nein! so
wie Christus seine Jünger damals lehrete, wie sie beten sollten, müssen auch wir
jetzt nicht unsere eigenen oder selbst erdachten Gebete, sondern diejenigen, die
Er uns
lehret, hervorbringen; das heisst: wir müssen solche Gebete verrichten, als Er
in uns wirket, und wozu Er uns, wie ehemals seinen Jüngern, die Fähigkeit
verleihet.

\section{5. Abschnitt} \label{kap6_ab5}

Denn, wenn wir nicht sorgen sollen, wie, oder was wir reden wollen, wenn wir vor
weltliche Fürsten 
geführet werden; weil es uns zu der Stunde gegeben werden soll; und da wir es
dann nicht sind, die da reden, sondern der Geist unsers himmlischen Vaters es
ist, der in und durch und redet; wie viel weniger wird dann unsere eigene
Geschicklichkeit erforderlich seyn, oder was haben wie dann nöthig, eine Form
unserer Anrede
auszustudiren, wenn wir uns dem grossen Fürsten aller Fürsten, dem König aller
Könige, dem Herrn aller Herren nahen. Denn, wollten wir es seiner Grösse wegen
thun, so verbietet und dies Christus; oder, weil wir seine Kinder sind, so
bedarf es dessen nicht. Er weiss, was wir bedürfen, und wird, als Vater, uns
helfen, wenn er wirklich unser Vater ist. Demnach muss nicht allein der Mund des
Körpers, sondern auch der Mund
der Seele verschlossen bleiben, bis Gott ihn öffnet; und dann hört er gern seine
Sprache. Aber der Leib sollte hierin der Seele nicht vorgreifen; da das Ohr des
Herrn zu solchen Bitten sich neiget, und sein Geist die Flehenden mächtig
vertritt.

\section{6. Abschnitt} \label{kap6_ab6}

Man wird aber vielleicht fragen, wie eine solche Vorbereitung des
Herzens,erlangt werden könne?

\medskip

Ich antworte: Dadurch, dass man geduldig, wachsam und mit genauer Aufmerksamkeit
aus Gott harret. "`Das Verlangen der Elenden,"' sagt der Psalmist, "`hörest du,
Herr! Ihr Herz ist gewiss, dass dein Ohr daraus merket."'\footnote{Ps. 10,17}
Und die Weisheit sagt: "`Der Mensch setzt sich etwas vor im Herzen, aber vom
Herrn kommt, was die Zunge reden soll."'\footnote{Spr. Sal. 16,1} (Nach dem
Englischen: "`Die Vorbereitung dee Herzens im Menschen und die Antwort der Zunge
ist vom Herrn."') Der Mensch muss daher nicht seine eigenen Gedanken verfolgen,
noch seine eigenen Worte reden; das heisst: er muss unter der Zucht des heiligen
Kreuzes ein wahres Schweigen beobachten; er muss in seinem Innern von allen
verworrenen Einbildungen und zerstreuenden Vorstellungen sich abwenden, die
sonst in dieser heiligen Zurückgezogenheit nur zu häufig auf das Gemüth
einzudringen und dasselbe niederzudrükken pflegen. Denke ja nicht, der
Allmächtige sei durch wohlgeordnete Vorträge zu gewinnen, oder dass die
gewähltesten Ausdrücke etwas bei ihm ausrichten werden.Nein, das Seufzen und
Sehnen einer verwundeten Seele, ein Herz, von wahrer Reue durchdrungen, eine
aufrichtige und göttliche Traurigkeit, vom Geiste des Herrn gewirkt, dieses sind
Dinge, die bei Gott gelten, und Alles über ihn vermögen. Darum suche dein Gemüth
im Schweigen zu erhalten, und harte und warte, bis du etwas Göttliches in dir
fühlest, das dein Herz vorbereiten und geschickt machen kann, Gott in der
Wahrheit und aus eine ihm wohlgefällige Art anzubeten. Und wenn du so das Kreuz
gegen deine Selbstliebe aufnimmst, wenn du die Thüren und Fenster deiner Seele
gegen Alles verschliessest, was ein solches aufmerksames Harren auf Gott
unterbrechen könnte; möchte der Gegenstand, der deine stille Aufmerksamkeit
stört, an sich auch noch so angenehm, möchte er zu andern Zeiten auch noch so
erlaubt oder nothwendig seyn; dann wird die Kraft des Allmächtigen dich
überschatten, sein Geist wird dein Herz bearbeiten und zubereiten, dass es ein
wohlgefälliges Opfer darbringen kann. Er ist es, der der Seele ihre Mängel
entdeckt, und sie dieselben fühlen lässt, und wenn sie dann nach Hülfe seufzt
und schreiet, so ist Er es allein, der sie ihr gewährt. Diejenigen Gebete
hingegen, die nicht auf solchen Gefühlen entspringen, und ohne solche
Vorbereitung des Herzens verrichtet werden, sind bloss leere Formalitäten und
Täuschungen, und keine wahre Gebete; -- weil die Menschen sie in ihrem eigenen
blinden Verlangen hersagen und nicht dem Willen Gottes gemäss beten. Darum ist
auch sein Ohr dagegen verschlossen. "`Aber um des Seufzens der Armen und um des
Geschreies der Elenden willen,"' sagt Gott, "`wolle er sich
aufmachen,"'\footnote{Ps. 12,5} nämlich, um der geistlich armen und dürftigen
Seelen willen, die seiner Hülfe bedürfen; die zu versinken glauben; die ihre
Noth fühlen und laut nach einem Erretter rufen; ja, die auf Erden Niemand
finden, der ihnen zu helfen vermöchte, und nur an ihn im Himmel sich wenden
können. Von Solchen sagt David: "`Er wird den Armen erretten, der da schreiet,
und den Elenden, der keine Hülfe hat. Er wird die Seelen der Armen aus dem
Betruge und Frevel erlösen, und ihr Blut wird theuer geachtet werden vor
ihm."'\footnote{Ps. 72,12+14} Und ferner: "`Das Angesicht Derer, die ihn ansehen
und anlaufen, wird nicht zu Schanden."' "`Da dieser Elende rief, hörte ihn der
Herr, und half ihn aus allen seinen Nöthen. Der Engel des Herrn lagert sich um
die her, die ihn fürchten, und hilft ihnen aus."' Dann ladet er Alle ein, zu
kommen, "`zu schmecken und zu sehen, wie freundlich der Herr ist."'\footnote{Ps.
14,6-8} Ja, "`Er segnet die, welche den Herrn fürchten, sowohl die Kleinen als
die Grossen."'\footnote{Ps. 115,13}

\section{7. Abschnitt} \label{kap6_ab7}

Allein, was gehet dieses Diejenigen an, deren Seelen keinen Hunger haben oder
keinen Mangel empfinden? "`Die Gesunden bedürfen des Arztes
nicht;"'\footnote{Matth. 9,12} die Satten brauchen nicht nach Speisen zu
seufzen; die Reichen haben nicht nöthig um Beistand anzusprechen. Was haben also
Diejenigen, die ihre innern Bedürfnisse nicht fühlen, die weder Furcht noch
Schrecken in sich erfahren, die von der Nothwendigkeit der Kraft Gottes, ihnen
zu helfen, des Lichts seines Angesichts, sie zu trösten, nicht überzeugt sind,
-- was haben diese mit dem Gebete zu thun? Gewiss, ihre Andachtsübungen
scheinen, aufs gelindeste genonnnen, bloss ein ernsthafter Scherz zu seyn, den
sie vor dem Allmächtigen treiben; da sie Dasjenige, um welches sie bitten, weder
kennen Noch nöthig zu haben glauben, und es auch nicht einmal verlangen. Sie
bitten: "`Der Wille Gottes möge geschehen,"' und thun dennoch immer ihren
eigenen. Es ist ihnen freilich ein Leichtes, die Worte auszusprechen; aber die
Sache ist ihnen doch schrecklich. Sie bitten um göttliche Gnade, und
missbrauchen die, welche sie haben; -- um den Geist Gottes; widerstreben ihm
aber in ihren eigenen Herzen und verspotten ihn in Andern. Sie flehen Gott um
seine Barmherzigkeit und Güte an; allein sie fühlen kein wahres Bedürfniss
derselben; und ein dieser innern Unempfindlichkeit sind sie auch eben so
unfähig, Gott für das, was sie haben, zu loben, als um das, was sie nicht haben,
zu bitten. Aber  "`die, Welche nach dem Herrn fragen,"' sagt David, "`werden ihn
preisen; denn er sättiget die durstige Seele, und füllet die hungrige mit
Gütern."'\footnote{Ps. 22,26 Ps. 107,9} Auch hat er für die Armen und Elenden
noch Dieses aufbewahrt: "`Lass die (geistlich) Armen und Elenden deinen Namen
preisen! Preiset den Herrn, ihr, die ihr ihn fürchtet! Es ehre ihn aller Same
Jakobs."'\footnote{Ps. 74,21 Ps. 22,24} Jakob war ein schlichter Mann von
aufrichtigem Gemüthe; und Alle, die seine Gesinnung haben, gehören zu seinem
Samen. Und wenn sie gleich, wie er, in ihren eigenen Augen so arm wie ein
Würmlein sind, so empfangen sie doch Kraft, dass sie, auch wie er, mit Gott
ringen und obsiegen können.

\section{8. Abschnitt} \label{kap6_ab8}

Ohne Vorbereitung und Heiligung des Herzens durch diese Kraft ist aber kein
Mensch in einem geschickten Zustande, vor Gott zu erscheinen; sonst würde es ja
auch unter der evangelischen Einrichtung weniger Heiligkeit und Ehrfurcht zur
Verehrung Gottes erfordern, als unter der Anordnung des Gesetzes nöthig war; wo
alle Opfer besprengt werden mussten, ehe sie dargebracht wurden, und auch
Diejenigen, welche sie darbrachten, erst geheiligt wurden, Hehe sie vor dem
Herrn erschienen.\footnote{4. Mose 8.+9. 2 Chron. 29,31+34 und Chron. 30,16-17}
Wenn nun damals das Berühren eines todten oder unreinen Thieres die Menschen
unfähig mache, den Tempel zu betreten und Opfer darzubringen, ja, wenn es sie
sogar von der Gesellschaft der Reinen ausschloss, bis sie wieder besprengt und
geheiligt waren; wie können denn wir von der Gottesverehrung, die Christus in
den Zeiten des Evangeliums angeordnet hat, einen so niedigen Begriff haben, als
dass sie unzubereitete und ungeheiligte Opfer zuliesse? Oder wie können wir
annehmen, dass Diejenigen, welche entweder in Gedanken, oder mit Worten und
Handlungen sich täglich moralisch verunreinigen, im Stande seyn sollten, den
reinen Gott aus eine ihm wohlgefällige Weise zu verehren und anzubeten, ohne
dass zuvor ihre Herzen durch das Blut Jesu, welches das Gewissen von todten
Werken reinigt, besprenget oder gehörig zubereitet worden sind?\footnote{Petri
1,2  Ebr. 10,19 und 9,14 und 12,24} Dieses wäre ja für den gesunden Verstand ein
offenbarer Widerspruch; da es unmöglich ist, dass die Unreinen den Reinen, die
Ungeheiligten den Heiligen auf eine ihm wohlgefällige Art anbeten können. Es
bestehet allerdings eine heilige Verbindung und Gemeinschaft zwischen Christo
und seinen Nachfolgern, aber durchaus. keine zwischen Christo und Belial; so
auch nicht zwischen Ihm und Denen, die seinen Geboten nicht gehorchen, die nicht
unter seinem heiligen Kreuze ein Leben der Selbstverleugnung führen.

\section{9. Abschnitt} \label{kap6_ab9}

So wenig Jemand mit seinen Sünden Gott dienen kann, eben so wenig kann er ihn
auch durch Beobachtungund Vollziehung äusserer Religionsgebräuche verehren;
selbst dann nicht, wenn diese von Gottes eigener Anordnung sind. Dieses war die
Ursache, warum der Prophet Micha, indem er die Gefühle einer- geängstigten
Herzens darstellte, in den Worten ausbrach: "`Womit soll ich den Herrn
versöhnen? Mit Bücken vor dem hohen Gott? Soll ich mit Brandopfern und jährigen
Kälbern ihn versöhnen? Meinest du, der Herr habe Gefallen an viel tausend
Widdern, oder an Oehl, wenn es gleich unzählige Ströme voll wären? Oder soll ich
meinen ersten Sohn für meine Uebertretung geben? Oder meines Leibes Frucht für
die Sünde meiner Seele? -- Es ist dir gesagt, o Mensch! was gut ist, und was der
Herr von dir fordert; nämlich: Gottes Wort halten, Liebe üben und demüthig seyn
vor deinem Gott."' (Nach der englischen Bibelübersetzung: "`Er hat dir gezeigt,
o Mensch! was gut ist; und was fordert der Herr von dir, als dass du recht
handelst, Barmherzigkeit liebest und demüthig vor deinem Gott
wandelst."')\footnote{Micha 6,6.7.8.} Auch der königliche Prophet, der dieses
wohl fühlte, rief Gott mit diesen Worten an:"` O Herr! thue du meine Lippen auf,
dass mein Mund deinen Ruhm verkündige."' (Oder: "`so soll mein Mund dein Lob
verkündigen."').\footnote{Ps.51,16. 17. 18.}  Er wollte es also nicht wagen,
seine. Lippen selbst zu öffnen, weil er wohl wusste, dass diese allein Gott
nicht preisen könnten; und er giebt auch den Grund davon an: "`Denn du hast
nicht Lust zum Opfer, sonst wollte ich es dir wohl geben."' -- Wenn mein
äusserer Formendienst dir genügen könnte, so sollte es dir daran nicht mangeln.
-- "`Brandopfer gefallen dir nicht. Die Opfer, die Gott gefallen, sind ein
geängstigter, (zerbrochener) Geist; ein geängistigtes und zerschlagenes Herz
wirst du, o Gott! nicht verachten."'\footnote{Ps.51,16. 17. 18.} Und warum
nicht? Weil dieses Gottes eigenes Werk, die Wirkung seiner Kraft ist, und nur
seine eigenen Werke ihn preisen können. Zu demselben Zwecke redet Gott selbst
durch den Mund des Propheten ''Jesaias'', indem er den äussern Formendienst und
die Lippenverehrung der ausgearteten Juden tadelt: "`So spricht der Herr: Der
Himmel ist mein Stuhl und, die Erde meine Fussbank. Was ist es denn für ein
Haus, dass ihr mir hauen wollt? Oder welches ist der Ort, da ich ruhen soll?
Meine Hand hat Allee gemacht, was da ist, spricht der Herr. -- Ich sehe aber an
den Elenden, und den, der zerbrochenen Geister- ist, und der sich fürchtet vor
meinem Worte."'\footnote{Jes.66, 1. 2.} Hier sehen wir den wahren Anbeter, den
Gott selbst zu seiner Verehrung vorbereitet hat; den an Herz und Ohren
Beschnittenen , der nicht, wie jene stolzen Bekenner des Judenthums, dem
heiligen Geiste widerstrebet. Und war dieses nun damals
so unter dem Gesetze, welches doch die Vorrichtung äusserer in Schatten und
Bilder gehüllter, gottesdienstlicher Handlungen vorschrieb, wie können wir denn
jetzt, in den für die Ausgiessung des heiligen Geists bestimmten Tagen des
Evangeliums, Annahme bei Gott erwarten, wenn unsere Herzen zu der Verehrung und
Anbetung, die wir ihm leisten wollen, nicht durch seinen Geist gehörig
vorbereitet sind? Dieses dürfen wir keinesweges hoffen. Denn Gott ist noch immer
derselbe, der er ehemals war; und es können nur
Diejenigen seine wahren Anbeter seyn, die ihn in seinem eigenen Geiste anbeten.
Diese liebt und behütet er wie seinen Augapfel; die Andern aber verachtet er,
weil sie ihn mit ihren eigenwilligen Verrichtungen doch nur verhöhnen. Doch
lasst uns hören, was der Herr weiter zu jenem Volke sagte; denn es betrifft den
Zustand der Christenheit in unsern Tagen: "`Wer einen Ochsen schlachtet, ist
eben, als der einen Mann erschlüge; wer ein Schaf opfert, ist als der einem
Hunde den Hals bräche; wer Speisopfer bringt, ist als der, welcher Saublut
opfert. Wer des Weihrauchs gedenkt, ist als der
dass Unrecht lobt. Solches erwählen sie in ihren (eigenen) Wegen, und ihre Seele
hat Gefallen an ihren Greueln."'\footnote{Jes 66, 3.} Niemand sage: Wir bringen
auch dergleichen Opfer nicht; denn davon ist die Rede nicht. Gott zürnte nicht
über die Opfer, sondern über die Opfernden. Jene wurden nach einer gesetzlichen
Vorschrift dargebracht, die Gott selbst angeordnet hatte; da aber Diejenigen,
welche sie darbrachten, sich nicht in der dazu erforderlichen Geistesstimmung
und rechten Gemüthsverfassung befanden, so erklärte Gott, und zwar auf eine sehr
nachdrückliche Weise, seinen Abscheu dagegen; und an einem andern Orte verbietet
er sogar, durch denselben Propheten, alle fernere Opfer solcher Art. "`Bringet
nicht mehr Speisopfer so vergeblich,"' sagt Gott. "`Das Rauchwerk ist mir ein
Greuel; der Neumonden und Sabbathe, da ihr
zusammenkommt, und Mühe und Angst habt, mag ich nicht. Und wenn ihr gleich eure
Hände ausbreitet, verberge ich doch meine Augen vor euch; und ob ihr gleich viel
betet, höre ich euch doch nicht."'\footnote{Jes. 1, 13--15.}  Welch eine
furchtbare Verwerfung ihres Gottesdienstes! Und warum verwarf ihn der Herr? --
Weil ihre Herzen verunreinigt waren; weil sie den Herrn nicht von ganzem Herzen
liebten, sondern sein Gesetz übertraten, sich gegen seinen Geist empörten, und
nicht thaten, was in seinen Augen recht war. Dieses gehet sehr klar aus der
Veränderung und Besserung hervor, die er in den folgenden Worten verlangt:
"`Waschet, reiniget euch, thut euer Wesen vor meinen Augen hinweg. Lasset ab vom
Bösen, lernet Gutes thun. Trachtet nach Recht, helfet dem Unterdrückten,
schaffet den Waisen Recht und helfet (führet) der Wittwen Sache."'\footnote{V.
16. 17.} Unter diesen Bedingungen, die unerlässlich sind, ladet et nun Alle ein,
zu ihm zu kommen, wenn er ferner sagt: "`Wenn denn auch eure Sünden blutroth
sind, sollen sie doch schneeweiss werden; und wären sie wie Rosinsarbe, so
sollen sie doch wie (weisse) Wolle werden."'\footnote{V. 18.}

\medskip

Ebenso wahr ist jene merkwürdige Stelle, wo der Psalmist sagt: "`Kommet her,
höret zu Alle, die ihr Gott fürchtet; ich will erzählen, was er an meiner Seele
gethan hat. Zu ihm rief ich mit meinem Munde, und pries ihn mit meiner Zunge.
Wenn ich etwas Unrechts vorhätte in meinem Herzen, so würde der Herr mich nicht
hören. Nun aber erhöret mich Gott und merket auf mein Flehen. Gelobet sei Gott,
der mein Gebet nicht verwirft, noch seine Güte von mir wendet."'\footnote{Ps.
66, 16-20.}

\section{10. Abschnitt} \label{kap6_ab10}

Es könnten noch mehrere dergleichen Stellen angeführt werden, welche das
Missfallen Gottes sogar an den von ihm selbst vorgeschriebenen Formen der
Gottesverehrung an den Tag legen, wenn nämlich diese nicht
unter. dem Einflusse seines Geistes und ohne die nöthige Vorbereitung des
menschlichen Herzens vollzogen wird, die nur durch ihn bewirkt und verliehen
werden kann. Die Nothwendigkeit einer solchen innern Bearbeitung unserer Herzen,
und das Mittel, sie zu erlangen, empfiehlt
uns aber unter allen andern Verfassern der heiligen Schrift keiner öfterer und
eindrücklicher durch sein eigenes Beispiel, als der ''Psalmist'', welcher, indem
er sich seiner grossen Fehltritte und der Veranlassungen zu denselben, wie auch
des Mittels, durch welches er Gott wohlgefällig ward, und Kraft und Trost von
ihm erhielt, fortwährend erinnert, sich öfters selbst auffordert und ermahnt,
auf Gott zu harten. "`Leite mich, Herr, in deiner Wahrheit,"' sagt er, "`und
lehre mich; denn du bist der Gott, der wir hilft. Auf dich harre ich
täglich."'\footnote{Ps. 25. 5.} Seine Seele war aus Gott gerichtet, von dem er
Errettung und Befreiung von den Schlingen und Vorführungen der Welt erwartete.
Dieses setzt eine innere Gemüthsübung, eine geistliche Aufmerksamkeit voraus,
die sich nicht mit der Beobachtung äusserer Formen beschäftigte, sondern nach
der innern göttlichen Hülfe aussah.

\medskip

''David'' musste in der That auch grosse Aufmunterung zu diesem innern Harren
aus den göttlichen Einfluss finden; da die Güte Gottes? ihn dazu einlud und ihn
darin stärkte. Denn er sagt selbst: "`Ich harrete (geduldig) auf den Herrn, und
er neigete sich zu mir und hörte mein Schreien. Er zog mich aus dem Schlamme und
stellte meine Füsse aus einen Felsen."'\footnote{Ps. 40, 1. 2.} Der Herr
erschien ''David'' in seinem Innern, seine Seele zu trösten, die auf seine Hülfe
harrete, und sie von den Versuchungen und Bekümmernissen zu befreien, die sie
überwältigen wollten, und er gewährte ihm Sicherheit und Frieden. Darum sagt er,
"`der Herr habe seine Tritte gewiss gemacht;"' das heisst: er habe sein Herz in
dem Wege der Gerechtigkeit bevestigt; denn zuvor befleckte ihn jeder Schritt den
er that; und er war kaum vermögend zu gehen, Ohne zu fallen; da er sich von
allen Seiten mit Versuchungen und Fallstricken umgeben sah. Aber er harrete
geduldig auf Gott, sein Herz zog sich, wachsam und aufmerksam aus das Gesetz des
Geistes Gottes, davon zurück, und da fühlte er, dass der Herrlich zu ihm
neigete. Das Geschrei seiner Seele, das aus einem wahren Gefühle seines Elendes
und des Bedürfnisses der göttlichen Hülfe hervorging, drang in den Himmel und
ward erhört. Da erfuhr David Rettung und Erlösung in der von Gott ersehenen,
nicht in der von ihm selbst gewünschten Zeit; und er empfing Kraft, alle seine
Kämpfe durchzugehen und alle seine Trübsale zu überstehen. Und dann, sagt er
uns, "`habe der Herr ihm ein neues Lied in seinen Mund gelegt, um unsern Gott zu
loben."'\footnote{Ps. 40,4} Dieses ward ihm also von Gott gegeben, und war
folglich nicht sein eigenes Werk.

\medskip

Zu einer andern Zeit hören wir ihn ausrufen: "`Wie der Hirsch schreiet nach
frischem Wasser, so schreiet meine Seele, o Gott! zu dir. Meine Seele dürstet
nach Gott, nach dem lebendigen Gott. Wann werde ich dahin kommen, dass ich
Gottes Angesicht schaue."'\footnote{Ps. 43,1+2} Dieses übertrifft alle aussere
Formalität und ist etwas, das sich nicht wie eine Aufgabe erlernen lässt. Wir
können aber daraus abnehmen, dass wahre Gottesverehrung ein inneres Wert ist;
dass die Seele in ihrer himmlischen Sehnsucht durch den himmlischen Geist
gerührt und erhoben werden muss; und dass die wahre Anbetung in der Gegenwart
Gottes verrichtet wirdy "`Wann werde ich dahin kommen, dass. ich Gottes
Angesicht schaue?"' oder "`vor Gottes Angesichte erscheine?"' David redet nicht
vom Erscheinen im Tempel mit äussern Opfern, sondern vom Erscheinen vor Gott, in
seiner heiligen Gegenwart, wo die Seelen der wahren Anbeter Gott schauen, wenn
sie vor ihm erscheinen; und dieser- ist die grosse Sache, nach welcher sie
verlangen und dürsten, und worauf sie harren. Aber die mehrsten Bekenner des
Christenthums sind hierin so sehr von Davids Beispiele abgewichen, dass es ihnen
sonderbar vorkommt, wenn dieser fromme Mann uns versichert: "`Wahrlich meine
Seele harret auf Gott!"' und dass er es seiner eigenen Seele zur Pflicht macht,
dieses zu thun, wenn er sagt: "`O meine Seele harre du nur auf Gott; denn er ist
meine Hoffnung."'\footnote{Ps. 62,6} Als sagte er: Kein Anderer kann mein Herz
zubereiten und meinen Bedürfnissen abhelfen. Daher erwarte ich Nichts von meinen
eigenen willkührlichen Religionsübungen, oder von der äussern leiblichen
Verehrung, die ich Gott darbringen kann. Dieses Alles hat keinen Werth, und kann
weder mir selbst etwas nützen, noch dem Allmächtigen wohlgefallen. Aber ich
harre auf ihn, dass er mir Stärke und Kraft Verleihe, mich so vor ihm
darzustellen, als es ihm am angenehmsten seyn wird; denn wenn er sich selbst ein
Opfer zubereitet, so wird es ihm gewiss auch angenehm seyn. Daher erwähnt er in
zwei Versen dreimal des Harrenes; "`Ich harre des Herrn; meine Seele harret. --
Meine Seele wartet (oder harret) auf den Herrn, von einer Morgenwache bis zur
andern."'\footnote{Ps. 130,5+6} (Nach der englischen Uebersetzung; "`mehr als
Diejenigen, die auf den Morgen warten."') Ja, mit so genauer Aufmerksamkeit und
so unermüdet harrte er, dass er an einem Orte sagt-: "`Das Gesicht vergehet mir,
indem ich so lange auf meinen Gott harre."'\footnote{Ps. 69,4} Er begnügt sich
nicht mit dem Hersagen einer gewissen Anzahl Gebete, oder mit der Verrichtung
der verordneten gottesdienstlichen Handlungen, oder mit einer beschränkten
Wiederhohlung derselben; nein! er lässt nicht nach, bis er den Herrn findet,
nämlich seine tröstliche Gegenwart, die seine Seele mit Liebe und Frieden
erfüllte.

\medskip

Dieses war nun aber nicht allein der Gebrauch ''Davids'', als eines mit einem
ausserordentlichen Einflusse des göttlichen Geistes begnadigten Mannes; denn er
redet davon, als von der Art der Gottesverehrung, die zu seiner Zeit unter dem
wahren Volke Gottes, dem geistlichen, mit der Beschneidung des Herzens bekannten
''Israels'', üblich war; "`Siehe!"' sagt er, "`wie die Augen der Knechte auf die
Hände ihrer Herren, und die Augen der Magd auf die Hände ihrer Frauen sehen,
also sehen unsere Augen auf den Herrn unsern Gott, bis er uns gnädig
ist."'\footnote{Ps. 123, 2.} Und an einem andern Orte: "`Unsere Seele harret auf
den Herrn; er ist unsere Hülfe und Schild. Ich will auf deinen Namen harren,
denn deine Heiligen haben Freude daran."'\footnote{Ps. 33, 20. u. 52, 11.} Auf
Gott zu harten, war also schon in jenen Tagen unter den wirklich gottseligen
Menschen gebräuchlich, und als das wahre Mittel bekannt, wodurch sie zum Genusse
der Gegenwart Gottes gelangten, und Fähigkeit erhielten, ihn auf die ihm
wohlgefälligste Art zu verehren und anzubeten. Daher fand sich auch ''David''
sowohl aus eigener Erfahrung der Vortheile, die ihm dieses Harren gewährte, als
auch wegen des nützlichen Gebrauchs, den die Heiligen zu seiner Zeit davon
machten, bewogen, es ebenfalls Andern zu empfehlen. "`Harre auf den Herrn!" sagt
er "`Sei getrost und unverzagt, und harre auf den Herrn."'\footnote{Ps. 27, 14.}
Harre nur im Glauben und in Geduld, so wird er gewiss zu deiner Errettung
erscheinen. Ferner: "`Vertraue auf den Herrn, und harre auf ihn in Geduld."' Das
heisst: Wirf dich ganz auf ihn! Gieb dich zufrieden, und harre in allen deinen
Bedrängnissen auf seine Hülfe. Du kannst dir nicht vorstellen, wie nahe er mit
seiner Hülfe Denen ist, die wahrhaft auf ihn harten. O! versuche es nur und habe
Glauben! Noch sagt er uns: "`Harre auf den Herrn und halte seinen
Weg."'\footnote{Ps. 37,34} Hier sehen wir die Ursache, warum so Wenige wahre
Fortschritte machen. Sie bleiben nicht in seinem Wege, und können daher nie
recht auf ihn harren. Indessen hatte David doch gute Gründe für Das, was er
sagte; da er so grossen Trost und so viele Vortheile in dem gesegneten Wege des
Herrn erfahren hatte.

\section{11. Abschnitt} \label{kap6_ab11}

Der Prophet Jesaias sagt uns, dass die Gläubigen im Volke, obgleich die
Züchtigungen des Herrn wegen ihrer Uebertretungen schwer auf ihnen lagen,
dennoch "`in dem Wege seiner Gerichte,"' unter Gefühlen seiner Bestrafungen und
seines Missfallens, "`aus ihn geharret haben,"' und dass "`das Verlangen ihrer
Seelen,"' -- welches die Hauptsache dabei ist, --  "`auf seinen Namen und
Gedächtniss gerichtet gewesen sei."'\footnote{Jes. 26,8} Sie waren es gern
zufrieden, Verweise und Züchtigungen von ihm zu erhalten, weil sie gesündigt
hatten, und es war ihr sehnliches Verlangen, dass er sich ihnen aus diese Weise
zu erkennen geben möchte. Aber erschien er ihnen denn nicht auch endlich in
seiner Barmherzigkeit?  -- Allerdings erkannten sie ihn auch darin, so dass sie
sagen konnten, "`Sehet! das ist unser Gott, auf den wir harren. Er wird uns
helfen.! das ist der Herr, auf den wir harten, damit wir in seinem Heile und
freuen und fröhlich sind."'\footnote{Jes. 25,9} Erfahrungen, welche die arme,
sinnliche Welt nicht kennt! O seliger Genuss! o köstliche Vertrauen! Dieses war
ein Harren im Glauben, welches obsiegte. Alle Gottesverehrung aber, die nicht im
Glauben verrichtet wird, ist nicht nur fruchtlos für die, welche sie vollziehen,
sondern auch missfällig in den Augen Gottes. Und dieser Glaube, der die
Eigenschaft hat, dass er das Herz, der wahren Gläubigen reinigt und ihnen den
Sieg über die Welt giebt, ist eine Gabe Gottes. Doch der Prophet sagt noch
ferner: "`Wohl Allen, die auf den Herrn harren!"' und warum? "`denn die, weiche
auf den Herrn harten, bekommen nene Kraft, dass sie nicht matt noch müde
werden."'\footnote{Jes. 30,18; 40,31} Gewiss eine grosse Aufmunterung, auf Gott
zu harren! aber es geht noch -- Weiter, wenn er sagt,  "`das von der Weit her
nicht gehöret noch mit Ohren vernommen sei, und kein Auge, ohne Gott, es gesehen
habe, was denen geschiehet, die auf ihn harren."'\footnote{Jes. 64,4} Sehet, das
ist das innere Leben und dies Freude der Gerechten, der wahren Gottersverehrer;
deren Geister vor der Erscheinung des Geistes Gottes in ihren Herzens sich
beugen; die Alles verlassen und verleugnen, wogegen derselbe zeuget, und Alles
ergreifen und einnehmen, wozu er sie leitet.

\medskip

Aber auch in Jeremia’s Zeiten harrten die wahren Anbeter auf Gott. (Denn er
versichert uns, "`der Herr sei freundlich gegen den, der auf ihn harret, und
gegen die Seele, die nach ihm fraget."'\footnote{Klagl. Jer. 3,25} So ermahnte
gleichfalls der Prophet Hofea die Kirche zu feiner Zeit, sich zu Gott zu wenden
und auf ihn zu harren: "`So bekehre dich nun zu deinem Gott!"' sagt er, "`halte
Barmherzigkeit und Recht, und hoffe, (nach dem Englischen: harre oder warte)
stets auf deinen Gott."'\footnote{Hos. 12,7} Auch Micha war sehr eifrig in
dieser guten Uebung, indem er mit Entschlossenheit sagt: "`Ich will auf den
Herrn schauen, und den Gott meines Heitz erwarten. Mein Gott wird mich
hören."'\footnote{Micha 7,7} So redeten die Kinder des Geistes, die ein
dürstendes Verlangen nach einem innern Gefühle von Gott hatten. Die Gottlosen
können so nicht sprechen; und auch Die nicht, welche ohne Vorbereitung des
Herzens beten, die durch stillen Harren erlangt wird. Es ward den Kindern
''Israels'' in der Wüste als Ursache ihres Ungehorsams und ihrer Undankbarkeit
gegen Gott zum Vorwurfe gemacht, dass sie nicht auf seinen Rath harreten. Und
wir können versichert seyn, dass es auch unsere Pflicht ist, deren Vollziehung
von uns erwartet wird; da Gott sie beim ''Zephanja'' fordert: "`Darum, spricht
der Herr, müsset ihr wieder auf mich harren, bis ich mich aufmache zu seiner
Zeit tc."'\footnote{Zeph. 3,8}. O! möchten doch Alle, die den Namen Gottes
bekennen, auf diese Weise harren, und es nicht wagen, sich zu seiner Verehrung
zu erheben, ohne feinen göttlichen Einfluss in ihren Herzen zu fühlen! Dann.
würden sie die Anregungen das Erheben seines Geistes zu ihrer Hülfe, Zubereitung
und Heiligung in ihrem Innern erfahren.

\medskip

Christus gebot seinen Jüngern ausdrücklich, nicht von Jerusalem zu weichen,
sondern auf die Verheissung des Vaters, auf die Taufe des Geistes zu
warten,\footnote{Ap. Gesch. 1,4} wodurch sie zur Verkündigung seines herrlichen
Evangeliums zubereitet und fähig gemacht werden sollten. Da erfolgte freilich
eine ausserordentliche Ausgiessung des heiligen Geistes zu einem
ausserordentlichen Werkes allein der hohe Grad einer Sache verändert doch ihre
Natur und Eigenschaft nicht; und wenn ein so langes Harren oder Warten und eine
solche Vorbereitung des Herzens durch den Einfluss des göttlichen Geistes
erforderlich war, die Jünger zum Predigen an die Menschen fähig zu machen, so
muss wenigstens ein gewisser Grad solcher Vorbereitung nothwendig seyn, um uns
die Fähigkeit zu verschaffen, mit Gott zu reden.

\section{12. Abschnitt} \label{kap6_ab12}

Ich will diese wichtige Lehre der heiligen Schrift, vom stillen Harren auf Gott,
mit jener Stelle im ''Johannes'' beschliessen, wo von dem ''Teiche Bethesda''
die Rede ist. "`Es war nämlich zu Jerusalem, bei dem Schafhause, ein Teich, der
auf Ebräisch ''Bethesda'' hiess, und fünf Hallen hatte. In diesen lagen viele
Kranke, Blinde, Lahme und Dürre, welche auf die Bewegung des Wassers warteten.
Denn es fuhr zu seiner Zeit ein Engel in den Teich herab, und bewegte das
Wasser. Wer nun, nachdem das Wasser bewegt war, zuerst hinein stieg, der ward
gesund; mit was für einer Krankheit er auch behaftet war."'\footnote{Joh. 5,2-4}
Dieses giebt uns nun eine genaue bildliche Darstellung von dem ganzen Sinne
dessen, was bisher über den Gegenstand des Harrens gesagt worden ist. Denn sowie
es damals ein äusseres geselzliches Jerusalem gab, so giebt es jetzt ein
evangelisches, geistliches Jerusalem, die Kirche Gottes, die aus den Gläubigen
bestehet. Der Teich in jenem alten Jerusalem stellte gewissermassen die Quelle
vor, die jetzt im neuen Jerusalem eröffnet ist. Jener Teich war für Solche, die
an körperlichen Krankheiten litten; diese Quelle ist für Alle, die an ihren
Seelen krank sind. Damals kam ein Engel, der das Wasser bewegte, um es
heilwirkend zu machen; jetzt erscheint der mächtige Engel der Gegenwart Gottes,
der diese geistliche Quelle mit heilbringendem Erfolge segnet. Diejenigen,
welche damals nicht auf die Ankunft des Engels warteten, und die Zeit seiner
Bewegung des Wassers nicht in Acht nahmen, sondern früher hineinstiegen, hatten
keinen Nutzen davon; so können auch jetzt Alle, welche die Bewegung des Engels
Gottes in ihren Gemüthern nicht abwarten, sondern mit ihren selbstgeformten
Andachtsübungen in ihrer selbstbestimmten Zeit vor Gott erscheinen, sicher
daraus rechnen, dass sie in ihren Erwartungen eines gesegneten Erfolges sich
getäuscht finden werden. So wie daher damals Diejenigen, welche. geheilet zu
werden begehrten, mit aller Geduld und Aufmerksamkeit aus die Bewegung des
Engels warteten, eben so thun dieses die wahren Anbeter Gottes auch jetzt; denn
sie fühlen das Bedürfnis; und seufzen nach dem Genusse seiner Gegenwart, die
ihre Seelen belebt, wie die Sonne die Pflanzen auf dem Felde. Diese haben aus
öftern Versuchen das Nutzlose ihrer eigenen Wirksamkeit eingesehen, und sind nun
zu dem wahren ''Sabbathe'' gelangt. Diese dürfen es nicht mehr wagen, ihre
eigenen Einfälle hervorzubringen, oder ungeheligtes Gebet zu opfern, noch
weniger aber leiblichen Gottesdienst zu verrichten, wobei die Seele wirklich
unempfindlich oder vom Herrn nicht vorbereitet ist. Diese müssen immer in dem
Lichte ''Jesu'' aus die nothwendige Vorbereitung ihrer Herzen warten, indem sie
eingekehrt und abgeschieden von allen Gedanken, weiche die geringste Zerstreuung
oder Unruhe in ihren Gemüthern erzeugen könnten, sich still verhalten, bis sie
die göttliche Bewegung des Engels gewahr werden, und es dem Geliebten ihrer
Seelen gefällt, zu erwachen, den sie vor seiner Zeit nicht wecken dürfen. Ja,
sie fürchten sich, in seiner Abwesenheit Andachtsübungen zu verrichten, von
denen sie wissen, dass sie ihnen nicht allein keinen Nutzen gewähren, sondern
sogar Tadel verdienen würden. "`Wer fordert Solches von euren
Händen?"'\footnote{Jes 1, 12. Kap. 28, 16.} sagt der Herr. "`Wer glaubt, der
fliehet nicht."' (Nach dem Englischen: "`der ''eilet'' nicht."') Diejenigen,
welche in der Abwesenheit ihres geistlichen Führers selbsterdachte
gottesdienstliche Handlungen verrichten, machen es nicht besser als jene
''Israeliten'', die in Moses Abwesenheit ihre goldenen Ohrringe in ein
gegossenes Bild verwandelten, aber auch mit ihren Bemühungen den Fluch des Herrn
sich zuzogen. Auch traf Jene vor Zeiten kein besseres Loos, "`die selbst ein
Feuer anzündeten, und, mit Flammen gerüstet, in dem Feuer wandelten, das sie
selbst angezündet hatten."' Denn Gott sagte ihnen, "`dass sie in Schmerzen
liegen müssten."'\footnote{Jes. 50, 11.} Es sollte ihnen nicht allein keinen
Nutzen bringen, sondern sogar ein Gericht vom Herrn nach sich ziehen; Kummer und
Seelenangst sollte ihr Theil seyn. Der sinnliche Mensch möchte freilich auch
gern beten, wiewohl er nicht harren kann, und auch keine Geduld hat, die
Vorbereitung seines Herzens abzuwarten; ja er möchte gern ein Heiliger seyn,
obgleich es ihm unerträglich ist, den Willen Gottes zu thun oder zu leiden. Mit
seiner Zunge lobt er Gott, und mit derselben Zunge fluchet er dem Menschen, der
nach dem Bilde Gottes gemacht ist.\footnote{Jak. 3, 8-10.}  Er nennt Jesum
seinen Herrn, aber nicht durch den heiligen Geist; er führet oft den Namen
''Jesu'' im Munde, ja, er beugt seine Knie vor diesem Namen; lässt aber nicht ab
von der Ungerechtigkeit.\footnote{Kor. 12, 3. 2 Tim. 2, 19.} Dieses ist Gott ein
Greuel.

\section{13. Abschnitt} \label{kap6_ab13}

Es giebt vier Stücke, die zur wahren Anbetung Gottes unumgänglich nothwendig
sind, und durch welche die Vollziehung derselben so ganz dem eigenen Vermögen
des Menschen entzogen wird, dass es nur nöthig zu seyn scheint, sie zu nennen,
um dieses einleuchtend zu machen. Das erste ist, die Heiligung, oder der durch
die Kraft Gottes belebte, gereinigte und geheiligte Gemüthszustand des Anbeters.
Das zweite ist die Einweihung oder nöthige Zubereitung des Opfers, wovon schon
oben ziemlich weitläuftig gehandelt worden ist. Das dritte ist die Sache, um
welche man bittet, die Niemand kennet, der nicht durch die Hülfe oder unter dem
Einflusse des heiligen Geistes betet; weshalb denn auch Niemand, ohne diesen
göttlichen Einfluss zu haben, recht beten kann. Dieses setzt der Apostel
''Paulus'' ausser allen Zweifel, wenn er sagt: "`Der Geist hilft unserer
Schwachheit auf. Denn wir wissen nicht, was wir bitten sollen, wie sichs
gebühret; sondern der Geist selbst vertritt uns  aufs Beste mit
unaussprechlichem Seufzen."'\footnote{Röm. 8, 26.} Menschen, die mit der Kraft
und den Wirkungen des heiligen Geistes unbekannt sind, können den Willen und die
Absicht Gottes; nicht erkennen, und ihm daher mit ihren Gebeten auch gewiss
nicht gefallen. Es ist nicht genug, zu wissen, dass wir Bedürfnisse und
Schickungen haben: wir sollten auch einsehen lernen, ob nicht oft dass, was uns
widerfährt, einen Segen für und enthält. Gott lässt über die Stolzen
Demüthigungen, über die Geizigen Verluste, über die Ungehorsamen Züchtigungen
ergehen; diese entfernen zu wollen, hiesse die Beförderung des Verderbens, nicht
des Heils der Seele suchen.

\medskip

Die sinnliche Welt betrachtet und beurtheilt Alles nur nach sinnlichen
Begriffen; und nur zu Viele, die für erleuchtet gehalten seyn wollen, sind
geneigt, den Fügungen der Vorsehung falsche Namen zu geben. So, pflegen sie, zum
Beispiele, Trübsale Gerichte, und Prüfungen, die köstlicher als das so sehr
geliebte Gold sind, Unglück zu nennen, indem sie, im Gegentheile, den
Beförderungen der Welt den Namen: Ehre, und dem Besitze ihrer Güter, den der
Glückseligkeit beilegen; da diese doch, wenn sie es auch in einen Falle seyn
mögen, in hundert andern, - wie sehr zu befürchten ist, -- von Gott als
Gerichte, wenigstens als Prüfungen, für ihre Besitzer, zugelassen werden. Es ist
daher schwer zu entscheiden, was der Mensch behalten, verwerfen oder begehren
soll; eine Aufgabe, die nur Gott seiner Seele auflösen kann, Denn da Gott unsere
Bedürfnisse besser kennt, als wir sie selbst einsehen, so kann er auch besser
uns, als wir ihm, sagen, was wir nöthig haben. Deswegen ermahnte Christus seine
Jünger, dass sie lange Gebete und Wiederhohlungen vermeiden sollten, indem er
ihnen sagte: "`Euer himmlischer Vater weiss, was ihr bedürfet, ehe ihr ihn
bittet."' \footnote{Matth. 6,7+8} Und darum gab er ihnen auch ein Muster des
Gebets; aber nicht, wie Einige der Meinung sind, dass er jenen menschlichen
Liturgien zum Texte dienen sollte; die unter allen religiösen Gebräuchen am
mehrsten durch Lange und Wiederhohlung sich auszeichnen; sondern ausdrücklich,
um diese zu tadeln und sie uns vermeiden zu lehren. Gesetzt aber auch, man wäre
über jene Bedürfnisse, welche den Gegenstand des Gebets ausmachen sollten,
einig; -- wiewohl dieses immer ein schwieriger Punkt sehn würde; -- so ist es
doch von noch grösserer Wichtigkeit, zu wissen, wie, als was man bitten soll; da
hierbei nicht sowohl der Gegenstand des Gebets, als die Gemüthsverfassung des
Betenden in Betrachtung kommt. Die Sache, um welche man bittet, kann recht und
gut, aber der Zustand, in dem man bittet, mangelhaft seyn. Obgleich also, - wie
schon erwähnt ist, -- Gott nicht bedarf, dass wir ihm unsere Bedürfnisse
anzeigen, da Er sie uns zu erkennen geben muss; so will er dennoch, dass wir
unser Anliegen vor ihm sollen kund werden lassen, damit wir ihn suchen mögen,
und er sich zu uns herablassen könne. Und wenn dieses nun geschiehet,
<!-- Beginn der Wörtlichen Rede nach engl. Original gesetzt. -->"`so will der
Herr den Elenden ansehen, der zerbrochenen Geistes ist, und der sich fürchtet
der seinem Worte."'\footnote{Jes. 66,2} Das sind die kranken Herzen, die
verwundeten Seelen, die Hungrigen und Durstigen, die Müden und schwer Beladenen,
die sich aufrichtig nach einem Helfer sehnen.

\section{14. Abschnitt} \label{kap6_ab14}

Doch ist dieses Alles zu einer vollkommnen evangelischen Anbetung Gottes noch
nicht hinreichend, wenn nicht auch das vierte Erforderniss dabei ist, nämlich:
der Glaube; der wahre köstliche Glaube der Auserwählten Gottes, der das Herz
reinigt, die Welt überwindet und der Sieg der Gläubigen ist.\footnote{1. Tim.
1,5; Ap. Gesch. 15,9; Tit. 1,1; 2 Petri. 1,1 ; 1. Joh. 5,4} Dieser muss das
Gebet beleben, und es durchdringend machen, wie bei dem anhaltenden cananaischen
Weihe, das sich nicht wollte abweisen lassen, und zu welcher Christus, indem er
sie zu bewundern schien, sagte: "`O Weib! dein Glaube ist
gross!"'\footnote{Math. 15,28} Dieser Glaube ist von der grössten Wichtigkeit
und Nothwendigkeit für uns, wenn wir wünschen, dass unser Gebet Annahme bei Gott
finden möge; wiewohl auch er nicht in unserer Macht stehel, sondern Gottes Gabe
ist, von dem wir ihn empfangen müssen. Ein Körnlein dieses Glaubens aber richtet
mehr aus, wirkt mehr Erlösung und Befreiung, und verschafft uns reichlichere
Gnade und Barmherzigkeit, als alles eigene Laufen, Wollen und Wirken, und alle
menschliche Erfindungen und leibliche Andachtsübungen nicht hervorbringen
können. Wenn wir dieses gehörig erwägen, so werden wir leicht die wahre Ursache
entdecken, warum die vielen gottesdienstlichen Verrichtungen und Uebungen, die
wir in der Welt wahrnehmen, den Menschen so wenig Nutzen bringen; da diese
offenbar keine andere ist, als weil es ihnen am wahren Glauben mangelt. Sie
bitten, und erlangen nicht;\footnote{Jak. 4,3} sie suchen, und finden nicht; sie
klopfen an, und es wird ihnen nicht aufgethan. Die Ursache davon liegt klar am
Tage: ihre Bitten sind nicht mit der reinigenden Kraft des Glaubens verbunden,
wie Jakobs Bitten waren, als er mit Gott rang und obsiegte. Und die Wahrheit
ist, leider! die, dass die Mehrsten noch in ihren Sünden leben und den Lüsten
ihrer Herzen folgen, den Ergözlichkeiten der Welt nachgehen und mit diesem
köstlichen Glauben ganz unbekannt sind, Diese Ursache des geringen Nutzens, den
das Predigen des Wortes Gottes bei Einigen in vorigen Zeiten hatte, giebt der
tiefblickende Verfasser der Epistel an die Hebräer
%<!-- im Original falsch "Ebräer" geschrieben -->
deutlich an, wenn er sagt: "`Das Wort der Predigt half Jenen nichts, weil
Diejenigen, die es hörten, nicht glaubten."'\footnote{Hebr. 4,2} -- Kann ein
Prediger wohl ohne Glauben recht und nützlich predigen? Nein! So kann auch noch
viel weniger Jemand ohne Glauben zu seinem wahren Nutzen beten. Denn wahre
Anbetung Gottes ist die erhabenste Handlung, der dass Gemüth des Menschen fähig
ist; und wenn bei den weniger erhabenen religiösen Verrichtungen der Glaube
nothwendig ist, so darf er gewiss bei dieser nicht fehlen.

\section{15. Abschnitt} \label{kap6_ab15}

Dieses kann bei Einigen ihre Bewunderung mässigen, warum Christus so oft seine
Jünger mit den Worten: "`O ihr. Kleingläubigen"' tadelte, und dennoch sagte,
dass ein so geringes Mass wahren lebendigen Glaubens, als einem Senfkörnchen,
einem der kleinsten Samenkörner, zu vergleichen ist, Berge versetzen könne. Als
wenn er gesagt hätte: es giebt keine Versuchung, die so mächtig wäre, dass der
wahre Glaube an ihn sie nicht überwinden könnte. Die wahre Ursache also, warum
Diejenigen, welche mit Versuchungen umgeben sind, in ihrer geistlichen Noth
keine Hülfe erfahren, kann nur die seyn, dass sie diesen kräftigen Glauben nicht
haben. Auch war derselbe vor Zeiten so nothwendig, dass Christus an manchen
Orten nicht viele mächtige Werke verrichten konnte, weil die Leute daselbst
keinen Glauben hatten; und wenn er durch seine Kraft an andern Orten Wunder
that, so wirkte doch der Glaube dabei mit; so dass es schwer zu bestimmen ist,
ob seine Kraft durch den Glauben, oder der Glaube durch seine Kraft die Wunder
verrichtete. Erinnern wir uns, was für merkwürdige Dinge ein wenig Tonerde und
Speichel, eine blosse Berührung des Saums seines Gewandes, ein paar Worte aus
seinem Munde, durch die Macht des Glaubens bei den Kranken
verrichteten.\footnote{Joh. 9,11; Lik. 8,47+48} "`Glaubet ihr, dass ich euch die
Augen öffnen kenne?"' sagte Christus zu den Blinden; -- "`ja Herr!"' antworteten
sie, und sahen.\footnote{Matth. 9,28-30} "`Glaube nur!"' sagte er zu dem
Obersten; er that’s, und seine Tochter erhielt das Leben wieder.\footnote{Mark.
5,36} Bei einer andern Gelegenheit sagte er: "`Wenn du glauben kannst?"' --
"`Ich glaube,"' schrie der bekümmerte Vater mit Thränen, "`Herr! hilf meinem
Unglauben!"'\footnote{Mark. 9,13-24} Da musste der böse Geist weichen, und der
Knabe ward gesund. Zu Einem sagte er: "`Gehe hin, dein Glaube hat dich gesund
gemacht."'\footnote{Mark. 10,52} Zu einem Andern: "`dein Glaube hat dir
geholfen; deine Sünden sind dir vergeben."'\footnote{Luk. 7,48-50} Und als seine
Jünger sich wunderten, wie schnell sein Ausspruch über den unfruchtbaren
Feigenbaum in Erfüllung gegangen war, sagte er ihnen zu ihrer Aufmunterung im
Glauben:  "`Wahrlich ich sage euch, wenn ihr Glauben habt, und nicht zweifelt,
so werdet ihr nicht allein Solches mit dem Feigenbaume thun, sondern wenn ihr zu
diesem Berge sagen werdet: hebe dich aus, und wirf dich ins Meer, so wird es
geschehen; und Alles, was ihr bittet im Gebet, wenn ihr glaubet, so werdet ihr
es empfangen."'\footnote{Matth. 21,21-22}. Schon diese eine Stelle überführt die
allgemeine Christenheit ihres grossen Unglaubens; indem sie bittet, aber nicht
empfängt.

\section{16. Abschnitt} \label{kap6_ab16}

Einige werden vielleicht sagen, es sei unmöglich, dass ein Mensch Alles, was er
bitten mag, erhalten könne. Es ist aber nicht unmöglich, dass der Mensch, der im
wahren Glauben an die Kraft Gottes und nach der Leitung seines Geistes betet,
alles Das, um welches er so bittet, empfange, und die Früchte des Glaubens sind
keinesweges unerreichbar für Diejenigen, welche wahrhaft an den Gott glauben,
der sie darreichen kann. Denn, als Jesus zu dem Obersten sagte: "`Wenn du
glauben kannst?"' fügte er hinzu: "`Alle Dinge sind möglich dem, der da
glaubt."'\footnote{Mastth 9,13} Hierauf werden Andere erwiedern, es sei
unmöglich, einen solchen Glauben zu haben.; denn die Ungläubigen möchten gern
ihren Mangel an Glauben damit entschuldigen, dass sie es für unmöglich halten,
ihn zu besitzen. Allein Christus widerlegt diese Einwendung vollkommen in seiner
Antwort, die er den Ungläubigen jenes Zeitalters in den Worten gab: "`Was bei
den Menschen unmöglich ist, das ist bei Gott möglich."'\footnote{Matth. 19,26
Lik. 18,17} Hirraus folgt nun, dass es eben so wenig bei Gott unmöglich ist,
einen solchen Glauben zu geben, als es gewiss ist, dass man ohne denselben
"`Gott unmöglich gefallen könne;"' wie der Verfasser der Epistel an die Hebräer
lehrer. Und wenn es also nicht möglich ist, dass Jemand ohne Glauben Gott
gefallen könne, so kann auch gewiss Niemand ohne diesen köstlichen Glauben
erhörlich zu Gott beten.

\section{17. Abschnitt} \label{kap6_ab17}

Vielleicht werden Einige auch fragen: Was ist denn dieser Glaube, der zur
Anbetung Gottes so nothwendig ist, und dem Menschen die grosse Wohlthat gewahrt,
dass sein Gebet Annahme bei Gott findet? Ich antworte: Dieser Glaube bestehet in
einer heiligen Ergebung in den Willen Gottes, und in einem festen Vertrauen aus
ihn, welches sich durch religiösen Gehorsam gegen seine göttlichen Forderungen
beweiset, und wodurch der Seele eine klare Ueberzeugung von Dem, was sie noch
nicht siehet, und ein gewisses Vorgefühl von dem Wesen der Dinge, aus welche sie
hoffet, nämlich, von der Herrlichkeit, die hernach geoffenbaret werden soll,
erlangt. Da nun dieser Glaube Gottes Gabe ist, so reinigt er auch die Herzen
Derer, die ihn empfangen; und der Apostel Paulus bezeuget, dass er nur in einem
reinen Gewissen wohne. Darum verbindet er an einem Orte "`ein reines Gewissen
mit ungeheucheltem Glauben,"' und an einem andern: "`Glauben mit einem guten
Gewissen."'\footnote{1. Tim. 1,5 und 3,9} Jakobus vereinigt "`Glauben mit
Gerechtigkeit,"'\footnote{Jak. 2,14 bis Ende} und Johannes mit dem Siege über
die Welt: "`Unser Glaube,"' sagt er, "`ist der Sieg, der die Welt überwunden
hat."'\footnote{1 Jak. 5,4+5}

\section{18. Abschnitt} \label{kap6_ab18}

Die Besitzer dieses Glaubens sind, -- wie Paulus erklärt, -- auch ohne äussere
Beschneidung, die wahren Kinder Abrahams; indem sie, dem Gehorsame des Glaubens
gemäss, in seinen Fussstapfen wandeln,\footnote{Röm. 4,12} wodurch allein die
Menschen ein Recht zu dieser Benennung erwerben können. Dieser Glaube erhebt
nicht nur über die Sünde, sondern auch über die Gerechtigkeit der
Welt.\footnote{Joh. 16, 9.10.} Aber Niemand kann ihn anders erlangen, als wenn
er durch die Kraft des Kreuzes Christi den Tod seiner Selbstheit erduldet, und
allein durch ''Christum'' sein ganzes Vertrauen auf Gott setzt.

\medskip 

Die merkwürdigen Thaten, die zu allen Zeiten durch diese göttliche Gabe des
Glaubens verrichtet wurden, sind gross und berühmt. Es würde mir aber an Zeit
fehlen, sie herzuzählen, da die ganze heilige Geschichte davon voll ist. Mag es
daher genügen, zu bemerken, dass durch den Glauben die heiligen Alten alle
Prüfungen erduldeten, alle Feinde überwunden, selbst Gott übermochten, seine
Wahrheit verbreiteten und verherrlichten, ihre Zeugnisse vollendeten und die
Belohnung der Gläubigen, eine Krone der Gerechtigkeit, empfingen, welche die
ewige Seligkeit der Gerechten ist.




