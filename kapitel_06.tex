

\chapter{6. Kapitel}  \label{kap6}

\section{Zusammenfassung des 6. Kapitel}
\footnotesize
\begin{description}
\item[1. Abschnitt] Auch Menschen, die in ihrem Glauben und in der Ausübung
ihrer religiösen Grundsätze mehr verfeinert sind, machen sich dennoch
hinsichtlich der Religion, der unerlaubten Eigenliebe schuldig.
\\(\textit{Seite \pageref{kap6_ab1}})
\item[2. Abschnitt] Gott siehet bei der Verehrung und Anbetung, die wir ihm
leisten, auf die Beweggründe, die wir dazu haben.
\\(\textit{Seite \pageref{kap6_ab2}})
\item[3. Abschnitt] Wahre Gottesverehrung kann nur mit einem Herzen vollzogen
werden, das durch den Geist Gottes dazu vorbereitet ist.
\\(\textit{Seite \pageref{kap6_ab3}})
\item[4. Abschnitt] Ohne den göttlichen Athem des Lebens ist die Seele des
Menschen geistlich todt, und daher nicht im Stande, den lebendigen Gott richtig
\footnote{\texttt{'gehörig' ersetzt durch 'richtig'}}
anzubeten.
\\(\textit{Seite \pageref{kap6_ab4}})
\item[5. Abschnitt] Wir sollten keine Gebetsformeln ersinnen, nach denen wir
beten wollen. -- Wie die Christen beten sollen; und von dem Beistande, den ihnen
der Geist Gottes dabei leistet.
\\(\textit{Seite \pageref{kap6_ab5}})
\item[6. Abschnitt] Wie die Vorbereitung des Herzens zur wahren Gottesverehrung
erlangt wird. Das Mittel dazu ist, dass man, wie David und Andere der Alten
gethan haben, im heiligen Schweigen auf Gott harre. Dadurch können wir sowohl
unsere Mängel, als auch unsere Hilfsmittel, am besten erkennen.
\\(\textit{Seite \pageref{kap6_ab6}})
\item[7. Abschnitt] Diejenigen, welche sich für gesund halten, oder von sich
selbst angefüllt und satt sind, glauben dieses Harrens nicht zu bedürfen, und
wenden es daher auch nicht an. Aber die Armen am Geiste denken anders; darum
erhöret sie auch der Herr, und füllet sie mit seinen Gütern.
\\(\textit{Seite \pageref{kap6_ab7}})
\item[8. Abschnitt] Wäre eine solche
Vorbereitung des Herzens nicht notwendig würden die jüdischen Zeiten heiliger
und geistlicher gewesen sein, als die Zeit des Evangeliums ist; sie war aber
schon damals erforderlich, und wie viel mehr muss sie es nicht jetzt sein?
\\(\textit{Seite \pageref{kap6_ab8}})
\item[9. Abschnitt] So wie die Sünde Gott nicht ehret, so kann es auch keine
leere Formalität; das sagt David, Jesaias, u. a.
\\(\textit{Seite \pageref{kap6_ab9}})
\item[10. Abschnitt] Selbst die Formen und Einrichtungen, die Gott verordnet
hat, sind ihm missfällig, wenn sein eigener Geist sie nicht belebt: wievielmehr
werden es denn nicht die menschlichen Erfindungen sein?
\\(\textit{Seite \pageref{kap6_ab10}})
\item[11. Abschnitt] Gottes Kinder fanden Gott zu allen Zeiten auf seinem Wege,
nicht auf den ihrigen; und auf dem seinigen fanden sie auch allezeit Hilfe und
Trost. So war es zu den Zeiten Jeremia's; die Güte des Herrn offenbarte sich
seinen Kindern, die wahrhaft auf ihn harreten, und ein dürstendes Verlangen nach
einem innern Gefühle und Genusse seines Naheseins hatten. Auch Christus befahl
seinen Jüngern, dass sie auf den heiligen Geist warten sollten.
\\(\textit{Seite \pageref{kap6_ab11}})
\item[12. Abschnitt] Fernere Erklärung dieser Lehre vom stillen Harren auf den
Einfluss Gottes. Sie endigt mit einer Anspielung auf den Teich zu Bethesda,
einem treffenden Bilde des innern Harrene oder Wartens und der segenereichen
Wirkungen desselben.
\\(\textit{Seite \pageref{kap6_ab12}})
\item[13. Abschnitt] Zur wahren Anbetung Gottes sind vier Stücke nothwendig: die
Heiligung des Anbeters; die Weihe seines Opfers; die Sache, um welche er bittet
und endlich der Glaube, in dem er bittet. Alle diese Stücke müssen recht und gut
sein, d. h. sie müssen aus dem Einflusse des Geistes Gottes entspringen.
\\(\textit{Seite \pageref{kap6_ab13}})
\item[14. Abschnitt] Von der grossen Kraft des Glaubene im Gebete, wovon die
Anhaltsamkeit des Weibes, Matthäus 15,28., zum Beweise dienet. -- Die Gottlosen
und Formalisien bitten, und empfangen nicht; -- Erklärung der Ursache, warum sie
vergeblich bitten; -- Jakob aber und seine Wahren Nachkommen, die Nachfolger
seiner Glaubens, erringen
den Seelen.
\\(\textit{Seite \pageref{kap6_ab14}})
\item[15. Abschnitt] Dieses zeigt, warum Christus seinen Jüngern ihre
Kleingläubigkeit vorwarf. -- Notwendigkeit des Glaubens; da Christus ohne
denselben kein Gutes in dem Menschen wirken kann.
\\(\textit{Seite \pageref{kap6_ab15}})
\item[16. Abschnitt] Ein solcher Glaube ist auch jetzt nicht allein möglich,
sondern auch durchaus nothwendig.
\\(\textit{Seite \pageref{kap6_ab16}})
\item[17. Abschnitt] Fernere Erklärung des wahren Glaubens.
\\(\textit{Seite \pageref{kap6_ab17}})
\item[18. Abschnitt] Welches die rechtten Nachfolger dieses Glaubens sind, und
was für edle Werke derselbe unter den vorigen Geschlechtern der Gerechten
hervorgebracht hat.
\\(\textit{Seite \pageref{kap6_ab18}})

\end{description}
\normalsize


\section{1. Abschnitt} \label{kap6_ab1}

\index{Gottesdiens!pompöser}Nun giebt es noch Andere, deren Begriffe mehr
verfeinert sind, und welche ihre
Religionsübungen so weit
gereinigt und verbessert haben, dass sie dabei sich keiner Bilder von Holz,
Steinen, Silber oder Gold bedienen,
und noch weniger dieselben anbeten dürfen; die auch bei ihrer Gottesverehrung
sich den jüdischen \index{Pomp!jüdischer} oder vielmehr heidnischen Pomp nicht
erlauben, wovon Andere
Gebrauch
machen. \index{Selbst!-betrug} -- als wenn die Anbetung des Vaters, die Christus
gelehret hat, von
dieser Welt sein könnte, obgleich sein Reich von einer andern ist; -- ja, die
sogar in ihrer Lehre solchen abergläubischen Gebräuchen widersprechen, und
dennoch nicht einsehen, dass sie bei dem Allen sich vor ihrer eigenen hohen
Meinung beugen, die sie von ihren religiösen Übungen haben, indem sie die
Vollziehung einiger Stücke ihres Gottesdienstes, die sie in ihrer fleischlichen
Ruhe und Bequemlichkeit störet, und die Pünktlichkeit \index{Pünktlichkeit}, mit
welcher sie dieselbe
beobachten, als
kein kleines Kreuz \index{Kreuz!kleines}für sie betrachten. Solche stehen
gewöhnlich in dem Wahn,
dass sie in dem Bezirke der Nachfolge Jesu und im Schoße der christlichen
Kirche sicher genug sind, wenn sie sich nur von groben und schändlichen Sünden
enthalten, \index{Sünde!kleine} \index{sündigen!in Gedanken} oder das Böse
nicht wirklich und mit der Tat ausüben, obgleich sie
den Gedanken an dasselbe Raum geben und in ihren Gemüthern freien Lauf lassen;
Allein auch dieses ist ein viel zu niedriger Begriff von dem Charakter und der
Wirkung des Kreuzes Christi, und es werden gewiss Alle, die sich mit dem
Gedanken schmeicheln, dass ein solches Aufnehmen und Tragen desselben
hinreichend sei, \index{Gericht!Jüngstes} am Ende beim Mitternachtsgeschrei als
Solche, die auf den Sand
gebauet haben, sich schrecklich betrogen finden; denn Christus erklärt:
\textit{"`Ich sage euch aber, dass die Menschen am Tage des Gerichts von jedem
unnützen Worte, das sie geredet haben, Rechenschaft geben müssen."'}
\footnote{Matthäus 12,36}

\section{2. Abschnitt} \label{kap6_ab2}

\index{Absichten, Wahre} \index{Äusserlichkeiten} Gott siehet bei den religiösen
Handlungen der Menschen nicht auf die äussere
Verrichtung derselben, sondern vielmehr auf die Triebfedern, die sie dazu
bewegen. \index{Gebote!Selbst auf erlegte} Die Menschen können oft -- und Viele
tun es -- ihren eigenen Willen
eigenwillig kreuzigen, indem sie willkührlich, oder nach eigener Wahl, Etwas
tun oder
unterlassen. Darum sagte einst der Herr zu den Juden\index{Personen:!Juden}, als sie ihm
mit vielem
Eifer zu dienen schienen:
\textit{"`Wer fordert solches von euern Händen?"'}
\footnote{Jesaja 1,2}
\index{Bibelstellen:!Jesaja 1)}
\index{Gottesdiensgestaltung} Ihre
gottesdienstlichen Handlungen bestanden in Werken ihrer eigenen Erfindung und
Einrichtung, die sie nach ihrem eigenen Willen und sind ihrer selbst gewählten
Zeit, aber nicht mit einem von der Kraft Gottes wahrhaft gerührten und zu ihrem
Gottesdienste vorbereiteten Herzen vollzogen. Es waren bloss leibliche Übungen,
von denen
Paulus uns sagt,
\textit{"`dass sie wenig nützen."'} Auch ist es offenbar eine
Hauptursache des Aberglaubens, der noch
immer die Welt beunruhigt, dass die Menschen sowohl in ihrer Gottesverehrung,
als auch in andern Dingen, das wahre Kreuz zu tragen aufgehört haben. Sie sind
eben so abgeneigt, die Beschaffenheit ihres Gottesdienstes, als er den
sündlichen Zustand ihrer Herzen zu untersuchen. Ja,
die Mehrsten bekümmern sich um die Untersuchung ihres Gottesdienstes am
wenigsten; da sie aus Unwissenheit in
dem Wahne stehen, dass die Vollziehung desselben als eine Art von Genugtuung
oder Entschuldigung für ihre Übertretungen diene, und glauben nicht, dass auch
die religiösen Handlungen des Menschen eines Kreuzes für seine eigene
Wirksamkeit bedürfen.

\section{3. Abschnitt} \label{kap6_ab3}

\index{Vorbereitung!des Herzens} \index{Heiliger Geis!Leitung}\textbf{Wahre
Gottesverehrung kann nur mit einem Herzen vollzogen werden, das vom Herrn
dazu vorbereitet ist.} Diese Vorbereitung des Herzens kann aber nur
durch den Einfluss des göttlichen Geistes bewirkt werden, dessen Leitung, wenn,
nach Pauli Lehre, die Kinder Gottes dieselbe im gewöhnlichen Laufe ihres Lebens
bedürfen, ihnen gewiss bei der Verehrung und Anbetung ihres Schöpfers und
Erlösers ganz unentbehrlich ist; da kein Beten oder Lehren, das nicht aus dem
Einflusse des
heiligen Geistes entspringt, und unter der Leitung des selben hervorgebracht
wird, bei Gott Annahme findet. \textbf{Auch kann eine solche Gottesverehrung,
die ohne
gehörige Vorbereitung des Herzens vollzogen wird, nicht die Wahre evangelische
Anbetung sein, welche nur im Geiste und in der Wahrheit, nämlich nur unter dem
Einflusse und durch Hilfe des göttlichen Geistes geleistet werden
kann.} \index{Gottesdienst!Zeiten und Orte}Denn was kann eine Menge der
erhabensten und rührendsten Worte bei dem
allmächtigen Gott ausrichten, oder was sind geweihete Orte und Zeiten vor
Ihm, der ein Geist ist, und mit dessen Wesen, -- wenn wir es genau betrachten,
-- Worte, Orte und Zeiten in gar keinem Verhältnisse stehen. Wir bedienen uns
ihrer freilich als Hülfsmittel bei unserer öffentlichen Gottesverehruug; allein
sie sind doch nur körperliche und sichtbare Mittel, die keinesweges dem Anliegen
unserer
Seelen etwas beifügen, und dasselbe befördern oder den unsichtbaren Gott
empfehlen können. Sie sind auch nur
zum Dienste der versammelten Gemeine erforderlich; \index{Gott!Erhörung}denn
Gott hört die Sprache
der Seele an, und diese kann
nur auf eine geistige Art reden\index{Bethen!geisig ohne Worte}, nur durch
Hilfe des heiligen Geistes aus die
rechte Weise zudem Allmächtigen seufzen.

\section{4. Abschnitt} \label{kap6_ab4}

\index{Seele!im toten Zustand} So lebendig die Seele des Menschen auch in andern
Dingen sein mag, so befindet
sie sich doch in Ansehung des Lebens, das aus Gott ist, in einem Zustande des
Todes, bis der Allmächtige ihr den Geist des Lebens einathmet, ohne welchen sie
vor ihm nicht leben und noch weniger ihn recht anbeten kann. Dieses erklärt uns
Gott durch Hesekiel, als derselbe ein Bild\footnote{\texttt{'Gesicht' ersetzt duch
'Bild'}} von der Wiederherstellung des
Menschengeschlechts hatte, und,
auf eine unter den Propheten \index{Personen:!Propheten} übliche, aber oft auch
missverstandene, Art zu
reden, unter der Benennung von Israel \index{Personen:!Israel, Das Volk} zu dem
ganzen Volke des Herrn sagte:
\textit{"`So spricht der Herr: Ich will eure Gräber auftun, ... und will meinen
Geist in
euch geben, dass ihr wieder leben sollt."'}
\footnote{Hes. 37,12-15}
Obgleich nun auch Christus seines
Jünger lehrete, wie sie beten sollten, so waren sie doch, ehe er ihnen diesen
Unterricht ertheilte, in gewissem Betracht schon wirkliche Jünger, und keine
weltlichgesinnte Menschen, deren Gebet dem Herrn ein Greuel ist. Auch
lässt sich daraus, dass Christus seinen Jüngern die Gegenstände anzeigte, um
welche sie bitten sollten, nicht
vernünftig schliessen, dass jeder Mensch das Gebet, welches er seinen Jüngern zu
beten empfahl, nachbeten \index{Gebet!nachbeten} müsse, er möge es mit
demselben Herzen oder in
derselben Gemüthsverfassung jener armen Nachfolger tun oder nicht, wie heutiges
Tages aus eine nur zu abergläubische und anmassende Weise geschiehet.
\textbf{Nein! so
wie Christus seine Jünger damals lehrete, wie sie beten sollten, müssen auch wir
jetzt nicht unsere eigenen oder selbst erdachten Gebete, sondern diejenigen, die
Er uns
lehret, hervorbringen; das heisst: wir müssen solche Gebete verrichten, als Er
in uns wirket, und wozu Er uns, wie ehemals seinen Jüngern, die Fähigkeit
verleihet.}

\section{5. Abschnitt} \label{kap6_ab5}

\index{Reden!aus dem Geist} \index{Beten!aus dem Geist} \index{Reden!unvorbereitet}Denn, wenn wir nicht sorgen sollen, wie, oder was wir reden
wollen, wenn wir vor
weltliche Fürsten
geführet werden; weil es uns zu der Stunde gegeben werden soll; und da wir es
dann nicht sind, die da reden, sondern der Geist unsers himmlischen Vaters es
ist, der in und durch und redet; wie viel weniger wird dann unsere eigene
Geschicklichkeit erforderlich sein, oder was haben wie dann nöthig, eine Form
unserer Anrede
auszustudiren, wenn wir uns dem grossen Fürsten aller Fürsten, dem König aller
Könige, dem Herrn aller Herren nahen. Denn, wollten wir es seiner Grösse wegen
tun, so verbietet uns dies Christus; oder, weil wir seine Kinder sind, so
bedarf es dessen nicht. Er weiss, was wir bedürfen, und wird, als Vater, uns
helfen, wenn er wirklich unser Vater ist. Demnach muss nicht allein der Mund des
Körpers, sondern auch der Mund
der Seele verschlossen bleiben, bis Gott ihn öffnet; und dann hört er gern seine
Sprache. Aber der Leib sollte hierin der Seele nicht vorgreifen; da das Ohr des
Herrn zu solchen Bitten sich neiget, und sein Geist die Flehenden mächtig
vertritt.

\section{6. Abschnitt} \label{kap6_ab6}

Man wird aber vielleicht fragen, wie eine solche Vorbereitung des
Herzens,erlangt werden könne?

\medskip

\index{Vorbereitung!des Herzens} Ich antworte: Dadurch, dass man geduldig,
wachsam und mit genauer Aufmerksamkeit
aus Gott harret.\index{Gott!Warten auf}
\textit{"`Das Verlangen der Elenden,"'} sagt der Psalmist,\textit{"`hörest du,
Herr! Ihr Herz ist gewiss, dass dein Ohr darauf merket."'}
\footnote{Psalm 10,17}
\index{Bibelstellen:!Psalm 10)}
Und die Weisheit sagt:
\textit{"`Der Mensch setzt sich etwas vor im Herzen, aber vom
Herrn kommt, was die Zunge reden soll."'}
\footnote{Weisheit Salomos 16,1}
\index{Bibelstellen:!Weisheit Salomos 16)}
(Nach dem
Englischen:
\textit{"`Die Vorbereitung dee Herzens im Menschen und die Antwort der Zunge
ist vom Herrn."'}) \index{Andacht!Gesteszustand} \textbf{Der Mensch muss
daher nicht seine eigenen Gedanken verfolgen,
noch seine eigenen Worte reden; das heisst: er muss unter der Zucht des heiligen
Kreuzes ein wahres Schweigen beobachten; er muss in seinem Innern von allen
verworrenen Einbildungen und zerstreuenden Vorstellungen sich abwenden, die
sonst in dieser heiligen Zurückgezogenheit nur zu häufig auf das Gemüth
einzudringen und dasselbe niederzudrücken pflegen. Denke ja nicht, der
Allmächtige sei durch wohlgeordnete Vorträge zu gewinnen, oder dass die
gewähltesten Ausdrücke etwas bei ihm ausrichten werden.Nein, das Seufzen und
Sehnen einer verwundeten Seele, ein Herz, von wahrer Reue durchdrungen, eine
aufrichtige und göttliche Traurigkeit, vom Geiste des Herrn gewirkt, dieses sind
Dinge, die bei Gott gelten, und Alles über ihn vermögen. Darum suche dein Gemüt
im Schweigen zu erhalten, und harte und warte, bis du etwas Göttliches in dir
fühlest, das dein Herz vorbereiten und geschickt machen kann, Gott in der
Wahrheit und aus eine ihm wohlgefällige Art anzubeten.} \index{Kreuz!aufnehmen}
Und wenn du so das Kreuz
gegen deine Selbstliebe aufnimmst, wenn du die Türen und Fenster deiner Seele
gegen Alles verschliessest, was ein solches aufmerksames Harren auf Gott
unterbrechen könnte; möchte der Gegenstand, der deine stille Aufmerksamkeit
stört, an sich auch noch so angenehm, möchte er zu andern Zeiten auch noch so
erlaubt oder nothwendig sein; dann wird die Kraft des Allmächtigen dich
überschatten, sein Geist wird dein Herz bearbeiten und zubereiten, dass es ein
wohlgefälliges Opfer darbringen kann. \textbf{Er ist es, der der Seele ihre
Mängel
entdeckt, und sie dieselben fühlen lässt, und wenn sie dann nach Hilfe seufzt
und schreit, so ist Er es allein, der sie ihr gewährt.} \index{Gebetserhörung}
Diejenigen Gebete
hingegen, die nicht auf solchen Gefühlen entspringen, und ohne solche
Vorbereitung des Herzens verrichtet werden, sind bloss leere Formalitäten und
Täuschungen, und keine wahre Gebete; -- weil die Menschen sie in ihrem eigenen
blinden Verlangen hersagen und nicht dem Willen Gottes \index{Gott!Willen Gottes}
gemäss beten. Darum ist
auch sein Ohr dagegen verschlossen.
\textit{"`Aber um des Seufzens der Armen und um des Geschreies der Elenden
willen,"'}
sagt Gott,
\textit{"`wolle er sich aufmachen,"'}
\footnote{Psalm 12,5}
\index{Bibelstellen:!Psalm 12)}
nämlich, um der geistlich armen und dürftigen
Seelen willen, die seiner Hilfe bedürfen; die zu versinken glauben; die ihre
Not fühlen und laut nach einem Erretter rufen; ja, die auf Erden Niemand
finden, der ihnen zu helfen vermöchte, und nur an ihn im Himmel sich wenden
können. Von Solchen sagt David:
\textit{"`Er wird den Armen erretten, der da schreiet,
und den Elenden, der keine Hilfe hat. Er wird die Seelen der Armen aus dem
Betruge und Frevel erlösen, und ihr Blut wird theuer geachtet werden vor
ihm."'}
\footnote{Psalm 72,12+14}
\index{Bibelstellen:!Psalm 72)}
Und ferner:
\textit{"`Das Angesicht Derer, die ihn ansehen und anlaufen, wird nicht zu
Schanden."'}
\textit{"`Da dieser Elende rief, hörte ihn der Herr, und half ihn aus allen
seinen Nöten. Der Engel des Herrn lagert sich um die her, die ihn fürchten, und
hilft ihnen aus."'}
Dann ladet er Alle ein, zu kommen,
\textit{"`zu schmecken und zu sehen, wie freundlich der Herr ist."'}
\footnote{Psalm 14,6-8}
\index{Bibelstellen:!Psalm 14)}
Ja,\index{Sehgen} \textbf{\textit{"`Er segnet die, welche den Herrn fürchten,
sowohl die Kleinen als
die Grossen."'}
\footnote{Psalm 115,13}}
\index{Bibelstellen:!Psalm 115)}

\section{7. Abschnitt} \label{kap6_ab7}

\index{Selbstzufriedenen} \index{Personen:!Reiche} Allein, was gehet dieses
Diejenigen an, deren Seelen keinen Hunger haben oder keinen Mangel empfinden?
\textit{"`Die Gesunden bedürfen des Arztes nicht;"'}
\footnote{Matthäus 9,12}
\index{Bibelstellen:!Matthäus 9)}
die Satten brauchen nicht nach Speisen zu
seufzen; die Reichen haben nicht nöthig um Beistand anzusprechen. Was haben also
Diejenigen, die ihre innern Bedürfnisse \index{Bedürfnisse!innern} nicht
fühlen, die weder Furcht noch
Schrecken in sich erfahren, die von der Notwendigkeit der Kraft Gottes, ihnen
zu helfen, des Lichts seines Angesichts, sie zu trösten\index{Trost}, nicht
überzeugt sind,
-- \textbf{was haben diese mit dem Gebete zu tun? Gewiss, ihre Andachtsübungen
scheinen, aufs gelindeste genommen, bloss ein ernsthafter Scherz zu sein, den
sie vor dem Allmächtigen treiben; da sie Dasjenige, um welches sie bitten, weder
kennen Noch nöthig zu haben glauben, und es auch nicht einmal verlangen.} Sie
bitten:
\textit{"`Der Wille Gottes möge geschehen,"'}
und tun dennoch immer ihren
eigenen\index{Eigenwille}\index{Floskeln}. Es ist ihnen freilich ein
Leichtes, die Worte auszusprechen; aber die
Sache ist ihnen doch schrecklich. Sie bitten um göttliche Gnade, und
missbrauchen die, welche sie haben; -- um den Geist Gottes; widerstreben ihm
aber in ihren eigenen Herzen und verspotten ihn in Andern\index{Spott}.
\index{Gebet!unaufrichtiges}Sie flehen Gott um
seine Barmherzigkeit und Güte an; allein sie fühlen kein wahres Bedürfniss
derselben; und ein dieser innern Unempfindlichkeit sind sie auch eben so
unfähig, Gott für das, was sie haben, zu loben, als um das, was sie nicht haben,
zu bitten. Aber  \textit{"`die, Welche nach dem Herrn fragen,"'} sagt David,
\textit{"`werden ihn preisen; denn er sättiget die durstige Seele, und füllet
die hungrige mit Gütern."'}
\footnote{Psalm 22,26) Psalm 107,9}
\index{Bibelstellen:!Psalm 22)}
\index{Bibelstellen:!Psalm 107)}
Auch hat er für die Armen und Elenden
noch Dieses aufbewahrt:
\index{Personen:!Arme}\textit{"`Lass die (geistlich) Armen und Elenden deinen Namen
preisen! Preiset den Herrn, ihr, die ihr ihn fürchtet! Es ehre ihn aller Same
Jakobs."'}\index{Personen:!Jakob}
\footnote{Psalm 74,21) Psalm 22,24}
\index{Bibelstellen:!Psalm 22)}
\index{Bibelstellen:!Psalm 74)}
Jakob war ein schlichter Mann von
aufrichtigem Gemüthe; und Alle, die seine Gesinnung haben, gehören zu seinem
Samen. Und wenn sie gleich, wie er, in ihren eigenen Augen so arm wie ein
Würmlein sind, so empfangen sie doch Kraft, dass sie, auch wie er, mit Gott
ringen und obsiegen können.

\section{8. Abschnitt} \label{kap6_ab8}

\index{Heiligung!des Herzens}\textbf{Ohne Vorbereitung und Heiligung des Herzens
durch diese Kraft ist aber kein
Mensch in einem geschickten Zustande, vor Gott zu erscheinen; sonst würde es ja
auch unter der evangelischen Einrichtung weniger Heiligkeit und Ehrfurcht zur
Verehrung Gottes erfordern, als unter der Anordnung des Gesetzes nöthig war; wo
alle Opfer besprengt werden mussten, ehe sie dargebracht wurden, und auch
Diejenigen, welche sie darbrachten, erst geheiligt wurden, Hehe sie vor dem
Herrn erschienen.}
\footnote{4. Mose 8.+9) 2 Chronik 29,31+34 und Chronik 30,16-17}
\index{Bibelstellen:!4. Mose 8)}
\index{Bibelstellen:!2 Chronik 29)}
\index{Bibelstellen:!2 Chronik 30)}
Wenn nun damals das Berühren eines todten oder unreinen Thieres die Menschen
unfähig mache, den Tempel zu betreten und Opfer darzubringen, ja, wenn es sie
sogar von der Gesellschaft der Reinen ausschloss, bis sie wieder besprengt und
geheiligt waren; wie können denn wir von der Gottesverehrung, die Christus in
den Zeiten des Evangeliums angeordnet hat, einen so niedigen Begriff haben, als
dass sie unzubereitete und ungeheiligte Opfer zuliesse? \textbf{Oder wie können
wir
annehmen, dass Diejenigen, welche entweder in Gedanken, oder mit Worten und
Handlungen sich täglich moralisch verunreinigen, im Stande sein sollten, den
reinen Gott auf eine ihm wohlgefällige Weise zu verehren und anzubeten, ohne
dass zuvor ihre Herzen durch das Blut Jesu, welches das Gewissen von todten
Werken \index{Werke!tote} reinigt, besprenget oder gehörig zubereitet worden
sind?}
\footnote{1. Petrus 1,2)  Hebräer 10,19 und 9,14 und 12,24}
\index{Bibelstellen:!1. Petrus 1)}
\index{Bibelstellen:!Hebräer 9)}
\index{Bibelstellen:!Hebräer 10)}
\index{Bibelstellen:!Hebräer 12)}
Dieses wäre ja für den gesunden Verstand ein
offenbarer Widerspruch; da es unmöglich ist, dass die Unreinen den Reinen, die
Ungeheiligten den Heiligen \index{Heilige} \index{Unheilige} auf eine ihm
wohlgefällige Art anbeten können. Es
bestehet allerdings eine heilige Verbindung und Gemeinschaft \index{Gemeinschaft!Heilige} zwischen Christo
und seinen Nachfolgern \index{Personen:!Nachfolger}, aber durchaus keine zwischen Christo
und Belial \index{Personen:!Belial}; so
auch nicht zwischen Ihm und Denen, die seinen Geboten nicht gehorchen, die nicht
unter seinem heiligen Kreuze ein Leben der
Selbstüberwindung\footnote{\texttt{'Selbstverleugnung' ersetzt durch
'Selbstüberwindung'}} führen.

\section{9. Abschnitt} \label{kap6_ab9}

\textbf{So wenig Jemand mit seinen Sünden Gott dienen kann, eben so wenig kann
er ihn
auch durch Beobachtung und Vollziehung äusserer Religionsgebräuche verehren;
selbst dann nicht, wenn diese von Gottes eigener Anordnung sind.}
\index{Gebote!Gottesgesetz anwendung}\index{Gott!Gottesgesetz anwendung} Dieses war die
Ursache, warum der Prophet Micha \index{Personen:!Micha}, indem er die Gefühle
eines geängstigten
Herzens darstellte, in den Worten ausbrach:
\textit{"`Womit soll ich den Herrn versöhnen? Mit Bücken vor dem hohen Gott?
Soll ich mit Brandopfern und jährigen Kälbern ihn versöhnen? Meinest du, der
Herr habe Gefallen an viel tausend Widdern, oder an Oehl, wenn es gleich
unzählige Ströme voll wären? Oder soll ich meinen ersten Sohn für meine
Übertretung geben? Oder meines Leibes Frucht für die Sünde meiner Seele? -- Es
ist dir gesagt, o Mensch! was gut ist, und was der Herr von dir fordert;
nämlich: Gottes Wort halten, Liebe üben und demüthig sein vor deinem Gott."'}
(Nach der englischen Bibelübersetzung:)\textit{"`Er hat dir gezeigt,
o Mensch! was gut ist; und was fordert der Herr von dir, als dass du recht
handelst, Barmherzigkeit liebest und demüthig vor deinem Gott
wandelst."'}
\footnote{Micha 6,6-8.}
\index{Bibelstellen:!Micha 6)}
Auch der königliche Prophet, der dieses
wohl fühlte, rief Gott mit diesen Worten an:
\textit{"` O Herr! thue du meine Lippen auf,
dass mein Mund deinen Ruhm verkündige."'} (Oder:)\textit{"`so soll mein Mund
dein Lob
verkündigen."'}.
\footnote{Psalm 51,17.}
\index{Bibelstellen:!Psalm 51)}
Er wollte es also nicht wagen,
seine. Lippen selbst zu öffnen, weil er wohl wusste, dass diese allein Gott
nicht preisen könnten; und er giebt auch den Grund davon an:
\textit{"`Denn du hast
nicht Lust zum Opfer, sonst wollte ich es dir wohl geben."'} -- Wenn mein
äusserer Formendienst dir genügen könnte, so sollte es dir daran nicht mangeln.
--
\textit{"`Brandopfer \index{Brandopfer} gefallen dir nicht. \textbf{Die Opfer,
die Gott gefallen, sind ein geängstigter, (zerbrochener) Geist; ein
geängistigtes und zerschlagenes Herz wirst du, o Gott! nicht verachten."'}}
\footnote{Psalm 51,16-18.}
\index{Bibelstellen:!Psalm 51)}
Und warum
nicht? Weil dieses Gottes eigenes Werk, die Wirkung seiner Kraft ist, und nur
seine eigenen Werke ihn preisen können. Zu demselben Zwecke redet Gott selbst
durch den Mund des Propheten Jesaias\index{Personen:!Jesaias}, indem er den
äussern Formendienst und
die Lippenverehrung der ausgearteten Juden \index{Personen:!Juden}tadelt:
\textit{"`So spricht der Herr: Der
Himmel ist mein Stuhl und, die Erde meine Fussbank. Was ist es denn für ein
Haus, dass ihr mir hauen wollt? Oder welches ist der Ort, da ich ruhen soll?
Meine Hand hat Allee gemacht, was da ist, spricht der Herr. -- Ich sehe aber an
den Elenden, und den, der zerbrochenen Geistes ist, und der sich fürchtet vor
meinem Worte."'}
\footnote{Jesaja 66,1+2.}
\index{Bibelstellen:!Jesaja 66)}
Hier sehen wir den wahren Anbeter, den
Gott selbst zu seiner Verehrung vorbereitet hat; den an Herz und Ohren
Beschnittenen\index{Beschneidung}, der nicht, wie jene stolzen Bekenner des
Judenthums\index{Judenthum}, dem
heiligen Geiste widerstrebet. Und war dieses nun damals
so unter dem Gesetze\index{Gebote!unter dem Gesetz}, welches doch die Vorrichtung äusserer
in Schatten und
Bilder gehüllter, gottesdienstlicher Handlungen vorschrieb, wie können wir denn
jetzt, in den für die Ausgiessung des heiligen Geists bestimmten Tagen des
Evangeliums, Annahme\index{Rechtfertigung} bei Gott erwarten, wenn unsere Herzen
zu der Verehrung und
Anbetung, die wir ihm leisten wollen, nicht durch seinen Geist gehörig
vorbereitet sind? Dieses dürfen wir keinesweges hoffen. Denn Gott ist noch immer
derselbe, der er ehemals war; und es können nur
Diejenigen seine wahren Anbeter\index{Gebet!richtiges} sein, die ihn in seinem
eigenen Geiste anbeten.
Diese liebt und behütet er wie seinen Augapfel; die Andern aber verachtet er,
weil sie ihn mit ihren eigenwilligen Verrichtungen doch nur verhöhnen. Doch
lasst uns hören, was der Herr weiter zu jenem Volke sagte; denn es betrifft den
Zustand der Christenheit in unsern Tagen:
\textit{"`Wer einen Ochsen schlachtet, ist
eben, als der einen Mann erschlüge; wer ein Schaf opfert, ist als der einem
Hunde den Hals bräche; wer Speisopfer bringt, ist als der, welcher Saublut
opfert. Wer des Weihrauchs gedenkt, ist als der
dass Unrecht lobt. Solches erwählen sie in ihren (eigenen) Wegen, und ihre Seele
hat Gefallen an ihren Greueln."'}
\footnote{Jesaja 66,3}
\index{Bibelstellen:!Jesaja 66)}
Niemand sage: Wir bringen
auch dergleichen Opfer nicht; denn davon ist die Rede nicht. Gott zürnte nicht
über die Opfer, sondern über die Opfernden. Jene wurden nach einer gesetzlichen
Vorschrift dargebracht, die Gott selbst angeordnet hatte; da aber Diejenigen,
welche sie darbrachten, sich nicht in der dazu erforderlichen Geistesstimmung
und rechten Gemüthsverfassung befanden, so erklärte Gott, und zwar auf eine sehr
nachdrückliche Weise, seinen Abscheu dagegen; und an einem andern Orte verbietet
er sogar, durch denselben Propheten, alle fernere Opfer solcher Art.
\textit{\textit{"`Bringet nicht mehr Speisopfer so vergeblich,"'} sagt Gott.
\textbf{"`Das Rauchwerk ist mir ein Greuel; der Neumonden und Sabbathe, da ihr
zusammenkommt, und Mühe und Angst habt, mag ich nicht. Und wenn ihr gleich eure
Hände ausbreitet, verberge ich doch meine Augen vor euch; und ob ihr gleich viel
betet, höre ich euch doch nicht."'} \index{Gebete!nicht erhörte}
\footnote{Jesaja 1,13-15}
\index{Bibelstellen:!Jesaja 1)}
Welch eine
furchtbare Verwerfung ihres Gottesdienstes! Und warum verwarf ihn der Herr? --
Weil ihre Herzen verunreinigt waren; weil sie den Herrn nicht von ganzem Herzen
liebten, sondern sein Gesetz übertraten, sich gegen seinen Geist empörten, und
nicht taten, was in seinen Augen recht war.}Dieses gehet sehr klar aus der
Veränderung und Besserung hervor, die er in den folgenden Worten verlangt:
\textit{"`Waschet, reiniget euch, tut euer Wesen vor meinen Augen hinweg. Lasset
ab vom
Bösen, lernet Gutes tun. Trachtet nach Recht, helfet dem Unterdrückten,
schaffet den Waisen Recht und helfet (führet) der Wittwen Sache."'}
\index{Gerrechtigkeit}
\footnote{Jesaja 1,16+17}
\index{Bibelstellen:!Jesaja 1)}
Unter diesen Bedingungen, die unerlässlich sind, ladet er nun Alle ein,
zu ihm zu kommen, wenn er ferner sagt:
\textit{"`Wenn denn auch eure Sünden blutroth
sind, sollen sie doch schneeweiss werden; und wären sie wie Rosinsarbe, so
sollen sie doch wie (weisse) Wolle werden."'}
\footnote{Jesaja 1,18.}

\medskip

Ebenso wahr ist jene merkwürdige Stelle, wo der Psalmist sagt:
\textit{"`Kommet her,
höret zu Alle, die ihr Gott fürchtet; ich will erzählen, was er an meiner Seele
gethan hat. Zu ihm rief ich mit meinem Munde, und pries ihn mit meiner Zunge.
Wenn ich etwas Unrechts vorhätte in meinem Herzen, so würde der Herr mich nicht
hören. Nun aber erhöret mich Gott und merket auf mein Flehen. Gelobet sei Gott,
der mein Gebet nicht verwirft, noch seine Güte von mir wendet."'}
\footnote{Psalm 66,16-20}
\index{Bibelstellen:!Psalm 66)}

\section{10. Abschnitt} \label{kap6_ab10}
% olaf
\index{Gebote!für Gottesdienst}Es könnten noch mehrere dergleichen Stellen
angeführt werden, welche das
Missfallen Gottes sogar an den von ihm selbst vorgeschriebenen Formen der
Gottesverehrung an den Tag legen, wenn nämlich diese nicht
unter dem Einflusse seines Geistes und ohne die nöthige Vorbereitung des
menschlichen Herzens vollzogen wird, die nur durch ihn bewirkt und verliehen
werden kann. Die Notwendigkeit einer solchen innern Bearbeitung unserer Herzen,
und das Mittel, sie zu erlangen, empfiehlt
uns aber unter allen andern Verfassern der heiligen Schrift keiner öfterer und
eindrücklicher durch sein eigenes Beispiel, als der
Psalmist\index{Personen:!Psalmist}, welcher, indem
er sich seiner grossen Fehltritte und der Veranlassungen zu denselben, wie auch
des Mittels, durch welches er Gott wohlgefällig ward, und Kraft und Trost von
ihm erhielt, fortwährend erinnert, sich öfters selbst auffordert und ermahnt,
auf Gott zu harten.
\textit{"`Leite mich, Herr, in deiner Wahrheit,"'} \index{Stilles!waren auf
Gott} sagt er,
\textit{"`und lehre mich; denn du bist der Gott, der wir hilft. Auf dich harre
ich
täglich."'}
\footnote{Psalm 25,5}
\index{Bibelstellen:!Psalm 25)}
Seine Seele war auf Gott gerichtet, von dem er
Errettung und Befreiung von den Schlingen und Vorführungen der Welt erwartete.
Dieses setzt eine innere Gemüthsübung, eine geistliche Aufmerksamkeit voraus,
die sich nicht mit der Beobachtung äusserer Formen beschäftigte, sondern nach
der innern göttlichen Hilfe aussah.

\medskip

David \index{Personen:!David} musste in der That auch grosse Aufmunterung zu
diesem innern Harren
auf den göttlichen Einfluss finden; da die Güte Gottes ihn dazu einlud und ihn
darin stärkte. Denn er sagt selbst:
\textit{"`Ich harrete (geduldig) auf den Herrn, und
er neigete sich zu mir und hörte mein Schreien. Er zog mich aus dem Schlamme und
stellte meine Füsse aus einen Felsen."'}
\footnote{Psalm 40,1+2}
\index{Bibelstellen:!Psalm 40)}
Der Herr
erschien David in seinem Innern, seine Seele zu trösten\index{Trost}, die auf
seine Hilfe
harrete, und sie von den Versuchungen und Bekümmernissen zu befreien, die sie
überwältigen wollten, und er gewährte ihm Sicherheit und Frieden. Darum sagt er,
\textit{"`der Herr habe seine Tritte gewiss gemacht;"'}
das heisst: er habe sein Herz in
dem Wege der Gerechtigkeit befestigt; denn zuvor befleckte ihn jeder Schritt den
er that; und er war kaum vermögend zu gehen, Ohne zu fallen; da er sich von
allen Seiten mit Versuchungen und Fallstricken umgeben sah. Aber er harrte
geduldig auf Gott, sein Herz zog sich, wachsam und aufmerksam auf das Gesetz des
Geistes Gottes, davon zurück, und da fühlte er, dass der Herrlich zu ihm
neigete. Das Geschrei seiner Seele\index{Seele!Schrei der}, das aus einem
wahren Gefühle seines Elendes
und des Bedürfnisses der göttlichen Hilfe hervorging, drang in den Himmel und
ward erhört. Da erfuhr David Rettung und Erlösung in der von Gott ersehenen,
nicht in der von ihm selbst gewünschten Zeit; und er empfing Kraft, alle seine
Kämpfe durchzugehen und alle seine Trübsale zu überstehen. Und dann, sagt er
uns,
\textit{"`habe der Herr ihm ein neues Lied in seinen Mund gelegt, um unsern Gott
zu
loben."'}
\footnote{Psalm 40,4}
\index{Bibelstellen:!Psalm 40)}
Dieses ward ihm also von Gott gegeben, und war
folglich nicht sein eigenes Werk.

\medskip

\index{Heimsuchung} \index{Offenbarung} Zu einer andern Zeit hören wir ihn
ausrufen:
\textit{"`Wie der Hirsch schreiet nach
frischem Wasser, so schreiet meine Seele, o Gott! zu dir. Meine Seele dürstet
nach Gott, nach dem lebendigen Gott. Wann werde ich dahin kommen, dass ich
Gottes Angesicht schaue."'}
\footnote{Psalm 43,1+2}
\index{Bibelstellen:!Psalm 43)}
Dieses übertrifft alle äussere
Formalität \index{Formalität} und ist etwas, das sich nicht wie eine
Aufgabe erlernen lässt. Wir
können aber daraus abnehmen, dass wahre Gottesverehrung ein inneres Wert ist;
dass die Seele in ihrer himmlischen Sehnsucht durch den himmlischen Geist
gerührt und erhoben werden muss; und dass die wahre Anbetung in der Gegenwart
Gottes verrichtet wird
\textit{"`Wann werde ich dahin kommen, dass. ich Gottes
Angesicht schaue?"'} oder \textit{"`vor Gottes Angesichte erscheine?"'} David
redet nicht
vom Erscheinen im Tempel \index{Tempel!äusserer} mit äussern Opfern, sondern
vom Erscheinen vor Gott, in
seiner heiligen Gegenwart, wo die Seelen der wahren Anbeter Gott schauen, wenn
sie vor ihm erscheinen; und dieses ist die grosse Sache, nach welcher sie
verlangen und dürsten, und worauf sie harren. Aber die meisten Bekenner des
Christenthums sind hierin so sehr von Davids Beispiele abgewichen, dass es ihnen
sonderbar vorkommt, wenn dieser fromme Mann uns versichert:
\textit{"`Wahrlich meine Seele harret auf Gott!"'}
und dass er es seiner eigenen Seele zur Pflicht macht,
dieses zu tun, wenn er sagt:
\textit{"`O meine Seele harre du nur auf Gott; denn er ist meine Hoffnung."'}
\footnote{Psalm 62,6}
\index{Bibelstellen:!Psalm 62)}
Als sagte er:.\textit{Kein Anderer kann mein Herz
zubereiten und meinen Bedürfnissen abhelfen} Daher erwarte ich Nichts von meinen
eigenen willkührlichen Religionsübungen\index{Religionsübungen!willkührliche},
oder von der äussern leiblichen
Verehrung, die ich Gott darbringen kann. Dieses Alles hat keinen Wert, und kann
weder mir selbst etwas nützen, noch dem Allmächtigen wohlgefallen. Aber ich
harre auf ihn, dass er mir Stärke und Kraft Verleihe, mich so vor ihm
darzustellen, als es ihm am angenehmsten sein wird; denn wenn er sich selbst ein
Opfer \index{Opfer!dabieten} zubereitet, so wird es ihm gewiss auch angenehm
sein. Daher erwähnt er in
zwei Versen dreimal des Harrenes;
\textit{"`Ich harre des Herrn; meine Seele harret. --
Meine Seele wartet (oder harret) auf den Herrn, von einer Morgenwache bis zur
andern."'}
\footnote{Psalm 130,5+6}
\index{Bibelstellen:!Psalm 130)}
(Nach der englischen Uebersetzung;)\textit{"`mehr als
Diejenigen, die auf den Morgen warten."'} Ja, mit so genauer Aufmerksamkeit und
so unermüdet harrte er, dass er an einem Orte sagt:
"`Das Sehen\footnote{\texttt{'Gesicht' ersetzt duch 'Sehen'}} vergehet mir,
indem ich so lange auf meinen Gott harre."'
\footnote{Psalm 69,4}
\index{Bibelstellen:!Psalm 69)}
Er begnügt sich
nicht mit dem Hersagen einer gewissen Anzahl Gebete\index{Gebete!förmliche},
oder mit der Verrichtung
der verordneten gottesdienstlichen Handlungen, oder mit einer beschränkten
Wiederhohlung derselben; nein! er lässt nicht nach, bis er den Herrn findet,
nämlich seine tröstliche Gegenwart, die seine Seele mit Liebe und Frieden
erfüllte.

\medskip

Dieses war nun aber nicht allein der Gebrauch Davids, als eines mit einem
ausserordentlichen Einflusse des göttlichen Geistes begnadigten Mannes; denn er
redet davon, als von der Art der Gottesverehrung, die zu seiner Zeit unter dem
wahren Volke Gottes\index{Personen:!wahres Volke Gottes}, dem geistlichen, mit der
Beschneidung des Herzens \index{Beschneidung!des Herzens} bekannten
Israels \index{Orte:!Israel} , üblich war; \textit{"`Siehe!"'} sagt er,
\textit{"`wie die Augen der Knechte auf die
Hände ihrer Herren, und die Augen der Magd auf die Hände ihrer Herrin
\footnote{\texttt{'Frauen' ersetzt duch 'Herrin'}} sehen,
also sehen unsere Augen auf den Herrn unsern Gott, bis er uns gnädig
ist."'} \index{Gott!Gnade Gottes}\index{Gnade}
\footnote{Psalm 123,2}
\index{Bibelstellen:!Psalm 123)}
Und an einem andern Orte:
"`Unsere Seele harret auf den Herrn; er ist unsere Hilfe und Schild. Ich will
auf deinen Namen harren,
denn deine Heiligen haben Freude daran."'
\footnote{Psalm 33,20)  Psalm 52,11}
\index{Bibelstellen:!Psalm 33)}
\index{Bibelstellen:!Psalm 52)}
Auf Gott zu harten, war also schon in jenen Tagen unter den wirklich gottseligen
Menschen gebräuchlich, und als das wahre Mittel bekannt, wodurch sie zum Genusse
der Gegenwart Gottes \index{Gott!Gegenwart} gelangten, und Fähigkeit
erhielten, ihn auf die ihm
wohlgefälligste Art zu verehren und anzubeten. Daher fand sich auch David'
sowohl aus eigener Erfahrung \index{Erfahrung!eigene}der Vortheile, die ihm
dieses Harren gewährte, als
auch wegen des nützlichen Gebrauchs, den die Heiligen zu seiner Zeit davon
machten, bewogen, es ebenfalls Andern zu empfehlen.
\textit{"`Harre auf den Herrn!"'} \index{Harre-auf-den-Herrn!} sagt
er
\textit{"`Sei getrost und unverzagt, und harre auf den Herrn."'}
\footnote{Psalm 27,14
\begin{description}
 \item[Einheitsübersetzung:] \textit{"`Hoffe auf den Herrn und sei stark! Hab
festen Mut und hoffe auf den Herrn!"'}
 \item[Luther rev.1984:] \textit{"`Harre des HERRN! Sei getrost und unverzagt
und harre des HERRN!"'}
 \item[Elberfelder:] \textit{"`Harre auf den HERRN! Sei mutig, und dein Herz sei
stark, und harre auf den HERRN!"'}
 \item[Schlachter:] \textit{"`Harre auf den Herrn! Sei stark, und dein Herz
fasse Mut, und harre auf den Herrn!"'}
 \item[Hoffnung für Alle:] \textit{"`Vertraue auf den Herrn! Sei stark und
mutig, vertraue auf den Herrn!"'}
 \item[King James]
\textit{ "`Wait on the LORD: be of good courage, and he shall strengthen thine
heart: wait, I say, on the LORD."'}
 \item[New International:] \textit{ "`Wait for the Lord. Be strong and don't
lose hope. Wait for the Lord."'}
 \end{description}
}
\index{Bibelstellen:!Psalm 27)}
Harre nur im Glauben und in Geduld, so wird er gewiss zu deiner Errettung
erscheinen. Ferner:
\textit{"`Vertraue auf den Herrn, und harre auf ihn in Geduld."'} Das
heisst: Wirf dich ganz auf ihn! Gieb dich zufrieden, und harre in allen deinen
Bedrängnissen auf seine Hilfe. Du kannst dir nicht vorstellen, wie nahe er mit
seiner Hilfe Denen ist, die wahrhaft auf ihn harten. O! versuche es nur und habe
Glauben! Noch sagt er uns:
\textit{"`Harre auf den Herrn und halte seinen Weg."'}
\footnote{Psalm 37,34}
\index{Bibelstellen:!Psalm 37)}\index{Gott!Gottes Wege}
Hier sehen wir die Ursache, warum so Wenige wahre
Fortschritte machen. Sie bleiben nicht in seinem Wege, und können daher nie
recht auf ihn harren. Indessen hatte David doch gute Gründe für Das, was er
sagte; da er so grossen Trost und so viele Vortheile in dem gesegneten Wege des
Herrn erfahren hatte.

\section{11. Abschnitt} \label{kap6_ab11}

Der Prophet Jesaias sagt uns, dass die Gläubigen im Volke, obgleich die
Züchtigungen des Herrn wegen ihrer Übertretungen schwer auf ihnen lagen,
dennoch
\textit{"`in dem Wege seiner Gerichte,"'} unter Gefühlen seiner Bestrafungen und
seines Missfallens, \textit{"`aus ihn geharret haben,"'} und dass \textit{"`das
Verlangen ihrer Seelen,"'} -- welches die Hauptsache dabei ist, --
\textit{"`auf seinen Namen und Gedächtniss gerichtet gewesen sei."'}
\footnote{Jesaja 26,8}
\index{Bibelstellen:!Jesaja 26)}\index{Gott!Bestrafung durch}
Sie waren es gern
zufrieden, Verweise und Züchtigungen von ihm zu erhalten, weil sie gesündigt
hatten, und es war ihr sehnliches Verlangen, dass er sich ihnen aus diese Weise
zu erkennen geben möchte.\index{Gott!Offenbarung} Aber erschien er ihnen denn
nicht auch endlich in
seiner Barmherzigkeit?  -- Allerdings erkannten sie ihn auch darin, so dass sie
sagen konnten,
\textit{"`Sehet! das ist unser Gott, auf den wir harren. Er wird uns
helfen.! das ist der Herr, auf den wir harten, damit wir in seinem Heile und
freuen und fröhlich sind."'}
\footnote{Jesaja 25,9}
\index{Bibelstellen:!Jesaja 25)}
Erfahrungen, welche die arme,
sinnliche Welt nicht kennt! \index{Welt!sinliche} O seliger Genuss! o köstliche
Vertrauen! Dieses war
ein Harren im Glauben, welches obsiegte. Alle Gottesverehrung aber, die nicht im
Glauben verrichtet wird, ist nicht nur fruchtlos
\index{Gott!fruchtlose verehrung} für die, welche sie vollziehen,
sondern auch missfällig in den Augen Gottes. Und dieser Glaube, der die
Eigenschaft hat, dass er das Herz, der wahren Gläubigen reinigt und ihnen den
Sieg über die Welt\index{Welt!Sieg über die} giebt, ist eine Gabe Gottes. Doch
der Prophet sagt noch
ferner:
\textit{"`Wohl Allen, die auf den Herrn harren!"'} und warum?\\textit{"`denn
die, weiche
auf den Herrn harten, bekommen nene Kraft, dass sie nicht matt noch müde
werden."'}
footnote{Jesaja 30,18 Jesaja 40,31}
\index{Bibelstellen:!Jesaja 30)}
\index{Bibelstellen:!Jesaja 40)}
Gewiss eine grosse Aufmunterung, auf Gott
zu harren! aber es geht noch Weiter, wenn er sagt,
\textit{"`das von der Welt her
nicht gehöret noch mit Ohren vernommen sei, und kein Auge, ohne Gott, es gesehen
habe, was denen geschiehet, die auf ihn harren."'}
\footnote{Jesaja 64,4}
\index{Bibelstellen:!Jesaja 64)}
Sehet, das ist das innere Leben und dies Freude der Gerechten, der wahren
Gottersverehrer \index{Gott!wahren verehrer};
deren Geister vor der Erscheinung des Geistes Gottes in ihren Herzens sich
beugen; die Alles verlassen und überwinden \footnote{\texttt{'verleugnen' sersetzt
durch 'überwinden'}}, wogegen derselbe zeuget, und Alles
ergreifen und einnehmen, wozu er sie leitet.

\medskip

Aber auch in Jeremia’s \index{Personen:!Jeremia} Zeiten harrten die wahren
Anbeter auf Gott. (Denn er
versichert uns,
\textit{ "`der Herr sei freundlich gegen den, der auf ihn harret, und
gegen die Seele, die nach ihm fraget."'}
\footnote{Klagelieder 3,25}
\index{Bibelstellen:!Klagelieder 3)}
So ermahnte
gleichfalls der Prophet Hofea \index{Personen:!Hofea} die Kirche zu seiner Zeit,
sich zu Gott zu wenden
und auf ihn zu harren:
\textit{"`So bekehre dich nun zu deinem Gott!"'} sagt er,
\textit{"`halte
Barmherzigkeit und Recht, und hoffe, (nach dem Englischen: harre oder warte)
stets auf deinen Gott."'}
\footnote{Hosea 12,7}
\index{Bibelstellen:!Hosea 12)}
Auch Micha \index{Personen:!Micha} war sehr eifrig in
dieser guten Übung, indem er mit Entschlossenheit sagt:
\textit{"`Ich will auf den
Herrn schauen, und den Gott meines Heitz erwarten. Mein Gott wird mich
hören."'}
\footnote{Micha 7,7}
\index{Bibelstellen:!Micha 7)}
So redeten die Kinder!des Geistes \index{Personen:!Kinder des Geistes}, die ein
dürstendes Verlangen nach einem innern Gefühle von Gott hatten. Die Gottlosen
\index{Personen:!Gottlosen, die}
können so nicht sprechen; und auch Die nicht, welche ohne Vorbereitung des
Herzens beten, die durch stillen Harren erlangt wird. Es ward den Kindern
Israels \index{Personen:!Kindern
Israels} in der Wüste als Ursache ihres Ungehorsams und ihrer Undankbarkeit
gegen Gott zum Vorwurfe gemacht, dass sie nicht auf seinen Rath harreten. Und
wir können versichert sein, dass es auch unsere Pflicht ist, deren Vollziehung
von uns erwartet wird; da Gott sie beim Zephanja \index{Personen:!Zephanja}
\index{Personen:!Zefanja} fordert:
\textit{"`Darum, spricht
der Herr, müsset ihr wieder auf mich harren, bis ich mich aufmache zu seiner
Zeit tc."'}
\footnote{Zefanja 3,8}.
\index{Bibelstellen:!Zefanja 3)}
\textbf{O! möchten doch Alle, die den Namen
Gottes
bekennen, auf diese Weise harren, und es nicht wagen, sich zu seiner Verehrung
zu erheben, ohne feinen göttlichen Einfluss in ihren Herzen zu fühlen!} Dann.
würden sie die Anregungen das Erheben seines Geistes zu ihrer Hilfe, Zubereitung
und Heiligung in ihrem Innern erfahren.

\medskip
%olaf
Christus gebot seinen Jüngern ausdrücklich, nicht von Jerusalem
\index{Orte:!Jerusalem} zu weichen,
sondern auf die Verheissung des Vaters, auf die Taufe des Geistes \index{Taufe!des Geistes} zu
warten,
\footnote{Apostelgeschichte 1,4}
\index{Bibelstellen:!Apostelgeschichte 1)}
wodurch sie zur Verkündigung \index{Missionierung} seines herrlichen
Evangeliums zubereitet und fähig gemacht werden sollten. Da erfolgte freilich
eine ausserordentliche Ausgiessung \index{Pfingstwunder} des heiligen Geistes zu
einem
ausserordentlichen Werkes allein der hohe Grad einer Sache verändert doch ihre
Natur und Eigenschaft nicht; und wenn ein so langes Harren oder Warten und eine
solche Vorbereitung des Herzens durch den Einfluss des göttlichen Geistes
erforderlich war, die Jünger zum Predigen an die Menschen fähig zu machen, so
muss wenigstens ein gewisser Grad solcher Vorbereitung nothwendig sein, um uns
die Fähigkeit zu verschaffen, mit Gott zu reden.

\section{12. Abschnitt} \label{kap6_ab12}

Ich will diese wichtige Lehre der heiligen Schrift, vom stillen Harren auf Gott,
mit jener Stelle im Johannes beschliessen, wo von dem Teiche Bethesda
\index{Orte:!Bethesda, Teiche}
die Rede ist.
\textit{"`Es war nämlich zu Jerusalem, bei dem Schafhause, ein Teich, der
auf Ebräisch 'Bethesda' hiess, und fünf Hallen hatte. In diesen lagen viele
Kranke, Blinde, Lahme und Dürre, welche auf die Bewegung des Wassers warteten.
Denn es fuhr zu seiner Zeit ein Engel in den Teich herab, und bewegte das
Wasser. Wer nun, nachdem das Wasser bewegt war, zuerst hinein stieg, der ward
gesund; mit was für einer Krankheit er auch behaftet war."'}
\footnote{Johannes 5,2-4}
\index{Bibelstellen:!Johannes 5)}
Dieses giebt uns nun eine genaue bildliche Darstellung von dem ganzen Sinne
dessen, was bisher über den Gegenstand des Harrens gesagt worden ist.
\textbf{Denn sowie
es damals ein äusseres gesetzliches Jerusalem \index{Orte:!Jerusalem!gesetzliches}
gab, so giebt es jetzt ein
evangelisches \index{Orte:!Jerusalem!evangelisches} , geistliches Jerusalem
\index{Orte:!Jerusalem!geistliches}, die Kirche Gottes \index{Kirche!Kirche Gottes}\index{Gott!Kirche Gottes}, die aus
den Gläubigen \index{Kirche!Gemeinschaft der Gläubigen} \index{Gemeinschaft!der Gläubigen}
bestehet.} Der Teich in jenem alten Jerusalem \index{Orte:!Jerusalem!altes}
stellte gewissermassen die Quelle
vor, die jetzt im neuen Jerusalem \index{Orte:!Jerusalem!neues} eröffnet ist.
Jener Teich war für Solche, die
an körperlichen Krankheiten litten; diese Quelle ist für Alle, die an ihren
Seelen krank sind. Damals kam ein Engel, der das Wasser bewegte, um es
heilwirkend zu machen; jetzt erscheint der mächtige Engel \index{Personen:!Engel} der
Gegenwart Gottes \index{Gott!Gegenwart},
der diese geistliche Quelle \index{geistliche Quelle} mit heilbringendem Erfolge
segnet. Diejenigen,
welche damals nicht auf die Ankunft des Engels warteten, und die Zeit seiner
Bewegung des Wassers nicht in Acht nahmen, sondern früher hineinstiegen, hatten
keinen Nutzen davon; so können auch jetzt Alle, welche die Bewegung des Engels
Gottes in ihren Gemüthern nicht abwarten, sondern mit ihren selbstgeformten
Andachtsübungen in ihrer selbstbestimmten Zeit vor Gott erscheinen, sicher
daraus rechnen, dass sie in ihren Erwartungen eines gesegneten Erfolges sich
getäuscht finden werden. So wie daher damals Diejenigen, welche. geheilet zu
werden begehrten, mit aller Geduld und Aufmerksamkeit aus die Bewegung des
Engels warteten, eben so tun dieses die wahren Anbeter Gottes auch jetzt; denn
sie fühlen das Bedürfnis; und seufzen nach dem Genusse seiner Gegenwart, die
ihre Seelen belebt, wie die Sonne die Pflanzen auf dem Felde. Diese haben aus
öftern Versuchen das Nutzlose ihrer eigenen Wirksamkeit eingesehen, und sind nun
zu dem wahren \textit{Sabbathe} \index{Sabbathe} gelangt. Diese dürfen es nicht
mehr wagen, ihre
eigenen Einfälle hervorzubringen, oder ungeheligtes Gebet zu opfern, noch
weniger aber leiblichen Gottesdienst zu verrichten, wobei die Seele wirklich
unempfindlich oder vom Herrn nicht vorbereitet ist. \index{Andacht!innerehaltung,
}Diese müssen immer in dem
Lichte Jesu \index{Lichte!Jesu} aus die nothwendige Vorbereitung ihrer Herzen
warten, indem sie
eingekehrt und sich lösen \footnote{\texttt{'abgeschieden' ersetzt duch 'sich
lösen'}} von allen Gedanken, weiche die geringste Ablenkung
\footnote{\texttt{'Zerstreuung' ersetzt duch 'Ablenkung'}}
oder Unruhe in ihren Gemüthern erzeugen könnten, sich still verhalten, bis sie
die göttliche Bewegung des Engels gewahr werden, und es dem Geliebten ihrer
Seelen gefällt, zu erwachen, den sie vor seiner Zeit nicht wecken dürfen. Ja,
sie fürchten sich, in seiner Abwesenheit Andachtsübungen zu verrichten, von
denen sie wissen, dass sie ihnen nicht allein keinen Nutzen gewähren, sondern
sogar Tadel verdienen würden.
\textit{"`Wer fordert Solches von euren Händen?"'}
\footnote{Jesaja 1,12) Jesaja 28,16.}
\index{Bibelstellen:!Jesaja 1)}
\index{Bibelstellen:!Jesaja 28)}
sagt der Herr.
\textit{"`Wer glaubt, der fliehet nicht."'} (Nach dem Englischen: \textit{"`der
'eilet' nicht."'}) Diejenigen,
welche in der Abwesenheit ihres geistlichen Führers selbsterdachte
gottesdienstliche Handlungen verrichten, machen es nicht besser als jene
Israeliten \index{Personen:!Israeliten}, die in Moses \index{Personen:!Mose}
Abwesenheit ihre goldenen Ohrringe in ein
gegossenes Bild verwandelten \index{Goldenes Kalb}, aber auch mit ihren
Bemühungen den Fluch \index{Fluch} des Herrn sich zuzogen. Auch traf Jene vor
Zeiten kein besseres Loos,
\textit{"`die selbst ein Feuer anzündeten, und, mit Flammen gerüstet, in dem
Feuer wandelten, das sie
selbst angezündet hatten."'} Denn Gott sagte ihnen,
\textit{"`dass sie in Schmerzen liegen müssten."'}
\footnote{Jesaja 50,11}
\index{Bibelstellen:!Jesaja 50)}
Es sollte ihnen nicht allein keinen
Nutzen bringen, sondern sogar ein Gericht vom Herrn nach sich ziehen; Kummer und
Seelenangst sollte ihr Teil sein. Der sinnliche Mensch \index{Personen:!Menschen!sinnliche} möchte freilich auch
gern beten, wiewohl er nicht harren kann, und auch keine Geduld \index{Geduld}
hat, die
Vorbereitung seines Herzens abzuwarten; ja er möchte gern ein Heiliger
\index{Personen:!Heilige} sein,
obgleich es ihm unerträglich ist, den Willen Gottes zu tun \index{Gott!Willen Gottes tun} oder zu leiden \index{Leiden}. Mit
seiner Zunge lobt er Gott, und mit derselben Zunge fluchet er dem Menschen, der
nach dem Bilde Gottes gemacht ist.
\footnote{Jakobus 3,8-10}
\index{Bibelstellen:!Jakobus 3)}
\index{Lippenbekenntnis} Er nennt Jesum
seinen Herrn, aber nicht durch den heiligen Geist; er führet oft den Namen
Jesu im Munde, ja, er beugt seine Knie vor diesem Namen; lässt aber nicht ab
von der Ungerechtigkeit. \index{Ungerechtigkeit}
\footnote{1. Korinther 12,3) 2. Timotheus 2,19.}
\index{Bibelstellen:!1. Korinther 12)}
\index{Bibelstellen:!2. Timotheus 2)}
Dieses ist Gott ein \index{Gott!Gott ein Greuel}
Greuel.

\section{13. Abschnitt} \label{kap6_ab13}

\index{Anbetung!nötigen vier Dinge} \textbf{Es giebt vier Dinge
\footnote{'Stücke' ersetzt durch 'Dinge'}, die zur wahren Anbetung Gottes
unumgänglich
nothwendig
sind,} und durch welche die Vollziehung derselben so ganz dem eigenen Vermögen
des Menschen entzogen wird, dass es nur nöthig zu sein scheint, sie zu nennen,
um dieses einleuchtend zu machen. \textbf{Das erste ist}, die Heiligung, oder
der durch
die Kraft Gottes belebte, gereinigte und geheiligte Gemüthszustand des Anbeters.
\textbf{Das zweite} ist die Einweihung oder nöthige Zubereitung des Opfers,
wovon schon
oben ziemlich weitläuftig gehandelt worden ist. \textbf{Das dritte} ist die
Sache, um
welche man bittet, die Niemand kennet, der nicht durch die Hilfe oder unter dem
Einflusse des heiligen Geistes \index{Personen:!heiligen Geist} betet; weshalb denn auch
Niemand, ohne diesen
göttlichen Einfluss zu haben, recht beten kann. Dieses setzt der Apostel
Paulus ausser allen Zweifel, wenn er sagt:
\textit{"`Der Geist hilft unserer
Schwachheit auf. Denn wir wissen nicht, was wir bitten sollen, wie sichs
gebühret; sondern der Geist selbst vertritt uns  aufs Beste mit
unaussprechlichem Seufzen."'}
\footnote{Römer 8,26.}
\index{Bibelstellen:!Römer 8)}
\textbf{Menschen, die mit der
Kraft
und den Wirkungen des heiligen Geistes unbekannt sind, können den Willen und die
Absicht Gottes; nicht erkennen, und ihm daher mit ihren Gebeten auch gewiss
nicht gefallen.} Es ist nicht genug, zu wissen, dass wir Bedürfnisse und
Schickungen haben: \index{Gott!Gottesstrafe} \textbf{wir sollten auch einsehen
lernen, ob nicht oft dass,
was uns
widerfährt, einen Segen \index{Segen} für uns enthält. Gott lässt über die
Stolzen
Demüthigungen\index{Demüthigung}, über die Geizigen Verluste, über die
Ungehorsamen Züchtigungen
ergehen; diese entfernen zu wollen, hiesse die Beförderung des Verderbens, nicht
des Heils der Seele suchen.}

\medskip

Die sinnliche Welt betrachtet und beurtheilt Alles nur nach sinnlichen
Begriffen; und nur zu Viele, die für erleuchtet \index{Personen:!Erleuchtete Menschen}
gehalten sein wollen, sind
geneigt, den Fügungen der Vorsehung falsche Namen zu geben. \index{Schicksal!Grund für}\textbf{So, pflegen
sie, zum
Beispiele, Trübsale Gerichte, und Prüfungen \index{Prüfungen}, die köstlicher
als das so sehr
geliebte Gold sind, Unglück \index{Unglück} zu nennen,} indem sie, im
Gegentheile, den
Beförderungen der Welt den Namen: Ehre, und dem Besitze ihrer Güter, den der
Glückseligkeit \index{Glückseligkeit} beilegen; da diese doch, wenn sie es auch
in einen Falle sein
mögen, in hundert andern, -- wie sehr zu befürchten ist, -- von Gott als
Gerichte, wenigstens als Prüfungen\index{Prüfung}, für ihre Besitzer, zugelassen
werden. Es ist
daher schwer zu entscheiden, was der Mensch behalten, verwerfen oder begehren
soll; eine Aufgabe, die nur Gott seiner Seele auflösen
kann\index{Gott!Gottesführung}, Denn da Gott unsere
Bedürfnisse \index{Bedürfnisse, gerechtfertigte} besser kennt, als wir sie
selbst einsehen, so kann er auch besser
uns, als wir ihm, sagen, was wir nöthig haben. Deswegen ermahnte Christus seine
Jünger, dass sie lange Gebete und Wiederhohlungen vermeiden sollten, indem er
ihnen sagte:
\textit{"`Euer himmlischer Vater weiss, was ihr bedürfet, ehe ihr ihn
bittet."'}
\footnote{Matthäus 6,7+8}
\index{Bibelstellen:!Matthäus 6)}
\index{Gebet!Das Vaterunser-}
Und darum gab er ihnen auch ein Muster des
Gebets; aber nicht, wie Einige der Meinung sind, dass er jenen menschlichen
Liturgien zum Texte dienen sollte; die unter allen religiösen Gebräuchen am
meisten durch Lange und Wiederhohlung sich auszeichnen; sondern ausdrücklich,
um diese zu tadeln und sie uns vermeiden zu lehren. Gesetzt aber auch, man wäre
über jene Bedürfnisse, welche den Gegenstand des Gebets ausmachen sollten,
einig; -- wiewohl dieses immer ein schwieriger Punkt sehn würde; -- so ist es
doch von noch grösserer Wichtigkeit, zu wissen, wie, als was man bitten soll; da
hierbei nicht sowohl der Gegenstand des Gebets, als die Gemüthsverfassung des
Betenden in Betrachtung kommt. Die Sache, um welche man bittet, kann recht und
gut, aber der Zustand, in dem man bittet, mangelhaft sein. Obgleich also, -- wie
schon erwähnt ist, -- Gott nicht bedarf, dass wir ihm unsere Bedürfnisse
anzeigen, da Er sie uns zu erkennen geben muss; so will er dennoch, dass wir
unser Anliegen vor ihm sollen kund werden lassen, damit wir ihn suchen mögen,
und er sich zu uns herablassen könne. Und wenn dieses nun geschiehet,
\textit{"`so will der Herr den Elenden ansehen, der zerbrochenen Geistes ist,
und der sich fürchtet der seinem Worte."'}
\footnote{Jesaja 66,2}
\index{Bibelstellen:!Jesaja 66)}
Das sind die kranken Herzen, die
verwundeten Seelen, die Hungrigen und Durstigen, die Müden und schwer Beladenen,
die sich aufrichtig nach einem Helfer sehnen.

\section{14. Abschnitt} \label{kap6_ab14}

Doch ist dieses Alles zu einer vollkommnen evangelischen Anbetung Gottes noch
nicht hinreichend, wenn nicht auch \textbf{das vierte Erforderniss} dabei ist,
nämlich:
der Glaube; der wahre köstliche Glaube der Auserwählten Gottes, der das Herz
reinigt, die Welt überwindet und der Sieg der Gläubigen ist.
\footnote{1. Timotheus 1,5) Apostelgeschichte 15,9) Titus 1,1) 2 Petrus 1,1) 1.
Johannes 5,4}
\index{Bibelstellen:!1. Timotheus 1)}
\index{Bibelstellen:!Apostelgeschichte 15)}
\index{Bibelstellen:!Titus 1)}
\index{Bibelstellen:!2 Petrus 1)}
\index{Bibelstellen:!1. Johannes 5)}
Dieser muss das
Gebet beleben, und es durchdringend machen, wie bei dem anhaltenden cananaischen
Frau \footnote{\texttt{'Weibe' ersetzt durch 'Frau'}} \index{Personen:!Frau,
Kanaanitische} \index{Personen:!Nichtjüdisch} \index{Personen:!Heiden}, das sich
nicht wollte abweisen lassen, und zu welcher Christus, indem er
sie zu bewundern schien, sagte:
\textit{"`O Weib! dein Glaube ist gross!"'}
\footnote{Matthäus 15,28}
\index{Bibelstellen:!Matthäus 15)}
\textbf{Dieser Glaube ist von der grössten
Wichtigkeit
und Notwendigkeit für uns, wenn wir wünschen, dass unser Gebet Annahme bei Gott
finden möge;} wiewohl auch er nicht in unserer Macht stehel, sondern Gottes Gabe
ist, von dem wir ihn empfangen müssen. Ein Körnlein dieses Glaubens aber richtet
mehr aus, wirkt mehr Erlösung und Befreiung\index{Erlösung}, und verschafft uns
reichlichere
Gnade und Barmherzigkeit, als alles eigene Laufen, Wollen und Wirken, und alle
menschliche Erfindungen und leibliche Andachtsübungen nicht hervorbringen
können. Wenn wir dieses gehörig erwägen, so werden wir leicht die wahre Ursache
entdecken, warum die vielen gottesdienstlichen Verrichtungen und Übungen, die
wir in der Welt wahrnehmen, den Menschen so wenig Nutzen bringen; da diese
offenbar keine andere ist, als weil es ihnen am wahren Glauben mangelt. Sie
bitten, und erlangen nicht;
\footnote{Jakobus 4,3}
\index{Bibelstellen:!Jakobus 4)}
\index{Gebete!unerhörte}sie suchen, und finden nicht; sie
klopfen an, und es wird ihnen nicht aufgethan. Die Ursache davon liegt klar am
Tage: ihre Bitten sind nicht mit der reinigenden Kraft des Glaubens verbunden,
wie Jakobs \index{Personen:!Jakob} Bitten waren, als er mit Gott rang und
obsiegte. Und die Wahrheit
ist, leider! die, dass die Mehrsten noch in ihren Sünden leben und den Lüsten
ihrer Herzen folgen, den Ergözlichkeiten der Welt nachgehen und mit diesem
köstlichen Glauben ganz unbekannt sind, Diese Ursache des geringen Nutzens, den
das Predigen \index{Predigen} des Wortes Gottes bei Einigen in vorigen Zeiten
hatte, giebt der
tiefblickende Verfasser der Epistel an die Hebräer
deutlich an, wenn er sagt:
\textit{"`Das Wort der Predigt half Jenen nichts, weil
Diejenigen, die es hörten, nicht glaubten."'}
\footnote{Hebräer 4,2}
\index{Bibelstellen:!Hebräer 4)}
-- \textbf{Kann
ein
Prediger wohl ohne Glauben recht und nützlich predigen? Nein!} So kann auch noch
viel weniger Jemand ohne Glauben zu seinem wahren Nutzen beten. Denn wahre
Anbetung Gottes ist die erhabenste Handlung, der dass Gemüth des Menschen fähig
ist; und wenn bei den weniger erhabenen religiösen Verrichtungen der Glaube
nothwendig ist, so darf er gewiss bei dieser nicht fehlen.

\section{15. Abschnitt} \label{kap6_ab15}

Dieses kann bei Einigen ihre Bewunderung mässigen, warum Christus so oft seine
Jünger mit den Worten:
\textit{"`O ihr. Kleingläubigen"'} tadelte, und dennoch sagte,
dass ein so geringes Mass wahren lebendigen Glaubens, als einem Senfkörnchen,
einem der kleinsten Samenkörner, zu vergleichen ist, Berge versetzen könne. Als
wenn er gesagt hätte: es giebt keine Versuchung, die so mächtig wäre, dass der
wahre Glaube an ihn sie nicht überwinden könnte. \textbf{Die wahre Ursache also,
warum
Diejenigen, welche mit Versuchungen umgeben sind, in ihrer geistlichen Not
keine Hilfe erfahren, kann nur die sein, dass sie diesen kräftigen Glauben nicht
haben.}\index{Glaube!fehlender} Auch war derselbe vor Zeiten so nothwendig,
dass Christus an manchen
Orten nicht viele mächtige Werke verrichten konnte, weil die Leute daselbst
keinen Glauben hatten; und wenn er durch seine Kraft an andern Orten Wunder
\index{Wunder}
tat, so wirkte doch der Glaube dabei mit; so dass es schwer zu bestimmen ist,
ob seine Kraft durch den Glauben, oder der Glaube durch seine Kraft die Wunder
verrichtete. Erinnern wir uns, was für merkwürdige Dinge ein wenig Tonerde und
Speichel, eine blosse Berührung des Saums seines Gewandes, ein paar Worte aus
seinem Munde, durch die Macht des Glaubens bei den Kranken
verrichteten.
\footnote{Johannes 9,11) Lukas 8,47+48}
\index{Bibelstellen:!Johannes 9)}
\index{Bibelstellen:!Lukas 8)}
\textit{"`Glaubet ihr, dass ich euch die
Augen öffnen kenne?"'} sagte Christus zu den Blinden; -- \textit{"`ja Herr!"'}
antworteten
sie, und sahen.
\footnote{Matthäus 9,28-30}
\index{Bibelstellen:!Matthäus 9)}
\textit{"`Glaube nur!"'} sagte er zu dem
Obersten; er tat es, und seine Tochter erhielt das Leben wieder.
\footnote{Markus 5,36}
\index{Bibelstellen:!Markus 5)}
Bei einer andern Gelegenheit sagte er:
\textit{"`Wenn du glauben kannst?"'} --
\textit{"`Ich glaube,"'} schrie der bekümmerte Vater mit Thränen,
\textit{"`Herr! hilf meinem Unglauben!"'}
\footnote{Markus 9,13-24}
\index{Bibelstellen:!Markus 9)}
Da musste der böse Geist weichen, und der
Knabe war gesund. Zu Einem sagte er:
\textit{"`Gehe hin, dein Glaube hat dich gesund gemacht."'}
\footnote{Markus 10,52}
\index{Bibelstellen:!Markus 10)}
Zu einem Andern: \textit{"`dein Glaube hat dir
geholfen; deine Sünden sind dir vergeben."'}
\footnote{Lukas  7,48-50}
\index{Bibelstellen:!Lukas  7)}
Und als seine
Jünger sich wunderten, wie schnell sein Ausspruch über den unfruchtbaren
Feigenbaum in Erfüllung gegangen war, sagte er ihnen zu ihrer Aufmunterung im
Glauben:
\textit{"`Wahrlich ich sage euch, wenn ihr Glauben habt, und nicht zweifelt,
so werdet ihr nicht allein Solches mit dem Feigenbaume tun, sondern wenn ihr zu
diesem Berge sagen werdet: hebe dich aus, und wirf dich ins Meer, so wird es
geschehen; und Alles, was ihr bittet im Gebet, wenn ihr glaubet, so werdet ihr
es empfangen."'}
\footnote{Matthäus 21,21-22}
\index{Bibelstellen:!Matthäus 21)}
. Schon diese eine Stelle überführt die allgemeine Christenheit ihres grossen
Unglaubens\index{Unglauben}; indem sie bittet, aber nicht
empfängt.

\section{16. Abschnitt} \label{kap6_ab16}

\textbf{Einige werden vielleicht sagen, es sei unmöglich, dass ein Mensch Alles,
was er
bitten mag, erhalten könne.} Es ist aber nicht unmöglich, dass der Mensch, der
im
wahren Glauben an die Kraft Gottes und nach der Leitung seines Geistes betet,
alles Das, um welches er so bittet, empfange, und die Früchte des Glaubens sind
keinesweges unerreichbar für Diejenigen, welche wahrhaft an den Gott glauben,
der sie darreichen kann. Denn, als Jesus zu dem Obersten sagte:
\textit{"`Wenn du glauben kannst?"'} fügte er hinzu: \textit{"`Alle Dinge sind
möglich dem, der da glaubt."'}
\footnote{Matthäus 9,23}
\index{Bibelstellen:!Matthäus 9)}
Hierauf werden Andere erwiedern, es sei
unmöglich, einen solchen Glauben zu haben.; \textbf{denn die Ungläubigen möchten
gern
ihren Mangel an Glauben \index{Glaube!Mangel an} damit entschuldigen, dass
sie es für unmöglich halten,
ihn zu besitzen.} Allein Christus widerlegt diese Einwendung vollkommen in
seiner
Antwort, die er den Ungläubigen jenes Zeitalters in den Worten gab:
\textit{"`Was bei den Menschen unmöglich ist, das ist bei Gott möglich."'}
\footnote{Matthäus 19,26) Lukas 18,17}
\index{Bibelstellen:!Matthäus 19)}
\index{Bibelstellen:!Lukas 18)}
\textbf{Hirraus folgt nun, dass es eben so wenig bei Gott unmöglich
ist,
einen solchen Glauben zu geben, als es gewiss ist, dass man ohne denselben
\textit{"`Gott unmöglich gefallen könne;"'}} wie der Verfasser der Epistel an
die Hebräer
lehrer. Und wenn es also nicht möglich ist, dass Jemand ohne Glauben Gott
gefallen könne, so kann auch gewiss Niemand ohne diesen köstlichen Glauben
erhörlich zu Gott beten.

\section{17. Abschnitt} \label{kap6_ab17}

Vielleicht werden Einige auch fragen: Was ist denn \textbf{dieser Glaube, der
zur
Anbetung Gottes so nothwendig ist, und dem Menschen die grosse Wohlthat gewahrt,
dass sein Gebet Annahme bei Gott findet? Ich antworte: Dieser Glaube bestehet in
einer heiligen Ergebung in den Willen Gottes, und in einem festen Vertrauen aus
ihn, welches sich durch religiösen Gehorsam \index{Gehorsam!religiöser}gegen
seine göttlichen Forderungen
beweiset, und wodurch der Seele eine klare Ueberzeugung von Dem, was sie noch
nicht siehet, und ein gewisses Vorgefühl von dem Wesen der Dinge, aus welche sie
hoffet, nämlich, von der Herrlichkeit, die hernach geoffenbaret werden soll,
erlangt.} Da nun dieser Glaube Gottes Gabe ist, so reinigt er auch die Herzen
Derer, die ihn empfangen; und der Apostel Paulus bezeuget, dass er nur in einem
reinen Gewissen wohne. Darum verbindet er an einem Orte
\textit{"`ein reines Gewissen mit ungeheucheltem Glauben,"'} und an einem
andern: \textit{"`Glauben mit einem guten
Gewissen."'}
\footnote{1. Timotheus 1,5) 1. Timotheus 3,9}
\index{Bibelstellen:!1. Timotheus 1)}
\index{Bibelstellen:!1. Timotheus 3)}
Jakobus vereinigt
\textit{"`Glauben mit Gerechtigkeit,"'}
\footnote{Jakobus 2,14 bis Ende}
\index{Bibelstellen:!Jakobus 2)}
und Johannes mit dem Siege über
die Welt:\textit{ "`Unser Glaube,"'} sagt er, \textit{"`ist der Sieg, der die
Welt überwunden hat."'}\index{Sieg} \index{Welt!überwunden}
\footnote{1. Jakobus 5,4+5}
\index{Bibelstellen:!1. Jakobus 5)}

\section{18. Abschnitt} \label{kap6_ab18}

Die Besitzer dieses Glaubens sind, -- wie Paulus erklärt, -- auch ohne äussere
Beschneidung\index{Beschneidung}, die wahren Kinder
Abrahams\index{Personen:!Kinder!Abrahams}; indem sie, dem Gehorsame des Glaubens
gemäss, in seinen Fussstapfen wandeln,
\footnote{Römer 4,12}
\index{Bibelstellen:!Römer 4)}
wodurch allein die
Menschen ein Recht zu dieser Benennung erwerben können. Dieser Glaube erhebt
nicht nur über die Sünde\index{Sünde}, sondern auch über die
Gerechtigkeit\index{Gerechtigkeit} der Welt.
\footnote{Johannes 16,9+10.}
\index{Bibelstellen:!Johannes 16)}
Aber Niemand kann ihn anders erlangen, als wenn
er durch die Kraft des Kreuzes Christi den Tod seiner Selbstheit\index{Tod!seiner Selbstheit} erduldet, und
allein durch Christum sein ganzes Vertrauen auf Gott setzt.

\medskip

Die merkwürdigen Taten, die zu allen Zeiten durch diese göttliche Gabe des
Glaubens verrichtet wurden, sind gross und berühmt. Es würde mir aber an Zeit
fehlen, sie herzuzählen, da die ganze heilige Geschichte davon voll ist. Mag es
daher genügen, zu bemerken, dass durch den Glauben die heiligen Alten alle
Prüfungen erduldeten, alle Feinde überwunden, selbst Gott übermochten, seine
Wahrheit verbreiteten\index{Wahrheit!verbreiteten} und verherrlichten, ihre
Zeugnisse \index{Zeugnis}vollendeten und die
Belohnung \index{Belohnung}der Gläubigen, eine Krone der
Gerechtigkeit\index{Krone!der Gerechtigkeit}, empfingen, welche die
ewige Seligkeit\index{Seligkeit!ewige} der Gerechten ist.




