\section*{L}

\articlesize

\begin{description}

 \item[Lay, Benjamin] $\ast$26.1.1682 in Colchester, England, \dag3.2.1759 in Abington, Pennsylvania. Englischer Quaker, Philanthrop und Schriftsteller. Trat er als Abolitionist gegen die Sklaverei ein und war Bekennt für Provokative Inszenierungen und Auftritte.

 \item[Lay down]
    the action properly taken upon a committee, meeting or ministry that is no longer needed; "`to lay down"' a meeting is to disband it.

 \item[Leading]
    a course of action, belief or conviction that a Friend feels is divinely inspired.

 \item[Levellers]


 \item[Liberales Quäkertum]
% Ganz wie der Begriff \textit{Chrisozentrische Quäker} eher eine Wage Bezeichnung für eine Grundhaltung oder Tendenz. Die meisten Europäschen Jahresversammlungen könnte man dort einordnen. Aber mit Vorbehalt! wie wir bei den <i>independent Quaker</i> schon gesehen haben, gibt es auch "Ausreißer". Mit <i>Liberales Quäkertum</i> ist meist gemeint das man so ziemlich jede Vorstellung haben kann, die man will. Was in der Regel nicht unter Liberales Quäkertum verstanden wird, ist aggressives Missionieren, Gottesdienste mit Gesang, Liturgie und Prediger/Pastor.</li>

 \item[Lightfoot, Hannah]

 \item[Lilburne, John] $\ast$ca.~1614, \dag28.8.1657. Auch bekannt als \textit{"`Freeborn John"'}. Er war einer der bekanntesten Wortführer der radikaldemokratischen $\to$Levellers in England des siebzehnten Jahrhunderts. Er ist der Bruder von Robert Lilburne.

 \item[Luffe, John 'Love'] $\ast$? \dag~?. Quaker. Bekanntgeworden durch den Versuch 1658 Papst Alexander VII bekehren zu wollen. Er kam dabei in Rom ums leben.\endnote{Biographisch-Bibliographischen Kirchenlexikon, Band XX (2002)Spalten 1160-1167 Claus Bernet über Perrot, John}

 \end{description}
\normalsize
