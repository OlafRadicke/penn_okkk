\section*{L}

\articlesize

\begin{description}

 \item[Lay, Benjamin] $\ast$26.1.1682 in Colchester, England, \dag3.2.1759 in Abington, Pennsylvania. Englischer Quaker, Philanthrop und Schriftsteller. Trat er als Abolitionist gegen die Sklaverei ein und war Bekennt für Provokative Inszenierungen und Auftritte.

 \item[Levellers]

 \item[Lightfoot, Hannah]

 \item[Lilburne, John] $\ast$ca.~1614, \dag28.8.1657. Auch bekannt als \textit{"`Freeborn John"'}. Er war einer der bekanntesten Wortführer der radikaldemokratischen $\to$Levellers in England des siebzehnten Jahrhunderts. Er ist der Bruder von Robert Lilburne.

 \item[Luffe, John 'Love'] $\ast$? \dag~?. Quaker. Bekanntgeworden durch den Versuch 1658 Papst Alexander VII bekehren zu wollen. Er kam dabei in Rom ums leben.\endnote{Biographisch-Bibliographischen Kirchenlexikon, Band XX (2002)Spalten 1160-1167 Claus Bernet über Perrot, John}

 \end{description}
\normalsize
