\section*{G}

\articlesize

\begin{description}

\item[Gathered Meeting] Es gibt im Deutschen kein stehenden Begriff dafür.
Gemeint ist, das die Teilnemer einer Andacht das Gefühl hatten, das die
Versammlung ganz besonders von dem Geist Gottes erfüllt war. Vielleicht könnte
mann von einer "`gnadensreichen"', "`erfüllten"' oder "`intensiven Andacht"'
im Deutschen sprechen. \endnote{Artikel "`Religious Society of Friends"'.
Abgerufen 2010, January 12. In Wikipedia, Die freie Enzyklopädie.
Bearbeitungsstand: 16:41 Uhr 2010-01-13, \\ \texttt{http://en.wikipedia.org/w/index.php?title=\\
$\hookrightarrow$~Religious\_Society\_of\_Friends\&oldid=337403234}
\\ Hier der Abschnitt "`Quaker terminology"'.}

\item[Germantown] Wurde von 13 deutschen Familien aus dem Raum Krefeld gegründet.
Die meisten Falilien waren Mennoniten, die zum Quakertum konvertierten. Siehe
auch $\to$\textit{Abraham Isacks op den Graeff} und $\to$\textit{Franz Daniel
Pastorius}.

 \item[Geschäftsversammlung] Die Geschäftsversammlung ist eine Zusammenkunft
 der Mitglieder einer Versammlung, (as kann eine $\to$\textit{Monats-,
 Vierteljahres-} oder $\to$\textit{Jahresversammlung} sein. Oder auch einfach
 eine $\to$\textit{Andachsgruppe}) um organisatorisch-administrative Dinge
 zu besprechen und zu entscheiden.
 \medskip
 Die Geschäftsversammlung wird so verstanden - ganz ähnlich wie die Andacht -
 das auf den Wellen Gottes gewahrtet wird und versucht wird zu ergünden, was
 in Gottes Sinne ist. Entscheidungen werden nicht durch Wahlen oder Abstimmungen
 entschiedern. Sondern es wird versucht zu einer einheitliche Wahrnemung (des
 Willens Gottes) zu kommen. Im Englisch $\to$\textit{Sense of the Meeting}
 genannt. Im ideal Fall ist sich die Versammlung einig, wenn der Schreiber
 (engl. $\to$\textit{Clerk}) einem Beschluss vormuliert und des keine weiteren
 Einwürfe oder Wiedereden gibt. Unter Umständen, gibt es aber Bedenkenträger
 die ihrer Bedenken zu Protokoll geben, und dann erklären, das sie sich einen
 Beschluss nicht entgegenstellen wollen. Es ist aber auch denkbar, das eine
 Versammlung auch gegen den ausdrücklichen Protest oder Wiedrrede eines
 Mitglieds zu einem Beschluss kommt. So war es zumk Beispiel im 18. Jahrundert,
 als die ersten Jahresversammlungen gegen die Wiederstände einzelner beschlossen,
 Sklaven-Händler und -Besitzer aus ihren Reihen auszuschißen. \endnote{Sie
 hierzu: "`Die Aufzeichnung von John Woolmann"', 1964, bei Leonhard Friedrich,
 Seite 89 bis 93.}

 \item[Gesellschaft, Die] Kurzform von "`Die
 $\to$\textit{Religiöse Gesellschaft der Freunde}"'.

\item[Gewichtige Freunde] $\to$\textit{Weighty Friends}

\item[Glücksspiel] Ist unter Quakern verpöhnt und führte bei den
$\to$\textit{Frühe Freund} unweigerlich zum Ausschluß. Hintergrund ist der,
das Ziel des Glücksspiel Betrug oder immindestens Übervorteilung des Anderen
ist. Und das mit Glücksspiel weder eine sinnvolle Wertschöpfung verbunden ist,
noch eine sinnvolle Freizeitbeschäftigung darstellt. Es fördert weder das
kooperative Verhalten, noch verbessert es wünschenswerte Vertikeiten oder
Fähigkeiten.\endnote{Siehe zum Beispiel die Ausführungen von William Penn, in
diesem Buch auf Seite \pageref{ref:10_08_zeitvertreib},
\pageref{ref:14_01_wahre_nachfolger} und \pageref{kap17_ab3}.}

\item[Gottesdienst] $\to$\textit{Andacht}

 \item[Graeff, Abraham Isacks op den]  $\ast$1649 in Krefeld; \dag25. März 1731
 in Perkiomen (heute Montgomery). Quaker. Entstammte einer mennonitischen Familie.
Er gehörte der ersten geschlossenen Gruppe deutscher Auswanderer nach Amerika
an, der sogenannten "`Original 13"'. Er war einer der ersten Unterzeichner einer
Petition gegen die Sklaverei. Im Jahre 1692 wurde er zum Bürgermeister von
Germantown ernannt.\endnote{Seite "`Abraham Isacks op den Graeff"'. In:
Wikipedia, Die freie Enzyklopädie. Bearbeitungsstand: 31. Dezember 2009,
07:28 UTC. URL: \\ \texttt{http://de.wikipedia.org/w/index.php?title=\\
$\hookrightarrow$~Abraham\_Isacks\_op\_den\_Graeff\&oldid=68641530} }

 \item[great separation] Die erste grosse Spaltung im Quakertum. Sie begann im
 Jahre 1827/28 im $\to$\textit{Philadelphia Yearly Meeting}. Dem war ein
 längerer schwelender Streit vorrausgeggangen der sich an der Person
 $\to$\textit{Elias Hicks} polarisierte. Bei der Spaltung schlugen sich
 ungefähr $\nicefrac{2}{3}$ der Mitglieder auf die Seite dessen Flügel dann
 spähter $\to$\textit{"`Hicksite"'} genannt wurde. Der andere Teil der
 Versammlung bildete den Flügel der als $\to$\textit{"`Orthodox"'} bezeichnet
 wurde. \textit{Hicksites} betonten die Bedeutung des $\to$\textit{Inneren
 Lichtes}, der Glaubens- und Gewissensfreiheit. Während den Orthodoxen die
 Bedeutung der Bibel und der Buse betonten. In der folgezeit kam es zu
 ähnliche Spaltungen in New York, Baltimore und andern Versammlungen.
 \endnote{\texttt{http://www.quakerinfo.org/quakerism/brancheshistory.html}
 \\ "`A Brief History of the Branches of Friends"', Stand:~2010-01-09}

 \item[Greenpeace] Eine Organisation die sich dem Umweldschutz verschrieben hat.
 Gegründet am 14.10.1979. Die Organisation war aber nie eine reine
 Quäkerorganisation, auch wenn vieele Gründungsmitglieder und spähtere Aktivisten
 Quaker waren.\endnote{\texttt{http://green.wikia.com/wiki/Greenpeace}}

\item[Gruppe] $\to$\textit{Monatsversammlung}

\item[Gurneys] Einflussreiche alte Quakerfamilie. Bekannteste Mitglieder sind:
$\to$\textit{Joseph John Gurney}, $\to$\textit{Elizabeth (Gurney) Fry},
$\to$\textit{Samuel Gurney}


\item[Gurney, Joseph John] $\ast$2.8.1788 \dag4.1.1847; Quaker, englischer Banker und
und evangelikaler Prediger. Seine Ansichten und Predigttätikeiten fürten zu
dem zweiten großen Schisma im Nordamerikanischen Quakertum. Dem
$\to$\textit{Gurney-Wilbur-Schisma}, bei dem sich die Orthodox-Linie noch ein
mal auf spaltete.

\item[Gurneyite Quakers] Abspaltung die aus dem $\to$\textit{Gurney-Wilbur-Schisma}
hervor ging. Gurneyite Quakers gingen dazu über, bezahlte Pediger/Pastoren,
festgelegte Zeromomien und Kirchengesang ein zu führen. Und lehnten sich dabei
sehr stark an den Protestantismus an.

\item[Gurney-Wilbur-Schisma]
Eine aufspaltung der Orthodox-Linie die aus dem ersten Grossen Schisma in
Nordamerika hervor ging ($\to$\textit{"`Hicksites-Orthodox-Schisma"'}). Der
eine Flügel der Abspaltung wurde \textit{Gurneyite Quakers} genannt, nach
ihrem Wortführer $\to$\textit{Joseph John Gurney}. Der Ander Zweig wurde
\textit{Wilburite Quakers} genannt; nach deren Wortführer
$\to$\textit{John Wilbur}.\endnote{"`History of the Religious Society of Friends"'.
(2009, December 30). In Wikipedia,. Retrieved 13:53, February 8, 2010, von \\
\texttt{http://en.wikipedia.org/w/index.php?title=\\
$\hookrightarrow$~History\_of\_the\_Religious\_Society\_of\_Friends\&oldid=334827469}}


\item[Gurney, Samuel] $\ast$18.10.1786 \dag5.6.1856; Quaker, englischer Banker und
Philanthrop.

\item[Gurney's bank] 1770 Gegrundet von den Quaker Brüdern John and Henry Gurney und
ihren Söhnen John und Bartlett Gurney. 1896 fusioniert mit
$\to$\textit{Barclays Bank}.\endnote{"`Gurney's bank"'. (2009, December 4). In
Wikipedia. Retrieved 13:51, February 8, 2010, von \\
\texttt{http://en.wikipedia.org/w/index.php?title=\\
$\hookrightarrow$~Gurney\%27s\_bank\&oldid=329765837}}

\item[GYM] \textit{German Yearly Meeting}, \textit{Deutsche Jahresversammlung}. $\to$\textit{Jahresversammlung}
 \end{description}

\normalsize