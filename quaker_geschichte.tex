




\chapter{Geschichte des Quakertums}\label{ref:entwiklung_quakertum}
% Das Kapitel muss noch verschlagwortet werden


\begin{flushright}
\begin{footnotesize}
\textit{Von Olaf Radicke}
\end{footnotesize}
\end{flushright}
\smallskip 

Das Werk \textit{"`no cross no crown"'} -- zu Deutsch: \textit{Ohne Kreuz keine
Krone} --
steht am Anfang der Quaker-Geschichte \index{Quaker!Geschichte}. Quasi mitten im
\textit{Quaker-Urknall}
um die 1660er Jahre. Aus dem Werk spricht die ganze Radikalität und
Aufbruchstimmung der jungen Glaubensgemeinschaft. In folgenden kann nur ein
Überblick über die weitere Entwicklung bis Heute gegeben werden. Mir ist kein
Deutscher Buchtitel bekannt, den ich zur Vertiefung empfehlen kann. Im
Englischen wird man aber reichlich Material finden.

\medskip

Es gibt eine Reihe von Hypothesen wie oder warum das Quakertum
entstanden\index{Quakertum!Ursprung} ist.
So
wollten einige im Quakertum ein Zweig der Mystik \index{Mystik} sehen.
Nachweislich ist, das
sich Quaker mit der Mittelalter-Mystik beschäftigten \index{G. Fox und die
Londoner Jahresversammlung standen der Mystik aber skeptisch bis ablehend
gegenüber}, allerdings erst mit dem
Beginn der Kontakte zum Festland, durch ihrer Missioniereungsversuche
\index{Missioniereung} unter
anderem in Holland \index{Orte:!Holland} und Deutschland
\index{Orte:!Deutschland}. Das war zu einem Zeitpunkt wo die
theologischen Eckpfeiler für die Quaker schon ausformuliert waren. Also konnte
die Mystik nicht der Ursprung des Quakertums sein.

\medskip

Das Quakertum entstand aber nicht in einem luftleeren Raum und man kann Faktoren
die zur Entstehung beitrugen nennen. Das war zum einen die Übersetzung der Bibel
\index{Bibel!Übersetzung}
in englische Sprache und der englische Bürgerkrieg
\index{Bürgerkrieg!englischer}. Ersteres befähigte auch
einfache Leute -- wie den Quaker George Fox \index{Personen:!Fox, George} -- zum
lesen der Bibel und somit zum
errichten eines theologischen Gedankengebeude, und letzteres sorgte für die
Verunsicherung und zum Teil Entmachtung der etablierten Kirche
\index{Kirche!etabierte} und ihrer
Theologie.

\medskip

Viele Ideen, die wir im Quakertum finden, geisterten schon vorher durch die
Gesellschaft und Köpfen der zahllosen Splittergruppe, Sekten
\index{Personen:!Sekten}, Suchenden und
Dissidenten \index{Personen:!Dissidenten}, die in der Brürgerkrigszeit
\index{Brürgerkrig} entstanden waren. Und das Quakertum
währe höchst wahrscheinlich genauso von der Bildfläche wieder verschwunden, wie
alle anderen auch, wenn es nicht die begnadeten Organisationstalente unter den
Quaker gegeben hätte. Natürlich wird auch einfach eine ordentliche Portion Glück
dabei gewesen sein. Zu den -- heute noch allgemein bekannten --
Organisationstalente gehörte G. Fox, der die so genannten Monats- Vierteljahres-
und Jahres-Versammlungen \index{Jahresversammlungen} \index{Monatsversammlungen}
\index{Vierteljahresversammlungen} organisierte, die massgeblich zu
funktionierenden
Strukturen führte. Damit verbunden die etablierte "`Zucht des Zusammenlebens"'
\index{Zucht des Zusammenlebens}
--
Englisch: \textit{THE BOOK OF DISCIPLINE} \index{BOOK OF DISCIPLINE} . Also ein
\textit{Kirchenrecht} \index{Kirche!Kirchenrecht} die
Regelte, wie Dinge zu laufen haben. Eben so wichtig war seine spätere Frau
Margaret Fell \index{Personen:!Fell, Margaret} (auch: \textit{Mother of
Quakerism} \index{Personen:!Mother of Quakerism}). Während ihr späterer Mann
lange und oft im Gefängnis war, war sie unermüdlich damit beschäftigt, die
gegenseitige Hilfe unter den verfolgten \index{Verfolgten} Quakern zu
koordinieren. Ihr Anwesen
war in den Ersten Jahren das Zentrum der Bewegung. Von hier aus wurden die
Missionsreisen organisiert, Seelsorge \index{Seelsorge} geleistet,
Rechtsbeistand geleistet und
politische Einflussnahme \index{Politik!Einflussnahme} geübt (das was man heute
\textit{Lobbyarbeit} \index{Lobbyarbeit} nennen
würde).

\medskip

W. Penn ist für die Quaker in mehrfacher Weise von unschätzbaren wert gewesen.
Zum einen durch seine Kontakte in aller höchste Kreise durch seine Abstammung,
und
durch sein schriftstellerisches Wirken. Aber allen voran, wegen seinem so
genannten \textit{Heiligen Experiment}\index{Heilig!Heilige Experiment}. Der
Gründung des Quakerstaates \index{Orte:!Quakerstaat} \index{Orte:!Pennsylvania}
Pennsylvania. Mit einem Schlag wurde aus einer kleinen \textit{Sekte von
Schwärmern und Ketzern}, eine Staatstragene soziale Gruppe. So haben die
damaligen Quaker noch bis heute sichtbare Spuren in der heutigen USA
\index{Orte:!USA}
hinterlasse.
Der damalige Quakerstaat wurde, auf Grund seiner liberalen Verfassung
\index{Verfassung}, die
rettende neue Heimat, vieler wegen ihres Glaubens verfolgte Gläubigen. Das ist
auch der Grund, warum man bis Heute noch zu erst an die USA bei Quakern denkt.

\medskip

Erst erst \textit{Toleration Act} \index{Toleration Act} von 1689 beendete die
Verfolgung der Quaker im
Mutterland. Was ihrer Popularität und Verbreitung aber kaum ein dämmte.
\index{Quaker!verbreitung} 1660 gab
es in Britanien \index{Orte:!Britanien} und Irland \index{Orte:!Irland} zusammen
66.000 Quaker was etwa 0,76\% der
Gesamtbevölkerung war\footnote{P.Dandelion, "`Quakerism"', Seite 43, ISBN
0-521-60088-X}. Übertragen auf die heutige Befölkerungsgrösse währen das 492.684
Quaker Heute sind es aber tatsächlich nur noch ungefähr
17.000\footnote{P.Dandelion, "`Quakerism"', Seite 178, ISBN 0-521-60088-X}.

\medskip

W. Penn verbrachte nur wenige Monate seines Lebens in Nordamerika
\index{Orte:!Nordamerika}. Die beiden
Quaker-Gruppen entwickelten sich auf Grund der geographischen Entfernung der der
unterschiedlichen politischen und sozialen Rahmenbedingungen unterschiedlich. In
England integrierten sich die Quaker nach dem \textit{Toleration Act} zusehends
in die Gesellschaft. Sie wurden \index{Wirtschaftliche!Erfolge} wirtschaftlich
erfolgreich und betätigten sich
auf dem verschiedensten Gebieten. Wie dem Bankgewerbe \index{Bankgewerbe}, der
Lebensmittelindustrie \index{Lebensmittelindustrie}
(bzw. Manufaktur) und der Schifffahrt. Ihre aufrührerischen und provokanten
Verhaltensweisen legten sie dabei zum Teil ab. So war es bald unüblich die
Gottesdienste \index{Gottesdienst!stören} anderer Konfessionen
\index{Konfession} zu stören. Das abnehmen des Hutes und die
Verwendung von Ehrentitel wurde aber weiter abgelehnt.

\medskip

Im Achtzehnten Jahrhundert lebten die Quaker in sich zurückgezogen.
Universitäten \index{Universität} und der Akademische Betrieb waren verpönt.
\footnote{Theater} Theater, Tanz \index{Tanz} und Musik \index{Musik}
wurden weiter abgelehnt. Geheiratet \index{Heirat} wurde nur innerhalb der
Glaubensgemeinschaft. Die Kleidung \index{Kleidung} war schmucklos und in
gedeckten Farben
(zumeist Grau) gehalten. Die Missionstätigkeit wurde weitestgehend eingestellt.
Theologische Auseinandersetzungen fanden zu meist nur innerhalb der Gemeinschaft
statt. Dann aber zum Teil um so heftiger. Ein Beispiel ist die
\textit{Keit-Kontroverse} \index{Keit-Kontroverse} die schon zu einem Stehden
Begriff geworden ist.
George Keith (1638/9 -- 1716) \index{Orte:!Keith, George} war schottischer
Quaker und Missionar, der in
Nordamerika tätig war, und durch seine Ansichten für heftige Kontroversen
sorgte. 

\medskip

In Nordamerika wurden die Quaker ebenfalls wirtschaftlich sehr erfolgreich. Dort
sind sie untrennbar mit dem Walfang \index{Walfang} verbunden. In
\textit{Moby-Dick} setzt
Herman Melville \index{Personen:!Melville, Herman} dem bludtigen Gewerbe ein
Literarisches Denkmal, wenn man so
will. Auch dieses Werk, trug zum allgemeinen Bild der Quaker bei.

\medskip

Aber es gab auch andere Gebiete als die Wirtschaft, in den sich Quaker noch
einen
Namen gemacht haben. So in der Sklaven-Frage \index{Sklaverei}. Nur fünf Jahre
nach eintreffen
der ersten deutschen Quaker in Pennsylvania inizierte Franz Daniel Pastorius
\index{Personen:!Pastorius, Franz Daniel}
1688 \textit{Germantown Quaker Petition Against Slavery}
\index{Orte:!Germantown}
\footnote{Wikipedia,
\\The 1688 Germantown Quaker Petition Against Slavery,
\\http://en.wikipedia.org/wiki/The\_1688\_Germantown\_Quaker\_Petition\_Against\
\_Slavery.} Das Thema Sklaverei wird die Quaker die nächsten fast 100 Jahre
nicht
los lasse. Denn erst 1758 beschloss Philadelphia \index{Orte:!Philadelphia} als
erste Jahresversammlung,
unter Androhung des Ausschlusses, ihren eigenen Mitglieder, die Sklaverei.

\medskip

Auch in Britanien kämpften die Quaker um diese Zeit gegen die Sklaverei.
Allerdings nicht in ihren eigenen Reihen, sondern bei bzw. gegen die Hersteller
und Händler von Produkten der Sklavenbesitzer. Also z.T. gegen ihrer
Glaubensbrüber in Nordamerika. Den Quakern in Britanien darf wohl zugestanden
werden, in diesen Zusammenhang sogar die Erfinder des Produkt-Boykotts
\index{Boykott} zu sein.
Sie zogen durch das Land und hielten Vorträge über Sklaverei um so Konsumenten
zu
bewegen, keine Produkte zu kaufen, die mit Sklaverei zu tun hatte. Und sie
wahren überaus erfolgreich dabei. Was bedeutet, das sie einige Profitöre der
Sklaverei in den Ruin trieben.

\medskip

Aber auch in Nordamerika \index{Orte:!Nordamerika} scheuten sich einige Quaker
nicht davor,
Sklavenbesitzer unmittelbar schaden an ihrem \textit{Eigentum} zuzuführen, in
dem sie aktiv an der Sklavenbefreiunge beteiligen. An der so genannten
\textit{Underground Railroad} \index{Underground Railroad} ein ausgeklügeltes
System von Fluchthelfer. Das
besondere dabei war, das jeder Fluchthelfer \index{Personen:!Fluchthelfer} immer
nur ein Knotenpunkt in einem
Netzwerk darstellte. So das wenn ein Knoten ausfiel, er durch andere ersetzt
wurde. Das macht das System sehr robust gegen Störungen. Heute würde man wohl
von ein \textit{Peer-to-Peer-Netzwerk} sprechen

\medskip

Auch wenn das \textit{Heilige Experiment} also der Quakerstaat nur 80 Jahre
bestand hatte \footnote{immer hin! Die Täufer in Neuen Jerusalem (Münster)
\index{Orte:!Jerusalem!Neuen} \index{Orte:!Münster} haben
nicht mal zwei Jahre durchgehalten.} in Pennsylvania, so verlief die weiter
Quakergeschichte in Nordamerika wesentlich interessant, als in der \textit{Alten
Welt} -- Europa. Die Missionierungsversuche \index{Missionierung} der Britischen
Quaker in
\textit{Europa} \index{Orte:!Europa} -- also dem Festland, blieben ohne
langfristigen Erfolg.
Entweder waren ihre Verfolger \index{Verfolgung} und Feinde so erbarmungslos,
das eine
Konvertierung einem Todesurteil gleich kam, so zumeist in den katholischen
\index{Orte:!katholischen Ländern}
Ländern, die von den Quakern \textit{dark countries} \index{Orte:!dark
countries} -- also: Länder der
Finsternis -- genannt wurden, oder es gab innergemeinschaftliche Probleme.
Gerade
für junge Männer war es attraktiver als konvertierter Quaker nicht Nordamerika
aus zuwandern, als sich mit den Sorgen und Nöten der \textit{Alten Welt} herum
zu
schlagen. Gerade auch auf Hinblick auf Heiratschangsen \index{Heirat}. In den
kleinen
Gemeinschaften auf dem Europäischen Festland, waren Wahl in der eigenen
Gemeinschaft sehr begrenzt und Heiraten ausserhalb der Gemeinschaft war (noch)
nicht erlaubt. Und so missionierten die Britischen Quaker indirekt für
Nordamerika.

\medskip

Typisch für die Nordamerikanische Quaker-Geschichte waren die zahlreichen
Spaltungen. Die erste Spaltung \index{Spaltung} entzündete sich anlässlich des
Amerikanischer
Unabhängigkeitskrieg an der Frage des Friedenzeugnisse \index{Friedenzeugnis}.
Die Meisten Quaker
standen fest zu der Überzeugung, sich an keinem Krieg \index{Krieg} zu
beteiligen. Einige
wenige hielten das Friedenzeugnis für entbehrlich und spalteten sich zu den
\textit{Freien Quaker} \index{Personen:!Freien Quaker} \index{Quaker!Freien} ab,
die sich dann an dem Unabhängigkeitskrieg \index{Unabhängigkeitskrieg}
beteiligten. Dieser Zweig starb 1834 aus. Die nächste grösser Spaltung war die
der Ann Lee \index{Personen:!Lee, Ann}. deren Abspaltung -- die sich theologisch
völlig verselbständigte --
wurde \textit{Shaker} \index{Personen:!Shaker} genannt. Den Höhepunkt hatte
diese Bewegung in der Mitte
des 19. Jahrhunderts.

\medskip

Zum Beginn des 19. Jahrhunderts kam es zu einer Spaltung, die das Quakertum in
zwei Lager spaltete. In die \textit{Hicksites} \index{Quaker!Hicksites} und die
\textit{Orthodox
Friends} \index{Quaker!Orthodox Friends}. Im Gegensatz zu der Shaker-Spaltung,
beanspruchten beide weiter für
sich das \textit{wahre Quaakertum} zu vertreten. Der Streit entzündete sich an
der Person Elias Hicks (1748 -- 1830) \index{Personen:!Hicks, Elias}. Deshalb
auch der Name \textit{Hicksites}.
Hicks war seid sein 27 Lebensjahr von seiner Andachtsgemeinschaft als Prediger
anerkannt. Er war Abolitionisten \index{Personen:!Abolitionisten}, also Kämpfer
für die Sklavenbefreiung. Auch
er, organisierte Produkt-Boykotts von Sklaverei-produzierten Waren. In dem
Konflikt ging es vermutlich nicht nur um theologische Fragen, sondern
unterschwellig auch um die Antipathie zwischen den Ländlichen und Urbanen
Quakern. Die theologischen kontroversen drehten sich um die Frage ob Jesus
göttlicher Natur sei, ob die Errettung nur durch den Opfertod Jesus möglich ist
und ob das Innere Licht \index{Inneres Licht} -- also die innere Offenbarung --
über der Bibel \index{Bibel} stehe.
Die Gegner warfen Hicks Härisi vor und unterstellten ihm, von zentralen
Aussagen des Christentums abzuweichen. Hicks ist in der quietistischen Tradition
von John Woolman \index{Personen:!Woolman, John} und Job Scott
\index{Personen:!Scott, Job} zu sehen. Seine Gegner, die \textit{Orthodox
Friends} \index{Quaker!Orthodox Friends} vertraten er Positionen, quakerfremder
christlicher Traditionen. Die
\textit{Orthodox Friends} störten sich auch an der Position von Hicks, der nicht
an einen Teufel \index{Personen:!Teufel} glaubte, der das Böse zu den Menschen
brächte, sondern er
glaubte, das die Leidenschaft (im negativen Sinne) ein Teil der Menschlichen
Natur war, und das die Sünde \index{Sünde} somit im Menschen ist. 1827 spaltete
sich das
Philadelphia-Jahresversammlung und es gab auf einmal zwei Jahresversammlungen,
die von sich behaupteten \textit{die} Philadelphia-Jahresversammlung zu sein.
Ihr folgten dann in den nächsten Jahren New York \index{Orte:!New York}, in
Baltimore \index{Orte:!Baltimore}, Ohio \index{Orte:!Ohio} und
Indiana \index{Orte:!Indiana}.

\medskip

Ein paar Jahre später (1843) kam es dann im \textit{Orthodox Friends}-Flügel
noch mal zu einer Aufspaltung. zum Gurneyite-Wilburite-Spaltung
\index{Gurneyite-Wilburite-Spaltung}. Die beiden
Kontrahenten waren diesmal John Wilbur (1774 -- 1856) \index{Personen:!Wilbur,
John} und Joseph John Gurney
(1788 -- 1847) \index{Personen:!Gurney, John}. Gurney war Britte, Bankier
\index{Personen:!Bankier} und Bruder von Elizabeth Fry \index{Personen:!Fry,
Elizabeth}, die für
ihre Gefängnisarbeit \index{Orte:!Gefängnis!Gefängnisarbeit} und ihr Eintreten gegen
die Todesstrafe \index{Todesstrafe} bekannt werden
sollte. Er war Auch er war Prediger \index{Personen:!Prediger} und in dieser
Funktion von seiner
Versammlung autorisiert (im englischen \textit{recorded Quaker minister}).Als
Gurney auf seiner Reise in Nordamerika predigte, kritisierte er, das unter den
Quakern die Bibel \index{Bibel} vernachlässigt würde und er betonte den Glauben
an die
Rettung durch Glauben an Christus. Somit vertrat er traditionellen
protestantische Theologie \footnote{Theologie!protestantische}. Aus seinen
Anhängern, die \textit{Gurneyite Quakers}
genannt wurden, spalteten sich später die evangelikalen Quaker
\footnote{Quaker!evangelikale} ab.

\medskip

Wilbur war in Rhode Island \index{Orte:!Rhode Island} geboren. Also
Nordamerikaner. Er reiste seinerseits
nach Britanien, und bemerkte das die dortigen Quaker nach dem Hicksites-Schisma
in seinen Augen über das Ziel hinaus schossen, in dem sie die Bibel nun
überbetonen und das \textit{Innere Licht} durch Rationalismus drohte verdrängt
zu werden. Wilbur zitierte wieder die Frühen Freunde Robert Barclay
\index{Personen:!Barclay, Robert}, William
Penn \index{Personen:!Penn, William}, and George Fox \index{Personen:!Fox,
George}, um an die Basis des Quäkertums zu erinnern. Auch wenn er
nach wie vor nicht in zweifel zog, das die Bibel durch Gott inspiriert war und
ein Leitfaden der Frühen Quaker darstellte. Die Anhänger von Wilbur wurden
\textit{Wilburites}. Sie waren deutlich in der Minderzahl zu den
\textit{Gurneyite Quakers}. Aus den \textit{Wilburites} sollte später die
\textit{Conservative Friends} \footnote{Quaker!Conservative Friends} -- also die
\textit{Konservativen Quaker} --
hervorgehen.

\medskip

In den nächsten Jahren gab es dann noch zahlreiche kleinere Spaltungen
Umbenennungen, Neugründung und Wiedervereinigungen, wie zum Beispiel die
\textit{Beanite Qaker} \index{Quaker!Beanite Qaker}, die \textit{Nontheist
Friend} \index{Quaker!Nontheist Friend}, das \textit{Quaker Universalist
Fellowship} \index{Quaker!Quaker Universalist Fellowship}, das \textit{New
Foundation Fellowship} \index{Quaker!New Foundation Fellowship}, die
\textit{Richmond Declaration} \index{Richmond Declaration}, das \textit{Five
Years Meeting}, das \textit{Evangelical Friends International}
\index{Quaker!Evangelical Friends International}, das \textit{Friends United
Meeting} \index{Quaker!Friends United Meeting}, die \textit{Friends General
Conference} \index{Quaker!Friends General Conference} und das \textit{Friends
World Committee for Consultation} und soweiter... Wo man so schnell auch nicht
durchblickt.
Die äusserlichen Erkennungsmerkmale, die \textit{Quaker-Uniformem}
\index{Quaker!Uniformem} wurden
langsam ausrangiert. Je nach Quaker-Zweig früher oder etwas später. Im 20.
Jahrhundert war sie dann weitestgehend verschwunden. Es wurde noch weiterhin auf
nicht all zu dekadentes Auftreten geachtet, aber an sonsten passte man sich
optisch weitestgehend der Gesellschaft an. Die Evangelikalen Quaker begannen
sogar in ihren Versammlungen zu singen, richteten Pastoren-Ämter ein und nannten
fortan ihrer Versammlungshäuser \textit{Kirchen} \index{Kirche!Bezeichnung}
(englisch: \textit{church}),
ein Begriff gegen den die Frühen Freunde gewettert hatten. Auch in "`Ohne Kreuz
keine Krone"' machte W. Penn klar, das die Kirche die \textit{Gemeinschaft der
Gläubigen} war, und nicht ein Gebäude aus Holz oder Stein. G. Fox verspottete
die "`Gotteshäuser"' der "`Namenschristen"' gar als
"`Turmhäuser"'\index{Kirche!Turmhäuser}, wegen ihrer
Glockentürme. Bis Heute hat sich der Evangelikale Flügel des Quakertums soweit
den konventionellen protestantischen Kirchen angenähert, das es kaum noch ein
erkennbaren unterschied gibt. Aber dieser Zweig war der schnellst wachsende!

\medskip

Zum Beginn des 20. Jahrhundert steckte das Quakertum in einer Identitätskrise.
Gab kam mehr neue Impulse. Weder von Aussen noch von innen. Das Quakertum war
nach Innen und nach Aussen immer weniger attraktiv geworden. Dann brachen die
beiden Katastrophen -- der erste und der zweite Weltkrieg\index{Weltkriege} --
über die Welt
herein. In dieser Zeit stand auf ein mal das Friedenszeugnis
\index{Friedenszeugnis} -- \textit{wie
Phönix aus der Asche} -- wieder auf. Allen voran war es Rufus Matthew Jones
\index{Personen:!Jones, Rufus Matthew} dem
es gelang, die heillos zerstrittenen Quaker-Flügel wieder unter der Flagge des
Friedenszeugnisses zu vereinen. Das Friedenszeugnis war auf einmal ein
gemeinsamer Identitässtifter \index{Identitässtiftent}. Gerade Deutschland
\index{Orte:!Deutschland} profitierte von der Hilfe der
Quaker. Vermutlich wurden tausende Kinder nach dem ersten und zweiten Weltkrieg,
durch die Quakerspeisung \index{Quaker!Speisung} vor dem Verhungern gerettet.
Zwischen den Weltkriegen
wurden bei den so genannten \textit{Kindertransporten} \index{Kindertransporten}
tausende jüdische Kinder \index{Personen:!Kinder!jüdische}
aus der Vernichtungsmaschinerie der Nazis gerettet. In beiden Weltkriegen war es
für Quaker bestimmt nicht immer leicht, sich gegen einen \textit{gerechten
Krieg} zu stellen. Und bis Heute kann man darüber diskutieren, ob es richtig
war, das die Quaker sich weigerten die Waffen gegen Nazi-Deutschland zu
ergreifen.

\medskip

In der zweiten Hälfte des 20. Jahrhundert wurde das Quakertum quasi auf den Kopf
gestellt. Hatten noch 1940 die meisten Quaker auf der Nordhalbkugel der Erde
gewohnt. So waren es im Jahre 2000 mehr, die auf der Südhalbkugel gewohnt haben:

\begin{center}
\begin{tabular}{|l|r|r|} \hline
                        & \textbf{1940}        & \textbf{2000}    \\ \hline
\hline
Nordamerika             & 112.300              & 92.300           \\ \hline
Europa                  & 22.600               & 19.100           \\ \hline
\textbf{Nordhalbkugel}  & \textbf{134.900}     & \textbf{111.400} \\ \hline
Afrika                  & 13.200               & 156.200          \\ \hline
Mittel/Südamerika       & 7.200                & 60.600           \\ \hline
\textbf{Südhalbkugel}   & \textbf{20.400}      & \textbf{216.800} \\ \hline
\end{tabular}
\end{center}


Da die Evangelikalen Quaker in erster Linie in der Mission tätig waren, ist es
auch so, das die meisten Quaker auf der Südhalbkugel (und somit auf der Welt)
Evangelikale Quaker sind.