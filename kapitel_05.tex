% STATUS=Unfertig
% FUSSNOTEN=1.Durchlauf
% ANFÜHRUNGSZEICHEN=ok
% KORSIVSCHRIFT=ok
% FETTSCHRIFT=ok
% RANDBEMERKUNGEN=%
% MODERNE_RECHTSCHREIBUNG=noch nicht begonnen
% WORTERSÄTZUNG=ok
% VERSCHLAGWORTUNG(INDEX)=ok
% REFERENZIERUNG=abgeschlossen
% BIBELSTELLEN=ungeprüft
% EDITORFUSSNOTEN=ok



\chapter{5. Kapitel} \label{kap5}

\section{Zusammenfassung des 5. Kapitel}
\footnotesize
\begin{description}
\item[1. Abschnitt] Von der unerlaubten Eigenliebe; sie ist zweifach: 1. in der
Religion  2. in der Moralität. (Seite: \pageref{kap5_ab1})
\item[2. Abschnitt] Von Denen, welche die meisen Formalitäten
\footnote{\texttt{'mehrste Formalität' ersetzt duchr 'meisen Formalitäten'}}, den
stärksten
Aberglauben und die größte Prachtliebe in ihrem Gottesdienste eingeführt haben.
(Seite: \pageref{kap5_ab2})
\item[3. Abschnitt] Gott tadelt dergleichen weltlichen Auffassungen
\footnote{\texttt{'fleischliche Begriffe' ersetzt duch 'weltlichen Auffassungen'}}.
(Seite:
\pageref{kap5_ab3})
\item[4. Abschnitt] Christus zog seine Jünger von der äußern jüdischen
Gottesverehrung ab, und stiftete eine geistigere. (Seite: \pageref{kap5_ab4})
\item[5. Abschnitt] Stephanus redet über diesen Gegenstand ganz deutlich und
ausführlich. (Seite: \pageref{kap5_ab5})
\item[6. Abschnitt] Paulus wendet den Ausdruck: Tempel, zweimal auf das Herz des
Menschen an. (Seite: \pageref{kap5_ab6})
\item[7. Abschnitt] Von dem Kreuze jener weltlichen Gottesverehrer. (Seite:
\pageref{kap5_ab7})
\item[8. Abschnitt] Da ihr Kreuz von Fleisch und Blut (von fleischlich gesinnten
Menschen) verfertigt ist; so kann dasselbe Fleisch und Blut nicht kreuzigen,
(Seite: \pageref{kap5_ab8})
\item[9. Abschnitt] Solche Kreuze sind Joche. die keinen Zwang auflegen. (Seite:
\pageref{kap5_ab9})
\item[10. Abschnitt] Von der Pracht dieser Kreuze, und von der Achtung. die man
ihnen erweiset. (Seite: \pageref{kap5_ab10})
\item[11. Abschnitt] Ein Einsiedlerleben ist keine wahre evangelische
Selbstüberwindung \footnote{\texttt{'Selbstverleugnung' ersetzt duch
'Selbstüberwindung'}}. (Seite: \pageref{kap5_ab11})
\item[12. Abschnitt] Vergleichung zwischen der Selbstüberwindung
\footnote{\texttt{'Selbstverleugnung' ersetzt duch 'Selbstüberwindung'}} Christi, und
derjenigen, der falschen Christen. Die seinige (Christi) führt zu einem reinen
Leben in
der Welt; die ihrige (der falschen Christen) zu einer selbsterwählten
Einkerkerung, damit sie nicht
durch die Welt in Versuchung gerathen. Von dem Unheile, welches in der Welt
entstehen würde, wenn solche Beispiele allgemeine Nachahmung fänden; es würde
nützliche Gesellschaften und ehrliche Betriebsamkeit zerstören. -- Ein solches
unthätiges Leben ist die gewöhnliche Zuflucht der Müßiggänger, der vornehmen
Armen, und des mit Schuld beladenen Alters. (Seite: \pageref{kap5_ab12})
\item[13. Abschnitt] Von den Wirrungen des Kreuzes Christi in diesen Fällen, und
von der Unmöglichkeit, daß hier die Anwendung eines äußern Mittels einen innern
Schaden heilen könne. (Seite: \pageref{kap5_ab13})
\item[14. Abschnitt] Eine Ermahnung an die Leute dieses Glaubens, daß sie sich
nicht selbst betrügen mögen. (Seite: \pageref{kap5_ab14})

\end{description}
\normalsize

\section{1. Abschnitt} \label{kap5_ab1}

\index{Eigenliebe!unerlaubte}
Ich komme nun zur Betrachtung der unerlaubten Eigenliebe, deren Befriedigung,
mehr oder weniger, die wichtigsten Angelegenheit des größten Teiles der
Menschheit ist. Diese unerlaubte Eigenliebe erstreckt sich auf zwei Gegenstände,
\textbf{erstlich:} auf die religiöse Verehrung und Anbetung Gottes;
\textbf{zweitens:} auf das
moralische und bürgerliche Verhalten der Menschen überhaupt; und in beiden
Hinsichten ist eine Untersuchung derselben von der größten Bedeutung für uns.
Ich werde mich jedoch hierin so kurz fassen, als mein Gewissen es mir erlaubt,
und die Wichtigkeit der Sache es zuläßt.

\section{2. Abschnitt} \label{kap5_ab2}

\index{Gottesdienst}
Die unerlaubte Eigenliebe in der Religion, welche durch das Kreuz Christi
ertödtet werden muß, bestehet in jener menschlichen Erfindung und Verrichtung
einer Gottesverehrung, die für göttlich gehalten wird, es aber sowohl in ihrer
Stiftung als in ihrer Ausübung nicht ist. In diesem großen Irrthume stehen unter
Allen, die sich den Namen der Christen beilegen, diejenigen oben an, welche den
mehrsten äußern Schein, die größte Pracht, und den stärksten Aberglauben mit
ihrem Gottesdienste verbinden. Denn sie fehlen nicht allein in der Art der
Vollziehung ihres Gottesdienstes darin, daß sie dem allmächtigen Gott, der ein
ewiger Geist ist, ohne geistliche Vorbereitung \index{geistliche Vorbereitung}
des Herzens Verehrung und
Anbetung leisten; sondern ihre Gottesverehrung selbst ist aus Sachen
zusammengesetzt, die mit der Vorschrift und Ausübung der Lehre Christi und dem
apostolischen Beispiele gänzlich unvereinbar sind. Denn, statt daß jene wahre
Gottesverehrung einfach und geistlich war, ist die ihrige prachtvoll und
weltlich; statt daß die Anbetung Gottes, die Christus lehrete, ganz innerlich
war, und mit dem Gemüthe verrichtet wurde, ist die ihrige äußerlich, und wird
nur körperlich. vollzogen; statt daß jene der Natur und dem Wesen Gottes, als
eines unsichtbaren Geistes, angemessen war, ist diese nur zur Unterhaltung der
Sinnlichkeit des Menschen eingerichtet; so daß wir also da, wo Fleisch und Blut,
oder die fleischlichen Begriffe und Neigungen des Menschen ganz ausgeschlossen
sein sollten, eine Gottesverehrung erblicken, die völlig darauf berechnet ist,
denselben zu behagen; als wenn es das Geschäft der Gottesverehrer nicht wäre,
Gott eine ihm wohlgefällige Verehrung und Anbetung zu leisten, sondern vielmehr
darin bestände, daß sie eine Art des Gottesdienstes hervorbringen, die ihnen
selbst am besten gefällt. Daraus entstand jene Gottesverehrung, die mit
prachtvollen Gebäuden und Bildern, reichen Verzierungen und Gewanden, seltenen
Stimmen und auserlesener Musik\index{Kirchenmusik}, mit kostbaren Lampen,
Wachslichtern und
Räucherwerk\index{Räucherwerk}, und dieß Alles mit einer für die äußern Sinne so
einnehmenden
Abwechselung, als die Kunst nur erfinden und Kostenaufwand es erschwingen
konnte, aufs glänzendste geschmücket wurde. \textbf{Als ob die Menschen wieder
Juden
oder Egypter \index{Personen:!Juden}\index{Personen:!Egypter} werden sollten;
oder als wenn Gott \index{Gott!Gottesbild} wirklich ein sehr alter Mann
wäre, und Christus ein kleiner Knabe, dem man mit einer Art religiösen
Spielwerks eine Freude machen könnte; denn so bilden sie ihn in ihren Tempeln
und nur zu häufig auch in ihren Gemüthern ab. Man muß freilich gestehen, daß
eine solche Gottesverehrung solchen Begriffen von Gott völlig entspricht; denn,
wenn Menschen sich ihn so, wie sie selbst sind, vorstellen können, so ist es
auch nicht zu  bewundern, daß sie sich auf eine Art an ihn wenden und ihn so zu
unterhalten suchen, wie es ihnen selbst von Andern am besten gefällt.}

\section{3. Abschnitt} \label{kap5_ab3}


Was sagte aber einst der Allmächtige \index{Gott!Allmächtige, Der} zu einem so
sinnlichen Volke bei einer ähnlichen Veranlassung?
\textit{"`Du meinest, ich werde sein wie du; aber ich will dich
strafen, und will dir’s unter Augen stellen. Merket doch das, die ihr Gottes
vergesset, damit ich nicht einmal hinreisse, und sei kein Retter mehr da. Wer
Dank opfert, der preiset mich, und das ist der Weg, daß ich ihm zeige das Heil
Gottes."'} (Nach der englischen Bibel:)\textit{"`Und wer seinen Wandel richtig
ordnet, dem will ich das Heil Gottes zeigen."'}
\footnote{Psalm 50,21-23.}
\index{Bibelstellen:!Psalm 50)}
Die Verehrung, die Gott wirklich wohlgefällt, ist die, \label{Rechtfertigung} \index{Rechtfertigung} 
\textit{"`daß man Gerechtigkeit übe, Barmherzigkeit liebe und demüthig vor ihm
wandle."'}
\footnote{Micha 6, 8.}
\index{Bibelstellen:!Micha 6)}
Denn er,
\textit{"`der  Herzen und Nieren prüft, der dem Menschen seine Sünden vor Augen
stellt, der allein der Gott der Geister alles Fleisches ist,"'} er siehet nicht
auf das äußere Machwerk des Menschen, sondern auf die innere Stimmung und
Neigung seines Herzens. Es ist auch für Verständige nicht denkbar, daß Er, von
dem es heißt:
\textit{"`Du bist sehr herrlich, du bist schön und prächtig geschmückt;
Licht ist dein Kleid, das du anhast; du breitest den Himmel aus wie einen
Teppich; du wölbest ihn oben mit Wasser; du fährst auf den Wollen wie auf
einem.Wagen, und gehest auf den Fittigen des Windes; der du deine Engel zu
Winden (Geistern), und deine Diener zu Feuerflammen machst; der du das Erdreich
gründetest aus seinen Boden, daß es immer und ewig bleibe;"'}
\footnote{Psalm 104,1-6}
\index{Bibelstellen:!Psalm 104)}
-- ich sage, es ist nicht denkbar, daß dieses Wesen durch solche
menschliche Erfindungen verehret werden könne, zu welchen ein von der
ursprünglichen Kraft der Religion und von der geistlichen Beschaffenheit der
christlichen Gottesverehrung abgewichenes Volk seine Zuflucht genommen hat.

\section{4. Abschnitt} \label{kap5_ab4}

\index{Liturgie} \index{Tempel} \index{Orte:!Jerusalem}
\textbf{Christus zog seine Jüngers von jener prachtvollen und zeremoniellen
Gottesverehrung im äußern Tempel gänzlich ab, und stiftete eine innere und
geistliche Anbetung, worin er sie unterrichtete.
\index{Personen:!Samaritisches Weibe}
\textit{"`Ihr sollt weder auf diesem Berge, noch zu Jerusalem,"'}
sagte er zu dem samaritischen Weibe,
\textit{"`den Vater anbeten; Gott ist ein· Geist, und diejenigen, welche ihn
anbeten, müssen ihn im Geiste und in der Wahrheit anbeten."'}
\footnote{Johannes 4,21+24.}
\index{Bibelstellen:!Johannes 4)}
}
Als hätte er
gesagt: Gott, hat zwar vorzeiten um der Schwachheit des Volke willen sich
herabgelassen, die Verehrung, die ihm geleistet werden sollte, auf Zeit und Ort,
auf einen äußern Tempel und äußere zeremonielle Vorrichtungen zu beschränken;
allein dieses geschah zur Zeit der Unwissenheit der Menschen, als sie seine
Allgegenwart noch nicht erkannten, und nicht einsahen, was Gott ist, und wo er
ist. Jetzt bin ich aber gekommen, ihn Allen, die mich aufnehmen, zu offenbaren.
Und nun sage ich euch, daß Gott ein Geist ist, und daß er im Geiste und in der
Wahrheit will angebetet sein. Die Menschen müssen ihn als einen Geist kennen
lernen, ihn als Solchen betrachten, verehren und anbeten. Es ist nicht diese
leibliche Anbetung, auch sind es nicht diese seht unter euch üblichen Gebräuche
und zeremoniellen Beobachtungen, womit ihm gedient ist, oder welche euch diesem
Gott, der ein Geist ist, angenehm machen können. Nein! \textbf{ihr müßt seinem
Geiste
Gehorsam leisten, der innerlich mit euch ringet \index{Gott!Mit Gott ringe}, um euch
von den bösen Dingen in
der Welt. abzuziehen, damit ihr, indem ihr aus die Belehrungen dieses Geistes in
euern Herzen aufmerksam seid, und euch seinen Geboten unterwerfet, einsehen und
verstehen lernet, was es heißt, Gott als einen Geist anzubeten. Dann werdet ihr
begreifen, wie die wahre Gottesverehrung nicht darin bestehet, daß man auf
diesen Berg oder nach Jerusalem gehe, sondern darin: daß man den Willen Gottes
thue, seine Gebote halte, mit seinem eigenen Herzen sich unterrede und nicht
mehr sündige; daß man sein Kreuz aufnehme, das heilige Gesetz Gottes in seinem
Herzen betrachte, und dem Beispiele Desjenigen folge, den der Vater gesandt
hat.}

\section{5. Abschnitt} \label{kap5_ab5}

\index{Personen:!Stephanus} \index{Märtyrer} \index{Personen:!Juden}
Da nun hernach Stephanus, jener kühne und standhafte Märtyrer Jesu, als
Gefangener vor die Rathsversammlung der Juden geführt, und wegen seiner
Behauptung, daß es mit ihrem geliebten Tempel und mit den Bedienungen desselben
ein Ende nehmen sollte, fälschlich der Gotteslästerung. \index{Gott!Gotteslästerung}
beschuldigt wurde, gab
er ihnen diese Erklärung:
\textit{"`Salomeo,"'} \index{Personen:!Salomo} sagte er,
\textit{"`erbauete Gott ein Haus. Aber der Allmachtige wohnet nicht in Tempeln,
die mit Händen gemacht sind, wie der Prophet spricht: Der Himmel ist mein Stuhl,
und die Erde meiner Füße Schemel, was für ein Haus wollt ihr mir denn bauen?
spricht der Herr, oder welches ist die Stätte meiner Ruhe? Hat nicht meine Hand
das Alles gemacht?"'}
\footnote{Apostelgeschichte 7.}
\index{Bibelstellen:!Apostelgeschichte 7)}
Hier sehen wir klar die gänzliche Überflüssigkeit der weltlichen
Tempel und aller ihnen anklebenden Zeremonien \index{Zeremonien}. -- Aber der
Märtyrer Jesu \index{Märtyrer} dringt
noch schärfer auf jene abgewichenen Juden \index{Personen:!Juden, Die
abgewichenen}ein, welche in jenen Zeiten die
prachitliebenden, zeremoniellen und weltlichgesinnten Gottesverehrer waren:
\textit{"`Ihr Halsstarrigen und Unbeschnittenen an Herzen und Ohren!"'} sagt er,
\textit{"`ihr widerstrebet allezeit dem heiligen Geiste; wie eure Väter taten,
so tut auch
ihr."'}, Das ist so viel, als wenn er gesagt hätte: Eure äußern Tempel,
Beobachtungen und Schattendienste, eure Ansprüche auf eure natürliche Abkunft
von Abraham  \index{Personen:!Abraham} und aus eure Nachfolge in der Religion
Mosis \index{Personen:!Mose}, helfen euch nichts;
ihr widerstrebet dem heiligen Geiste. und widersprechet seinen Belehrungen ihr
wollt euch seinem Rathe nicht unterwerfen, und eure Herzen sind nicht ungetheilt
auf Gott gerichtet. \textbf{Ihr seid Nachfolger der Ungerechtigkeit eurer Väter,
und
wenn ihr auch mit dem Munde die Propheten bewundert und lobet, so seid ihr
dennoch nicht ihre Nachfolger im Glauben und Leben.} \index{Eigenlob}

\medskip

Der Prophet Jesaias \index{Personen:!Jesaias} geht aber noch etwas weiter, als
was Stephanus von ihm
anführt; denn nachdem er gezeigt hat, was Gottes Haus, nämlich der Ort seiner
Ruhe, nicht sei, fährt er mit diesen Worten fort:
\textit{"`Ich sehe aber an den Elenden, der eines zerbrochenen Geistes ist, und
der sich fürchtet vor meinem Worte."'}
\footnote{Jesaja 66,2}
\index{Bibelstellen:!Jesaja 66)}
Siehe hier, o sinnlicher, abergläubiger Mensch! den
wahren Anbeter! Siehe hier den Ort der Ruhe Gottes! das Haus und den Tempel
Desjenigen, den der Himmel aller Himmel nicht fassen kann! Ein Haus, welches
die Eigenliebe nicht zu erbauen versteht, und das weder die Kunst noch die Macht
der Menschen zu bereiten oder einzuweihen vermag.

\section{6. Abschnitt} \label{kap5_ab6}

\index{Personen:!Paulus} \index{Tempel} 
Paulus, der große Apostel der Heiden, wendet zweimal das Wort: Tempel,
ausdrücklichl auf den Menschen an; einmal in seiner Epistel an die Gemeine zu
Korinth, wo er sagt:
\textit{"`Wisset ihr nicht, daß euer Leib ein Tempel des heiligen
Geistes ist, der in euch ist, welchen ihr von Gott habt?"'} u. s. w. ,
\footnote{1. Korinther 6,19}
\index{Bibelstellen:!1. Korinther 6)}
also kein Gebäude der menschlichen Hand und Kunst. \textbf{Dann sagt er
wieder zu denselben Leuten in seiner zweiten Epistel:
\textit{"`Ihr seid der Tempel des lebendigen Gottes wie Gott spricht:"'} (hier
führt er die durch den Propheten
geredeten Worte Gottes an,)
\textit{"`Ich will in ihnen wohnen, und in ihnen wandeln, und will ihr Gott
sein, und sie sollen mein Volk sein."'}
\footnote{2. Korinther 6,16}
\index{Bibelstellen:!2. Korinther 6)}
Dieses ist der
evangelische Tempel\index{Evangelische Tempel}, die christliche Kirche, deren
Schmuck nicht in Stickereien
und Verzierungen weltlicher Kunst und irdischen Reichthums, sondern in Gaben des
Geistes\index{Gaben des Geistes}; nämlich in Sanftmuth, Liebes, Glauben, Geduld,
Selbstverleugnung und
Wohltatigkeit bestehet.} Hier ist der Ort, den die ewige Weisheit,  "`die von
Ewigkeit her, ehe die Berge eingesenkt und die Hügel gebildet waren, bei Gott
war, zu ihrer Wohnung erkor, und wo sie wie die Weisheit es selbst ausdrückt, --
auf dem Erdboden spielet, und ihre Lust. bei den Menschenkindern
hat;"'
\footnote{Weisheit Salomos 8,22+23+25+31.}
\index{Bibelstellen:!Salomos 8)}
aber nicht in Häusern von Holz
und Steinen. Dieser lebendige Tempel ist herrlicher als Salomo's lebloses
Gebäude, welches denselben auch nur vorbildlich darstellte, so wie Salomo
selbst, als der Erbauer jenes Hauses, ein Vorbild von Christo war, der uns zu
geistlichen Tempeln Gottes erbauet. Es war in vorigen Zeiten verheißen,
\textit{"`daß die Herrlichkeit des leztern Hauses größer als die des erstern
werden
sollte."'}
\footnote{Hagai 2,10.}
\index{Bibelstellen:!Hagai 2)}
Dieses läßt sich sehr wohl hieraus deuten, und
ist nicht so zu verstehen, daß ein äußeres Tempelgebäude ein anderes an äußerm
Glanze übertreffen solle; denn wozu würde das nützen? Nein! die göttliche
Herrlichkeit, die Schönheit der Heiligkeit in dem evangelischen Gebäude, oder in
der christlichen Kirche\index{Kirche!Definition} \index{Kirche!christliche},
die aus wiedergebornen \index{Wiedergeborne} Gläubigen bestehet, sollte jene
äußere Herrlichkeit des Tempels Salomo’s weit übertreffen, welche in
Vergleichung mit der Klarheit der lettztern Tage, sich wie Fleisch zu Geist, wie
ein schwinden der Schatten zu dem unvergänglichen Wesen verhalten würde.

\medskip

Dessenungeachtet haben. die Christen ihre Versammlungsörter; die aber nicht auf
eine jüdische oder heidnische Art prachtvoll ausgerüstet, sondern schmucklos und
ohne Pomp sind; so daß sie mit der Einfachheit ihres gottseligen Lebens und mit
ihrer heiligen Lehre übereinstimmen. Denn Gottes Gegenwart \index{Gott!Gottes
Gegenwart} theilt sich nicht dem
Hause, sondern den in demselben Versammelten mit, und aus diesen, nicht aber aus
dem Hause, bestehet die evangelische Kirche \index{Kirche!Definition}
\index{Kirche!evangelische}. O! möchten doch Alle, die sich
Christen nennen, statt jener eingebildeten Heiligkeit\index{Heilig!eingebildete Heiligkeit}, die Häusern und Örtern
zugeschrieben wird, die wahre Heiligkeit durch dar: Bad oder Abwaschen der
wiedergebärenden Kraft und Gnade Gottes in ihren eigenen Herzen aus Erfahrung
kennen lernen, so würden sie wissen, was die Kirche ist, und wo man in diesen
Tagen der evangelischen Einrichtung den Ort findet, da Gott erscheint. Dieses
veranlasste den königlichen Propheten David \index{Personen:!David}, zu sagen:
"`Des Könige Tochter ist inwendig ganz herrlich; sie ist mit goldenen Stücken
bekleidet."'
\footnote{Psalm 45,14.}
\index{Bibelstellen:!Psalm 45)}
Worin bestehet diese Herrlichkeit im Innern der wahren Kirche\index{Kirche!wahre
}? oder was
ist das Gold, das diese Herrlichkeit ausmacht? Sage mir, o abergläubiger Mensch!
bestehet es in deinen prächtigen Tempeln, Altären\index{Altar}, Tischen,
Teppichen, Tapeten?
-- In deinen Gewänden, Orgeln\index{Orgel}, Stimmen, Lichtern, Lampen,
Rauchpfannen? -- In
deinen Silbergeräthen, Juwelen und dergleichen Verzierungen deiner weltlichen
Tempel\index{Tempel!weltlicher}? Keinesweges. Alle diese Dinge haben gar keine
Ähnlichkeit mit dem
göttlichen Schmucke der Tochter des Königs der Himmel, der gebenedeieten und
erlöseten Gemeine Christi. O! des bejammernswerthen Abfalles! Welch ein elender
Ersatz für den Verlust und die Abwesenheit des, apostolischen Lebens, der
geistlichen Herrlichkeit der \textit{ersten Kirche} \index{Kirche!Die ersten}.

\section{7. Abschnitt} \label{kap5_ab7}

\index{Kreuz!Als Schmuk} Dennoch wollen einige dieser Bewunderer der äußern
Pracht und Herrlichkeit bei
ihrem Gottesdienste für Verehrer des Kreuzes Christi gehalten sein, und haben
daher eine ganze Menge Kreuze gemacht. Aber ach! wie können sie hoffen, etwas
mit dem Christenthume zu vereinigen, daß, jemehr es den Schein desselben
annimmt, nur desto weiter von seinem Wesen entfernt ist? Denn gerade ihr Kreuz
und ihre Selbstüberwindung \footnote{\texttt{'Selbstverleugnung' durch 'Selbstüberwindung' ersetzt}}
gehen aus den Wirkungen einer höchst unerlaubten
Eigenliebe \index{Eigenliebe!}hervor, und während sie sich einbilden, dadurch
Gott zu dienen,
verirren sie sich aus eine gefahrvolle Weise von dem wahren Kreuze Christi und
von jener heiligen Überwindung \footnote{\texttt{'Verleugnung' duch 'Überwindung'
ersezt}}, die Er vorgeschrieben hat. Es ist wahr, sie
haben wirklich ein Kreuz; allein es scheint nur ein Stellvertreter des wahren
Kreuzes \index{Kreuz!Das wahre}zu sein; und es ist so bescheiden und
nachgiebig, daß es Alles zugiebt,
was seine Träger nur wollen. Denn anstatt ihren Willen dadurch zu überwinden
\footnote{\texttt{'ertödten' ersetz durch 'überwinden'}},
richten sie es nach ihrem Willen ein, und gebrauchen es auch nur ihrem Belieben
gemäß! \textbf{Auf diese Art ist das Kreuz das Zeichen Derjenigen geworden, die
Alles
tun,  was sie gelüstet. Dennoch wollen sie, dieses Kreuzes wegen, für Jünger
Desjenigen gelten, der niemals seinen eigenen Willen, sondern allezeit den
Willen seines himmlischen Vaters tat.}\index{Eigenwillen}

\section{8. Abschnitt} \label{kap5_ab8}

\index{Kreuze!Als Schmuk} Da die weltlichen \footnote{\texttt{'fleischlichen' ersetzt
durch 'weltlichen'}} Begriffe der Menschen dieses Kreuz erdacht haben, so ist
es
so beschaffen, daß Weltlichgesinnte \footnote{\texttt{'Fleischlichgesinnte' ersetzt
duch 'Weltlichgesinnte'}} dasselbe gar wohl tragen können. Eben
darum ist es aber auch nicht das Kreuz Christi, (nicht die göttliche Kraft,)
wodurch Fleisch und Blut gekreuzigt werden. Tausende solcher Kreuze haben nicht
mehr Kraft und Wirkung, als das kleinste Stückchen Holz; es sind. armselige,
leere Schatten, die nicht einmal wirkliche Bilder des wahren Kreuzes sind.
Einige tragen sie als Zaubermittel bei sich, können aber niemals ein einziges
Uebel damit von sich abwehren. Sie sündigen mit dem Kreuze auf ihrem Rücken; und
wenn sie es auch an ihrem Busen tragen, so ruhen dennoch in demselben ihre
Lieblingslüste ganz ungestört. Wie die stummen Götzen \index{Kreuz!als
Götze}, welche Elia \index{Personen:!Elia} verspottete,
sind auch sie ohne Kraft und Leben. Und wie könnten sie auch andere sein; da sie
aus irdischer Masse bestehen, und ihre Gestalt und Verfertigung der Erfindung
und Arbeit weltlicher Künstler verdanken? \textbf{Ist es also wohl möglich, daß
solche
Kreuze Diejenigen, die sie tragen, zu bessern Menschen machen können? Nein!
wahrlich nicht.}

\section{9. Abschnitt} \label{kap5_ab9}

Es sind Joche, die den verderbten Neigungen der Menschen keinen Zwang anlegen;
\textbf{Kreuze, die ihnen nie widersprechen. Eine ganze Ladung solcher Kreuze
würde den
Menschen immer in demselben unverbesserten Zustande lassen, worin sie ihn
findet; denn sie können die sündlichen Neigungen aus seinem Herzen nicht
vertreiben.} Hiervon, fürchte ich, sind leider nur zu Viele, die sich des
falschen Kreuzes \index{Kreuz!das falsche} bedienen, ja dasselbe sogar anbeten
\index{Kreuz!anbetung}, in ihrem Innern überzeugt,
wiewohl sie sich den noch viel darauf einbilden daß sie es tragen.
\textbf{Dieses kann
aber nur der Fall mit den falschen; nie mit dem wahren Kreuze sein; da dieser
bei Denen, die es aufrichtig tragen, gar keinen Stolz duldet.}

\section{10. Abschnitt} \label{kap5_ab10}

So wie aber die Religion solcher Menschen beschaffen ist, so hat auch ihr Kreuz
ein sehr prächtiges, triumphirendes, hinreißendes Ansehen. Aber worin bestehet
dieses? In kostbaren Metallen und Edelsteinen; -- eine Beute, die der Aberglaube
ihren Geldsäcken raubt. Statt daß diese Kreuze die Herzen Derer, die sie tragen,
belehren sollten, die irdischen Schätze zu verleugnen, sind sie vielmehr aus
Kostbarkeiten zusammengesetzt, und werden auch, wie ihre Besitzer selbst, nach
ihrer Pracht geschätzte Ein reich besetztes Kreuz findet viele Angaffer und
Bewunderer, ein ungeschmücktes hingegen wird, wie dieses auch bei andern Dingen
der Fall ist, nur wenig geachtet. Ich könnte mich in Ansehung dieses törichten
und eitlen Aberglaubens \index{Aberglaube} auf die eigene Überzeugung dieser Kreuzträger berufen;
und o! wie sehr verschieden ist das ihrige von dem heiligen Kreuze Jesu,
welches. die Sünde im Menschen zerstöret!

\section{11. Abschnitt} \label{kap5_ab11}

Auch ist ein eingesperrtes Klosterleben\index{Klosterleben}; worin so Viele eine
prahlerische
Gerechtigkeit suchen, eben so wenig zu empfehlen, oder nur im mindesten mit der
Natur und Eigenschaft des wahren Kreuzes vereinbar. Denn, wenn dasselbe auch
nicht, wie andere Dinge, unerlaubt wäre, so ist es doch unnatürlich; und wahre
Religion leitet nicht dazu an. Die Zelle des wahren Christen ist in seinem
Innern, wo die Seele, von der Sünde geschieden, eingeklöstert ist. \textbf{Diese
geistliche Zelle tragen die wahren Nachfolger Christi beständig mit sich, indem
sie sich nicht dem Umgange mit der Welt entziehen, sondern in ihrem Umgange mit
derselben, vor dem Bösen, das sie hat, in Acht nehmen. Klöster sind Wohnsitze
einer tragen, schmutzigen, unnützen Selbstüberwindung \footnote{\texttt{'Selbstverleugnung' erstzt durch 'Selbstüberwindung'}}
\index{Selbstüberwindung!unnützen}, wodurch die ,Menschen
Andern lästig fallen, um ihren Müßiggang \index{Müßiggang} zu nähren; eine
Art religiöser Narrenhäuser\index{religiöser Narrenhäuser}, in welche man die Verrückten einsperrt, damit sie draußen kein
Unheil anrichten; das mag wohl heißen: \textit{Patience par forçe}, (Geduld aus
Zwang); eine Selbstüberwindung \footnote{\texttt{'Selbstverleugnung' erstzt durch 'Selbstüberwindung'}} wider  Willen, welche die Menschen eher dumm als
tugendhaft macht, und sie lieber vor der Versuchung verschließt, als sie zur
Standhaftigkeit in derselben anleitet.} Als wenn es ein Verdienst wäre, daß nicht
zu tun, wozu man nicht versucht wird. Was aber das Auge nicht betrachtet, wird
das Herz nicht begehren und auch nicht zu bereuen haben.

\section{12. Abschnitt} \label{kap5_ab12}

Das Kreuz Christi bringt ganz andere Wirkungen hervor. Es macht den Menschen
fähig, die Welt wirklich zu überwinden und, selbst im Angesichte ihrer Lockungen
zum Bösen, ein reines Leben zu führen. Diejenigen, welche es tragen, sind nicht
an Ketten gelegt, aus Furcht, daß sie beißen möchten; oder eingeschlossen, damit
sie nicht gestohlen werden; nein! Christus, der Heerführer ihrer Seelen, giebt
ihnen Kraft, dem Bösen zu widerstehen, und das, was in den Augen Gottes recht
und gut ist, zu tun; die Welt in ihrem eitlen Wesen zu verachten, und lieber
ihren Tadel zu ertragen, als ihren Beifall zu suchen. \index{Beleidigung} Diese werden auch
angewiesen, nicht nur Andere nicht zu beleidigen, sondern selbst Diejenigen, von
denen sie beleidigt werden, -- wiewohl nicht ihrer Beleidigung wegen, -- zu
lieben.

\medskip

Was für eine Welt würden wir haben, wenn Jeder, aus Furcht zu übertreten, sich
zwischen vier Wänden einkerkern wollte? So soll es nicht sein. Ein vollkommen
christliches Leben verträgt sich mit jedem, unter den Menschen üblichen,
ehrlichen Geschäfte und Gewerbe. Auch ist jene strenge Abgeschiedenheit nicht
die Wirkung des Geistes Christi, sondern eine selbsterwählte fleischliche
Demuth; ein bloßer Kappzaum, den der Mensch ohne Grund oder Vorschrift \index{Vorschrift} sich
selbst macht oder anlegt. In allen solchen Dingen ist er offenbar sein eigener
Gesetzgeber, der sich selbst Regeln, Strafe und Lösepreis vorschreibt. Daher
jene unnatürliche Strenge\index{Strenge, unnatürliche}, die in keinem Zusammenhänge mit den Zwecken der
Schöpfung stehet, von Welchen das gesellschaftliche Leben eine der vornehmsten
ist. Dieses solle aber nicht aus Furcht vor dem Bösen, das wir in der
Gesellschaft antreffen, zerstöret, sondern, durch standhafte Bestrafung des
Lasters und durch hervorstechende Beispiele erprobter Tugend, von dem Bösen,
welches dasselbe verdirbt, gereinigt und befreiet werden. \textbf{Wahre
Gottseligkeit
treibt die Menschen nicht aus der Welt, sondern setzt sie in Stand, desto besser
in ihr zu leben, erwecket in ihnen ein Bestreben, sie zu verbessern,}und lehret
sie, ihr Licht nicht unter einen Scheffel, sondern mit dem Leuchter auf einen
Tisch zustellen. Ueberdieß ist das Klosterleben bloß eine Erfindung der
Selbstsucht, und daher kann das nicht die rechte Art, das Kreuz zu tragen, sein,
welche aus dem entsprungen ist, was durch das wahre Kreuz vernichtet werden muß.
Noch mehr: Diese Grille macht, daß Jeder für sich allein davon läuft \index{davon laufen}und die
Welt hinter sich zurückläßt, ohne sich darum zu bekümmern, ob sie verloren
gehet. \index{Verantwortung übernehmen}Die Christen sollten in den Fahrzeugen der Welt am Ruder stehen, und sie
ihrem Hafen zuführen, aber nicht an den Hintertheilen der Schiffe sich feige
davon schleichen, und die Andern in denselben ohne Steuermänner der Gefahr
überlassen, daß sie durch die Wuth der bösen Zeiten auf die Felsen oder
Sandbänke des Verderbens getrieben werden.

\medskip

\index{Müßiggang}\index{Armut}Ergreifen junge Leute das Klosterleben, so haben sie gewöhnlich dabei keinen
andern Zweck, als ihren Hang zum Müßiggange zu bemänteln oder sie haben sich
dazu bereden lassen, damit die Mitgabe ihrer übrigen Geschwister durch ihre
Einkerkerung vermehret werde. Der Müßiggänger sucht sich dadurch den
schmerzhaften Folgen seiner Untatigkeit, und der unbemittelte Vornehme der
Schande der Armuth zu entziehen. Der Eines hat zur Arbeit keine Lust, der Andere
verachtet sie als unter seiner Würde. Sind es endlich alte Leute, die sich dem
Klosterleben ergeben, so haben sie oft keinen andern Beweggrund dazu, als weil
ihr langes, mit Schuld belastetes Leben bei dem Aberglauben \index{Aberglauben}Zuflucht sucht.
Nachdem sie lange genug in andern Dingen ihrer Selbstliebe gefolgt sind, wollen
sie nun damit enden, daß sie eine selbsterwählte Religionsform ergreifen, um
Gott zu versöhnen.

\section{13. Abschnitt} \label{kap5_ab13}

Die wahre Aufnahme des Kreuzes Jesu bestehet in einer innern Gemüthsübung \index{Gemüthsübung}, in
einer genauen Vorsichtigkeit, Wachsamkeit und Inachtnehmung der Seele, daß sie
in Uebereinstimmung mit dem ihr geoffenbarten Willen Gottes
\index{Gott!offenbarter Wille} handeln möge. Muß
aber dabei nicht die Seele den Körper, und nicht der Körper die Seele regieren?
Und sollten daher nicht solche Menschen bedenken, daß keine äußere Zelle die
Seele vor bösen Begierden verschließen oder das Gemüth vor einer unendlichen
Menge sündlicher Vorstellungen bewahren kann? Die Gedanken des verderbten
menschlichen Herzens sind böse, und zwar unaufhörlich;\textbf{das Böse kommt von innen,
und nicht von außen} -- denn, wenn es sich auch von außen zeigt, und im Innern
des Herzens keine Ausnahme findet, so kann es keine. böse Wirkung hervorbringen;
-- wie kann denn nun hier die Anwendung eines äußern Mittels eine innere Ursache
aufheben? oder wie kann ein bloß körperlicher Zwang eine Einschränkung des
Geistes bewirken? Dieses wird gewiß weit weniger im eingesperrten und
gezwungenen als einem freien Zustande des Menschen der Fall sein; denn wo er am
wenigsten handelt, da hat er die mehrste Zeit zu denken, und insofern seine
Gedanken nicht durch einen höhern Einfluß geleitet werden, \textbf{sind in der Tat
Klöster der Welt schädlicher als Jahrmärkte und Börsen. Dennoch ist
Zurückgezogenheit eine eben so vortreffliche als nothwendige Sache.} \index{Zurückgezogenheit}

\section{14. Abschnitt} \label{kap5_ab14}

Dann aber, o Mensch! untersuche deinen Grund wohl. Prüfe, was es ist, das dich
zu deiner Abgeschiedenheit bewog, und wer dich in dieselbe versetzt hat; damit
es nicht am Ende sich zeige, wie du deine eigene Seele auf die ganze Ewigkeit
betrogen hast. \index{Missionierung, Grund für}Ich muß gestehen, daß ich mit Eifer für das Heil meiner
Mitmenschen besorgt bin, Denn da mir von meinem himmlischen Vater
Barmherzigkeit widerfahren ist, so wünsche ich vornehmlich, daß Keiner seine
Seele durch Selbstbetrug in Betreff der Religion verlieren möge, worin die
Mehrsten nur zu geneigt sind, Alles ununtersucht für ausgemacht anzunehmen, und
aus diese Weise durch Schmeichelei ihrer Eigenliebe und Vernachlässigung ihres
Heils einen unersetzlichen Verlust erleiden. Die innere bleibende Gerechtigkeit
Christi ist etwas ganz Anderes, als jene erzwungene Andacht, worin der arme
abergläubige Mensch seine Gerechtigkeit setzt; und in den Augen, Gottes mit
Beifall bestehen, ist weit vortrefflicher, als jene leibliche Übung in der
Religion, die bloß einer menschliche Erfindung ist. Die durch die Kraft des
Geistes Gottes erweckte und bewahrte Seele lebt ihm nach seiner eigenen
Anordnung, und betet ihn in seinem Geiste an; nämlich, in dem heiligen Gefühle
des Lebens und nach der Leitung seines Geistes, welches die wahre evangelische
Anbetung Gottes ist.

\medskip

\index{Zurückgezogenheit} \textbf{Jedoch wünsche ich auch nicht, so verstanden zu werden, als ob ich wahre
Zurückgezogenheit geringschätzte. Nein, sich billige nicht nur die Einsamkeit;
ich liebe sie auch und schätze sie hoch.} Christus selbst hat sie uns durch
sein Beispiel empfohlen. Er liebte und suchte oft einsame Örter auf Bergen, in
Gärten und an Seegestaden. \textbf{Einsamkeit ist zum Wachstume in der Gottseligkeit
nothwendig;} ich ehre die Tugend, die sie sucht und benützt, und wünsche, daß
mehr davon in der Welt anzutreffen wäre. Nur sollte sie frei gewählt, nicht
erzwungen sein. Denn wad für Vortheil kann das Gemüt aus der Einsamkeit ziehen,
wenn sie ihm eher zur Strafe als zum Vergnügen dienet. \textbf{Ja, ich habe es
bei
Andern, die daß Klosterleben \index{Klosterleben} nicht billigen, schon lange als einen Fehler
angesehen, daß sie keine Zufluchtsörter \index{Zufluchtsort} für Bekümmerte, Versuchte, Verlassene
und Religiösgesinnte haben, wo diese ungestört auf Gott harren und ihre
religiösen Gemüthsübungen durchgehen könnten, um hernach, gestärkt, und mit mehr
Kraft, sich selbst zu beherrschen, versehen, wieder in die Geschäfte des Lebens
treten zu können; wiewohl allerdings je weniger, die besser. Denn in freier
Einsamkeit wird göttliches Vergnügungen gefunden.}



