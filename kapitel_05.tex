

\chapter{Kapitel 5} \label{kap5}

\section{Zusammenfassung des 5. Kapitels}

\begin{description}
\item[1. Abschnitt] Von der unerlaubten Eigenliebe. Sie ist zweifach: 1. in der
Religion  2. in der Moral.
\dotfill \textit{Seite~\pageref{kap5_ab1}}\\
\item[2. Abschnitt] Von denen, welche die meisten Formalitäten
\footnote{\texttt{'mehrste Formalität' ersetzt duchr 'meisten Formalitäten'}}, den
stärksten
Aberglauben und die größte Prachtliebe in ihrem Gottesdienste eingeführt haben.
\dotfill \textit{Seite~\pageref{kap5_ab2}}\\
\item[3. Abschnitt] Gott tadelt dergleichen weltlichen Auffassungen
\footnote{\texttt{'fleischliche Begriffe' ersetzt duch 'weltlichen Auffassungen'}}.
\dotfill \textit{Seite~\pageref{kap5_ab3}}\\
\item[4. Abschnitt] Christus zog seine Jünger von der äußeren jüdischen
Gottesverehrung ab, und stiftete eine geistigere.
\dotfill \textit{Seite~\pageref{kap5_ab4}}\\
\item[5. Abschnitt] Stephanus redet über diesen Gegenstand ganz deutlich und
ausführlich.
\dotfill \textit{Seite~\pageref{kap5_ab5}}\\
\item[6. Abschnitt] Paulus wendet den Ausdruck "Tempel" zweimal auf das Herz des
Menschen an.
\dotfill \textit{Seite~\pageref{kap5_ab6}}\\
\item[7. Abschnitt] Von dem Kreuze jener weltlichen Gottesverehrer.
\dotfill \textit{Seite~\pageref{kap5_ab7}}\\
\item[8. Abschnitt] Da ihr Kreuz von Fleisch und Blut (von fleischlich gesinnten
Menschen) verfertigt ist, so kann dasselbe Fleisch und Blut nicht kreuzigen,
\dotfill \textit{Seite~\pageref{kap5_ab8}}\\
\item[9. Abschnitt] Solche Kreuze sind Joche, die keinen Zwang auflegen.
\dotfill \textit{Seite~\pageref{kap5_ab9}}\\
\item[10. Abschnitt] Von der Pracht dieser Kreuze und von der Achtung, die man
ihnen erweist.
\dotfill \textit{Seite~\pageref{kap5_ab10}}\\
\item[11. Abschnitt] Ein Einsiedlerleben ist keine wahre evangelische
Selbstüberwindung \footnote{\texttt{'Selbstverleugnung' ersetzt duch
'Selbstüberwindung'}}.
\dotfill \textit{Seite~\pageref{kap5_ab11}}\\
\item[12. Abschnitt] Vergleichung zwischen der Selbstüberwindung
\footnote{\texttt{'Selbstverleugnung' ersetzt duch 'Selbstüberwindung'}} Christi, und
derjenigen der falschen Christen. Die seinige (Christi) führt zu einem reinen
Leben in
der Welt, die ihre (der falschen Christen) zu einer selbsterwählten
Einkerkerung, damit sie nicht
durch die Welt in Versuchung geraten. Von dem Unheile, welches in der Welt
entstehen würde, wenn solche Beispiele allgemeine Nachahmung fänden, es würde
nützliche Gesellschaften und ehrliche Betriebsamkeit zerstören. -- Ein solches
untätiges Leben ist die gewöhnliche Zuflucht der Müßiggänger, der vornehmen
Armen und des mit Schuld beladenen Alters.
\dotfill \textit{Seite~\pageref{kap5_ab12}}\\
\item[13. Abschnitt] Von den Wirrungen des Kreuzes Christi in diesen Fällen und
von der Unmöglichkeit, dass hier die Anwendung eines äußern Mittels einen innern
Schaden heilen könne.
\dotfill \textit{Seite~\pageref{kap5_ab13}}\\
\item[14. Abschnitt] Eine Ermahnung an die Leute dieses Glaubens, dass sie sich
nicht selbst betrügen mögen.
\dotfill \textit{Seite~\pageref{kap5_ab14}}\\

\end{description}

\newpage

\section{1. Abschnitt} \label{kap5_ab1}

\index{Eigenliebe!unerlaubte}
Ich komme nun zur Betrachtung der unerlaubten Eigenliebe, deren Befriedigung,
mehr oder weniger, die wichtigsten Angelegenheit des größten Teils der
Menschheit ist. Diese unerlaubte Eigenliebe erstreckt sich auf zwei Gegenstände,
\textbf{erstlich:} auf die religiöse Verehrung und Anbetung Gottes,
\textbf{zweitens:} auf das
moralische und bürgerliche Verhalten der Menschen überhaupt und in beiden
Hinsichten ist eine Untersuchung derselben von der größten Bedeutung für uns.
Ich werde mich jedoch hierin so kurz fassen, als mein Gewissen es mir erlaubt,
und die Wichtigkeit der Sache es zulässt.

\section{2. Abschnitt} \label{kap5_ab2}

\index{Gottesdienst}
Die unerlaubte Eigenliebe in der Religion, welche durch das Kreuz Christi
getötet werden muss, bestehet in jener menschlichen Erfindung und Verrichtung
einer Gottesverehrung, die für göttlich gehalten wird, es aber sowohl in ihrer
Stiftung als in ihrer Ausübung nicht ist. In diesem großen Irrtum stehen unter
allen, die sich den Namen der Christen beilegen, diejenigen oben an, welche den
meisten äußeren Schein, die größte Pracht und den stärksten Aberglauben mit
ihrem Gottesdienste verbinden. Denn sie fehlen nicht allein in der Art der
Vollziehung ihres Gottesdienstes darin, dass sie dem allmächtigen Gott, der ein
ewiger Geist ist, ohne geistliche Vorbereitung \index{Vorbereitung!geistliche}
des Herzens Verehrung und
Anbetung leisten, sondern ihre Gottesverehrung selbst ist aus Sachen
zusammengesetzt, die mit der Vorschrift und Ausübung der Lehre Christi und dem
apostolischen Beispiel gänzlich unvereinbar sind. Denn, statt dass jene wahre
Gottesverehrung einfach und geistlich wäre, ist die ihre prachtvoll und
weltlich, statt dass die Anbetung Gottes, die Christus lehrte, ganz innerlich
wäre, und mit dem Gemüt verrichtet würde, ist die ihrige äußerlich und wird
nur körperlich vollzogen, statt dass jene der Natur und dem Wesen Gottes, als
eines unsichtbaren Geistes, angemessen wäre, ist diese nur zur Unterhaltung der
Sinnlichkeit des Menschen eingerichtet, so dass wir also da, wo Fleisch und Blut,
oder die fleischlichen Begriffe und Neigungen des Menschen ganz ausgeschlossen
sein sollten, eine Gottesverehrung erblicken, die völlig darauf berechnet ist,
denselben zu behagen, als wenn es das Geschäft der Gottesverehrer nicht wäre,
Gott eine ihm wohlgefällige Verehrung und Anbetung zu leisten, sondern vielmehr
darin bestände, dass sie eine Art des Gottesdienstes hervorbringen, die ihnen
selbst am besten gefällt. Daraus entstand jene Gottesverehrung, die mit
prachtvollen Gebäuden und Bildern, reichen Verzierungen und Gewändern, seltenen
Stimmen und auserlesener Musik\index{Kirche!Musik}, mit kostbaren Lampen,
Wachslichtern und
Räucherwerk\index{Räucherwerk}, und dies alles mit einer für die äußeren Sinne so
einnehmenden
Abwechselung, als die Kunst nur erfinden und Kostenaufwand es erschwingen
konnte, aufs glänzendste geschmückt wurde. \label{ref:05_02_kasperletheater}
\textbf{Als ob die Menschen wieder Juden
oder Ägypter \index{Personen:!Juden}\index{Personen:!Ägypter} werden sollten,
oder als wenn Gott \index{Gott!Gottesbild} wirklich ein sehr alter Mann
wäre und Christus ein kleiner Knabe, dem man mit einer Art religiösen
Spielwerks eine Freude machen könnte, denn so bilden sie ihn in ihren Tempeln
und nur zu häufig auch in ihren Gemütern ab. Man muss freilich gestehen, dass
eine solche Gottesverehrung solchen Begriffen von Gott völlig entspricht, denn,
wenn Menschen sich ihn so, wie sie selbst sind, vorstellen können, so ist es
auch nicht zu  verwundern, dass sie sich auf eine Art an ihn wenden und ihn so zu
unterhalten suchen, wie es ihnen selbst von anderen am besten gefällt.}

\section{3. Abschnitt} \label{kap5_ab3}


Was sagte aber einst der Allmächtige \index{Gott!Allmächtige, Der} zu einem so
sinnlichen Volk bei einer ähnlichen Veranlassung?
\textit{"`Du meinst, ich werde sein wie du, aber ich will dich
strafen, und will dir’s unter die Augen stellen. Merkt doch das, die ihr Gottes
vergesset, damit ich nicht einmal hinreise, und sei kein Retter mehr da. Wer
Dank opfert, der preist mich, und das ist der Weg, dass ich ihm das Heil
Gottes zeige."'} (Nach der englischen Bibel:)\textit{"`Und wer seinen Wandel richtig
ordnet, dem will ich das Heil Gottes zeigen."'}
\footnote{Psalm 50,21-23.}
\index{Bibelstellen:!Psalm 50)}
Die Verehrung, die Gott wirklich gefällt, ist die, \label{Rechtfertigung} \index{Rechtfertigung}
\textit{"`dass man Gerechtigkeit übt, Barmherzigkeit liebt und demütig vor ihm
wandelt."'}
\footnote{Micha 6,8.}
\index{Bibelstellen:!Micha 6)}
Denn er,
\textit{"`der  Herzen und Nieren prüft, der dem Menschen seine Sünden vor Augen
stellt, der allein der Gott der Geister alles Fleisches ist,"'} sieht nicht
auf das äußere Machwerk des Menschen, sondern auf die innere Stimmung und
Neigung seines Herzens. Es ist auch für Verständige nicht denkbar, dass er, von
dem es heißt:
\textit{"`Du bist sehr herrlich, du bist schön und prächtig geschmückt,
Licht ist dein Kleid, das du anhast. Du breitest den Himmel aus wie einen
Teppich, du wölbst ihn oben mit Wasser, du fährst auf den Wolken wie auf
einem Wagen, und gehst auf den Fittichen des Windes, der du deine Engel zu
Winden (Geistern) und deine Diener zu Feuerflammen machst, der du das Erdreich
gründetest aus seinen Boden, dass es immer und ewig bleibe;"'}
\footnote{Psalm 104,1-6}
\index{Bibelstellen:!Psalm 104)}
-- ich sage, es ist nicht denkbar, dass dieses Wesen durch solche
menschliche Erfindungen verehrt werden könne, zu welchen ein von der
ursprünglichen Kraft der Religion und von der geistlichen Beschaffenheit der
christlichen Gottesverehrung abgewichenes Volk seine Zuflucht genommen hat.

\section{4. Abschnitt} \label{kap5_ab4}

\index{Liturgie} \index{Tempel} \index{Orte:!Jerusalem}
\textbf{Christus zog seine Jünger von jener prachtvollen und zeremoniellen
Gottesverehrung im äußeren Tempel gänzlich ab und stiftete eine innere und
geistliche Anbetung, worin er sie unterrichtete.
\index{Personen:!Samaritisches Weibe}
\textit{"`Ihr sollt weder auf diesem Berge, noch zu Jerusalem,"'}
sagte er zu dem samaritischen Weibe,
\textit{"`den Vater anbeten, Gott ist ein Geist, und diejenigen, welche ihn
anbeten, müssen ihn im Geist und in der Wahrheit anbeten."'}
\footnote{Johannes 4,21+24.}
\index{Bibelstellen:!Johannes 4)}
}
Als hätte er
gesagt: Gott hat zwar vorzeiten um der Schwachheit des Volks willen sich
herabgelassen, die Verehrung, die ihm geleistet werden sollte, auf Zeit und Ort,
auf einen äußern Tempel und äußere zeremonielle Vorrichtungen zu beschränken,
allein dieses geschah zur Zeit der Unwissenheit der Menschen, als sie seine
Allgegenwart noch nicht erkannten und nicht einsahen, was Gott ist, und wo er
ist. Jetzt bin ich aber gekommen, ihn allen, die mich aufnehmen, zu offenbaren.
Und nun sage ich euch, dass Gott ein Geist ist, und dass er im Geiste und in der
Wahrheit will angebetet sein. Die Menschen müssen ihn als einen Geist kennen
lernen, ihn als solchen betrachten, verehren und anbeten. Es ist nicht diese
leibliche Anbetung, auch sind es nicht diese
jezt unter euch üblichen Gebräuche
und zeremoniellen Beobachtungen, womit ihm gedient ist, oder welche euch diesem
Gott, der ein Geist ist, angenehm machen können. Nein!
\label{ref:05_04_wahre_anbetung}
\textbf{Ihr müsst seinem Geiste
Gehorsam leisten, der innerlich mit euch ringt \index{Gott!Mit Gott ringen}, um euch von den bösen Dingen in
der Welt abzuziehen, damit ihr, indem ihr auf die Belehrungen dieses Geistes in
eueren Herzen aufmerksam seid, und euch seinen Geboten unterwerft, einsehen und
verstehen lernt, was es heißt, Gott als einen Geist anzubeten. Dann werdet ihr
begreifen, wie die wahre Gottesverehrung nicht darin besteht, dass man auf
diesen Berg oder nach Jerusalem geht, sondern darin, dass man den Willen Gottes
tut, seine Gebote hält, mit seinem eigenen Herzen sich unterredet und nicht
mehr sündigt, dass man sein Kreuz aufnimmt, das heilige Gesetz Gottes in seinem
Herzen betrachtet und dem Beispiel desjenigen folgt, den der Vater gesandt
hat.}

\section{5. Abschnitt} \label{kap5_ab5}

\index{Personen:!Stephanus} \index{Personen:!Märtyrer} \index{Personen:!Juden}
Da nun hernach Stephanus, jener kühne und standhafte Märtyrer Jesu, als
Gefangener vor die Ratsversammlung der Juden geführt und wegen seiner
Behauptung, dass es mit ihrem geliebten Tempel und mit den Bedienungen desselben
ein Ende nehmen sollte, fälschlich der Gotteslästerung. \index{Gott!Gotteslästerung}
beschuldigt wurde, gab
er ihnen diese Erklärung:
\textit{"`Salomo,"'} \index{Personen:!Salomo} sagte er,
\textit{"`erbaute Gott ein Haus. Aber der Allmachtige wohnt nicht in Tempeln,
die mit Händen gemacht sind, wie der Prophet spricht: Der Himmel ist mein Stuhl
und die Erde meiner Füße Schemel, was für ein Haus wollt ihr mir denn bauen?
Spricht der Herr, oder welches ist die Stätte meiner Ruhe? Hat nicht meine Hand
das alles gemacht?"'}
\footnote{Apostelgeschichte 7.}
\index{Bibelstellen:!Apostelgeschichte 7)}
Hier sehen wir klar die gänzliche Überflüssigkeit der weltlichen
Tempel und aller ihnen anklebenden Zeremonien \index{Zeremonien}. -- Aber der
Märtyrer Jesu \index{Personen:!Märtyrer} dringt
noch schärfer auf jene abgewichenen Juden \index{Personen:!Juden!abgewichenen} ein, welche in jenen Zeiten die
prachtliebenden, zeremoniellen und weltlich gesinnten Gottesverehrer waren:
\textit{"`Ihr Halsstarrigen und Unbeschnittenen an Herzen und Ohren!"'} sagt er,
\textit{"`ihr widerstrebt allezeit dem heiligen Geist, wie euere Väter taten,
so tut auch
ihr."'}, Das ist so viel, als wenn er gesagt hätte: Eure äußeren Tempel,
Beobachtungen und Schattendienste, eure Ansprüche auf euere natürliche Abkunft
von Abraham  \index{Personen:!Abraham} und aus euere Nachfolge in der Religion
Mose \index{Personen:!Mose} helfen euch nichts,
ihr widerstrebt dem heiligen Geiste und widersprecht seinen Belehrungen. Ihr
wollt euch seinem Rathe nicht unterwerfen und euere Herzen sind nicht ungeteilt
auf Gott gerichtet. \label{ref:05_05_wahre_nachfolge}
\textbf{Ihr seid Nachfolger der Ungerechtigkeit eurer Väter, und
wenn ihr auch mit dem Munde die Propheten bewundert und lobt, so seid ihr
dennoch nicht ihre Nachfolger im Glauben und Leben.} \index{Eigenlob}

\medskip

Der Prophet Jesaias \index{Personen:!Jesaias} geht aber noch etwas weiter, als
was Stephanus von ihm
anführt. Denn nachdem er gezeigt hat, was Gottes Haus, nämlich der Ort seiner
Ruhe, nicht sei, fährt er mit diesen Worten fort:
\textit{"`Ich sehe aber an den Elenden, der eines zerbrochenen Geistes ist und
der sich fürchtet vor meinem Worte."'}
\footnote{Jesaja 66,2}
\index{Bibelstellen:!Jesaja 66)}
Siehe hier - o sinnlicher, abergläubiger Mensch! - den
wahren Anbeter! Siehe hier den Ort der Ruhe Gottes! Das Haus und den Tempel
desjenigen, den der Himmel aller Himmel nicht fassen kann! Ein Haus, welches
die Eigenliebe nicht zu erbauen versteht, und das weder die Kunst noch die Macht
der Menschen zu bereiten oder einzuweihen vermag.

\section{6. Abschnitt} \label{kap5_ab6}

\index{Personen:!Paulus} \index{Tempel}
Paulus, der große Apostel der Heiden, wendet zweimal das Wort \textit{Tempel}
ausdrücklichl auf den Menschen an, einmal in seiner Epistel an die Gemeine zu
Korinth, wo er sagt:
\textit{"`Wisst ihr nicht, dass euer Leib ein Tempel des heiligen
Geistes ist, der in euch ist, welchen ihr von Gott habt?"'} u.s.w.,
\footnote{1. Korinther 6,19}
\index{Bibelstellen:!1. Korinther 6)}
also kein Gebäude der menschlichen Hand und Kunst.
\label{ref:05_06_tempel}
\textbf{Dann sagt er
wieder zu denselben Leuten in seiner zweiten Epistel:
\textit{"`Ihr seid der Tempel des lebendigen Gottes wie Gott spricht:"'} (hier
führt er die durch den Propheten
geredeten Worte Gottes an,)
\textit{"`Ich will in ihnen wohnen, und in ihnen wandeln, und will ihr Gott
sein, und sie sollen mein Volk sein."'}
\footnote{2. Korinther 6,16}
\index{Bibelstellen:!2. Korinther 6)}
Dieses ist der
evangelische Tempel\index{Tempel!Evangelische}, die christliche Kirche, deren
Schmuck nicht in Stickereien
und Verzierungen weltlicher Kunst und irdischen Reichtums, sondern in Gaben des
Geistes\index{Gaben!des Geistes}, nämlich in Sanftmut, Liebe, Glauben, Geduld,
Selbstverleugnung und
Wohltätigkeit besteht.} Hier ist der Ort, den die ewige Weisheit,  "`die von
Ewigkeit her, ehe die Berge eingesenkt und die Hügel gebildet waren, bei Gott
war, zu ihrer Wohnung erkor, und wo sie wie die Weisheit es selbst ausdrückt, --
auf dem Erdboden spielte und ihre Lust bei den Menschenkindern
hatte;"'
\footnote{Weisheit Salomos 8,22+23+25+31.}
\index{Bibelstellen:!Weisheit Salomos 8)}
aber nicht in Häusern von Holz
und Steinen. Dieser lebendige Tempel ist herrlicher als Salomos lebloses
Gebäude, welches denselben auch nur vorbildlich darstellte, so wie Salomo
selbst, als der Erbauer jenes Hauses, ein Vorbild von Christo war, der uns zu
geistlichen Tempeln Gottes erbaut. Es war in vorigen Zeiten verheißen,
\textit{"`dass die Herrlichkeit des lezten Hauses größer als die des erstern
werden
sollte."'}
\footnote{Hagai 2,10.}
\index{Bibelstellen:!Hagai 2)}
Dieses läßt sich sehr wohl hieraus deuten, und
ist nicht so zu verstehen, dass ein äußeres Tempelgebäude ein anderes an äußerem
Glanze übertreffen sollte, denn wozu würde das nützen? Nein! Die göttliche
Herrlichkeit, die Schönheit der Heiligkeit in dem evangelischen Gebäude, oder in
der christlichen Kirche\index{Kirche!Definition} \index{Kirche!christliche},
die aus wiedergeborenen \index{Wiedergeborn} Gläubigen besteht, sollte jene
äußere Herrlichkeit des Tempels Salomos weit übertreffen, welche in
Vergleichung mit der Klarheit der letzten Tage sich wie Fleisch zu Geist, wie
ein Schwinden der Schatten zu dem unvergänglichen Wesen verhalten würde.

\medskip

Dessen ungeachtet haben die Christen ihre Versammlungsorte, die aber nicht auf
eine jüdische oder heidnische Art prachtvoll ausgerüstet, sondern schmucklos und
ohne Pomp sind, so dass sie mit der Einfachheit ihres gottseligen Lebens und mit
ihrer heiligen Lehre übereinstimmen. Denn Gottes Gegenwart \index{Gott!Gottes
Gegenwart} teilt sich nicht dem
Hause, sondern den in demselben Versammelten mit, und aus diesen, nicht aber aus
dem Hause, besteht die evangelische Kirche \index{Kirche!Definition}
\index{Kirche!evangelische}. O! Möchten doch alle, die sich
Christen nennen, statt jener eingebildeten Heiligkeit\index{Heilig!eingebildete Heiligkeit}, die Häusern und Örtern
zugeschrieben wird, die wahre Heiligkeit durch das Bad oder Abwaschen der
wiedergebärenden Kraft und Gnade Gottes in ihren eigenen Herzen aus Erfahrung
kennen lernen, so würden sie wissen, was die Kirche ist, und wo man in diesen
Tagen der evangelischen Einrichtung den Ort findet, da Gott erscheint. Dieses
veranlasste den königlichen Prophet David \index{Personen:!David} zu sagen:
"`Des Königs Tochter ist inwendig ganz herrlich, sie ist mit goldenen Stücken
bekleidet."'
\footnote{Psalm 45,14.}
\index{Bibelstellen:!Psalm 45)}
Worin besteht diese Herrlichkeit im Innern der wahren Kirche\index{Kirche!wahre
}? oder was
ist das Gold, das diese Herrlichkeit ausmacht? Sage mir, o abergläubiger Mensch,
besteht es in deinen prächtigen Tempeln, Altären\index{Altar}, Tischen,
Teppichen, Tapeten?
-- In deinen Gewändern, Orgeln\index{Orgel}, Stimmen, Lichtern, Lampen,
Rauchpfannen? -- In
deinen Silbergeräten, Juwelen und dergleichen Verzierungen deiner weltlichen
Tempel\index{Tempel!weltlicher}? Keinesweges. Alle diese Dinge haben gar keine
Ähnlichkeit mit dem
göttlichen Schmuck der Tochter des Königs der Himmel, der gesegneten\footnote{\texttt{'gebenedeieten' ersetzt durch 'gesegneten'}}
und
erlösten Gemeine Christi. O des bejammernswerten Abfalls! Welch ein elender
Ersatz für den Verlust und die Abwesenheit des apostolischen Lebens, der
geistlichen Herrlichkeit der \textit{ersten Kirche} \index{Kirche!erste}.

\section{7. Abschnitt} \label{kap5_ab7}

\index{Kreuz!Als Schmuck} Dennoch wollen einige dieser Bewunderer der äußern
Pracht und Herrlichkeit bei
ihrem Gottesdienst für Verehrer des Kreuzes Christi gehalten werden und haben
daher eine ganze Menge Kreuze gemacht. Aber ach! Wie können sie hoffen, etwas
mit dem Christentum zu vereinigen, dass, je mehr es den Schein desselben
annimmt, nur desto weiter von seinem Wesen entfernt ist? Denn gerade ihr Kreuz
und ihre Selbstüberwindung \footnote{\texttt{'Selbstverleugnung' durch 'Selbstüberwindung' ersetzt}}
gehen aus den Wirkungen einer höchst unerlaubten
Eigenliebe \index{Eigenliebe}hervor, und während sie sich einbilden, dadurch
Gott zu dienen,
verirren sie sich auf eine gefahrvolle Weise von dem wahren Kreuze Christi und
von jener heiligen Überwindung \footnote{\texttt{'Verleugnung' duch 'Überwindung'
ersezt}}, die er vorgeschrieben hat. Es ist wahr, sie
haben wirklich ein Kreuz, allein es scheint nur ein Stellvertreter des wahren
Kreuzes \index{Kreuz!Das wahre}zu sein, und es ist so bescheiden und
nachgiebig, daß es alles zugibt,
was seine Träger nur wollen. Denn anstatt ihren Willen dadurch zu überwinden
\footnote{\texttt{'ertödten' ersetzt durch 'überwinden'}},
richten sie es nach ihrem Willen ein, und gebrauchen es auch nur ihrem Belieben
gemäß! \label{ref:05_07_kreuz}
\textbf{Auf diese Art ist das Kreuz das Zeichen derjenigen geworden, die
alles
tun,  was ihnen gelüstet. Dennoch wollen sie, dieses Kreuzes wegen, für Jünger
desjenigen gelten, der niemals seinen eigenen Willen, sondern allezeit den
Willen seines himmlischen Vaters tat.}\index{Eigenwillen}

\section{8. Abschnitt} \label{kap5_ab8}

\index{Kreuz!Als Schmuck} Da die weltlichen \footnote{\texttt{'fleischlichen' ersetzt durch 'weltlichen'}} Begriffe der Menschen dieses Kreuz erdacht haben, so ist es so beschaffen, dass weltlich Gesinnte \footnote{\texttt{'Fleischlichgesinnte' ersetzt
duch 'weltlich Gesinnte'}} dasselbe gar wohl tragen können. Eben
darum ist es aber auch nicht das Kreuz Christi, (nicht die göttliche Kraft,)
wodurch Fleisch und Blut gekreuzigt werden. Tausende solcher Kreuze haben nicht
mehr Kraft und Wirkung als das kleinste Stückchen Holz, es sind armselige,
leere Schatten, die nicht einmal wirkliche Bilder des wahren Kreuzes sind.
Einige tragen sie als Zaubermittel bei sich, können aber niemals ein einziges
Übel damit von sich abwehren. Sie sündigen mit dem Kreuz auf ihrem Rücken, und
wenn sie es auch an ihrem Busen tragen, so ruhen dennoch in demselben ihre
Lieblingslüste ganz ungestört. Wie die stummen Götzen \index{Kreuz! als
Götze}, welche Elia \index{Personen:!Elia} verspottete,
sind auch sie ohne Kraft und Leben. Und wie könnten sie auch andere sein, da sie
aus irdischer Masse bestehen, und ihre Gestalt und Verfertigung der Erfindung
und Arbeit weltlicher Künstler verdanken? \label{ref:05_08_kreuz}
\textbf{Ist es also wohl möglich, dass solche
Kreuze diejenigen, die sie tragen, zu besseren Menschen machen können? Nein!
wahrlich nicht.}

\section{9. Abschnitt} \label{kap5_ab9}

\label{ref:05_09_kreuz}
Es sind Joche, die den verderbten Neigungen der Menschen keinen Zwang anlegen,
\textbf{Kreuze, die ihnen nie widersprechen. Eine ganze Ladung solcher Kreuze
würde den
Menschen immer in demselben unverbesserten Zustande lassen, worin sie ihn
findet, denn sie können die sündlichen Neigungen aus seinem Herzen nicht
vertreiben.} Hiervon, fürchte ich, sind leider nur zu viele, die sich des
falschen Kreuzes \index{Kreuz!das falsche} bedienen, ja dasselbe sogar anbeten
\index{Kreuz! Anbetung}, in ihrem Innern überzeugt,
wiewohl sie sich dennnoch viel darauf einbilden, dass sie es tragen.
\textbf{Dieses kann
aber nur der Fall mit den falschen, nie mit dem wahren Kreuze sein, da dieser
bei denen, die es aufrichtig tragen, gar keinen Stolz duldet.}

\section{10. Abschnitt} \label{kap5_ab10}

So wie aber die Religion solcher Menschen beschaffen ist, so hat auch ihr Kreuz
ein sehr prächtiges, triumphirendes, hinreißendes Ansehen. Aber worin besteht
dieses? In kostbaren Metallen und Edelsteinen, -- eine Beute, die der Aberglaube
ihren Geldsäcken raubt. Statt dass diese Kreuze die Herzen derer, die sie tragen,
belehren sollte, die irdischen Schätze zu verleugnen, sind sie vielmehr aus
Kostbarkeiten zusammengesetzt, und werden auch, wie ihre Besitzer selbst, nach
ihrer Pracht geschätzte. Ein reich besetztes Kreuz findet viele Angaffer und
Bewunderer, ein ungeschmücktes hingegen wird, wie dieses auch bei anderen Dingen
der Fall ist, nur wenig geachtet. Ich könnte mich in Ansehung dieses törichten
und eitlen Aberglaubens \index{Aberglaube} auf die eigene Überzeugung dieser Kreuzträger berufen,
und o! wie sehr verschieden ist das ihrige von dem heiligen Kreuz Jesu,
welches die Sünde im Menschen zerstört!

\section{11. Abschnitt} \label{kap5_ab11}

Auch ist ein eingesperrtes Klosterleben\index{Klosterleben}, worin so viele eine
prahlerische
Gerechtigkeit suchen, eben so wenig zu empfehlen, oder nur im mindesten mit der
Natur und Eigenschaft des wahren Kreuzes vereinbar. Denn, wenn dasselbe auch
nicht, wie andere Dinge, unerlaubt wäre, so ist es doch unnatürlich und wahre
Religion leitet nicht dazu an. Die Zelle des wahren Christen ist in seinem
Inneren, wo die Seele, von der Sünde geschieden, eingeklostert ist.
\label{ref:05_11_kloster} \textbf{Diese
geistliche Zelle tragen die wahren Nachfolger Christi beständig mit sich, indem
sie sich nicht dem Umgang mit der Welt entziehen, sondern in ihrem Umgange mit
derselben, vor dem Bösen, das sie hat, in Acht nehmen. Klöster sind Wohnsitze
einer trägen, schmutzigen, unnützen Selbstüberwindung \footnote{\texttt{'Selbstverleugnung' erstzt durch 'Selbstüberwindung'}}
\index{Selbst!-überwindung!unnützen}, wodurch die Menschen
anderen lästig fallen, um ihren Müßiggang \index{Müßiggang} zu nähren, eine
Art religiöser Narrenhäuser\index{Narrenhäuser!religiöser}, in welche man die Verrückten einsperrt, damit sie draußen kein
Unheil anrichten, das mag wohl heißen: \textit{Patience par forçe}, (Geduld aus
Zwang), eine Selbstüberwindung \footnote{\texttt{'Selbstverleugnung' erstzt durch 'Selbstüberwindung'}} wider  Willen, welche die Menschen eher dumm als
tugendhaft macht, und sie lieber vor der Versuchung verschließt, als sie zur
Standhaftigkeit in derselben anleitet.} Als wenn es ein Verdienst wäre, dass nicht
zu tun, wozu man nicht versucht wird. Was aber das Auge nicht betrachtet, wird
das Herz nicht begehren und auch nicht zu bereuen haben.

\section{12. Abschnitt} \label{kap5_ab12}

Das Kreuz Christi bringt ganz andere Wirkungen hervor. Es macht den Menschen
fähig, die Welt wirklich zu überwinden und, selbst im Angesichte ihrer Lockungen
zum Bösen, ein reines Leben zu führen. Diejenigen, welche es tragen, sind nicht
an Ketten gelegt, aus Furcht, dass sie beißen möchten, oder eingeschlossen, damit
sie nicht gestohlen werden. Nein! Christus, der Heerführer ihrer Seelen, gibt
ihnen Kraft, dem Bösen zu widerstehen, und das, was in den Augen Gottes recht
und gut ist, zu tun, die Welt in ihrem eitlen Wesen zu verachten, und lieber
ihren Tadel zu ertragen als ihren Beifall zu suchen. \index{Beleidigung} Diese werden auch
angewiesen, nicht nur andere nicht zu beleidigen, sondern selbst diejenigen, von
denen sie beleidigt werden, -- wiewohl nicht ihrer Beleidigung wegen, -- zu
lieben.

\medskip

Was für eine Welt würden wir haben, wenn jeder, aus Furcht zu übertreten, sich
zwischen vier Wänden einkerkern wollte? So soll es nicht sein. Ein vollkommen
christliches Leben verträgt sich mit jedem, unter den Menschen üblichen,
ehrlichen Geschäfte und Gewerbe. Auch ist jene strenge Abgeschiedenheit nicht
die Wirkung des Geistes Christi, sondern eine selbsterwählte fleischliche
Demut, ein bloßer Kappzaum \footnote{\texttt{Kappzaum: ist eine Art Zügel}}, den der Mensch ohne Grund oder Vorschrift \index{Gebote!Vorschrift} sich
selbst macht oder anlegt. In allen solchen Dingen ist er offenbar sein eigener
Gesetzgeber, der sich selbst Regeln, Strafe und Lösepreis vorschreibt. Daher
jene unnatürliche Strenge\index{Gebote!unnatürliche Strenge}, die in keinem Zusammenhang mit den Zwecken der
Schöpfung steht, von welchen das gesellschaftliche Leben eine der vornehmsten
ist. Dieses solle aber nicht aus Furcht vor dem Bösen, das wir in der
Gesellschaft antreffen, zerstört, sondern durch standhafte Bestrafung des
Lasters und durch hervorstechende Beispiele erprobter Tugend, von dem Bösen,
welches dasselbe verdirbt, gereinigt und befreiet werden. \textbf{Wahre
Gottseligkeit
treibt die Menschen nicht aus der Welt, sondern setzt sie in Stand, desto besser
in ihr zu leben, erweckt in ihnen ein Bestreben, sie zu verbessern,}und lehrt
sie, ihr Licht nicht unter einen Scheffel, sondern mit dem Leuchter auf einen
Tisch zu stellen. Überdies ist das Klosterleben bloß eine Erfindung der
Selbstsucht, und daher kann das nicht die rechte Art sein, das Kreuz zu tragen,
welche aus dem entsprungen ist, was durch das wahre Kreuz vernichtet werden muss.
Noch mehr: Diese Grille macht, dass jeder für sich allein davon läuft \index{davon laufen}und die
Welt hinter sich zurücklässt, ohne sich darum zu kümmern, ob sie verloren
geht. \index{Verantwortung übernehmen}Die Christen sollten in den Fahrzeugen der Welt am Ruder stehen, und sie
ihrem Hafen zuführen, aber nicht an den Hinterteilen der Schiffe sich feige
davonschleichen, und die anderen in denselben ohne Steuermänner der Gefahr
überlassen, dass sie durch die Wut der bösen Zeiten auf die Felsen oder
Sandbänke des Verderbens getrieben werden.
\label{ref:05_11_kloster_ende}

\medskip

\index{Müßiggang}\index{Armut}Ergreifen junge Leute das Klosterleben, so haben sie gewöhnlich dabei keinen
andern Zweck, als ihren Hang zum Müßiggang zu bemänteln oder sie haben sich
dazu bereden lassen, damit die Mitgabe ihrer übrigen Geschwister durch ihre
Einkerkerung vermehrt werde. Der Müßiggänger sucht sich dadurch den
schmerzhaften Folgen seiner Untätigkeit, und der unbemittelte Vornehme der
Schande der Armut zu entziehen. Der eine hat zur Arbeit keine Lust, der andere
verachtet sie als unter seiner Würde. Sind es endlich alte Leute, die sich dem
Klosterleben ergeben, so haben sie oft keinen andern Beweggrund dazu, als weil
ihr langes, mit Schuld belastetes Leben bei dem Aberglauben \index{Aberglaube}Zuflucht sucht.
Nachdem sie lange genug in anderen Dingen ihrer Selbstliebe gefolgt sind, wollen
sie nun damit enden, dass sie eine selbsterwählte Religionsform ergreifen, um
Gott zu versöhnen.

\section{13. Abschnitt} \label{kap5_ab13}

Die wahre Aufnahme des Kreuzes Jesu besteht in einer inneren Gemütsübung \index{Gemütsübung}, in
einer genauen Vorsichtigkeit, Wachsamkeit und Inachtnehmung der Seele, dass sie
in Übereinstimmung mit dem ihr geoffenbarten Willen Gottes
\index{Gott!offenbarter Wille} handeln möge. Muss
aber dabei nicht die Seele den Körper, und nicht der Körper die Seele regieren?
Und sollten daher nicht solche Menschen bedenken, dass keine äußere Zelle die
Seele vor bösen Begierden verschließen oder das Gemüt vor einer unendlichen
Menge sündlicher Vorstellungen bewahren kann? Die Gedanken des verderbten
menschlichen Herzens sind böse, und zwar unaufhörlich,\textbf{das Böse kommt von innen,
und nicht von außen} -- denn, wenn es sich auch von außen zeigt, und im Inneren
des Herzens keine Ausnahme findet, so kann es keine böse Wirkung hervorbringen,
-- wie kann denn nun hier die Anwendung eines äußeren Mittels eine innere Ursache
aufheben? Oder wie kann ein bloß körperlicher Zwang eine Einschränkung des
Geistes bewirken? Dieses wird gewiss weit weniger im eingesperrten und
gezwungenen als einem freien Zustand des Menschen der Fall sein, denn wo er am
wenigsten handelt, da hat er die meiste Zeit zu denken, und insofern seine
Gedanken nicht durch einen höheren Einfluss geleitet werden,
\label{ref:05_13_zurueckgezogenheit} \textbf{sind in der Tat
Klöster der Welt schädlicher als Jahrmärkte und Börsen. Dennoch ist
Zurückgezogenheit eine ebenso vortreffliche als notwendige Sache.} \index{Zurückgezogenheit}

\section{14. Abschnitt} \label{kap5_ab14}

Dann aber, o Mensch, untersuche deinen Grund wohl. Prüfe, was es ist, das dich
zu deiner Abgeschiedenheit bewog, und wer dich in dieselbe versetzt hat, damit
es nicht am Ende sich zeigt, wie du deine eigene Seele auf die ganze Ewigkeit
betrogen hast. \index{Missionierung!Grund für}Ich muss gestehen, dass ich mit Eifer für das Heil meiner
Mitmenschen besorgt bin, denn da mir von meinem himmlischen Vater
Barmherzigkeit widerfahren ist, so wünsche ich vornehmlich, dass keiner seine
Seele durch Selbstbetrug in Betreff der Religion verlieren möge, worin die
meisten nur zu geneigt sind, alles ununtersucht für ausgemacht anzunehmen, und
auf diese Weise durch Schmeichelei ihrer Eigenliebe und Vernachlässigung ihres
Heils einen unersetzlichen Verlust erleiden. Die innere bleibende Gerechtigkeit
Christi ist etwas ganz anderes, als jene erzwungene Andacht, worin der arme
abergläubige Mensch seine Gerechtigkeit setzt, und in den Augen Gottes mit
Beifall bestehen, ist weit vortrefflicher, als jene leibliche Übung in der
Religion, die bloß eine menschliche Erfindung ist. Die durch die Kraft des
Geistes Gottes erweckte und bewahrte Seele lebt ihm nach seiner eigenen
Anordnung und betet ihn in seinem Geiste an, nämlich in dem heiligen Gefühl
des Lebens und nach der Leitung seines Geistes, welches die wahre evangelische
Anbetung Gottes ist.

\medskip

\index{Zurückgezogenheit} \textbf{Jedoch wünsche ich auch nicht, so verstanden zu werden, als ob ich wahre
Zurückgezogenheit geringschätzte. Nein, ich billige nicht nur die Einsamkeit,
ich liebe sie auch und schätze sie hoch.} Christus selbst hat sie uns durch
sein Beispiel empfohlen. Er liebte und suchte oft einsame Örter auf Bergen, in
Gärten und an Seegestaden. \textbf{Einsamkeit ist zum Wachstume in der Gottseligkeit
notwendig,} ich ehre die Tugend, die sie sucht und benutzt, und wünsche, dass
mehr davon in der Welt anzutreffen wäre. Nur sollte sie frei gewählt, nicht
erzwungen sein. Denn was für Vorteil kann das Gemüt aus der Einsamkeit ziehen,
wenn sie ihm eher zur Strafe als zum Vergnügen dient. \textbf{Ja, ich habe es
bei
anderen, die das Klosterleben \index{Klosterleben} nicht billigen, schon lange als einen Fehler
angesehen, dass sie keine Zufluchtsorte \index{Zufluchtsort} für Bekümmerte, Versuchte, Verlassene
und eligiös Gesinnte haben, wo diese ungestört auf Gott harren und ihre
religiösen Gemütsübungen durchgehen könnten, um hernach, gestärkt und mit mehr
Kraft, sich selbst zu beherrschen, versehen, wieder in die Geschäfte des Lebens
treten zu können, wiewohl allerdings je weniger, desto besser. Denn in freier
Einsamkeit wird göttliches Vergnügen gefunden.}
\label{ref:05_13_zurueckgezogenheit_ende}