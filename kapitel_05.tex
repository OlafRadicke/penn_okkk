
\chapter{5. Kapitel}
\section{Zusammenfassung des 5. Kapitel}
\small
\begin{description}
\item[1. Abschnitt] Von der unerlaubten Eigenliebe; sie ist zweifach: 1. in der Religion  2. in der Moralität.
\item[2. Abschnitt] Von Denen, welche die mehrste Formalität, den stärksten Aberglauben und die größte Prachtliebe in ihrem Gottesdienste eingeführt haben.
\item[3. Abschnitt] Gott tadelt dergleichen fleischliche Begriffe.
\item[4. Abschnitt] Christus zog seine Jünger von der äußern jüdischen Gottesverehrung ab, und stiftete eine geistigere.
\item[5. Abschnitt] Stephanus redet über diesen Gegenstand ganz deutlich und ausführlich.
\item[6. Abschnitt] Paulus wendet den Ausdruck: Tempel, zweimal auf das Herz des Menschen an.
\item[7. Abschnitt] Von dem Kreuze jener weltlichen Gottesverehrer.
\item[8. Abschnitt] Da ihr Kreuz von Fleisch und Blut (von fleischlich gesinnten Menschen) verfertigt ist; so kann dasselbe Fleisch und Blut nicht kreuzigen,
\item[9. Abschnitt] Solche Kreuze sind Joche. die keinen Zwang auflegen.
\item[10. Abschnitt] Von der Pracht dieser Kreuze, und von der Achtung. die man ihnen erweiset.
\item[11. Abschnitt] Ein Einsiedlerleben ist keine wahre evangelische Selbstverleugnung.
\item[12. Abschnitt] Vergleichung zwischen der Selbstverleugnung Christi, und derjenigen: der falschen Christen; die seinige führt zu einem reinen Leben in der Welt; die ihrige zu einer selbsterwählten Einkerkerung, damit sie nicht durch die Welt in Versuchung gerathen. – Von dem Unheile, welches in der Welt entstehen würde, wenn solche Beispiele allgemeine Nachahmung fänden; es würde nützliche Gesellschaften und ehrliche Betriebsamkeit zerstören. – Ein solches unthätiges Leben ist die gewöhnliche Zuflucht der Müßiggänger, der vornehmen Armen, und des mit Schuld beladenen Alters.
\item[13. Abschnitt] Von den Wirrungen des Kreuzes Christi in diesen Fällen, und von der Unmöglichkeit, daß hier die Anwendung eines äußern Mittels einen innern Schaden heilen könne.
\item[14. Abschnitt] Eine Ermahnung an die Leute dieses Glaubens, daß sie sich nicht selbst betrügen mögen.

\end{description}
\normalsize

\section{1. Abschnitt}

Ich komme nun zur Betrachtung der unerlaubten Eigenliebe, deren Befriedigung, mehr oder weniger, die wichtigsten Angelegenheit des größten Theiles der Menschheit ist. Diese unerlaubte Eigenliebe erstreckt sich auf zwei Gegenstände, erstlich: auf die religiöse Verehrung und Anbetung Gottes; zweitens: auf das moralische und bürgerliche Verhalten der Menschen überhaupt; und in beiden Hinsichten ist eine Untersuchung derselben von der größten Bedeutung für uns. Ich werde mich jedoch hierin so kurz fassen, als mein Gewissen es mir erlaubt, und die Wichtigkeit der Sache es zuläßt.

\section{2. Abschnitt}

Die unerlaubte Eigenliebe in der Religion, welche durch das Kreuz Christi ertödtet werden muß, bestehet in jener menschlichen Erfindung und Verrichtung einer Gottesverehrung, die für göttlich gehalten wird, es aber sowohl in ihrer Stiftung als in ihrer Ausübung nicht ist. In diesem großen Irrthume stehen unter Allen, die sich den Namen der Christen beilegen, diejenigen oben an, welche den mehrsten äußern Schein, die größte Pracht, und den stärksten Aberglauben mit ihrem Gottesdienste verbinden. Denn sie fehlen nicht allein in der Art der Vollziehung ihres Gottesdienstes darin, daß sie dem allmächtigen Gott, der ein ewiger Geist ist, ohne geistliche Vorbereitung des Herzens Verehrung und Anbetung leisten; sondern ihre Gottesverehrung selbst ist aus Sachen zusammengesetzt, die mit der Vorschrift und Ausübung der Lehre Christi und dem apostolischen Beispiele gänzlich unvereinbar sind. Denn, statt daß jene wahre Gottesverehrung einfach und geistlich war, ist die ihrige prachtvoll und weltlich; statt daß die Anbetung Gottes, die Christus lehrete, ganz innerlich war, und mit dem Gemüthe verrichtet wurde, ist die ihrige äußerlich, und wird nur körperlich. vollzogen; statt daß jene der Natur und dem Wesen Gottes, als eines unsichtbaren Geistes, angemessen war, ist diese nur zur Unterhaltung der Sinnlichkeit des Menschen eingerichtet; so daß wir also da, wo Fleisch und Blut, oder die fleischlichen Begriffe und Neigungen des Menschen ganz ausgeschlossen seyn sollten, eine Gottesverehrung erblicken, die völlig darauf berechnet ist, denselben zu behagen; als wenn es das Geschäft der Gottesverehrer nicht wäre, Gott eine ihm wohlgefällige Vdrehrung und Anbetung zu leisten, sondern vielmehr darin bestände, daß sie eine Art des Gottesdienstes hervorbringen, die ihnen selbst am besten gefällt. Daraus entstand jene Gottesverehrung, die mit prachtvollen Gebäuden und Bildern, reichen Verzierungen und Gewanden, seltenen Stimmen und auserlesener Musik, mit kostbaren Lampen, Wachslichtern und Räucherwerk, und dieß Alles mit einer für die äußern Sinne so einnehmenden Abwechselung, als die Kunst nur erfinden und Kostenaufwand es erschwingen konnte, aufs glänzendste geschmücket wurde. Als ob die Menschen wieder Juden oder Egypter werden sollten; oder als wenn Gott wirklich ein sehr alter Mann wäre, und Christus ein kleiner Knabe, dem man mit einer Art religiösen Spielwerks eine Freude machen könnte; denn so bilden sie ihn in ihren Tempeln und nur zu häufig auch in ihren Gemüthern ab. Man muß freilich gestehen, daß eine solche Gottesverehrung solchen Begriffen von Gott völlig entspricht; denn, wenn Menschen sich ihn so, wie sie selbst sind, vorstellen können, so ist es auch nicht zu  bewundern, daß sie sich auf eine Art an ihn wenden und ihn so zu unterhalten suchen, wie es ihnen selbst von Andern am besten gefällt.

\section{3. Abschnitt}

Was sagte aber einst der Allmächtige zu einem so sinnlichen Volke bei einer ähnlichen Veranlassung? "`Du meinest, ich werde seyn wie du; aber ich will dich strafen, und will dir’s unter Augen stellen. Merket doch das, die ihr Gottes vergesset, damit ich nicht einmal hinreisse, und sey kein Retter mehr da. Wer Dank opfert, der preiset mich, und das ist der Weg, daß ich ihm zeige das Heil Gottes."' (Nach der englischen Bibel: "`Und wer seinen Wandel richtig ordnet, dem will ich das Heil Gottes zeigen."')\footnote{Ps. 50, 21.22. 23,} Die Verehrung, die Gott wirklich wohlgefällt, ist die, "`daß man Gerechtigkeit übe, Barmherzigkeit liebe und demüthig vor ihm wandle."' \footnote{Micha 6, 8.} Denn er, "`der  Herzen und Nieren prüft, der dem Menschen seine Sünden vor Augen stellt, der allein der Gott der Geister alles Fleisches ist,"' er siehet nicht auf das äußere Machwerk des Menschen, sondern auf die innere Stimmung und Neigung seines Herzens. Es ist auch für Verständige nicht denkbar, daß Er, von dem es heißt: "`Du bist sehr herrlich, du bist schön und prächtig geschmückt; Licht ist dein Kleid, das du anhast; du breitest den Himmel aus wie einen Teppich; du wölbest ihn oben mit Wasser; du fährst auf den Wollen wie auf einem.Wagen, und gehest auf den Fittigen des Windes; der du deine Engel zu Winden (Geistern), und deine Diener zu Feuerflammen machst; der du das Erdreich gründetest aus seinen Boden, daß es immer und ewig bleibe;"'\footnote{Ps. 104,1-6} – ich sage, es ist nicht denkbar, daß dieses Wesen durch. solche menschliche Erfindungen verehret werden könne, zu welchen ein von der ursprünglichen Kraft der Religion und von der geistlichen Beschaffenheit der christlichen Gottesverehrung abgewichenes Volk seine Zuflucht genommen hat.

\section{4. Abschnitt}

Christus zog seine Jüngers von jener prachtvollen und zeremoniellen Gottesverehrung im äußern Tempel gänzlich ab, und stiftete eine innere und geistliche Anbetung, worin  er sie unterrichtete. "`Ihr sollt weder auf diesem Berge, noch zu Jerusalem,"' sagte er zu dem samaritischen Weibe, "`den Vater anbeten; Gott ist ein· Geist, und diejenigen, welche ihn anbeten, müssen ihn im Geiste und in der Wahrheit anbeten."'\footnote{Joh. 4,21. 24.}  Als hätte er gesagt: Gott, hat zwar vorzeiten um der Schwachheit des Volke willen sich herabgelassen, die Verehrung, die ihm geleistet werden sollte, auf Zeit und Ort, auf einen äußern Tempel und äußere zeremonielle Vorrichtungen zu beschränken; allein dieses geschah zur Zeit der Unwissenheit der Menschen, als sie seine Allgegenwart noch nicht erkannten, und nicht einsahen, was Gott ist, und wo er ist. Jetzt bin ich aber gekommen, ihn Allen, die mich aufnehmen, zu offenbaren. Und nun sage ich euch, daß Gott ein Geist ist, und daß er im Geiste und in der Wahrheit will angebetet seyn. Die Menschen müssen ihn als einen Geist kennen lernen, ihn als Solchen betrachten, verehren und anbeten. Es ist nicht diese leibliche Anbetung, auch sind es nicht diese seht unter euch üblichen Gebräuche und zeremoniellen Beobachtungen, womit ihm gedient ist, oder welche euch diesem Gott, der ein Geist ist, angenehm machen können. Nein! ihr müßt seinem Geiste Gehorsam leisten, der innerlich mit euch ringet, um euch von den bösen Dingen in der Welt. abzuziehen, damit ihr, indem ihr aus die Belehrungen dieses Geistes in euern Herzen aufmerksam seid, und euch seinen Geboten unterwerfet, einsehen und verstehen lernet, was es heißt, Gott als einen Geist anzubeten. Dann werdet ihr begreifen, wie die wahre Gottesverehrung nicht darin bestehet, daß man auf diesen Berg oder nach Jerusalem gehe, sondern darin: daß man den Willen Gottes thue, seine Gebote halte, mit seinem eigenen Herzen sich unterrede und nicht mehr sündige; daß man sein Kreuz aufnehme, das heilige Gesetz Gottes in seinem Herzen betrachte, und dem Beispiele Desjenigen folge, den der Vater gesandt hat.

\section{5. Abschnitt}

Da nun hernach Stephanus, jener kühne und standhafte Märtyrer Jesu, als Gefangener vor die Rathsversammlung der Juden geführt, und wegen seiner Behauptung, daß es mit ihrem geliebten Tempel und mit den Bedienungen desselben ein Ende nehmen sollte, fälschlich der Gotteslästerung. beschuldigt wurde, gab er ihnen diese Erklärung: "`Salome,"' sagte er, "`erbauete Gott ein Harfe. Aber der Allmachtige wohnet nicht in "`Tempeln, die mit Händen gemacht sind, wie der Prophet spricht: Der Himmel ist mein Stuhl, und die Erde meiner Füße Schemel, was für ein Haus wollt ihr mir denn bauen? spricht der Herr, oder welches ist die Stätte meiner Ruhe? Hat nicht meine Hand das Alles gemacht?"'\footnote{Ap. Gesch. 7.} Hier sehen wir klar die gänzliche Ueberflüssigkeit der weltlichen Tempel und aller ihnen anklebenden Zeremonien. -- Aber der Märtyrer Jesu dringt noch schärfer auf jene abgewichenen Juden ein, welche in jenen Zeiten die prachitliebenden, zeremoniellen und weltlichgesinnten Gottesverehrer waren: "`Ihr Halsstarrigen und Unbeschnittenen an Herzen und Ohren!"' sagt er, "`ihr widerstrebet allezeit dem heiligen Geiste; wie eure Väter thaten, so thut auch ihr."', Das ist so viel, als wenn er gesagt hätte: Eure äußern Tempel, Beobachtungen und Schattendienste, eure Ansprüche auf eure natürliche Abkunft von Abraham  und aus eure Nachfolge in der Religion Mosis, helfen euch nichts; ihr widerstrebet dem heiligen Geiste. und widersprechet seinen Belehrungen ihr wollt euch seinem Rathe nicht unterwerfen, und eure Herzen sind nicht ungetheilt auf Gott gerichtet. Ihr seid Nachfolger der Ungerechtigkeit eurer Väter, und wenn ihr auch mit dem Munde die Propheten bewundert und lobet, so seid ihr dennoch nicht ihre Nachfolger im Glauben und Leben.

Der Prophet Jesaias geht aber noch etwas weiter, als was Stephanus von ihm anführt; denn nachdem et gezeigt hat, was Gottes Haus, nämlich der Ort seiner Ruhe, nicht sei, fährt er mit diesen Worten fort: "`Ich sehe aber an den Elenden, der eines zerbrochenen Geistes "`ist, und der sich fürchtet vor meinem Worte."'\footnote{Jes. 66,2} Siehe hier, o sinnlicher, abergläubiger Mensch! den wahren Anbeter! Siehe hier den Ort der Ruhe Gottes-! das Hand und den Tempel Desjenigen, den- der Himmel aller Himmel nicht fassen kann! Ein Hand, welches die Eigenliebe nicht zu erbauen versieht, und das weder die Kunst noch die Macht der Menschen zu bereiten oder einzuweihen vermag.

\section{6. Abschnitt}

Paulus, der große Apostel der Heiden, wendet zweimal das Wort: Tempel, ausdrücklichl auf den Menschen an; einmal in seiner Epistel an die Gemeine zu Korinth, wo er sagt: "`Wisset ihr nicht, daß euer Leib ein Tempel des heiligen Geistes ist, der in euch ist, welchen ihr von Gott habt?"' u. s. w. ,\footnote{1 Kor. 6,19} also kein Gebäude der menschlichen Hand und Kunst. Dann sagt er wieder zu denselben Leuten in seiner zweiten Epistel: "`Ihr seyd der Tempel des lebendigen Gottes wie Gott spricht:"' (hier führt er die durch den Propheten geredeten Worte Gottes an,) "`Ich will in ihnen wohnen, und in ihnen wandeln, und will ihr Gott seyn, und sie sollen mein Volk seyn."'  Dieses ist der evangelische Tempel, die christliche Kirche, deren Schmuck nicht in Stickereien und Verzierungen weltlicher Kunst und irdischen Reichthums, sondern in Gaben des Geistes; nämlich in Sanftmuth, Liebes, Glauben, Geduld, Selbstverleugnung und Wohlthatigkeit bestehet. Hier ist der Ort, den die ewige Weisheit,  die von Ewigkeit her, ehe die Berge eingesenkt und die Hügel gebildet waren, bei Gott war, zu ihrer Wohnung erkor, und wo sie wie die Weisheit es selbst ausdrückt, – auf dem Erdboden spielet, und ihre Lust. bei den Menschenkindern hat;"'\footnote{Sprichw. Sal. 8,22. 23. 25. 31.} aber nicht in Häusern von Holz und Steinen. Dieser lebendige Tempel ist herrlicher als Salomo's lebloses Gebäude, welches denselben auch nur vorbildlich darstellte, so wie Salomo selbst, als der Erbauer jenes Hauses, ein Vorbild von Christo war, der uns zu geistlichen Tempeln Gottes erbauet. Es war in vorigen Zeiten verheißen, "`daß die Herrlichkeit des leztern Hauses größer als die des erstern werden sollte."'\footnote{Hagai 2,10.} Dieses läßt sich sehr wohl hieraus deuten, und ist nicht so zu verstehen, daß ein äußeres Tempelgebäude ein anderes an äußerm Glanze übertreffen solle; denn wozu würde das nützen? Nein! die göttliche Herrlichkeit, die Schönheit der Heiligkeit in dem evangelischen Gebäude, oder in der christlichen Kirche, die aus wiedergebornen Gläubigen bestehet, sollte jene äußere Herrlichkeit des Tempels Salomo’s weit übertreffen, welche in Vergleichung mit der Klarheit der lettztern Tage, sich wie Fleisch zu Geist, wie ein schwinden- der Schatten zu dem unvergänglichen Wesen verhalten würde.

Dessenungeachtet haben. die Christen ihre Versammlungsörter; die aber nicht auf eine jüdische oder heidnische Art prachtvoll ausgerüstet, sondern schmucklos und ohne Pump sind; so daß sie mit der Einfachheit ihres gottseligen Lebens und mit ihrer heiligen Lehre übereinstimmen. Denn Gottes Gegenwart theilt sich nicht dem Hause, sondern den in demselben Versammelten mit, und aus diesen, nicht aber aus dem Hause, bestehet die evangelische Kirche. O! möchten doch Alle, die sich Christen nennen, statt jener eingebildeten Heiligkeit, die Häusern und Oertern zugeschrieben wird, die wahre Heiligkeit durch dar: Bad oder Abwaschen der wiedergebärenden Kraft und Gnade Gottes in ihren eigenen Herzen aus Erfahrung kennen lernen, so würden sie wissen, was die Kirche ist, und wo man in diesen Tagen der evangelischen Einrichtung den Ort findet, da Gott erscheint. Dieses veranlasste den königlichen Propheten David, zu sagen: "`Des Könige Tochter ist inwendig ganz herrlich; sie ist mit goldenen Stücken bekleidet."'\footnote{Ps. 45,14.} Worin bestehet diese Herrlichkeit im Innern der wahren Kirche? oder was ist das Gold, das diese Herrlichkeit ausmacht? Sage mir, o abergläubiger Mensch! bestehet es in deinen prächtigen Tempeln, Altären, Tischen, Teppichen, Tapeten? – In deinen Gewänden, Orgeln, Stimmen, Lichtern, Lampen, Rauchpfannen? – In deinen Silbergeräthen, Juwelen und dergleichen Verzierungen deiner weltlichen Tempel? Keinesweges. Alle diese Dinge haben gar keine Aehnlichkeit mit dem göttlichen Schmucke der Tochter des Königs der Himmel, der gebenedeieten und erlöseten Gemeine Christi. O! des bejammernswerthen Abfalles! Welch ein elender Ersatz für den Verlust und die Abwesenheit des, apostolischen Lebens, der geistlichen Herrlichkeit der ''ersten Kirche''.

\section{7. Abschnitt}

Dennoch wollen einige dieser Bewunderer der äußern Pracht und Herrlichkeit bei ihrem Gottesdienste für Verehrer des Kreuzes Christi gehalten seyn, und haben   daher eine ganze Menge Kreuze gemacht. Aber ach! wie können sie hoffen, etwas mit dem Christenthume zu vereinigen, daß, jemehr es den Schein desselben annimmt, nur desto weiter von seinem Fliesen entfernt ist? Denn gerade ihr Kreuz und ihre Selbstverleugnung gehen aus den Wirkungen einer höchst unerlaubten Eigenliebe hervor-, und während sie sich einbilden, dadurch Gott zu dienen, verirren sie sich aus eine gefahrvolle Weise von dem wahren Kreuze Christi und von jener heiligen Verleugnung, die Er vorgeschrieben hat. Es ist wahr, sie haben wirklich ein Kreuz; allein es scheint nur ein Stellvertreter des wahren Kreuzes zu seyn; und es ist so bescheiden und nachgiebig, daß es Alles zugiebt, was seine Träger nur wollen. Denn anstatt ihren Willen dadurch zu ertödten, richten sie es nach ihrem Willen ein, und gebrauchen es auch nur ihrem Belieben gemäß! Auf diese Art ist das Kreuz das Zeichen Derjenigen geworden, die Alles thun,  was sie gelüstet. Dennoch wollen sie, dieses Kreuzes wegen, für Jünger Desjenigen gelten, der niemals seinen eigenen Willen, sondern allezeit den Willen seines himmlischen Vaters.3 that.

\section{8. Abschnitt}

Da die fleischlichen Begriffe der Menschen dieses Kreuz erdacht haben, so ist es so beschaffen, daß Fleischlichgesinnte dasselbe gar wohl tragen können. Eben darum ist es aber auch nicht das Kreuz Christi, (nicht die göttliche Kraft,) wodurch Fleisch und Blut gekreuzigt werden. Tausende solcher Kreuze haben nicht mehr Kraft und Wirkung, als das kleinste Stückchen Holz; es sind. armselige, leere Schatten, die nicht einmal wirkliche Bilder des wahren Kreuzes sind. Einige tragen sie als Zaubermittel bei sich, können aber niemals ein einziges Uebel damit von sich abwehren. Sie sündigen mit dem Kreuze auf ihrem Rücken; und wenn sie es auch an ihrem Busen tragen, so ruhen dennoch in demselben ihre Lieblingslüste ganz ungestört. Wie die stummen Gehen, welche Elia verspottete, sind auch sie ohne Kraft und Leben. Und wie könnten sie auch andere seyn; da sie aus irdischer Masse bestehen, und ihre Gestalt und Verfertigung der Erfindung und Arbeit weltlicher Künstler verdanken? Ist es also wohl möglich, daß solche Kreuze Diejenigen, die sie tragen, zu bessern Menschen machen können? Nein! wahrlich nicht.

\section{9. Abschnitt}

Es sind Joche, die den- verderbten Neigungen der Menschen keinen Zwang anlegen; Kreuze, die ihnen nie widersprechen. Eine ganze Ladung solcher Kreuze würde den Menschen immer in demselben unverbesserten Zustande lassen, worin sie ihn findet; denn sie können die sündlichen Neigungen aus seinem Herzen nicht vertreiben. Hiervon, fürchte ich, sind leider nur zu Viele, die sich dee falschen Kreuzes bedienen, ja dasselbe sogar anbeten, in ihrem Innern überzeugt, wiewohl sie sich den noch viel darauf einbilden daß sie es tragen. Dieses kann aber nur der Fall mit den falschen; nie mit dem wahren Kreuze seyn; da dieser bei Denen, die es aufrichtig tragen, gar keinen Stolz duldet.

\section{10. Abschnitt}

So wie aber die Religion solcher Menschen beschaffen ist, so hat auch ihr Kreuz ein sehr prächtiges, triumphirendes, hinreißendes Ansehen. Aber worin bestehet dieses? In kostbaren Metallen und Edelsteinen; – eine Beute, die der Aberglaube ihren Geldsäcken raubt. Statt daß diese Kreuze die Herzen Derer, die sie tragen , belehren sollten, die irdischen Schätze zu verleugnen, sind sie vielmehr aus Kostbarkeiten zusammengesetzt, und werden auch, wie ihre Besitzer selbst, nach ihrer Pracht geschätzte Ein reich besetztes Kreuz findet viele Angaffer und Bewunderer, ein ungeschmücktes hingegen wird, wie dieses auch bei andern Dingen der Fall ist, nur wenig geachtet. Ich könnte mich in Ansehung dieses thörichten und eitlen Aberglaubens auf die eigene Ueberzeugung dieser Kreuzträger berufen; und o! wie sehr verschieden ist das ihrige von dem heiligen Kreuze Jesu, welches. die Sünde im Menschen zerstöret!

\section{11. Abschnitt}

Auch ist ein eingesperrtes Klosterleben; worin so Viele eine prahlerische Gerechtigkeit suchen, eben so wenig zu empfehlen, oder nur im mindesten mit der Natur und Eigenschaft des wahren Kreuzes vereinbar. Denn, wenn dasselbe auch nicht, wie andere Dinge, unerlaubt wäre, so ist es doch unnatürlich; und wahre Religion leitet nicht dazu an. Die Zelle des wahren.Christen ist in seinem Innern, wo die Seele, von der Sünde geschieden, eingeklöstert ist. Diese geistliche Zelle tragen die wahren Nachfolger Christi beständig mit sich, indem sie sich nicht dem Umgange mit der Welt entziehen, sondern in ihrem Umgange mit derselben, vor dem Bösen, das sie hat, in Acht nehmen. Klöster sind Wohnsitze einer tragen, schmutzigen, unnützen Selbstverleugnung, wodurch die ,Menschen Andern lästig fallen, um ihren Müßiggang zu nähren; eine Art religiöser Narrenhäuser, in welche man die Verrückten einsperrt, damit sie draußen kein Unheil anrichten; das mag wohl heißen: <tt>Patience par forçe</tt>, (Geduld aus Zwang); eine- Selbstverleugnung wider  Willen, welche die Menschen eher dumm als tugendhaft macht, und sie lieber vor der Versuchung verschließt, als sie zur Standhaftigkeit in derselben anleitet. Als wenn es ein Verdienst wäre, daß nicht zu thun, wozu man nicht versucht wird. Wade aber das Auge nicht betrachtet, wird das Herz nicht begehren und auch nicht zu bereuen haben.

\section{12. Abschnitt}

Das Kreuz Christi bringt ganz andere Wirkungen hervor. Es macht den Menschen fähig, die Welt wirklich zu überwinden und, selbst im Angesichte ihrer Lockungen zum Bösen, ein reines Leben zu führen. Diejenigen, welche es tragen, sind nicht an Ketten gelegt, aus Furcht, daß sie beißen möchten; oder eingeschlossen, damit sie nicht gestohlen werden; nein! Christus, der Heerführer ihrer Seelen, giebt ihnen Kraft, dem Bösen zu widerstehen, und das, was in den Augen Gottes recht und gut ist, zu thun; die Welt in ihrem eitlen Wesen zu verachten, und lieber ihren Tadel zu ertragen, als ihren Beifall zu suchen. Diese werden auch angewiesen, nicht nur Andere nicht zu beleidigen, sondern selbst Diejenigen, von denen sie beleidigt werden, – wiewohl nicht ihrer Beleidigung wegen, – zu lieben.

Was für eine Welt würden wir haben, wenn Jeder, aus Furcht zu übertreten, sich zwischen vier Wänden einkerkern wollte? So soll es nicht seyn. Ein vollkommen  christliches Leben verträgt sich mit jedem, unter den Menschen üblichen, ehrlichen Geschäfte und Gewerbe. Auch ist jene strenge Abgeschiedenheit nicht die Wirkung des Geistes Christi, sondern eine selbsterwählte fleischliche Demuth; ein bloßer Kappzaum, den der Mensch ohne Grund oder Vorschrift sich selbst macht oder anlegt. In allen solchen Dingen ist er offenbar sein eigener Gesetzgeber, der sich selbst Regeln, Strafe und Lösepreis vorschreibt. Daher jene unnatürliche Strenge, die in keinem Zusammenhänge mit den Zwecken der Schöpfung stehet, von Welchen das gesellschaftliche Leben eine der vornehmsten ist. Dieses solle aber nicht aus Furcht vor dem Bösen, das wir in der Gesellschaft antreffen, zerstöret, sondern, durch standhafte Bestrafung des Lasters und durch hervorstechende Beispiele erprobter Tugend, von dem Bösen, welches dasselbe verdirbt, gereinigt und befreiet werden. Wahre Gottseligkeit treibt die Menschen nicht aus der Welt, sondern setzt sie in Stand, desto besser in ihr zu leben, erwecket in ihnen ein Bestreben, sie zu verbessern, und lehret sie, ihr Licht nicht unter einen Scheffel, sondern mit dem Leuchter auf einen Tisch zustellen. Ueberdieß ist das Klosterleben bloß eine Erfindung der Selbstsucht, und daher kann das nicht die rechte Art, das Kreuz zu tragen, seyn, welche aus dem entsprungen ist, was durch das wahre Kreuz vernichtet werden muß. Noch mehr: Diese Grille macht, daß Jeder für sich allein davon läuft und die Welt hinter sich zurückläßt, ohne sich darum zu bekümmern, ob sie verloren gehet. Die Christen sollten in den Fahrzeugen der Welt am Ruder stehen, und sie ihrem Hafen zuführen., aber nicht an den Hintertheilen der Schiffe sich feige davon schleichen, und die Andern in denselben ohne Steuermänner der Gefahr überlassen, daß sie durch die Wuth der bösen Zeiten auf die Felsen oder Sandbänke des Verderbens getrieben werden.

Ergreifen junge Leute das Klosterleben, so haben sie gewöhnlich dabei keinen andern Zweck, als ihren Hang zum Müßiggange zu bemänteln oder sie haben sich dazu bereden lassen, damit die Mitgabe ihrer übrigen Geschwister durch ihre Einkerkerung vermehret werde. Der Müßiggänger sucht sich dadurch den schmerzhaften Folgen seiner Unthatigkeit, und der unbemittelte Vornehme der Schande der Armuth zu entziehen. Der Eines hat zur Arbeit keine Lust, der Andere verachtet sie als unter seiner Würde. Sind es endlich alte Leute, die sich dem Klosterleben ergeben, so haben sie oft keinen andern Beweggrund dazu, als weil ihr langes, mit Schuld belastetes Leben bei dem Aberglauben Zuflucht sucht. Nachdem sie lange genug in andern Dingen ihrer Selbstliebe gefolgt sind, wollen sie nun damit enden, daß sie eine selbsterwählte Religionsform ergreifen, um Gott zu versöhnen.

\section{13. Abschnitt}

Die wahre Aufnahme des Kreuzes Jesu bestehet in einer innern Gemüthsübung, in einer genauen Vorsichtigkeit, Wachsamkeit und Inachtnehmung der Seele, daß sie in Uebereinstimmung mit dem ihr geoffenbarten Willen Gottes handeln möge. Muß aber dabei nicht die Seele den Körper, und nicht der Körper die Seele regieren? Und sollten daher nicht solche Menschen bedenken, daß keine äußere Zelle die Seele vor bösen Begierden verschließen oder das Gemüth vor einer unendlichen Menge sündlicher Vorstellungen bewahren kann? Die Gedanken des verderbten menschlichen Herzens sind böse, und zwar unaufhörlich; das Böse kommt von innen, und nicht von außen; – denn, wenn es sich auch von außen zeigt, und im Innern des Herzens keine Ausnahme findet, so kann es keine. böse Wirkung hervorbringen; – wie kann denn nun hier die Anwendung eines äußern Mittels eine innere Ursache aufheben? oder wie kann ein bloß körperlicher Zwang eine Einschränkung des Geistes bewirken? Dieses wird gewiß weit weniger im eingesperrten und gezwungenen als einem freien Zustande des Menschen der Fall seyn; denn wo er am wenigsten handelt, da hat er die mehrste Zeit zu denken, und insofern seine Gedanken nicht durch einen höhern Einfluß geleitet werden, sind in der That Klöster der Welt schädlicher als Jahrmärkte und Börsen. Dennoch ist Zurückgezogenheit eine eben so vortreffliche als nothwendige Sache.

\section{14. Abschnitt}

Dann aber, o Mensch! untersuche deinen Grund wohl. Prüfe, was es ist, das dich zu deiner Abgeschiedenheit bewog, und wer dich in dieselbe versetzt hat; damit es nicht am Ende sich zeige, wie du deine eigene Seele auf die ganze Ewigkeit betrogen hast. Ich muß gestehen, daß ich mit Eifer für das Heil meiner Mitmenschen besorgt bin., Denn da mir von meinem himmlischen Vater Barmherzigkeit widerfahren ist,. so wünsche ich vornehmlich, daß Keiner seine Seele durch Selbstbetrug in Betreff der Religion verlieren möge, worin die Mehrsten nur zu geneigt sind, Alles ununtersucht für ausgemacht anzunehmen, und aus diese Weise durch Schmeichelei ihrer Eigenliebe und Vernachlässigung ihres Heils einen unersetzlichen Verlust erleiden. Die innere bleibende Gerechtigkeit Christi ist etwas ganz Anderes, als jene erzwungene Andacht, worin der arme abergläubige Mensch seine Gerechtigkeit setzt; und in den Augen, Gottes mit Beifall bestehen, ist weit vortrefflicher, als jene leibliche Uebung in der Religion, die bloß einer menschliche Erfindung ist. Die durch die Kraft des Geistes Gottes erweckte und bewahrte Seele lebt ihm nach seiner eigenen Anordnung, und betet ihn in seinem Geiste an; nämlich, in dem heiligen Gefühle des Lebens und nach der Leitung seines Geistes, welches die wahre evangelische Anbetung Gottes ist.

Jedoch wünsche ich auch nicht, so verstanden zu werden, als ob ich wahre Zurückgezogenheit geringschätzte. Nein, sich billige nicht nur die Einsamkeit; ich liebe sie auch und schütze sie hoch. ''Christus'' selbst hat sie uns durch sein Beispiel empfohlen. Er liebte und suchte oft einsame Oerter auf Bergen, in Gärten und an Seegestaden. Einsamkeit ist zum Wachsthume in der Gottseligkeit nothwendig; ich ehre die Tugend, die sie sucht und benützt, und wünsche, daß mehr davon in der Welt anzutreffen wäre. Nur sollte sie frei gewählt, nicht erzwungen seyn. Denn wad für Vortheil kann das Gemüth aus der Einsamkeit ziehen, wenn sie ihm eher zur Strafe als zum Vergnügen dienet. Ja, ich habe es bei Andern, die daß Klosterleben nicht billigen, schon lange als einen Fehler angesehen, daß sie keine Zufluchtsörter für Bekümmerte, Versuchte, Verlassene und Religiösgesinnte haben, wo diese ungestört auf Gott harren und ihre religiösen Gemüthsübungen durchgehen könnten, um hernach, gestärkt, und mit mehr Kraft, sich selbst zu beherrschen, versehen, wieder in die Geschäfte des Lebens treten zu können; wiewohl allerdings je weniger, die besser. Denn in ''freier Einsamkeit'' wird göttliches 



