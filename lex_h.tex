\section*{H}

\articlesize

\begin{description}

 \item[Hanson, Elizabeth]

 \item[Heirat]

 \item[Helmont, Franciscus Mercurius van] $\ast$um~1614 in Vilvoorde bei Brüssel, Spanische Niederlande, heute Belgien, \dag1699 in Cölln bei Berlin. Quäker, Alchemist, Kabbalist, Theosoph, Arzt, Rosenkreuzer.\endnote{"`Helmont, Franciscus Mercurius van,"', Biographisch-Bibliographischen Kirchenlexikon, Band XXV (2005) Spalten 586-597 Autor: Claus Bernet, http://www.bautz.de/bbkl/h/helmont\_f\_m.shtml}

 \item[Hicks, Elias] $\ast$19.3.1748 \dag27.2.1830. Quäker, Wanderprediger, Abolitionisten. An seiner Person entzündete sich ein heftiger Streit, der dann in den  $\to$\textit{"`Hicksites-Orthodox-Schisma"'} der Quäker mündete. Daraus ging dann die $\to$Hicksites und die $\to$Orthodoxe Linie hervor.

 \item[Hodgkin, Thomas]

 \item[Hoover, Herbert C.]

 \item[Hopkins, Johns]

 \item[Hutchinson, Anne] $\ast$Juli 1591, \dag~August 1643. War keine Quakerin. Wurde als Anne Marbury in England geboren; heiratete mit 21 Jahren William Hutchinson und siedelte mit ihm 1634 in die britische Kolonie Massachusetts in Nordamerika um. Sie trat hervor als selbstsichere Frau, was für ihre Zeit ungewöhnlich war. Sie Vorkämpferin des Feminismus und eine Kämpferin für Menschenrechte und Toleranz. Sie lehrte, dass Gott direkt zu jedermann spräche und nicht allein zum Klerus. Sie Beeinflusste die spätere Quäkerin Mary Dyer, die sich beide persönlich kannten.


 \end{description}

\normalsize
