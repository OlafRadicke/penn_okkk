\section*{H}

\articlesize

\begin{description}

 \item[Hanson, Elizabeth] $\ast$1684; \dag1737, neuenglische Quäkerin. Bekannt
 durch ihren Erlebnisbericht über ihre Zeit in indianischer Gefangenschaft.
 Gefangen genommen wurde sie im Sommer 1724. Der Bericht erschien 1728 und
 wurde im 18. Jahrhundert in Amerika und in England mehrfach nachgedruckt.


 \item[Heirat] Die Quakerhochzeit findet - wie fast alles - formlos statt. Zur
 vermählung wird kein Geistlicher, Pfarrer oder Prediger bestellt. Viel mehr
 geben sich die künftigen Eheleute gegenseitig die treue Beteuerung. Es wird
 darauf verwiesen, das auch in der Bibel keine Heiratszerominie beschrieben
 wird. Der Rest wird frei gestaltet. Meist kombiniert mit einer
 $\to$\textit{Andacht}. Statt den üblichen Trauzeugen, gibt es ein schriftliches
 Trauzeugnis-/urkund, die von allen anwesenden unterzeichnet werden können.
 Dabei spielt die konfessionelle ZTugehörigkeit keine Rolle. Es bereits aus dem
 17.~Jahrhundert Fälle dokumentiert, wo Vermeählungsurkunden der Quaker von
 Mennoniten unterschrieben wurden.\endnote{Sünne Juterczenka, "`Über Gott und die Welt - Entzeitvisionen, Reformdebatten, und die europäische Quäkermission in der frühen Neuzeit"', Vandenhoeck\&Ruprecht, 2008, ISBN 978-3-525-35458-2, Seite~209}

 \item[Helmont, Franciscus Mercurius van] $\ast$um~1614 in Vilvoorde bei Brüssel, Spanische Niederlande, heute Belgien, \dag1699 in Cölln bei Berlin. Quäker, Alchemist, Kabbalist, Theosoph, Arzt, Rosenkreuzer.\endnote{"`Helmont, Franciscus Mercurius van,"', Biographisch-Bibliographischen Kirchenlexikon, Band XXV (2005) Spalten 586-597 Autor: Claus Bernet, http://www.bautz.de/bbkl/h/helmont\_f\_m.shtml}

 \item[Hicks, Elias] $\ast$19.3.1748 \dag27.2.1830. Quäker, Wanderprediger, Abolitionisten. An seiner Person entzündete sich ein heftiger Streit, der dann in den  $\to$\textit{"`Hicksites-Orthodox-Schisma"'} der Quäker mündete. Daraus ging dann die $\to$Hicksites und die $\to$\textit{Orthodoxe Linie} hervor.

 \item[Hicksite] Name des eine der entstandenen Flügel, bei der $\to$\textit{"`great separation"'}. Bennant nach $\to$\textit{Elias Hicks}.

 \item[Histrischen Friedenszeugnis] Als Verfasserin des Histrischen Friedenszeugnis gilt $\to$\textit{Margaret Fell}. Die erste Version wurde noch beim Druck konfisziert. Die Zeite Version -- ebenfalls von Margaret Fell -- wurde 21. Januar 1661 feierlich, (aber ohne den Hut abzunehmen) dem Köng Charles II. überreicht, der jedoch kaum darauf reagierte und die Quäker weiter verfolgte. Unterschrieben wurde die Erklährung von 13 Männern, aber nicht von Margaret Fell selbst. $\to$\textit{Friedenszeugnisses}.
 \endnote{In einer ausführlichen Abhandlung ging Claus Bernet in der Ausgabe
 6/2007 auf Seite 282 bis 286 in der Zeitschrift "`Quäker"' auf das "`Histrischen
 Friedenszeugnis"' ein. issn 1619-0394}

 
 \item[Hodgkin, Thomas] $\ast$17.8.1798 , \dag5.4.1866 in Jaffa, damals
 Palästina. Britischer Quaker, Arzt und Pathologe.

\item[Hold in the Light] Im Deutschen wird gesagt "`\textit{Im Licht halten}"'.
Synonym für "`\textit{jemanden in das Gebet einschlissen}"'.

 \item[Hoover, Herbert C.] $\ast$10.8.1874 \dag20.10.1964. Quaker und der 31.
 Präsident der Vereinigten Staaten von Amerika (zwischen 1929 und 1933). 1938
 traf Hoover als erster und einziger Quäker mit Adolf Hitler zusammen.
 Allerdings ohne im geringsten die Gefahr die von den Nationalsozialisten
 ausführlichenging, richti ein zu schätzen!

 \item[Hopkins, Johns] $\ast$19.5.1795, \dag24.12.1873. US-amerikanischer Quaker,
 Geschäftsmann und Philanthrop. Mit seiner beträchtlichen Hinterlassenschaft
 wurden die Universität und das Krankenhaus gegründet, die seinen Namen tragen.

 \item[Hutchinson, Anne] $\ast$Juli 1591, \dag~August 1643. War keine Quakerin. Wurde als Anne Marbury in England geboren; heiratete mit 21 Jahren William Hutchinson und siedelte mit ihm 1634 in die britische Kolonie Massachusetts in Nordamerika um. Sie trat hervor als selbstsichere Frau, was für ihre Zeit ungewöhnlich war. Sie Vorkämpferin des Feminismus und eine Kämpferin für Menschenrechte und Toleranz. Sie lehrte, dass Gott direkt zu jedermann spräche und nicht allein zum Klerus. Sie Beeinflusste die spätere Quäkerin Mary Dyer, die sich beide persönlich kannten.


 \end{description}

\normalsize
