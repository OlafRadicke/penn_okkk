\section*{H}

\articlesize

\begin{description}

 \item[Heirat] Die Quakerhochzeit findet - wie fast alles - formlos statt. Zur
 vermählung wird kein Geistlicher, Pfarrer oder Prediger bestellt. Viel mehr
 geben sich die künftigen Eheleute gegenseitig die treue Beteuerung. Es wird
 darauf verwiesen, das auch in der Bibel keine Heiratszerominie beschrieben
 wird. Der Rest wird frei gestaltet. Meist kombiniert mit einer
 $\to$\textit{Andacht}. Statt den üblichen Trauzeugen, gibt es ein schriftliches
 Trauzeugnis-/urkund, die von allen anwesenden unterzeichnet werden können.
 Dabei spielt die konfessionelle ZTugehörigkeit keine Rolle. Es bereits aus dem
 17.~Jahrhundert Fälle dokumentiert, wo Vermeählungsurkunden der Quaker von
 Mennoniten unterschrieben wurden.\endnote{Sünne Juterczenka, "`Über Gott und
 die Welt - Entzeitvisionen, Reformdebatten, und die europäische Quäkermission
 in der frühen Neuzeit"', Vandenhoeck\&Ruprecht, 2008, ISBN 978-3-525-35458-2,
 Seite~209}


 \item[Hicksite] Name des eine der entstandenen Flügel, bei der
 $\to$\textit{"`great separation"'}. Bennant nach $\to$\textit{Elias Hicks}.

 H\item[Hicksites-Orthodox-Schisma]  $\to$\textit{"`great separation"'}

 \item[Histrischen Friedenszeugnis] Als Verfasserin des Histrischen Friedenszeugnis gilt $\to$\textit{Margaret Fell}. Die erste Version wurde noch beim Druck konfisziert. Die Zeite Version -- ebenfalls von Margaret Fell -- wurde 21. Januar 1661 feierlich, (aber ohne den Hut abzunehmen) dem Köng Charles II. überreicht, der jedoch kaum darauf reagierte und die Quäker weiter verfolgte. Unterschrieben wurde die Erklährung von 13 Männern, aber nicht von Margaret Fell selbst. $\to$\textit{Friedenszeugnisses}.
 \endnote{In einer ausführlichen Abhandlung ging Claus Bernet in der Ausgabe
 6/2007 auf Seite 282 bis 286 in der Zeitschrift "`Quäker"' auf das "`Histrischen
 Friedenszeugnis"' ein. issn 1619-0394}

\item[Hold in the Light] Im Deutschen wird gesagt "`\textit{Im Licht halten}"'.
Synonym für "`\textit{jemanden in das Gebet einschlissen}"'.


 \end{description}

\normalsize
