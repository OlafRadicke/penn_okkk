
\chapter{Kapitel 14} \label{kap14}

\section{Zusammenfassung des 14. Kapitel}

\begin{description}
\item[1. Abschnitt] Von der Maßlosigkeit
\footnote{\texttt{'Üppigkeit' ersetzt durch 'Maßlosigkeit'.}},
worin sie besteht und was für Nachteile
sie der menschlichen Gesellschaft bringt. -- Sie ist eine Feindin des Kreuzes
Christi.
\dotfill \textit{Seite~\pageref{kap14_ab1}}\\
\item[2. Abschnitt] Von der Maßlosigkeit
\footnote{\texttt{'Üppigkeit' ersetzt durch 'Maßlosigkeit'.}}
im Essen und Trinken, wie sehr dieselbe
dem Beispiel und den Lehren Christi und der heiligen Schrift zuwider ist.
\dotfill \textit{Seite~\pageref{kap14_ab2}}\\
\item[3. Abschnitt] Was für Schaden sie dem Menschen an Leib und Seele zufügt.\\
.\dotfill \textit{Seite~\pageref{kap14_ab3}}\\
\item[4. Abschnitt] Von der Maßlosigkeit\footnote{\texttt{'Üppigkeit' ersetzt
durch 'Maßlosigkeit'}} in Kleidern und Vergnügungen. Die Sünde
machte die die erste Kleidung notwendig. Der Mensch sollte daher nicht stolz auf
etwas sein, das ein Zeichen seines Falls ist.
\dotfill \textit{Seite~\pageref{kap14_ab4}}\\
\item[5. Abschnitt] Die jetzt gebräuchlichen Vergnügungen schaden der Tugend,
und
entspringen aus dem entarteten Zustand des Menschen.
\dotfill \textit{Seite~\pageref{kap14_ab5}}\\
\item[6. Abschnitt] Der Zweck der Kleidung ist erlaubt, aber der Missbrauch
desselben ist zu tadeln.
\dotfill \textit{Seite~\pageref{kap14_ab6}}\\
\item[7. Abschnitt] Das vornemste Vergnügen guter Menschen der Vorzeit bestand
darin, dass sie Gott dienten, ihren Mitmenschen Gutes erwiesen und ehrlichen
Berufsgeschäften, aber nicht eitlen Vergnügungen und Belustigungen nachgingen.
\dotfill \textit{Seite~\pageref{kap14_ab7}}\\
\item[8. Abschnitt] Schon die Heiden kannten und taten etwas besseres. --
Einige wilde Völker übertreffen die Christentumsbekenner an Mäßigkeit und
Nüchternheit des Lebens.
\dotfill \textit{Seite~\pageref{kap14_ab8}}\\
\item[9. Abschnitt] In dem Gleichnis vom reichen Mann wird das
maßlose\footnote{\texttt{'üppige' ersetzt durch 'maßlose'.}} Leben
verdammt.
\dotfill \textit{Seite~\pageref{kap14_ab9}}\\
\item[10. Abschnitt] Die Lehren der heiligen Schrift sind durchaus dagegen.
\dotfill \textit{Seite~\pageref{kap14_ab10}}\\
\end{description}

\newpage

\section{1. Abschnitt} \label{kap14_ab1}

Jetzt komme ich zur Betrachtung einer dem Geist ganz entgegengesetzten
Leidenschaft: der Maßlosigkeit\footnote{\texttt{'Üppigkeit' ersetzt durch
'Maßlosigkeit'}}\index{Maßlosigkeit}, die in Befriedigung der Prachtliebe und
eines
ausschweifenden\index{Ausschweifung} Hangs zur
Bequemlichkeit\index{Bequemlichkeit} und zu Vergnügungen\index{Vergnügen}
besteht. Sie ist
das letzte Laster\index{Laster}, welches ich in der gegenwärtigen Abhandlung
über das heilige
Kreuz Christi angreife, dessen göttliche Kraft und Wirkung allein vermögend ist,
dieses eben so ansteckende als tödliche Übel\index{Übel!tödliche} zu
zerstören\index{zerstören} und auszurotten\index{auszurotten}. Es
ist eine Untugend, die sich in alle Stände\index{Stände} einschleicht, welche
die Ärmsten
hinreisst, dass sie ihre Kräfte überschreiten, um ihre
maßlose\footnote{\texttt{'üppigen' ersetzt durch 'maßlose'.}} Neigungen zu
befriedigen und die Reichen so sehr einnimmt, dass sie die inneren Züchtigungen
Jesu\index{Züchtigungen!durch Jesu}, den sie dennoch ihren
Heiland\index{Heiland} nennen, ganz aus den Augen verlieren, und sich
den Vergnügungen der Augenlust\index{Augenlust}, der
Fleischeslust\index{Fleischeslust} und des
hofartigen Lebens so
sehr ergeben, als wenn nicht
Selbstüberwindung\footnote{\texttt{'Selbstverleugnung' ersetzt
durch 'Selbstüberwindung'}}
\index{Selbst!-überwindung} und Unterwerfung\index{Kreuz!Unterwerfen unter das}
unter das Kreuz
Christi, sondern Befriedigung der Maßlosigkeit\footnote{\texttt{'Üppigkeit'
ersetzt durch 'Maßlosigkeit'}},
der uns verordnete Weg zum Himmel
wäre. Was wollen wir essen? Was wollen wir trinken? Wie wollen wir uns kleiden?
-- Diese Sorgen, wodurch einst die maßlosen\footnote{\texttt{'üppigen' ersetzt
durch 'maßlosen'}}
Heiden\index{Heide} sich auszeichneten, sind jetzt
allgemein, ja, was noch schlimmer ist, das Dichten\index{Dichten} und Trachten
der Bekenner des
Christentums geworden. Möchten sie doch darüber erröten! Möchten sie ihre
Abweichung\index{Abweichung} bereuen und bedenken, dass Christus solche
Sorgen\index{Sorgen machen} an jenen Heiden
nicht darum tadelte, dass er sie seinen Nachfolgern erlauben wollte.
\label{ref:14_01_wahre_nachfolger} \textbf{Diejenigen,
welche sich in Wahrheit zu der Nachfolge Christi bekennen, müssen auch zeigen,
dass sie ihm angehören, sie müssen
\textit{"`gesinnt sein, wie er war,"'} ein nüchternes\index{Nüchternheit}
und mäßiges Leben\index{mäßiges} führen, und sich stets erinnern,
\textit{"`dass der Herr nahe ist."'}\index{der-Herr-ist-nahe}
Prächtige Kleider\index{Kleider!Prächtige}, schöne Putzsachen, kostbare
Möbel\index{Möbel!kostbare},
eine reiche Tafel und die
gewöhnlichen Vergnügungen\index{Vergnügen} und
Unterhaltungen\index{Unterhaltungen} der Welt,
also: Schauspiele\index{Schauspiel}, Bälle\index{Bälle},
Spiele\index{Spiele}, Romane\index{Romane}, und dergleichen sind auf dem
heiligen Pfade,
den Jesus und seine wahren
Jünger und Nachfolger betraten, nicht anzutreffen.} Nein! Einer, der keiner der
Geringsten unter den Aposteln war, sagt uns,
\textit{"`dass wir durch viele Trübsal ins
Reich Gottes eingehen müssen."'}\index{Reich!Gottes}
\footnote{Apostelgeschichte 14,22}
Es ist daher meine
ernste Bitte an alle, die der Prachtliebe und
Maßlosigkeit\footnote{\texttt{'Üppigkeit' ersetzt durch
'Maßlosigkeit'}} ergeben sind, und
denen diese Abhandlung vielleicht in die Hände fallen möge, dass sie die Gründe
und Beispiele, die ich gegen ihre Lebensweise hier anführe, ernstlich in
Erwägung ziehen wollen. Vielleicht können sie dadurch zu der Einsicht gelangen,
wie sehr ihr Leben und Betragen von dem wahren
Christentum\index{Christentum!wahres} entfernt
und ihrem
ewigen Frieden\index{Frieden!ewigen} gefährlich sei.
\label{ref:14_01_wahre_nachfolger_suenetod}
\textbf{Möge sie der allmächtige Gott für diesen
Unterricht empfänglich machen, und seine zärtliche Liebe in ihre Herzen
ausgießen, damit sie zu einem Gefühl der Reue\index{Reue} gebracht werden
und den heiligen
Weg des Kreuzes Jesu\index{Kreuz!Jesu}, des Erlösers\index{Erlöser} der
gefallenen
Menschheit\index{gefallenen Menschheit}, lieb gewinnen,
denn sie dürfen nicht glauben, dass sein Versönungstod\index{Versönungstod}
ihnen etwas nützen werde,
so lange sie sich weigern, aus Liebe zu ihm ihre Sünden\index{Sünde} zu
verlassen, da er doch
aus Liebe zu ihnen sein Leben ließ, oder dass er ihnen eine Stätte im Himmel
bereiten werde, wenn sie ihm einen Platz in ihrem Herzen auf Erden versagen.}
Wir
wollen nun die Maßlosigkeit\footnote{\texttt{'Üppigkeit' ersetzt durch
'Maßlosigkeit'}}
in ihren verschiedenen Teilen betrachten.

\section{2. Abschnitt} \label{kap14_ab2}

Unter den verschiedenen Arten der Maßlosigkeit\footnote{\texttt{'Üppigkeit'
ersetzt durch 'Maßlosigkeit'}} betrifft die erste, die Christus,
der uns ein Muster der Selbstüberwindung\footnote{\texttt{'Selbstverleugnung'
ersetzt durch 'Selbstüberwindung'}} gegeben hat, ausdrücklieh verbietet,
die Sorge für den Bauch.
"`Ihr sollt nicht sorgen, und sagen: Was werden wir
essen? Oder was werden wir trinken? ... Nach solchen Dingen trachten die
Heiden."'\index{Heide}\footnote{Matthäus 6,31+32.}
\index{Bibelstellen:!Matthäus 06@Matthäus 6)}
Das heißt: Jenen Heiden, die ohne
Erkenntnis des wahren Gottes\index{Gott!wahrer} in der Welt leben, die ihren
Bauch zu ihrem Gott
machen, und deren Trachten nur auf die Befriedigung ihrer
Begierden\index{Begierde}, aber nicht
auf die Erlangung des Reiches Gottes\index{Reich!Gottes} gerichtet ist, denen
sollt ihr nicht gleich
sein.
\textit{"`Trachtet ihr zuerst nach dem Reich Gottes und nach seiner
Gerechtigkeit, und alles andere soll euch dazugegeben werden."'} Was für euch
gut
ist, wird schon erfolgen. Lasst nur alles seine rechte Zeit\index{Zeit!rechte}
und Ordnung haben.

\medskip

In diesen Worten liegt ein scharfer Tadel\index{Tadel} für alle, die der
Maßlosigkeit\footnote{\texttt{'Üppigkeit' ersetzt durch 'Maßlosigkeit'.}} im Essen
und Trinken ergeben sind, deren ganze Sorge nur dahin geht, wie sie ihren
Gaumen ergötzen und ihren Bauch füllen, was sie essen und trinken wollen. Diese
sind nicht selten verlegen, was für eine Speise sie für die nächste Mahlzeit
anordnen sollen, und einige halten sich daher einen eigenen
Speisemeister\index{Speisemeister}, um
neue Gerichte\index{Gerichte!neue} für den Koch zu erfinden,
der sie so künstlich würzen\index{Geschmack!künstlicher}, verändern
und zubereiten muss, dass Auge und Geschmack getäuscht werden, und man sie für
etwas ganz Neues und Sonderbares hält, bloß um die Esslust zu reizen oder
Bewunderung zu erregen. Der Schwelger\index{Schwelger, Der} fordert in der
Tat eine große
Verschiedenheit seltener und köstlicher Gerichte, wobei nicht selten die Brühe
weit höher als die Speise selbst zu stehen kommt, denn er ist beständig so voll
und satt, dass er ohne Anwendung solcher künstlichen
Reizmittel\index{Appetitanreger} gar keinen
Appetit haben würde. Dieses heißt aber: den Hunger erzwingen, nicht: ihn
befriedigen. Und so, wie der Maßlose\footnote{\texttt{'Ueppige' ersetzt durch
'Maßlose'}} isst,
so trinkt er auch, selten aus Durst,
bloß zum Vergnügen\index{Vergnügen} und dem Gaumen zu Gefallen. Darum muss er
auch eine
Verschiedenheit von Getränken\index{Getränke!kostbare} haben, die er alle kosten
muss. Eine Sorte, wäre
sie auch noch so gut, wurde ihm bald unschmackhaft und widerlich werden. Die
Abwechslung muss ihm die Getränke behaglich machen, und daher ist die ganze Welt
kaum im Stande, seinen Keller mit kostbaren Getränken anzufüllen. Wäre er dabei
noch mäßig im Genusse, so könnte man seine Abwechslung eher der Neugier als der
maßlosen\footnote{\texttt{'üppigen' ersetzt durch 'maßlosen'.}} Schwelgerei
Zuschreiben. Allein was
der Mäßige als Herzstärkung
geniesst, trinkt er mit vollen Zügen, bis er, \index{Betrunkenkheit}durch
Übermaß erhitzt, zu
Torheiten schreitet, und, wo nicht anderen, doch immer sich selbst Schaden
zufügt, ja, oft so tief unter das unvernünftige Tier herabsinkt, dass er sich
selbst seines Bewusstseins beraubt. Dieses ist die Wirkung der Fleischeslust,
\textit{"`die nicht vom Vater, sondern von der Welt ist,"'} und in deren Gefolge
Musik\index{Musik}
und Tanz\index{Tanz}, Ausgelassenheit und Gelächter, -- welches der weise Mann
Tollheit
nennt, -- gewöhnlich zur Hand sind, damit der rauschende Lärm der einen
Ergözlichkeit, das Gefühl der Sündhaftigkeit\index{Sündhaftigkeit} der anderen
übertäuben\index{Beteuben, sich}, und die
Stimme des Gewissens\index{Gewissen!Stimme des} nicht zu laut und deutlich mit
dem Schwelger
reden möge. So
leben die Üppigen, die Gott vergessen und des Unglücklichen nicht gedenken. O!
Möchten doch die Kinder der Menschen das Törichte und Gottlose dieser Dinge
einsehen! Wie schlecht erwidern sie die Güte Gottes durch einen solchen
Missbrauch des Überflusses, den er ihnen verleiht! Wie grausam gehen sie oft
mit seinen Geschöpfen\index{Tiere!quälen}, wie verschwenderisch mit ihrem
eigenen Leben und mit
ihren Kräften um, und wie undankbar\index{Undank} sind sie für die Wohltaten,
die sie
genießen? Sie vergessen den Geber, indem sie seine Gaben zur Befriedigung ihrer
Lüste missbrauchen, allen Rat verachten und jede Belehrung von sich stossen.
Dadurch verlieren sie endlich alles Zartgefühl und vergessen ihre
Pflichten\index{Pflichtvergessenheit},
indem sie im Taumel sinnlicher Genüsse von einer Ausschweifung zur anderen
übergehen.

\medskip

Gott rügte diese Sünden bei den Juden\index{Jude}, als er durch den
Propheten Amos sagte:
\textit{"`Wehe euch! ... die ihr euch weit vom bösen Tage achtet, (den bösen Tag
weit
hinaussetzt,) und trachtet immer nach Frevel. Regiment! (oder: rücket den Stuhl
des Frevels oder der Gewalttätigkeit herbei!) Ihr schlafet auf elfenbeinernen
Lagern und treibt Überfluss mit eueren Betten! Ihr esst die Lämmer aus der
Heerde, und die gemästeten Kälber, und spielt auf dem Psalter, (singt zu dem
Saitenspiel,) und erdichtet euch Lieder, (erfindet euch musikalische
Instrumente,) wie David! Ihr trinkt Wein aus den Schalen, und salbt euch mit
Balsam, aber ihr bekümmert euch nicht um den Schaden Josephs!"'}\footnote{Amos
6,3-6}
\index{Bibelstellen:!Amos 06@Amos 6)}
Dieses waren, wie es scheint, die Laster, denen die entarteten
Juden\index{Jude!entarteter} bei allen ihren Ansprüchen auf
Religiosität ergeben waren,
und sind nicht
heutigen Tags dieselben Übel auch unter den Christen\index{Christ}
herrschend geworden? --
Allerdings! Und sie machen einen großen Teil der
Maßlosigkeit\footnote{\texttt{'Üppigkeit' ersetzt
durch 'Maßlosigkeit'}} aus, die ich in
dieser Abhandlung angreife. Denke an den reichen Mann, der mit seiner ganzen
Pracht und Herrlichkeit, und bei allen seinen hohen Genüssen verloren ging, und
erinnere dich, was für ein schweres Wehe der Apostel über diejenigen ausspricht,
\textit{"`deren Bauch ihr Gott ist, und die in ihrer Schande eine Ehre
suchen"'}\footnote{Philipper 5,19}
\index{Bibelstellen:!Philipper 05@Philipper 5)}

\medskip

Christus redet von der Maßlosigkeit\footnote{\texttt{'Üppigkeit' ersetzt durch
'Maßlosigkeit'}} als von
etwas, das an den Höfen weltlicher
Könige anzutreffen ist, in seinem Reich aber nicht stattfindet und seinen
Nachfolgern nicht geziemt. Daher war auch sein Fest, das er der Menge gab, und
welches eins seiner Wunder war, ganz einfach und ohne Vorbereitung. Er gab den
Leuten reichlich, aber nichts Seltenes, oder etwas, das durch die
Kochkunst\index{Kochkunst}
zubereitet war, dennoch aßen sie es gern, weil sie Hunger hatten, welcher auch
immer das beste und schicklichste Mittel zur Beförderung der Esslust ist. -- In
einer Anweisung, die Paulus seinem geliebten Timotheus erteilte, setzt er die
Liebhaber des weltlichen Überflusses sehr herab, indem er behauptet,
\textit{"`dass Gottseligkeit mit Genügsamkeit verbunden, großer Gewinn sei,"'}
und noch
hinzufügt, \textit{"`dass wir, wenn wir Nahrung und Kleider haben, uns begnügen
sollen."'}\footnote{1. Timotheus 6,6-11}
\index{Bibelstellen:!Timotheus 1 06@1. Timotheus 6)}
Dieses gibt uns einen Begriff von dem
enthaltsamen und genügsamen Leben jener erhabenen
Pilger\index{Pilger}, jener
Söhne des Himmels\index{Söhne des Himmels} und wahren Abkömmlinge der
ewigen Kraft
Gottes, die sich oft in der
Notwendigkeit befanden, zu fasten\index{Fasten} und Gefahren entgegen zu gehen,
sich aber
mit dem begnügten, \textit{"`was ihnen vorgesetzt wurde."'} O! Heilige Menschen!
O!
Selige Geister! Möge meine Seele ewig bei den eurigen wohnen!

\section{3. Abschnitt} \label{kap14_ab3}

Die Krankheiten, welche die Maßlosigkeit\footnote{\texttt{'Üppigkeit' ersetzt
durch 'Maßlosigkeit'}} erzeugt
und nährt, machen sie zu einer
wahren Feindin des Menschengeschlechts. Denn außer dem Nachteil, den sie der
Seele des Menschen zufügt, untergräbt sie auch seine
Gesundheit\index{Gesundheit} und verkürzt ihm
das Leben, indem sie ihm nur elende Nahrung reicht, wodurch schlechte Säfte
erzeugt werden, die seinen Körper ungesund, träge,
skorbutisch\footnote{\texttt{gemeint ist
wohl die Krankheit \textit{Skorbut}. Siehe: de.wikipedia.org/wiki/Skorbut}} und
für ein tätiges,
arbeitsames Leben\index{arbeitsames Leben} ganz untauglich machen. Und sind
einmal die Lebensgeister des
Menschen von seinem elenden Fleische unterdrückt und seine Seelenkräfte
erschlafft, so macht sein Hang zur Untätigkeit\index{Untätigkeit} ihn für die
bürgerliche Gesellschaft
\index{Gesellschaft} ganz unbrauchbar,
denn die Maßlosigkeit\footnote{\texttt{'Üppigkeit' ersetzt durch 'Maßlosigkeit'.}}
macht die Menschen nicht
allein krank, sondern auch zu Müßiggängern\index{Müßiggang}, die alles Gute
verschlingen und
sich selbst so sehe lieben, dass sie Gott vergessen. Diese sind wirklich der
Welt
zur Last\index{Unsoziales Verhalten}, und die Schrift sagt uns, -- was eben so
schreklich
als gerecht ist,
-- \textit{"`dass diejenigen, welche Gott vergessen, ins Verderben gehen
sollen."'}\index{Verdammnis}\footnote{Psalm 9,18}
\index{Bibelstellen:!Psalm 009@Psalm 9)}

\section{4. Abschnitt} \label{kap14_ab4}

Es gibt aber noch eine andere Art der Maßlosigkeit\footnote{\texttt{'Üppigkeit'
ersetzt
durch 'Maßlosigkeit'}}, die bei eitlen Personen\index{Eitelkeit} beiderlei
Geschlechts häufig angetroffen wird, nämlich die Kleiderpracht\index{Kleidung},
eine der
größten Torheiten, welcher der Mensch sich schuldig machen kann, da sie eine
der kostbarsten und zugleich nichtigsten und unnützesten Ausschweifungen ist.
\textbf{Die
Schrift der Warheit lehrt uns zu glauben, dass die Sünde die erste Kleidung in
der
Welt notwendig machte;}\footnote{1. Mose 3,21}
\index{Bibelstellen:!Mose 1 03@1. Mose 3)}
und wenn die Übereinstimmung
mehrerer Schriftsteller einiges Gewicht hat, so erstreckte sich der Fall des
Menschen sowohl auf sein Äußeres als auch auf sein Inneres. Ich richte daher
diese
Abhandlung an diejenigen, welche dieses glauben, da diese, wie ich hoffe, die
größere Anzahl meiner Leser ausmachen. -- Ich sage, wenn die Sünde zuerst die
Kleidung veranlasste, so haben die armen Adam’s\index{Adam}
Abkömmlinge wenig Ursache, auf
ihre Kleidung stolz zu sein, oder sich darin zu suchen, da ihr Ursprung so
niedrig ist, und der Schmuck derselben weder den Menschen veredelt, noch zu
seiner Unschuld zurückführen kann. Selig war aber gewiss jene Zeit, als noch
Unschuld, nicht Unwissenheit, Kleider unnötig machte. Damals waren unsere ersten
Eltern nackt, und kannten keine Scham\index{Scham}, bis die Sünde sie beschämt
machte,
länger nackt zu bleiben. \label{ref:14_04_wahre_nachfolger_kleidung}
\textbf{Da also das Bewusstsein der Schuld die Scham erzeugte,
und diese eine Bedeckung erforderte, wie tief sind denn diejenigen gefallen, die
in ihrer Schande sich brüsten und auf ihren Fall stolz tun? Dieses tun
unstreitig aber alle, die Mühe und Kosten anwenden, ihre Kleidung, das wahre
Kennzeichen jenes beklagenswerten Falls, zu schmücken und herauszuputzen. Und
ist es also wohl bei denen, die sich Christen nennen, zu entschuldigen, wenn sie
eine so große Vorliebe zu prächtigen Kleidern hegen, dass sie die erste und
vornehmste Bekleidung des Menschen, das Kleid der Unschuld, darüber
vernachlässigen?} Beweist es nicht die größte Torheit der Menschen, wenn sie so
viel Zeit, Mühe und Geld anwenden, um ihre Schande recht zur Schau zu stellen?
Heißt dieses nicht Vergnügen an den Folgen einer Sache finden, deren Entstehung
man bejammern sollte? Ja, wenn ein Verbrecher verurteilt wäre, lebenslänglich
in Ketten zu gehen, würde es wohl seine Schande vermindern, wenn seine Ketten
von Gold\index{Gold} gemacht und sehr künstlich gearbeitet wären? Gewiss, dieser
Umstand
würde die Folgen seines Verbrechens nur noch auffallender machen und seine
Schande vermehren. Dieses ist nun aber gerade der Fall mit den
Sklaven\index{Sklave} der
eitlen Moden unseres Zeitalters, die dessenungeachtet Christen, Verurteiler
religiöser Gegenstände, fromme Leute, und wer weiss was noch mehr, sein wollen.
O! Des bejammernswerten Zustandes der Menschen, welche die Augenlust, die
Fleischeslust und das hoffärtige Leben so sehr verblendet\index{Verblendung}
hat, dass sie sich mit
Dingen zieren und schminken\index{Schminken}, die ihren Fall beweisen und sogar
erfinderisch und
verschwenderisch in diesen Dingen sind, die ihnen doch eigentlich zur
Demütigung dienen sollten! Ihre große Anhänglichkeit an diese
Eitelkeiten\index{Eitelkeit}
zeigt offenbar, wieweit sie von der ersten Unschuld\index{Unschuld} abgewichen
sind. Und wem ist
nicht bekannt, wie viele Moden\index{Mode} schon erfunden sind und noch immer
erdacht
werden, die keinen anderen Zweck haben, als Lüsternheit zu erregen, wodurch das
Herz des Menschen immer mehr von der Einfalt und Unschuld entfernt und in die
Sklaverei\index{Sklave} niedriger Sinnlichkeit geführt wird.

\section{5. Abschnitt} \label{kap14_ab5}

Ebenso verhält es sich mit ihren Vergnügungen oder Erhohlungen\index{Erhohlung},
wie man sie
nennt, denn diese stehen mit der
Maßlosigkeit\footnote{\texttt{'Üppigkeit' ersetzt durch 'Maßlosigkeit'.}} in
Kleidern in naher
Verbindung. Der
Mensch wurde als ein edles, vernünftiges und ernstes Wesen erschaffen.
\index{Zweck des
Menschen}\textbf{Sein
Vergnügen bestand in der Erfüllung seiner Pflicht, und diese war: Gott zu
gehorchen, das heißt, ihn zu lieben, zu fürchten, zu verehren und anzubeten,}
und
seine Geschöpfe mit wahrer Mäßigkeit und heiliger Genügsamkeit zu gebrauchen,
indem er wohl wusste, dass der Herr, sein Richter, nahe war, der seine
Handlungen
beobachtete und vergelten würde. Kurz, seine Glückseligkeit bestand in der
Gemeinschaft mit Gott und sein Fehler war, dass er diese selige Gemeinschaft
verliess und mit seinem Gemüte vergänglichen Dingen nachhing. Waren nun aber
die Ergötzungen der gegenwärtigen Welt wirklich so angenehm und notwendig, als
man vorgibt, so würde es ja Adam und Eva in dem Stande ihrer Glückseligkeit an
der Freude, die sie gewähren, gemangelt haben, da sie dieselben gar nicht
kannten. Allein es verhält sich anders. Diese Vergnügungen sind das nicht, was
sie zu sein scheinen, und wenn unsere ersten Eltern nicht gefallen wären, und
die Welt mit ihrer Torheit und mit ihrem bösen Beispiel nicht angesteckt
hätten, so würden wir warscheinlich den Gebrauch und die Notwendigkeit vieler
solcher Vergnügungen nicht kennen, die, ebensowohl als die Kleiderpracht, aus
der Sünde entsprungen sind. \index{Sündenfall} Denn, als Adam und Eva gesündigt
hatten, fürchteten
sie die Gegenwart des Herrn, die in ihrer Unschuld sie erfreuete, und nun ließen
sie ihre Gemüter auf andere Gegenstände ausschweifen, suchten andere
Vergnügungen, und fingen an, Gott, ihren Schöpfer, zu vergessen. Daher klagte
Gott hernach durch den Propheten Amon:
\textit{"`Sie achten sich weit vom bösen Tag,
sie essen die Lämmer aus der Heerde, trinken Wein aus Schalen, salben sich mit
Balsam, schlafen auf elfenbeinernen Lagern, spielen auf dem Psalter, erdichten
sich Lieder, wie David, und bekümmern sich nicht um den Schaden
Josephs."'}\footnote{Amos 6,3-6}
\index{Bibelstellen:!Amos 06@Amos 6)}
Sie verkauften Joseph auf eine gottlose
Weise, verbanneten die Unschuld gänzlich, und gewöhnten sich bald an die Schande
so sehr, dass sie über die Nachahmung schändlicher Handlungen gar nicht mehr
erröteten. Und jetzt schämen sich Menschen, die wahrlich eben so sehr, sich der
ursprünglichen Unschuld und bescheidenen Einfachheit wieder zu nähern, als Adam
sich schämte, da er diese Tugenden verloren hatte, und deswegen, -- wiewoh1
vergeblich, -- seine Zuflucht zu einer Bedeckung von Feigenblättern nahm. Ebenso
vergeblich suchen jetzt die Menschen mit scheinbaren Ansprüchen auf Religion
sich zu bedecken, und ihren armen Seelen durch Zueignung der schönen
Benennungen: christlich, unschuldig, gütig, tugendhaft, usw. zu schmeicheln,
während doch Eitelkeit und Torheit die Herrschaft über sie haben.

\medskip

Darum fühle ich mich von dem ewigen Gott verpflichtet, allen solchen zu sagen:
Ihr verspottet \textit{Ihn},
\textit{"`der sich nicht will spotten lassen,"'}\footnote{Galater 6,7}
\index{Bibelstellen:!Galater 06@Galater 6)}
und betrügt euch selbst\index{Selbst!-betrug}! Solche Unmäßigkeit, als unter
euch herrscht, muss
überwunden\footnote{\texttt{'verleugnet ersetzt durch 'überwunden'.}} werden.
\label{ref:14_06_wahre_nachfolger_umkehr}
\textbf{Ihr müsst eine Veränderung euerer Gesinnungen erfahren und euch
mehr der ursprünglichen Reinheit des Christentums nähern, ehe ihr ein Recht zu
jenen Benennungen haben könnt, die ihr euch jetzt nur anmasst.}
\index{Christ!wahrer} Denn
\textit{"`die wahren
Kinder Gottes bestehen nur aus solchen, die vom Geist Gottes geleitet werden,
der zur Mäßigkeit, Sanftmut, Keuschheit und allen anderen Tugenden
führt."'}\footnote{Römer 8,14}
\index{Bibelstellen:!Römer 08@Römer 8)}

\section{6. Abschnitt} \label{kap14_ab6}

Die christliche Welt, oder vielmehr die weltlichen
Christen\index{Christ!weltlicher},
verdienen aber auch
deswegen billigen Tadel, da sie den rechten Zweck, wozu die Kleidung zuerst
eingeführt wurde, ganz verkehrt haben. Der erste Dienst, den die Kleider dem
Menschen leisten sollte, als die Sünde ihn seiner angeborenen Unschuld beraubt
hatte, war, wie schon bemerkt wurde, Bedeckung, und sie mussten schon in dieser
Hinsicht einfach und anspruchslos sein. \textbf{Ihr nächster Zweck war, ihn
gegen Kälte
zu schützen, daher mussten sie Stärke und Festigkeit haben. Und endlich sollten
sie auch zur Unterscheidung der beiden Geschlechter dienen und mussten folglich
in dieser Rücksicht für beide verschieden sein.}\index{Kleidung!Beschaffenheit
der} Statt dass aber
damals das
Bedürfnis die notwendige Bekleidung erforderte, macht jetzt der Stolz und die
eitle Prachtliebe eine Menge Kleider notwendig, statt dass sie damals nur zum
Nutzen des Menschen gebraucht wurden, dienen sie jetzt nur, seine
Eitelkeit\index{Eitelkeit} und
Gefallsucht zu befriedigen. Anfänglich brauchte man die Kleider wirklich zur
Bedeckung, jetzt macht diese den kleinsten Teil ihres Zwecks aus. Das
unersättliche Auge der eitlen Menschen verlangt prachtvolle Überflüssigkeiten,
als ob ihre Kleider bloß der anzubringenden Verzierungen wegen gemacht werden
müssten, und nur zur Schau, aber nicht zum Tragen bestimmt wären. In wiefern sie
die Blöße bedecken, gegen Kälte schützen und der Anständigkeit des Geschlechts
entsprechen, darum scheint man in der Tat sich wenig zu bekümmern, indem
sie nur zu augenscheinlich so eingerichtet werden, dass sie den Stolz, die
Eitelkeit und Lüsternheit derer, die sie tragen, verraten.

\section{7. Abschnitt} \label{kap14_ab7}

\label{ref:14_07_wahre_nachfolger_erholung}
\index{Beschäftigung!gute} \textbf{Vor Zeiten bestanden die
besten \textit{Erhohlungen}\index{Erhohlungen} der Menschen darin, dass sie Gott
dienten, Gerechtigkeit übten, ihren Berufsgeschäften nachgingen, ihre Heerden
weideten, anderen Gutes taten und körperliche Übungen vornahmen, die mit der
Mäßigkeit und Tugend übereinstimmten und ihrer Ernsthaftigkeit und Würde
angemessen waren.} Jetzt wird der Name \textit{Erholung} fast jeder Torheit
beigelegt, die nur einigermaßen über offenbar schändliche Ausschweifungen
erhaben ist, wiewohl selbst diejenigen, die sie begehen, darüber erröten
müssen, wenn sie bedenken, was sie getan haben. So weit sind die Menschen sogar
von dem Zustande, in welchem Adam\index{Adam} nach seinem Ungehorsam
sich befand, entartet
und abgewichen! So viel vertrauter sind sie mit allen Lastern und so viel klüger
und geschickter in ihrer Ausübung geworden! Ja, so unempfindlich hat endlich die
Gewohnheit ihre Gemüter gegen den Zwang und die Unbequemlichkeit gemacht, die
fast beständig mit der Kleiderpracht verbunden sind, dass diese Mittel zur
Bedeckung der Blöße des Menschen, die einst die Notwendigkeit einführte, jetzt
die Sorge, das Vergnügen, die Ergötzung, ja, das Dichten und Trachten aller
Stände geworden sind. -- Aber wie unedel, wie schimpflich, wie unwürdig des
Charakters eines vernünftigen Wesens ist dieses! Wie kann der mit Verstand
begabte, des Nachdenkens über Unsterblichkeit fähige Mensch, der den
Engeln\index{Engeln}
gleich, wo nicht noch höher, geachtet ist, -- wie kann der denkende Mensch einen
solchen Wert auf so nichtswürdige Dinge, auf einige ärmliche Lampen legen, die
bloß der Stolz erfunden und die Maßlosigkeit\footnote{\texttt{'Üppigkeit' ersetzt
durch 'Maßlosigkeit'}}
eingeführt hat? Die unbedeutendsten
Spielsachen kleiner Kinder\index{Kinder!-spiele} können keine so lächerliche und
unwürdige
Beschäftigung für Erwachsene abgeben, als die törichten Erfindungen der Mode
für verständige Menschen sein würden, wenn sie dieselben als Gegenstände ihrer
Sorge und ihres Vergnügens betrachten wollten. Und es verrät in der Tat immer
große Beschränktheit des Verstandes, wenn solche Eitelkeiten das edle Gemüt des
Menschen beschäftigen, der das Ebenbild des großen Schöpfers des Himmels und der
Erde ist.

\section{8. Abschnitt} \label{kap14_ab8}

Hiervon hatten schon viele der alten Heiden\index{Heide} so klare
Begriffe, dass sie alle
solche eitlen Dinge verabscheueten, und sowohl die Kleiderpracht, als auch die
jetzt unter den Bekennern des Christentums üblichen und so sehr beliebten
Vergnügungen und Erhohlungen, als sittenverderbende Übel betrachteten, weil sie
die Gemüter der Menschen von der Nüchternheit und Mäßigkeit zur
Maßlosigkeit\footnote{\texttt{'Üppigkeit'
ersetzt durch 'Maßlosigkeit'}}, zum
Müßiggange und zur Weichlichkeit\index{Verweichlichung} verleiteten, und sie
endlich
unter das
vernunftlose Tier\index{Tiere!Rangordnung} herabsezten. Davon zeugen:
Naragoras\index{Personen:!Naragoras}, Sokrates\index{Personen:!Sokrates},
Plato\index{Personen:!Plato},
Aristides\index{Personen:!Aristides}, Cato\index{Personen:!Cato},
Seneca\index{Personen:!Seneca},
Epietet\index{Personen:!Epietet}, und andere, welche der Meinung waren, dass nur
in
Tugend und Unsterblichkeit wahre Ehre\index{Ehre!wahre} und wahrer
Genuss\index{Genuss!wahrer} in
finden sei. Selbst
unter einigen Schwarze\footnote{\texttt{'Negern' ersetzt durch 'Schwarze'.}}
\index{Neger}\index{Schwarze} und Indianern\index{Indianer} findet man noch
jetzt unverkennbare
Spuren
von Unschuld und Reinheit der Sitten. Sie beobachten bei ihrem Handel ein
einfaches und gerades Benehmen\index{Benehmen!gerades}, und wenn ein Christ, --
der freilich wohl von
einer besondern Art sein muss, -- ein schmutziges Wort ausstösst, so haben sie
den
Gebrauch, dass sie ihm, um ihn seine Unschicklichkeit fühlen zu lassen, ein
Gefäss mit Wasser reichen, damit er seinen Mund reinigen möge. Wie sehr klagen
nicht solche Tugenden und vernünftige Beispiele diejenigen, die sich zum
Christentum bekennen, der gröbsten Torheit und Zügellosigkeit an! O! Möchten
doch die Menschen die Furcht Gottes vor Augen haben! Möchten sie so viel Mitleid
mit sich selbst haben, dass sie bedächten, welchen Ursprungs sie sind, was sie
tun und treiben, und wohin sie zurückkehren müssen, damit sie edlere,
tugendhaftere, vernünftigere und himmlische Dinge zu Gegenständen ihrer Freude
und Erholung wählten, und einmal überzeugt würden, wie unverträglich die
Torheiten, Eitelkeiten und Vergnügungen, denen sie sich größtenteils ergeben
haben, mit dem wahren Adel vernünftiger Wesen sind! Ja, mögen sie endlich dem
göttlichen Einflusse, der die tugendhaften
Heiden\index{Heide!tugendhafter} leitete,
Gehör geben;
damit es nicht diesen am Tag des Gerichts\index{Jüngstes Gericht} erträglicher
als ihnen erginge! Denn
wenn jene, von geringer Einsicht und unvollkommener Erkenntnis, schon so viele
eitle Dinge entdeckten und einsetzen, wenn ihr geringeres Maß von Licht
dieselben verwarf, und sie, aus Gehorsam gegen dasselbe, sich von ihnen
losrissen, was wird dann nicht von den Christen erwartet? --
\label{ref:14_08_wahre_nachfolger_rational}
\textbf{Christus kam nicht,
um diese Erkenntnis auszulöschen, sondern sie zu vermehren, und wenn daher
jemand denkt, er brauche es so genau nicht zu nehmen, als jene, der hat gewiss
nötig, besser zu handeln als er denkt.} Ich behaupte, dass die Moden und
Vergnügungen, die man jetzt so sehr erhebt, dem Zwecke der Schöpfung des
Menschen sehr entgegen wirken, und dass die Untugenden, die sie begleiten,
nämlich: Ausschweifung\index{Ausschweifung}, Müßiggang\index{Müßiggang},
Verschwendung\index{Verschwendung}, Stolz, Lüsternheit, Ansehen
der Person um der prächtigen Kleider willen, und mehrere solcher schlechten
Früchte, sowohl der Pflicht und Vernunft, als auch dem wahren Vergnügen des
Menschen zuwider sind, und mit der Weisheit, Erkenntnis, Selbstständigkeit,
Mäßigkeit\index{Mäßigkeit} und Betriebsamkeit, welche ihn wahrhaft edel und gut
machen, auf keine
Weise sich vereinigen lassen.

\section{9. Abschnitt} \label{kap14_ab9}

Auch haben die heiligen Menschen der Vorzeit\index{Vorzeit}, welche uns die
Schrift als würdige
Muster zur Nachahmung empfiehlt, die hier getadelten Dinge weder gestattet noch
selbst gebraucht. Abraham\index{Personen:!Abraham}, Isaak\index{Personen:!Isaak}
und
Jakob\index{Personen:!Jakob} waren Männer von einfachen Sitten,
Hirten, aber doch Fürsten in ihren Familien und über ihre Heerden. Sie
kümmerten sich nicht um die eitlen Dinge, in welchen das gegenwärtige
Geschlecht lebt und sich ergözt, denn sie suchten in allem, was sie taten,
\textit{"`durch den Glauben Gott wohlgefällig zu werden."'} Abraham verließ sein
väterliches Haus, seine Verwandten und sein Vaterland, und ward dadurch ein
wahres Vorbild und Muster der
Selbstüberwindung\footnote{\texttt{'Selbstverleugnung'
ersetzt durch 'Selbstüberwindung'}}, welche alle üben müssen, die
seine Nachfolger sein wollen. Diese dürfen nicht denken, in den Moden,
Gebräuchen und Vergnügungen fortleben zu wollen, welche sie zu verlassen
ermahnt und aufgefordert werden. Nein! Sie müssen allem
entsagen\index{entsagen!allem}, und, in der
\textit{"`Hoffnung einer reichen Belohnung und jenes besseren ewigen Landes im
Himmel,"'}\index{Jenseitige entlohnung}\footnote{Hebräer 11,26) 2. Korinther
5,1-2.}
\index{Bibelstellen:!Hebräer 11)}
\index{Bibelstellen:!Korinther 2 05@2. Korinther 5)}
von allem sich trennen. Die
Propheten\index{Prophet} waren größtenteils arme Handwerker und
Hirten.\footnote{Amos 7,14}
\index{Bibelstellen:!Amos 07@Amos 7)}
Sie riefen oft den übermütigen, zügellosen
Israeliten\index{Personen:!Israeliten} zu, dass sie
sich bekehren\index{bekehren},
den lebendigen Gott fürchten\index{Gott!fürchten} und scheuen, und ihre
Sünden\index{Sünde} und Eitelkeiten, in
denen sie lebten, verlassen sollten, aber nie ahmten sie die Torheiten
derselben nach. Johannes der Täufer\index{Personen:!Johannes der Täufer}, der
Vorläufer unseres
Herrn, der schon im
Mutterleib geheiligt\index{geheiligt} ward, verkündigte der Wert seine Botschaft
in einem
Gewand von Kamelhaaren, welches ohne Zweifel eine einfache und unansehnliche
Kleidung\index{Kleidung!unansehnliche} war.\footnote{Lukas 1,15) Matthäus
3,1-5.}
\index{Bibelstellen:!Lukas 01@Lukas 1)}
\index{Bibelstellen:!Matthäus 03@Matthäus 3)}
Wir können auch sicher annehmen,
dass Jesus Christus selbst ebenso einfach gekleidet ging, da er dem Fleische
nach von armer Herkunft\index{Herkunft!arme} war, und ein sehr einfaches
Leben\index{Leben!einfaches}
führte. Daher pflegte
man auch vorwurfsweise von ihm zu sagen:
\textit{"`Ist dieser nicht Jesus, der
Zimmerman, Marias Sohn?"'}\footnote{Markus 6,3) Matthäus 13,55}
\index{Bibelstellen:!Markus 06@Markus 6)}
\index{Bibelstellen:!Matthäus 13)}
Derselbe Jesus sagte
zu seinen Jüngern,
\textit{"`dass weiche und herrliche Kleider und ein üppiges Leben an
den Höfen der Könige anzutreffen wären;"'}\footnote{Lukas 7,25}
\index{Bibelstellen:!Lukas 07@Lukas 7)}
womit er sagte,
dass er und seine Nachfolger nach solchen Dingen nicht trachteten und zugleich
die große Verschiedenheit ausdrückte, die zwischen den Anhängern an den Moden
und Gebrauchen der Welt, und denen, die er aus der Welt erwählt hatte, statt
fand. Er erschien auch in einer so niedrigen und verächtlichen Gestalt, nicht
allein in der Absicht,
\textit{"`um den Stolz alles Fleisches zu demütigen,"'} er gab
auch dadurch seinen Nachfolgern ein Beispiel des Lebens der
Selbstüberwindung\footnote{\texttt{'Selbstverleugnung' ersetzt durch
'Selbstüberwindung'}},
welches sie führen müssten, wenn sie seine wahren Jünger sein wollten. Um ihnen
dieses desto tiefer einzudrücken, und es recht einleuchtend zu machen, wie
unverträglich eine prächtige und weltliche Lebensweise mit dem Reich sei, das
er zu stiften gekommen war und zu welchem er die Menschen einlud, hinterließ
er zur Belehrung für alle noch die treffende Erzählung vom reichen
Manne.\footnote{Lukas 16,19}
\index{Bibelstellen:!Lukas 16)}
Dieser wird nämlich zuerst als ein Reicher, dann als
ein Maßloser\footnote{\texttt{'Üppiger' ersetzt durch 'Maßloser'.}}, der sich
prächtig
kleidete und eine köstliche Tafel führte, und
endlich als ein Unbarmherziger geschildert, dem es weit wichtiger war und mehr
anlag, die Vergnügungen der Augenlust, der Fleischeslust und des hoffärtigen
Lebens zu geniessen, und alle Tage herrlich und in Freuden zu leben, als sich
des
armen Lazarus\index{Personen:!Lazarus} vor seiner Tür zu erbarmen. Ja, sogar
seine Hunde\index{Hunde}
bewiesen mehr
Mitleid\index{Mitleid} als er! Was war aber auch das Los dieses lustigen, großen
und reichen
Mannes? Wir lesen, dass er ewige Qual\index{Qual!ewige} leiden musste, und dass
hingegen Lazarus mit
Abraham\index{Personen:!Abraham}, Isaak\index{Personen:!Isaak} und
Jakob\index{Personen:!Jakob}
im Reiche Gottes\index{Reich!Gottes} ewige Freude teilte. Der eine war ein
guter, der andere ein großer Mann, jener arm und mäßig, dieser reich und
schwelgerisch. Von der letzteren Gattung gibt es aber heutigen Tages nur zu
viele, und es würde ein Glück für sie sein, wenn sie durch das Schicksal des
reichen Mannes sich zur Reue und Sinnesänderung erwecken ließen!

\section{10. Abschnitt} \label{kap14_ab10}

Auch waren die zwölf Apostel\index{Zwölf Apostel}, die unmittelbaren
Gesandten unseres Herren Jesu
Christi, nur arme Leute. Einer war ein Fischer\index{Fischer}, ein
anderer ein
Teppichmacher\index{Teppichmacher},
(Zeltmacher oder Sattler,) und einer, der noch das ansehnlichste, wiewohl
vielleicht nicht das beste Geschäft getrieben hatte, war ein
Zöllner\index{Zöllner}
gewesen.\footnote{Matthäus 4,18) Matthäus 9,9) Apostelgeschichte 18,3}
\index{Bibelstellen:!Matthäus 04@Matthäus 4)}
\index{Bibelstellen:!Matthäus 09@Matthäus 9)}
\index{Bibelstellen:!Apostelgeschichte 18)}
Es ist daher nicht wahrscheinlich, dass sie den
maßlosen\footnote{\texttt{'üppigen' ersetzt
durch 'maßlosen'}} Gebräuchen der Welt ergeben
gewesen waren. Sie waren in der Tat so weit davon entfernt, dass sie vielmehr,
wie es den Nachfolgern Christi geziehmte, ein mit Verachtung und Trübsal
verkürztes Leben der Selbstüberwindung\footnote{\texttt{'Selbstverleugnung'
ersetzt durch
'Selbstüberwindung'}} führten,\footnote{1. Korinther 4,9-14}
\index{Bibelstellen:!Korinther 1 04@1. Korinther 4)}
weshalb sie denn auch die Gemeinden ermahnen konnten,
\textit{"`so zu wandeln, wie sie ihnen zu Vorbildern
dienten."'}\footnote{Philipper 3,17}
\index{Bibelstellen:!Philipper 03@Philipper 3)}
So gaben sie auch von den heiligen
Frauen\index{Frau!heilige}\index{Personen:!Heilige!Frauen} der
Vorzeit
eine würdevolle Beschreibung, indem sie dieselben als Muster
der göttichen Genügsamkeit und Sittsamkeit empfahlen,
\textit{"`dem Schmuck nicht in
äußeren Zierraten, also Haarflächten, Goldumhänge oder schönen Kleider und
dergleichen,
sondern in dem verborgenen Menschen des Herzens und in einem sanften, stillen
Geiste bestand, der in den Augen Gottes köstlich ist."'}\footnote{1. Petrus
3,3+4.}
\index{Bibelstellen:!Petrus 1 03@1. Petrus 3)}
Von denen aber, die ein maßloses\footnote{\texttt{'üppiges' ersetzt durch
'maßloses'}} Leben
führten, behaupteten sie mit Recht,
\textit{"`dass sie lebendig tot wären,"'}\index{lebendig tot}\footnote{1.
Timotheus 5,6}
\index{Bibelstellen:!Timotheus 1 05@1. Timotheus 5)}
denn die Sorgen und
Ergötzungen dieses Lebens ersticken und zerstören den Samen des Reich
Gottes\index{Reich!Gottes!Samen des} im
Herzen und verhindern allen Fortgang in dem verborgenen göttlichen
Leben.\footnote{Lukas 8,14}
\index{Bibelstellen:!Lukas 08@Lukas 8)}
So sehen wir also, dass Vorzeiten die Gläubigen sich
nicht an die eitlen Vergnügungen und sogenannten Erhohlungen der Welt gewöhnt
hatten, sondern ihre Gemüter auf bessere Dinge im Himmel\index{Himmel}
richteten, und nach
einem Reiche trachteten,
"`dass in Gerechtigkeit, in Friede und in Freude im
heiligen Geist besteht."'\footnote{Römer 14,17}
\index{Bibelstellen:!Römer 14)}
Alle diese
"`haben ein gutes
Zeugnis bekommen, und sind in die ewige Ruhe eingegangen, wo ihre Werke ihnen
nachfolgen und sie selig preisen."'\index{Zeugnis}\footnote{Hebräer 4,9) Hebräer
11,2+14-16)
Offenbarung 14,13}
\index{Bibelstellen:!Hebräer 04@Hebräer 4)}
\index{Bibelstellen:!Hebräer 11)}
\index{Bibelstellen:!Offenbarung 14)}



