% STATUS=Unfertig
% FUSSNOTEN=1.Durchlauf
% ANFÜHRUNGSZEICHEN=
% KORSIVSCHRIFT=
% FETTSCHRIFT=
% RANDBEMERKUNGEN=
% MODERNE_RECHTSCHREIBUNG=noch nicht begonnen
% WORTERSÄTZUNG=
% VERSCHLAGWORTUNG(INDEX)=
% REFERENZIERUNG=abgeschlossen
% BIBELSTELLEN=ungeprüft


\chapter{14. Kapitel} \label{kap14}

\section{Zusammenfassung des 14. Kapitel}
\small
\begin{description}
\item[1. Abschnitt] Von der Üppigkeit; worin sie bestehet. und was für Nachteile
sie der menschlichen Gesellschaft bringt. - Sie ist eine Feindin des Kreuzes
Christi. (Seite: \pageref{kap14_ab1})
\item[2. Abschnitt] Von der Üppigkeit im Essen und Trinken; wie sehr die Selbe
dem Beispiele und den Lehren Christi und der heiligen Schrift zuwider ist. (Seite: \pageref{kap14_ab2})
\item[3. Abschnitt] Was für Schaden sie dem Menschen an Leib und Seele zufügt. (Seite: \pageref{kap14_ab3})
\item[4. Abschnitt] Von der Üppigkeit in Kleinen und Vergnügungen. Die Sünde
machte die die erste Kleidung notwendig. Der Mensch sollte daher nicht stolz auf
Etwas seyn, das ein Zeichen seines Falles ist. (Seite: \pageref{kap14_ab4})
\item[5. Abschnitt] Die jezt gebräuchlichen Vergnügungen schaden der Tugend, und
entspringen aus dem entarteten Zustande des Menschen. (Seite: \pageref{kap14_ab5})
\item[6. Abschnitt] Der Zweck der Kleidung ist erlaubt, aber der Mißbrauch
desselben zu tadeln. (Seite: \pageref{kap14_ab6})
\item[7. Abschnitt] Das vornemste Vergnügen guter Menschen der Vorzeit bestand,
darin, das sie Gott dienten, ihren Nebemmenschen Gutes erwiesen, und ehrlichen
Berufsgeschäften, aber nicht eitlen Vergnügungen und Belustigungen nachgingen. (Seite: \pageref{kap14_ab7})
\item[8. Abschnitt] Schon die Heiden kannten und thaten etwas Besseres. --
Einige wilde Völker übertreffen die Christentumsbekenner an Maßigkeit und
Nüchternheit des Lebens. (Seite: \pageref{kap14_ab8})
\item[9. Abschnitt] In dem Gleichnisse vom reichen Manne wird das üppige Leben
verdammt. (Seite: \pageref{kap14_ab9})
\item[10. Abschnitt] Die Lehren der heiligen Schrift sind durchaus dagegen. (Seite: \pageref{kap14_ab10})
\end{description}
\normalsize

\section{1. Abschnitt} \label{kap14_ab1}

Jetzt komme ich zur Betrachtung einer dem Geiste ganz entgegengesetzten
Leidenschaft: der Ueppigkeit, die in Befriedigung der Prachtliebe und eines
ausschweifenden Hanges zur Bequemlichkeit und zu Vergnügungen bestehet. Sie ist
das letzte Laster, welches ich in der gegenwärtigen Abhandlung üder daß heilige
Kreuz Christi angreife, dessen göttliche Kraft und Wirkung allein vermögend ist,
dieses eben so ansteckende als tödliche Übel zu zerstören und auszurotten. Es
ist eine Untugend, die sich in alle Stände einschleicht; welche die Aermsten
hinreißt, daß sie ihre Kräfte überschreiten, um ihre üppigen Neigungen zu
befriedigen, und die Reichen so sehr einnimmt, daß sie die innern Zuchtigungen
Jesu, den sie dennoch ihren Heiland nennen, ganz aus den Augen setzen, und sich
den Vergnügungen der Augenlust, der Fleischeslust und des hoffartigen Lebens so
sehr ergeben, als wenn nicht Selbstverleugnung und Unterwerfung unter daß Kreuz
Christi, sondern Befriedigung der Ueppigkeit, der uns verordnete Weg zum Himmel
wäre. Was wollen wir essen? Was wollen wir trinken? Wie wollen wir uns kleiden?
-- Diese Sorgen, wodurch einst die üppigen Heiden sich auszeichneten, sind jetzt
allgemein, ja, was noch schlimmer ist, das Dichten und Trachten der Bekenner des
Chriftenthumes geworden. Möchten sie doch darüber erröthen! Möchten sie ihre
Abweichung bereuen, und bedenken, daß Christus solche Sorgen an jenen Heiden
nicht darum tadelte, daß er sie seinen Nachfolgern erlauben wollte. Diejenigen,
welche sich in Wahrheit zu der Nachfolge Christi bekennen, müssen auch zeigen,
daß sie ihm angehören; sie müssen "`gesinnet seyn, wie Er war,"' ein nüchternes
und mäßigen Leben fuhren, und sich stets erinnern, "`daß der Herr nahe ist."'
Prächtige Kleider, schöne Putzsachen, kostbare Möbeln, eine reiche Tafel und die
gewöhnlichen Vergnügungen und Unterhaltungen der Welt, als: Schauspiele, Bälle,
Spiele, Romane, u. dgl. sind auf dem heiligen Pfade, den Jefus und seine wahren
Junger und Nachfolger betraten, nicht anzutreffen. Nein! Einer, der keiner der
Geringsten unter den Apesteln war, sagt uns, "`daß wir durch viele Trüdsale ins
Reich Gottes eingehen müssen."'\footnote{Ap. Gesch. 14,22} Es ist daher meine
ernstliche Bitte an Alle, die der Prachtliebe und Ueppigkeit ergeben sind, und
denen diese Abhandlung vielleicht in die Hände fallen möchte, daß sie die Grunde
und Beispiele, die ich gegen ihre Lebensweise hier anführe, ernstlich in
Erwägung ziehen wollen. Vielleicht können sie dadurch zu der Einsicht gelangen,
wie sehr ihr Leben und Betragen von dem wahren Christenthume entfernt und ihrem
ewigen Frieden gefährlich sei. Möge sie der allmächtige Gott für diesen
Unterricht empfänglich machen, und seine zertliche Liebe in ihre Herzen
ausgießen, damit sie zu einem Gefühle der Reue gebracht werden und den heiligen
Weg des Kreuzes Jesu, des Erlösers der gefallenen Menschheit, lieb gewinnen;
denn sie dürfen nicht glauben, daß sein Versönungstod ihnen Etwae nützen werde,
so lange sie sich weigern, aus Liebe zu ihm ihre Sünden zu verlassen, da er doch
aus Liebe zu ihnen sein Leben ließ; oder daß er ihnen eine Stätte im Himmel
bereiten werde, wenn sie ihm einen Platz in ihrem Herzen auf Erden versagen. Wir
wollen nun die Ueppigkeit in ihren verschiedenen Theilen betrachten.

\section{2. Abschnitt} \label{kap14_ab2}

Unter den verschiedenen Arten der Ueppigkeit betrifft die erste, die Christus,
der uns ein Muster der Selbstverleugnung gegeben hat, ausdrücklieh verbietet,
die Sorge für den Bauch. "`Ihr sollt nicht sorgen, und sagen: was werden wir
essen? oder was werden wir trinken? ..... Nach solchen Dingen trachten die
Heiden."'\footnote{Matth. 6 6,31. 32.} Das heißt: Jenen Heiden, die ohne
Erkenntniß des wahren Gottes in der Welt leben, die ihren Bauch zu ihrem Gott
machen, und deren Trachten nur auf die Befriedigung ihrer Begierden, aber nicht
auf die Erlangung des Reichen Gottes gerichtet ist, denen sollt ihr nicht gleich
seyn. "`Traehtet ihr zuerst nach dem Reiche Gottes und nach seiner
Gerechtigkeit, und alles Andere soll euch zugegeben werden."' Was für euch gut
ist, wird schon erfolgen. Laßt nur Alles seine rechte Zeit und Ordnung haben.

\medskip

In diesen Worten liegt ein scharfer Tadel für Alle die der Ueppigieit im Essen
und Trinken ergeben sind; deren ganze Sorge nur dahin gehet, wie sie ihren
Gaumen ergötzen und ihren Bauch füllen, was sie essen und trinken wollen. Diese
sind nicht selten verlegen, was für eine Speise sie für die nächste Mahlzeit
anordnen sollen; und Einige halten sich daher einen eigenen Speisemeister, um
neue Gerichte für den Koch zu erfinden, der sie so künstlich würzen, verändern
und zubereiten muß, daß Auge und Geschmack getäuscht werden, und man sie für
etwas ganz Neues und Sonderdares hält, bloß um die Eßlust zu reinen oder
Bewunderung zu erregen. Der Schwelger fordert in der That eine große
Verschiedenheit seltener und köstlicher Gerichte, wobei nicht selten die Brühe
weit hoher als die Speise selbst zu stehen kommt; denn er ist beständig so voll
und satt, daß er ohne Anwendung solcher künstlichen Reitzmittei gar keinen
Appetit haben wurde. Dieses heißt aber: den Hunger erzwingen, nicht: ihn
befriedigen. Und so, wie der Ueppige ißt, so trinkt er auch; selten aus Durst;
bloß zum Vergnügen und dem Gaumen zu Gefallen. Darum muß er auch eine
Verschiedenheit von Getränken haben, die er alle kosten muß. Eine Sorte, wäre
sie auch noch so gut, wurde ihm bald unschmackhaft und wiederlich werden. Die
Abwechslung muß ihm die Getränke behaglich machen; und daher ist die ganze Welt
kaum im Stande, seinen Keller mit kostbaren Getränken anzufüllen. Wäre er dabei
noch maßig im Genusse, so konnte man seine Abwechselung eher der Neugier als der
üppigen Schwelgerei Zusschreiben. Allein was der Maßige als Herzstärkung
genießt, trinkt er mit vollen Zügen, bie. er, durch Uebermaß erhitzt, zu
Thorheiten schreitet, und, wo nicht Andern, doch immer sich selbst, Schaden
zufügt; ja, oft so tief unter das unvernünftige Thier herabsinkt, daß er sich
selbst seines Bewußtseyns beraubt. Diesers ist die Wirkung der Fleischeslust,
"`die nicht vom Vater, sondern von der Welt ist,"' und in deren Gefolge Musik
und Tanz, Ausgelassenheit und Gelächter, -- welches der weise Mann Tollheit
nennt, -- gewöhnlich zur Hand sind, damit der rauschende Lärm der einen
Ergözlichkeit, das Gefühl der Sündhaftigkeit der andern üdertäuben, und die
Stimme des Gewissens nicht zu laut und deutlich mit dem Schwelger reden möge. So
leben die Ueppigen, die Gott vergessen und des Unglücklichen nicht gedenken. O!
möchten doch die Kinder der Menschen das Thörichte und Gottlose dieser Dinge
einsehen! Wie schlecht erwiedern sie die Güte Gottes durch einen solchen
Mißbrauch des Ueberflusses, den er ihnen verleihet! Wie grausam gehen sie oft
mit seinen Geschöpfen, wie verschwenderisch mit ihrem eigenen Leben und mit
ihren Kräften um, und wie undankbar sind sie für die Wohlthaten, die sie
genießen? Sie vergessen den Geber, indem sie seine Gaben zur Befriedigung ihrer
Lüste mißbrauchen, allen Rath verachten und jede Belehrung von sich stoßen.
Dadurch verlieren sie endlich alles Zartgefühl, und vergessen ihre Pflichten,
indem sie im Taumel sinnlicher Genüsse von einer Ausschweifung zur andern
übergehen.

\medskip

Gott rügte diese Sünden bei den Juden, als er durch den Propheten Amoe sagte:
"`Wehe euch! ..... die ihr euch weit vom bösen Tage achtet, (den bösen Tag weit
hinaussetzt,) und trachtet immer noch Frevel. Regiment! (oder: rücket den Stuhl
des Frevels oder der Gewaltthätigkeit herbei!) Ihr schlafet auf elfenbeinernen"'
 Lagern und treibet Ueberfluß mit eueren Betten! Ihr esset die Lämmer aus der
Heerde, und die gemästeten Kälber; und spielet auf dem Pfalter, (singet zu dem
Saitenspiele,) und erdichtet euch Lieder, (erfindet euch musikalische
Instromente,) wie David! Ihr trinket Wein aus den Schalen, und salbet euch mit
Balsam; aber ihr bekümmert euch nicht um den Schaden Josephs!"'\footnote{Amos
6,3. 4. 5. 6.} Dieses waren, wie es scheint, die Laster, denen die entarteten
Juden bei allen ihren Ansprüchen aus Religiosität ergeben waren; und sind nicht
heutiges Tages dieselben Uebel auch unter den Christen herrschend geworden? --
Allerdiugs! und sie machen einen großen Theil der Ueppigkeit aus, die ich in
dieser Abhandlung angreife. Denk an den reichen Mann, der mit seiner ganzen
Pracht und Herrlichkeit, und bei allen seinen hohen Genüssen verloren ging; und
erinnere dich, was für ein schweres Wehe der Apostel über Diejenigen ausspricht,
"`deren Bauch ihr Gott ist, und die in ihrer Schande eine Ehre
suchen"'\footnote{Phil. 5,19}

\medskip

Christus redet von der Ueppigkeit als von Etwas, dar an den Höfen weltlicher
Könige anzutreffen ist, in seinem Reiche aber nicht Statt findet, und seinen
Nachsotgern nicht geziemet. Daher war auch sein Fest, das er der Menge gab, und
welches eins seiner Wunder war, ganz einfach und ohne Vorbereitung. Er gab den
Leuten reichlich, aber nichts Seltenes, oder Etwas das durch die Kochkunst
zubereitet war; dennoch aßen sie es gern, weil sie Hunger hatten, welcher auch
immer das beste und schiklichste Mittel zur Beförderung der Eßlust ist. -- In
einer Anweisung, die Paulus seinem geliebten Timotheus ertheilte, setzt er die
Liebhaber des weltlichen Ueberflusses sehr herab, indem er behauptet, "`daß
Gottseligkeit mit Genügsamkeit verbunden, großer Gewinn sei,"' und noch
hinzufügt, "`daß wir, wenn wir Nahrung und Kleider haben, uns begnügen
sollen."'\footnote{1 Tim 6,6.-11.} Dieses giebt uns einen Begriff von dem
enthaltsamen und genügsamen Leben jener erhabenen Pilger, jener Söhne des
Himmels und wahren Abkömmlinge der ewigen Kraft Gottes, die sich oft in der
Nothwendigkeit befanden, zu fasten, und Gefahren entgegen zu gehen; sich aber
mit Dem begnügten, "`was ihnen vorgesetzt wurde."' O! heilige Menschen! O!
selige Geister! Möge meine Seele ewig bei den eurigen wohnen!

\section{3. Abschnitt} \label{kap14_ab3}

Die Krankheiten, welche die Ueppigkeit erzeugt und nähret, machen sie zu einer
wahren Feindin des Menschengeschlechts. Denn außer dem Nachtheile den sie der
Seele des Menschen zufügt, untergrabt sie auch seine Gesundheit und verkürzt ihm
das Leben, indem sie ihm nur elende Nahrung reicht, wodurch schlechte Säfte
erzeugt werden, die seinen Körper ungesund, träge, skorbutisch <!-- gemeint ist
wohl die Krankheit de.wikipedia.org/wiki/Skorbut --> und für ein thätiges,
arbeitsames Leben, ganz untauglich machen. Und sind einmal die Lebensgeister des
Menschen von seinem elenden Fleisehe unterdrückt, und seine Seelenkrafte
erschlafft, so macht sein Hang zur Unthätigkeit ihn für die bürgerliche
Gesellschaft ganz unbrauchbar; denn die Ueppigkeit macht die Menschen nicht
allein krank, sondern auch zu Mußiggängern, die alles Gute verschlingen, und
sich selbst so sehe lieben, daß sie Gott vergessen. Diese sind wirklich der Welt
zur Last, und die Schrift sagt uns, -- was eben so schreklich als gerecht ist,
-- "`daß Diejenigen, welche Gott vergessen, ins Verderben gehen
sollen."'\footnote{Ps. 9,18}

\section{4. Abschnitt} \label{kap14_ab4}

Es giebt aber noch eine andere Art der Üppigkeit, die bei eitlen Personen beides
Geschlechtes häufig angetroffen wird, nämlich die Kleiderpracht; eine der
größten Thorheiten, welcher der Mensch sich schuldig machen kann; da sie eine
der kostbarsten und zugleich nichtigsten und unnüzesten Ausschweifungen ist. Die
Schrift der Warheit lehret uns glauben, daß die Sünde die erste Kleidung in der
Welt notwendig machte;\footnote{1 Mos. 3,21} und wenn die Uebereinstimmung
mehrerer Schriftsteller einiges Gewicht hat, so erstreckte sich der Fall des
Menschen sowohl ausf sein Aeußeres als auf sein Inneres. Ich richte daher diese
Abhandlung an Diejenigen welche dieses glauben; da diese, wie ich hoffe, die
grössere Anzahl meiner Leser ausmachen. -- Ich sage, wenn die Sünde zuerst die
Kleidung veranlaßte, so haben die armen Adam’s Abkömmlinge wenig Ursache, auf
ihre Kleidung stolz zu sein, oder sich darin zu suchen; da ihr Ursprung so
niedrig ist, und der Schmuck derselben weder den Menschen veredelt: noch zu
seiner Unschuld zurückführen kann. Selig war aber gewiß jene Zeit, als noch
Unschuld, nicht Unwissenheit, Kleider unnötig machte. Damals waren unsere ersten
Eltern natkend, und kannten keine Scham, bis die Sünde sie beschämt machte,
länger nackend zu bleiben. Da also das Bewußtseyn der Schuld die Scham erzeugte,
und diese eine Bedeckung erforderte, wie tief sind denn Diejenigen gefallen, die
in ihrer Schande sich brüsten und auf ihren Fall stolz thun? Dieses thun
unstreitig aber Alle, die Mühe und Kosten anwenden, ihre Kleidung, das wahre
Kennzeichen jenes beklagenswerthen Falles, zu schmücken und herauszuputzen. Und
ist es also wohl bei Denen, die sich Christen nennen, zu entschuldigen, wenn sie
eine so große Vorliebe zu prächtigen Kleidern hegen, daß sie die erste und
vornehnrste Bekleidung des Menschen, das Kleid der Unschuld, darüber
vernachlässigen? Beweiset es nicht die größte Thorheit der Menschen, wenn sie so
viel Zeit, Mühe und Geld anwenden, um ihre Schande recht zur Schau zu stellen?
Heißt dieses nicht Vergnügen an den Folgen einer Sache finden, deren Entstehung
man bejammern sollte? Ja! wenn ein Verbrecher verurtheilt wäre, lebenslänglich
in Ketten zu gehen, würde es wohl seine Schande vermindern, wenn seine Ketten
von Gold gemacht und sehr künstlich gearbeitet wären? Gewiß, dieser Umstand
würde die Folgen seines Verbrechens nur noch auffallender machen und seine
Schande vermehren. Dieses ist nun aber gerade der Fall mit den Sklaven der
eitlen Moden unsere Zeitalters, die dessenungeachtet Christen, Veurtheiler
religiöser Gegenstände, fromme Leute, und wer weiß was noch mehr, seyn wollen.
O! des dejammernswerthen Zustandes, der Menschen, welche die Angenlust, die
Fleischeslust und das hoffartige Leben so sehr verblendet hat, daß sie sich mit
Dingen zieren und schminken, die ihren Fall beweisen, und sogar erfinderisch und
verschwenderisch in diesen Dingen sind, die ihnen doch eigentlich zur
Deinüthigung dienen sollten! Ihre große Anhänglichkeit an diese Eitelkeiten
zeigt offenbar, wieweit sie von der ersten Unschuld abgewichen sind. Und wem ist
nicht bekannt, wie viele Moden schon erfunden sind, und noch immer erdacht
werden, die keinen andern Zweck haben, als Lüsternheit zu erregen, wodurch das
Herz des Menschen immer mehr von der Einfall und Unschuld entfernt und in die
Sklaverei niedriger Sinnlichkeit geführt wird.

\section{5. Abschnitt} \label{kap14_ab5}

Ebenso verhält es sich mit ihren Vergnügungen oder Erhohlungen, wie man sie
nennt; denn diese stehen mit der Ueppigkeit in Kleidern in naher Verbindung. Der
Mensch ward zu einem edlen, vernünftigen und ernsten Wesen erschaffen. Sein
Vergnügen bestand in der Erfüllung seiner Pflicht; und diese war: Gott zu
gehorchen, d. h. ihn zu lieben, zu fürchten, zu verehren und anzubeten, und
seine Geschöpfe mit wahrer Mäßigkeit und heiliger Genügsamkeit zu gebrauchen,
indem er wohl wußte, daß der Herr, sein Richter, nahe war, der seine Handlungen
beobachtete, und vergelten würde. Kurz, seine Glückseligkeit bestand in der
Gemeinschaft mit Gott; und sein Fehler war, daß er diese selige Gemeinschaft
verließ, und mit seinem Gemnüthe vergänglichen Dingen nachhing. Waren nun aber
die Ergötzungen der gegenwärtigen Welt wirklich so angenehm und nothwendig, als
man vorgiebt, so würde es ja Adam und Eva in dem Stande ihrer Glückseligkeit an
der Freude, die sie gewähren, gemangelt haben; da sie dieselben gar nicht
kannten. Allein es verhält sich anders. Diese Vergnügungen sind das nicht, was
sie zu seyn scheinen; und wenn unsere ersten Eltern nicht gefallen wären, und
die Welt mit ihrer Thorheit, und mit ihrem bösen Beispiele nicht angesteckt
hätten, so würden wir warscheinlich den Gebrauch und die Nothwendigkeit vieler
solcher Vergnügungen nicht kennen, die, ebensowohl als die Kleiderpracht, aus
der Sünde entsprungen sind. Denn, als Adam und Eva gesündigt hatten, fürchtetetn
sie die Gegenwart des Herrn, die in ihrer Unschuld sie erfreuete; und nun ließen
sie ihre Gemüther auf andere Gegenstände ausschweifen, suchten andere
Vergnügungen, und fingen an, Gott, ihren Schöpfer, zu vergessen. Daher klagte
Gott hernach durch den Propheten Amon: "`Sie achten sich weit vom bösen Tage;
sie essen die Lämmer aus der Heerde; trinken Wein aus Schalen; salben sich mit
Balsam; schlafen auf elfendeinernen Lagern; spielen auf dem Pfalter; erdichten
sich Lieder, wie David, und bekümmern sich nicht um den Schaden
Josephs."'\footnote{Amos 6, 3. 4. 5. 6.} Sie verkauften Joseph auf eine gottlose
Weise, verbanneten die Unschuld gänzlich, und gewöhnten sich bald an die Schande
so sehe, daß sie über die Nachahmung schändlicher Handlungen gar nicht mehr
errötheten. Und jetzt schämen sich Menschen, die wahrlich eben so sehr, sich der
ursprünglichen Unschuld und bescheidenen Einfachheit wieder zu nähern, als Adam
sich schämte, da er diese Tugenden verloren hatte, und deswegen, -- wiewo1
vergeblich, -- seine Zuflucht zu einer Bedeckung von Feigenblättern nahm. Eben
so vergeblich suchen jetzt die Menschen mit scheinbaren Ansprüchen auf Religion
sich zu bedecken, und ihren armen Seelen durch Zueignung der schönen
Benennungen: christlich, unschuldig, gütig, tugendhaft, u. s. w. zu schmeicheln,
während doch Eitelkeit und Thorheit die Herrschaft über sie haben.

\medskip

Darum fühle ich mich von dem ewigen Gott verpflichtet, allen Solchen zu sagen:
Ihr verspottet ''Ihn,'' "`sich nicht will spotten lassen,"'\footnote{Gal. 6,7}
und betrüget euch selbst! Solche Unmäßigkeit, als unter euch herrscht, muß
verleugnet werden. Ihr müßt eine Veränderung eurer Gesinnungen erfahren und euch
mehr der ursprünglichen Reinheit des Christenthumes nähern, ehe ihr ein Recht zu
jenen Benennungen haben könnt, die ihr euch jetzt nur anmaßet. Denn "`die wahren
Kinder Gottes bestehen nur aus Solchen, die vom Geiste Gottes geleitet werden,
der zur Mäßigkeit, Sanftmuth, Keuschheit und allen andern Tugenden
führet."'\footnote{Röm. 8,14}

\section{6. Abschnitt} \label{kap14_ab6}

Die christliche Welt, oder vielmehr die weltlichen Christen, verdienen aber auch
deswegen billigen Tadel, daß sie den rechten Zweck, wozu die Kleidung zuerst
eingeführt wurde, ganz verkehret haben. Der erste Dienst den die Kleider dem
Menschen leisten sollten, als die Sünde ihn seiner angebornen Unschuld beraubt
hatte, war, wie schon bemerkt ist, Bedeekung; und sie mußten schon in dieser
Hinsicht einfach und anspruchslos seyn. Ihr nächster Zweck war, ihn gegen Kälte
zu schützen; daher mußten sie Starke und Festigkeit haben. Und endlich sollten
sie auch zur Unterscheidung der beiden Geschlechter dienen, und mußten folglich
in dieser Rücksicht für beide verschieden seyn. Statt daß aber damals das
Bedürfniß die nothwendige Bekleidung erforderte, macht jetzt der Stolz und die
eitle Prachtliebe eine Menge Kleider nothwendig; statt daß sie damals nur zum
Nutzen des Menschen gebraucht wurden, dienen sie jetzt nur, seine Eitelkeit und
Gefallsucht zu befriedigen. Anfänglich brauchte man die Kleider wirklich zur
Bedeckung; jetzt macht diese den kleinsten Theil ihres Zwecken aus. Das
unersättliche Auge der eitlen Menschen verlangt prachtvoll Ueberflüffigkeiten;
als ob ihre Kleider bloß der anzubringenden Verzierungen wegen gemacht werden
müßten, und nur zur Schau, aber nicht zum Tragen bestimmt wären. In wiefern sie
die Blöße bedecken, gegen Kälte schützen und der Anstäudigkeit des Geschlechtes
entsprechen, darum scheint man in der That sich wenig mehr zu bekümmern; indem
sie nur zu augenscheinlich so eingerichtet werden, daß sie den Stolz, die
Eitelkeit und Lüsternheit Derer, die sie tragen, verrathen.

\section{7. Abschnitt} \label{kap14_ab7}

Vorzeiten bestanden die besten Erhohlungen der Menschen darin, daß sie Gott
dienten, Gerechtigkeit übten, ihren Berufsgeschäften nachgingen, ihre Heerden
weideten, Anderen Gutes thaten und körperliche uebungen vornahmen, die mit der
Mäßigkeit und Tugend übereinstimmten und ihrer Ernsthaftigkeit und Wurde
angemessen waren. Jetzt wird der Name ''Erholung'' fast jeder Thorheit
beigelegt, die nur einigermaßen über offenbar schändliche Ausschweifungen
erhaben ist, wiewohl selbst Diejenigen, die sie begehen. darüber erröthen
müssen, wenn sie bedenken, was sie gethan haben. So weit sind die Menschen sogar
von dem Zustande, in welchem Adam nach feinem Ungehorsame sich befand, entartet
und abgewichen! So viel Vertrauter sind sie mit allen Lastern und so viel klüger
und geschickter in ihrer Ausübung geworden! Ja, so unempfindlich hat endlich die
Gewohnheit ihre Gemüther gegen den Zwang und die Unbequemlichkeit gemacht, die
fast beständig mit der Kleiderpracht verbunden sind, daß diese Mittel zur
Bedeckung der Blöße des Menschen, die einst die Nothwendigkeit einführte, jetzt
die Sorge, das Vergnügen, die Ergötzung, ja, das. Dichten und Trachten aller
Stände geworden sind. -- Ader wie unedel, wie schimpflich, wie unwürdig des
Charakters eines vernünftigen Wesens ist dieses! Wie kann der mit Verstand
begabte, des Nachdenkens über Unsterblichkeit fähige Mensch, der den Engeln
gleich, wo nicht noch höher, geachtet ist, -- wie kann der denkende Mensch einen
solchen Werth auf so nichtswürdige Dinge, auf einige ärmliche Lampen legen, die
bloß der Stolz erfunden und die Ueppigkeit eingeführt hat? Die unbedeutendsten
Spielsachen kleiner Kinder können keine so lächerliche und unwürdige
Beschäftigung für Erwachsene abgeben, als die thörichten Erfindungen der Mode
für verständige Menschen seyn würden, wenn sie dieselben als Gegenstände ihrer
Sorge und ihres Vergnügens betrachtete wollten. Und es verrät in der That immer
große Beschränktheit des Verstandes, wenn solche Eitelkeiten das edle Gemütn des
Menschen beschäftigen, der das Edenbild des großen Schöpfers des Himmels und der
Erde ist.

\section{8. Abschnitt} \label{kap14_ab8}

Hiervon hatten schon viele der alten Heiden so klare Begriffe, daß sie alle
solche eitle Dinge verabscheueten, und sowohl die Kleiderpracht, als auch die
jetzt unter den Bekennern des Christentums üblichen und so sehr beliebten
Vergnügungen und Erhohlungen, als sittenverderbende Uebel betrachteten; weil sie
die Gemüther der Menschen von der Nüchternheit und Mäßigkeit zur Ueppigkeit, zum
Müßiggange und zur Weichlichkeit verleiteten, und sie endlich unter das
vernunftlose Thier herabsezten. Davon zeugen: naragoras, Sokrates, Plato-
Aristides, Cato, Seneca, Epietet, u. a. welche der Meinung waren, daß nur in
Tugend und Unsterblichkeit wahre Ehre und wahrer Genuß in finden sei. Selbst
unter einigen Negern und Indianern findet man noch jetzt unverkennbare Spuren
von Unschuld und Reinheit der Sitten. Sie beobachten bei ihrem Handel ein
einfachen und gerades Benehmen, und wenn ein Christ, -- der freilich wohl von
einer besondern Art seyn muß, -- ein schmutziges Wort ausstößt, so haben sie den
Gebrauch , daß sie ihm, um ihn seine Unschicklichkeit fühlen zu lassen, ein
Gefäß mit Wasser reichen, damit er seinen Mund reinigen möge. Wie sehr klagen
nicht solche Tugenden und vernünftige Beispiele Diejenigen, die sich zum
Christentum bekennen, der gröbsten thorheit und Zugellosigkeit an! O! Möchten
doch die Menschen die Furcht Gottes vor Augen haben! Möchten sie so viel Mitleid
mit sich selbst haben, das; sie dedächten, welches Ursprunges sie sind, was sie
thun und treiben, und wohin sie zurckkehren müssen; damit sie edlere,
tugendhaftere, vernünftigere und himmlische Dinge zu Gegenständen ihrer Freude
und Erholung wählten, und einmal überzeugt würden, wie unverträglich die
Thorheiten, Eitelkeiten und Vergnügungen, denen sie sich gröstenteils ergeben
haben, mit dem wahren Adel vernünftiger Wesen sind! Ja, mögen sie endlich dem
göttlichen Einflusse, der die tugendhaften Heiden leitete, Sie Gehör geben;
damit es nicht diesen am Tage des Gerichte erträglicher als ihnen erginge! Denn
wenn Jene, von geringer Einsicht und unvollkommender Erkenntniß, schon so viele
eitle Dinge entdeckten und einsetzen; wenn ihr geringeres Maß von Licht
dieselben verwarf, und sie, aus Gehorsam gegen dasselbe, sich von ihnen
losrissen; was wird denn nicht von den Christen erwartet? -- Christus kam nicht,
um diese Erkenntniß auszulöschen, sondern, sie zu vermehren; und wenn daher
Jemand denkt, er brauche es so genau nicht zu nehmen, als Jene, der hat gewiß
nöthig, besser zu handeln als er denkt. Ich behaupte, daß die Moden und
Vergnügungeu, die man jetzt so sehr erhebt, dem Zwecke der Schöpfung des
Menschen sehr entgegen wirken, und daß die Untugenden, die sie begleiten,
nämlich: Ausschweifung, Müßiggang, Verschwendung , Stolz, Lüsternheit, Ansehen
der Person um der prächtigen Kleider willen, und mehrere solcher schlechten
Früchte, sowohl der Pflicht und Vernunft, als auch dem wahren Vergnügen des
Menschen zuwider sind, und mit der Weisheit, Erkenntniß, Selbstständigkeit,
Mäßigkeit und Betriebsamkeit, welche ihn wahrhaft edel und gut machen, aus keine
Weise sich vereinigen lassen.

\section{9. Abschnitt} \label{kap14_ab9}

Auch haben die heiligen Menschen der Vorzeit, welche und die Schrift als würdige
Muster zur Nachahmung empfiehlt, die hier getadelten Dinge weder gestattet noch
selbst gebraucht. Abraham, Isaak und Jakob, waren Männer von einfachen Sitten;
Hirten, aber doch Fürsten in ihren Familien und über ihre Heerden. Sie
bekümmerten sich nicht um die eitlen Dinge, in welchen das gegenwärtige
Geschlecht lebt und sich ergözt; denn sie suchten in Allem, was sie thaten,
"`durch den Glauben Gott wohlgefällig zu werden."' Abraham verließ sein
väterliches Haus, seine Verwandte und sein Vaterland, und ward dadurch ein
wahres Vorbild und Muster der Selbstverleugnung, welche alle üben müssen, die
seine Nachfolger seyn wollen. Diese dürfen nicht denken, in den Moden,
Gebräuchen und Vergnügungen fortleben zu wollen, welche sie zu verlassen
ermahnet und aufgefordert werden. Nein! sie müssen Allem entsagen, und, in der
"`Hoffnung einer reichen Belohnung und jenes bessern ewigen Landes im
Himmel,"'\footnote{Hebe. 11,26. Koor. 5,1. 2.} von Allem sich trennen. Die
Propheten waren größtenteils arme Handwerker und Hirten.\footnote{Amos 7,14} Sie
riefen oft den übermüthigen, zügellosen Israeliten zu, daß sie sich bekehren,
den lebendigen Gott fürchten und scheuen, und ihre Sünden und Eitelkeiten, in
denen sie lebten, verlassen sollten; aber nie ahmten sie die Thorheiten
derselben nach. Johannes der Täufer, der Vorläufer unseres Herrn, der schon in
Mutterleibe geheiligt ward, verkündigte der Wert seine Botschaft in einem
Gewande von Kameelhaaren, welchers ohne Zweifel eine einfache und unansehnliche
Kleidung war.\footnote{Luk. 1,15 Matth. 3,1-5.} Wir können auch sicher annehmen,
daß Jesus Christus selbst eben so einfach gekleidet ging; da er dem Fleische
nach von armer Abkunft war, und ein sehr einfaches Leben führte. Daher pflegte
man auch vorwurfsweise von ihm zu sagen: "`Ist dieser nicht Jesus, der
Zimmerman, Maria Sohn?"'\footnote{Mark. 6,3 Matth. 13,55} Derselbe Jesus sagte
zu seinen Jüngern, "`daß weiche und herrliche Kleider und ein üppiges Leben an
den Höfen der Könige anzutreffen wären;"'\footnote{Luk. 7,25} womit er sagte,
daß er und seine Nachfolger nach solchen Dingen nicht trachteten, und zugleich
die große Verschiedenheit ausdrückte, die zwischen den Anhängern an den Moden
und Gebrauchen der Welt, und Denen, die er aus der Welt erwählet hatte, Statt
fand. Er erschien auch in einer so niedrigen und verächtlichen Gestalt, nicht
allein in der Absicht, "`um den Stolz alles Fleisches zu demüthigen;"' er gab
auch dadurch seinen Nachfolgern ein Beispiel des Lebens der Selbstverleugnung,
welches sie führen müßten, wenn sie seine wahren Jünger seyn wollten. Um ihnen
Dieses desto tiefer einzudrücken, und es recht einleuchtend zu machen, wie
unverträglich eine prächtige und weltliche Lebenensweise mit dem Reiche sei, das
er zu stiften gekommen war, und zu welchen: er die Menschen einlud, hinterließ
er zur Belehrung für Alle noch die treffende Erzählung vom reichen
Manne.\footnote{Luk. 16,19} Dieser wird nämlich zuerst als ein Reicher, dann als
ein Ueppiger, der sich prächtig kleidete und eine köstliche Tafel führte, und
endlich als ein Unbarmherziger geschildert, dem es weit wichtiger war und mehr
anlag, die Vergnügungen der Augenlust, der Fleischeslust und des hofartigem
Lebens zu genießen, und alle Tage herrlich und in Freuden zu leben, als sich des
armen Lazarus vor seiner Thür zu erbarmen. Ja, sogar seine Hunde bewiesen mehr
Mitleid, als er! Was war aber auch das Loos dieses lustigen, großen und reichen
Mannes? Wir lesen, daß er ewige Qual leiden mußte, und daß hingegen Lazarus mit
Abraham, Isaak und Jakob im Reiche Gottes ewige Freude theilte. Der Eine war ein
guter, der Andere ein großer Mann; Jener arm und mäßig, Dieser reich und
schwelgerisch. Von der letztern Gattung giebt es aber heutiges Tages nur zu
Viele, und es würde ein Glück für sie seyn, wenn sie durch das Schicksal des
reichen Mannes sich zur Reue und Sinnesänderung erwecken ließen!

\section{10. Abschnitt} \label{kap14_ab10}

Auch waren die zwölf Apostel, die unmittelbaren Gesandten unsers Herrn Jesu
Christi, nur arme Leute. Einer war ein Fischer, ein Anderer ein Teppichmacher,
(Zeltmacher oder Sattler,) und einer, der noch das ansehnlichste, wiewohl
vielleicht nicht das beste Geschäft getrieben hatte, war ein Zöllner
gewesen.\footnote{Matth. 4,18 Matth. 9,9. Apost. Geschicht. 18,3} Es ist daher
schon nicht wahrscheinlich, daß sie den üppigen Gebranchen der Welt ergeben
gewesen waren. Sie waren in der That so weit davon entfernt, daß sie vielmehr,
wie es den Nachfolgern Christi gezieemte, ein mit Verachtung und Trübsal
verkürztes Leben der Selbstverleumdung führten;\footnote{Kor. 4,9.-14.} weshalb
sie denn auch die Gemeinen ermahnen konnten, "`so zu wandeln, wie sie ihnen zu
Vorbildern dienten."'\footnote{Phil. 3,17} So gaben sie auch von den heiligen
Weibern der Vorzelt eine würdefolle Beschreibung, indem sie dieselben als Muster
der gottiechen Genügsamkeit und Sittsamkeit empfahlen, "`dem Schmuck nicht in
äußern Zierrathen, als: Haarflächten, Goldumhänge, oder schönen Kleidern u.dgl.;
sondern in dem. Verborgenen Menschen des Hetzens und in einem sanften, stillen
Geiste bestand, der in den Augen Gottes köstlich ist."'\footnote{1 Petr. 3,3.
4.} Von Denen aber, die ein üppiges Leben führten, behaupteten sie mit Recht,
"`daß sie lebendig todt wären;"'\footnote{1 Tim. 5,6} denn die Sorgen und
Ergöhzungen dieses Lebens ersticken und zerstören den Samen des Reichs Gottes im
Herzen und verhindern allen Fortgang in dem verborgenen göttlichen
Leben.\footnote{Luk. 8,14} So sehen wir also, daß vorzeiten die Gläubigen sich
nicht an die eitlen Vergnügungen und sogenannten Erhohlungen der Welt gewöhnt
hatten, sondern ihre Gemüther auf bessere Dinge im Himmel richteten, und nach
einem Reiche trachteten, "`daß in Gerechtigkeit, in Frieden und in Freude im
heiligen Geiste bestehet."'\footnote{Röm 14,17} Alle diese "`haben ein gutes
Zeugniß bekommen, und sind in die ewige Ruhe eingegangen, wo ihre Werke ihnen
nachfolgen und sie selig preisen."'\footnote{Ebr. 4,9 Ebr. 11,2. 14. 15. 16.
Offenb. 14,13}

