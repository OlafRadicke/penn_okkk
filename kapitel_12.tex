
\chapter{Kapitel 12} \label{kap12}

\section{Zusammenfassung des 12. Kapitels}

\begin{description}
\item[1. Abschnitt] Charakter des Stolzen. Seine Eigenliebe ist unersättlich,
er
ist stolz auf seine Geburt.
\dotfill \textit{Seite~\pageref{kap12_ab1}}\\
\item[2. Abschnitt] Er ist trozig und zänkisch, aber feige, und doch grausam.
\dotfill \textit{Seite~\pageref{kap12_ab2}}\\
\item[3. Abschnitt] Ein eben so schlechter Sohn als schlechter Untertan und
Diener.
\dotfill \textit{Seite~\pageref{kap12_ab3}}\\
\item[4. Abschnitt] Ein Feind der Gastfreundschaft.
\dotfill \textit{Seite~\pageref{kap12_ab4}}\\
\item[5. Abschnitt] Keines Menschen Freund.
\dotfill \textit{Seite~\pageref{kap12_ab5}}\\
\item[6. Abschnitt] Als Gewalthaber gefährlich und schädlich.
\dotfill \textit{Seite~\pageref{kap12_ab6}}\\
\item[7. Abschnitt] Vor allen anderen sind stolze Prediger zu tadeln.
\dotfill \textit{Seite~\pageref{kap12_ab7}}\\
\item[8. Abschnitt] Sie verlangen Vorrechte vor anderen,
\dotfill \textit{Seite~\pageref{kap12_ab8}}\\
\item[9. Abschnitt] Sie nennen sich die
Geistlichkeit\footnote{\texttt{'Klerisei' erstetzt durch 'Geistlichkeit'.}}, --
ihre Herrschsucht und ihr Geiz.
\dotfill \textit{Seite~\pageref{kap12_ab9}}\\
\item[10. Abschnitt] Der Tod verschlingt alle.
\dotfill \textit{Seite~\pageref{kap12_ab10}}\\
\item[11. Abschnitt] Mittel, den bösen Folgen des Stolzes zu entgehen.
\dotfill \textit{Seite~\pageref{kap12_ab11}}\\
\end{description}

\newpage

\section{1. Abschnitt} \label{kap12_ab1}

\index{Charakter!Stolzer} Um endlich diese lange Abhandlung über den Stolz
zu beschließen, wollen wir noch
in der Kürze untersuchen, worin im Ganzen genommen der Charakter eines stolzen
Menschen an sich und in Beziehung auf andere besteht. Der Stolze ist eine Art
des
selbstsüchtigen Schwelgens, denn er wird nie satt, sich selbst zu lieben und zu
bewundern, während alles andere in seinen Augen weder Liebe noch Bewunderung
verdient. \index{Unterordnung} \index{Demut!Unfähigkeit zur} \index{Gott!
sich unterordnen} \label{ref:12_01_egoisten}
\textbf{Das Verdienst, welchen er allenfalls anderen
Gegenständen noch
einräumt, besteht bloß darin, dass sie seinen Zwecken dienen, als ob alles nur
für ihn geschaffen, oder vielmehr, als wenn er sein eigener Schöpfer wäre. So
wie er daher andere Menschen deswegen verachtet, weil er seines Gleichen nicht
dulden mag, so liebt er auch Gott nicht, weil er keinen Höheren über sich haben
will. Der Gedanke, sein Dasein einem anderen zuzuschreiben, ist ihm
unerträglich,
da dieser ihn in die Notwendigkeit versetzt, ein höheres Wesen anzuerkennen.}
Er
ist stolz auf die Ehre seiner Vorfahren, aber nicht auf die Tugenden, durch
welche sie dazu gelangten, auch gibt er sich nicht die geringste Mühe, ihnen
darin nachzuahmen. Seine Erzählungen von seinen Geschlechtsregister, von der
uralten Herkunft, von den Besitzungen und Verbindungen seiner Vorfahren nehmen
kein Ende, aber er vergisst, dass sie nicht mehr sind und dass auch er sterben
muss.

\section{2. Abschnitt} \label{kap12_ab2}

Wer ist wohl lästiger in der Gesellschaft, als der Stolze?
Er hat an jede Kleinigkeit etwas auszusetzen und ist
herablassend\footnote{\texttt{Ursprünglich lautete der Satz: 'Er bekritelt jede
Kleinigkeit und spricht gebieterisch über alles.'}}.
\label{ref:12_02_eitle_menschen_streit}
\textbf{Gibt man ihm nicht nach,
so wird er beleidigend und zänkisch, doch wenn es aufs Äußerste ankommt, so
zeigt
er sich feige\index{Feigheit}, aber grausam, sobald er sieht, dass er die
Obermacht hat.} Für das
Elend hat er kein Mitgefühl\index{Mitgefühl!kein}, als wäre es unter seiner
Würde, gefühlvoll zu sein. \label{ref:12_02_eitle_menschen_mitgefuehl}
\textbf{Das Unglück eines anderen rührt ihn so wenig, als wenn er selbst kein
Mensch
wäre, oder als hielte er Mitleid für eine Sünde\index{Sünde}. Was nicht gerade
ihn angeht,
fesselt auch seine Aufmerksamkeit nicht. Er will sich über das Unglück anderer
keine unruhigen Gedanken machen, ihm genügt die Überzeugung, dass sie es
verdient haben, und er möchte es ihnen lieber gerade heraus sagen, dass es ihre
eigene Schuld\index{Schuld} sei, als sich bereit finden lassen, sie zu bedauern
oder ihnen zu
helfen\index{Helfen}.} Daher scheinen ihm Mitleid und Wohlthätigkeit eben so
überflüssig zu
sein, als Demut\index{Demut} und Sanftmut ihm verhasst sind.

\section{3. Abschnitt} \label{kap12_ab3}

Der Stolze ist sowohl ein schlechter Sohn\index{Personen:!Sohn}, als schlechter
Diener\index{Personen:!Diener} und Untertan\index{Personen:!Untertan}.
Er verachtet seine Eltern\index{Personen:!Eltern}, seine Herren und seinen
Fürsten. Sich
zu unterwerfen\index{Unterordnung}
ist ihm unerträglich. \label{ref:12_03_eitle_menschen_ehe}
\textbf{Auch dünkt er sich zu weise, oder hält sich für zu alt, um
sich vorschreiben zu lassen, als wenn Gehorsam Sklaverei
\index{Personen:!Sklaven} wäre,
und Freiheit\index{Freiheit}
darin bestände, dass man tun dürfe, was man wolle,}welches jedoch alle Pflicht
aufheben und alles Ansehn herabsetzen würde. - \textbf{Ist der Stolze
verheiratet\index{Ehe}, ist
er Vater oder Herr, so ist es fast unmöglich, es bei ihm auszuhalten. Er ist so
eigen und wunderlich, dass es wirklich ein Trübsal ist, mit ihm zu leben, weil
es fast unmöglich ist, es ihm recht zu machen. Der kleinste Fehler in Betreff
seiner Kleidung, Speise, Wohnung oder Aufwartung, bringt ihn ganz ausser
Fassung,
besonders wenn er sich einbildet, dass man es mit den Achtungs- und
Ehrenbezeigungen, die er erwartet, nicht genau genug nehme. So zerstört der
Stolze alle natürlichen Bande der Verwandtschaft\index{Verwandtschaft}, indem er
auf der einen Seite
Pflicht\index{Pflicht} und Schuldigkeit verachten lehrt und auf der anderen
Liebe in Furcht
verwandelt, aus seiner Frau\index{Personen:!Ehefrau} eine Magd und aus seinen
Kindern\index{Personen:!Kinder} und Dienstboten
Sklaven macht.}

\section{4. Abschnitt} \label{kap12_ab4}

Auch ist der Stolze immer ein schlechter Nachbar, weil er ein Feind der
Gastfreundschaft\footnote{\texttt{'Gastfreiheit' ersetzt durch
'Gastfreundschaft'}}
\index{Gastfreundschaft} ist. Er hasst alle
Freundschaftsdienste, aus Furcht, sie erwidern
zu müssen, oder damit es nicht das Ansehen haben möge, dass er derselben
bedürfe.
Überdies kann er sich nicht damit abgeben, weil sie seinem Hochmut zu viel
Gleichheit\index{Gleichheit} und Vertraulichkeit zu verraten scheinen. Mit
Größern wetteifern und
seinesgleichen verkleinern ist sein Element, denn er ist zu neidisch, um
anderen Gerechtigkeit widerfahren zu lassen, damit nicht das Lob\index{Lob}, das
er ihren
Verdiensten erteilen müsste, dasjenige, auf welches er - wiewohl ohne allen
Grund -- Anspruch macht, verdunkeln oder vermindern möge. Er befürchtet, was er
wünschen sollte, nämlich dass andere Gutes tun möchten. Aber dabei lässt seine
Bösartigkeit es nicht bewenden, \label{ref:12_04_eitle_menschen_tugent}
\textbf{er gibt auch den tugendhaften Handlungen
anderer schlechte Namen, weil er sich unfähig fühlt, sie nachzuahmen und ihnen
den guten Ruf derselben missgönnt.} Fehlt es ihm an einer Gelegenheit, Schaden
zu
tun, so weiss er sich eine zu schaffen: Man hat ihn entweder schlecht behandelt,
oder etwas Böses gegen ihn beabsichtigt, da oder dort hat man ihn nicht gegrüßt,
den Hut nicht vor ihm abgenommen oder ihm die Achtung und Ehre nicht erwiesen,
die, seiner Meinung nach, seinem Stande, seinen Verdiensten und seinen
Eigenschaften gebührt. \label{ref:12_04_eitle_menschen_beleidigung}
\textbf{Es bedarf nur einer geringen Kleinigkeit, um den
Stolzen
zu Zank und Streit\index{Streit} zu bewegen, denn er ist unter allen
menschlichen Geschöpfen
das eifersüchtigste, eigensinnigste, feindseligste und rachsüchtigste, und es
ist ihm ebenso unmöglich, Beleidigungen zu vergeben, als er es unterlassen kann,
sie anderen zuzufügen.}

\section{5. Abschnitt} \label{kap12_ab5}

\label{ref:12_05_eitle_menschen_freundschaft}
\textbf{Überdies kann ein Stolzer nie jemandes Freund sein.} Denn sobald es
seiner
Ehre
oder seines Emporsteigens gilt, wird immer sein Ehrgeiz die Bande der
Freundschaft
zerreißen; und dann ist er auch zu ungesellig. Er nimmt weder Unterricht noch
Rat, noch vielweniger Zurechtweisung\index{Zurechtweisung} an, und kann
durchaus keinen Widerspruch\index{Widerspruch}
ertragen. Auch ist er mit den Einsichten, die er zu besitzen glaubt, so geizig,
dass er anderen nichts davon abgeben oder sich ihnen nicht mitteilen will.
Kurz, er ist zu sehr von sich eingenommen und viel zu unbiegsam und empfindlich,
als dass er irgend jemand solche Freiheiten einräumen sollte, als wahre
Freundschaft fordert. Eigentlich ist ihm der wahre Charakter der Freundschaft
verächtlich, es scheint ihm zu viel Vertraulichkeit und Erniedrigung darin zu
liegen. \textbf{Seine erhabene Seele wünscht nur ihre eigene Größe zu kennen,
und alles
um sich her in tiefer Abhängigkeit zu sehen. Daher schätzt er die Menschen auch,
wie man gewöhnlich das Vieh schätzt, nämlich nach Maßgabe des Nutzens, den es
gewährt, und wenn er könnte, so würde er sie auch so behandeln, aber glücklicher
Weise sind sie ihm an Anzahl und Kraft überlegen.}

\section{6. Abschnitt} \label{kap12_ab6}

Wenn aber ein Stolzer Gewalt\index{Macht!gefähliche} hat, so ist er höchst
schädlich, denn sein Ehrgeiz
wird durch seine Größe um so gefährlicher, da er in Tyrannei\index{Tyrannei}
ausartet. Er will
allein regieren, ja, er möchte lieber nur allein leben, als Nebenbuhler haben:
\texttt{Aut Caesar, aut nullus}. Weder die Zügel der Vernunft, noch die
Schranken des
Gesetzes können seinen Schritten Einhalt tun, denn er glaubt nichts Unrechtes
tun zu können und hält es daher schon für Empörung, wenn über seine ungerechten
Handlungen Klagen geführt werden. Solche Menschen wollen, dass man nichts von
dem, was sie tun, für Unrecht halte, wenigstens halten sie es für gefährlich,
wenn jemand ihren unrechten Handlungen die rechten Namen gibt, weil dieses
anzeigen würde, dass sie geirrt hätten und das darf ihre Politik nie zugeben.
Nein! Sie wollen lieber in ihrer Hartnäckigkeit umkommen, als durch Nachgeben
eingestehen, dass ihre Untergebenen eine Sache besser als sie beurteilt hätten,
sollte selbst auch die Klugheit ihnen anraten, dies einzuräumen.
\label{ref:12_06_eitle_menschen_sturz}
\textbf{Und in der
Tat, die einzige Genugtuung, welche die stolzen Großen für alles Unheil, das
sie angerichtet haben, der Welt geben, besteht darin, dass sie, früher oder
später, ihren wahren Vorteil hintansetzen, um irgend einer Laune ihres Stolzes
zu folgen, und dadurch fast immer ihren Untergang sich selbst bereiten. So enden
endlich die Stolzen in dem Umsturz ihres eigenen Gebäudes, nachdem sie lange
genug anderen zur Strafe gedient haben.}

\section{7. Abschnitt} \label{kap12_ab7}

\index{Personen:!Klerus} \label{ref:12_07_eitle_menschen_religion}
\textbf{Vor allen anderen ist aber der Stolz\index{Stolz} bei denen
unerträglich, die auf Religion
Anspruch machen,} und unter diesen vornehmlich bei \textit{Dienern der
Religion}\index{Personen:!Diener!der-Religion}, denn
Religion und Stolz sind einander ganz widersprechende Dinge. \textbf{Ich rede
ohne
Rücksicht und ohne Voreingenommenheit gegen irgend eine Person oder Partei, ich
greife nur das Böse\index{Böse, Das} in allen an. Aber wie kann sich der Stolz
mit der Religion
vertragen, da diese ihn tadelt und verwirft? Oder wie lässt sich Ehrgeiz mit den
Gesinnungen wahrer Diener der Religion vereinbaren, deren Amt und Pflicht es
ist,
Demut zu lehren und durch ihr Beispiel zu befördern?} Und doch gibt es deren,
leider, nur zu viele, die nicht allein mit anderen an dem stolzen und eitlen
Wesen der Welt Anteil nehmen, sondern sogar auf einen Namen und auf ein Amt
stolz sind, das sie doch beständig an
Selbstüberwindung\footnote{\texttt{'Selbstverleugung' ersetzt durch
'Selbstüberwindung'}}
erinnern sollte. \index{Amtsmissbrauch}
\label{ref:12_07_eitle_menschen_religion_2} \textbf{O! Sie
bedienen sich desselben nur, wie die Bettler des Namens Gottes und Christi sich
bedienen, nämlich, um etwas dadurch zu erlangen, indem sie die Vorteile dieses
ehrwürdigen Standes sich zueignen, und so ihr Amt nur als ein politisches
Hilfsmittel gebrauchen, um sich in der Welt hervorzutun.} Wie können solche
aber
Diener desjenigen sein, der gesagt hat:
\textit{"`Mein Reich ist nicht von dieser
Welt."'}\footnote{Johannes 18,36}
\index{Bibelstellen:!Johannes 18)}
Gibt es in der Welt wohl Menschen, die mehr von
sich eingenommen wären, als diese? Widerspricht man ihnen, so zeigen sie so viel
Stolz und Anmaßung, dass man glauben sollte, diese gehören zu ihrem Amte. Gib
ihnen guten Rat, so verachten sie dich. Wagst du es, einen von ihnen zu tadeln
oder zu bestrafen, so ist er bereit, dich auf der Stelle in den Bann\index{Bann}
zu tun. \label{ref:12_07_eitle_menschen_geislicher}
\textbf{‚\textit{Ich bin Geistlicher, ein Ältester der Kirche}’ ruft er aus,
als ob diese Titel
ihn gegen gerechten Tadel schützen könnten, da sie ihn doch in der Tat
demselben nur um so vielmehr aussetzen,} als Fehler und Verweigerung der
Zurechtweisung bei einem Religionsdiener strafbarer als bei anderen Menschen
sind.

\section{8. Abschnitt} \label{kap12_ab8}

\index{Personen:!Klerus}Aber er beruft sich darauf, dass er, vermöge seines
Amtes, eine
Ausnahme mache.
Denn, sollte er die Küken dazu aufgezogen haben, dass sie ihm die Augen
aushacken könnten? Soll er Tadel oder Zurechtweisung von einem
Laien\index{Personen:!Laien, im Glauben} oder von
einem seiner Pfarrkinder annehmen? Oder von jemand, der vielleicht jünger, nicht
so gelehrt und weniger talentvoll als er ist? Das kann man nicht erwarten. Wir
müssen wissen, dass das Vorrecht seines Amtes ihn über allen Tadel des Volks
erhebt, und er folglich der Beurteilung gewöhnlicher Menschen nicht unterworfen
ist. Selbst Fragen über religiöse Gegenstände sind schon
Ketzerei\index{Personen:!Ketzer}. Glaube was er
sagt, und sei nicht so neugierig, dass du in die Geheimnisse der
Religion\index{Religion!Geheimnisse} zu
schauen begehrest! \label{ref:12_08_eitle_menschen_leihen_vs_geisltichkeit}
\textbf{Seitdem die Laien sich so viel um die Angelegenheiten
der
Geistlichen bekümmern, stehen die Sachen nicht mehr, wie sie sollten! -- Armer
Mann! Du denkst wohl nicht, dass gerade das Gegenteil der Fall ist? Dass,
seitdem
die Geistlichkeit sich so sehr in die Angelegenheiten der Laien mischet, die
Sachen schlecht stehen?} Wiewohl eigentlich zwischen den Religionsdienern und
sogenannten Laien kein weiterer Unterschied stattfinden kann, als in so fern
die Erstern durch geistliche Gaben, und durch deren Ausbildung und fleißige
Anwendung zum Besten ihrer Mitmenschen sich von den Letzteren unterscheiden.

Solche heilsame Worte als diese:
\textit{"`Seid lehrhaft, freundlich gegen jedermann,
lasset einen jeden reden nach der Gabe Gottes, die in ihm ist. Wenn aber ein
anderer, der da sizt, eine Offenbarung hat, so schweige der Erstere. Seid nicht
solche, die über das Erbe Gottes herrschen, sondern seid sanftmütig und
deumütig, bereit, anderen die Füße zu waschen, wie Jesus seinen armen Jüngern die
Füße wusch;"'}\footnote{2. Timotheus 2,24+25.}
\index{Bibelstellen:!2. Timotheus 2)}
\textbf{Solche treffende Worte werden von
\index{Bibel!Anwendbarkeit}einigen, die sich zu der Geistlichkeit rechnen, als
unanwendbare, veraltete
Vorschriften betrachtet, und es wird heutiges Tages fast für
Ketzerei\index{Personen:!Ketzer} gehalten,
wenn man sie daran erinnert, ja, man zeigt sich dadurch, ihrer Meinung nach, nur
als ein Feind der Kirche\index{Personen:!Feind der Kirche}\index{Kirche!Feind
der Kirche}.} Denn ihr Stolz hat sie nun
schon so weit gebracht,
dass sie sich selbst als die Kirche und das Volk etwa nur als die Vorhalle
derselben betrachten. Ja, sie sehen dasselbe gleichsam wie eine Null an, denn so
wie diese ohne ihre vorgesetzten Grundzahlen nichts gilt, so ist auch das Volk
in
ihren Augen nichts, wenn sie ihm nicht voranstehen. Sie sollten aber bedenken,
dass sie, wenn sie wirklich wären, was sie sein sollten, doch nur Diener,
Haushalter, Unterhirten, nämlich Diener der Kirche\index{Kirche!Diener der
Kirche}
\index{Personen:!Diener!der Kirche}, der
Haushaltung, der Herde
oder des Erbes Gottes sein könnten, und folglich diese Kirche, diese
Haushaltung, diese Herde und dieses Erbe nicht selbst sind. Auch müssten sie
sich erinnern, dass Christus ausdrücklich gesagt hat:
\textit{"`Wenn jemand unter euch
gewaltig sein will, der sei euer Diener, und wer der Vornehmste sein will, der
sei euer Knecht."'}\footnote{Matthäus 20,26+27.}
\index{Bibelstellen:!Matthäus 20)}

\section{9. Abschnitt} \label{kap12_ab9}

Es findet sich in der heiligen Schrift nur eine Stelle, wo das Wort
Klerus\index{Personen:!Klerus},
eigentlich auf die Kirche angewendet werden kann, und diese haben sie sich
angeeignet. Daher nennen sie sich die Kleriker, d.h. das Erbgut oder Erbe
Gottes. Hingegen ermahnt Petrus\index{Personen:!Petrus} die Diener des
Evangeliums mit den Worten:
\textit{"`Weidet die Herde Christi nicht um schändlichen Gewinns willen. --
Nicht
als solche, die über das Volk (oder Erbteil des Herrn) herrschen."'}
Wahrscheinlich sah es Petrus voraus, dass Stolz und Geiz den Kirchendienern zur
Versuchung gereichen würden, auch haben sie in der Tat sich nur zu oft in
diesen beiden Schlingen fangen lassen, und in schlimmere hätten sie, in
Wahrheit, nicht geraten können. Und was sie auch in beiden Hinsichten zu ihrer
Rechtfertigung vorbringen mögen, so wird immer die Entschuldigung ebenso
schlimm als der Fehler selbst sein. Denn, wenn sie sagen, dass sie nicht über
das
Erbe oder das Volk des Herrn herrschen, so ist es nur dewegen nicht der Fall,
weil sie sich selbst als dieses Erbe betrachten und das Volk seines Rechts
entsetzt haben. Auf diese Weise könnten sie denn freilich Herren des Volks sein
und von den Ermahnungen des guten alten Petrus eine Ausnahme machen.

\medskip

\index{Kirche!-steuer}Was den anderen Punkt, nämlich das Laster des Geizes
betrifft, so können einige
der Beschuldigung desselben nicht anders ausweichen, als wenn sie mit Wahrheit
sagen, man könne ihnen, da sie sich um die Herde gar nicht bekümmern, nicht zur
Last legen, dass sie dieselbe um schändlichen Gewinns willen weideten, woraus
dann aber folgt, dass sie das Geld von den Leuten umsonst nehmen. Davon finden
wir treffende Beispiele in der Schrift, wo Gott selbst sich über das stolze und
habsüchtige Benehmen der vormaligen falschen Propheten beklagt, wenn er zum Beispiel
durch Jesaias sagt:
\textit{"`das Volk gibt sein Geld für etwas hin\footnote{\texttt{In
Original:'das Volk zähle sein Geld für etwas dar'}},
dass kein Brot sei,
und wendet seine Arbeit an das, wovon es nicht satt werden
kann."'}\footnote{Jesaja 55,2}
\index{Bibelstellen:!Jesaja 55)}
Und was war die Ursache davon? --
\textit{"`Die Hirten hatten keinen Verstand, ein
jeder sah auf seinen Weg, und geizte für sich in seinem
Stande,"'}\footnote{Jesaja 56,11}
\index{Bibelstellen:!Jesaja 56)} und
\textit{"`die Priester und Propheten hatten keine wahren
Gesichte,"'}\footnote{Ezechiel 13}
\index{Bibelstellen:!Ezechiel 13)}
welche auch heutiges Tages nur zu viele
verachten.

\section{10. Abschnitt} \label{kap12_ab10}

Aber ach! Wie viel Torheit und Mangel an Religion verrät nicht endlich der
Mensch, der von sich selbst eingenommen oder aus irgend etwas, das er besitzt,
stolz ist? Kann er doch mit allen seinen hohen Gedanken seine Gestalt nicht um
einen Zoll vergrößern! Welche Widerwärtigkeiten kann sein stolzer Sinn abwenden?
Welchen Unfällen kann er abhelfen, welchem Übel vorbeugen? Er ist nicht
vermögend, sich nur vor einem der Schläge zu schützen, denen alle Menschen ohne
Ausnahme ausgesetzt sind.
\index{Vergänglichkeit}\index{Krankheit}\index{Leid}Eine Krankheit entstellt,
Schmerz und Leiden
verändern die Züge, und der Tod zerstört den ganzen schönen Körperbau auch der
Stolzesten unter den Sterblichen.
\label{ref:12_10_eitle_menschen_tod}
\textbf{Ein kleiner Haufen kalter Erde schließt nun
den Leichnam des Mannes ein, dessen hochfahrende Gedanken keine Grenzen kannten.
Seine zarte Person, der -- vielleicht noch vor kurzem -- kein Ort und keine
Gesellschaft gut genug war, muss sich jetzt in dem engen Raum einer kleinen
finstern Höhle behelfen und sich die Gesellschaft der geringsten Geschöpfe
gefallen lassen, der Würmer nämlich, denen sie bald zur Speise dienen soll. So
nehmen die Stolzen und Prachtliebenden ein gleiches Ende mit allen anderen,
jedoch mit dem Unterschied, dass sie von den Überlebenden\index{Trauer} weniger
bedauert
werden und im Sterben eine furchtbare und peinvolle Aussicht in die Ewigkeit
haben.} Denn so wenig den Stolzen seine vornehme Herkunft vor dem Tod schüzen
kann, eben so wenig ist auch sein Geschlechtsregister vermögend, ihn vor dem
Gericht zu schützen, dass ihn nach dem Tode erwartet. Die schauerliche Stunde
des Hinscheidens löst alle seine Titel und Ehrzeichen in ein nichts auf, und
keine irdische Macht, weder Reichtum noch Hoheit oder Ansehn, ist vermögend,
ihn zu erretten oder in Schutz zu nehmen. \textbf{\textit{"`Wie der Baum fällt,
so wird er
liegen,"'} und wie der Tod den Menschen verlässt, so findet ihn das Gericht.}
\index{Gericht!Jüngste}

\section{11. Abschnitt} \label{kap12_ab11}

Aber ach! \textbf{Wie kann man nun einem so elenden Ende vorbeugen?} Was für ein
Mittel
gibt es gegen diese bejammernswerte Abweichung und Entfernung von der Demut,
Sanftmut und echten Frömmigkeit, von jenem heiligen Leben der Gläubigen in den
ersten und reinsten Jahrhunderten des Christentums, und von der göttlichen
Kraft, die sich so fühlbar und augenscheinlich sowohl in ihrem mächtigen
Predigen, als auch in ihrem musterhaften Betragen bewies? \textbf{Wahrlich kein
anderes,
als dass man zu dem \index{Selbst!-prüfung}Zeugnisse des Geistes Jesu in sich
selbst einkehre, und in
seinem heiligen Licht den Zustand seines eigenen Herzens untersuche und prüfe,
in wiefern man ihm ähnlich sei oder nicht, und dass man zu diesem Zweck die in
den Urkunden der heiligen Schrift\index{Gebote!heilige Schrift} enthaltenen
Lehren und
Beispiele mit genauer
Aufmerksamkeit betrachte. Christus führte einst selbst die Klage,
\textit{"`dass das
Licht in die Welt gekommen sei, dass aber die Menschen die Finsternis mehr als
das Licht liebten, weil ihre Werke böse waren."'}\footnote{Johannes 3,19}
\index{Bibelstellen:!Johannes 3)}
Willst du
nun ein Kind Gottes\index{Personen:!Kind!Gottes} sein, ein Gläubiger, der an
Christum
glaubt, so musst du erst
ein Kind des Lichtes\index{Personen:!Kind!des Lichtes} werden. Du musst deine
Werke an das
Licht in deinem Inneren
bringen und sie bei dieser heiligen Lampe deiner Seele prüfen. Denn es ist das
Licht des Herrn, das dir deinen Stolz und deine Anmaßung zu erkennen gibt,}
und
dich über das Vergnügen, welches du an den eitlen Moden\index{Mode} und
Gebräuchen der Welt
findest, insgeheim bestraft.
\label{ref:12_11_eitle_menschen_erloesung}
\textbf{Wahre Religion ist Selbstüberwindung
\footnote{\texttt{'Selbstverleugnung' ersetzt durch 'Selbstüberwindung'.}}, ja,
und auch Überwindung\footnote{\texttt{'Verleugnung' ersetzt durch
'Überwindung'}}
aller selbsterwählten Religion\index{Religion!selbsterwählte} und eigenwilligen
Frömmigkeit.\index{Frömmigkeit!eigenwillige}} Sie
ist ein starkes Band der Seele, welches die Menschen zu einem heiligen Leben
verbindet, das zur Glückseligkeit\index{Glückseligkeit} führt. Denn dadurch
gelangen sie zum Anschauen
Gottes wie Jesus sagt:
\textit{"`Selig sind die, welche reines Herzens sind, denn sie
werden Gott schauen."'}\footnote{Matthäus 5,8}
\index{Bibelstellen:!Matthäus 5)}
Wer einmal dahin gelangt ist, dass
er das Joch Christi standhaft trägt, der lässt sich von den Lockungen des
Feindes
seiner Seele nicht hinreißen. Er findet höhere Freuden in der Wachsamkeit über
sein Herz und im Gehorsam gegen das Gesetz Christi\index{Gebote!Gesetz Christi},
als die Vergnügungen der
Welt ihm bieten können. Wenn die Menschen wirklich das Kreuz
Christi\index{Kreuz!Kreuz Christi}, seine
Gebote und Lehren, liebten, so würden sie ihren eigenen Willen
kreuzigen\index{Kreuz!kreuzigen}, der
sie verleitet, dem heiligen Willen Christi entgegen zu leben, und den Willen des
Feindes ihrer Glückseligkeit\index{Glückseligkeit} zu tun, wodurch sie sich um
das Heil\index{Heil} ihrer Seelen
bringen. Hätte Adam\index{Personen:!Adam} im Paradies\index{Orte:!Paradies}
nicht auf die Lockpfeife der Schlange\index{Personen:!Schlange}, sondern
auf das göttliche Licht in seiner Seele geachtet und sein Gemüt auf seinen
Schöpfer, den Vergelter der Treue, gerichtet, so würde er die Schlinge des
Feindes gesehen und ihm in seinen listigen Versuchungen widerstanden haben. O
darum ergötze dich nicht an verbotenen Dingen. Siehe das Böse\index{Böse} nicht
an, wenn du
nicht davon gefesselt werden willst. Belaste deine Seele nicht mit der Schuld,
gegen deine Erkenntnis zu sündigen. Unterwarf nicht Christus seinen Willen dem
Willen seines himmlischen Vaters? Erduldete er nicht nur der ihm bevorstehenden
Freude Willen das Kreuz\index{Belohnung!Jenseitige}? Und verachtete er nicht die
Schmach und Schande des
neuen unbetretenen Weges zur Herrlichkeit?\footnote{Hebräer 12,2}
\index{Bibelstellen:!Hebräer 12)}
\label{ref:12_11_opfer} \textbf{So musst auch du
deinen Willen dem heiligen Gesetz\index{Gebote!heiliges Gesetz} und Licht
Christi\index{Licht!Christi} in deinem Herzen
unterwerfen, und um der Belohnung\index{Belohnung} willen, die er dir vorhält,
sein Kreuz erdulden und die Schande desselben nicht achten. Alle möchten gern
mit Christo
sich freuen, aber wenige wollen mit ihm oder um seinentwillen
leiden\index{Leid}.} Viele sind
bereit, an seinem Tische teil zu nehmen, wenigen aber gefällt seine
Enthaltsamkeit\index{Enthaltsamkeit}. Seiner Austeilung des Brotes wollen sie
nachfolgen, aber den
Kelch\index{Kelch!bitteren} seiner Todesangst lassen sie stehen. Dieser ist
ihnen zu bitter, als dass
sie ihn trinken möchten. -- Manche erheben seine Wunder\index{Wunder}, und
ärgern sich doch an
der Schwäche\index{Christi!Schwäche} seines Kreuzes. Darum höre, o Mensch, was
er zu deinem Heil für
dich tat, das musst auch du aus Liebe zu ihm tun. Du musst dich demütigen, und
es dir gefallen lassen, um seinetwillen Verachtung zu leiden, damit du ihn
wahrhaft nachfolgen\index{Nachfolge} kannst, nämlich nicht auf eine bloß
förmliche, leblose Weise, oder nach den Erfindungen und
Vorschriften\index{Gebote!erfundene Vorschriften} eitler Menschen, sondern, wie
der heilige Geist durch den Apostel sich ausdrückt:
\textit{"`auf dem neuen und
lebendigen Wege,"'}\footnote{Hebräer 10,20}
\index{Bibelstellen:!Hebräer 10)}
den Jesus geheiligt hat, der alle, die
auf demselben wandeln, zu der ewigen Ruhe\index{ewige!Ruhe} Gottes führt, zu
welcher er, der
allein heilige und glorreiche Erlöser, selbst eingegangen ist.




