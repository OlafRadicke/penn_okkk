% STATUS=Unfertig
% FUSSNOTEN=1.Durchlauf
% ANFÜHRUNGSZEICHEN=
% KORSIVSCHRIFT=
% FETTSCHRIFT=
% RANDBEMERKUNGEN=
% MODERNE_RECHTSCHREIBUNG=noch nicht begonnen
% WORTERSÄTZUNG=
% VERSCHLAGWORTUNG(INDEX)=
% REFERENZIERUNG=abgeschlossen
% BIBELSTELLEN=ungeprüft


\chapter{12. Kapitel} \label{kap12}

\section{Zusammenfassung des 12. Kapitel}
\small
\begin{description}
\item[1. Abschnitt] Charakter des Stolzen. Seine Eigenliebe ist unerfättlich; er
ist stolz auf feine Geburt; (Seite: \pageref{kap12_ab1})
\item[2. Abschnitt] Er ist trozig und zänkisch; aber feige, und doch grausam. (Seite: \pageref{kap12_ab2})
\item[3. Abschnitt] Ein eben so schlechter Sohn als schlechter Unterthan und
Diener. (Seite: \pageref{kap12_ab3})
\item[4. Abschnitt] Ein Feind der Gastfreundschaft. (Seite: \pageref{kap12_ab4})
\item[5. Abschnitt] Keines Menschen Freund. (Seite: \pageref{kap12_ab5})
\item[6. Abschnitt] Als Gewalthaber gefährlich und schädlich. (Seite: \pageref{kap12_ab6})
\item[7. Abschnitt] Vor allen Andern sind stolze Prediger zu tadeln. (Seite: \pageref{kap12_ab7})
\item[8. Abschnitt] Sie verlangen Vorrechte vor Andere, (Seite: \pageref{kap12_ab8})
\item[9. Abschnitt] Sie nennen sich die Klerisei; -- ihre Herrschsucht und ihr
Geiz. (Seite: \pageref{kap12_ab9})
\item[10. Abschnitt] Der Tod verschlingt Alle. (Seite: \pageref{kap12_ab10})
\item[11. Abschnitt] Mittel, den bösen Folgen des Stolzes zu entgehen. (Seite: \pageref{kap12_ab11})
\end{description}
\normalsize

\section{1. Abschnitt} \label{kap12_ab1}

Um endlich diese lange Abhandlung über den Stolz zu beschließen, wollen wir noch
in der Kürze untersuchen, worin im Ganzen genommen der Charakter eines stolzen
Menschen an sich und in Beziehung auf Andere bestehet. Der Stolze ist eine Art
selbstsüchtigen Schwelgens denn er wird nie satt, sich selbst zu lieben und zu
bewundern, während alles Andere in seinen Augen weder Liebe noch Bewunderung
verdient. Daß Verdienst, welchen er allenfalls andern Gegenständen noch
einräumt, bestehet bloß darin, daß sie seinen Zwecken dienen; als ob Alles nur
für ihn geschaffen, oder vielmehr, als wenn er sein eigener Schöpfer wäre. So
wie er daher andere Menschen deswegen verachtet, weil er seines Gleichen nicht
dulden mag, so liebt er auch Gott nicht, weil er keinen Höheren über sich haben
will. Der Gedanke, sein Dasein einem Andern zuzuschreiben, ist ihm unerträglich,
da dieser ihn in die Nothwendigkeit versetzt, ein höheres Wesen anzuerkennen. Er
ist stolz auf die Ehre seiner Vorfahren, aber nicht auf die Tugenden, durch
welche sie dazu gelangten; auch giebt er sich nicht die geringste Mühe, ihnen
darin nachzuahmen. Seine Erzählungen von seinen: Geschlechtsregister, von der
uralten Herkunft, von den Besitzungen und Verbindungen seiner Vorfahren, nehmen
kein Ende; aber er vergißt, daß sie nicht mehr sind, und daß auch er sterben
muß.

\section{2. Abschnitt} \label{kap12_ab2}

Wer ist wohl lästiger in der Gesellschaft, als der Stolze? Er bekritelt jede
Kleinigkeit, und spricht gebieterisch über Alles ab. Giebt man ihm nicht nach,
so wird er beleidigend und zänkisch; doch wenn es aufs Aeußerste kommt, so zeigt
er sich feige; aber grausam, sobald er siehet, daß er die Obermacht hat. Für das
Elend hat er kein Mitgefühl; als wäre es unter seiner Würde, gefühlvoll zu seyn.
Das Unglück eines Andern rührt ihn so wenig, als wenn er selbst kein Mensch
wäre, oder als hielte er Mitleid für eine Sünde. Was nicht gerade ihn angehet,
fesselt auch seine Aufmerksamkeit nicht. Er will sich über das Unglück Anderer
keine unruhige Gedanken machen; ihm genügt die  Ueberzeugung, daß sie es
verdient haben, und er möchte es ihnen lieber gerade heraus sagen, daß es ihre
eigene Schuld sei, als sich bereit finden lassen, sie zu bedauern oder ihnen zu
helfen. Daher scheinen ihm Mitleid und Wohlthütigkeit eben so überflüssig zu
seyn, als Demuth und Sanftmuth ihm verhaßt sind.

\section{3. Abschnitt} \label{kap12_ab3}

Der Stolze ist sowohl ein schlechter Sohn, als schlechter Diener und Unterthan;
er verachtet seine Eltern, seine Herren und seinen Fürsten. Sich zu unterwerfen,
ist ihm unerträglich. Auch dünkt er sich zu weise, oder hält sich für zu alt, um
sich vorschreiben zu lassen; als wenn Gehorsam Sklaverei wäre, und Freiheit
darin bestände, daß man thun dürfe, was man wolle; welches jedoch alle Pflicht
aufheben und alles Ansehn herabsezen würde. - Ist der Stolze verheirathet, ist
er Vater oder Herr; so ist es fast unmöglich, es bei ihm auszuhalten. Er ist so
eigen und wunderlich, daß es wirklich eine Trübsal ist, mit ihm zu leben, weil
es fast unmöglich ist, es ihm recht zu machen. Der kleinste Fehler in Betreff
seiner Kleidung, Speise, Wohnung oder Aufwartung, bringt ihn ganz außer Fassung,
besonders wenn er sich einbildet, daß man es mit den Achtungs- und
Ehrenbezeigungen, die er erwartet, nicht genau genug nehme. So zerstört der
Stolze alle natürlichen Bande der Verwandtschaft, indem er auf der einen Seite
Pflicht und Schuldigkeit verachten lehret, und auf der andern Liebe in Furcht
verwandelt, aus seiner Frau eine Magd und aus seinen Kindern und Dienstboten
Sklaven macht.

\section{4. Abschnitt} \label{kap12_ab4}

Auch ist der Stolze immer ein schlechter Nachbar, weil er ein Feind der
Gastfreiheit ist. Er haßt alle Freundschaftsdienste, aus Furcht, sie erwiedern
zu müssen, oder damit es nicht das Ansehen haben möge, daß er derselben bedürfe.
Ueberdieß kann er sich nicht damit abgeben, weil sie seinem Hochmnthe zu viel
Gleichheit und Vertraulichkeit zu verrathen scheinen. Mit Größern wetteifern und
seines Gleichen verkleinern, ist sein Element; denn er ist zu neidisch, um
Andern Gerechtigkeit widerfahren zu lassen; damit nicht das Lob, das er ihren
Verdiensten ertheilen müßte, dasjenige, auf welches er - wiewohl ohne allen
Grund -- Anspruch macht, verdunkeln oder vermindern möge. Er befürchtet, was er
wünschen sollte, nämlich: daß Andere Gutes thun möchten. Aber dabei läßt seine
Bösartigkeit es nicht dewenden; er giebt auch den tugendhaften Handlungen
Anderer schlechte Namen, weil er sich unfähig fühlt, sie nachzuahmen, und ihnen
den guten Ruf derselben mißgönnt. Fehlt es ihm an einer Gelegenheit, Schaden zu
thun, so weiß er sich eine zu schaffen: Man hat ihn entweder schlecht behandelt,
oder etwas Böses gegen ihn beabsichtigt; da oder dort hat man ihn nicht gegrüßt,
den Hut nicht vor ihm abgenommen, oder ihm die Achtung und Ehre nicht erwiesen,
die, seiner Meinung nach, seinem Stande, seinen Verdiensten und seinen
Eigenschaften gebühret. \textbf{Es bedarf nur einer geringen Kleinigkeit, um den Stolzen
zu Zank und Streit zu bewegen; denn er ist unter allen menschlichen Geschöpfen
das eifersüchtigste, eigensinnigste, feindseligste und rachsüchtigste, und es
ist ihm eben so unmöglich Beleidigungen zu vergeben, als er es unterlassen kann,
sie Andern zuzusugen.}

\section{5. Abschnitt} \label{kap12_ab5}

Ueberdieß kann ein Stolzer nie Jemands Freund seyn. Denn sobald es seine Ehre
oder sein Emporsteigen gilt, wird immer sein Ehrgeiz die Bande der Freundschaft
zerreißen; und dann ist er auch zu ungesellig. Er nimmt weder Unterricht noch
Rath, noch vielweniger Zurechtweisung an, und kann durchaus keinen Widerspruch
ertragen. Auch ist er mit den Einsichten, die er zu besitzen glaubt, so geitzig,
das; er Anderen Nichts davon abgeben oder sich ihnen nicht mittheilen will.
Kurz, er ist zu sehr von sich eingenommen und viel zu unbiegsam und empfindlich,
als daß er irgend Jemand solche Freiheiten einräumen sollte, als wahre
Freundschaft fordert. Eigentlich ist ihm der wahre Charakter der Freundschaft
veraehtlich; es scheint ihm zu viel Vertraulichkeit und Erniedrigung darin zu
liegen. Seine erhabene Seele wünscht nur ihre eigene Größe zu kennen, und Alles
um sich her in tiefer Abhäugkeit zu sehen. Daher schätzt er die Menschen auch,
wie man gewöhnlich das Vieh schätzt, nämlich nach Maßgabe des Nutzens, den es
gewährt; und wenn er könnte, so Würde er sie auch so behandeln; aber glücklicher
Weise sind sie ihm an Anzahl und Kraft überlegen.

\section{6. Abschnitt} \label{kap12_ab6}

Wenn aber ein Stolzer Gewalt hat, so ist er höchst schädlich; denn sein Ehrgeiz
wird durch seine Größe um so gefährlicher, da er in Tyrannei ausartet. Er will
allein regieren; ja, er möchte lieber nur allein leben, als Nebenbuhler haben:
Aut Caesar, aut Nullus. Weder die Zügel der Vernunft, noch die Schranken des
Gesetzes, können seinen Schritten Einhalt thun; denn er glaubt nichts Unrechts
thun zu können, und hält es daher schon für Empörung wenn über seine ungerechten
Handlungen Klagen geführt werden. Solche Menschen wollen, man solle Nichts von
dem, was sie thun, für Unrecht halten; wenigstens halten sie es für gefährlich,
wenn Jemand ihren unrechten Handlungen die rechten Namen giebt, weil dieses
anzeigen würde, daß sie geirrt hätten und das darf ihre Politik nie zugeben.
Nein! sie Wollen lieber in ihrer Hartnäckigkeit umkommen, als durch Nachgeben
eingestehen, daß ihre Untergebenen eine Sache besser als sie beurtheilt hätten;
sollte selbst auch die Klugheit ihnen anrathen, dies einzuräumen. Und in der
That, die einzige Genugthnung, welche die stolzen Großen für alles Unheil, das
sie angerichtet haben, der Welt geben, besteht darin, daß sie, früher oder
später, ihren wahren Vortheil hintansezen, um irgend einer Laune ihren Stolzes
zu folgen, und dadurch fast immer ihren Untergang sich selbst bereiten. So enden
endlich die Stolzen in dem Umsturze ihres eigenen Gebäudes, nachdem sie lange
genug Andern zur Strafe gedient haben.

\section{7. Abschnitt} \label{kap12_ab7}

\textbf{Vor allen Andern ist aber der Stolz bei Denen unerträglich, die auf Religion
Anspruch machen,} und unter Diesen vornehmlich bei ''Dienern der Religion''; denn
Religion und Stolz sind einander ganz widersprechende Dinge. \textbf{Ich rede Ohne
Rücksicht, und ohne Eingenommenheit gegen irgend eine Person oder Partei; ich
greife nur das Böse in Allen an.} Aber wie kann sich der Stolz mit der Religion
vertragen, da diese ihn tadelt und verwirft? Oder wie läßt sich Ehrgeiz mit den
Gesinnungen wahrer Diener der Religion vereinigen, deren Amt und Pflicht es ist,
Demuth zu lehren und durch ihr Beispiel zu befördern? Und doch giebt es deren,
leider! nur zu Viele, die nicht allein mit Andern an dem stolzen und eitlen
Wesen der Welt Antheil nehmen, sondern sogar auf einen Namen und auf ein Amt
stolz sind, das sie doch beständig an Selbstverleugung erinnern sollte. O! sie
bedienen sich desselben nur, wie die Bettler des Namens Gottes und Christi sich
bedienen, nämlich: um Etwas dadurch erlangen; indem sie die Vortheile dieses
ehrwürdigen Standes sich zueignen, und so ihr Amt nur als ein politisches
Hilfsmittel gebrauchen, um sich in der Welt hervorzuthun. Wie können Solche aber
Diener Desjenlgen seyn, der gesagt hat: "`Mein Reich ist nicht von dieser
Welt."'\footnote{Joh. 18,36} Giebt es in der Welt wohl Menschen, die mehr von
sich eingenommen wären, als diese? Widerspricht man ihnen, so zeigen sie so viel
Stolz und Anmaßung, daß man glauben sollte, diese gehörten zu ihrem Amte. Gieb
ihnen guten Rath, so verachten sie dich. Wagst du es, einen von ihnen zu tadeln
oder zu bestrafen, so ist er bereit, dich auf der Stelle in den Bann zu thun.
‚Ich bin Geistlicher, ein Aeltester der Kirchel’ ruft er aus; als ob diese Titel
ihn gegen gerechten Tadel schützen könnten, da sie ihn doch in der That
demselben nur um so vielmehr aussetzen, als Fehler und Verweigerung der
Zurechtweisung bei einem Religionsdiener strafbarer, als bei andern Menschen
sind.

\section{8. Abschnitt} \label{kap12_ab8}

Aber er beruft sich daraus, daß er, vermöge seines Amtes, eine Ausnahme mache.
Denn, sollte er die Küchlein dazu aufgezogen haben, daß sie ihm die Augen
aushacken könnten? Soll er Tadel oder Zurechtweifung von einem Laien oder von
einem seiner Pfarrkinder annehmen? oder von Jemand, der vielleicht jünger, nicht
so gelehrt und weniger talentvoll als er ist? Das kann man nicht erwarten. Wir
müssen wissen, daß daß Vorrecht seines Amtes ihn über allen Tadel des Volks
erhebt, und er folglich der Beurtheilung gewöhnlicher Menschen nicht unterworfen
ist. Selbst Fragen über religiöse Gegenstände sind schon Ketzerei. Glaube was er
sagt, und sei nicht so neugierig, daß du in die Geheimnisse der Religion zu
schauen begehrest! Seitdem die Laien sich so viel um die Angelegenheiten der
Geistlichen bekiunntern, stehen die Sachen nicht mehr, wie sie sollten! -- Armer
Mann! du denkst wohl nicht, daß gerade das Gegentheil der Fall ist? Daß, seitdem
die Geistlichkeit sich so sehr in die Angelegenheiten der Laien mischet, die
Sachen schlecht stehen? wiewohl eigentlich zwischen den Religionsdienern und
sogenannten Laien kein weiterer Unterschied Statt finden kann, als in so fern
die Erstern durch geistliche Gaben, und durch deren Ausbildung und fleißige
Anwendung zum Besten ihrer Mitmenschen sich von den Letztern unterscheiden.

Solche heilsame Worte als diese: "`Seid lehrhaft, freundlich gegen Jedermann,
lasset einen Jeden reden,"' nach der Gabe Gottes, die in ihm ist. Wenn aber ein
Anderer, der da sizt, eine Offenbarung hat, so schweige der Erstere. Seid nicht
als Solche, die über das Erbe Gottes herrschen; sondern seid sanftmüthig und
deumüthig; bereit Andern die Füße zu waschen, wie Jesus seinen armen Jüngern die
Füße wusch;"'\footnote{2 tim. 2,24. 25.} solche treffende Worte werden von
Einigen, die sich zu der Geistlichkeit rechnen, als unanwendbare, veraltete
Vorschriften betrachtet, und es wird heutiges Tages fast für Kelzerei gehalten,
wenn man sie daran erinnert; ja, man zeigt sich dadurch, ihrer Meinung nach, nur
als einen Feind der Kirche. Denn ihr Stolz hat sie nun schon so weit gebracht,
daß sie sich selbst als die Kirche, und das Volk etwa nur als die Vorhalle
derselben betrachten. Ja, sie sehen dasselbe gleichsam wie eine Null au; denn so
wie diese ohne ihr vorgelelzte Grundzahlen nichts gilt, so ist auch das Volk in
ihren Augen Nichts, wenn sie ihm nicht voranstehen. Sie sollten aber bedenken,
daß sie, wenn sie wirklich wären, was sie seyn sollten, doch nur Diener,
Hanshalter, Unterhirten, nämlich Diener der Kirche, der Haushaltung, der Heerde
oder des Erbes Gottes seyn könnten, und folglich diese Kirche, diese
Haushaltung, diese Heerde und dieses Erbe nicht selbst sind. Auch müßten sie
sich erinnern, daß Christus ausdrücklich gesagt hat: "`Wenn Jeniand unter euch
gewaltig sehn will, der sei euer Diener; und wer der Vornehmste seyn will, der
sei euer Knecht."'\footnote{Matth. 20,26. 27.}

\section{9. Abschnitt} \label{kap12_ab9}

Es findet sich in der heiligen Schrift nur eine Stelle, wo das Wort Clerus,
eigentlich auf die Kirche angewendet werden kann, und diese haben sie sich
zugeeignet. Daher nennen sie sich die Klerisei, d. h. das Erbgut oder Erbe
Gottes. Hingegen ermahnt Petrus die Diener des Evangeliums mit den Worten:
"`Weidet die ''Heerde Christi'' nicht um schändlichen Gewinns willen. -- Nicht
als Solche, die über das Volk (oder Erbtheil des Herrn) herrschen."'
Wahrscheinlich sah es Petrus voraus, daß Stolz und Geiz den Kirchendienern zur
Versuchung gereichen würden; auch haben sie in der That sich nur zu oft in
diesen beiden Schlingen fangen lassen, und in schlimmere hatten sie, in
Wahrheit, nicht gerathen können. Und was sie auch in beiden Hinsichten zu ihrer
Rechtfertigung vorbringen mögen, so wird immer die Entschuldigung eben so
schlimm als der Fehler selbst seyn. Denn, wenn sie sagen, daß sie nicht über das
Erbe oder das Volk des Herrn herrschten, so ist es nur darum nicht der Fall,
weil sie sich selbst als dieses Erbe betrachten und das Volk seines Rechts
entsetzt haben. Auf diese Weise könnten sie denn freilich Herren des Volks seyn,
und von den Ermahnungen des guten alten Petrus eine Ausnahme machen.

\medskip

Was den andern Punkt, nämlieh, das Laster des Geizes betrifft, so können Einige
der Beschuldigung desselben nicht anders ausweichen, als wenn sie mit Wahrheit
sagen, man könne ihnen, da sie sich um die Heerde gar nicht bekümmern, nicht zur
Last legen, daß sie dieselbe um schändlichen Gewinns willen weideten; woraus
dann aber folgt, daß sie das Geld von den Leuten umsonst nehmen. Davon finden
wir treffende Beispiele in der Schrift, wo Gott selbst sich über das stolze und
habsüchtige Benehmen der vormaligen falschen Propheten beklagt, wenn er z. B.
durch Jesaias sagt: "`das Volk zähle sein Geld für Etwas dar, das kein Brot sei,
und wende seine Arbeit an das, wovon es nicht satt werden könne."'\footnote{Jes.
55,2} Und was war die Ursache davon? -  "`Die Hirten hatten keinen Verstand, ein
Jeder sah auf seinen Weg, und geizte für sich in seinem Stande,"'\footnote{Jes.
56,11} und "`die Priester und Propheten hatten keine wahre
Gesichte,"'\footnote{Ezech. 13} welche auch heutiges Tages nur zu Viele
verachten.

\section{10. Abschnitt} \label{kap12_ab10}

Aber ach! wie viel Thorheit und Mangel an Religion verräth nicht endlich der
Mensch, der von sich selbst eingenommen oder aus irgend Etwas, das er besitzt,
stolz ist? Kann er doch mit allen seinen hohen Gedanken seine Gestalt nicht um
einen Zoll vergrößern! Welche Widerwärtigkeiten kann sein stolzer Sinn abwenden?
Welchern Unfalle kann er abhelfen, welchem Uebel vorbeugen? Er ist nicht
vermögend, nur vor einem der Schläge zu schützen, denen alle Menschen ohne
Auesnahme ausgesetzt sind. Eine Krankheit entstellt, Schmerz und Leiden
verändern die Züge, und der Tod zerstört den ganzen schönen Körperbau auch der
Stolzesten unter den Sterblichen. Ein kleiner Haufen kalter Erde schließt nun
den Leichnam des Mannes ein, dessen hochfahrende Gedanken keine Grenzen kannten.
Seine zarte Person, der -- vielleicht noch vor Kurzem -- kein Ort und keine
Gesellschaft gut genug war, muß sich jetzt in dem engen Raume einer kleinen
finstern Höhle behelfen und sich die Gesellschaft der geringsten Geschöpfe
gefallen lassen, der Würmer nämlich, denen sie bald zur Speise dienen soll. So
nehmen die Stolzen und Prachtliebenden ein gleiches Ende mit allen Andern;
jedoch mit dem Unterschiede, daß sie von den Ueberlebenden weniger bedauert
werden, und im Sterben eine furchtbare und peinvolle Aussicht in die Ewigkeit
haben. Denn so wenig den Stolzen seine vornehme Abkunft vor dem Tode schüzen
kann, eben so wenig ist auch sein Geschlechtsregister vermögend, ihn vor dem
Gerichte zu schützen, daß ihn nach dem Tode erwartet. Die schauerliche Stunde
des Hinscheidens löst alle seine Titel und Ehrenzeichen in ein Nichte auf, und
keine irdische Macht, weder Reichthum noch Hoheit oder Ansehn, ist vermögend,
ihn zu erretten oder in Schuz zu nehmen. "`Wie der Baum fällt, so wird er
liegen,"' und wie der Tod den Menschen verläßt, so findet ihn das Gericht.

\section{11. Abschnitt} \label{kap12_ab11}

Aber ach! \textbf{wie kann man nun einem so elenden Ende vorbeugen?} Was für ein Mittel
giebt es gegen diese bejammernswerthe Abweichung und Entfernung von der Demuth,
Sanftmuth und ächten Frömmigkeit, von jenem heiligen Leben der Gläubigen in den
ersten und reinsten Jahrhunderten des Christenthumes, und von der göttlichen
Kraft, die sich so fühlbar und augenscheinlich sowohl in ihrem mächtigen
Predigen, als auch in ihrem musterhaften Betragen bewies? \textbf{Wahrlich kein anderes,
als daß man zu dem Zeugnisse des Geister; Jesu in sich selbst einkehre, und in
seinem heiligen Lichte den Zustand seinen eigenen Herzens untersuche, und prüfe,
in wiefern man ihm ähnlich sei oder nicht; und daß man zu diesem Zwecke die in
den Urkunden der heiligen Schrift enthaltenen Lehren und Beispiele mit genauer
Aufmerksamkeit betrachte. Christus führte einst selbst die Klage, "`daß daß
Licht in die Welt gekommen sei; daß aber die Menschen die Finsterniß mehr als
das Licht liebten, weil ihre Werke böse wären."'\footnote{Joh. 3,19} Willst du
nun ein Kind Gottes seyn; ein Gläubiger, der an Christum glaubt; so muß du erst
ein Kind des Lichte werden. Du mußt deine Werke an das Licht in deinem Innern
bringen, und sie bei dieser heiligen Lampe deiner Seele prüfen. Denn es ist daß
Licht des Herrn, das dir deinen Stolz und deine Annraßung zu erkennen giebt,} und
dich über das Vergnügen, welchen du an den eitlen Moden und Gebräuchen der Welt
findest, ins Geheim bestraft. \textbf{Wahre Religion ist Seldstverleugnung; ja, und auch
Verleugnung aller selbsterwählten Religion und eigenwilligen Frömmigkeit.} Sie
ist ein starkes Band der Seele, welches die Menschen zu einem heiligen Leben
verbindet, das zur Glückseligkeit führt. Denn dadurch gelangen sie zum Anschauen
Gottes wie Jesus sagt: "`Selig sind die, welche reines Herzenösfindz denn sie
werden Gott schauen."'\footnote{Matth. 5,8} Wer einmal dahin gelangt ist, daß er
das Joch Christi standhaft tragt, der laßt sich von, den Lokungen des Feindes
seiner Seele nicht hinreißen. Er findet höhere Freuden in der Wachsamkeit über
sein Herz und im Gehorsame gegen das Gesetz Christi, als die Vergnügungen der
Welt ihm darbieten können. Wenn die Menschen wirklich das Kreuz Christi, seine
Gebote und Lehren, liebten, so würden sie ihren eigenen Willen kreuzigen, der
sie verleitet, dem heiligen Willen Christi entgegen zu leben, und den Willen des
Feindes ihrer Glückseligkeit zu thun, wodurch sie sich um deas Heil ihrer Seelen
bringen. Hätte Adam im Paradiese nicht auf die Lockpfeiffe der Schlange, sondern
auf daß göttliche Licht in feiner Seele geachtet, und sein Gemuth auf seinen
Schöpfer, den Vergelter der Treue, gerichtet, so wurde er die Schlinge des
Feindes gesehen und ihm in seinen listigen Versuchungen widerstanden haben. O
darum ergötze dich nicht an verbotenen Dingen. Siehe das Böse nicht an, wenn du
nicht davon gefesselt werden willst. Belaste deine Seele nicht mit der Schuld,
gegen deine Erkenntnis zu sündigen. Unterwarf nicht Christus seinen Willen dem
Willen seines himmlischen Vaters? Erduldete er nicht nur der ihm bevorstehenden
Freude Willen das Kreuz? und verachtete er nicht die Schmach und Schande des
neuen unbetretenen Weges zur Herrlichkeit?\footnote{Ebr. 12,2} \textbf{So mußt auch du
deinen Willen dem heiligen Gesetze und Lichte Christi in deinem Herzen
unterwerfen, und um der Belohnung willen, die er dir vorhält, sein Kreuz
erdulden und die Schande desselben nicht achten. Alle möchten gern mit Christo
sich freuen, aber Wenige wollen mit ihm oder um seinentwillen leiden.} Viele sind
bereit, an seinem Tische Theil zu nehmen; Wenigen aber gefällt seine
Enthaltsamkeit. Seiner Austeilung des Brodes wollen sie nachfolgen; aber den
Kelch seiner Todezangst lassen sie stehen. Dieser ist ihnen zu bitter, als daß
sie ihn trinken mochten. -- Manche erheben seine Wunder, und ärgern sich doch an
der Schwäch seines Kreuzes. Darum höre, o ’Mensch! Was er zu deinem Heile für
dich that, das mußt auch du aus Liebe zu ihm thun. Du mußt dich demüthigen, und
es dir gefallen lassen, um seinetwillen Verachtung zu leiden, damit du ihn:
wahrhaft nachfolgen könnest; nämlich nicht auf eine bloß förmliche leblose
Weise, oder nach den Erfindungen und Vorschriften eitler Menschen, sondern, wie
der heilige Geist durch den Apostel sich ausdrückt: "`auf dem neuen und
lebendigen Wege,"'\footnote{Ebr. 10,20} den Jesus geheiligt hat, der Alle, die
auf demselben wandeln, zu der ewigen Ruhe Gottes führet, zu welcher Er, der
allein heilige und glorreiche Erlöser, selbst eingegangen ist.


