\section*{I}

\articlesize

\begin{description}

\item[\textit{I hope so}] Eine Redewendung aus dem Britischen Raum. Wenn in der
$\to$\textit{Geschäftsversammlung} der $\to$\textit{Schreiber}(engl. Clerk) das
Protokoll verliest und die anwesenden fragt, ob es so fest gehaten werden
soll/kann, dann antworten die Anwesenden "`I hope so"' ("`Ich hoffe"') oder
"`yes"', im Sinne von "`I hope that this is the true guidance of the Holy
Spirit"' ("`Ich hoffe, das dies unter der Leitung des Heiligen Geistes steht"').

\item[\textit{Im Licht halten}] $\to$\textit{Hold in the Light}

\item[Independent Quaker] Traditonell sind die Quakerversammlungen (Gemeinden)
sehr locker und autonom organisiert. Die $\to$\textit{Virteljahresversammlungen}
und die $\to$\textit{Jahresversammlungen} haben in erster Linie informellen
Charakter. Trozdem gibt es auch Versammlungen die sich überhauptnicht
überregional organisieren. Diese werden "`Independent Meetings"' genannt. 
\endnote{\texttt{http://www.rcquakers.lomaxes.me.uk/about/about.htm}}

% Solange es Quaker
% gab, solange gab es auch \textit{"`theologische Insellösungen"'}. Ein solches Beispiel
% stellt einer Gruppe wie diese aus Großbritannien, die nicht der $\to$Jahresversammlung
% ($\to$Britain Yearly Meeting) angeschlossen ist da. Entscheidend ist im Grunde immer
% die jeweilige verwendete $\to$\textit{"`Ordnung des Zusammenlebens"'} in diesem Fall
% benutzt das selbständige $\to$Meeting die des \textit{$\to$Ohio Yearly Meeting}, auch
% wenn es Geographisch eigendlich zum \textit{Britain Yearly Meeting} gehört. Der Gruppe
% war in diesem Fall die theologische Ausrichtung der Jahresversammlung zu liberal.

 \item[Innere Bestrafung/Züchtigung] Gerade Willam Penn sahr in dem
 $\to$\textit{innerem Licht} nicht nur die Stimme der Tröstung sondern auch die
 der Anklage. (Siehe: Seite \pageref{kap10_ab10} und
 Seite \pageref{kap14_ab1}) und letzlich auch der Strafe und Züchtigung (in Sinne
 von Erziehung).
 
 \item[Innere Gnade]
 
 \item[Innere Offenbarung] $\to$\textit{Inneres Licht}
 \item[Innere Kraft] $\to$\textit{Inneres Licht}
 \item[Innere Stimme] $\to$\textit{Inneres Licht}
 \item[Innerer Christus] $\to$\textit{Inneres Licht}
 \item[Innerer Judas]
 \item[Inneres Licht]
 \item[Inneres Ohr] $\to$\textit{Inneres Licht}
 \item[Inneres Zeugnis] $\to$\textit{Inneres Licht}
 \item[Innerlich kreuzigen]
 \end{description}

\normalsize