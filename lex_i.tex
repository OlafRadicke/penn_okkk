\section*{I}

\articlesize

\begin{description}

\item[I hope so]
    (British term) during a meeting for worship for business, when the clerk asks those present if they agree with a minute, Friends will usually say "I hope so" rather than "`yes"'. It is meant in the sense of "`I hope that this is the true guidance of the Holy Spirit"'.

 \item[\textit{Im Licht halten}] $\to$\textit{Hold in the Light}

\item[Independent Quaker]
% \endnote{\texttt{http://www.rcquakers.lomaxes.me.uk/about/about.htm}} Solange es Quaker
% gab, solange gab es auch \textit{"`theologische Insellösungen"'}. Ein solches Beispiel
% stellt einer Gruppe wie diese aus Großbritannien, die nicht der $\to$Jahresversammlung
% ($\to$Britain Yearly Meeting) angeschlossen ist da. Entscheidend ist im Grunde immer
% die jeweilige verwendete $\to$\textit{"`Ordnung des Zusammenlebens"'} in diesem Fall
% benutzt das selbständige $\to$Meeting die des \textit{$\to$Ohio Yearly Meeting}, auch
% wenn es Geographisch eigendlich zum \textit{Britain Yearly Meeting} gehört. Der Gruppe
% war in diesem Fall die theologische Ausrichtung der Jahresversammlung zu liberal.

 \item[Innere Bestrafung/Züchtigung]
 \item[Innere Gnade]
 \item[Innere Offenbarung] $\to$\textit{Inneres Licht}
 \item[Innere Kraft] $\to$\textit{Inneres Licht}
 \item[Innere Stimme] $\to$\textit{Inneres Licht}
 \item[Innerer Christus] $\to$\textit{Inneres Licht}
 \item[Innerer Judas]
 \item[Inneres Licht]
 \item[Inneres Ohr] $\to$\textit{Inneres Licht}
 \item[Inneres Zeugnis] $\to$\textit{Inneres Licht}
 \item[Innerlich kreuzigen]
 \end{description}

\normalsize