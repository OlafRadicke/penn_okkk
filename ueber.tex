

\part{Über dieses Buch}


\chapter{Für wen ist dieses Buch}

Diese Ausgabe hat das Ziel, den über 300 Jahre alten Text, dem Leser
so leicht wie möglich zugänglich und verständlich zu machen. Grundlage
des Textes ist eine Übersetzung von 1825\footnote{
Titel: "`Ohne Kreuz keine Krone"' \\
Entstehungsdatum: 1825, \\
Erscheinungsdatum: 1826, \\
Verlag: Georg Uslar, \\
Drucker: Heinrich Gelpke, \\
Erscheinungsort: Pyrmont (Bad-Pyrmont),\\
Übersetzer: ?, \\
Originaltitel: No Kross No Krown, \\
Quelle: \textbf{http://de.wikisource.org/wiki/Ohne\_Kreuz\_keine\_Krone}}. Es
ist nicht Ziel dieser Ausgabe, diesen Text in seinen historischen Original-
zustand wiederzugeben, um z.B. daraus als Quelle zu zitieren. Fehler die im
Urtext enthalten sind -- z.B. fehlerhafte Nummerierungen -- werden hier nicht
übertragen. Wer mit dem Quelltext zu wissenschaftlichen Zwecken arbeiten will,
sei hier auf \texttt{http://de.WikiSource.org} verwiesen, wo er den
Originaltext als Scann findet.

\medskip

Um der Zielstellung (den Text möglichst vielen Lesern zugänglich zu machen)
gerecht zu werden, haben wir eine Reihe von Änderungen und Ergänzungen gemacht.
Zum einen haben wir den Text der modernen Rechtschreibung angepasst. Die
Interpunktion wurde der heutigen Verwendung angepasst. An einigen wenigen Stellen,
wurden die Sätze, zur besseren Verständlichkeit, ein wenig umgestellt. An vielen
Stellen wurden Wörter ersetzt, die heute ungebräuchlich und deshalb
unverständlich sind. Oder wenn sich die Bedeutung von Wörtern geändert hat. So
hat das Wort \textit{Geize} in der Vergangenheit einen starken Bedeutungswandel
erfahren. Früher war in erster Linie das damit gemeint, was wir heute als
\textit{Gier} bezeichnen würden. So haben wir also im kompletten Text
\textit{Geize} durch \textit{Gier} ersetzt\footnote{Wer zu diesem konkreten
Beispiel mehr wissen will, der sei auf den Artikel "`Geiz"' in de.Wikipedia.org
verwiesen}. In der folgenden Tabelle sind einige
weitere Beispiele, für Wörter die wir ersetzt haben, aufgeführt. An allen
Stellen, wo Wörter ersetzt wurden, haben wir das mit einer Fussnote kenntlich
gemacht.

\medskip

\begin{center}
\label{ref:tab_wortersetzungen}
\begin{tabular}{|l|l|} \hline
\textbf{Original}       & \textbf{Ersetzung}            \\ \hline \hline
allezeit                & immer                         \\ \hline
arge                    & böse                          \\ \hline
einerlei                & gleichen                      \\ \hline
figürlicher             & sinnbildlicher                \\ \hline
fleischlichen           & materiellen / weltliche       \\ \hline
Geize                   & Gier                          \\ \hline
gehörig                 & richtig                       \\ \hline
hofartig                & hochmütig                     \\ \hline
lautere                 & erliche                       \\ \hline
reiflich erwägest       & genau prüfst                  \\ \hline
unbillig                & in Irrtum                     \\ \hline
uneingedenk             & ungeachtet                    \\ \hline
Üppigkeit               & Verschwendung / Masslosigkeit \\ \hline
Vergleichung            & Konkurrenz                    \\ \hline
vertilgen               & vernichten                    \\ \hline
Weiber genommen         & geheiratet                    \\ \hline
\end{tabular}
\end{center}
\medskip

\medskip

Als weitere Hilfestellung für den Leser, haben wir noch einige erklärende
Begleittexte verfasst. Im hinteren Teil, gehen wir noch mal in separaten
Abschnitten darauf ein, was es interessantes zur Entstehungsgeschichte des
Textes zu sagen gibt; wer der Autor war und was für ein Leben er führte; und wie
sich das Quakertum seid der Entstehung des Textes weiter entwickelt hat.

\chapter{Wie man das Buch lesen sollte}

Man kann das Buch von vorne nach hinten durchlesen. Man kann auch nur den Text
von W. Penn lesen und den Rest weglassen. Ich bin aber der Meinung, man kann
durchaus auch, mitten in den Text, irgendwo reinspringen. Ich glaube, man
erfährt eine ganze Menge über das Denken der frühen Quaker, auch wenn man nicht
alles oder chronologisch gelesen hat.

\medskip

Für diejenigen, die vom Quakertum nur wenig oder gar nichts wissen, empfehle ich
 schon, mit dem Kapitel über die Quakergeschichte zu beginnen
 (\textit{Seite~\pageref{ref:entwicklung_quakertum}})

\medskip

Für die, die sich nur für bestimmte Aspekte des Quakertum interessieren oder
sich erst mal \textit{Appetit} holen wollen, empfehle ich sich mit der
(chronologischen) Themenübersicht im Anhang zu beschäftigen
(\textit{Seite~\pageref{ref:theme_nuebersicht}}).

\chapter{Typographische Konventionen}

In dem Buch wird der Begriff "`Quaker"' \index{Quaker!Schreibweise} in seiner
englischen original Schreibweise verwendet. Mir scheint es nicht sinnvoll
"`Quäker"' zu schreiben. Englische Begriffe wie "`Safe"' werden im Deutschen
auch nicht "`Säfe"' geschrieben. Ich vermag also kein Sinn darin zu sehen
"`Quäker"' zu schreiben.

\medskip

Es gibt zwei verschiedene Fussnoten in dem Abschnitt des Textes von "`Ohne Kreuz
keine Krone"' Die Fussnoten in normaler Schrift, sind die des Autors, des
Originaltextes -- Also W. Penn (und mit einer Ausnahme: Ludwig Seebohm, der vermutliche Übersetzer). Hingegen werden Fussnoten der Herausgeber in
\texttt{Maschinenschrift kenntlich gemacht}. Das sind meist Anmerkungen zum
textlichen Veränderungen oder Erklärungen.

\medskip

Einige Stellen sind mit \textbf{Fettschrift herausgestellt}. Das soll dem Leser
helfen auf wichtige Kernaussagen aufmerksam zu werden. Es gibt einige langatmige
Passagen, in denen sich W. Penn in Widerlegungen seiner Gegner auslässt, die
heute von ihrer Bedeutung kaum noch von belang sind. So z.B. der Abschnitt
"`10. Kapitel"'~(\textit{Seite~\pageref{kap10}~--~\pageref{kap10_ende}}) über
Ehrentitel, die heute ohnehin nicht mehr gebräuchlich sind. Trotzdem gibt es
(auch hier) einige Sätze, die Dinge sehr prägnant auf den Punkt bringen.
Diese haben wir mit Fettschrift herausgestellt.

\chapter{Nutzungs- und Urheberrechte}
Der Urtext im Original von W. Penn ist \textit{gemeinfrei}. Er ist im Internet
unter dieser Adresse zu finden:

\begin{center}
\texttt{http://de.wikisource.org/wiki/Index:Penn\_Ohne\_Kreuz\_keine\_Krone}
\end{center}

Die Bearbeitungen, Fussnoten, Einleitungen, Anmerkungen, Erläuterungen und
Extra-Kapitel unterstehen dem Urheberrecht von Claus Bernet und Olaf Radicke.





