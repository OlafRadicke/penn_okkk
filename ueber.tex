

\part{Über dieses Buch}


\chapter{Für wen ist dieses Buch?}

Diese Ausgabe hat das Ziel, den über dreihundert Jahre alten Text dem Leser
so leicht wie möglich zugänglich und verständlich zu machen. Grundlage
des Textes ist eine Übersetzung von 1825\footnote{
Titel: "`Ohne Kreuz keine Krone"' \\
Entstehungsdatum: 1825, \\
Erscheinungsdatum: 1826, \\
Verlag: Georg Uslar, \\
Drucker: Heinrich Gelpke, \\
Erscheinungsort: Pyrmont (Bad-Pyrmont),\\
Übersetzer: vermutlich Ludwig Seebohm, \\
Originaltitel: No Cross No Crown, \\
Quelle: \textbf{http://de.wikisource.org/wiki/Ohne\_Kreuz\_keine\_Krone}}. Es
ist nicht Ziel dieser Ausgabe, diesen Text in seinem historischen Originalzustand wiederzugeben, um zum Beispiel daraus als Quelle zu zitieren. Fehler, die im
Urtext enthalten sind -- zum Beispiel fehlerhafte Nummerierungen --, wurden hier nicht
übertragen. Wer mit dem Quelltext zu wissenschaftlichen Zwecken arbeiten will,
sei hier auf \texttt{http://de.WikiSource.org} verwiesen, wo er den
Originaltext als Scan findet.

\medskip

Um der Zielstellung (den Text möglichst vielen Lesern zugänglich zu machen)
gerecht zu werden, haben wir eine Reihe von Änderungen und Ergänzungen gemacht.
Zum einen haben wir den Text der modernen Rechtschreibung angepasst. Die
Interpunktion wurde der heutigen Verwendung angepasst. An einigen wenigen Stellen
wurden die Sätze, zur besseren Verständlichkeit, ein wenig umgestellt. An vielen
Stellen wurden Wörter ersetzt, die heute ungebräuchlich und deshalb
unverständlich sind, oder, wenn sich die Bedeutung von Wörtern geändert hat. So
hat das Wort \textit{Geiz} in der Vergangenheit einen starken Bedeutungswandel
erfahren. Früher war in erster Linie das damit gemeint, was wir heute als
\textit{Gier} bezeichnen würden. So haben wir also im kompletten Text
\textit{Geiz} durch \textit{Gier} ersetzt\footnote{Wer zu diesem konkreten
Beispiel mehr wissen will, der sei auf den Artikel "`Geiz"' in de.Wikipedia.org
verwiesen.}. In der folgenden Tabelle sind einige
weitere Beispiele für Wörter, die wir ersetzt haben, aufgeführt. An allen
Stellen, wo Wörter ersetzt wurden, haben wir das mit einer Fußnote kenntlich
gemacht.

\medskip

\begin{center}
\label{ref:tab_wortersetzungen}
\begin{tabular}{|l|l|} \hline
\textbf{Original}       & \textbf{Ersetzung}            \\ \hline \hline
allezeit                & immer                         \\ \hline
arge                    & böse                          \\ \hline
einerlei                & gleichen                      \\ \hline
figürlicher             & sinnbildlicher                \\ \hline
fleischlichen           & materiellen/weltlichen        \\ \hline
Geiz                    & Gier                          \\ \hline
gehörig                 & richtig                       \\ \hline
hofartig                & hochmütig                     \\ \hline
lautere                 & erliche                       \\ \hline
reiflich erwägest       & genau prüfst                  \\ \hline
unbillig                & in Irrtum                     \\ \hline
uneingedenk             & ungeachtet                    \\ \hline
Üppigkeit               & Verschwendung/Masslosigkeit   \\ \hline
Vergleichung            & Konkurrenz                    \\ \hline
vertilgen               & vernichten                    \\ \hline
Weiber genommen         & geheiratet                    \\ \hline
\end{tabular}
\end{center}
\medskip

\medskip

Als weitere Hilfestellung für den Leser haben wir noch einige erklärende
Begleittexte verfasst. Im hinteren Teil gehen wir noch einmal in separaten
Abschnitten darauf ein, was es Interessantes zur Entstehungsgeschichte des
Textes zu sagen gibt; wer der Autor war und was für ein Leben er führte; und wie
sich das Quakertum seit der Entstehung des Textes weiter entwickelt hat.

\chapter{Wie man das Buch lesen sollte}

Man kann das Buch selbstverständlich von vorne nach hinten durchlesen. Man kann aber auch nur den Text
von William Penn lesen und den Rest weglassen. Ich bin aber der Meinung, man kann
durchaus auch, mitten in den Text, irgendwo hineinspringen. Ich glaube, man
erfährt eine ganze Menge über das Denken der frühen Quaker, auch wenn man nicht
alles oder chronologisch gelesen hat.

\medskip

Für diejenigen, die vom Quakertum nur wenig oder gar nichts wissen, empfehle ich
 schon, mit dem Kapitel über die Quakergeschichte zu beginnen
 (\textit{Seite~\pageref{ref:entwicklung_quakertum}}).

\medskip

Für diejenigen, die sich nur für bestimmte Aspekte des Quakertums interessieren oder
sich erst einmal \textit{Appetit} holen wollen, empfehle ich, sich mit der
(chronologischen) Themenübersicht im Anhang zu beschäftigen
(\textit{Seite~\pageref{ref:theme_nuebersicht}}).

\chapter{Typographische Konventionen}

In dem Buch wird der Begriff "`Quaker"' \index{Quaker!Schreibweise} in seiner
englischen originalen Schreibweise verwendet. Mir scheint es nicht sinnvoll
"`Quäker"' zu schreiben. Englische Begriffe wie "`Safe"' werden im Deutschen
auch nicht "`Säfe"' geschrieben. Ich vermag also kein Sinn darin zu sehen
"`Quäker"' zu schreiben.

\medskip


Es gibt zwei verschiedene Fußnoten in dem Abschnitt des Textes von "`Ohne Kreuz
keine Krone"'. Die Fußnoten in normaler Schrift sind die des Autors des
Originaltextes -- also William Penn (und mit einer Ausnahme: Ludwig Seebohm, der vermutliche Übersetzer).
Hingegen werden Fußnoten der Herausgeber in
\texttt{Maschinenschrift kenntlich gemacht}. Das sind meist Anmerkungen zu
textlichen Veränderungen oder Erklärungen. Bibelzitate werden durch einen kleine Raute $\diamond$\marginpar{\footnotesize{Rand\-bemerkung}} makiert und am Rand wird die genaue Stelle bezeichnet.

\medskip

Einige Stellen sind mit \textbf{Fettschrift herausgestellt}. Das soll dem Leser
helfen, auf wichtige Kernaussagen aufmerksam zu werden. Es gibt einige langatmige
Passagen, in denen sich Penn in Widerlegungen seiner Gegner auslässt, die
heute von ihrer Bedeutung kaum noch von Belang sind. So zum Beispiel der Abschnitt im
"`10. Kapitel"'~(\textit{Seite~\pageref{kap10}~--~\pageref{kap10_ende}}) über
Ehrentitel, die heute ohnehin nicht mehr gebräuchlich sind. Trotzdem gibt es
(auch hier) einige Sätze, die die Dinge sehr prägnant auf den Punkt bringen.
Diese haben wir mit Fettschrift herausgestellt.

\chapter{Nutzungs- und Urheberrechte}
Der Urtext im Original von William Penn ist \textit{gemeinfrei}. Er ist im Internet
unter dieser Adresse zu finden:

\begin{center}
\texttt{http://de.wikisource.org/wiki/Index:Penn\_Ohne\_Kreuz\_keine\_Krone}
\end{center}

Dasselbe gilt für die Bilder, die für den Umschlag verwendet wurden. Sie sind zu finden unter:

\begin{center}
\texttt{http://commons.wikimedia.org/wiki/File:William\_Penn.png}\\
\texttt{http://www.loc.gov/pictures/item/99401118/}\\
und\\
\begin{footnotesize}
\texttt{http://commons.wikimedia.org/wiki/File:Treaty\_of\_Penn\_with\_Indians\_by\_Benjamin\_West.jpg}\\
\end{footnotesize}
\end{center}

Die Bearbeitungen, Fußnoten, Einleitungen, Anmerkungen, Erläuterungen und
Extra-Kapitel unterstehen dem Urheberrecht von Claus Bernet und Olaf Radicke.

Sollte dieser Text länger als ein Jahr nicht mehr im regulären Buchhandel
erhältlich sein, fallt der vollständige Text mit seinen Bearbeitungen von
Claus Bernet und Olaf Radicke unter die Creative Commons Licence (Namensnennung - Weitergabe unter
gleichen Bedingungen) in Version 4.0 oder höher der International Lizenz. Details
unter: 

\begin{center}
 \begin{footnotesize}
  \texttt{http://creativecommons.org/licenses/by-sa/4.0/}
 \end{footnotesize}
\end{center}

Der Source Code ist unter der dieser URL hinterlegt:

\begin{center}
 \begin{footnotesize}
\texttt{https://github.com/OlafRadicke/penn\_okkk/tree/auflage\_04} 
 \end{footnotesize}
\end{center}




