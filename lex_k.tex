\section*{K}

\articlesize

\begin{description}


 \item[Keith-Kontroverse] $\to$\textit{George Keith}.

 \item[Klassen, Hans] $\ast$30.9.1893 \dag18.9.1959 Fällt durch seinen sein bizarrer Lebensweg etwas aus dem Ramen. Klassen war Rußlandmennonit, Lebensreformer und Bürokrat, Nationalsozialist und Kommunist und Quäker.\endnote{Mennonitischen Geschichtsblätter 2009, 66. Jahrgang. von Claus Bernet.}


 \item[Kryptoquäker] Von griechisch "`krypto"': geheim, verborgen. Ein Quäker, der sich nicht öffentlich als Quäker zu erkennen gibt, also ein "`Geheimquäker"'. Das gab es gelegentlich im 17. Jahrhundert, zu Zeiten schwerer Verfolgung, als nicht alle Quäker den Mut hatten, zu ihrer Gemeinschaft zu stehen. Verdächtigen Personen, die irgendwie als Quäker erschienen, wurde auch vorgeworfen, "`Kryptoquäker"' zu sein.\endnote{Verwendungsbeispiel: Claus Bernet, 2007, "`Deutsche Quäkerschriften. Band 2: Deutsche Quäkerschriften des 18. Jahrhunderts"', ISBN 9783487134086, ISBN 348713408X, Seite 371}

 \item[Kulturquakertum] $\to$\textit{Wharton, Joseph}

 \end{description}
\normalsize
