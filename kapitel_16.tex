
\chapter{16. Kapitel} \label{kap16}
\section{Zusammenfassung des 16. Kapitel}
\small
\begin{description}
\item[1. Abschnitt] Die Christen dürfen die üppigen Moden und Gebräuche der Welt
nicht mitmachen, weil sie mit dem Geiste des Christentums unverträglich sind. (Seite: \pageref{kap16_ab1})
\item[2. Abschnitt] Was für einen Kelch die wahren Jünger Jesu zu trinken haben. (Seite: \pageref{kap16_ab2})
\item[3. Abschnitt] O! Wer will diesen Kelch trinken! (Seite: \pageref{kap16_ab3})
\item[4. Abschnitt] Beantwortung einer Einwendung in Ansehung der Beschaffenheit
des Reichs Gottes. - Worin dasselbe bestehe.(Seite: \pageref{kap16_ab4})
\item[5. Abschnitt] Gemühtsbeschaffenheit der Nachfolger Christi.(Seite: \pageref{kap16_ab5})

\end{description}
\normalsize

\section{1. Abschnitt} \label{kap16_ab1}

Die üppigen Moden, Gebräuche und Ergötzungen der Welt, denen in dieser
Abhandlung widersprochen und entgegengearbeitet wird, sollten unter den Christen
durchaus nicht geduldet werden; weil der Geist, der sie erfunden und eingeführt
hat, der ein so großes Vergnügen in ihnen findet, und sie daher so sorgfältig in
Schutz nimmt, mit dem Geiste und Wesen der wahren christlichen Religion ganz
unvereinbar ist. Denn der Zweck, warum "`ewiges Leben und Unsterblichkeit ans
Licht gebracht sind,"' ist der: daß die Menschen dadurch bewogen und fähig
gemacht werden sollen, alle jene verderblichen Freuden und Vergnügungen den
sterblichen Lebens der Welt zu verleugnen und zu verlassen; und darum sind ihnen
so große und unermeßliche Belohnungen und ewige Wohnungen des Friedens
verheißen, daß sie dadurch aufgemuntert und willig gemacht werden möchten, sich
von den eitlen, sinnlichen Ergötzungen der Welt zu trennen, und mit Muth und
Standhaftigkeit aller Schmach und allen Leiden entgegenzugehen, welche sie
deshalb, vielleicht selbst von Seiten ihrer vertrautesten Freunde und nächsten
Verwandten, zu erwarten haben.\footnote{Luk. 16,15. Joh. 15,18. 19. Joh. 16,20.
Joh. 17, 15. 16. Ebr. 11, 14. 25. 26. 27. Röm. 8,19 2 Tim. 1,10 Ebr. 12,1. 2. }

\medskip

Wenn die christliche Religion es gestattete, daß der Genuß dieser Welt sich
weiter als auf den bloßen Gebrauch der Dinge erstrecke, die Gott zur
Befriedigung der Bedürfnisse seiner ganzen Schöpfung gegeben hat; wenn sie, z.
B. all den Stolz, die Eitelkeit, Üppigkeit und Pracht in Kleidern, die vielen
verschwenderischen Moden, die schmeichelhaften Ehrenbezeigungen und
Auszeichnungen, die verschiedenen Lustbarkeiten der Welt und Alles, was der
Sinnlichkeit nur behagen und schmeicheln kann, den Menschen erlaubte; wozu
bedurfte es dann eines täglichen Kreuzes und eines Lebens der Selbstverleugnung?
Was hatten wir nöthig, "`unsere Seligkeit mit Furcht und Zittern zu schaffen,
nach Dem zu trachten, das droben ist, und unsern Schaz und unser Herz im Himmel
zu haben?"' Warum sollten wir uns denn auch jedes unnützen Geschwätzes, jedes
eitlen Scherzes enthalten, und alle Tage in heiliger Furcht und Wachsamkeit
leben? Und warum so viel Tadel, Schmach Verfolgung und harte Behandlung
erdulden? Wozu sollte alles dieses dienen, und wie könnte es durchaus
erforderlich seyn, um zu jener ewigen Herrlichkeit und unverganglichen Krone
desLebens zu gelangen, wenn die Religion Jesu die Eitelkeit, den Stolz, die
Verschwendung, den Müßiggang, die Ueppigkeit, den Geiz, den Neid und die
Arglist, kurz, die ganze jetzt unter den Bekennern des Christenthumes
herrschende Lebensweise billigte und erlaubte? Aber nein! so ist es nicht.
Sondern, da der Herr Jesus Christus wohl wußte, wie sehr die Gemüther der
Menschen sowohl thörichten Kleinigkeiten, als auch groben Ausschweifungen
nachhingen, und wie sie von dem himmlischen Einflusse des Lebens so weit
abgewichen wahren, daß sie nicht allein den unerlaubten Genüssen der Welt
begierig nachtrachteten, sondern auch täglich neue Mittel erfanden, um ihre
sinnlichen Neigungen zu befriedigen; so sahe er auch sehr wohl die
Schwierigkeiten voraus, welche Alle zu bekämpfen haben würden, wenn sein Ruf:
alle diese Dinge zu verlassen, an sie ergehen würde, und wie ungern sie sich von
ihnen trennen und entwöhnen wurden. Um aber die Menschen dazu aufzumuntern und
zu bewegen, führt er nicht eine Sprache, wie unter dem Gesetze nöthig war. Er
verheißt kein irdisches Canaan, nicht hohes Ansehen, zahlreiche Nachkommnschaft,
langes Leben und dergleichen; nein, eher daß Gegentheil; höchstens nur den
mäßigen Gebrauch der irdischen Dinge in ihrem gewöhnlichen Laufe. Seine
Einladungen sind höherer Art; er versichert Allen, die an ihn glauben und ihm
Gehorsam leisten, ein ewiges Reichs und eine unvergängliche Krone, die weder
Zeit noch Grausamkeit, weder Tod noch Grab, oder alle Werkzeuge der Zerstörung,
jemals im Stande seyn werden, ihnen zu vereiteln oder zu entreißen. Ueberdies
verspricht er ihnen, daß er sie in ein inniges Freundschaftsverhältniß, ja, in
die genaueste göttliche Verwandtschaft als geliebte Brüder aufnehmen und zu
Miterben aller seiner himmlischen Seligkeit und unvergänglichen Herrlichkeit
machen wolle. -- Möge es daher wohl erwogen werden, was wir als Wahrheit
beurkundet finden, "`daß wenn Jene, die Moses nicht hören wollten, sterben
mußten, noch vielweniger Diejenigen entfliehen werden, die sich weigern, die
Vorschriften und Gebote des ewigen Vergelters zu befolgen, der Allen, die ihn
fleißig suchen und ihm Gehorsam sind, so große Belohnungen verheißen
hat."'\footnote{Luk. 6,20. Kap. 12,32. Kap. 22,29. Koloss. 1,13. 1 Thes. 2,12
Ebr. 12,28. Jak. 2,5 Joh. 15,14. 15. Röm. 8,17 Ebr. 2,11  Kap. 12,2. 1 Petri
2,21. Luk. 12, 29-31 2 Tim. 4,8 Matth. 19, 27-29. Luk. 6.22. 23 Ebr. 11,6 Kap.
12,25}

\section{2. Abschnitt} \label{kap16_ab2}

Darum gefiel es ihm, durch sein eigenes Beispiel uns einen Begriff von dem
Kelche zu geben, den seine Nachfolger in großem Maße zu trinken erwarten müssen;
nämlich den Kelch der Selbstverleugnung, harter Prüfungen und bitterer Leiden.
Denn er hat den Weg zur ewigen Ruhe nicht so für uns eingeweihet und ihn uns
nicht auf die Art beschieden, daß wir mit Gold, Silber, Juwelen, Bändern,
Spitzen und Stickereien behängt, in prachtvollea Modeanzügen, und unter
abwechselnden Lustbarkeiten und angenehmen Zeitvertreiben ungestört auf
demselben einhertreten können; ach nein! sondern so, daß wir alle solche
weltliche Freuden und thörichte Unterhaltungen verlassen, ja, zuweilen sogar
erlaubte Genüsse verleugnen, und sowohl den Verlust unserer Güter, als auch den
Spott und die Verachrung der Unwissenden und die ungereehte Behandlung unserer
grausamen Verfolger mit freudiger Ergebung erdulden müssen.\footnote{Matth.
10,37. 38. } Wäre der Gebrauch und Genuß weltlicher Pracht und Freuden der Natur
und Eigenschaft seines Reiches angemessen gewesen, so würde es ihm gewiß an
einer großen Verschiedenheit solcher Vergnügungen nie gemangelt haben; da er, --
wie auch seinen Nachfolgern wiederfährt, -- mit nichts Geringerin, als mit der
Herrlichkeit der ganzen Welt versucht ward. Ader Er, der seinen Jüngern befahl,
"`ein anderers Vaterland zu suchen, und sich Schätze für den Himmel zu sammeln,
die nie vergehen;"' der ihnen gebot, "`nicht ängstlich zu sorgen, was sie essen,
oder trinken, oder anziehen sollten, weil dieses die Sorgen der Heiden wären,
die Gott nicht kenneten;"' – daher denn auch die Christen, die ihn zu kennen
vorgaben, sich mit Nahrung und Kleidern begnügen müssen;\footnote{1 Tim.
6,6-11.} -- Er, der Herr Jesus Christus, sag’ ich, entsagte Allem, und schärfte
durch ein himmlisches Beispiel seine Lehre ein, die in der bestimmten Erklärung
bestand, daß Jeder, "`der sein Jünger seyn wolle, dasselbe Kreuz aufnehmen und
ihn nachfolgen müsse."'\footnote{Luk. 14,26. 27. 33.}

\section{3. Abschnitt} \label{kap16_ab3}

Und o! Wer will ihm folgen? Wer will ein wahrer Christ seyn? -- Wir dürfen nicht
denken, wir könnten einen andern Weg einschlagen, als der Herzog unserer
Seligkeit nahm,\footnote{Ebr. 2,10} oder einen andern Kelch trinken, als er
trank. O! nein! denn als die beiden Söhne des Zebedäus, Jakobus und Johannes, zu
seiner Rechten und Linken in seinem Reiche zu sizen verlangten, legte er ihnen
die Frage vor: "`Könnet ihr den Kelch trinken, den ich trinken werde, und euch
mit der Taufe taufen lassen, womit ich getauft werde?"'\footnote{Matth. 20,22}
Auf andere Weise kann Niemand ein Jiniger Jesu oder ein wahrer Christ seyn. Wer
also zu Christo kommen und ein wirklicher Christ werden will, der muß freiwillig
allen Vergnügnngen entsagen, die seine Gemüthsneigungen gefangen nehmen und dem
Einflusse des göttlichen Geistes entziehen können; ja, er muß jeder geliebten
Eitelkeit, – und alles Vergängliche unter der Sonne ist eitel, wenn es mit dem
Ewigen und Unvergänglichen verglichen wird, - auf immer den Abschied geben.

\section{4. Abschnitt} \label{kap16_ab4}

Es giebt jedoch Menschen, welche für die Befriedigung ihrer verderbten Neigungen
sogar einen Grund in der Schrift finden wollen, wiewohl sie dieselbe offenbar
unrichtig anwenden, wenn sie den Einwurf machen: das Reich Gottes bestehe nicht
in Essen und Trinken oder Kleidern, etc. Ich antworte ihnen: Ihr habt darin ganz
recht; und eben deswegen haben wir auch das Üederflüssige in diesen Dingen
abgeschafft. Aber gewiß hat Niemand weniger ein Recht, uns diese Einwendung zu
machen, als ihr, die ihr, ja solche Dinge für so unentbehrlich haltet, daß ihr
uns deswegen nur um so mehr euren Tadel empfinden lasset, weil wir uns euch
hierin nicht gieichstellen wollen. Ja wie fern ihr nun aber hierin christlich
handelt, oder ob euer Betragen hierin mit der Gerechtigkeit, mit dem Frieden und
mit der Freude im heiligen Geiste, worin das Reich Gottes bestehet,
übereinstimmet, das lasset den gerechten Zeugen der Wahrheit in eurem Gewissen
entscheiden. Unser Betragen in diesen Stücken gründet sich auf Maßigkeit, und
diese bestehet mit der Gerechtigkeit, durch welche wir jenes Reich erlangt
haben, wovon eure Unmaßigkeit und eure Auschweifungen euch ausschließen. Denn,
wenn nur Diejenigen wahre Jünger Jesu seyn können, welche sein Kreuz täglich
aufnehmen, und nur Solche es wirklich tragen, die dem Beispiele ihres theuern
Herrn in seiner Taufe, in seinen Leiden und in seinen Versuchungen treulich
folgen; wenn nur Diejenigen seine Taufe kennen und erfahren haben, welche ihre
Gemüther von den Eitelieiten und Thorheiten, in welchen der große Haufe der Welt
lebt, zurückgezogen haben, und dem heiligen Lichte der göttlichen Gnade gehorsam
geworden sind, das ihre Seelen erleuchtet und sie täglich in der Uebung erhält,
jede ihm widerstrebende Neigung zu kreuzigen, und den Gegenständen der
Unsterblichkeit nachzutrachten;\footnote{Phil. 3,10. 1 Petri 4,13. Tit. 2,11
Joh. 1,9 Röm. 6,6 Gal. 2,20 Kap. 5,24. Kap. 6,4} -- wenn nur Solche wahre Jünger
Jesu sind, wie denn ohne Zweifel keine Andere es seyn können: so müssen gewiß
die Menschen Unserer Zeit, wenn sie nur ein wenig ernsilich nachdenken wollen,
die richtige Schlußfolge ziehen, daß Niemand, der sein Vergnügen in den eitlen
Gebräuchen der Welt sucht, und einen dem Leben Christi so ganz unähnlichen
Lebenswandel führt, ein wahrer Christ oder Nachfolger des gekreuzigten Jesu sehn
könne. Denn worin bestände sonst dar Kreuz? oder was machte das christliche
Leben denn so schwer und zu einem Gegenstande des Spottes und der Verachtung der
Welt? Und stände das Christenthum nicht im Widerspruche mit jenen üppigen Moden,
Gebräuchen und eitlen Ergözungen der Welt, so würde "`das Ärgerniß des Kreuzes"'
bald aufhören;\footnote{Gal. 5,11} des Kreuzes, das Denen, die da glauben und
selig werden, "`eine Kraft Gottes ist,"'\footnote{1 Kor. 1,17. 18.} wodurch jede
verderbte und eitle Neigung überwunden und das Gemüth der Menschen zu einer
heiligen Unterwerfung unter den Willen seines himmlischen Schöpfers gebracht
wird. Darum ward gesagt, daß Jesus Christus geoffenbaret worden sei, und noch
geoffenbaret werde, damit er durch sein heiliges Leben, seine Lehre und sein
Beispiel der Selbstverleugnung, den Stolz, des menschlichen Geistes zu Schanden
mache,\footnote{Kor. 1,27-29.} und durch dass ewige "`Leben und undfergängliche
Wesen,"' welches er ans Licht gebracht hat, und noch täglich ans Licht bringt,
die vergängliche Herrlichkeit der irdischen Ruhe und Freude
verdunkele;\footnote{1 Kor. 1,27. 28. 29.}. so daß die Menschen, wenn ihre
Gemüther davon entwöhnet und der Welt gekreuzigt sind, ein auderes Vaterland
suchen und ein ewiges Erbe erlangen mögen. "`Denn was sichtbar ist, das ist
zeitlich,"'\footnote{2 Kor. 4,18} und davon werden alle wahre Christen erlöset,
so daß sie keine Ruhe darin suchen noch finden; "`was aber unsichtbar ist, das
ist ewig;"' darauf sind die Neigungen ihrer Herzen vornehmlich gerichtet, und
darin finden sie Ruhe.

\section{5. Abschnitt} \label{kap16_ab5}

Ein wahrer Jünger des Herrn Jesu Christi sucht daher sein Gemüth so sehr an den
Umgang mit himmlischen Gegenständen zu gewöhnen, daß ihm die Dinge dieser Welt
ganz gleichgültig werden, und er sie gebraucht, als wenn er sie nicht
gebrauchte. Er läßt sich mit dem, was er zur Nothdurft bedarf, begnügen, ohne
den Ueberfluß der Welt zu begehren,\footnote{1 Tim. 6,8.} und dabei wird ihm das
Vergnügen, welches er zur Zeit der Unwissenheit in der Gleichstellung der Welt
zu finden glaubte, in dem verborgenen himmlischen Leben mit Jesu überflüssig
ersetzt. Wer aber nicht in ihm bleibt, der kann unmöglich alle die Früchte
hervorbringen, die er von seinen Nachfolgern erwartet, und wodurch sein Vater
verherrlicht wird.\footnote{Joh. 15, 4. 7. 8.} So wie es nun aber klar erhellet,
daß Alle, die in den eitlen Gebrächen, Ergötzungen und Lüsten der Welt leben,
weder in Christo bleiben noch ihn kennen; - denn wer ihn kennt, verläßt den Weg
der Ungerechtigkeit; -- so leuchtet es auch deutlich ein, daß ihr Bleiben und
Ergötzen in jenen bezaubernden Thorheiten die wahre Ursache ist, warum sie ihn
nicht kennen und nahe fühlen, der doch beständig "`an den Thüren ihrer Herzen
stehet und anklopft,"'\footnote{Offenb. 3,20} und dem sie den Eingang nicht
verweigern sollten, damit sie seine göttliche Kraft als das heilige Kreuz
erkenneten, wodurch jede Lieblingslust und jede verführerische Eitelkeit
gekreuzigt und ertödtet wird, und so zu der seligen Erfahrung gelangten, daß sie
das göttliche Leben ihren Herzen emporsteigen fühlten, ein Verlangen nach
himmlischen Dingen erwecket fänden, und eine gegründete Hoffnung hätten, "`daß,
wenn Christus, ihr Leben, sich offenbaren wird, auch sie mit ihm in der
Herrlichkeit offenbar werden sollen, der da ist Gott über Alles, gelobet in
Ewigkeit! Amen!"'\footnote{Kol. 3,1-4. Röm. 9,5.}
