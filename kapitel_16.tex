\chapter{Kapitel 16} \label{kap16}
\section{Zusammenfassung des 16. Kapitels}

\begin{description}
\item[1. Abschnitt] Die Christen dürfen die maßlosen\editnote{'üppigen'
ersetzt durch 'maßlosen'.}
Moden und Gebräuche der Welt
nicht mitmachen, weil sie mit dem Geist des Christentums unverträglich sind.
\dotfill \textit{Seite~\pageref{kap16_ab1}}\\
\item[2. Abschnitt] Was für einen Kelch die wahren Jünger Jesu zu trinken haben.\\
.\dotfill \textit{Seite~\pageref{kap16_ab2}}\\
\item[3. Abschnitt] O! Wer will diesen Kelch trinken?
\dotfill \textit{Seite~\pageref{kap16_ab3}}\\
\item[4. Abschnitt] Beantwortung einer Einwendung in Ansehung der Beschaffenheit
des Reichs Gottes. -- Worin dasselbe besteht.
\dotfill \textit{Seite~\pageref{kap16_ab4}}\\
\item[5. Abschnitt] Gemütsbeschaffenheit der Nachfolger Christi.
\dotfill \textit{Seite~\pageref{kap16_ab5}}\\

\end{description}

\newpage

\section{1. Abschnitt} \label{kap16_ab1}

Die maßlosen\editnote{'üppigen' ersetzt durch 'maßlosen'.}
Moden, Gebräuche und Ergötzungen der Welt, denen in dieser
Abhandlung widersprochen und entgegengearbeitet wird, sollten unter den Christen
durchaus nicht geduldet werden, weil der Geist, der sie erfunden und eingeführt
hat, der ein so großes Vergnügen an ihnen findet und sie daher so sorgfältig in
Schutz nimmt, mit dem Geist und Wesen der wahren christlichen Religion ganz
unvereinbar ist. Denn der Zweck, warum
\textit{"`ewiges Leben und Unsterblichkeit ans
Licht gebracht sind,"'} ist der: Dass die Menschen dadurch bewogen und fähig
gemacht werden sollen, alle jene verderblichen Freuden und Vergnügungen des
sterblichen Lebens der Welt zu überwinden\editnote{'verleugnen' ersetzt
durch 'überwinden'.}
und zu verlassen, und darum sind ihnen
so große und unermeßliche Belohnungen\index{Belohnung} und ewige
Wohnungen\index{Wohnung!ewige} des Friedens
verheißen, dass sie dadurch aufgemuntert und willig gemacht werden möchten, sich
von den eitlen, sinnlichen Ergötzungen der Welt zu trennen und mit Mut und
Standhaftigkeit aller Schmach\index{Schmach} und allen Leiden\index{Leid}
entgegen zu gehen, welche sie
deshalb, vielleicht selbst von Seiten ihrer vertrautesten Freunde und nächsten
Verwandten, zu erwarten haben.\bibmarginpar{
Lukas 16,15)
Johannes 15,18+19)
Johannes 16,20)
Johannes 17,15+16.)
Hebräer 11,14+25-27)
Römer 8,19)
2.~Timotheus 1,10)
Hebräer 12,1+2.}
\bibleindex{Lukas 16}
\bibleindex{Johannes 15}
\bibleindex{Johannes 16}
\bibleindex{Johannes 17}
\bibleindex{Hebräer 11}
\bibleindex{Römer 08@Römer 8}
\bibleindex{Timotheus 2 01@2. Timotheus 1}
\bibleindex{Hebräer 12}


\medskip

Wenn die christliche Religion es gestattete, dass der Genuss dieser Welt sich
weiter als auf den bloßen Gebrauch der Dinge erstreckt, die Gott zur
Befriedigung der Bedürfnisse seiner ganzen Schöpfung gegeben hat, wenn sie, zum Beispiel
all den Stolz, die Eitelkeit, Maßlosigkeit\editnote{'Üppigkeit' ersetzt
durch 'Maßlosigkeit'.}
und Pracht in Kleidung, die vielen
verschwenderischen Moden, die schmeichelhaften Ehrenbezeigungen und
Auszeichnungen, die verschiedenen Lustbarkeiten der Welt und alles, was der
Sinnlichkeit nur behagen und schmeicheln kann, den Menschen erlaubt, wozu
bedürfte es dann eines täglichen Kreuzes\index{Kreuz!tägliches} und eines Lebens
der Selbstüberwindung\editnote{'Selbstverleugnung' ersetzt durch
'Selbstüberwindung'.}?
Was hätten wir nötig,
\textit{"`unsere Seligkeit mit Furcht und Zittern zu
erarbeiten\editnote{'schaffen' ersetzt durch 'erarbeiten'.},
nach dem zu trachten, das droben ist, und unseren Schatz und unser Herz im
Himmel
zu haben?"'} \label{ref:16_01_warum}
\textbf{Warum sollten wir uns denn auch jedes unnützen Geschwätzes, jedes
eitlen Scherzes enthalten, und alle Tage in heiliger Furcht und Wachsamkeit
leben? Und warum so viel Tadel, Schmach, Verfolgung und harte Behandlung
erdulden? Wozu sollte alles dieses dienen, und wie könnte es durchaus
erforderlich sein, um zu jener ewigen Herrlichkeit und unvergänglichen Krone
des Lebens\index{Krone!des Lebens} zu gelangen, wenn die Religion Jesu die
Eitelkeit, den Stolz, die
Verschwendung, den Müßiggang, die Maßlosigkeit\editnote{'Üppigkeit'
ersetzt durch 'Maßlosigkeit'.},
den Geiz, den Neid und die
Arglist, kurz, die ganze jetzt unter den Bekennern des Christentums
herrschende Lebensweise billigte und erlaubte? Aber nein, so ist es nicht.}
Sondern, da der Herr Jesus Christus wohl wusste, wie sehr die Gemüter der
Menschen sowohl törichten Kleinigkeiten als auch groben Ausschweifungen
nachhingen, und wie sie von dem himmlischen Einflusse des Lebens so weit
abgewichen waren, dass sie nicht allein den unerlaubten Genüssen der Welt
begierig nachtrachteten, sondern auch täglich neue Mittel erfanden, um ihre
sinnlichen Neigungen zu befriedigen, so sah er auch sehr wohl die
Schwierigkeiten voraus, welche alle zu bekämpfen haben würden, wenn sein Ruf,
alle diese Dinge zu verlassen, an sie ergehen würde, und wie ungern sie sich von
ihnen trennen und entwöhnen würden. Um aber die Menschen dazu aufzumuntern und
zu bewegen, führt er nicht eine Sprache, wie sie unter dem Gesetze nötig war. Er
verheißt kein irdisches Canaan\index{Orte:!Canaan}, nicht hohes
Ansehen, zahlreiche Nachkommenschaft,
langes Leben und dergleichen, nein, eher das Gegenteil; höchstens nur den
mäßigen Gebrauch der irdischen Dinge in ihrem gewöhnlichen Laufe. Seine
Einladungen sind höherer Art,
\label{ref:16_01_warum_2}
\textbf{er versichert allen, die an ihn glauben und ihm
Gehorsam leisten, ein ewiges Reich\index{Reich!ewiges} und eine unvergängliche
Krone\index{Krone!unvergängliche}, die weder
Zeit noch Grausamkeit, weder Tod noch Grab oder alle Werkzeuge der Zerstörung
jemals im Stande sein werden, ihnen zu vereiteln oder zu entreißen. Überdies
verspricht er ihnen, dass er sie in ein inniges Freundschaftsverhältnis, ja, in
die genaueste göttliche Verwandtschaft als geliebte Brüder aufnehmen und zu
Miterben aller seiner himmlischen Seligkeit\index{Seligkeit} und unvergänglichen
Herrlichkeit
machen wolle.}-- Möge es daher wohl erwogen werden, was wir als Wahrheit
beurkundet finden,
\textit{"`dass wenn jene, die Moses nicht hören wollten, sterben
mussten, noch viel weniger diejenigen entfliehen werden, die sich weigern, die
Vorschriften und Gebote des ewigen Vergelters zu befolgen, der allen, die ihn
fleißig suchen und ihm Gehorsam sind, so große Belohnungen verheißen
hat."'}\bibmarginpar{
Lukas 6,20)
Lukas 12,32)
Lukas 22,29)
Kolosser 1,13)
1.~Thessalonicher 2,12)
Hebräer 12,28)
Jakobus 2,5)
Johannes 15,14+15)
Römer 8,17)
Hebräer 2,11)
Hebräer 12,2)
1.~Petrus 2,21)
Lukas 12,29-31)
2.~Timotheus 4,8)
Matthäus 19,27-29)
Lukas 6,22+23)
Hebräer 11,6)
Hebräer 12,25.}
\bibleindex{Lukas 06@Lukas 6}
\bibleindex{Lukas 12}
\bibleindex{Lukas 22}
\bibleindex{Kolosser 01@Kolosser 1}
\bibleindex{Thessalonicher 1 02@1. Thessalonicher 2}
\bibleindex{Hebräer 12}
\bibleindex{Hebräer 02@Hebräer 2}
\bibleindex{Hebräer 11}
\bibleindex{Jakobus 02@Jakobus 2}
\bibleindex{Johannes 15}
\bibleindex{Römer 08@Römer 8}
\bibleindex{Petrus 1 02@1. Petrus 2}
\bibleindex{Timotheus 2 04@2. Timotheus 4}
\bibleindex{Matthäus 19}

\section{2. Abschnitt} \label{kap16_ab2}

Darum gefiel es ihm, durch sein eigenes Beispiel uns einen Begriff von dem
Kelch\index{Kelch!trinken} zu geben, den seine Nachfolger in großem Maß zu
trinken erwarten müssen,
nämlich den Kelch der Selbstüberwindung\editnote{'Selbstverleugnung'
ersetzt durch 'Selbstüberwindung'.},
harter Prüfungen\index{Prüfung} und bitterer Leiden\index{Leid}.
Denn er hat den Weg zur ewigen Ruhe\index{Ruhe!ewige} nicht so für uns
eingeweiht und ihn uns
nicht auf die Art beschieden, dass wir mit Gold, Silber, Juwelen, Bändern,
Spitzen und Stickereien behängt, in prachtvollen Modeanzügen und unter
abwechselnden Lustbarkeiten und angenehmen Zeitvertreiben ungestört auf
demselben einhertreten können; ach nein, sondern so, dass wir alle solche
weltliche Freuden und törichte Unterhaltungen verlassen, ja, zuweilen sogar
erlaubte Genüsse überwinden\editnote{'verleugnen' ersetzt durch
'überwinden'.},
und sowohl den Verlust unserer Güter als auch den
Spott und die Verachtung der Unwissenden und die ungerechte Behandlung unserer
grausamen Verfolger\index{Verfolgung} mit freudiger Ergebung erdulden
müssen.\biblecite{Matthäus 10,37+38.}{Matthäus 10}
Wäre der Gebrauch und Genuss weltlicher Pracht und Freuden der Natur
und Eigenschaft seines Reiches angemessen gewesen, so würde es ihm gewiss an
einer großen Verschiedenheit solcher Vergnügungen nie gemangelt haben, da er, --
wie auch seinen Nachfolgern widerfährt,~-- mit nichts Geringerem, als mit der
Herrlichkeit der ganzen Welt versucht war. Aber er, der seinen Jüngern befahl,
\textit{"`ein anderes Vaterland zu suchen und sich Schätze für den Himmel zu
sammeln, die nie vergehen;"'} der ihnen gebot,
\textit{"`nicht ängstlich zu sorgen, was sie essen,
trinken oder anziehen sollten, weil dieses die Sorgen der Heiden wären,
die Gott nicht kennen;"'} -- daher denn auch die Christen, die ihn zu kennen
vorgaben, sich mit Nahrung und Kleidern begnügen müssen,\biblecite{1.~Timotheus
6,6-11.}{Timotheus 1 06@1. Timotheus 6}
-- Er, der Herr Jesus Christus, sag’ ich, entsagte allem und schärfte
durch ein himmlisches Beispiel seine Lehre ein, die in der bestimmten Erklärung
bestand, dass jeder,
\textit{"`der sein Jünger sein wolle, dasselbe Kreuz aufnehmen und
ihm nachfolgen müsse."'}\biblecite{Lukas 14,26+27+33.}{Lukas 14}

\section{3. Abschnitt} \label{kap16_ab3}

Und o! Wer will ihm folgen? Wer will ein wahrer Christ sein? -- Wir dürfen nicht
denken, wir könnten einen anderen Weg einschlagen als der Herzog unserer
Seligkeit nahm,\biblecite{Hebräer 2,10.}{Hebräer 02@Hebräer 2}
oder einen anderen Kelch trinken\index{Kelch}, als er
trank. O nein! Denn als die beiden Söhne des Zebedäus\personenindex{Zebedäus},
Jakobus\personenindex{Jakobus} und Johannes\personenindex{Johannes}, zu
seiner Rechten und Linken in seinem Reich zu sitzen verlangten, legte er ihnen
die Frage vor:
\label{ref:16_03_kelch_tringen}
\textbf{\textit{"`Könnt ihr den Kelch trinken, den ich trinken werde, und euch
mit der Taufe taufen lassen, womit ich getauft werde?"'}
\index{Taufe}\biblecite{Matthäus 20,22.}{Matthäus 20}
Auf andere Weise kann niemand ein Jünger Jesu oder ein wahrer Christ sein. Wer
also zu Christo kommen und ein wirklicher Christ werden will, der muss
freiwillig
allen Vergnügungen entsagen, die seine Gemütsneigungen gefangen nehmen und dem
Einflusse des göttlichen Geistes entziehen können, ja, er muss jeder geliebten
Eitelkeit, -- und alles Vergängliche unter der Sonne ist eitel, wenn es mit dem
Ewigen und Unvergänglichen verglichen wird, - auf immer den Abschied geben.}

\section{4. Abschnitt} \label{kap16_ab4}

\index{Bibelauslegung!falsche} Es gibt jedoch Menschen, welche für die
Befriedigung ihrer verderbten Neigungen
sogar einen Grund in der Schrift finden wollen, wiewohl sie dieselbe offenbar
unrichtig anwenden, wenn sie den Einwurf machen:
\textit{das Reich Gottes bestehe nicht
in Essen und Trinken oder Kleidern,} etc. Ich antworte ihnen: Ihr habt darin
ganz
recht, und eben deswegen haben wir auch das Überflüssige in diesen Dingen
abgeschafft. Aber gewiss hat niemand weniger ein Recht, uns diese Einwendung zu
machen, als ihr, die ihr ja solche Dinge für so unentbehrlich haltet, dass ihr
uns deswegen nur um so mehr eueren Tadel empfinden lasst, weil wir uns euch
hierin nicht gleichstellen wollen. Ja wie fern ihr nun aber hierin christlich
handelt, oder ob euer Betragen hierin mit der Gerechtigkeit, mit dem Frieden und
mit der Freude im heiligen Geist\index{Geist!Heiliger}, worin das
Reich Gottes\index{Reich!Gottes} besteht,
übereinstimmt, das lasst den gerechten Zeugen der Wahrheit\index{Wahrheit} in
eurem Gewissen\index{Gewissen}
entscheiden. Unser Betragen in diesen Stücken gründet sich auf Mäßigkeit, und
diese besteht mit der Gerechtigkeit, durch welche wir jenes Reich
erlangt\index{Reich!erlangt haben}
haben, wovon eure Unmäßigkeit und eure Ausschweifungen euch ausschließen. Denn,
wenn nur diejenigen wahre Jünger Jesu sein können, welche sein
Kreuz\index{Kreuz!aufnehmen} täglich
aufnehmen, und nur solche es wirklich tragen, die dem Beispiel ihres teueren
Herrn in seiner Taufe\index{Taufe}, in seinen Leiden und in seinen Versuchungen
treulich
folgen, wenn nur diejenigen seine Taufe kennen und erfahren haben, welche ihre
Gemüter von den Eitelkeiten und Torheiten, in welchen der große Haufen der Welt
lebt, zurückgezogen haben, und dem heiligen Licht der göttlichen
Gnade\index{Gnade} gehorsam
geworden sind, dass ihre Seelen erleuchtet und sie täglich in der Übung erhält,
jede ihm widerstrebende Neigung zu kreuzigen\index{Kreuz!kreuzigen} und den
Gegenständen der
Unsterblichkeit\index{Unsterblichkeit} nachzutrachten;\bibmarginpar{Philipper 3,10)
1.~Petrus 4,13)
Titus 2,11)
Johannes 1,9)
Römer 6,6)
Galater 2,20)
Galater 5,24)
Galater 6,4.}
\bibleindex{Philipper 03@Philipper 3}
\bibleindex{Petrus 1 04@1. Petrus 4}
\bibleindex{Titus 02@Titus 2}
\bibleindex{Johannes 01@Johannes 1}
\bibleindex{Römer 06@Römer 6}
\bibleindex{Galater 02@Galater 2}
\bibleindex{Galater 05@Galater 5}
\bibleindex{Galater 06@Galater 6}
-- wenn nur solche wahre Jünger
Jesu sind, wie denn ohne Zweifel keine andere es sein können, so müssen gewiss
die Menschen unserer Zeit, wenn sie nur ein wenig ernstlich nachdenken wollen,
die richtige Schlussfolge ziehen, dass niemand, der sein Vergnügen in den eitlen
Gebräuchen der Welt sucht und einen dem Leben Christi so ganz unähnlichen
Lebenswandel führt, ein wahrer Christ oder Nachfolger des gekreuzigten Jesu sein
kann. Denn worin bestände sonst das Kreuz? Oder was macht das christliche
Leben denn so schwer und zu einem Gegenstand des Spotts und der Verachtung der
Welt? Und stände das Christentum nicht im Widerspruch mit jenen üppigen Moden,
Gebräuchen und eitlen Ergötzungen der Welt, so würde
\textit{"`das Ärgernis des Kreuzes"' bald aufhören;}\biblecite{Galater 5,11.}{Galater 05@Galater 5}
des Kreuzes, das denen, die da glauben und
selig werden,
\textit{"`eine Kraft Gottes ist,"'}\biblecite{1.~Korinther 1,17+18.}{Korinther 1 01@1. Korinther 1}
wodurch jede
verderbte und eitle Neigung überwunden und das Gemüt der Menschen zu einer
heiligen Unterwerfung unter den Willen seines himmlischen Schöpfers gebracht
wird. Darum wurde gesagt, dass Jesus Christus geoffenbart worden ist, und noch
geoffenbart werde, damit er durch sein heiliges Leben, seine Lehre und sein
Beispiel der Selbstüberwindung\editnote{'Selbstverleugnung' ersetzt
durch 'Selbstüberwindung'.}, den Stolz des menschlichen Geistes zu Schanden
mache,\biblecite{1.~Korinther 1,27-29.}{Korinther 1 01@1. Korinther 1}
und durch das ewige \textit{"`Leben und unvergängliche
Wesen,"'} welches er ans Licht\index{Licht} gebracht hat, und noch täglich ans
Licht bringt,
die vergängliche Herrlichkeit der irdischen Ruhe und Freude
verdunkele,\biblecite{1.~Korinther 1,27-29.}{Korinther 1 01@1. Korinther 1}
so dass die Menschen, wenn ihre
Gemüter davon entwöhnt und der Welt gekreuzigt sind, ein anderes Vaterland
suchen und ein ewiges Erbe erlangen mögen.
\textit{"`Denn was sichtbar ist, das ist zeitlich,"'}\biblecite{2.~Korinther
4,18.}{Korinther 2 04@2. Korinther 4}
und davon werden alle wahren Christen erlöst,\index{Erlösung}
so dass sie keine Ruhe darin suchen noch finden,
\textit{"`was aber unsichtbar ist, das
ist ewig;"'} darauf sind die Neigungen ihrer Herzen vornehmlich gerichtet, und
darin finden sie Ruhe.

\section{5. Abschnitt} \label{kap16_ab5}

\label{ref:16_05_weltliches}
\textbf{Ein wahrer Jünger des Herrn Jesu Christi sucht daher sein Gemüt so sehr
an den
Umgang mit himmlischen Gegenständen zu gewöhnen, dass ihm die Dinge dieser Welt
ganz gleichgültig werden, und er gebraucht sie, als wenn er sie nicht
gebrauchen würde.} Er lässt sich mit dem, was er als
Allernötigstes\editnote{'zur Notdurft' ersetzt durch 'als
Allernötigstes'.} bedarf, begnügen, ohne
den Überfluss der Welt zu begehren,\biblecite{1.~Timotheus 6,8.}{Timotheus 1 06@1. Timotheus 6}
und dabei wird ihm das
Vergnügen, welches er zur Zeit der Unwissenheit in der Gleichstellung der Welt
zu finden glaubte, in dem verborgenen himmlischen Leben mit Jesu überflüssig
ersetzt. Wer aber nicht in ihm bleibt, der kann unmöglich all die Früchte
hervorbringen, die er von seinen Nachfolgern erwartet und wodurch sein Vater
verherrlicht wird.\biblecite{Johannes 15,4+7+8.}{Johannes 15}
\label{ref:16_05_weltliches_2}
\textbf{So wie es nun aber klar erhellt,
dass alle, die in den eitlen Gebräuchen, Ergötzungen und Lüsten der Welt leben,
weder in Christo bleiben noch ihn kennen; -- denn wer ihn kennt, verlässt den
Weg der Ungerechtigkeit,~-- so leuchtet es auch deutlich ein, dass ihr Bleiben und
Ergötzen in jenen bezaubernden Torheiten die wahre Ursache ist, warum sie ihn
nicht kennen und nahe fühlen, der doch beständig
\textit{"`an den Türen ihrer Herzen
steht und anklopft,"'}}\biblecite{Offenbarung 3,20.}{Offenbarung 03@Offenbarung 3}
und dem sie den Eingang nicht
verweigern sollten, damit sie seine göttliche Kraft als das heilige Kreuz
erkennen, wodurch jede Lieblingslust und jede verführerische Eitelkeit
gekreuzigt\index{Kreuzigung} und ertötet wird, und so zu der seligen
Erfahrung gelangen, dass sie
das göttliche Leben ihren Herzen emporsteigen fühlten, ein Verlangen nach
himmlischen Dingen erweckt finden, und eine gegründete Hoffnung haben,
\textit{"`dass,
wenn Christus, ihr Leben, sich offenbaren wird, auch sie mit ihm in der
Herrlichkeit offenbar werden sollen, der da ist Gott über alles, gelobt in
Ewigkeit! Amen!"'}\bibmarginpar{Kolosser 3,1-4) Römer 9,5.}
\bibleindex{Kolosser 03@Kolosser 3}
\bibleindex{Römer 09@Römer 9}


