\section*{P}

\articlesize

\begin{description}

 \item[Paquet, Alfons]

 \item[Pastoral Friends] Werden Versammlungen genannt die Pastoren haben, ihre
 Versammlungshäusert "`Kirchen"' ("`churches"') oder "`meeting houses"' nennen,
 $\to$\textit{Christ-centered} aber nicht unbedingt Missionarisch/Evangelikal
 sind. \endnote{\texttt{http://www.quakerinfo.org/quakerism/fandplinks.html}}

 \item[Pastoral Friends] Ähnlich wie die $\to$\textit{Evangelical Friends}. Sie
 sind dazu übergegangen wieder Pastoren zu bestellen und haben
 $\to$\textit{programmierte Andachten} eingeführt. Aber \textit{Pastoral
 Friends} sind nicht so missionarisch wie \textit{Evangelical Friends}.

 \item[Pastorius, Franz Daniel]

 \item[Penn, Willam] $\ast$14.10.1644 in London, \dag30.7.1718 in Ruscombe,
 Berkshire. Quaker. Gründete die Kolonie Pennsylvania im Gebiet der heutigen
 USA, verfasste 1693 das Essay \textit{"`towards the Present and Future Peace
 of Europe"'}.

 \item[Perrot, John] $\ast$?, \dag1665 in Jamaika. Quaker. Wollte 1658 mit
 $\to$John 'Love' Luffe zusammen Papst Alexander VII bekehren.

  \item[Philadelphia Yearly Meeting] Verbot von Sklavenhandel/Besitz, Erste
  große Spaltung ($\to$\textit{great separation})

  \item[Proceed as Way Opens]
    to undertake a service or course of action without prior clarity about all the details but with confidence that divine guidance will make these apparent and assure an appropriate outcome.

 \item[Programmierte] meint $\to$\textit{Programmierte Andacht}.

 \item[Programmierte Andacht]

 \item[Projekt Alternativen zur Gewalt (PAG)]

 \end{description}
\normalsize


