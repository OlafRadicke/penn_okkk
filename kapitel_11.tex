
\chapter{Kapitel 11} \label{kap11}
\section{Zusammenfassung des 11. Kapitel}

\begin{description}
\item[1. Abschnitt] Der Stolz verleitet die Menschen zur Überschätzung ihrer
Personen.
\dotfill \textit{Seite~\pageref{kap11_ab1}}\\
\item[2. Abschnitt] Dieses beweist der Lärm, den man über Familienherkunft,
Geburt, Gestalt und Schönheit erhebt.
\dotfill \textit{Seite~\pageref{kap11_ab2}}\\
\item[3. Abschnitt] Die Tugend, nicht die Geburt, kann wahren Adel verleihen.\\
.\dotfill \textit{Seite~\pageref{kap11_ab3}}\\
\item[4. Abschnitt] Die Tugend ist kein Emporkömmling, und ohne sie kann alte
Herkunft keinen Adel gewähren, sonst würden in unseren Tagen Alter und Geburt
die
Tugend davon ausschließen.
\dotfill \textit{Seite~\pageref{kap11_ab4}}\\
\item[5. Abschnitt] Gott, der alle Geschlechter von einem Blut abstammen ließ,
lehrt uns, was wahrer Adel ist.
\dotfill \textit{Seite~\pageref{kap11_ab5}}\\
\item[6. Abschnitt] Die Menschen von adeliger Herkunft sind, ihrer äußeren
Zierden
beraubt, anderen Menschen ganz gleich
\dotfill \textit{Seite~\pageref{kap11_ab6}}\\
\item[7. Abschnitt] Dieses wird nicht gesagt, um den Edelmann zu verwerfen,
sondern um ihn zur Demut anzuleiten. -- Von dem Vorteil, den dieser Stand vor
anderen voraus hat. Eine Ermahnung an den Adel, den gesunkenen Zustand ihrer
Familien um ihres eigenen Vorteiles willen wieder herzustellen.
\dotfill \textit{Seite~\pageref{kap11_ab7}}\\
\item[8. Abschnitt] Der Verfasser hat aber noch einen höheren Beweggrund dazu:
das Evangelium, zu dem die Adeligen sich bekennen.
\dotfill \textit{Seite~\pageref{kap11_ab8}}\\
\item[9. Abschnitt] Vom persönlichen Stolz in Hinsicht auf Gestalt und
Schönheit. Die Ausgaben für Wohlgerüche, Schönheitswasser, Schminke, Putz, etc.
würden viele arme Familien ernähren können. -- Von den Nachteilen, die damit
verknüpft sind.
\dotfill \textit{Seite~\pageref{kap11_ab9}}\\
\item[10. Abschnitt] Stolz und Eitelkeit ist bei alten und hässlichen Personen
noch verwerflicher als bei anderen, wird aber doch häufig angetroffen. Die große
Torheit solcher Personen. Rat an die Schönen, ihre Seelen ihren Körpern
ähnlich zu machen, und an die Hässlichen, ihren Mangel an körperlicher Schönheit
dadurch zu ersetzen, dass sie den unvergänglichen Teil ihres Wesens, ihre
Seelen, mit Heiligkeit schmücken. -- Vor Gott ist nichts hässlich als die Sünde.
-- Von der Seligkeit derer, die das Joch oder Kreuz Christi tragen, und der Welt
gekreuzigt sind.
\dotfill \textit{Seite~\pageref{kap11_ab10}}\\

\end{description}

\newpage

\section{1. Abschnitt} \label{kap11_ab1}

Der Stolz bleibt aber dabei noch nicht stehen, er verleitet auch die Menschen
zur Überschätzung ihrer selbst und zu einer übertriebenen Sorgfalt für ihre
Personen. Daher müssen sie so viele und pünktliche Aufwartung, kostbare
Möbel\index{Möbel!kostbare},
reiche und schöne Kleider\index{Kleidung} und dergleichen haben, worin ein
großer Teil
der "`Hoffart
des Lebens"' besteht, wovon Johannes uns sagt,
"`dass sie nicht vom Vater,
sondern von der Welt ist."'\biblecite{1.~Johannes 2,16.}{Johannesbrief 1 02@1. Johannes 2}
Dieses war die Sünde, die Gott
den übermütigen Töchtern Zions\personenindex{Zions Töchter} und dem stolzen
Fürsten und Volk von Tyrus\index{Orte:!Tyrus} zur
Last legte. Man lese das 3. Kapitel bei Jesaias\bibleindex{Jesaja 03@Jesaja 3},
und das 27. und 28. bei
Ezechiel\bibleindex{Ezechiel 27}\bibleindex{Ezechiel 28},
beurteile dann das gegenwärtige Zeitalter nach den Sünden, die nun im
Schwange gehen, und schließe, was für Gerichte die jetzt lebenden Völker zu
erwarten haben. Ich will hier nur den ersten Gegenstand dieses Kapitels, nämlich
den übertriebenen Wert, den die Menschen auf ihre Personen legen, in
Betrachtung ziehen, und das Übrige im letzten Abschnitt dieser Abhandlung, der
von der Üppigkeit handelt, und wo es eine passende Stelle finden wird,
weiter auseinandersetzen.

\section{2. Abschnitt} \label{kap11_ab2}

Dass die Menschen im Allgemeinen stolz auf ihre Person sind, besonders, wenn sie
einigen Anspruch auf vornehme Herkunft\index{Stammbaum!vornehmer} oder auf
Schönheit\index{Schönheit} machen können, ist eben
so bekannt, als unangenehm zu betrachten. Der erstere Fall hat oft schon unter
dem männlichen, und der letztere unter dem weiblichen Geschlecht sehr heftige
Streitigkeiten veranlasst, und es sind auch nicht selten schon Männer, um der
Frauen willen, oder durch deren Anreizung, in solche Zänkereien verwickelt
worden. Was für Lärm und Gezänk ist nicht schon in der Welt vorgefallen, wenn
es die Bestimmung des Alters eines Stamms oder einer Familie galt? Oder wenn
entschieden werden sollte, wessen Vater oder Mutter, oder Urgroßvater und
Urgroßmutter die meisten Ahnen zählte? Welchem Stamme oder Zweige dieser oder
jener angehöre? Was für Wappen\index{Wappen} der eine oder der andere geführt
habe, und wer
nun das Recht zum Vorrang besitze? Wiewohl, meines Erachtens, unter allen
Torheiten des Menschen keine weniger vernünftig scheinende Gründe zu ihrer
Entschuldigung aufzuweisen hat, als diese.

\section{3. Abschnitt} \label{kap11_ab3}

\index{Herabsetzen!jemanden}Denn erstlich: Was liegt daran, von wem jemand
abstammt, insofern er selbst
keinen bösen Ruf hat, da nur seine eigenen Tugenden ihn erheben, und keine
anderen als seine eigenen Laster\index{Laster} ihn
erniedrigen\index{erniedrigen!sich selbst} können? Die Verdienste eines
Vorfahren können die schlechten Handlungen eines Menschen nicht
entschuldigen\index{entschuldigen};
sie beweisen vielmehr seine Entartung. Und da Tugend nicht durch Geburt erlangt
werden kann, so werde ich auch in der Tat durch meine Herkunft weder besser noch
schlechter. Auch gilt diese eben so wenig in Gottes Auge, als sie in den Augen
der Menschen etwas gelten sollte. \label{ref:11_03_beleidigung}
\textbf{Kein Verständiger wird Beleidigungen bloß
deswegen leichter ertragen, weil sie ihm von einem Manne von hoher Geburt
zugefügt werden, oder Gunstbezeigungen darum desto eher ausschlagen, weil
derjenige, der sie ihm erweist, von geringer Herkunft ist.} Ich gestehe, dass es
eine große Ehre sein würde, gar keine Familienflecken\index{Schwarzes
Schaf!in
Familie} zu haben, und sein
Erbteil von einem Stamm herleiten zu können, dem man nichts zum Vorwurf
machen konnte. Allein, das hat man noch nie angetroffen, selbst nicht in der
gesegnetsten Familie, die jemals auf Erden lebte, ich meine, auch in
Abrahams\personenindex{Abrahams!Familie}
Familie nicht. Auch kann der Umstand, dass jemand von reichen und hoch
betitelten
Vorfahren abstammt, weder seinen Kopf mit Verstand, noch sein Herz mit Wahrheit
erfüllen, diese Eigenschaften sind eines höheren Ursprungs. Demnach ist es bloß
Eitelkeit und höchst verwerflicher Stolz\index{Stolz}, wenn jemand von großem
Gewicht und
Ansehen in der Welt einen anderen deswegen gering achtet, weil dieser ihm an
Macht und Rang nicht gleich ist. Es kann ja leicht sein, dass der Letztere die
Verdienste seiner Vorfahren besitzt, während der Erstere nur die Früchte
genießt, welche die Verdienste seiner Vorfahren hervorbrachten. So kann also der
eine durch seine Vorfahren, der andere aber durch sich selbst groß sein, und
welcher von beiden verdient nun den Vorzug?

\section{4. Abschnitt} \label{kap11_ab4}

Ach! Sagt der auf seine Herkunft stolze Mensch, alles geht schlecht in der Welt,
seitdem wir so viele Emporkömmlinge\index{Emporkömmling} unter dem
Adel zählen! Was mögen aber wohl
andere von seinen Vorfahren gesagt haben, als diese zuerst anfingen, sich in der
Welt auszuzeichnen? Denn sie und alle Menschen und Familien, ja, alle Staaten
und Reiche der Welt mussten doch auch ihr Emporkommen haben, oder ihren Anfang
nehmen. Wenn daher Familien ihren Adel auf ihre alte Herkunft, und nicht auf ihre
Tugenden gründen, \label{ref:11_04_abstammung} \textbf{so machen sie es ebenso,
wie\index{Kirche!beste}\index{Kirche!wahre} die Kirche, welche behauptet,
die Wahre zu sein, weil sie die älteste, nicht weil sie die beste ist.} So kann
es aber nicht gehen. Wenn das Alter etwas gelten soll, so muss es ein Alter in
der Tugend sein, und dieses muss vor jeder anderen alten Abstammung den Vorzug
haben. \textbf{Sonst könnte jemand von höherem Adel als sein Vorfahr, von dem er
seinen Adel ableitet, und der Vorfahr, der ihm seinen Adel erwarb, von geringerem Adel
als er sein, eine Seltsamkeit, welche die Geschicktesten in der
Wappenkunde\index{Wappen!Wappenkunde} wohl
schwerlich werden erklären können.} Es ist allerdings sonderbar, dass jemand
einen höheren Adel besitzen sollte als seine Vorfahren, denen er den seinigen
verdankt. Wenn dieses aber ungereimt ist, wie es denn wirklich ist, so ist der
Neugeadelte, der sich seinen Adel durch seine Tugenden erworben hat, der wahre
Edelmann, und es können daher auch nur diejenigen, die seine Tugenden
nachahmen, auf seine Ehre Anspruch machen, obgleich seine übrigen Abkömmlinge
sich auf ihre Herkunft von ihm berufen und auch seinen Namen führen mögen. Wenn
also nur die Tugend wahren Adel verleihen kann, wie selbst die
Heiden\index{Heide} behaupten,
so kann folglich eine Familie auch nur so lange wahrhaft adelig sein, als sie
tugendhaft ist. Und da ferner die Tugend nicht der Geburt oder Herkunft anhängt,
sondern in den guten Eigenschaften der Nachkommen besteht, so folgt klar, dass
die Herkunft kein Vorrecht gewährt, sonst würde diese die Tugend als unnötig
ausschließen, und derjenige, dem es an Herkunft mangelte, auch auf die Wohltaten
der Tugend Verzicht leisten müsse, und dadurch würde dann der Adel, sobald es
ihm an Alter fehlte, herabgesetzt, und die Tugend bei den Adeligen entbehrlich
gemacht werden.

\medskip

Nein, man lasse immer die Namen der Herkunft folgen, aber Adel und Tugend müssen
Hand in Hand gehen, da sie so nahe Verwandte sind. So hat Gott selbst es
verordnet, der am besten alle Dinge recht und billig zu verteilen weiß. Er
richtet sich mit seinem Wohlgefallen oder Missfallen nicht nach der Herkunft der
Menschen, und sieht auch nicht auf das, was sie waren, sondern auf das, was sie
sind. Denn
\textit{"`er gedenkt nicht der Gerechtigkeit des Menschen, der seine
Gerechtigkeit verlässt;"'}\biblecite{Ezechiel 18.}{Ezechiel 18}
und wird noch viel weniger den
Ungerechten wegen der Tugenden seiner Vorfahren als einen Gerechten betrachten.

\section{5. Abschnitt} \label{kap11_ab5}

Wenn indessen diejenigen, die ihre Herkunft so hoch anrechnen, sich verpflichtet
halten wollen, Gott dadurch zu ehren, dass sie dem, was er in der heiligen
Schrift uns bekannt machen lässt, Glauben beimessen, so werden sie finden,
\textit{"`dass
er am Anfang alle Menschengeschlechter, die auf dem ganzen Erdboden wohnen,
aus einem Blut gemacht hat,"'}\biblecite{Apostelgeschichte 17,26.}{Apostelgeschichte 17}
und dass wir alle
von einem Elternpaar abstammen. Eine untrüglichere Geschlechtskunde werden auch
die Besten unter uns nicht aufweisen können. Weiter herab kommen wir auf
Noah\personenindex{Noah}, den zweiten Stammvater des menschlichen Geschlechts,
und bis auf ihn
wissen wir also etwas Zuverlässiges von unsern Vorfahren. Was für einen Anteil
uns nun aber in unseren Tagen an den Titeln und Auszeichnungen gebührt, welche
andere seit jener Zeit entweder durch Gewalt\index{Gewalt} sich zugeeignet oder
durch Tugend
erworben haben, dürfte wohl -- wenn wir auch nur einige Jahrhunderte
zurückgehen,~-- schwer zu bestimmen sein.

\section{6. Abschnitt} \label{kap11_ab6}

Ich denke\editnote{'Mich dünkt' ersetzt durch 'Ich denke'.} auch, es
müsste uns klar einleuchten, dass die Geburt niemand zu einem
vorzüglichen Menschen machen könne, wenn wir bemerken, dass diese Leute von
vornehmer Herkunft, wenn sie ihren ganzen Staat und Putz abgelegt haben, von
Natur kein besonderes Merkmal an sich tragen, das sie von ihren geringeren
Mitmenschen unterschiede. Und lassen wir sie selbst urteilen, so werden sie
uns gestehen müssen, dass sie bei allen Vorzügen ihrer Geburt dennoch jenen
Leidenschaften\index{Leidenschaft} unterworfen sind, welche die Menschen mit
einander gemein haben,
und folglich die Vornehmeren den Geringeren gleich machen, wenn nicht noch
weiter
als andere von der Tugend entfernen, die allein wahren Adel verleihen kann.
Hiervon geben uns, leider, die beklagenswerte Unwissenheit und Zügellosigkeit,
die wir unter nur zu vielen unserer vornehmen Leute antreffen, die
überzeugendsten Beweise. Und was für einer Geburt wollen sie dieses zuschreiben?

\section{7. Abschnitt} \label{kap11_ab7}

Wie dem nun auch sein mag, so ist es, nachdem ich dieses alles gesagt habe, um
eine falsche Eigenschaft herabzuwürdigen, keineswegs meine Absicht, einer
anderen, die nicht weniger tadelswert ist, das Wort zu reden. \textbf{Man wolle
mich
nicht so verstehen, als suchte ich die eingebildete
Klasse\index{Gesellschaft!Klassen abschaffen} über den vornehmen
Stand zu erheben. Davon bin ich weit entfernt, denn das grobe
Benehmen\index{Benehmen!grobes} roher
Menschen würde die Sache nicht verbessern. Mein Zweck ist nur, allen zu zeigen,
woraus der wahre Adel besteht,} damit jeder auf dem Weg der Tugend und des
Edelmuts danach streben möge. Auch muss ich, nach allem, dem Edelmann große
Vorzüge einräumen, die seinen Stand wirklich erheben, so wie der Apostel
Paulus\personenindex{Paulus}
den Juden\index{Juden}, welche, stolz auf ihre Gesetze und äußeren
Beobachtungen, die Christen
beleidigten, nachdem er sie gedemütigt hatte, in Ansehung ihrer Verfassung und
Einrichtung vor allen anderen Völkern den Vorzug gab.
\label{ref:11_07_standesvorteil} \textbf{Ich muss gestehen, die Lage
unserer\index{Gehobener Stand!Vorteile} Großen ist der des
niedrigeren
Standes weit vorzuziehen. Denn erstlich
haben sie größere Macht, Gutes zu tun, und wenn hierin ihre Herzen mit ihrer
Kraft und Fähigkeit übereinstimmen, so dienen sie in jedem Land dem Volke zum
Segen. Zweitens, da die Augen der Menge auf sie gerichtet sind, so ist es ihnen
leicht, durch Güte, Gerechtigkeit und Wohltätigkeit\index{Wohltätigkeit} sich
allgemeine Zuneigung
und Achtung zu erwerben. Drittens sind ihre Umstände nicht so beschränkt als die
der niedrigeren Klasse, und sie haben folglich mehr Hilfsmittel, mehr
Muße\index{Müßiggang} und
Gelegenheit, durch Lesen\index{Erkenntnis!Lesen} und Umgang ihren Verstand
auszubilden\index{Erkenntnis!Verstand auszubilden} und ihre
Leidenschaften zu zügeln. Viertens haben sie auch mehr Zeit zu
Reisen\index{Erkenntnis!Reisen}, um das
Betragen anderer Nationen zu beobachten, ihre Gesetze, Sitten, Gebräuche und
Vorteile kennen zu lernen, und das Nachahmungswerte derselben in ihr Vaterland
zu verpflanzen.} Alles dieses macht es den Großen leichter als anderen, sich
Achtung und Ehre zu erwerben, und wem es unter ihnen um wahrhaft guten Ruf zu
tun ist, dem stehen Mittel und Wege genug zu Gebote, auf die beste Weise dazu
zu gelangen. Da es aber, leider, oft der Fall ist, dass die Großen wenig daran
denken, für ihren Wohlstand Gott die Ehre zu geben, und für die Wohltaten, die
sie von ihm genießen, durch ein ihm wohlgefälliges Leben sich dankbar zu
erweisen, sondern da sie, im Gegenteil, nicht selten wie
\textit{"`ohne Gott in der
Welt leben,"'} und bloß der Befriedigung ihrer sinnlichen Neigungen nachgehen,
so zeigt auch der Allmächtige seine Hand\index{Gott!Strafe} oft dadurch, dass
er sie arm macht\index{Verarmung}, oder
sie ausrottet\index{Ausrottung}, und andere von mehr Tugend und Demut zum
Besitz ihrer Stellen
und Würden erhebt. Doch muss ich auch bemerken, dass es unter den Leuten von
hohem
Rang auch schon einige gegeben hat, die sich durch ungewöhnliche Tugenden
auszeichneten, und deren Beispiel wie glänzende Lichter in ihren Familien
schienen, so dass ihren Nachkommen ein beständiges Streben, den guten Namen des
Hauses den Verdiensten seines Stifters gemäß aufrecht zu erhalten, gewissermaßen
natürlich geworden war. \textit{Wenn aber, in Wahrheit, irgend ein Vorzug mit
einer
vornehmen Herkunft\editnote{'Abkunft' ersetzt durch'Herkunft'.}
verbunden ist, so liegt er in der Erziehung\index{Erziehung},
die der Stand
gewährt, und nicht in der Herkunft selbst,} deren Echtheit übrigens auch oft
zweifelhaft oder ungewiss ist, wohingegen die Erziehung immer einen mächtigen
Einfluss auf die Neigungen und Handlungen des Menschen hat. Hierin zeichneten
sich einst die Adeligen und Vornehmen in unserem Königreich besonders aus,
und es wäre sehr zu wünschen, dass unsere jetzigen Großen es sich möchten
angelegen sein lassen, die Wirtschaftlichkeit, gute Ordnung und tugendhafte
Erziehung in ihren Familien wieder herzustellen, die unter ihren würdigen
Vorfahren üblich war, als die Adeligen ihrer großen und edlen Taten wegen
geehrt wurden, und nichts jemand mehr der Schande und Verachtung aussetzte, als
wenn er von adeliger Herkunft war, und keine Tugend besaß, um seinen Adel
aufrecht zu halten.

\section{8. Abschnitt} \label{kap11_ab8}

Doch ich habe noch einen höheren Beweggrund hinzuzufügen. Das herrliche
Evangelium Jesu Christi ist ja auch dieser nördlichen Insel verkündigt
worden, und da ihre Bewohner aus allen Ständen an dasselbe zu glauben bekennen,
so lasst mich euch bitten und bewegen, die Ehre zu suchen und nach der
Auszeichnung zu trachten, die durch diese himmlische
\index{Gnade}Gnadenausteilung Gottes
allen wahren Gläubigen widerfährt, die dem \index{Lamm}Lamm folgen,
\textit{"`welches die Sünden
der Welt hinwegnimmt."'}\index{Sünde!Tilgung der}\biblecite{Johannes 1,29.}{Johannes 01@Johannes 1}
Nehmt mit Sanftmut sein gnädiges
Wort in eure Herzen auf, welches die sinnlichen Lüste\index{Lust} der Welt
überwindet und die Seele auf den heiligen Pfad leitet, der zur wahren Glückseligkeit führt.
Hier gibt es Freuden zu genießen, die noch kein fleischliches Auge gesehen,
kein sinnliches Ohr gehört, kein weltlich gesinntes Herz\index{Herz!weltlich
gesinntes} vernommen hat, aber den
Demütigen, die sich in Wahrheit zu Gott bekehren, durch seinen Geist
geoffenbart werden. Bedenkt, dass ihr nur Geschöpfe seid, dass ihr sterben
müsst,
und nach dem Tod vor's Gericht kommt. \index{Gericht!Jüngstes}

\section{9. Abschnitt} \label{kap11_ab9}

Persönlicher Stolz beschränkt sich aber nicht allein auf den hohen Wert, den
die Menschen auf ihre Herkunft legen, er verleitet sowohl die Unadeligen als
auch
die Adeligen zu einer übertriebenen Schätzung ihrer Personen, besonders wenn sie
auf körperliche Schönheit\index{Schönheit} Anspruch machen können.
\label{ref:11_09_putzsucht}
\textbf{Es ist zum Verwundern, wenn man
sieht, wie sehr einige von ihrer Person eingenommen sind, als wenn sonst nichts
in der Welt ihrer Aufmerksamkeit wert sei, oder die Achtung anderer verdiene!
Doch würde es noch ihre Torheit vermindern, wenn ihr Herz sich entschließen
könnte, nur die Hälfte der Zeit, die sie mit Waschen,
Schminken\index{Schminken}, Parfümieren und
Putzanlegen verschwenden, darauf anzuwenden, dass sie an Gott und an ihr Ende
dachten.} Alle jene Dinge müssen aufs Pünklichste besorgt und aufs Künstlichste
gemacht werden, und an Ersparung der Kosten ist dabei nicht zu denken. Was daher
das Übel noch vergrößert, ist, dass mit dem, was der Stolz eines einzelnen
verlangt, die Bedürfnisse von zehn anderen Menschen befriedigt werden könnten.
\textbf{Ja, ist es nicht grobe und entsetzliche Sünde\index{Sünde}, dass der
Stolz und die Eitelkeit
einer Nation mehr verschlingen, als die Unterhaltung aller ihrer Armen kosten
würde?} Und was haben die Menschen bei allen diesen Torheiten für einen Zweck?
-- Um sich bewundern und verehren zu lassen, um Liebe einzuflößen, die Augen der
Zuschauer auf sich zu ziehen und ihre Neigungen zu gewinnen. Und bei dem allen
sind sie noch dazu so eigen, dass es schwer ist, ihnen zu gefallen. Nichts ist
ihnen gut, oder fein oder modig genug. \index{Sonnenbräune}Ach, die Sonne, diese
die Erde erquickende Wohltat des Himmels, darf sie nicht bescheinen, sie möchte ihre
Haut verderben! Der Wind muss nicht wehen, er könnte ihren Putz in Unordnung
bringen! O! Der schändlichen Verzärtelung! -- Während sie aber so über alles in
der Welt sich selbst schätzen, \textbf{sind sie doch
Sklaven\index{Sklave}, gefesselt vom Geist des
Stolzes und der Eitelkeit,} der sie beherrscht und den sie durch Bewunderung
ihres Wuchses, ihrer Gesichtsbildung und ihrer Haut, verehren und anbeten.

Der Zweck aller solcher künstlichen und kostbaren Bemühungen und Anstrengungen
ist nur zu oft kein anderer, als zu gefallen und unerlaubte Liebe zu erwecken,
wodurch beide Geschlechter oft in eine ebenso traurige als strafbare Lage
versetzt werden. \index{Ehe}\index{Fremdgehen}\index{Eifersucht}
\label{ref:11_09_kaputte_ehen} \textbf{Bei
unverheirateten Personen sind immer die Folgen einer
solchen Liebe verderblich, denn, wenn sie auch nicht zu unkeuschen Begierden
Veranlassung gibt, so legt sie doch nie den Grund zu einer festen und
dauerhaften Verbindung, die nur auf gegenseitiger innerer Achtung und
Wertschätzung sicher ruhen kann, so wie der Mangel an einer solchen Grundlage
die Hauptursache ist, dass es so viele unglückliche Ehen in der Welt gibt. Bei
Verehelichten ist ein solches Benehmen noch sündlicher, dass beide Teile, dem
heiligen Gesetze der Ehe gemäß, nur einander zu gefallen suchen sollten. Und
wenn sie darin der üppigen und eitlen Jugend nachahmen, so ist dieses immer ein
schlechter Beweis ihrer gegenseitigen Liebe und häuslichen Glückseligkeit. Das
eitle Schminken und Herausputzen ihrer Personen gibt ihnen das Aussehen, als
wollten sie auf Eroberungen ausgehen, und wo dieses wirklich der Fall ist, da
sind die Folgen davon oft schrecklich, sie brechen in Missvergnügen und
Eifersucht, endlich in Hass aus, und enden gewöhnlich mit Trennungen und
Ehescheidungen, ja, nicht selten sogar mit Vergiftungen und anderen schändlichen
Ermordungen.} Kein Zeitalter kann uns von dergleichen traurigen und entsetzlichen
Wirkungen des Stolzes und der Eitelkeit deutlichere Begriffe geben, als das
gegenwärtige, welches, vornehmlich in unserm Königreich, den nachteiligen
Einfluss eines üppigen und ausschweifenden Lebens auf die Tugend, Ruhe,
Mäßigkeit
und Gesundheit der Familien aus unzähligen Beispielen erklärt.

\section{10. Abschnitt} \label{kap11_ab10}

Noch muss ich notwendig bemerken, dass unter allen menschlichen Geschöpfen
solche
Beweise des Stolzes und der Eitelkeit am wenigsten\index{Menschen!alte}\index{Menschen!hässliche} alten und hässlichen Personen
anstehen,~-- wenn ich die Verunstalteten und von der Natur schlecht
Ausgestatteten hässlich nennen darf. \textbf{Denn alte Personen können nur auf
das, was
sie einst waren, stolz sein, welches zu ihrer Schande beweist, dass ihr Stolz
ihre Schönheit überlebt hat, und sie sich, statt ihre Torheit zu bereuen, nur
neuen Stoff zur Reue bereiten.} Die Hässlichen machen es aber noch schlimmer,
sie
sind stolz auf etwas, das sie nie besaßen, und auch niemals erlangen können. Ja,
es scheint, als wenn ihre Gestalt ihnen zu einer beständigen Demütigung ihres
eitlen Geistes dienen solle, und hierauf stolz sein, heißt in der Tat, den
Stolz des Stolzes wegen lieben, und sich seiner Herrschaft ergeben, ohne eine
Versuchung dazu zu haben. Und dennoch habe ich in meinem ganzen Leben keine
Menschen gesehen, die mehr von sich eingenommen gewesen waren, als gerade diese.
Wie ist es doch möglich, dass der Stolz die Menschen so sehr betören und
verblenden kann? Ist denn die Parteilichkeit ihrer Herzen so groß, dass sie mit
ihren Augen nicht mehr recht sehen können? -- Dieses beweist in der Tat, dass
die Eigenliebe die Menschen blind macht. Dass sie aber mit ihrer Eitelkeit noch
Verschwendung verbinden, und so viele Kosten auf etwas verwenden, das gar nicht
zu ändern ist, zeigt wirklich die höchste Torheit, besonders wenn man bedenkt,
dass ihre Anwendung der Dinge, die für schön gehalten werden, ihre Hässlichkeit
nur noch mehr heraushebt und erst recht bemerkbar macht, weil diese ihnen so
übel anstehen.

\medskip

Aber aus der Torheit solcher Personen können wir deutlich lernen, was für ein
Wesen der Mensch geworden ist, nachdem er seinen ursprünglichen herrlichen
Zustand verloren hat. Und all diese Übel kommen, wie Jesus einst von der
Sünde\index{Sünde} sagte,
\textit{"`aus seinem Innern"'}\biblecite{Matthäus 15,11+18-20.}{Matthäus 15}.
Sie sind nämlich die Folgen seines Mangels an Aufmerksamkeit auf
\textit{"`dass Wort seines Schöpfers in
seinem Herzen"'}\bibmarginpar{1.~Mose 30,14.) Römer 10,8.}
\bibleindex{Mose 1 30@1. Mose 30} \bibleindex{Römer 10},
welches ihm seinen Stolz entdeckt, ihn Demut\index{Demut} und
Selbstüberwindung\editnote{'Selbstverleugnung' ersetzt durch
'Selbstüberwindung'.} lehrt, und sein Gemüt auf den
wahren Gegenstand seiner Verehrung und Anbetung richtet, indem es dasselbe mit
einer feierlichen Achtung und Ehrfurcht erfüllt, die der höchsten Herrschaft und
Majestät des Allmächtigen gebührt. Oh,\index{Tod}\index{Gericht!Jüngstes}
\index{Vergänglichkeit} \label{ref:11_10_juengstes_gericht}
\textbf{was ist der arme sterbliche Mensch? --
Ein lebender Staub, aus dem gleichen Staub der Erde gebildet, die er mit seinen
Füßen tritt! Kann er doch mit all' seinem Stolz sich nicht vor der Zerstörung
einer Krankheit sichern, und noch vielweniger den Streich des Todes abwenden!
Möchten daher doch die Menschen die Unbeständigkeit alles Sichtbaren, die
Trübsale und Widerwärtigkeiten ihres Lebens und die Gewissheit sowohl ihres
Hinscheidens, als auch der Erwartung des ewigen Gerichts in reifliche
Betrachtung ziehen, so ließe sich hoffen, dass sie
\textit{"`ihre Werke an das Licht
Christi in ihren Herzen bringen würden, damit sie erkennen würden, ob sie in
Gott
getan wären oder nicht,"'} }\biblecite{Johannes 3,20+21.}{Johannes 03@Johannes 3}
wie der Lieblingsjünger, als
Worte aus dem Munde seines teueren Meisters uns sagt. -- Bist du nun
wohlgestaltet, von angenehmer Gesichtsbildung und schön, ein musterhafter
Abdruck der Würde des menschlichen Wesens? O! So bewundere die Macht, die dich
so schuf. Führe ein Leben, das der edlen Gestalt und wundervollen Einrichtung
deines Körpers entspricht, und lass die Schönheit desselben dich lehren, deine
Seele mit Heiligkeit\index{Heiligkeit}, dem Schmuck\index{Schmuck} der
Lieblinge Gottes, zu zieren. Und bist du
ungestaltet und hässlich, so preise nichtsdestoweniger die ewige Güte, dass sie
dich nicht zu einem unvernünftigen Wesen erschuf, und lass es dein Bestreben
sein, durch Hilfe der Gnade, die dir verliehen ist, --\index{Gnade} denn die
göttliche Gnade
ist allen erschienen,~-- deine Seele mit unvergänglicher Schönheit zu schmücken.
Bedenke, \textit{"`dass die Tochter des Königs der Himmel,"'} die Kirche Christi,
\index{Kirche!Kirche Christi} deren
\index{Kirche!Mitgliedschaft} Mitglieder alle wahre
Christen\index{Christ!wahrer}
sind, \textit{"`inwendig ganz herrlich geschmückt ist,"'}
und wenn deine Seele sich darin hervortut, so wird das Mangelhafte deines
Körpers ihren Glanz nur umso mehr erheben. Denn in Gottes Augen ist nichts
hässlich, als die Sünde\index{Sünde}, und alle, welches Geschlechts und Stands
sie auch
sind und wie sie auch gestaltet sein mögen,
\textit{"`die mit ihrem Herzen reden und
nicht sündigen;"'}\biblecite{Psalm 4,5.}{Psalm 004@Psalm 4},
die in dem heiligen Licht Jesu über die
Regungen und Neigungen ihrer Herzen wachen und jedes Böse in der Geburt
ersticken, die das Joch oder Kreuz Christi\index{Kreuz} lieben und dadurch
täglich der Welt
gekreuzigt\index{Kreuzigung} werden, alle diese führen ein inneres Leben
mit
Gott, welches die
vergänglichen Lebensgenüsse der Welt an Schönheit und Dauer unendlich
übertrifft.



