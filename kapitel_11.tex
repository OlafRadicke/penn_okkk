
\chapter{11. Kapitel} \label{kap11}
\section{Zusammenfassung des 11. Kapitel}
\footnotesize
\begin{description}
\item[1. Abschnitt] Der Stolz verleitet die Menschen zur Überschätzung ihrer
Personen.
\\(\textit{Seite \pageref{kap11_ab1}})
\item[2. Abschnitt] Dieses beweiset der Lärm, den man über Familienabkunft,
Geblüt, Gestalt und Schönheit erhebt.
\\(\textit{Seite \pageref{kap11_ab2}})
\item[3. Abschnitt] Die Tugend, nicht das Geblüt, kann wahren Adel verleihen.
\\(\textit{Seite \pageref{kap11_ab3}})
\item[4. Abschnitt] Die Tugend ist kein Emporkömmling, und ohne sie kann alte
Abkunft keinen Adel gewähren; sonst würden in unsern Tagen Alter und Geburt die
Tugend davon ausschließen.
\\(\textit{Seite \pageref{kap11_ab4}})
\item[5. Abschnitt] Gott, der alle Geschlechter von einem Blute abstammen ließ,
lehret uns, was wahrer Adel ist.
\\(\textit{Seite \pageref{kap11_ab5}})
\item[6. Abschnitt] Die Menschen von adeliger Abkunft sind, ihrer äußern Zierden
beraubt, andren Menschen ganz gleich
\\(\textit{Seite \pageref{kap11_ab6}})
\item[7. Abschnitt] Dieses wild nicht gesagt um den Edelmann zu verwerfen,
sondern um ihn zur Demuth anzuleiten. -- von den Vortheil, die dieser Stand vor
andern voraus hat. Eine Ermahnung an den Adel, den gesunkenen Zustand ihrer
Familien um ihres eigenen Vortheiles willen wieder herzustellen.
\\(\textit{Seite \pageref{kap11_ab7}})
\item[8. Abschnitt] Der Verfasser hat aber noch einen höheren Beweggrund dazu:
das Evangelium, zu dem die Adelinen sich bekennen.
\\(\textit{Seite \pageref{kap11_ab8}})
\item[9. Abschnitt] Vom persönlichen Stolze in Hinsicht auf Gestalt und
Schönheit. Die Ausgaben für Wohlgerüche, Schönheitswasser, Schminke, Putz, etc.
würden viele arme Familien ernähren können. -- Von den Nachtheilen, die damit
verknüpft sind.
\\(\textit{Seite \pageref{kap11_ab9}})
\item[10. Abschnitt] Stolz und Eitelkeit ist bei alten und häßlichen Personen
noch verwerflicher als bei Andern; wird aber doch häufig angetroffen. Die große
Thorheit solcher Personen. Rath an die Schönen, ihre Seelen ihren Körpern
ähnlich zu machen; und an die Häßlichen, ihren Mangel an körperlicher Schönheit
dadurch zu ersetzen, daß sie den unvergänglichen Theil ihres Wesens: ihre
Seelen, mit Heiligkeit schmücken. -- Vor Gott ist nichts häßlich als die Sünde.
-- Von der Seligkeit Derer, die das Joch oder Kreuz Christi tragen, und der Welt
gekrenzigt sind.
\\(\textit{Seite \pageref{kap11_ab10}})

\end{description}
\normalsize

\section{1. Abschnitt} \label{kap11_ab1}

Der Stolz bleibt aber dabei noch nicht stehen; er verleitet auch die Menschen
zur Überschätzung ihrer selbst und zu einer übertriebenen Sorgfalt für ihre
Personen. Daher müssen sie so viele und pünktliche Aufwartung, kostbare
Möbeln\index{Möbeln!kostbare},
reiche und schöne Kleider\index{Kleidung} u. dgl. haben, worin ein großer Theil
der "`Hoffahrt
des Lebens"' bestehet, wovon Johannes uns sagt,
"`daß sie nicht vom Vater,
sondern von der Welt ist."'
\footnote{1. Johannes 2,16}
\index{Bibelstellen:!1. Johannes 2)}
Dieses war die Sünde, die Gott
den übermüthigen Töchtern Zions\index{Personen:!Zions Töchter} und dem stolzen
Fürsten und Volke von Tyrus \index{Orte:!Tyrus}zur
Last legte. Man lese das 3te Kapitel beim Jesaias\index{Bibelstellen:!Jesaja
3)}, und das 27ste und 28ste beim
Ezechiel\index{Bibelstellen:!Ezechiel 27)} \index{Bibelstellen:!Ezechiel 28)},
beurtheile dann das genwärtige Zeitalter nach den Sünden, die nun im
Schwange gehen, und schließe, was für Gerichte die jetzt lebenden Völker zu
erwarten haben. Ich will hier nur den ersten Gegenstand dieses Kapitels, nämlich
den übertriebenen Werth, den die Menschen auf ihre Personen legen, in
Betrachtung ziehen, und das Übrige im letzten Abschnitte dieser Abhandlung, der
von der Üppigkeit handelt, und wo es eine schickliche Stelle finden wird,
weiter auseinandersetzen.

\section{2. Abschnitt} \label{kap11_ab2}

Daß die Menschen im Allgemeinen stolz auf ihre Personen sind, besonders wenn sie
einigen Anspruch auf vornehme Abkunft\index{Stammbaum!vornehmer} oder auf
Schönheit\index{Schönheit} machen können, ist eben
so bekannt, als unangenehm zu betrachten. Der erstere Fall hat oft schon unter
dem männlichen, und der letztere unter dem weiblichen Geschlechte sehr heftige
Streitigkeiten veranlaßt; und es sind auch nicht selten schon Männer, um der
Weiber willen, oder durch deren Anreizung, in solche Zänkereien verwickelt
worden. Was für Lärm und Gezänke ist nicht schon in der Welt vorgefallen, wenn
es die Bestimmung des Alters eines Stammes oder einer Familie galt? Oder wenn
entschieden werden sollte, wessen Vater oder Mutter, oder Urgroßvater und
Urgroßmutter die mehrsten Ahnen zählte? Welchem Stamme oder Zweige Dieser oder
Jener angehöre? Was für Wappen\index{Wappen} der Eine oder der Andere gefühlt
habe, und wer
nun das Recht zum Vorrange besitze? wiewohl, meines Erachtens, unter allen
Thorheiten des Menschen keine weniger vernünftig scheinende Gründe zu ihrer
Entschuldigung aufzuweisen hat, als diese.

\section{3. Abschnitt} \label{kap11_ab3}

\index{Herabsetzen!Jemanden}Denn erstlich: Was liegt daran, von wem Jemand
abstammt, insofern er selbst
keinen bösen Ruf hat; da nur seine eigenen Tugenden ihn erheben, und keine
andere als seine eigenen Laster\index{Laster} ihn erniedrigen\index{erniedrigen!sich selbst} können? Die Verdienste eines
Vorfahren können die schlechten Handlungen eines Menschen nicht
entschuldigen\index{entschuldigen};
sie beweisen vielmehr seine Entartung. Und da Tugend nicht durch Geburt erlangt
werden kann; so werde ich auch in der Tat durch meine Abkunft weder besser noch
schlechter. Auch gilt diese eben so wenig in Gottes Augen, als sie in den Augen
der Menschen etwas gelten sollte. \textbf{Kein Verständiger wird Beleidigungen
bloß
deswegen leichter ertragen, weil sie ihm von einem Manne von hoher Geburt
zugefügt werden, oder Gunstbezeigungen darum desto eher ausschlagen, weil
Derjenige, der sie ihm erzeigt, von geringer Abkunft ist.} Ich gestehe, daß es
eine große Ehre sein würde, gar keine Familienflekken\index{Personen:!Schwarzes Schaf in
Famiele} zu haben, und sein
Erbtheil von einem Stamme herleiten zu können, dem man nichts zum Vorwurfe
machen konnte. Allein das hat man noch nie angetroffen; selbst nicht in der
gesegnetsten Familie, die jemals auf Erden lebte; ich meine, auch in
Abrahams\index{Personen:!Abrahams Familie}
Familie nicht. Auch kann der Umstand, daß Jemand von reichen und hoch betitelten
Vorfahren abstammt, weder seinen Kopf mit Verstand, noch sein Herz mit Wahrheit
erfüllen; diese Eigenschaften sind eines höheren Ursprunges. Demnach ist es bloß
Eitelkeit und höchst verwerflicher Stolz\index{Stolz}, wenn Jemand von großem
Gewichte und
Ansehen in der Welt einen Andern deswegen geringachtet, weil dieser ihm an
Gebiete und Rang nicht gleich ist. Es kann ja leicht sein, daß der Letztere die
Verdienste seiner Vorfahren besitzt, während der Erstere nur die Früchte
genießt, welche die Verdienste seiner Vorfahren hervorbrachten. So kann also der
Eine durch seine Vorfahren, der Andere aber durch sich selbst groß sein; und
Welcher von Beiden verdient nun den Vorzug?

\section{4. Abschnitt} \label{kap11_ab4}

Ach! sagt der auf seine Abkunft stolze Mensch, Alles geht schlecht in der Welt,
seitdem wir so viele Emporkömnmlinge\index{Personen:!Emporkömnmlinge} unter dem Adel
zahlen! Was mögen aber wohl
Andere von seinen Vorfahren gesagt haben, als diese zuerst anfingen, sich in der
Welt auszuzeichnen? Denn sie und alle Menschen und Familien, ja, alle Staaten
und Reiche der Welt mußten doch auch ihr Emporkommen haben, oder ihren Anfang
nehmen. Wenn daher Familien ihren Adel aus ihre alte Adkunft, und nicht auf ihre
Tugenden gründen, \textbf{so machen sie es eben so, wie \index{Kirche!die
beste}\index{Kirche!die Wahre} die Kirche, welche behauptet,
die Wahre zu sein, weil sie die älteste, nicht weil sie die beste ist.} So kann
es aber nicht gehen. Wenn das Alter etwas gelten soll, so muß es ein Alter in
der Tugend sein, und dieses muß vor jeder andern alten Abstammung den Vorzug
haben. \textbf{Sonst könnte Jemand von höheren Adel als sein Vorfahr, von dem er
seinen
Adel ableitet, und der Vorfahr, der ihm seinen Adel erwarb, von geringeren Adel
als Er sein; eine Seltsamkeit, welche die Geschicktesten in der
Wappenkunde\index{Wappen!Wappenkunde} wohl
schwerlich werden erklären können.} Es ist allerdings sonderbar, daß Jemand
sollte einen höhern Adel besitzen, als seine Vorfahren, denen er den seinigen
verdankt. Wenn dieses aber ungereimt ist, wie es denn wirklich ist, so ist der
Neugeadelte der sich seinen Adel durch seine Tugenden erworben hat, der wahre
Edelmann, und es können daher auch nur Diejenigen, die seinen Tugenden
nachahmen, auf seine Ehre Anspruch machen, obgleich keine übrigen Abkömmlinge
sich auf ihre Abkunft von ihn: berufen. und auch seinen Namen führen mögen. Wenn
also nur die Tugend wahren Adel verleihen kann, wie selbst die
Heiden\index{Personen:!Heiden} behaupten;
so kann folglich eine Familie auch nur so lange wahrhaft adelig sein, als sie
tugendhaft ist. Und da ferner die Tugend nicht dem Geblüte oder Abkunft anhängt,
sondern in den guten Eigenschaften der Nachkommen bestehet, so folgt klar, daß
die Abkunft kein Vorrecht gewahrt; sonst würde diese die Tugend, als unnöthig,
ausschließen, und Derjenige, dem es an Abkunft mangelte, auch auf die Wohlthaten
der Tugend Verzicht leisten müssen, und dadurch würde dann der Adel, sobald es
ihm an Alter fehlte, herabgesetzt, und die Tugend bei den Adeligen entbehrlich
gemacht werden.

\medskip

Nein, man lasse immer die Namen der Abkunft folgen, aber Adel und Tugend müssen
Hand in Hand gehen; da sie so nahe Verwandte sind. So hat Gott selbst es
verordnet, der am besten alle Dinge recht und billig zu vertheilen weiß. Er
richtet sich mit seinem Wohlgefallen oder Mißfallen nicht nach der Abkunft der
Menschen, und siehet auch nicht auf das, was sie waren, sondern auf das, was sie
sind. Denn
\textit{"`er gedenkt nicht der Gerechtigkeit des Menschen, der seine
Gerechtigkeit verläßt;"'}
\footnote{Ezechiel 18}
\index{Bibelstellen:!Ezechiel 18)}
und wird noch viel weniger den
ungerechten wegen der Tugenden seiner Vorfahren als einen Gerechten betrachten.

\section{5. Abschnitt} \label{kap11_ab5}

Wenn indessen Diejenigen, die ihre Abkunft so hoch anrechnen, sich verpflichtet
halten wollen, Gott dadurch zu ehren, daß sie denn, was Er in der heiligen
Schrift uns bekannt machen läßt, Glauben beimessen, so werden sie finden,
\textit{"`daß
Er im Anfange alle Menschengeschlechter, die auf dem ganzen Erdboden wohnen,
aus einem Blute gemacht hat,"'}
\footnote{Apostelgeschichte 17,26}
\index{Bibelstellen:!Apostelgeschichte 17)}
und daß wir alle
von einem Elternpaare abstammen. Eine untrüglichere Geschlechtskunde werden auch
die Besten unter uns nicht aufweisen können. Weiter herab kommen wir auf
Noah\index{Personen:!Noah}, den zweiten Stammvater des menschlichen Geschlechts,
und bis auf ihn
wissen wir also etwas Zuverlässiges von unsern Vorfahren. Was für einen Antheil
uns nun aber in unsern Tagen an den Titeln und Auszeichnungen gebühret, welche
Andere seit jener Zeit entweder durch Gewalt\index{Gewalt} sich zugeeignet oder
durch Tugend
erworben haben, dürfte wohl -- wenn wir auch nur einige Jahrhunderte
zurückgehen, -- schwer zu bestimmen sein.

\section{6. Abschnitt} \label{kap11_ab6}

Mich dünkt auch, es müsse uns klar einleuchten, daß die Geburt Niemand zu einem
vorzüglichen Menschen machen könne, wenn wir bemerken, daß diese Leute von
vornehmer Abkunft, wenn sie ihren ganzen Staat und Putz abgelegt haben, von
Natur kein besonderes Merkmal an sich tragen, das sie von ihren geringern
Nebenmenschen unterschiede. Und lassen wir sie selbst urtheilen, so werden sie
uns gestehen müssen, daß sie bei allen Vorzügen ihrer Geburt dennoch jenen
Leidenschaften\index{Leidenschaften} unterworfen sind, welche die Menschen mit
einander gemein haben,
und folglich die Vornehmern den Geringern gleich machen, wo; nicht noch weiter
als Andere von der Tugend entfernen, die allein wahren Adel verleihen kann.
Hiervon geben uns, leider! die beklagenswerthe Unwissenheit und Zügellosigkeit,
die wir unter nur zu Vielen unserer vornehmen Leute antreffen, die
überzeugendsten Beweise. Und was für einer Geburt wollen sie dieses zuschreiben?

\section{7. Abschnitt} \label{kap11_ab7}

Wie dem nun auch sein mag, so ist es, nachdem ich dieses Alles gesagt habe, um
eine falsche Eigenschaft herabzuwürdigen, keinesweges meine Absicht, einer
andern, die nicht weniger tadelnswerth ist, dar Wort zu reden. \textbf{Man wolle
mich
nicht so verstehen, als suchte ich die eingebildete
Klasse\index{Gesellschaftsklassen!abschaffen} über den vornehmen
Stand zu erheben. Davon bin ich weit entfernt; denn das grobe
Benehmen\index{Benehmen!grobes} roher
Menschen würde die Sache nicht verbessern. Mein Zweck ist nur, Allen zu zeigen,
worin der wahre Adel bestehet,} damit Jeder auf dem Wege der Tugend und des
Edelmuthes darnach streben möge. Auch muß ich, nach Allem, dem Edelmanne große
Vorzüge einräumen, die seinen Stand wirklich erheben; so wie der Apostel
Paulus\index{Personen:!Paulus}
den Juden\index{Personen:!Juden}, welche, stolz auf ihre Gesetze und äußern
Beobachtungen, die Christen
beleidigten, nachdem er sie gedemüthiget hatte, in Ansehung ihrer Verfassung und
Einrichtung vor allen andern Völkern den Vorzug gab. \textbf{Ich muß gestehen,
die Lage
unserer \index{Personen:!Gehobener Stand, Vorteile} Großen ist der des niedrigeren
Standes weit vorzuziehen. Denn erstlich
haben sie größere Macht, Gutes zu thun, und wenn hierin ihre Herzen mit ihrer
Kraft und Fähigkeit übereinstimmen, so dienen sie in jedem Lande dem Volke zum
Segen. Zweitens, da die Augen der Menge auf sie gerichtet sind, so ist es ihnen
leicht, durch Güte, Gerechtigkeit und Wohlthätigkeit\index{Wohlthätigkeit} sich
allgemeine Zuneigung
und Achtung zu erwerben. Drittens sind ihre Umstände nicht so beschränkt als die
der niedrigern Klasse, und sie haben folglich mehr Hilfsmittel, mehr
Muße\index{Muße} und
Gelegenheit, durch Lesen\index{Erkenntnis!Lesen} und Umgang ihren Verstand
auszubilden\index{Erkenntnis!Verstand auszubilden} und ihre
Leidenschaften zu zügeln. Viertens haben sie auch mehr Zeit zu
reisen\index{Erkenntnis!reisen}, um das
Betragen anderer Nationen zu beobachten, ihre Gesetze, Sitten, Gebräuche und
Vortheile kennen zu lernen, und das Nachahmungswerthe derselben in ihr Vaterland
zu verpflanzen.} Alles dieses macht es den Großen leichter als Andern, sich
Achtung und Ehre zu erwerben; und wenn es unter ihnen um wahrhaft guten Ruf zu
thun ist, dem stehen Mittel und Wege genug zu Gebote, auf die beste Weise dazu
zu gelangen. Da es aber, leider! oft der Fall ist, daß die Großen wenig daran
denken, für ihren Wohlstand Gott die Ehre zu geben, und site die Wohlthaten, die
sie von ihm genießen, durch ein ihm wohlgefälliges Leben sich dankbar zu
erweisen, sondern da sie, im Gegentheile, nicht selten wie
\textit{"`ohne Gott in der
Welt leben,"'}  und bloß der Befriedigung ihrer sinnlichen Neigungen nachgehen;
so zeigt auch der Allmächtige seine Hand\index{Gott!Strafe Gottes} oft dadurch, daß
er sie arm macht\index{Verarmung}, oder
sie ausrottet\index{ausrottung}, und Andere von mehr Tugend und Demuth zum
Besitze ihrer Stellen
und Würden erhebt. Doch muß ich auch bemerken, daß es unter den Leuten von hohem
Range auch schon oft Einige gegeben hat, die sich durch ungewöhnliche Tugenden
auszeichneten, und deren Beispiele wie glänzende Lichter in ihren Familien
schienen, so daß ihren Nachkommen ein beständiges Streben, den guten Namen des
Hauses den Verdiensten seines Stifters gemäß aufrecht zu erhalten, gewissermaßen
natürlich geworden war. \textit{Wenn aber, in Wahrheit, irgend ein Vorzug mit
einer
vornehmen Abkunft verbunden ist, so liegt er in der Erziehung\index{Erziehung},
die der Stand
gewahrt, und nicht in der Abkunft selbst,} deren Ächtheit übrigens auch oft
zweifelhaft oder ungewiß ist, wohingegen die Erziehung immer einen mächtigen,
Einfluß auf die Neigungen und Handlungen des Menschen hat. Hierin zeichneten
sich vorzeiten die Adeligen und Vornehmen in unserm Königreiche besondere aus;
und es wäre sehr zu wünschen, daß unsere jetzigen Großen es sich möchten
angelegen sein lassen, die Wirthschaftlichkeit, gute Ordnung und tugendhafte
Erziehung in ihren Familien wieder herzustellen, die unter ihren würdigen
Vorfahren üblich war, als die Adeligen ihrer großen und edlen Taten wegen
geehrt wurden, und Nichts Jemand mehr der Schande und Verachtung aussezte, als
wenn er von adeliger Abkunft war, und keine Tugend besaß, um seinen Adel
aufrecht zu erhalten.

\section{8. Abschnitt} \label{kap11_ab8}

Doch ich habe noch einen höhern Beweggrund hinzuzufügen. Das herrliche
Evangelium Jesu Christi ist ja auch dieser nördlichen Insel verkündigt
worden, und da ihre Bewohner aus allen Ständen an dasselbe zu glauben bekennen;
so laßt mich euch bitten und bewegen, die Ehre zu suchen und nach der
Auszeichnung zu trachten, die durch diese himmlische
\index{Gnade}Gnadenaustheilung Gottes
allen wahren Gläubigen widerfahrt, die dem \index{Lamme, Das}Lamme folgen,
\textit{"`welches die Sünden
der Welt hinwegnimmt."'}\index{Sünde!tilgung}
\footnote{Johannes 1,29.}
\index{Bibelstellen:!Johannes 1)}
Nehmet mit Sanftmuth sein gnädiges
Wort in euren Herzen auf, welches die sinnlichen Lüste\index{Lüste} der Welt
überwindet, und
die Seele auf den heiligen Pfad leitet, der zur wahren Glückseligkeit führet.
Hier giebt es Freuden zu genießen, die noch kein fleischliches Auge gesehen,
kein sinnliches Ohr gehört, kein weltlichgesinntes Herz\index{weltlichgesinntes!Herz} vernommen hat, aber den
Demüthigen, die sich in Wahrheit zu Gott bekehren, durch seinen Geist
geoffenbaret werden. Bedenket, daß ihr nur Geschöpfe seid; daß ihr sterben müßt,
und nach dem Tode in’s Gericht kommt.\index{Gericht!Jüngstes}

\section{9. Abschnitt} \label{kap11_ab9}

Persönlicher Stolz beschränkt sich aber nicht allein auf den hohen Werth, den
die Menschen aus ihre Abkunft legen; er verleitet auch sowohl die Unadeligen als
die Adeligen zu einer üdertriebenen Schätzung ihrer Personen, besondere wenn sie
auf körperliche Schönheit\index{Schönheit} Anspruch machen können. \textbf{Es
ist zum Bewundern, wenn man
siehet, wie sehr Einige von ihrer Person eingenommen sind; als wenn sonst nichts
in der Welt ihrer Aufmerksamkeit werth sei, oder die Achtung Anderer verdiene!
Doch würde es noch ihre Thorheit vermindern, wenn ihr Herz sich entschließen
könnte, nur die Hälfte der Zeit, die sie mit Waschen,
Schminken\index{Schminken}, Parfümiren und
Puzanlegen verschwenden, dazu anzuwenden, daß sie an Gott und an ihr Ende
dachten.} Alle jene Dinge müssen aufs pünklichste besorgt und aufs künstlichste
gemacht werden, und an Ersparung der Kosten ist dabei nicht zu denken. Was daher
daß Übel noch vergrößert, ist, daß mit dem, was der Stolz eines Einzelnen
verlangt, die Bedürfnisse von zehn andern Menschen befriedigt werden könnten.
\textbf{Ja, ist es nicht grobe und entsetzliche Sünde\index{Sünde}, daß der
Stolz und die Eitelkeit
einer Nation mehr verschlingen, als die Unterhaltung aller ihrer Armen kosten
würde?} Und was haben die Menschen bei allen diesen Thorheiten für einen Zweck?
-- Um sich bewundern und verehren zu lassen; um Liebe einzuflößenz die Augen der
Zuschauer auf sich zu ziehen und ihre Neigungen zu gewinnen. Und bei dem allen
sind sie noch dazu so eigen, daß es schwer ist, ihnen zu gefallen. Nichtes ist
ihnen gut, oder fein oder modig genug. \index{Sonnenbräune}Ach! die Sonne, diese
die Erde
erquickende Wohlthat des Himmels, darf sie nicht bescheinen; sie möchte ihre
Haut verderben! Der Wind muß nicht wehen, er könnte ihren Putz in Unordnung
bringen! O! der schändlichen Verzärtelung! -- Während sie aber so über alles in
der Welt sich selbst schätzen, \textbf{sind sie doch Sklaven\index{Personen:!Sklaven};
gefesselt vom Geiste des
Stolzes und der Eitelkeit,}der sie beherrscht, und den sie durch Bewunderung
ihres Wuchseo, ihrer Gesichtsbildung und ihrer Haut, verehren und anbeten.

Der Zweck aller solcher künstlichen und kostbaren Bemühungen und Anstrengungen
ist nur zu oft kein anderer, als zu gefallen und unerlaubte Liebe zu erwecken,
wodurch beide Geschlechter oft in eine eben so traurige als strafbare Lage
versetzt werden. \index{Ehe}\index{Fremdgehen}\index{Eifersucht}\textbf{Bei
unverheiratheten Personen sind immer die Folgen einer
solchen Liebe verderblich; denn, wenn sie auch nicht zu unkeuschen Begierden
Veranlassung giebt, so legt sie doch nie den Grund zu einer festen und
dauerhaften Verbindung, die nur auf gegenseitiger innerer Achtung umd
Werthschätzung sicher beruhen kann, so wie der Mangel an einer solchen Grundlage
die Hauptursache ist, daß es so viele unglückliche Ehen in der Welt giebt. Bei
Verehelichten ist ein solches Benehmen noch sündlicher; das beide Theile, dem
heiligen Gesetze der Ehe gemäß, nur einander zu gefallen suchen sollten. Und
wenn sie darin der üppigen und eitlen Jugend nachahmen, so ist dieses immer ein
schlechter Beweis ihrer gegenseitigen Liebe und häuslichen Glückseligkeit. Das
eitle Schminken und Herauspuzen ihrer Personen giebt ihnen das Ansehen, als
wollten sie auf Eroberungen ausgehen, und wo dieses wirklich der Fall ist, da
sind die Folgen davon oft schrecklich; sie brechen in Mißvergnügen und
Eifersucht, endlich in Haß aus, und enden gewöhnlich mit Trennungen und
Ehescheidungen, ja, nicht selten sogar mit Vergiftungen und andern schändlichen
Ermordungen.} Kein Zeitalter kann uns von dergleichen traurigen und entsezlichen
Wirkungen des Stolzes und der Eitelkeit deutlichere Begriffe geben, als das
gegenwärtige, welches vornehmlich in unserm Königreiche, den nachtheiligen
Einfluß eines üppigen und ausschweifenden Lebens auf die Tugend, Ruhe, Mäßigteit
und Gesundheit der Familien aus unzähligen Beispielen erklärt.

\section{10. Abschnitt} \label{kap11_ab10}

Noch muß ich nothwendig bemerken, daß unter allen menschlichen Gesthöpfen solche
Beweise des Stolzes und der Eitelkeit am wenigsten \index{Personen:!alten
Menschen}\index{Personen:!hessliche Menschen} alten und häßlichen Personen
anstehen, -- wenn ich die Verunstalteten und von der Natur schlecht
Ausgestatteten häßlich nennen darf. \textbf{Denn alte Personen können nur auf
deas, was
sie einst waren, stolz sein; welches zu ihrer Schande beweist, daß ihr Stolz
ihre Schönheit überlebt hat, und sie sich, statt ihre Thorheit zu bereuen, nur
neuen Stoff zur Reue bereiten.} Die Häßlichen machen es aber noch schlimmer; sie
sind stolz auf Etwas, das sie nie besaßen, und auch niemals erlangen können. Ja,
es scheint, als wenn ihre Gestalt ihnen zu einer deständigen Demütigung ihres
eitlen Geistes dienen solle; und hierauf stolz sein, heißt in der Tat, den
Stolz des Stolzes wegen lieben, und sich seiner Herrschaft ergeben, ohne eine
Versuchung dazu zu haben. Und dennoch habe ich in meinem ganzen Leben keine
Menschen gesehen, die mehr von sich eingenommen gewesen ware, als gerade diese.
Wie ist es doch möglich, daß der Stolz die Menschen so sehr bethören und
Verblenden könne? Ist denn die Parteilichkeit ihrer Herzen so groß, daß sie mit
ihren Augen nicht mehr recht sehen können? -- Dieses beweiset in der Tat, daß
die Eigenliebe die Menschen blind macht. Daß sie aber mit ihrer Eitelkeit noch
Verschwendung verbinden, und so viele Kosten aus Etwas verwenden, das gar nicht
zu ändern ist, zeigt wirklich die höchste Thorheit; besonders wenn man bedenkt,
daß ihre Anwendung der Dinge, die für schön gehalten werden, ihre Häßlichkeit
nur noch mehr heraushebt und erst recht bemerkbar macht, weil diese ihnen so
übel anstehen.

\medskip

Aber aus der Thorheit solcher Personen können wir deutlich abnehmen, was für ein
Wesen der Mensch geworden ist, nachdem er seinen ursprünglichen herrlichen
Zustand verloren hat. Und alle diese Übel kommen, wie Jesus einst von der
Sünde\index{Sünde}
sagte,
\textit{"`aus seinem Innern"'}
\footnote{Matthäus 15,11. und Matthäus 15,18-20.}
\index{Bibelstellen:!Matthäus 15)}
Sie sind nämlich
die Folgen seines Mangels an Aufmerksamkeit auf
\textit{"`daß Wort seines Schöpfers in
seinem Herzen"'}
\footnote{1. Mose 30,14. Römer 10,8}
\index{Bibelstellen:!1. Mose 30)}
\index{Bibelstellen:!Römer 10)}
welches ihm seinen Stolz
entdeckt, ihn Demuht\index{Demuht} und
Selbstüberwindung
\footnote{\texttt{'Selbstverleugnung' ersetzt durch 'Selbstüberwindung'}}
lehret, und sein Gemüth auf den
wahren Gegenstand seiner Verehrung und Anbetung richtet, indem es dasselbe mit
einer feierlichen Achtung und Ehrfurcht erfüllt, die der höchsten Herrschaft und
Majestät des Allmächtigen gebühret. O! \index{Tod} \index{Gericht!Jüngstes}
\index{Vergänglichkeit}\textbf{was ist der arme sterbliche Mensch? --
Ein lebender Staub; aus gleichem Staube der Erde gebildet, als er mit seinen
Füßen detritt! Kann er doch mit all seinem Stolze sich nicht vor der Zerstörung
einer Krankheit sichern, und noch vielweniger den Streich des Todes abwenden!
Möchten daher doch die Menschen die Unbeständigkeit alles Sichtbaren, die
Trübsale und Widerwärtigkeiten ihres Lebens, und die Gewißheit sowohl ihrers
Hinscheidens, als auch der Erwartung des ewigen Gerichts in reifliche
Betrachtung ziehen, so ließe sich hoffen, daß sie
\textit{"`ihre Werke an das Licht
Christi in ihren Herzen bringen würden, damit sie erkenneten, ob sie in Gott
gethan wären oder nicht,"'} }
\footnote{Johannes 3,20+21}
\index{Bibelstellen:!Johannes 3)}
wie der Lieblingsjünger, als
Worte aus dem Munde seines theuern Meisters, uns sagt. -- Bist du nun
wohlgestaltet, von angenehmer Gesichtsbildnng und schön; ein musterhafter
Abdruck der Würde des menschlichen Wesens? O! so bewundere die Macht, die dich
so schuf. Führe ein Leben, das der edlen Gestalt und wundervollen Einrichtung
deines Körpers entspricht, und laß die Schönheit desselben dich lehren, deine
Seele mit Heiligkeit\index{Heiligkeit}, dem Schmucke\index{Schmuck} der
Lieblinge Gottes, zu zieren. Und bist du
ungestaltet und häßlich, so preise nichtsdestoweniger die ewige Güte, daß sie
dich nicht zu einem unvernünftigen, Wesen erschuf, und laß es dein Bestreben
sein, durch Hülfe der Gnade, die dir verliehen ist, -- \index{Gnade} denn die
göttliche Gnade
ist Allen erschienen, -- deine Seele mit unvergänglicher Schönheit zu schmücken.
Bedenke, \textit{"`daß die Tochter des Könige der Himmel,"'} die Kirche Christi
\index{Kirche!Kirche Christi} deren
\index{Kirche!Mitgliedschaft} Mitglieder alle wahre Christen\index{Personen:!wahre Christen}
sind, \textit{"`in wendig ganz herrlich geschmückt ist,"'}
Und wenn deine Seele sich  darin hervorthut, so wird das Mangelhafte deines
Körpers ihren Glanz nur um so mehr erheben. Denn in Gottes Augen ist nichts
häßlich, als die Sünde\index{Sünde}; und Alle, welches Geschlechtes und Standes
sie auch
sind, und wie sie auch gestaltet sein mögen,
\textit{"`die mit ihrem Herzen reden und
nicht sündigen;"'}
\footnote{Psalm 4,5}
\index{Bibelstellen:!Psalm 4)}
die in dem heiligen Lichte Jesu über die
Regungen und Neigungen ihrer Herzen wachen und jedes Böse in der Geburt
ersticken; die das Joch oder Kreuz Christi \index{Kreuz} lieben, und dadurch
täglich der Welt
gekreuzigt\index{Kreuz!gekreuzigt} werden; alle Diese führen ein inneres Leben mit
Gott, welches die
vergänglichen Lebensgenüsse der Welt an Schönheit und Dauer unendlich
übertrifft.
