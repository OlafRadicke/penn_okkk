
In diesen Anhang werden die wichtigsten Begriffe des Quakertums erklärt. Die Erklährung beschrenken sich ausschlisslich auf das frühe Quakertum und den konservativen Flügel des Quakertums. Heut gibt es eine schier unübersehbare vielfalt - um es mal positiv aus zu drücken - von Quakertraditionen, Flügeln und Gruppen, die unmöglich alle abgehandet werden können. Hier eine unvollständige Liste:

\begin{itemize}
\item Nontheist Friends
\item Independent Quaker
\item Chrisozentrische Quaker
\item Liberales Quakertum
\item Religiöser Sozialismus
\item Esoterische Quaker
\item Beanite Friends
\item Conservative Friends
\item Evangelical Friends 
\item New Foundation Fellowship
\item Quaker Universalist Fellowship
\item Free Quakers
\end{itemize}

Deshalb beschrenke ich mich bewust auf die ursprünglichen Wurzeln. Die oben genannte Gruppen werden aber alle kurz angerissen. Wenn keine spezielle Gruppe oder Flügel genannt wird, sind aber immer die \textit{frühen Freunde} (Quaker) oder die \textit{Conservative Friends} (konservative Quaker) gemeint.


% Mögliche Schriftgrössen: \small  \small   \footnotesize
\newcommand{\articlesize}{\footnotesize}

\twocolumn


\section*{A}

\articlesize
\begin{description}

 \item[Abolitionisten] Einige Quaker waren bekannte Abolitionisten. So zum Beispiel $\to$Benjamin Lay und $\to$John Woolman. Wobei selbige gegen häftie Wiederstände in den eigenen Reihen ankämfen mussten.\endnote{Die Aufzeichnungen von John Woolman. Aus der Zeit der Sklavenbefreiung. 2. Auflage, Lorber U. Turm Verlag, 1964, ISBN 3799901019}

 \item[Amnesty International]

 \item[Anthony, Susan B.]

 \item[Ältester]

 \item[Andacht]

 \end{description}
\normalsize

\section*{B}

\articlesize

\begin{description}

 \item[Baker, Eric] $\ast$22.9.1920 \dag~July 1976. Quaker und Mitbegünder von $\to$\textit{Amnesty International}, deren Secretary-General er 1966 bis 1968 war. Er gründete \textit{Amnesty International} 1961 in London mit Peter Benenson zusammen. Für die Quaker betreute er von 1946 bis 1948 ein Zentrum in Delhi, India. In den spähten 1950er und den frühen 1960er Jahren war Baker für das \textit{the Friends Peace \& International Affairs Committee} (FPIAC) tätig. Gegen Ende seines Lebens war er im $\to$\textit{Yearly Meeting} der \textit{Religious Society of Friends in London} und dem \textit{Triennial Meeting of the Friends World Committee for Consultation} aktiv.

 \item[Barclay, Robert]

 \item[Barclays Bank PLC] Ein international agierendes Finanzunternehmen aus Großbritannien.

 \item[Beerdigung]

 \item[Bennenungsausschuß]
 
 \item[Bezirksversammlung]
 \item[Birthright Membership]
 \item[Birthright Friends] $\to$\textit{Birthright Membership}
 \item[Birthright Quaker] $\to$\textit{Birthright Membership}

 \item[Bunyan, John] $\ast$28. November 1628 zu Elstow bei Bedford, \dag31. August 1688 in London. Baptistenprediger. Bekannt durch seinen Hauptwerk \textit{"`Pilgerreise zur seligen Ewigkeit"'} (orig. \textit{The Pilgrim’s Progress from This World to That Which Is to Come}). Um 1656 bis 1657 lieferte er sich ein heftigen Schlagabtausch mit dem Quaker Edward Burrough, in dem sie Streitschriften veröffentlichten. Von John Bunyan stammte die Schriften \textit{"`Some Gospel Truths Opened in which he attacked Quaker beliefs"'} und \textit{"`A Vindication of Some Gospel Truths Opened"'}. Von Edward Burrough stammten \textit{"`The True Faith of the Gospel of Peace"'} und \textit{"`Truth (the Strongest of All) Witnessed Forth"'}. Darin griff er die Quaker dafür an, das sie das $\to$\textit{"`Innere Licht"'}, seiner Meinung nach, fälschlicher weise, über die Bibel stellten. Spähter behandelte auch noch mal $\to$George Fox die Auseinandersetzung mit John Bunyan in \textit{"`The Great Mystery of the Great Whore Unfolded"'}. Um so bemerkenswehrtet war, das sich viele Jahre spähter, die Quaker bei dem zweiten Gefängnisaufenthalt (Ab 1675) von John Bunyan erfolkreich für dessen Freilassung einsetzten.

 \item[Burrough, Edward] $\ast$1634 \dag1663 gehörte zu den so genanten $\to$\textit{Valiant Sixty}. $\to$\textit{Bunyan, John}
 \item[by convincement (Mitgliedschaft)]
 \end{description}

\normalsize


\section*{C}

\articlesize

\begin{description}

 \item[Cadbury, John] $\ast$12.8.1801 \dag11.5.1889, Quaker, Schokoladen-Fabrikant aus Birmingham (England).

 \item[Campaign for Nuclear Disarmament (CND)] Zu Deutsch ``Kampagne für nukleare Abrüstung''. Der Quaker $\to$\textit{Eric Baker} war Mitbegründer der Organisation. Das Logo der Organisation wurde dann zum Friedenszeichen schleht weg. Der erste der Märsche die von CND organisiert wurde, der \textit{Aldermaston March} zu Ostern (4. bis 7. April 1958), wurde Vorbild für so genannten ``Ostermärsche'' in verschiedenen westeuropäischen Ländern. Nach Deutschland wurden die Ostermärsche unter anderem von dem Quakerehepaar Tempel getragen.

 \item[Ceresole, Pierre] $\ast$17.8.1879 in Lausanne \dag23.10.1945 in Le Daley bei Lausanne. Schweizer Pazifist, Quäker, Mathematiker und Gründer des Service Civil International (SCI).

 \item[Chrisozentrische Quäker] Das ist ein Oberbegriff der nicht genau definiert ist und bezeichnet nur vage eine Tendenz, die Rolle und Bedeutung der Person Jesus zu betonen. Unter $\to$\textit{Evangelical Friends} spielt der Begriff keine rolle, da selbige der Rolle Jesus Christus ohne hin sehr betonen. Er wird er in leberalen Quakertum benutz, um sich innerhalb des Quakerzweigs von universalistischen Strömungen ($\to$\textit{Quaker Universalist Fellowship})zu distanzieren oder auch von den $\to$\textit{Nontheist Friends}. Inhaltlich gibt es überschneidungen zu dem $\to$\textit{New Foundation Fellowship}, da die Frühen Freunde klar in der christlichen Tradition standen.

 \item[Clarks] Schuhfirma, gegründet 1825 von den Quakerbrüdern Cyrus and James Clark.
 
 \item[Clearness (Committees)] Ein Vorgehen bei schwirigen Problemen, Fragen und Kriesen. Es soll dabei gemeinsam in der Gruppe nach dem Willen Gottes gesucht werden. Eine geplante Vermählung kann zum Beispiel ein solcher Anlass sein.

 \item[Clerk] Im Deutschem: "`Schreiber"'. In ermangelung eines Klerus (Pastor, Prister oder Pfarrer) gibt es eine Person in der Versammlung, die Protokoll fürt. Selbige Person hat keinerlei Auorität. Ist ist lediglich ihre Aufgabe, beschlüsse und Feststellungen schriftlich fest zu haten. Hier zu muss sie in einer dienenden Grundhaltung versuchen, zu erfassen, worin die Versammlung sich einig ist. Dazu werden vom Schreiber der Versammlug Formierungsvorschläge gemacht. Gibt es keine weiteren Wortmeldungen, signalisiert die Versammlung durch schweigen ihrer Zustimmung und der Schreiber kann den Beschluss niederschreiben. Bei Wortmeldungen obliegt es dem Schreiber (und seinem Gefühl) Wem und in welcher Reihenfolge er zum Sprechen aufruft. Der Wunsch zum Sprechen, signalisieren die Anwesenden dadurch, das sie aufstehen und warten.

\item[Concern] (Anliegen/Berufung)
    Friends believe that anyone may feel called by God. Friends consider carrying out a concern to be a form of ministry. Often there may be a meeting for clearness to test the concern after which the meeting may well support the person in their concern. Many well-known organisations, such as the American Friends Service Committee, Don't Make a Wave Committee (the predecessor organisation to Greenpeace), Oxfam and Amnesty International, have been founded by Friends "`acting under concern"'.


 \item[Conservative Friends] Conservative (Deutsch: \textit{Konsevativ}) ist hier nicht im politischen Sinne gemeint, sondern in theologischen Sinne. Dieser Flügel orientiert sich also eng an den Überzeugungen der $\to$\textit{Frühen Freunden}. Also \textit{"`Old School Quaker"'} wenn man so will. Das $\to$\textit{Ohio Yearly Meeting} wird z.B. in der Regel diesem Conservative-Flügel zugeordnet.

 \item[Convinced Friend] Gemeint sind Quaker die einmal bewust aus überzeugung zum Quakertum übergetreten (Konvertiert) sind. Das Pendant ist quasie das $\to$\textit{Birthright Membership}. Im gegensatz zum \textit{Birthright Membership} wird \textit{Member by Convincement} von allen Versammlungen anerkannt. Strittig ist aber wieder auf jemand "`Quaker"' ist, wenn er sich zum Quakertum bekennt und danach lebt; oder erst, wenn er in einer $\to$\textit{Monatsversammlung} als Mitglied aufgenommen wurde. Es gibt einige Beispiele in der Quakergeschichte, wie zum Beispiel bei $\to$\textit{Benjamin Lay}, wo keine formale Mitgliedschaft gestand (oder verlohren ging) und die trozdem unbestritten als Quaker gelten.

 \item[Cookworthy, William]$\ast$12.4.1705 in Kingsbridge, Devon, England; \dag17.10.1780 in Plymouth. Quaker, Apotheker, Chemiker und Erfinder. Er gilt als Pionier sowohl der Kaolin-Industrie in Cornwall und Devon als auch der Porzellanherstellung in England.

 \item[Cromwell, Oliver] $\ast$25.4.1599 in Huntingdon, \dag3.9.1658 in Westminster, Gründer der englischen Republik, regierte als Lordprotektor England, Schottland und Irland während der kurzen republikanischen Periode der britischen Geschichte. Die Quaker genossen zum Teil seinen Schutz auf Grund guter Beziehungen, die zum Teil aus der Zeit des englischen Bürgerkrigs stammen. Denn nicht wenige Quaker waren frühr Mitglieder der \textit{New Model Army}, deren Oberbefehlshaber Oliver Cromwell war.

 \end{description}

\normalsize
\section*{D}

\articlesize

\begin{description}
 \item[dark countries]

 \item[Deutsche Jahresversammlung]

\item[Diggers] Splittergruppe der $\to$Levellers.

 \item[Doppelmitgliedschaft]

 \item[Dewsbury, William] $\ast$1621 \dag1688 Quaker, Bemühte sich ervolkreich um die Aussöhnung zwischen $\to$George Fox und $\to$James Nayler.

 \item[Dyer, Mary] $\ast$um 1611 in London, \dag1.6.1660 in Boston, Quäkerin, die in Boston (Massachusetts) gehängt wurde, weil sie die Stadt wiederholt betreten hatte, obwohl die Quäker verbannt waren. Sie gilt als die erste religiöse Märtyrerin der Quaker und als die letzte religiöse Märtyrerin Nordamerikas.

 \end{description}

\normalsize
\section*{E}

\articlesize

\begin{description}

 \item[Eccles, Solomon] $\ast$1618 \dag1683, Quaker,  bekannt duch exzentrisches
 Verhalten. Etwar da durch, das er nackt auftrat, um damit zu zeigen, das er
 sich wieder im paradiesisch sündenfreien Urzustand befände (als Strafe dafür
 wurde er ausgepeitscht).

 \item[Ekklesiologie]



 \item[Elisabeth von der Pfalz] $\ast$26.12.1618 in Heidelberg; \dag8.2.1680
 in Herford. Empfing unter anderem $\to$William Penn und korospondierte mit
 Quaker. Trat aber nie zum Quakertum über auch wenn sie mit William Pann eine
 Freundschaft bis zu ihrem Lebensende verband. Beim englischen König setzte sie
 sich für die Duldung der Quäker ein und in ihnem Haus wurden Andachten und
 Gottesdienste abgehalten.\endnote{ \\
 \texttt{http://www.kirchenlexikon.de/e/elisabeth\_v\_h.shtml }, \\
 Band I (1990)Spalten 1494-1495 Autor: Friedrich Wilhelm Bautz}

  \item[Elders] $\to$\textit{Ältester}

  \item[Esoterische Quäker] Esoterische Strömungen gab es schon seid den Anfängen
  des Quakertums. Durch Gerorge Fox wurden diese aber bekämpft und zurückgedrängt.
  Eine überaus interessante Person und Vertrete diese Strömung in den Anfängen
  des Quäkertums war $\to$\textit{Franciscus Mercurius van Helmont}. Aber auch
  heute noch ist diese Strömung gerade im $\to$\textit{Liberales Quakertum} und
  $\to$\textit{Universalismus} des Quakertums überaus pressend. Und auch in der
  Deutschen Jahresversammlung.
 \endnote{Seite "`Deutsche Jahresversammlung"'. In: Wikipedia, Die freie
 Enzyklopädie. Bearbeitungsstand: 24. Dezember 2009, 10:46 UTC. URL: \\
 \texttt{http://de.wikipedia.org/w/index.php?title=Deutsche\_Jahresversammlung\&oldid=68383041} \\
 (Abgerufen: 5. Januar 2010, 22:26 UTC)  }

 \item[Evangelical Friends]
  Die \textit{Evangelical Friends} sind von den heute exsistirenden Flügeln -
  rein Äusserlich - am weitesten vo \textit{"`Urquakertum"'} entfehrnt. Sie haben
  viele Dinge wieder eingeführt, gegen die $\to$\textit{George Fox} und seine
  Freund gekämpft und wo für sie Gesundheit und Leben aufs Spiel gesetzt haben.
  Die \textit{Evangelical Friends} nennen ihrer Versammlungshäuser wieder
  "`Kirchen"', wo gegen auch William Penn in "`Ohne Kreuz keine Krohne"'
  ausfühlich stellung genommen hatt (Siehe hier zu Kapitel 5, 6.Absatz, Seite
  \pageref{kap5_ab6}). Bei den \textit{Evangelical Friends} gibt es auch wieder
  Pastoren, liturgische Gesänge und zum Teil sogar Taufen mit Wasser. Ein
  weiteres Merkmal ist die Betonung der Missionstätikeit und die Autorität der
  Bibel und der Heilsrelevanz von Jesus als der Christus.
\medskip 
  Die Geschichte der \textit{Evangelical Friends} ist etwas koplizierter. In
  den 1870er und 1880er Jahren spalteten sich eine Reihe von Jahresversammlungen
  (Western 1877, Kansas 1880, Canada 1881, Iowa 1883) über die Frage ab, ob
  Pastoren eingeführt werden sollten. Die meisten Versammlungen die sich zu dem
  Pastoralen Zeig zusammenschlossen, kam aus dem Gurneyite-Zweig der eine
  Abspaltung von den $\to$\textit{Orthodox Friends} waren. Sie schlossen sich
  zu dem $\to$\textit{Five Years Meeting} (dem spähteren $\to$\textit{Friends
  United Meeting}) zusammen. Das war ein Zusammenschluss von Yahresversammlungen
  die für sich die $\to$\textit{Richmond Declaration} von 1877 übernahmen.
\medskip 
  1947 kam es zu einer weiteren Spaltung und der Gründung der
  \textit{Association of Evangelical Friends}. Diese bestad bis 1970. 1965
  formte sich die \textit{Evangelical Friends Association} die dann 1989 zu
  \textit{Evangelical Friends Church International} wurde.

  Heute ist der Flügel der \textit{Evangelical Friends} deutlich der grösste und
  es sieht nichts danach aus, als wenn sich daran in zukunft etwas ändern wird.
  Der wachstum der \textit{Evangelical Friends} ist ungebrochen. Im gegensatz
  zum den $\to$\textit{Liberal Friends} die stagnieren und zum Teil auch
  schrumpfen.
\endnote{\texttt{http://www.evangelicalfriends.org/}}\endnote{Wikipedia
contributors, "`Evangelical Friends International,"' Wikipedia, The Free
Encyclopedia, \\
\texttt{http://en.wikipedia.org/w/index.php?title=Evangelical\_Friends\_International\&oldid=329570717}
(accessed January 9, 2010).}


 \item[Evangelical Friends International (EFI)] beziehungsweise
 \textit{Evangelical Friends Church International (EFCI)}
 $\to$\textit{Evangelical Friends}


 \end{description}

\normalsize

\section*{F}

\articlesize

\begin{description}

\item[Facing Benches] In älteren Versammlungshäusern (englisch: meeting haus)
gibt es eine spezelle anordnung der Sitzbänge. Dabe gibt es eine einzelne Bank
die etwas erhoben ist und den anderen Bänken gegenüber steht. Dort sitzen die
so genannten \textit{"`gewichtien Freunde"'} ($\to$\textit{Weighty Friends}),
von denen zu erwarten ist, das sie in der Andacht das Wort ergreifen und zur
Versammlung sprechen werden. Das sind oft und traditonell die Ältestent
($\to$\textit{Ältester}) und die $\to$\textit{recorded ministers}. Heute ist
diese Form der Bestuhlung unüblich geworden. Der liberale Flügel
($\to$\textit{Liberal Friends}) verzichtet meist völlig auf
\textit{recorded ministers} und der \textit{evangelikale} beziehungsweise
\textit{pastoralen Fügel} ($\to$\textit{Evangelical Friends},
$\to$\textit{Pastoral Friends}) hat im Grunde jetzt seine Pastoren.
\endnote{Artikel "`Religious Society of Friends"'. Abgerufen 2010, January 12.
In Wikipedia, Die freie Enzyklopädie. Bearbeitungsstand: 16:41 Uhr 2010-01-13, \\ \texttt{http://en.wikipedia.org/w/index.php?title=\\
$\hookrightarrow$~Religious\_Society\_of\_Friends\&oldid=337403234}}


 \item[Fell, Margaret] $\ast$1614 in Marsh Grange, \dag23.4.1702 in Swarthmore.
 Gelegentlich auch als \textit{"`Mother of Quakerism"'} (\textit{"`Mutter des
 Quäkertums"'}) bezeichnet. 1632 heiratete sie Thomas Fell ($\ast$1598
 \dag1658). Dieser war Richter und gehörte dem \textit{"`Long Parliament"'}
 unter $\to$Oliver Cromwell an. Im Juni 1652 traf sie zum ersten mal
 $\to$George Fox. Nach dem Tod ihres ersten Mannes heiratete sie am 27.10.1669
 George Fox in Bristol. Margaret Fell war Autorin unter anderem des
 $\to$\textit{"`histrischen Friedenszeugnisses"'} und maßgeblich mit der
 Organisation der Missionsreisen beschäftigt.
\medskip 
 Sie war unermüdlich damit
 beschäftigt Quaker aus der Gefangenschaft zu befreiehen und Rechtsbeistand vor
 gericht zu organisieren. Sie sammelte und verteilte Gelder um in Notgeratene
 und verfolgte Quakern zu helfen. Sie selbst war mehr mal im Gefängnis. Das
 erste mal 1663 für die Verweigerung des Eids vor Gericht. 1670
 sorgte sogar ihr eigener Sohn, wegen Erbstreitigkeiten, für ihre Inhaftierung.
 Bis 1684 kam sie immer wieder in gefängnis dafür, weil sie sich weigerte die
 örtlicher Gottesdienste zu besuchen (heute unvorstellbar, weil dann Heute ~80\%
 der Bundesbürger im Gefängnis sitzen würde). Die Bedeutung ihrer Person kann
 man daran sehen, das noch heute 500 an sie adressierte Briefe, aus den Jahren
 der schlimmsten Verfolgiung(1654 bis 1670) erhalten sind.

 \item[Finlayson, James] $\ast$vermutlich am 29. August 1772 in Penicuik,
 Schottland; \dag vermutlich am 18. August 1852 in Edinburgh. Schottischer
 Quäker, der wesentlich zum Aufbau der finnischen Textilindustrie beitrug. In
 inischen Tampere des Jahres 1820, beaufsichtigte er den Aufbau eines
 gewaltigen Industriekomplexes, dessen Werke von der Wasserkraft des gestauten
 Tammerkoski angetrieben wurden. Zunächst wurde eine Wollweberei errichtet,
 1828 dann eine Baumwollmühle und zwei Baumwollspinnereien. Aber auch
 karitative Projekte wie die Einrichtung eines Waisenhauses witmete er sich.
 Obwohl Finlayson schon früh kontakte zu Quäkern hatte, trat er selbst erst
 sehr späht der $\to$\textit{Gesellschaft} bei.

\item[Finsternis] $\to$\textit{dark countries}

 \item[Fisher, Mary] $\ast$1623 \dag1698. Quakerin. Wollte 1657 zusammen mit
 fünf anderen Quakern den Sultan von Konstantinopel bekehren. Selbiger empfing
 sie auch freundlich, hörte sich an was sie zu sagen hatte und liess sie dann
 wieder nach hause fahren.

 \item[Five Years Meeting] Vorläuferorganisation von $\to$\textit{Friends
 United Meeting}.

 \item[Fox, Geroge] $\ast$Juli 1624 in Drayton-in-the-Clay, Leicestershire,
 heute Fenny Drayton, \dag13.1.1691. War einer der Gründerväter der Quaker und
 erfolgreicher Missionar. Wie fiele Andere, sass fiele mal im Gefängnis. Sein
 wichtigstes Verdienst ist wohl, die administrative Grundlage zu schaffen, nach
 der sich bis Heute die Versammlungen organisieren. Sehr bekannt ist er auch
 als Autor der des "`The Journal of George Fox"', in den er seine Erlebnisse
 während seiner Missions-Jahre schildert.

 \item[Free Quakers] Eine Abspaltung von Quäkern, die von ihren Versammlungen
 ausgeschlossen wurden, weil sie sich an den amerikanischen Unabhängigkeiskrig
 beteiligten. Als Gründer ist Samuel Wetherill bekannt. Das einzigste
 Versammlungshaus von ihnen steht noch heute in der 5th und Arch Street in
 Philadelphia. Es wurde mit prominenter Unterstützung wie Washington und
 Benjamin Franklin 1783 errichtet. Es war ein Prizip der \textit{Free Quakers}
 niemanden aus zu schiessen. weder aus theologischen noch aus moralischen
 Gründen. Nachdem einige Jahre niemand mehr die Andachten besucht hatte, wurde
 das Versammlungshaus 1834 geschlossen.
\endnote{\texttt{http://www.qhpress.org/quakerpages/qwhp/freequakertoc.htm}}
\endnote{Claus Bernet: "`Samuel Wetherill"' In: Biographisch-Bibliographisches
Kirchenlexikon (BBKL). Band 30, Nordhausen 2009, ISBN 978-3-88309-478-6,
Spalte~1555–1557.}

 \item[Friedenszeugnis] Das Friedenszeugnis teilt sich in ein so genanntes
 $\to$\textit{"`historisches Friedenszeugnis"'} und ein \textit{generelles}
 oder \textit{praktisches} Friedenszeugnis, wo z.B. die Verweigerung des
 Wehrdienstes dazu gehört. Für dieses Friedenszeugnis wird das Quakertum der
 so genannten \textit{Friedenskirche} zugerechnet.\endnote{Seite "`Friedenskirche
 (Konfession)"'. In: Wikipedia, Die freie Enzyklopädie. Bearbeitungsstand:
 12.~Februar 2010, 08:16 UTC. URL: \\
 \texttt{http://de.wikipedia.org/w/index.php?title=Friedenskirche\_(Konfession)\&oldid=70569238}
 (Abgerufen: 7. März 2010, 12:28 UTC) }

 \item[Freund(e)] Zum Teil bezeichnen sich Quaker untereinander als "`Freund"'.
 Streckenweise wurde der Begriff inflationär oder floskelhaft benutzt, was
 als unaufrichtig empfunden wurde, so das es einige ablehnen, grundsätzlich
 aller Quaker als "`Freunde"' zu bezeichnen. Abgeleitet wird das von der
 Bezeichnung "`Religiöse Gesellschaft der Freunde"'. Und so benutzen viele
 Quaker unreflektiert "`Freund"' als stehenden Begriff für "`Mitglied"'.
 
 \item[Freunde der Freunde] Sind Menschen die mit dem Quakertum symparisieren
 und engen kontakt zu Quakern pflegen, aber selber keine Quaker sind.

 \item[Friedensthal] Eine ehemalige deutsche Quäkerkolonie, die von 1792 bis
 1870 im heutigen Bad Pyrmonter Ortsteil Löwensen existierte. Begründer und
 wichtigste Figur war $\to$\textit{Ludwig Seebohm}.

 \item[Friends] $\to$\textit{Freund}

 \item[Friends House] $\to$\textit{Meeting House}

 \item[Friends General Conference (FGC)] Gehört neben der
 $\to$\textit{Friends United Meeting} und $\to$\textit{Evangelical Friends
 International} zu einer der drei größten Quäkerdachorganisation. Die FGC
 vereint unter sich Quäkergruppen mit unprogrammierter Andacht. Die Mitglieder
 der FGC vertreten mehr zu liberalen theologischen Ansichten
 ($\to$\\textit{Liberal Friends}). \endnote{Artikel "`Friends General
 Conference"'. In: Wikipedia, Die freie Enzyklopädie. Bearbeitungsstand:
 13.6.2009, 12:54 UTC. URL: \\
 \texttt{http://de.wikipedia.org/w/index.php?title=\\
 $\hookrightarrow$~Friends\_General\_Conference\&oldid=62161112}}



 \item[Friends World Committee for Consultation (FWCC)] Eine weltweite
 Dachorganisation der Quäker. Im Gegensatz zu Dachorganisationen wie
 $\to$\textit{Friends General Conference}, hat das FWCC keine thologischa
 Austrichtung. Die Aufgabe von FWCC ist der rein infomelle Kontakt von Quakern.
 Gegründet wurde die Organisation 1937 in Swarthmore, Pennsylvania.
 Grundsätzlich sind nur Quaker-Organisationen Mitglied in der FWCC, es seiden
 es handelt sich um allein lebende Quaker, die kein Anschlus an einer
 Jahresversammlung haben. Die ca. 70 Jahresversammlungen, die dem FWCC
 angeschlossen sind, representieren etwar 265.000 der insgesammt 368.000
 Quäkern welt weit.\endnote{"`an introduction to quakerism"', Pnk Danelion,
 2007, ISBN 052160088-x}\endnote{Sie Artikel "`Friends World Committee for
 Consultation"', in der freie Enzyklopädie Wikipedia. Bearbeitungsstand:
 18. November 2009, 13:29 UTC. URL: \\
 \texttt{http://de.wikipedia.org/w/index.php?title=\\
 $\hookrightarrow$~Friends\_World\_Committee\_for\_Consultation\&oldid=66966169} }



 \item[Frühe Freund] (engl. \textit{early friends}) Damit ist die erste
 Quakergeneration gemein. Namentlich, zum Beispiel $\to$\textit{Geroge Fox}
 und $\to$\textit{Willam Penn}.

  \item[Friedrich, Leonhard] $\ast$1889 \dag1979 $\to$\textit{Freund der
  Freunde}. Gelegendlich
  wird behaubtet Leonhard Friedrich sei Quaker gewesen, was aber nicht der
  Fall war. Während der der Jahre des Nationalsozialismus, bemühte er sich
  um den Erhalt des $\to$\textit{Quakerhauses} in Bad Pyarmond.\endnote{In
  dem Austellungskatalog der Austellung "`Stille Helfer - Die Quäkerhilfe
  im Nachkriegsdeutschland"' (12. Januar bis 12. März 1996, Zeughaus des
  Deutschen Historischen Museum), ist ein Dokument abgebildet, woraus
  die Religionszugehörigkeit von Leonhard Friedrich herforgeht.}

 \item[Fry, Elizabeth] $\ast$21.5.1780 in Cartham Hall bei Norwich,
 \dag12.10.1845 in Ramsgate. Quäkerin. War britische Reformerin des
 Gefängniswesens. Sie war die Tochter von $\to$\textit{John Gurney}

 \item[Fuchs, Emil] $\ast$13.5.1874 in Beerfelden/Odenwaldkreis; \dag13.2.1971
 in Ost-Berlin. Deutscher Theologe, Religiösen Sozialist,
 evangelisch-lutherischen Pfarrer und Quaker. 1933 verlor Fuchs seine
 Anstellung als Pfarrer und trat im selben Jahr der $\to$\textit{Deutschen
 Jahresversammlung} bei. In den 30ern war Fuchs in der Berliner Gruppe eine
 prägende Persönlichkeit gewesen, in seinen Predigten fanden viele Besucher
 geistigen Halt.


 \end{description}

\normalsize
\section*{G}

\articlesize

\begin{description}

\item[Gathered Meeting]
    A meeting for worship, where those present feel that they were particularly in tune with the leadings of the Spirit.

 \item[Geschäftsversammlung]

 \item[Gesellschaft, Die] Kurzform von "`Die $\to$\textit{Religiöse Gesellschaft der Freunde}.

\item[Gewichtige Freunde] $\to$\textit{Weighty Friends}

\item[Glücksspiel]

\item[Gottesdienst] $\to$\textit{Andacht}

 \item[Graeff, Abraham Isacks op den]

 \item[great separation] Die erste grosse Spaltung im Quakertum. Sie begann im
 Jahre 1827/28 im $\to$\textit{Philadelphia Yearly Meeting}. Dem war ein
 längerer schwelender Streit vorrausgeggangen der sich an der Person
 $\to$\textit{Elias Hicks} polarisierte. Bei der Spaltung schlugen sich
 ungefähr $\nicefrac{2}{3}$ der Mitglieder auf die Seite dessen Flügel dann
 spähter $\to$\textit{"`Hicksite"'} genannt wurde. Der andere Teil der
 Versammlung bildete den Flügel der als $\to$\textit{"`Orthodox"'} bezeichnet
 wurde. \textit{Hicksites} betonten die Bedeutung des $\to$\textit{Inneren
 Lichtes}, der Glaubens- und Gewissensfreiheit. Während den Orthodoxen die
 Bedeutung der Bibel und der Buse betonten. In der folgezeit kam es zu
 ähnliche Spaltungen in New York, Baltimore und andern Versammlungen.
 \endnote{\texttt{http://www.quakerinfo.org/quakerism/brancheshistory.html} \\ "`A Brief History of the Branches of Friends"', Stand:~2010-01-09}

 \item[Greenpeace]

\item[Gruppe] $\to$\textit{Monatsversammlung}

\item[Gurneys] Alte Quakerfamilie

\item[Gurney, Joseph John] (2 August 1788 – 4 January 1847) was a banker in Norwich, England and an evangelical Minister of the Religious Society of Friends (Quakers), whose views and actions led, ultimately, to a schism among American Quakers.

\item[Gurneyite(s)]

\item[GYM] \textit{German Yearly Meeting}, \textit{Deutsche Jahresversammlung}. $\to$\textit{Jahresversammlung}
 \end{description}

\normalsize %bis hier...
\section*{H}

\articlesize

\begin{description}

 \item[Hanson, Elizabeth]

 \item[Heirat]

 \item[Helmont, Franciscus Mercurius van] $\ast$um~1614 in Vilvoorde bei Brüssel, Spanische Niederlande, heute Belgien, \dag1699 in Cölln bei Berlin. Quäker, Alchemist, Kabbalist, Theosoph, Arzt, Rosenkreuzer.\endnote{"`Helmont, Franciscus Mercurius van,"', Biographisch-Bibliographischen Kirchenlexikon, Band XXV (2005) Spalten 586-597 Autor: Claus Bernet, http://www.bautz.de/bbkl/h/helmont\_f\_m.shtml}

 \item[Hicks, Elias] $\ast$19.3.1748 \dag27.2.1830. Quäker, Wanderprediger, Abolitionisten. An seiner Person entzündete sich ein heftiger Streit, der dann in den  $\to$\textit{"`Hicksites-Orthodox-Schisma"'} der Quäker mündete. Daraus ging dann die $\to$Hicksites und die $\to$\textit{Orthodoxe Linie} hervor.

 \item[Hicksite] Name des eine der entstandenen Flügel, bei der $\to$\textit{"`great separation"'}. Bennant nach $\to$\textit{Elias Hicks}.

 \item[Histrischen Friedenszeugnisses]
 
 \item[Hodgkin, Thomas]

\item[Hold in the Light]
    To recognize concern in one's self for another person or situation. This is often considered to be synonymous with praying for someone.

 \item[Hoover, Herbert C.]

 \item[Hopkins, Johns]

 \item[Hutchinson, Anne] $\ast$Juli 1591, \dag~August 1643. War keine Quakerin. Wurde als Anne Marbury in England geboren; heiratete mit 21 Jahren William Hutchinson und siedelte mit ihm 1634 in die britische Kolonie Massachusetts in Nordamerika um. Sie trat hervor als selbstsichere Frau, was für ihre Zeit ungewöhnlich war. Sie Vorkämpferin des Feminismus und eine Kämpferin für Menschenrechte und Toleranz. Sie lehrte, dass Gott direkt zu jedermann spräche und nicht allein zum Klerus. Sie Beeinflusste die spätere Quäkerin Mary Dyer, die sich beide persönlich kannten.


 \end{description}

\normalsize

\section*{I}

\articlesize

\begin{description}

\item[I hope so]
    (British term) during a meeting for worship for business, when the clerk asks those present if they agree with a minute, Friends will usually say "I hope so" rather than "`yes"'. It is meant in the sense of "`I hope that this is the true guidance of the Holy Spirit"'.

\item[Independent Quaker]
% \endnote{\texttt{http://www.rcquakers.lomaxes.me.uk/about/about.htm}} Solange es Quaker
% gab, solange gab es auch \textit{"`theologische Insellösungen"'}. Ein solches Beispiel
% stellt einer Gruppe wie diese aus Großbritannien, die nicht der $\to$Jahresversammlung
% ($\to$Britain Yearly Meeting) angeschlossen ist da. Entscheidend ist im Grunde immer
% die jeweilige verwendete $\to$\textit{"`Ordnung des Zusammenlebens"'} in diesem Fall
% benutzt das selbständige $\to$Meeting die des \textit{$\to$Ohio Yearly Meeting}, auch
% wenn es Geographisch eigendlich zum \textit{Britain Yearly Meeting} gehört. Der Gruppe
% war in diesem Fall die theologische Ausrichtung der Jahresversammlung zu liberal.

 \item[Innere Bestrafung/Züchtigung]
 \item[Innere Gnade]
 \item[Innere Offenbarung] $\to$\textit{Inneres Licht}
 \item[Innere Kraft] $\to$\textit{Inneres Licht}
 \item[Innere Stimme] $\to$\textit{Inneres Licht}
 \item[Innerer Christus] $\to$\textit{Inneres Licht}
 \item[Innerer Judas]
 \item[Inneres Licht]
 \item[Inneres Ohr] $\to$\textit{Inneres Licht}
 \item[Inneres Zeugnis] $\to$\textit{Inneres Licht}
 \item[Innerlich kreuzigen]
 \end{description}

\normalsize
\section*{J}

\articlesize

\begin{description}
 \item[Jahresversammlung]

 \item[Jones, Rufus M.] $\ast$25.1.1863, \dag16.6.1948. US-amerikanischer Quäker, Autor, College-Professor, Philosoph und Mystiker.

 \item[Jones, Rice] \dag28.3.1693. Baptist, Quäker, Dissident. Er hielt in Nottingham religiöse Versammlungen in seinem Haus ab und seit 1657 auch im Schloß zu Nottingham. Lieferte sich Redeschlachten mit namhaften Quäkern seiner Zeit die ihn die ihm des Abfalls ($\to$Renterismus) bezichtigten.

 \item[Jungfreund]
 \end{description}

\normalsize
\section*{K}

\articlesize

\begin{description}


 \item[Keith-Kontroverse] $\to$\textit{George Keith}.

 \item[Klassen, Hans] $\ast$30.9.1893 \dag18.9.1959 Fällt durch seinen sein bizarrer Lebensweg etwas aus dem Ramen. Klassen war Rußlandmennonit, Lebensreformer und Bürokrat, Nationalsozialist und Kommunist und Quäker.\endnote{Mennonitischen Geschichtsblätter 2009, 66. Jahrgang. von Claus Bernet.}


 \item[Kryptoquäker] Von griechisch "`krypto"': geheim, verborgen. Ein Quäker, der sich nicht öffentlich als Quäker zu erkennen gibt, also ein "`Geheimquäker"'. Das gab es gelegentlich im 17. Jahrhundert, zu Zeiten schwerer Verfolgung, als nicht alle Quäker den Mut hatten, zu ihrer Gemeinschaft zu stehen. Verdächtigen Personen, die irgendwie als Quäker erschienen, wurde auch vorgeworfen, "`Kryptoquäker"' zu sein.\endnote{Verwendungsbeispiel: Claus Bernet, 2007, "`Deutsche Quäkerschriften. Band 2: Deutsche Quäkerschriften des 18. Jahrhunderts"', ISBN 9783487134086, ISBN 348713408X, Seite 371}

 \item[Kulturquakertum] $\to$\textit{Wharton, Joseph}

 \end{description}
\normalsize

\section*{L}

\articlesize

\begin{description}

 \item[Lay, Benjamin] $\ast$26.1.1682 in Colchester, England, \dag3.2.1759 in Abington, Pennsylvania. Englischer Quaker, Philanthrop und Schriftsteller. Trat er als Abolitionist gegen die Sklaverei ein und war Bekennt für Provokative Inszenierungen und Auftritte.

 \item[Levellers]


 \item[Liberales Quäkertum]
% Ganz wie der Begriff \textit{Chrisozentrische Quäker} eher eine Wage Bezeichnung für eine Grundhaltung oder Tendenz. Die meisten Europäschen Jahresversammlungen könnte man dort einordnen. Aber mit Vorbehalt! wie wir bei den <i>independent Quaker</i> schon gesehen haben, gibt es auch "Ausreißer". Mit <i>Liberales Quäkertum</i> ist meist gemeint das man so ziemlich jede Vorstellung haben kann, die man will. Was in der Regel nicht unter Liberales Quäkertum verstanden wird, ist aggressives Missionieren, Gottesdienste mit Gesang, Liturgie und Prediger/Pastor.</li>

 \item[Lightfoot, Hannah]

 \item[Lilburne, John] $\ast$ca.~1614, \dag28.8.1657. Auch bekannt als \textit{"`Freeborn John"'}. Er war einer der bekanntesten Wortführer der radikaldemokratischen $\to$Levellers in England des siebzehnten Jahrhunderts. Er ist der Bruder von Robert Lilburne.

 \item[Luffe, John 'Love'] $\ast$? \dag~?. Quaker. Bekanntgeworden durch den Versuch 1658 Papst Alexander VII bekehren zu wollen. Er kam dabei in Rom ums leben.\endnote{Biographisch-Bibliographischen Kirchenlexikon, Band XX (2002)Spalten 1160-1167 Claus Bernet über Perrot, John}

 \end{description}
\normalsize

\section*{M}

\articlesize

\begin{description}

 \item[Meeting for Sufferings]
 
 \item[meeting for worship]  $\to$\textit{Andacht}

 \item[Member by Convincement] $\to$\textit{Convinced Friend}
 
 \item[Ministry]
    the act of speaking during a meeting for worship. (Many Friends use the term more broadly to mean living their testimonies in everyday life). "`Vocal"' or "`proclamational"' refer to ministries that are verbal.
 
 \item[Monatsversammlung]
 
 \item[Musik]
 
 \end{description}

\normalsize

\section*{N}

\articlesize

\begin{description}
 \item[Nayler, James] $\ast$1618, \dag1660. Englischer Quäker. Ist der Nachwelt in erster Linie durch den Konflikt mit $\to$Geroge Fox in Erinnereung geblieben.
 \end{description}

\normalsize

\section*{O}

\articlesize

\begin{description}
 \item[Overseers]

 \item[Outreach]

 \item[Orthodox Friends]

 \item[Oxfam]
 \end{description}

\normalsize

\section*{P}

\articlesize

\begin{description}

 \item[Paquet, Alfons]

 \item[Pastorius, Franz Daniel]

 \item[Penn, Willam] $\ast$14.10.1644 in London, \dag30.7.1718 in Ruscombe, Berkshire. Quaker. Gründete die Kolonie Pennsylvania im Gebiet der heutigen USA, verfasste 1693 das Essay \textit{"`towards the Present and Future Peace of Europe"'}.

 \item[Perrot, John] $\ast$?, \dag1665 in Jamaika. Quaker. Wollte 1658 mit $\to$John 'Love' Luffe zusammen Papst Alexander VII bekehren.

  \item[Proceed as Way Opens]
    to undertake a service or course of action without prior clarity about all the details but with confidence that divine guidance will make these apparent and assure an appropriate outcome.

 \item[Projekt Alternativen zur Gewalt (PAG)]

 \end{description}
\normalsize



\section*{Q}

\articlesize

\begin{description}

 \item[Quaker (Name)]

 \item[Quäkerbüro]

 \item[Quaker by Convincement]

 \item[Quaker Gray]

 \item[Quäkerhaus]

 \item[Quaker Peace \& Social Witness (QPSW)]

 \item[Quäkerspeisung]

 \item[Quaker United Nations Office (QUNO)]

 \item[Quaker Universalist Fellowship] Das ist kein selbständiger Zweig des Quäkertums, sondern eine Bewegung innerhalb des Quäkertums, ähnlich wie $\to$\textit{New Foundation Fellowship} nur quasie in die genaue Gegenrichtung. Die Anhänger gehen so weit, das sie die Auffassung vertreten, das man auch als Hindu, Buddhist oder Moslem Quäker sein kann. Das Quäkertum eben etwas universelles ist.\endnote{\texttt{http://www.universalistfriends.org/}}

 \item[Quäkerzeugnis] $\to$Zeugnis


 \end{description}

\normalsize

\section*{R}

\articlesize

\begin{description}

 \item[Recorded Minister]
    A person whose vocal ministry (spoken contribution in meeting)—or another spiritual gift—is recognised as helpful and probably faithful to Divine leading, by the body of Friends to which they belong and formally recorded by that body. Not all Friends' organisations record ministers. Other Friends have adopted a defined process prerequisite for "`recording"'.
 

 \item[Religiöse Gesellschaft der Freunde]

  \item[Religiöser Sozialismus]
%  \endnode{\texttt{http://de.wikipedia.org/wiki/Religi\%C3\%B6ser\_Sozialismus}} Das ist eine Strömung die nicht nur auf das Quäkertum beschränkt war. Bekanntester Vertreter in Deutschland war Emil Fuchs\endnote{Wikipedia zu Emil Fuchs: http://de.wikipedia.org/wiki/Emil_Fuchs}. Seit Ende des zweiten Weltkriegs spielt die Strömung innerhalb der Deutschen Jahresversammlung keine Geige mehr.

 \item[Renterismus] Ein Schlagwort den Quakern benutzten, um andere zu dikreditieren, die in ihren Augen \textit{"`der Welt verfallen waren"'}. So zum Beispiel der Gruppe um   $\to$Rice Jones dessen Anhänger zu den landesbesten Football-Spielern und Ringern gehörten.\endnote{Biographisch-Bibliographischen Kirchenlexikons, Band XXIV (2005) Spalten 917-918 Autor: Claus Bernet, Zu Rice Jones}

 \item[Rest Home Projekt]

 \item[Richmond Declaration] 1877 \endnote{\texttt{http://en.wikipedia.org/wiki/Richmond\_Declaration}}.

 \item[Right ordering]
    has to do with proper conduct of a meeting for business. The term is often used in the negative, that is, if someone senses that something about the conduct of the meeting is not proper, they may object that "`this meeting is not in right ordering."'

 \end{description}

\normalsize

\section*{S}

\articlesize

\begin{description}
 \item[Schissma]

 \item[Schreiber]

 \item[Seekers]

 \item[Sense of the Meeting]

 \item[Service Civil International (SCI)]

 \item[Shaker]

 \item[Suffering]

 \end{description}

\normalsize

\section*{T}

\articlesize

\begin{description}

 \item[That of God in everyone]
    the belief in the presence of God within all people. Also referred to as the Inner Light.

 \item[travelling minute]

 \item[travelling in the ministry]
%     ein Wanderprediger mit einem Sendschreiben (minute)und Auftrag seines home meetings, kaum noch gebräuchlich. 

 \item[True Levellers] $\to$Diggers.

 \end{description}

\normalsize

\section*{U}

\articlesize

\begin{description}

\item[Unabhängige Quaker]

\item[Unattached Quakers]

 \item[Universalismus] $\to$\textit{Quaker Universalist Fellowship}

  \item[unprogrammed]
  
 \end{description}
\normalsize

\section*{V}

\articlesize

\begin{description}

 \item[Valiant Sixty] Eine Gruppe von Frühen Freunden, die eine Führungsrolle einnahmen und sich aktiv um die Verbreitung ihrer Botschaft bemühten. Zu den bekanntesden Mitgliedern zählen:
 
 \begin{itemize}
  \item Edward Burrough
  \item Margaret Fell
  \item Mary Fisher
  \item George Fox
  \item Elizabeth Hooton
  \item Francis Howgill
  \item James Nayler
  \item George Whitehead
 \end{itemize}

 \item[Versammlung] Je nach Kontext ist entweder der Gottesdienst gemeind, oder die Gemeinde. Das Wort "`Gemeinde"' wird meist nicht verwendet um eine Quakergemeinschaft zu bezeichnen. Das rührt aus dem Kirchenverständniss der Quaker ($\to$\~Ekklesiologie), die Kirche ausschlisslich als "`Gemeinschaft der Gläubigen"' versteht, und nicht als Gebäude oder Institution. Versammlung bezeichnet also lediglich eine nicht näher bezeichnete "`Organisationseinheit"'.
 
 \end{description}

\normalsize

\section*{W}

\articlesize

\begin{description}
 \item[Weighty Friends]

 \item[Whittier, John Greenleaf]

 \item[Winstanley, Gerrard] $\ast$1609 in Wigan, Lancashire, \dag10.9.1676) war ein protestantischer Reformer und politischer Aktivist in England. Er führte zur Zeit des Protektorats (1649–1658) von $\to$Oliver Cromwell die Gruppe der \textit{"`True Levellers"'} (englisch: wahre Gleichmacher) an, die von ihren Gegnern abfällig \textit{"`Diggers"'} (englisch: Buddler) genannt wurden.

\item[Woolman, John] $\ast$19.10.1720 in Northampton, New Jersey; \dag7.10.1772 in York. Amerikanischer Quäker-Prediger und vehementer Gegner der Sklaverei. Bekannt durch seine \textit{"`Aufzeichnungen"'} (englisch: \textit{Journal}).\endnote{Die Aufzeichnungen von John Woolman. Aus der Zeit der Sklavenbefreiung. 2. Auflage, Lorber U. Turm Verlag, 1964, ISBN 3799901019}

 \end{description}


\normalsize
\input{lex_x}
\section*{Y}

\articlesize

\begin{description}
 \item[Yearly Meeting] $\to$Jahresversammlung
 \end{description}
\normalsize
\input{lex_z}

\onecolumn

\theendnotes


\null\newpage