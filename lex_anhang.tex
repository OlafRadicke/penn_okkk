
In diesen Anhang werden die wichtigsten Begriffe des Quakertums erklärt. Die Erklährung beschrenken sich ausschlisslich auf das frühe Quakertum und den konservativen Flügel des Quakertums. Heut gibt es eine schier unübersehbare vielfalt - um es mal positiv aus zu drücken - von Quakertraditionen, Flügeln und Gruppen, die unmöglich alle abgehandet werden können. Hier eine unvollständige Liste:

\begin{itemize}
\item Nontheist Friends
\item Independent Quaker
\item Chrisozentrische Quaker
\item Liberales Quakertum
\item Religiöser Sozialismus
\item Esoterische Quaker
\item Beanite Friends
\item Conservative Friends
\item Evangelical Friends 
\item New Foundation Fellowship
\item Quaker Universalist Fellowship
\item Free Quakers
\end{itemize}

Deshalb beschrenke ich mich bewust auf die ursprünglichen Wurzeln. Die oben genannte Gruppen werden aber alle kurz angerissen. Wenn keine spezielle Gruppe oder Flügel genannt wird, sind aber immer die \textit{frühen Freunde} (Quaker) oder die \textit{Conservative Friends} (konservative Quaker) gemeint.


% Mögliche Schriftgrössen: \small  \small   \footnotesize
\newcommand{\articlesize}{\footnotesize}

\twocolumn


\section*{A}

\articlesize
\begin{description}

 \item[Abolitionisten] Einige Quaker waren bekannte Abolitionisten. So zum Beispiel $\to$Benjamin Lay und $\to$John Woolman. Wobei selbige gegen häftie Wiederstände in den eigenen Reihen ankämfen mussten.\endnote{"`Die Aufzeichnungen von John Woolman. Aus der Zeit der Sklavenbefreiung"'. 2. Auflage, Lorber U. Turm Verlag, 1964, ISBN 3799901019}

 \item[acting under concern] $\to$\textit{Concern}

 \item[Ältester] Diese Amt gibt es für jede Verwaltungsebene ($\to$\textit{Jahresversammlung, Monatsversamm, Viertel- bzw. Bezirksversammlung}). Haubt aufgabe sind seelsorgerische und organisatorische Aufgaben. Für diese Aufgabe werden in der Regel erfahrene und angesehende Mitglieder berufen. Die $\to$Ämterbesetzung erfolgt nicht per Wahl.

 \item[Amnesty International] Eine Organisation die sich für Menschenrechte einsetzt. Einer der Gründer $\to$Eric Baker war Quaker. Die Organisation selbst ist aber überkonfessionel.

 \item[Ämterbesetzung]  Die meisten Ämter werden nicht durch Wahlen besetzt, sondern ein von der Versammlung bestellter Ausschuß wird beauftragt geeignete Personen zu suchen, und der Versammlung vor zu schlagen. Die Versammlung prüft in einer $\to$\textit{Geschäftsversammlung} den Vorschlag. Gibt es ein $\to$\textit{Sense of the Meeting} wirde der vorgeschlagenen Person das Amt anvertraut. Herrscht keine Einigkeit oder gar Ablehnung, wir der Ausschuß gebeten einen neuen Vorschlag vor zu bereiten. Dazu kann der $\to$\textit{Bennenungsausschuß} neu zusammengesetzt werden.

 \item[Anliegen] $\to$\textit{Concern}

 \item[Anthony, Susan B.] $\ast$15.2.1820 in South Adams, Massachusetts \dag13.3.1906 in Rochester, New York. Qakerin. Auch \textit{"`Napoleon der Frauenbewegung"'} genatt, auf grund ihrer Pionierrolle in der US-amerikanischen Frauenrechtsbewegung. 1872 gab sie als erste Frau ihrer Stimme bei der Präsidentschaftswahl in den Vereinigten Staaten ab. Ein Jahr spähter wurde sie dafür verurteilt wegen \textit{"`unrechtmäßiger Wahlbeeinflussung"'}. Der Gerichtsprozess erregte sehr viel Aufsehen dabe. Erst 14 Jahre nach ihrem Tod 1920 wurde das Frauenwahlrecht in der USA einführt. Zwischen 1979 und 1981 sowie nochmals 1999 wurde eine Ein-Doller-Münze mit ihrem Porträt geprägt und ausgegeben.

 \item[Andacht] oder auch \textit{"`stille Andacht"'} (englisch: \textit{meeting for worschip}), wird der Gottesdienst genant bei Quakern. Findet meist Sonntags statt (muss aber nicht). Die Andacht zentrum und tragenenes Element des spirituellen Lebebs einer $\to$\textit{Versammlung}. Besonderes Merkmal der Quakerandacht ist, das es keine Liturgie gibt. In "`Ohne Kreuz keine Krohne"' Kapitel 5 ($\to$~Seite~\pageref{kap5_ab1}) geht William Penn ausführlich darauf ein, warum das so ist. Praktisch ist es so, das in der regel ein $\to$~\textit{Schreiber} die Leitung hat. Entweder dur eine kurze Ansprache wird von ihm die Andacht begonnen, oder dadurch das er sich einen Moment hinstellt, um zu signalisieren, das die Andacht begonnen hat und das Reden eingestellt werden soll. Hat jeder seinen Platz eingenommen und ist zur Ruhe gekommen. setzt sich der Schreiber. Im Verlauf der Andacht können Teilnehmer spontan aufstehen und zur Versammlung sprechen. Diese Redebeiträge sollten insperiert sein vom $\to$\textit{Innerein Licht}. Die Prüfung obliegt der Versammlung. Möglicherweise wird das gesate nach einiger Zeit der schweigenen Prüfung von einem weiteren Redner aufgegriffen. Oder es gibt verschiedenste Redebeiträge, die keinerlei inhaltliche Verbindungen haben. Die Redezeit wird nicht formal begrenst. Von $\to$~George Fox und $\to$~Ludwig Seebohm zum Beispiel ist bekannt das sie zum Teil über eine Stunde redeten (nicht immer zum Wohlgefallen der Anwesenden). Das sprechen/predigen ist jedem gestattet Männer, Freuen, Kinder und selbst Nichtquakern. So ist bekannt, das in Deutschland und Holland auch Mennoniten an Quakerandachten teilnamen und auch sprachen (dabei kam es auch gelegendlich auch zu so häftigen Tumulten, das weltliche Ordnungskräfte eingriffen, weil sich Anwohner über Lerm beschwerten)\endnote{Sünne Juterczenka, "`Über Gott und die Welt - Entzeitvisionen, Reformdebatten, und die europäische Quäkermission in der frühen Neuzeit"', Vandenhoeck\&Ruprecht, 2008, ISBN 978-3-525-35458-2, Dort Seite~188 und Seite~94}. Formal gibt es kein festgelegtes Ende, bei der Anacht. Hier ist es Aufgabe des Schreibers wahrzunehmen, ob die Zeit gekommen ist die Versammlung zu beänden.  Der Gottesdiens des evangelikalen Flügels des Quakertums ($\to$\textit{Evangelical Friends}) hat heute einen vorbestimmten Ablauf mit Gesang, Predigt und Gebet. Ähnelt also äusserlich kaum noch der ursprünglichen Quakerandacht.

 \end{description}
\normalsize

\section*{B}

\articlesize

\begin{description}

 \item[Baker, Eric] $\ast$22.9.1920 \dag~July 1976. Quaker und Mitbegünder von $\to$\textit{Amnesty International}, deren Secretary-General er 1966 bis 1968 war. Er gründete \textit{Amnesty International} 1961 in London mit Peter Benenson zusammen. Für die Quaker betreute er von 1946 bis 1948 ein Zentrum in Delhi, India. In den spähten 1950er und den frühen 1960er Jahren war Baker für das \textit{the Friends Peace \& International Affairs Committee} (FPIAC) tätig. Gegen Ende seines Lebens war er im $\to$\textit{Yearly Meeting} der \textit{Religious Society of Friends in London} und dem \textit{Triennial Meeting of the Friends World Committee for Consultation} aktiv.

 \item[Barclay, Robert] $\ast$23.12.1648 in Gordonstown, Morayshire; \dag3.10.1690 in Ury bei Aberdeen. War Theologe und Quaker und eine der wichtigsten Figuren des jungen Quakertums. Er war der Erste, der die thologischen rundsätze systematisch auf akademischen neveu ausformulierte. Seine Schriften erfreuten sich unter Quakern großer beliebtheit. Kaum ein Quaker-Autor seiner Zeit wurde in so vielen Sprachen übersetzt und in hoher Auflage vertrieben. Seine Bücher wurden oft auch für Missionszwecke verschenkt. Sein Elternhaus war zunächst calvinistisch geprägt. Ihm wurde eine umfassende Bildung zu Teil, die ihn sogar bis an die Collegium Scotorum nach Paris führte. Sein Vater konvertierte 1666 zum Quakertum und er selbst schon ein Jahr spähter. Schon kurz darauf reiste er mit den führenden Persönlichkeiten des Quakertums wie $\to$William Penn und $\to$George Fox zu missionszwecken nach Holland und Deutschland. Hier bei traf er auch mit $\to$Pfalzgräfin Elisabeth, der Äbtissin von Herford zusammen. Und Auch er setzte sich wie sie für der Quaker bei dem König von England ein. Er versuchte sein Einfluss am Hof von Jakob II. dazu zu nutzen. Wie viele andere Quaker, sass auch er für fünf monate im Gefängnis. Sein Werk ist stark gepregt von von der kabbalistischen Mystik wie sie unter anderem $\to$George Keith vertrat. Seine \textit{Apologie} gilt als sein Hauptwerk und prägte das Selbstverständnis des frühen Quakertums.\endnote{Seite "`Robert Barclay (Quäker)"'. In: Wikipedia, Die freie Enzyklopädie. Bearbeitungsstand: 27. November 2009, 10:13 UTC. URL:
 \\ \texttt{http://de.wikipedia.org/w/index.php?title=Robert\_Barclay\_(Qu\%C3\%A4ker)\&oldid=67322327}
 (Abgerufen: 19. Dezember 2009, 13:10 UTC) } \endnote{Sünne Juterczenka, "`Über Gott und die Welt - Entzeitvisionen, Reformdebatten, und die europäische Quäkermission in der frühen Neuzeit"', Vandenhoeck\&Ruprecht, 2008, ISBN 978-3-525-35458-2} \endnote{Claus Bernet: Robert Barclay (Quäker). In: Biographisch-Bibliographisches Kirchenlexikon (BBKL). Band 1, Hamm 1975, Sp. 367–368, \\  \texttt{http://www.bbkl.de/b/barclay\_r.shtml}}

 \item[Barclays Bank PLC] Ein international agierendes Finanzunternehmen aus Großbritannien. Gegründet wurde die Bank schon im Jahr 1690. Von Londoner Quaker. Der Name \textit{Barclay} wurde erstmals 1736 verwendet. Zu ihrem Reichtum kam sie während des Kolonialismus. Die fühe Geschichte der Bank steht in Verbindung mit der Quaker-Famiele $\to$Gurney. Und diese wiederum mit der \textit{Gurney's bank}. Aus der Famili  Gurney ent stammen die beiden - für die Quakergeschichte wichtigen - Persönlichkeiten $\to$Joseph John Gurney und $\to$Elizabeth Fry. \endnote{\\ \texttt{http://www.nzz.ch/2007/03/28/al/articleF1WAM.html} NZZ vom 28. März 2007 und \\ \texttt{http://en.wikipedia.org/wiki/Elizabeth\_Fry}} \endnote{Barbara Müller, "`Widerstand gegen Barclays"', In "`Finanzplatz Informationen"' 3/2005, \\ \texttt{http://www.aktionfinanzplatz.ch/pdf/fpi/2005/3/Barclays.pdf}}

 \item[Beanite Friends]
 Diese sind durch ein unfreiwilligen Schisma entstanden. Diese, kann man sie als $\to$liberale Quaker bezeichnen. Sie haben eine umprogrammierte Andacht und keine Pastoren. Sind aber eher Christozentrisch. Sie haben sich Abgespalten von den Iowa Conservative Yearly Meeting, als sich diese immer mehr dem $\to$evangelikalen Quakertum zuwanden. Das war um das Jahr 1881.
 \endnote{\texttt{http://en.wikipedia.org/wiki/Beanite}} \endnote{Beanite Quakerism. (2009, October 11). In Wikipedia, The Free Encyclopedia. Retrieved 13:28, December 19, 2009, from \\ 
 \texttt{http://en.wikipedia.org/w/index.php?title=Beanite\_Quakerism\&oldid=319319549}
 }


 \item[Beerdigung] Die Beedigung bei Quakern findet ganz wie der $\to$Gottesdienst, ohne Liturgie statt. In Sterbeanzeigen hieß es oft \textit{"`...der ganannt wurde...[Name des Toten]"'}. Das hing damit zusammen, das geglaubt wurde, das erst am \textit{Jüngsten Tag} oder im Hadis von Gott der richtige Name bestimmt wurde. Das rührt daher, das dass Quakertum ursprünglich eine eschatologischen Erweckungsbewegungs-Charakter hatte. Von Aussenstehenden wurde die Quakerbeerdigungen als pietätlos empfunden und sorgten für Spott und Entrüstung. \endnote{Seite "`Quäkertum"'. In: Wikipedia, Die freie Enzyklopädie. Bearbeitungsstand: 11. Dezember 2009, 00:40 UTC. URL:
 \\ \texttt{ http://de.wikipedia.org/w/index.php?title=Qu\%C3\%A4kertum\&oldid=67859053  }
\\ (Abgerufen: 22. Dezember 2009, 22:08 UTC) } \endnote{Sünne Juterczenka, "`Über Gott und die Welt - Entzeitvisionen, Reformdebatten, und die europäische Quäkermission in der frühen Neuzeit"', Vandenhoeck\&Ruprecht, 2008, ISBN 978-3-525-35458-2, Seite 275}

 \item[Bennenungsausschuß] Da die Ämter nicht durch Wahlen besetzt werden, wird eine Gruppe von Mitgliedern beauftragt der $\to$\textit{Versammlung} vorschläge für die Ämterbesetzung zu unterbreiten. Über die Besetzung der Ausschuss entscheidet eigendlich die Versammlung. 
 
 \item[Bezirksversammlung] Eine Bezirksversammlung vereinigt einige $\to$\textit{Monatsversammlungen} zu einer oranisatorischen-infomellen Einheit, um die Monatsversammlungen untereinander zu versammeln. Die \textit{Bezirksversammlung} ist eine Besonderheit des deutschen Quakertums. In anderen Ländern verwendet man den Begriff \textit{Vierteljahresversammlung}. Das ist eigendlich auch die korekte Selbstbeschreibung. Denn die Selbige trifft sich vier mal im Jahr und ist informelles Bindeglied zwischen Monatsversammlung und $\to$\textit{Jahresversammlung}. Im deutschen Quakertum wird auf der ebene der Bezierks- bzw. Vierteljahresversammlung die Mitgliederverwaltung abgewickelt. Also die Aufnahmen, Ausschlüsse und so weiter abgewickelt. Aus dem ganz einfachen Grund, weil in Deutschland die die meisten Mitglieder keine eigene Monatsversammlung haben, zu der sie gehen können (oder wollen). In anderen Ländern, ist die Monatsversammlung das eigendliche Zentrum des Gemeinschaftlichen Lebens. Es gibt Länder die auf den \textit{Luxus} einer Jahresversammlung verzichten.\endnote{Dazu zählt zum Beispiel das Moscow Monthly Meeting (MM), das Ramallah MM, Brummana MM, Barcelona MM, Hong Kong MM, Seoul MM und das Kinshasa MM. Pink Dandelion, "`An Introduction to Quakerism"', Seite 177, ISBN: 052160088X}
 
 \item[Birthright Membership] Das frühe Quakertum verstand sich als \textit{"`Gemeinschaft der Gläubigen"'} und ihrer Mitglieder wahren ausschlisslich Konvertiken. Schon eine Generation spähter, mit den in die Gemeinschaft \textit{hineingeborenen}, entstand sowas wie "`geborene Quaker"'. In laufe der Zeit entstanden fast sowas wie Quaker-Dynastin wie bei den $\to$\textit{Gurneys}. Man begann von "`Birthright Friends"' zu sprechen. Das Gegenstück dazu ist quasie das $\to$\textit{"`Member by Convincement"'}. Also dem Bekentnis zur Quakerüberzeugungen.
 
 \item[Birthright Friends] $\to$\textit{Birthright Membership}
 
 \item[Birthright Quaker] $\to$\textit{Birthright Membership}

 \item[Bunyan, John] $\ast$28. November 1628 zu Elstow bei Bedford, \dag31. August 1688 in London. Baptistenprediger. Bekannt durch seinen Hauptwerk \textit{"`Pilgerreise zur seligen Ewigkeit"'} (orig. \textit{The Pilgrim’s Progress from This World to That Which Is to Come}). Um 1656 bis 1657 lieferte er sich ein heftigen Schlagabtausch mit dem Quaker Edward Burrough, in dem sie Streitschriften veröffentlichten. Von John Bunyan stammte die Schriften \textit{"`Some Gospel Truths Opened in which he attacked Quaker beliefs"'} und \textit{"`A Vindication of Some Gospel Truths Opened"'}. Von Edward Burrough stammten \textit{"`The True Faith of the Gospel of Peace"'} und \textit{"`Truth (the Strongest of All) Witnessed Forth"'}. Darin griff er die Quaker dafür an, das sie das $\to$\textit{"`Innere Licht"'}, seiner Meinung nach, fälschlicher weise, über die Bibel stellten. Spähter behandelte auch noch mal $\to$George Fox die Auseinandersetzung mit John Bunyan in \textit{"`The Great Mystery of the Great Whore Unfolded"'}. Um so bemerkenswehrtet war, das sich viele Jahre spähter, die Quaker bei dem zweiten Gefängnisaufenthalt (Ab 1675) von John Bunyan erfolkreich für dessen Freilassung einsetzten.

 \item[Burrough, Edward] $\ast$1634 \dag1663 gehörte zu den so genanten $\to$\textit{Valiant Sixty}. $\to$\textit{Bunyan, John}
 \item[by convincement (Mitgliedschaft)]
 \end{description}

\normalsize


\section*{C}

\articlesize

\begin{description}

 \item[Cadbury, John] $\ast$12.8.1801 \dag11.5.1889, Quaker, Schokoladen-Fabrikant aus Birmingham (England).

 \item[Campaign for Nuclear Disarmament (CND)]

 \item[Ceresole, Pierre]

 \item[Chrisozentrische Quäker]
% Das ist ein Oberbegriff der nicht genau definiert ist und bezeichnet nur vage eine Tendenz, die Rolle und Bedeutung der Person Jesus zu betonen. Das das Beispiel oben von den \textit{Independent Quaker} zeigt das der Gruppe die Ausrichtung des Britain Yearly Meeting zu liberal war in Bezug auf die Christlichen Wurzeln. Es gibt aber keine "`Chrisozentrische Ordnung des Zusammenlebens"'. Unter diesen Begriff könnte man z.B. auch die Evangelical Friends sehen. Das Ohio Yearly Meeting wird sich theologisch aber überhaupt nicht in der Nähe von den Evangelicals sehen.

 \item[Clarks] Schuhfirma
 
 \item[Clearness Committees]


 \item[Conservative Friends]
%  Das schon genannte Ohio Yearly Meeting wird z.B. in der Regel diesem Flügel zugeordnet. Conservative ist hier nicht im politischen Sinne gemeint, sondern in theologischen Sinne. Dieser Flügel orientiert sich also eng an den <i>Frühen Freunden</i>. Also "Old School Quaker" währe für den Deutschsprachigen Raum weniger missverständlich. Ich persönlich tendiere hier zu. Mein Meeting ( http://WasKannstDuSelbstSagen.de/ ) ist aber er als Liberal ein zu ordnen, und wir kratzen uns nicht gegenseitig die Augen aus!</li>

 \item[Cookworthy, William]

 \item[Cromwell, Oliver] $\ast$25.4.1599 in Huntingdon, \dag3. 9.1658 in Westminster, Gründer der englischen Republik, regierte als Lordprotektor England, Schottland und Irland während der kurzen republikanischen Periode der britischen Geschichte. Die Quaker genossen zum Teil seinen Schutz.

 \end{description}

\normalsize
\section*{D}

\articlesize

\begin{description}

 \item[dark countries] Auf grund schlechter Erfahrungen (siehe: $\to$\textit{John
 'Love' Luffe}) mieden die frühen Quakermissionare katholische länder und nannten sie
 \textit{dark countries}. \textit{Dunkelheit} stand immer metaphorisch für dia
 abwesenheit von Gotteswirken. Menschen, die einen schlechten Lebenswandel führten,
 wurde oft vorgeworfen "`der finsternis verfallen"' zu sein. \endnote{Seite 74, in
 "`Über Gott und die Welt: Endzeitvisionen, Reformdebatten und die europäische
 Quäkermission in der Frühen Neuzeit"',
Band 143 von Veröffentlichungen des Max-Planck-Instituts für Geschichte, Max-Planck-
Institut für Geschichte (Göttingen), Autorin: Sünne Juterczenka, Verlag:
Vandenhoeck~\&~Ruprecht, 2008, ISBN: 3525354584 und 9783525354582, \\
\texttt{http://books.google.com/books?id=aP00obJjhT8C} }

\item[Deutsche Jahresversammlung] In den 1920er Jahren gab es zwei Deutsche
Jahresversammlungen. Eine er konservative ($\to$\textit{Conservative Friends} mit
Schwerpunkt Süddeutschland, und eine liberal bürgerliche ($\to$\textit{Liberales
Quäkertum}) mit Berlin, als geograpisches Zentrum. Heute existiert nur noch die
liberal-bürgerliche Deutsche Jahresversammlung. Die Deutsche Jahresversammlung hatte
nie mehr als unter 600 Mitglieder. Heute etwa 250. Die Deutsche Jahresversammlung hat
zwei besonderheiten. 1.) Betreut sie keine selbständigen
$\to$\textit{Monatsversammlungen} sondern Einzelmitglieder. 2.) Ist sie die
Jahresversammlung von zwei Ländern: Deutschland und Östereich. Also müsste sie
eigendlich \textit{"`Deutsch-Östereicher Jahresversammlung"'} heißen.
\endnote{Siehe hierzu den Abschnitt "`Albrecht, Hans"' in "`Quäker aus Politik,
Wissenschaft und Kunst: Ein biographisches Lexikon"' Auflage 2, Seite 17-19, ISBN
978-3883094694. Zitat: "`Hans Albrecht ist ein bedeutender Mitbegründer der Deutschen
Jahresversammlung. Er verkörperte den bürgerlichen Flügel der Quäker."'(S.18). Und im
Kapitel "`Stackelberg, Freiherr Traugott von (1891-1970)"' auf Seite 195 Zitat "`[...]
Hinderungsgründe schienen ihnen [den Stackelbergs] vor allem die überwigend bürgerlich
Sozialorientierung der Quäker in Deutschland, [...] und ihre zu intellektuell
ausgeprägte Einstellung."'}
\endnote{Zu den Mitgliderzahlen siehe, Claus Bernet, Leben zwischen evangelischer
Theologie und Quäkertum, aus dem Heft Materialdienst, 02/2008, ISSN 0934-8522,
Seite~29}


\item[Diggers] (Zu Deutsch "`Buddler"') also ein Spottname für eine Splittergruppe der
$\to$\textit{Levellers}. Ihr Gründer war $\to$\textit{Gerrard Winstanley}. Selber
nannte sich die Gruppe "`True Levellers"'. Also "`wahre Gleichmacher"'. Zwischen 1649
und 1651 existierten  einige Gruppen von ihnen, die öffentliches Land besetzten und Gütergemeinschaft lebten. Vorbild war die Urgemeinde der Aphostelgeschichte.\endnote{Seite „Diggers“. In: Wikipedia, Die freie Enzyklopädie. Bearbeitungsstand:
2. Januar 2010, 22:42 UTC. URL: \\
\texttt{http://de.wikipedia.org/w/index.php?title=Diggers\&oldid=68745857}
(Abgerufen: 2. Januar 2010, 22:46 UTC) }

\item[Dissenter] Überbegriff von einer Reihe von Gruppen die während oder kurz
nach dem englischen Bürgerkrig endstanden, und sich als Gegenbewegung zu
etablierten Konfesionen, wie der römisch-katholischen Kirche oder den Anglikanern
verstanden. Hier einige Beispiele:

\begin{itemize}
 \item \textit{Adamites}
 \item \textit{Anabaptists}
 \item \textit{Baptisten}~*
 \item \textit{Barrowists}
 \item \textit{Behmenists}
 \item \textit{Brownists}
 \item \textit{Congregationalists}~*
 \item \textit{Countess of Huntingdon's Connexion}~*
 \item \textit{Enthusiasts}
 \item \textit{Familists}
 \item \textit{Fifth Monarchists}
 \item \textit{Grindletonians}
 \item \textit{Muggletonians}
 \item \textit{Kongregationalisten}
 \item \textit{Levellers}
 \item \textit{Presbyterianer}~*
 \item \textit{Puritans}
 \item \textit{Philadelphians}
 \item \textit{Sabbatarians}
 \item \textit{Seekers}
 \item \textit{Socinians}
 \item \textit{Renters} bzw. \textit{Renterismus}
 \item \textit{True Levellers} bzw. \textit{Diggers}
 \item \textit{Unitarians}~*
\end{itemize}
*Existieren noch Heute

 \item[Doppelmitgliedschaft] Gemeint ist die Mitgliedschaft in Meheren Gemeinschaften
 oder Kirchen gleichzeitig. Das wahr in frühen Quakertum unmöglich. Selbst das
 Heiraten
 ausserhalb der Gemeinschaft war absolute Ausname und wurde nicht gerne gesehen. Im
 Quakertum des 20. Jahrhundert in Deutschland wahren aber die meisten Quaker der 2.
 Deutschen Jahresversammlung (also die von 1925 mit dem Zentrum in Berlin) Mitglieder
 einer Weitern meist evangelischen oder katholischen Landeskirche. Das war eine
 Konzesion an das vorwiegend bürgerliche Specktrum. Bis Heute ist aber eine
 Doppelmitgliedschaft in den meisten Jahresversammlungen auf der Welt unüblich.
 \endnote{Seite "`Deutsche Jahresversammlung"'. In: Wikipedia, Die freie Enzyklopädie.
 Bearbeitungsstand: 24. Dezember 2009, 10:46 UTC. URL: \\
 \texttt{http://de.wikipedia.org/w/index.php?title=Deutsche\_Jahresversammlung\&oldid=68383041}
\\ (Abgerufen: 5. Januar 2010, 19:06 UTC) }

 \item[Dewsbury, William] $\ast$1621 \dag1688 Quaker, Bemühte sich ervolkreich um
 die Aussöhnung zwischen $\to$\textit{George Fox} und $\to$\textit{James Nayler}.

 \item[Dyer, Mary] $\ast$um 1611 in London, \dag1.6.1660 in Boston, Quäkerin, die in
 Boston (Massachusetts) gehängt wurde, weil sie die Stadt wiederholt betreten hatte,
 obwohl die Quaker per Gesetz dort verbannt waren. Sie gilt als die erste religiöse
 Märtyrerin der Quaker und als die letzte religiöse Märtyrerin Nordamerikas.

 \end{description}

\normalsize
\section*{E}

\articlesize

\begin{description}

 \item[Eccles, Solomon] $\ast$1618 \dag1683, Quaker,  bekannt duch exzentrisches Verhalten. Etwar da durch, das er nackt auftrat, um damit zu zeigen, das er sich wieder im paradiesisch sündenfreien Urzustand befände (als Strafe dafür wurde er ausgepeitscht).

 \item[Ekklesiologie]

 \item[Evangelical Friends International (EFI)] beziehungsweise \textit{Evangelical Friends Church International (EFCI)}

 \end{description}

\normalsize

\section*{F}

\articlesize

\begin{description}


 \item[Fell, Margaret] $\ast$1614 in Marsh Grange, \dag23.4.1702 in Swarthmore. Gelegentlich auch als \textit{"`Mother of Quakerism"'} -- zu Deutsch: \textit{"`Mutter des Quäkertums"'} bezeichnet. 1632 heiratete sie Thomas Fell ($\ast$1598 \dag1658). Dieser war Richter und gehörte dem \textit{"`Long Parliament"'} unter $\to$Oliver Cromwell an. Im Juni 1652 traf sie zum ersten mal $\to$George Fox. Nach dem Tod ihres ersten Mannes heiratete sie am 27.10.1669 George Fox in Bristol. Margaret Fell war Autorin des $\to$\textit{"`histrischen Friedenszeugnisses"'} und maßgeblich mit der Organisation der Missionsreisen beschäftigt.


 \item[Fell, Margaret] $\ast$1614 in Marsh Grange, \dag23.4.1702 in Swarthmore. Gelegentlich auch als \textit{"`Mother of Quakerism"'} -- zu Deutsch: \textit{"`Mutter des Quäkertums"'} bezeichnet. 1632 heiratete sie Thomas Fell ($\ast$1598 \dag1658). Dieser war Richter und gehörte dem \textit{"`Long Parliament"'} unter $\to$Oliver Cromwell an. Im Juni 1652 traf sie zum ersten mal $\to$George Fox. Nach dem Tod ihres ersten Mannes heiratete sie am 27.10.1669 George Fox in Bristol. Margaret Fell war Autorin des $\to$\textit{"`histrischen Friedenszeugnisses"'} und maßgeblich mit der Organisation der Missionsreisen beschäftigt.


 \item[Fell, Margaret] $\ast$1614 in Marsh Grange, \dag23.4.1702 in Swarthmore. Gelegentlich auch als \textit{"`Mother of Quakerism"'} -- zu Deutsch: \textit{"`Mutter des Quäkertums"'} bezeichnet. 1632 heiratete sie Thomas Fell ($\ast$1598 \dag1658). Dieser war Richter und gehörte dem \textit{"`Long Parliament"'} unter $\to$Oliver Cromwell an. Im Juni 1652 traf sie zum ersten mal $\to$George Fox. Nach dem Tod ihres ersten Mannes heiratete sie am 27.10.1669 George Fox in Bristol. Margaret Fell war Autorin des $\to$\textit{"`histrischen Friedenszeugnisses"'} und maßgeblich mit der Organisation der Missionsreisen beschäftigt.

 \item[Finlayson, James]

 \item[Fisher, Mary] $\ast$1623 \dag1698. Quakerin. Wollte 1657 zusammen mit fünf anderen Quakern den Sultan von Konstantinopel bekehren.

 \item[Fox, Geroge] $\ast$Juli 1624 in Drayton-in-the-Clay, Leicestershire, heute Fenny Drayton, \dag13.1.1691. War einer der Gründerväter der Quaker und erfolgreicher Missionar.

 \item[Freund]

 \item[Free Quakers]
% <li><b><i></i></b> <KBD>( http://www.qhpress.org/quakerpages/qwhp/freequakertoc.htm )</KBD> Das wiederum ist ein insich abgeschlossenes Schismar. Entstanden während des Amerikanischen Bürgerkriegs. Nämlich eine Abspaltung von Quäkern, die im Bürgerkrieg nicht nur Verbal sondern auch mit der Waffe in der Hand Partei ergriffen. Sich somit vom "Friedenszeugnis" los sagten. Heute gibt es sie nicht mehr. Ausgestorben.</li>
 
 \item[Freunde der Freunde]

 \item[Friedensthal]

 \item[Friends General Conference]

 \item[Friends World Committee for Consultation]

 \item[Friends United Meeting (FUM)]

 \item[Frühe Freund]

 \item[Fry, Elizabeth] $\ast$21. 5.1780 in Cartham Hall bei Norwich, \dag12.10.1845 in Ramsgate. Quäkerin. War britische Reformerin des Gefängniswesens.

 \item[Fuchs Emil]


 \end{description}

\normalsize
\section*{G}

\articlesize

\begin{description}
 \item[Geschäftsversammlung]

 \item[Gesellschaft, Die] Kurzform von $\to$\textit{Religiöse Gesellschaft der Freunde}.

\item[Gewichtige Freunde] $\to$\textit{Weighty Friends}

\item[Glücksspiel]

\item[Gottesdienst] $\to$\textit{Andacht}

 \item[Graeff, Abraham Isacks op den]

 \item[Greenpeace]

\item[Gruppe] $\to$\textit{Monatsversammlung}

\item[Gurneys] Alte Quakerfamilie

\item[GYM] \textit{German Yearly Meeting}, \textit{Deutsche Jahresversammlung}. $\to$\textit{Jahresversammlung}
 \end{description}

\normalsize %bis hier...
\section*{H}

\articlesize

\begin{description}

 \item[Hanson, Elizabeth]

 \item[Heirat]

 \item[Helmont, Franciscus Mercurius van] $\ast$um~1614 in Vilvoorde bei Brüssel, Spanische Niederlande, heute Belgien, \dag1699 in Cölln bei Berlin. Quäker, Alchemist, Kabbalist, Theosoph, Arzt, Rosenkreuzer.

 \item[Hicks, Elias] $\ast$19.3.1748 \dag27.2.1830. Quäker, Wanderprediger, Abolitionisten. An seiner Person entzündete sich ein heftiger Streit, der dann in den  $\to$\textit{"`Hicksites-Orthodox-Schisma"'} der Quäker mündete. Daraus ging dann die $\to$Hicksites und die $\to$Orthodoxe Linie hervor.

 \item[Hodgkin, Thomas]

 \item[Hoover, Herbert C.]

 \item[Hopkins, Johns]

 \item[Hutchinson, Anne] $\ast$Juli 1591, \dag~August 1643. War keine Quakerin. Wurde als Anne Marbury in England geboren; heiratete mit 21 Jahren William Hutchinson und siedelte mit ihm 1634 in die britische Kolonie Massachusetts in Nordamerika um. Sie trat hervor als selbstsichere Frau, was für ihre Zeit ungewöhnlich war. Sie Vorkämpferin des Feminismus und eine Kämpferin für Menschenrechte und Toleranz. Sie lehrte, dass Gott direkt zu jedermann spräche und nicht allein zum Klerus. Sie Beeinflusste die spätere Quäkerin Mary Dyer, die sich beide persönlich kannten.


 \end{description}

\normalsize

\section*{I}

\articlesize

\begin{description}
 \item[Innere Bestrafung/Züchtigung]
 \item[Innere Gnade]
 \item[Innere Offenbarung] $\to$\textit{Inneres Licht}
 \item[Innere Kraft] $\to$\textit{Inneres Licht}
 \item[Innere Stimme] $\to$\textit{Inneres Licht}
 \item[Innerer Christus] $\to$\textit{Inneres Licht}
 \item[Innerer Judas]
 \item[Inneres Licht]
 \item[Inneres Ohr] $\to$\textit{Inneres Licht}
 \item[Inneres Zeugnis] $\to$\textit{Inneres Licht}
 \item[Innerlich kreuzigen]
 \end{description}

\normalsize
\section*{J}

\articlesize

\begin{description}
 \item[Jahresversammlung]

 \item[Jones, Rufus M.] $\ast$25.1.1863, \dag16.6.1948. US-amerikanischer Quäker, Autor, College-Professor, Philosoph und Mystiker.

 \item[Jones, Rice] \dag28.3.1693. Baptist, Quäker, Dissident. Er hielt in Nottingham religiöse Versammlungen in seinem Haus ab und seit 1657 auch im Schloß zu Nottingham. Lieferte sich Redeschlachten mit namhaften Quäkern seiner Zeit die ihn die ihm des Abfalls ($\to$Renterismus) bezichtigten.

 \item[Journal] Tagebuchähnliche Aufzeichnung. Die beiden Bekantesten, sind wohl die von $\to$\textit{George Fox} und $\to$\textit{John Woolman}.

 \item[Jungfreund]
 \end{description}

\normalsize
\section*{K}

\articlesize

\begin{description}

\item[Kappes, Heinz]

 \item[Keith, George] $\ast$1638 oder $\ast$1639, \dag27.3.1716. Schottischer Quaker-Missionar. Bekannt durch die \textit{"`Keith-Kontroverse"'}. Zusammen mit den damals angesehensten Quakern missionierte er in Deutschland und den Niederlanden. In der Zeit um 1691, als er in den Kolonien (den heutigen USA) lebte, begann eine Auseinandersetzung um seine Person. Er vertrat die Auffassung, die Gemeinschaft hätte sich zusehr von den Christlichen Wurzeln entfernt. Zuerst kam es zum Bruch mit dem Philadelphia Yearly Meeting und später mit dem London Yearly Meeting. Jahre später konvertierte er zur Anglikanische Kirche und wurde dort Priester.

 \item[Keith-Kontroverse] $\to$George Keith.

 \item[Kraus Hertha]

 \item[Kryptoquäker]

 \end{description}
\normalsize

\section*{L}

\articlesize

\begin{description}

 \item[Lay, Benjamin] $\ast$26.1.1682 in Colchester, England, \dag3.2.1759 in Abington, Pennsylvania. Englischer Quaker, Philanthrop und Schriftsteller. Trat er als Abolitionist gegen die Sklaverei ein und war Bekennt für Provokative Inszenierungen und Auftritte.

 \item[Lay down]
    the action properly taken upon a committee, meeting or ministry that is no longer needed; "`to lay down"' a meeting is to disband it.

 \item[Leading]
    a course of action, belief or conviction that a Friend feels is divinely inspired.

 \item[Levellers]


 \item[Liberales Quakertum]
% Ganz wie der Begriff \textit{Chrisozentrische Quäker} eher eine Wage Bezeichnung für eine Grundhaltung oder Tendenz. Die meisten Europäschen Jahresversammlungen könnte man dort einordnen. Aber mit Vorbehalt! wie wir bei den <i>independent Quaker</i> schon gesehen haben, gibt es auch "Ausreißer". Mit <i>Liberales Quäkertum</i> ist meist gemeint das man so ziemlich jede Vorstellung haben kann, die man will. Was in der Regel nicht unter Liberales Quäkertum verstanden wird, ist aggressives Missionieren, Gottesdienste mit Gesang, Liturgie und Prediger/Pastor.</li>

 \item[Lightfoot, Hannah]

 \item[Lilburne, John] $\ast$ca.~1614, \dag28.8.1657. Auch bekannt als \textit{"`Freeborn John"'}. Er war einer der bekanntesten Wortführer der radikaldemokratischen $\to$Levellers in England des siebzehnten Jahrhunderts. Er ist der Bruder von Robert Lilburne.

 \item[Luffe, John 'Love'] $\ast$? \dag~?. Quaker. Bekanntgeworden durch den Versuch 1658 Papst Alexander VII bekehren zu wollen. Er kam dabei in Rom ums leben.\endnote{Biographisch-Bibliographischen Kirchenlexikon, Band XX (2002)Spalten 1160-1167 Claus Bernet über Perrot, John}

 \end{description}
\normalsize

\section*{M}

\articlesize

\begin{description}

 \item[Meeting for Sufferings]
 
 \item[meeting for worship]  $\to$\textit{Andacht}

 \item[Member by Convincement] $\to$\textit{Quaker by Convincement}
 
 \item[Ministry]
 
 \item[Monatsversammlung]
 
 \item[Musik]
 
 \end{description}

\normalsize

\section*{N}

\articlesize

\begin{description}

\item[Notion] (Idee, Begriff, Vorstellung, Ahnung, Ansicht, Meinung, Konzept)
    An unfounded, unspiritual position. (Used by George Fox, often to refer to teachings or doctrines that were expressed but not fully understood or experienced.)

 \item[Nayler, James] $\ast$1618, \dag1660. Englischer Quäker. Ist der Nachwelt in erster Linie durch den Konflikt mit $\to$Geroge Fox in Erinnereung geblieben.

 \item[New Foundation Fellowship]
% <li><b><i>New Foundation Fellowship</i></b> <KBD>( http://www.nffellowship.org/index.html )</KBD> Das ist kein Schismar oder Flügel, als mehr eine "Erneuerungsbewegung". Einige Freunde, vorallem wohl in den Liberalen Flügeln, versuchen die Erinnerung an die Frühen Freunde wach zu halten und sich auf alte Traditionen zu besinnen. New Foundation Fellowship ist auch in der Deutschen Jahresversammlung vertreten. Ich glaube Annette Fricke\endnote{Nontheist Friend. (2009, May 18). In Wikipedia, The Free Encyclopedia. Retrieved 19:35, May 18, 2009, from http://en.wikipedia.org/w/index.php?title=Nontheist_Friend&oldid=290733148} kann man dazu zahlen. Sie bemühte sich seinerzeit vergeblich das Tagebuch von G. Fox neu zu übersetzen und über das GYM verlegen zu lassen. Es gibt zeit geraumerzeit im "Quäker" eine Reihe wo Rex Amblers Geogre-Fox-Verkexelungen übersetzt werden. Ob sie daran beteiligt ist, weiß ich nicht. Bei aller Begeisterung für G. Fox, aber mit dem Rex Amblers Zeugs kann ich überhaupt nichts anfangen.</li>

 \item[Nontheist Friends]
%  \endnote{\texttt{http://www.nontheistfriends.org/}} \endnote{Nontheist Friend. (2009, May 18). In Wikipedia, The Free Encyclopedia. Retrieved 19:35, May 18, 2009, from \texttt{http://en.wikipedia.org/w/index.php?title=Nontheist_Friend\&oldid=290733148}}
%  Ob man zu \textit{Nontheist Friends} auch Athistisch Quäker sagen kann ist für mich offen geblieben.
%  \endnote{Im "`Quäker"' 2/2008 Seite 88 veröffentlichte H. Konopatzky das Gedicht
%  "`Bekenntnis"', worin es Heißt:
%  \textit{"`Ist kein Jude oder Christ, weder Moslem noch Buddhist,
%  meint, er sei ein Atheist. - Sollt ich selbst mich bekennen, müsst ich mich ungläubig
%  nennen im Sinne aller Religionen, in deren Templn Herrscher thronen."'}}

 \end{description}

\normalsize

\section*{O}

\articlesize

\begin{description}
 \item[Overseers]

 \item[Outreach]

 \item[Oxfam]
 \end{description}

\normalsize

\section*{P}

\articlesize

\begin{description}

 \item[Paquet, Alfons]

 \item[Pastorius, Franz Daniel]

 \item[Penn, Willam] $\ast$14.10.1644 in London, \dag30.7.1718 in Ruscombe, Berkshire. Quaker. Gründete die Kolonie Pennsylvania im Gebiet der heutigen USA, verfasste 1693 das Essay \textit{"`towards the Present and Future Peace of Europe"'}.

 \item[Perrot, John] $\ast$?, \dag1665 in Jamaika. Quaker. Wollte 1658 mit $\to$John 'Love' Luffe zusammen Papst Alexander VII bekehren.

 \item[Projekt Alternativen zur Gewalt (PAG)]

 \end{description}
\normalsize



\section*{Q}

\articlesize

\begin{description}

 \item[Quaker (Name)]

 \item[Quäkerbüro]

 \item[Quaker by Convincement] $\to$\textit{Convinced Friend}.

 \item[Quaker Gray]

 \item[Quäkerhaus]

 \item[Quaker Meeting House] $\to$\textit{Meeting House}

 \item[Quaker Peace \& Social Witness (QPSW)]

 \item[Quäkerspeisung]

 \item[Quaker United Nations Office (QUNO)]

 \item[Quaker Universalist Fellowship] Das ist kein selbständiger Zweig des Quäkertums, sondern eine Bewegung innerhalb des Quäkertums, ähnlich wie $\to$\textit{New Foundation Fellowship} nur quasie in die genaue Gegenrichtung. Die Anhänger gehen so weit, das sie die Auffassung vertreten, das man auch als Hindu, Buddhist oder Moslem Quäker sein kann. Das Quäkertum eben etwas universelles ist.\endnote{\texttt{http://www.universalistfriends.org/}}

 \item[Quakerzeugnis] $\to$\textit{Zeugnis}

 \item[Quarterly Meeting] $\to$\textit{Vierteljahres Versammlung}

 \end{description}

\normalsize

\section*{R}

\articlesize

\begin{description}

 \item[Recorded Minister]
    A person whose vocal ministry (spoken contribution in meeting)—or another spiritual gift—is recognised as helpful and probably faithful to Divine leading, by the body of Friends to which they belong and formally recorded by that body. Not all Friends' organisations record ministers. Other Friends have adopted a defined process prerequisite for "`recording"'.
 

 \item[Religiöse Gesellschaft der Freunde]

  \item[Religiöser Sozialismus]
%  \endnode{\texttt{http://de.wikipedia.org/wiki/Religi\%C3\%B6ser\_Sozialismus}} Das ist eine Strömung die nicht nur auf das Quäkertum beschränkt war. Bekanntester Vertreter in Deutschland war Emil Fuchs\endnote{Wikipedia zu Emil Fuchs: http://de.wikipedia.org/wiki/Emil_Fuchs}. Seit Ende des zweiten Weltkriegs spielt die Strömung innerhalb der Deutschen Jahresversammlung keine Geige mehr.

 \item[Renterismus] Ein Schlagwort den Quakern benutzten, um andere zu dikreditieren, die in ihren Augen \textit{"`der Welt verfallen waren"'}. So zum Beispiel der Gruppe um   $\to$Rice Jones dessen Anhänger zu den landesbesten Football-Spielern und Ringern gehörten.\endnote{Biographisch-Bibliographischen Kirchenlexikons, Band XXIV (2005) Spalten 917-918 Autor: Claus Bernet, Zu Rice Jones}

 \item[Rest Home Projekt]

 \item[Richmond Declaration]

 \item[Right ordering]
    has to do with proper conduct of a meeting for business. The term is often used in the negative, that is, if someone senses that something about the conduct of the meeting is not proper, they may object that "`this meeting is not in right ordering."'

 \end{description}

\normalsize

\section*{S}

\articlesize

\begin{description}
 \item[Schissma]

 \item[Schreiber] $\to$\textit{Clerk}

 \item[Seebohm, Ludwig] 

 \item[Seekers]

 \item[Sense of the Meeting]

 \item[Service Civil International (SCI)]

 \item[Shaker]

 \item["`Speaks to my condition"' or "`Friend speaks my mind"'] (Frei übersetzt: "`Er spricht mir aus der Seele"' oder "`er spricht in meinem Sinne"')
    Commonly used during meetings for business to express that another Friend has spoken what is in the mind of the speaker; used to help add weight to the ministries of others.

 \item[Suffering]

 \end{description}

\normalsize

\section*{T}

\articlesize

\begin{description}

 \item[That of God in everyone]
    the belief in the presence of God within all people. Also referred to as the Inner Light.
    
 \item[Thoreau, Henry David] $\ast$12.6.1817 in Concord, Massachusetts, \dag6.5.1862 ebenda. Pazifist, Unitarist, Abolitionist. Entstammte einer Familie mit Quakerwurzeln. War selber aber nicht Quaker. Thoreau war Vordenker des Zivilen Ungehorsams. Sehe bekannt ist das Zitat: \textit{"`In einem Staat, der seine Bürger willkürlich einsperrt, ist es eine Ehre für einen Mann, im Gefängnis zu sitzen."'}


 \item[travelling minute]

 \item[True Levellers] $\to$Diggers.

 \end{description}

\normalsize

\section*{U}

\articlesize

\begin{description}
 \item[Universalismus] $\to$\textit{Quaker Universalist Fellowship}
 \end{description}
\normalsize

\section*{V}

\articlesize

\begin{description}

 \item[Valiant Sixty] Eine Gruppe von Frühen Freunden, die eine Führungsrolle einnahmen und sich aktiv um die Verbreitung ihrer Botschaft bemühten. Zu den bekanntesden Mitgliedern zählen:
 
 \begin{itemize}
  \item Edward Burrough
  \item Margaret Fell
  \item Mary Fisher
  \item George Fox
  \item Elizabeth Hooton
  \item Francis Howgill
  \item James Nayler
  \item George Whitehead
 \end{itemize}

 \item[Versammlung] Je nach Kontext ist entweder der Gottesdienst gemeind, oder die Gemeinde. Das Wort "`Gemeinde"' wird meist nicht verwendet um eine Quakergemeinschaft zu bezeichnen. Das rührt aus dem Kirchenverständniss der Quaker ($\to$\~Ekklesiologie), die Kirche ausschlisslich als "`Gemeinschaft der Gläubigen"' versteht, und nicht als Gebäude oder Institution. Versammlung bezeichnet also lediglich eine nicht näher bezeichnete "`Organisationseinheit"'.
 
 \end{description}

\normalsize

\section*{W}

\articlesize

\begin{description}
 \item[Weighty Friends]
    a Friend, respected for their experience and ability over their history of participation with Friends, whose opinion or ministry is especially valued.

 
 \item[Wehrdienst]

 \item[Whittier, John Greenleaf]

 \item[Winstanley, Gerrard] $\ast$1609 in Wigan, Lancashire, \dag10.9.1676) war ein protestantischer Reformer und politischer Aktivist in England. Er führte zur Zeit des Protektorats (1649–1658) von $\to$Oliver Cromwell die Gruppe der \textit{"`True Levellers"'} (englisch: wahre Gleichmacher) an, die von ihren Gegnern abfällig \textit{"`Diggers"'} (englisch: Buddler) genannt wurden.

\item[Woolman, John] $\ast$19.10.1720 in Northampton, New Jersey; \dag7.10.1772 in York. Amerikanischer Quäker-Prediger und vehementer Gegner der Sklaverei. Bekannt durch seine \textit{"`Aufzeichnungen"'} (englisch: \textit{Journal}).\endnote{Die Aufzeichnungen von John Woolman. Aus der Zeit der Sklavenbefreiung. 2. Auflage, Lorber U. Turm Verlag, 1964, ISBN 3799901019}

 \end{description}


\normalsize
\section*{X}

\articlesize

\begin{description}
 \item[X]
 \end{description}
\normalsize
\section*{Y}

\articlesize

\begin{description}
 \item[X]
 \end{description}
\normalsize
\section*{Z}

\articlesize

\begin{description}
 \item[Zeugnis]
 \end{description}


\normalsize

\onecolumn

\theendnotes


\null\newpage