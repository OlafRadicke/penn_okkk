
\chapter{Zusammenfassung der Kapitel}
%\footnotesize
\begin{description}
\item[1.und 2. Kapitel] Von der Notwendigkeit, das Kreuz Christi täglich zu tragen.
\item[3. Kapitel] Erklärung des Kreuzes Christi; worin es besteht. etc.
\item[4. Kapitel] Von den großen Wirkungen des Kreuzes.
\item[5.und 6. Kapitel] Von der unerlaubten Selbstheit in der Religion und Modalität.
\item[7. bis 12. Kapitel] Vom Stolze, als der ersten Hauptleidenschaft des Menschen; dessen Ursprung, nähere Bestimmung und Unterscheidung.
\item[13. Kapitel] Von der Gier, als der zweiten Hauptleidenschaft; nähere Bestimmung und Unterscheidung desselben.
\item[14. bis 18. Kapitel] Von der Verschwendung\footnote{In Ursprungstext stet das Wort "`Üppigkeit"'. Das Wort wird in der heutigen Verwendung anders gebraucht und wurde desshalb - der verständlichkeit halber - durch "`Verschwendung"' ersetzt}; worin sie besteht, und was für Unheil sie unter den Menschen anrichtet.
\end{description}
%\normalsize

\chapter{1. Kapitel}

\section{Zusammenfassung des 1. Kapitel}
\small
\begin{description}
\item[1. Abschnitt] Von der Nothwendigkeit des Kreuzen Christi überhaupt, und wie wenig dennoch die Christen sich darum kümmern.
\item[2. Abschnitt] Ausartung des Christentums von Reinheit in Lüste und Begierden, und von Mäßigkeit in Übermaß.
\item[3. Abschnitt] Weltliche Lüste und Vergnügungen sind so sehr das Ziel und Streben der Bekenner des Christentums geworden, dass sie die Gottlosigkeit der Ungläubigen darin übertreffen.
\item[4. Abschnitt]  Diese Ausartung bildet den zweiten Akt des Trauerspiels, welchen die Juden angefangen haben, und dieser ist schlimmer\footnote{Im Urtextr wird sas Wort "`ärger"' benutzt, statt "`schlimmer"'} als, der erste. -- Bemerkungen über die Verachtung, welche die Christen auf ihren Heiland gebracht haben.
\item[5. Abschnitt] Die Sünde ist in der ganzen Welt von der gleichen\footnote{im Org. "`einerlei"'} Natur und Beschaffenheit. -- Alle Gottlosen gehören zu einer und derselben Gemeine; sind alle Kinder des Argen. -- Bösewichter, welche Religion zu haben vorgeben, sind darum nur desto schlimmer.
\item[6. Abschnitt] Ein Wolf ist kein Lamm: ein Sünder kann, so lange es in der Sünden bleibt, kein Heiliger sein
\item[7. Abschnitt] Die Gottlosen verfolgen immer\footnote{"`immer"' statt "`allezeit"'} die Frommen; auch haben immer die falschen Christen die wahren verfolgt, weil diese ihrem Aberglauben nicht beipflichten wollten. -- Von den sonderbaren und fleischlichen Begriffen, welche die falschen Christen vom Christentum haben; und von der Gefahr eines solchen Selbstbetrugs.
\item[8. Abschnitt] Diese Betrachtungen, und meine Empfindungen darüber, haben es wir zur Pflicht gemacht, die gegenwärtige Abhandlung, als eine Warnung gegen die Lüste der Welt, und als eine Einladung zum täglichen Aufnehmen des Kreuzes Christi, zu schreiben, und zu zeigen, daß dieser das von Christo uns verordnete Mittel zu unserer Seligkeit ist.
\item[9. Abschnitt] Über die Selbstverdammung der Gottlosen. -- Wahre Religion und Gottesverehrung bestehen darin, daß man den Willen Gottes tut. -- Von dem Vorzug, den die Gerechten vor den Gottlosen im jüngsten Gerichte haben.
\item[10. Abschnitt] Gebet für die Christenheit, daß sie an jenem großen Gerichtstag der Welt nicht möge verworfen werden. -- Sie wird ermahnt, zu erwägen, worin sie Christo ähnlich ist; und, wenn er ihr Heiland und Erlöser ist, wie, und wovon er sie erlöst habe, und was ihre eigene Erfahrung von diesem großen Werk sei? -- Christus kam in die Welt, die Menschen von ihren Sünden, und also auch vom ewigen Zorn zu befreien; aber nicht, um sie in ihren Sünden selig zu machen. -- Indem er sie von der Sünde erlöst, errettet er sie auch vom ewigen Tode, welcher der Sold oder Lohn der Sünde ist.
\end{description}
\normalsize


\section{1. Abschnitt}

Obgleich die Kenntnis und Ausübung der Lehre vom Kreuze Christi, als dem einzigen Eingang zum wahren Christentum, und dem Pfade, den allezeit die Alten zu ihrer Seligkeit betraten, für die Seelen der Menschen von der höchsten Wichtigkeit ist; so wird dennoch diese Lehre, -- ich sage es mit tiefer Betrübnis! -- so wenig verstanden, so sehr vernachlässigt, und, -- was noch schlimmer ist, -- es wird ihr durch die Eitelkeit, den Aberglauben, und die Unmäßigkeit der Christenthumsbekenner so bitter widersprochen, daß wir entweder aufhören müssen, zu glauben, was der Herr Jesus Lukas 24, 27.\index{Bibelstellen:!Lukas 24} uns sagt, wo er nämlich erklärt, "`\textit{daß Niemand, der nicht sein Kreuz tragt und ihm nachfolgt, sein Jünger sein könne,}"' oder, -- wenn wir dieses als Wahrheit annehmen, -- nicht anders schließen können, als daß die Mehrheit der Bekenner des christlichen Namens, in der großen Angelegenheit der Religion und ihres eigenen Heils, auf eine bejammernswerthe Art sich täuschen und selbst betrügen.

\section{2. Abschnitt}

Wir mögen den Zustand der Völker, die auf die Wohltat des heiligen Namens Jesus Anspruch machen, noch so nachsichtsvoll und liebreich beurtheilen, so Müssen wir doch auch, wenn wir zugleich gerecht handeln wollen, notgedrungen gestehen, daß ungeachtet der gnädigen Vortheile des Lichts und der Erkenntniß, und der Aufmunterungen zur Treue, welche in diesen letztem Jahrhunderten durch die Erscheinung, das Leben, die Lehren und Wunder, durch den Tod, die Auferstehung und Himmelfahrt Christi, nebst den Gaben seines heiligen Geistes den Menschen verliehen worden sind; ungeachtet der Schriften, Arbeiten, Leiden und Erduldungen des Martertodes seiner theuern Zeugen in allen Zeiten, nicht viel mehr als der bloße Name vom wahren Christenthume übrig geblieben zu seyn scheint. Und wo nun die alte heidnische Natur der Menschen sich dieses Namens anmaßet, oder ihr zügelloses Leben damit zu bedecken sucht, da sind die Bekenner desselben in der Tat nichts anders, als wirkliche, wiewohl verkleidete Heiden. Denn wenn sie auch nicht dieselben Götzen der Heiden anbeten, so beten sie doch Christum mit einem heidnischen Herzen an; und sie können auch nicht anders, so lange sie in gleichen heidnischen Lüsten leben. So gehören also beide: der Christ, der sich nicht selbst überwindet, und der zügellose Heide zu einer und derselben Religion\index{Heiden}. Beide haben freilich verschiedene Gegenstände, an welche sie ihre Gebete richten, allein ihre Anbetung ist doch nur erzwungen, und bloße Zeremonie; denn die Gottheit, die sie im wahren Sinne verehren, ist der Gott dieser Welt, der große Beherrscher der weltlichen Lüste und Begierden. Vor ihm beugen sie sich mit allen Kräften der Seele und der Sinne. Was sollen wir essen? Was sollen wir trinken? Was sollen wir anziehen, und wie sollen wir unsere Zeit hinbringen? Auf welche Art können wir uns Reichthum erwerben? Wodurch können wir unsere Macht vergrößern, unsere Besitzungen ausdehnen, unsere Namen und Familien in der Welt berühmt machen und verewigen? -- Diese niedrige Sinnlichkeit faßt der geliebte Apostel Johannes sehr kurz und nachdrucksvoll in einigen Worten zusammen: "`\textit{Fleischeslust, Augenlust und hofärtiges Leben,}"' sagt er, "`\textit{sind nicht vom Vater, sondern von der Welt,}"' \footnote{Johannes 2,16} \index{Bibelstellen:!Johannes 2} die im Argen liegt.

\section{3. Abschnitt}

Es ist eine traurige Feststellung, aber eine durchaus nicht zu leugnende Wahrheit, daß diese weltlichen Lüste die Gegenstände des Nachsinnens, der Sorge und der Unterhaltung des größten Teils der unglücklichen Christenheit ausmachen, und -- was das Elend noch vergrößert -- mit der Zeit zugenommen haben. Denn, so wie die Welt älter geworden ist, hat sie sich auch verschlimmert. Die Beispiele früherer ausschweifenden Zeitalter, und die daraus zu ziehenden beklagenswerthen Folgerungen, haben das unsrige Zeitalter nicht abgeschreckt, sondern vielmehr noch gereizt; so, daß die, Menschen unserer Zeit den alten Vorrath von Gottlosigkeit noch mehr angehäuft haben. Ja, sie haben die ihnen gegebenen bösen Beispiele so sehr übertroffen, daß sie, statt in bessern Zeiten Fortschritte in der Tugend zu machen, auf eine abscheuliche Art tief unter die Heiden herabgesunken sind. -- \marginpar{Christen verhalten so, das sie auf Heiden abstossen wirken müssen.} Sie haben ihren Hochmuth, ihre wollüstige Ausgelassenheit, Unreinheit und Trunkenheit, ihr Fluchen, Schwören und Lügen, ihr Neiden und Verleumden, ihre Grausamkeit, Falschheit, Habsucht, Ungerechtigkeit und Unterdrückung, so allgemein verbreitet, und mit einem erfinderischen Geiste so hoch getrieben, daß sie darin den Ungläubigen zum Anstoße und Ärgernisse gedient, und ihnen die stärkste Veranlassung gegeben haben, die heilige Religion mit Verachtung zu betrachten, für welche sie durch gute Beispiele der Christen hätten gewonnen werden können.

\section{4. Abschnitt}

Diesen traurigen Abfall von der ursprünglichen Reinheit der ersten Zeiten des Christenthums, als der Ruhm desselben, in dem reinen Lebenswandel seiner Bekenner bestand, kann ich nicht andere als den zweiten und furchtbarsten Teil des Trauerspiels betrachten, welches die Juden \index{Personen:!Juden} mit dem glorreichen Heilande des Menschengeschlechts begannen. Diese, die durch die Macht der Unwissenheit, und der großen Vorurtheile, die sie gegen seine in den Augen der Welt unansehnliche Erscheinung hatten, so verblendet waren, daß sie ihn, als er erschien, nicht annehmen wollten, verfolgten ihn jedoch nur zwei oder drei Jahre, bis sie ihn zuletzt an einem Tage kreuzigten. \marginpar{Der Verrat der Christen Heute an Jesus ist viel schlimmer, als der der Juden damals.} Allein die Grausamkeit der falschen Christen ist von weit längerer Dauer. Nachdem sie, wie Judas \index{Personen:!Judas}, zuerst ihn anerkannt, und dann viele Jahrhunderte hindurch auf das schändlichste verrathen haben; hören sie nicht auf, ihn zu verfolgen und zu kreuzigen, indem sie von seiner Lehre, welche Selbstüberwindung und Heiligkeit vorschreibt, in ihren Sitten fortwährend abweichen, und durch ihren Lebenswandel ihrem Glaubensbekenntnisse beständig widersprechen. Von Solchen sagt uns der Verfasser der Epistel an die Hebraer, "`daß sie ihnen selbst den Sohn Gottes von neuem wieder kreuzigen, und öffentlich zum Gespötte machen."''\footnote{Hebräer 6,6} Johannes nennt ihre verunreinigten Herzen in seiner Offenbarung: "`die Gassen des geistlich so genannten Sodoms und Egyptens, wo unser Herr gekreuzigt ist."'\footnote{Offenbarung 11,8}. Und so wie Christus ehemals sagte: "`des Menschen Feinde werden seine eigenen Hausgenossen seyn,"' \footnote{Matthäus 10,36} so befinden sich jetzt die Feinde Christi vornehmlich unter seinen eigenen Bekennern, unter welchen es nicht Wenige giebt, die ihn anspeien, ans Kreuz nageln und durchboren, und ihm Essig mit Galle vermischt zu trinken geben.\footnote{Matthäus 27} Dieses ist auch nicht schwer einzusehen; da diejenigen Menschen , die nach ihrer verderbten Natur, und unter demselben bösen Einflusse leben, worunter die gottlosen Juden standen, welche Christum äußerlich kreuzigten, ihn gewiß innerlich kreuzigen, und Alle, welche jetzt die Erscheinung und Zucht seiner Gnade in ihren eigenen Herzen verwerfen, gleiches Stamme und Geschlechtes mit jenen verhärteten Juden sind, die damals derselben Gnade widerstanden den, als sie in Christo erschien und durch ihn geoffenbaret ward.

\section{5. Abschnitt}

\marginpar{In der Sünde sind alle Gleich. Ob Christ, Jude oder Heide.} Die Sünde ist, von einem Ende der Welt bis zum andern, von einerlei Natur und Beschaffenheit. Denn wenn auch ein Lügner kein Trunkenbold, oder ein Flucher kein Hurer, und keiner von ihnen eigentlich ein Mörder ist; so gehören sie doch alle zu \textbf{einer Gemeinschaft}, sind Alle Zweige aus einer und derselben bösen Wurzel, Alle eines Geschlechts. Die Gottlosen haben nur einen gemeinschaftlichen Vater; wie Christus den Bekennern des Judenthums, die in jenem Zeitalter die sichtbare Kirche ausmachten, frei erklärte, indem er Ihre Ansprüche auf Mose und Abraham \index{Personen:!Mose} \index{Personen:!Abraham} verwarf, und ihnen gerade heraus sagte, "`\textit{wer Sünde thue, sey der Sünde Knecht; sie thäten die Werke des Teufels, und wären folglich des Teufels Kinder.}"'\footnote{Johannes 8, 34-45.} \index{Bibelstellen:!Johannes 8}.
\marginpar{Nicht zu wem man sich bekennt macht einem zum \textit{Nachfolger}, sondern das man wirklich nachfolgt.} Diese Behauptung wird immer wahr bleiben, so lange dieselben Gründe dafür vorhanden sind. "`\textit{Wem ihr euch zum Gehorsam ergebet,}"' sagt Paulus, "`\textit{dessen Knechte seid ihr.}"'\footnote{Römer 6, 16.} \index{Bibelstellen:!Römer 6} Und Johannes sagt in seiner allgemeinen Epistel an die ersten Gemeinen: "`\textit{Lasset euch Niemand betrügen; wer Sünde tut, der ist vom Teufel.}"'\footnote{Johannes 3, 7. 8.} \index{Bibelstellen:!Johannes 3} -- War Judas index{Personen:!Judas} darum ein besserer Christ, "`\textit{daß er gegrüßet seist du, Meister!}"' ausrief, und Christum küßte?
Keinesweges.
Es war vielmehr das Zeichen seines Verraths;
die Losung, wodurch die blutdürstigen Juden Christum erkennen sollten, damit sie ihn greifen könnten. Judas nannte Christus \textit{Meister}, und verrieth ihn;
er küßte ihn, und verkaufte ihn zum Tode.
So verhält es sich mit der Religion der falschen \textbf{Namenchristen} \index{Personen:!Namenchristen} noch jetzt.
\marginpar{Es wird behauptet es sei unmöglich so zu leben, wie Christus es verlang hat.}
Fragt man sie, ob Christus ihr Herr sei,
so sind sie bereit auszurufen:
"`\textit{Behüte uns Gott, daß es anders wäre! Freilich ist er unser Herr!}"'
-- "`\textit{Wohlan denn! Haltet ihr aber auch seine Gebote?}"'
-- "`\textit{O Nein! Wie könnten wir das?}"'
-- "`\textit{Wie dürft ihr euch denn seine Jünger nennen?}"'
-- "`\textit{Es ist unmöglich}!"'
...antworten sie.
"`\textit{Wie kann man verlangen, daß wir seine Gebote halten sollen? Das kann ja kein Mensch.}"'
-- Wie? Es wäre unmöglich, das zu thun, ohne dessen Ausübung Christus es für unmöglich erklärt, ein Christ zu seyn? Ist Christus denn unbillig? Wird er "`\textit{da ernten wollen, wo er nicht gesäet hat?}"' \footnote{Matthäus 25, 24.} \index{Bibelstellen:!Matthäus 25} oder etwas den uns verlangen, wozu er uns keine Fähigkeit gab?
-- So geht es zu, daß die falschen Christen mit Judas zusammen, Christus ihren Herrn und Meister nennen, zu gleicher Zeit aber mit dem bösen Haufen der Welt sich verbinden, um ihn zu verrathen;
daß sie ihn umarmen und küssen, soweit ein scheinbares Namenbekenntniß \index{Namenbekenntniß} reicht, ihn aber treulos verkaufen, sobald es darauf ankommt, ihre herrschende Leidenschaft, der sie am meisten nachhängen, zu befriedigen.

\section{6. Abschnitt}

Möchte doch Keiner seine eigene Seele betrügen!
"`\textit{Man kann nicht Trauben von Dornen, oder Feigen von Disteln sammeln.}"' \footnote{Matthäus 7, 16.} \index{Bibelstellen:!Matthäus 7}
Ein Wolf ist kein Lamm, und ein Geier keine Taube.
Zu welcher äußern Religionsform, zu welcher religiösen Gesellschaft, oder zu welcher Kirche du dich auch bekennest, so ist es eine an dich und alle Menschen gerichtete Wahrheit Gottes, daß Diejenigen, welche die Form und den Schein der Gottseligkeit haben, aber durch ihr ungöttliches Leben die Kraft derselben verleugnen, nicht die wahre, sondern die falsche Kirche ausmachen, die, obgleich sie sich den Titel der Braut des Lammes, oder der Kirche Christi beilegt, dennoch jenes große Geheimniß, oder "`die geheimnißvolle \textit{Babylon}"' \index{Orte:!Babylon} \index{Personen:!Hure Babylon} ist, welche der heilige Geist so passend "`die Mutter der Hurerei und aller Greuel auf Erden"' nennt;\footnote{Offenbarung 17, 5.} \index{Bibelstellen:!Offenbarung 17} weil sie von der christlichen Keuschheit und Reinheit ausgeartet ist, in alle Greuel der heidnischen Babylon, einer prachtvollen Stadt, der Vorzeit, die als Sitz der babylonischen Könige und der größten Hoffahrt und Ueppigkeit in der damaligen Welt berühmt war. Was nun diese mystische Babylon damals war, das ist sie auch noch jetzt: die größte Feindin der Sache und des Volks Gottes.

\section{7. Abschnitt}

\marginpar{Die Weltabgewanten werden immer von den weltlichen Menschen verfolgt.} Es bleibt auch wahr, daß die, welche vom Fleische geboren sind, diejenigen, welche aus dem Geiste geboren sind und die \textit{Beschneidung des Herzens} \index{Beschneidung des Herzens} erfahren haben,\footnote{Galater 4, 29.} \index{Bibelstellen:!Galater 4} hassen und verfolgen; weil sie nach Babylons Erfindungen, Lehrarten und Vorschriften, Gott nicht verehren und anbeten, und weder ihre nichtigen Traditionen als Lehren annehmen, noch im Leben und Wandel nach ihren verderbten Moden und Gebrauchen sich bequemen können.
Wo dieses nun der Fall ist, da verwandelt die Abtrünnige sich in eine Verfolgerin.
-- Denn es ist nicht genug, daß sie selbst von der ersten Reinheit des Christenthums abgewichen ist;
Nein!
Andere sollen es ihr auch nachthun.
Darum läßt sie auch denen, die an ihrer Ausartung keinen Antheil haben oder ihr Maalzeichen \footnote{Anmerkung: Vermutlich die Anspielung auf Offenbarung 13,16} \index{Bibelstellen:!Offenbarung 13} nicht annehmen wollen, keine Ruhe.
-- Wer ist auch wohl weiser als sie?
Die Mutterkirche?
Und wer kann mit dem Thiere, auf dem sie reitet, streiten?

\medskip 

Die Abtrünnigen und Abergläubigen sind immer stolz auf ihren Irrthum , und unduldsam gegen Andere, die nicht ihrer Meinung sind.
Alle sollen ihnen beistimmen oder umkommen.
Daher werden "`\textit{die erschlagenen Zeugen, und das Blut der Seelen unter dem Altare}"'\footnote{Offenbarung 6, 9.} \index{Bibelstellen:!Offenbarung 6} innerhalb der Mauern dieser geheimnißvollen Babylon, dieser großen festen Stadt der falschen Christen, gefunden, und von dem heiligen Geiste in der Offenbarung ihr zur Last gelegt.
Es ist freilich nicht zu bewundern, daß sie, die zuerst den Herrn kreuzigte, hernach auch seine Knechte tötete;
aber höchst sonderbar und zugleich grausam ist es, daß sie ihren Bräutigam tödten, ihren Heiland ermorden kann;
da sie doch diese beiden Benennungen, die ihr so viel eingebracht haben, so sehr zu lieben scheint, und auch durch dieselben,
-- wiewohl ohne allen gerechten Anspruch, --
sich immer noch zu empfehlen sucht.
Indessen sind ihre Kinder, durch ihren fortwährenden Ungehorsam gegen die Offenbarung des göttlichen Lichts in ihren Seelen, so gänzlich unter die Herrschaft der Finsterniß gerathen, daß sie vergessen haben, was der Mensch einst war, oder was sie jetzt sein sollten, und wahres reines Christenthum, wenn sie es antreffen, nicht einmal kennen;
wiewohl sie sich viel darauf einbilden, zu der Zahl der Bekenner desselben zu gehören.
Ihre Begriffe vom Heil der Seele sind so fleischlich und falsch, daß sie Gutes bös, und Böses gut nennen.
Sie halten Menschen wie Teufel für Christen, und Heilige für Teufel.
-- Obgleich nun die Erwägung der gottlosen Ungebundenheit ihres Lebens, da dieselbe ihr Verderben nach sich ziehet, schon das tiefste Bedauern erregen muß;
so ist doch von allen Selbsttäuschungen, unter denen sie sich befinden, hinsichtlich ihres ewigen Zustandes, die verderblichste diese, daß sie indem allgemeinen Wahne stehen, sie könnten Kinder Gottes sein, während sie im Ungehorsame gegen seine heiligen Gebote leben;
sie dürften sich für Jünger Jesu halten, obgleich sie sich weigern, sein Kreuz zu tragen; und sie könnten sich auch als Glieder seiner wahren Kirche betrachten, welche heilig und ohne Tadel sein soll, ungeachtet sie ein unheiliges und tadelhaftes Leben führen.
So sind sie mitten in ihren Sünden im Frieden, und halten sich in ihren Uebertretungen für sicher.
Ihre eitle Hoffnung betäubt ihre bessere Ueberzeugung, und erstickt jede zarte Anmahnung zur Reue;
so daß also ihr Irrthum in Ansehung ihrer Pflichten gegen Gott, eben so gefährlich als ihre Empörung gegen ihn ist.
\marginpar{Die verlogenen Christen betrügen sich selbst. Aber am Tag des Gerichts, wird Gott sie aus ihren Träumen reissen.}
-- So wandeln sie an Abgründen, und täuschen sich selbst mit schmeichelhaften Vorstellungen, bis das Grab sie verschlingt, und das Gericht des lebendigen Gottes sie aus ihrer Schlafsucht weckt, wo dann in der Qual der Gottlosen, als dem Lohne ihrer Werke, ihre armen unglücklichen Seelen ihren Irrthum empfinden werden.

\section{8. Abschnitt}

Dieses war von jeher das Schicksal aller weltlichgesinnten Christen -- ist es noch und wird es immer seyn.
Ein so furchtbares Ende, daß ich, wenn mich auch meine Pflichten gegen Gott und meine Mitmenschen nicht aufforderten, schon als bloßer Mensch, und als Einer, der das Schrecken der Gerichte des Herrn in dem Wege und in der Bewirkung seiner eigenen Seligkeit aus Erfahrung kennt, allein durch das Mitleid mich hinreichend bewogen fühlen würde, diese Abhandlung zu schreiben, um die Bekenner des Christenthumes gegen die abergläubischen Meinungen, Gebräuche und Lüste der Welt zu warnen, und sie zu der Kenntniß des Kreuzes Christi, und zum täglichen Gehorsam gegen dasselbe, als dem einzigen uns von Christus angezeigten und verordneten Mittel zur Seligkeit, einzuladen;
damit Diejenigen, die sich jetzt des christlichen Namens bloß anmaßen, zum wahren Besitze der Sache gelangen, und durch die Kraft des Kreuzes,
-- gegen welches sie jetzt unempfindlich und tod sind;
statt daß sie durch das selbe der Welt gekreuzigt und abgestorben sein sollten,
-- Theilhaber an der Auferstehung in Christo Jesu werden, und zu einem neuen Leben kommen mögen.
Denn Alle, die wirklich in Christus sind, das heißt:
die eine Erlösung durch ihn, und eine Vereinigung mit ihm erfahren haben, sind "`\textit{neue Geborene}"' \footnote{"`neue Kreaturen"' ist ersetzt duch "`neue Geborene"'}''.\footnote{Galater 6, 14-16.} \index{Bibelstellen:!Galater 6} \index{Wiedergeboren}
Diese haben einen neuen Willen empfangen, womit sie den Willen Gottes und nicht ihr eigenes Wollen vollbringen.
Diese können in der Wahrheit beten, und sie verspotten Gott nicht, wenn sie sagen: "`\textit{Dein Wille geschehe auf Erden, wie im Himmel.}"'\footnote{Matthäus 6,10} \index{Bibelstellen:!Matthäus 6}
Diese haben ein neues Streben;
sie trachten nach dem, das droben ist, und ihr ewiger Schatz ist Christus.\footnote{Kolosser 3, 2. 3.} \index{Bibelstellen:!Kolosser 3}
Sie haben einen neuen Glauben, der die Eigenschaft hat, daß er die Fallstricke und Versuchungen des Geistes der Welt überwindet, wenn sie in ihnen selbst oder durch Andere erscheinen.
Sie haben endlich auch neue Werke, die nicht in abergläubischen Einrichtungen oder menschlichen Erfindungen, sondern in reinen Früchten des Geiste Christus \index{Geiste Christus} bestehen, welche derselbe in ihnen hervorbringt;
nämlich in Werken "`\textit{der Liebe, der Freude, des Friedens, der Geduld, der Freundlichkeit, der Gütigkeit, des Glaubens, der Sanftmuth, der Keuschheit, gegen welche das Gesetz nicht ist.}"'
Von denen hingegen, die den Geist Christus nicht haben und nach demselben nicht wandeln, sagt uns der Apostel, daß sie nicht zu den Seinigen gehören\footnote{Galater 5, 22 23.} \index{Bibelstellen:!Galater 5};
und auf Solchen liegt der Zorn Gottes;
und die Verdammung des göttlichen Gesetzes. \index{Gesetz, Das}
Denn, wenn, nach Paulus Lehre, "`\textit{nichts Verdammliches an denen ist, die in Jesus Christus sind, die nicht nach dem Fleische, sondern nach dem Geiste wandeln }"'\footnote{Römer 8, 9.} \index{Bibelstellen:!Römer 8};
so sind, nach derselben Lehre, Diejenigen, die nicht nach diesem heiligen Geiste wandeln, auch nicht in Christus.
Diese können also auch weder wahren Antheil an ihm haben, noch gerechte Ansprüche auf das durch ihn gebrachte Heil machen, und sind folglich der Verdammniß unterworfen. \index{Verdammniß}

\section{9. Abschnitt}

\marginpar{Den Ungerechten wird ihr Gewissen schon zur Lebzeiten zur Plage.} Es ist eine gewisse Wahrheit, daß die vorgebliche Religion der Gottlosen eine Lüge ist.
"`\textit{Die Gottlosen,}"' sagt der Prophet, "`\textit{haben keinen Frieden.}"' \footnote{Jesaja 48, 22.} \index{Bibelstellen:!Jesaja 48}
Wahren Gemüthsfrieden können sie in der Tat auch nicht haben; da sie bei allen ihren Werken des Ungehorsams in ihrem Gewissen bestraft und von ihrem eigenen Herzen verdammt werden.
Sie mögen gehen, wohin sie wollen, ihre Gewissensvorwürfe gehen mit ihnen, und oft verfolgt sie auch der Schrecken;
denn es ist ein beleidigter Gott, der sie beunruhigt, und durch sein Licht ihnen ihre Sünden der Reihe nach unter Augen stellt.
Zuweilen suchen sie ihn freilich durch ihre leibliche, selbstersonnene Andacht und Anbetung, zu versöhnen;
allein ihre Bemühungen sind vergeblich;
denn die wahre Gottesverehrung bestehet darin, daß man den Willen Gottes thue, den sie aber so oft übertreten.
Alles andere ist ein leeres Kompliment, wie es Jener machte, der da sagte, "`\textit{er wolle gehen, und doch nicht ging.}"' \footnote{Matthäus 21, 30.} \index{Bibelstellen:!Matthäus 21}
Zu andern Zeiten nehmen sie ihre Zuflucht zu Vergnügungen und Zerstreuungen in Gesellschaften, um die Stimme des göttlichen Bestrafers in ihren Herzen zu ersticken, oder seine Pfeile abzustumpfen, die beunruhigenden Gedanken zu verscheuchen, und sich außerhalb des Bezirkes dieses Störers ihrer Vergnügungen in Sicherheit zu begeben.
Aber der Allmächtige erreicht sie dennoch früher oder später gewiß.
Diejenigen, welche die Bedingungen seiner Barmherzigkeit verwerfen, können seiner endlichen Gerechtigkeit nicht entgehen.
Vergeblich werden dann die uneinsichtigen \footnote{"`unbußfertigen"' duch "`uneinsichtigen"' ersetzt} Empörer gegen sein Gesetz die Berge anrufen, und in den Höhlen der Erde Schutz suchen.
Sein alldurchforschendes Auge wird ihre dicksten, Bedeckungen durchdringen, und in ihrem Dunkel ein Licht anzünden, das ihre mit Schuld belasteten Seelen mit Schrecken erfüllen wird, und welches sie nie werden auslöschen können.
Gewiß!
ihr Ankläger ist bei ihnen, und sie können sich eben so wenig von ihm, als von sich selbst losmachen;
er ist in ihrer Mitte und wird sich fest an sie halten.
Derselbe Geist, der den Geistern der Gerechten Zeugniß giebt, wird gegen die ihrigen zeugen;
ja, ihre eigenen Herzen werden sich laut gegen sie erheben.
-- "`\textit{Wenn uns unser Herz verdamnmet,}"' sagt Johannes, "`\textit{so ist Gott noch größer als unser Herz, und er erkenne oder weiß Alles;}"' \footnote{Johannes 3, 20.} \index{Bibelstellen:!Johannes 3} das heißt:
Wenn der Mensch der Verdammung seines eigenen Herzens nicht ausweichen kann, so wird er auch gewiß den Gerichten Gottes nicht entgehen können;
da seine Macht unbegrenzt ist. \index{Allmacht Gottes}
\marginpar{Alle müssen eines Tages vor das Gericht.}
An jenem Tage werden die stolzen und üppigen Christen einsehen lernen, daß Gott die Person nicht ansiehet;
daß alle Sekten und Namen sich in zwei Gattungen: \index{Personen:!Sekten}
in Schafe und Böcke, nämlich in Gerechte und Ungerechte auslösen werden.
Und selbst der Gerechte hat eine so genaue Prüfung durchzugehen, daß deßhalb ein heiliger Mann zu dem Ausrufe bewogen wurde: "`\textit{Wenn der Gerechte kaum erhalten wird, wie will der Gottlose und Sünder erscheinen?}"' \footnote{1 Petrus 4,18.} \index{Bibelstellen:!1 Petrus 4}.
Wenn also die Gedanken, Worte und Handlungen der Gerechten eine solche Prüfung bestehen und vor dem unpartheiischen Richter des Himmels und der Erde untersucht werden müssen, wie sollte der Gottlose davon ausgenommen sein?
Nein!
Er, der nicht lügen kann, hat uns gesagt, daß Viele als dann \textit{Herr! Herr!} ausrufen, ihr Bekenntniß von ihm erheben, und alle die Werke, die sie in seinem Namen verrichtet haben, herzählen werden, um ihn gutmütig \footnote{"`geneigt"' durch "`gutmütig"' ersetzt} zu machen, und dennoch mit dem schrecklichen Aussprache verworfen werden sollen:
"`\textit{Weichet von mir ihr Uebelthäter; ich kenne euch nicht.}"' \footnote{Matthäus 7, 23.} \index{Bibelstellen:!Matthäus 7}
Als sagte er:
Geht nur fort, ihr Uebelthäter!
Ihr habt euch zwar zu mir bekannt;
aber ich will euch dennoch nicht anerkennen;
denn euer eitles und böses Leben hat euch für mein heiliges Reich untüchtig gemacht.
Gehet hin zu den Gözen, denen ihr gedienet habt;
zu euern geliebten Lüsten, die ihr angebetet, und zu der argen Welt, deren Freundschaft ihr so sehr gesucht, und die ihr so hoch verehrt habt;
laßt diese euch nun, wenn sie es können, von dem Zorne erretten, der, als der gerechte Lohn eurer Werke, über euch ausbrechen wird.
-- So endigt das Werk derer, die auf den Sand bauen;
der Athem des Richters bläset es um, und sein Fall ist schrecklich.
-- O dann, dann wird es sein, daß die Gerechten den Vorzug vor den Gottlosen haben werden!
Weßhalb auch schon in alten Zeiten ein Abtrünniger ausrief:
"`\textit{O möchte meine Seele den Tod des Gerechten sterben, und mein Ende wie das seinige sein!}"' \footnote{4.Mose 23,10.} \index{Bibelstellen:!4.Mose 23}
Ja, denn der Urtheilsspruch lautet anders;
der Richter lächelt freundlich!
Er wirft einen Blick voller Liebe auf seine eigenen Schafe, und ladet sie mit den holden Worten ein:
"`\textit{Kommet her, ihr Gesegneten meines Vaters!}"' \footnote{Matthäus 25, 34.} \index{Bibelstellen:!Matthäus 25}
Ihr, die ihr durch geduldiges Ausharren im Wohlthun schon lange der Unsterblichkeit entgegen sahet;
ihr seid die wahren Gefährten meiner Trübsale und meines Kreuzes gewesen, und habt mit unermüdeter Treue in der Unterwerfung unter meinen heiligen Willen muthvoll bis ans Ende ausgehalten, indem ihr auf mich, den Urheber eures köstlichen Glaubens, in Erwartung der Belohnung hinsahet, die ich Denen, die mich lieben, und nicht müde werden, verheißen habe.
-- "`\textit{Nun gehet ein, zu eures Herrn Freude, und ererbet das Reich, das vom Anfange der Welt her für euch bereitet ist.}"' \footnote{Matthäus 25, 34.}.

\section{10. Abschnitt}

O Christenheit!
Es ist das inbrünstige Gebet meiner Seele, daß, nach allem deinem hohen Bekenntnisse von Christus und von seiner sanften und heiligen Religion, dein unpassendes und dem Leben Christus so unähnliches Leben, dich an jenem großen Gerichtstage der Welt nicht verwerflich machen, und zuletzt um dein ewiges Heil bringen möge.
Höre mich daher noch ein wenig an;
Ich bitte dich darum.
Kann Christus wohl dein Herr sein, wenn du ihm keinen Gehorsam leistest?
oder kannst du dich seine Dienerin nennen, wenn du ihm gar nicht dienest? Irre dich nicht!
"`\textit{Was du säest, das wirst du auch ernten.}"' \footnote{Galater 6, 7.} \index{Bibelstellen:!Galater 6}
\marginpar{Erst wer sich duch Jesus von seinen Sünen erlösen läsast, ist tatsächlich in seiner Nachvolge.}
Er ist gewiß dein Erlöser und Heiland nicht, so lange du seine Gnade in deinem Herzen verwirfst, durch welche er dich erlösen und selig machen will.
Sage mir, wovon hat er dich erlöset? \index{Erlösung}
Hat er dich von deinen sündlichen Lüsten, von deinen weltlichen Begierden und von deinem eiteln Wandel erlöset?
-- Ist dieses nicht geschehen, so ist er auch dein Heiland und Erlöser nicht;
denn, obgleich er sich Allen zum Erlöser und Heilande darbietet, so kann er es doch eigentlich und in der Tat nur für Diejenigen sein, die sich durch ihn erlösen und selig machen lassen;
und es können Keine von ihm selig gemacht werden, die nicht aufhören wollen, in den sündlichen Dingen zu leben, die sie von Gott trennen, und von welchen er sie zu erlösen in die Welt kam.

\medskip

Christus ist gekommen, die Menschen von der Sünde und vom ewigen Tode, als dem Lohne derselben, zu erretten.
Allein Diejenigen, die sich nicht durch die in ihren Seelen wirkende Kraft Christi von der Macht und Herrschaft, welche die Sünde über sie ausübt, erlösen oder befreien lassen, können auch von dem ewigen Tode, dem sichern Lohne der Sünde, in der sie leben, nie errettet werden.

\medskip 

In wie fern also die Menschen über die bösen Neigungen und fleischlichen Lüste, denen sie ergeben waren, den Sieg erlangt haben, in so fern sind sie wahrhaft erlöset und selig gemacht, und wirkliche Zeugen \textbf{der Erlösung, die durch Jesum Christum geschiehet.}
Dieses wichtige Werk des Erlösers wird auch durch seinen Namen angezeigt: "`\textit{Seinen Namen,}"' sagte der Engel des Herrn, "`\textit{sollst du Jesus nennen; denn er wird sein Volk selig machen von \textbf{seinen Sünden.}}"'\footnote{Matthäus 1, 21.} \index{Bibelstellen:!Matthäus 1}
Und Johannes sagte von Christo: "`\textit{Siehe, das ist Gottes Lamm, welches die Sünden der Welt hinwegnimmt.}"'\footnote{Johannes 1, 19.} \index{Bibelstellen:!Johannes 1}
Das heißt so viel als:
\textit{betrachtet Ihn, den Gott gegeben hat, die Menschen zu erleuchten, und alle Diejenigen von ihren Sünden zu befreien und selig zu machen, die ihn, sein Licht und seine Gnade, in ihren Herzen aufnehmen, täglich sein Kreuz tragen, und ihn nachfolgen}.
Das sind Solche, die lieber dem Vergnügen der Befriedigung ihrer Lüste und Begierden entsagen, als gegen die Erkenntniß, die er ihnen von seinem Willen gegeben hat, sündigen, oder etwas thun, wovon sie wissen, daß sie es nicht thun sollten.




