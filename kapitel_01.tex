

\chapter{Kapitel 1} \label{kap1}



\section{Zusammenfassung des 1. Kapitel}


\begin{description}
\item[1. Abschnitt] Von der Notwendigkeit des Kreuzes Christi überhaupt, und
wie wenig dennoch die Christen sich darum kümmern.
\dotfill \textit{Seite~\pageref{kap1_ab1}}\\
\item[2. Abschnitt] Ausartung des Christentums von Reinheit in Lüste und
Begierden, und von Mäßigkeit in Übermaß.
\dotfill \textit{Seite~\pageref{kap1_ab2}}\\
\item[3. Abschnitt] Weltliche Lüste und Vergnügungen sind so sehr das Ziel und
Streben der Bekenner des Christentums geworden, dass sie die Gottlosigkeit der
Ungläubigen darin übertreffen.
\dotfill \textit{Seite~\pageref{kap1_ab3}}\\
\item[4. Abschnitt] Diese Ausartung bildet den zweiten Akt des Trauerspiels,
welchen die Juden angefangen haben, und dieser ist schlimmer\editnote{Im Urtext
wird das Wort 'ärger' benutzt, statt 'schlimmer'.} als der erste. --
Bemerkungen über die Verachtung, welche die Christen auf ihren Heiland gebracht
haben.
\dotfill \textit{Seite~\pageref{kap1_ab4}}\\
\item[5. Abschnitt] Die Sünde ist in der ganzen Welt von der
gleichen\editnote{im Org. 'einerlei'.} Natur und Beschaffenheit. -- Alle
Gottlosen gehören zu einer und derselben Gemeine;
\index{Böse!Gemeinde des Bösen}
\index{Kirche!Gemeinde des Bösen}
sind alle Kinder des Bösen\index{Böse!Kinder des Bösen}\editnote{'Argen' ersetzt durch 'Bösen'.}. --
Bösewichter, welche Religion zu haben vorgeben, sind darum nur desto schlimmer.
\dotfill \textit{Seite~\pageref{kap1_ab5}}\\
\item[6. Abschnitt] Ein Wolf ist kein Lamm: Ein Sünder kann, so lange er in der
Sünde bleibt, kein Heiliger sein
\dotfill \textit{Seite~\pageref{kap1_ab6}}\\
\item[7. Abschnitt] Die Gottlosen verfolgen immer\editnote{'immer' statt
'allezeit'.} die Frommen; auch haben immer die falschen Christen die wahren
verfolgt, weil diese ihrem Aberglauben nicht beipflichten wollten. -- Von den
sonderbaren und weltlichen\editnote{'fleischlichen' ersetzt durch 'weltlichen'.}
Begriffen, welche die falschen Christen vom
Christentum haben, und von der Gefahr eines solchen Selbstbetrugs.
\dotfill \textit{Seite~\pageref{kap1_ab7}}\\
\item[8. Abschnitt] Diese Betrachtungen, und meine Empfindungen darüber, haben
es mir zur Pflicht gemacht, die gegenwärtige Abhandlung, als eine Warnung gegen
die Lüste der Welt und als eine Einladung zum täglichen Aufnehmen des Kreuzes
Christi zu schreiben, und zu zeigen, dass dieses das von Christus uns verordnete
Mittel zu unserer Seligkeit ist.
\dotfill \textit{Seite~\pageref{kap1_ab8}}\\
\item[9. Abschnitt] Über die Selbstverdammung der Gottlosen. -- Wahre Religion
und Gottesverehrung \index{Religion!wahre} besteht darin, dass man den
Willen Gottes tut.~-- Von dem
Vorzug, den die Gerechten vor den Gottlosen im jüngsten Gerichte haben.
\dotfill \textit{Seite~\pageref{kap1_ab9}}\\
\item[10. Abschnitt] Gebet für die Christenheit, dass sie an jenem großen
Gerichtstag der Welt nicht möge verworfen werden. -- Sie wird ermahnt, zu
erwägen, worin sie Christo ähnlich ist; und, wenn er ihr Heiland und Erlöser
ist, wie, und wovon er sie erlöst habe, und was ihre eigene Erfahrung von diesem
großen Werk sei? -- Christus kam in die Welt, die Menschen von ihren Sünden, und
also auch vom ewigen Zorn zu befreien; aber nicht, um sie in ihren Sünden selig
zu machen. -- Indem er sie von der Sünde erlöst, errettet er sie auch vom ewigen
Tod, welcher der Sold oder Lohn der Sünde ist.
\dotfill \textit{Seite~\pageref{kap1_ab10}}\\
\end{description}

\newpage

\section{1. Abschnitt} \label{kap1_ab1}

Obgleich die Kenntnis und Ausübung der Lehre vom Kreuze Christi, als dem
einzigen Eingang zum wahren Christentum, und dem Pfade, den allezeit die Alten
zu ihrer Seligkeit betraten, für die Seelen der Menschen von der höchsten
Wichtigkeit ist, so wird dennoch diese Lehre, (ich sage es mit tiefer
Betrübnis!) so wenig verstanden, so sehr vernachlässigt, und (was noch
schlimmer ist) es wird ihr durch die Eitelkeit, den Aberglauben, und die
Unmäßigkeit der Christentumsbekenner so bitter widersprochen, dass wir entweder
aufhören müssen, zu glauben, was der Herr Jesus Lukas 24,27.\bibleindex{Lukas 24} uns sagt, wo er nämlich erklärt, "`\textit{dass
niemand, der nicht sein Kreuz trägt und ihm nachfolgt, sein Jünger sein
könne,}"' oder, (wenn wir dieses als Wahrheit annehmen) nicht anders
schließen können, als dass die Mehrheit der Bekenner des christlichen Namens, in
der großen Angelegenheit der Religion und ihres eigenen Heils, auf eine
bejammernswerte Art sich täuscht und selbst betrügt.

\section{2. Abschnitt} \label{kap1_ab2}

\label{ref:01_02_urchristentum} Wir mögen den Zustand der Völker, die auf die Wohltat des heiligen Namens Jesu
Anspruch machen, noch so nachsichtsvoll und liebreich beurteilen, so müssen wir
doch auch, wenn wir zugleich gerecht handeln wollen, notgedrungen gestehen, dass
ungeachtet der gnädigen Vorteile des Lichts und der Erkenntnis, und der
Aufmunterungen zur Treue, welche in diesen letzten Jahrhunderten durch die
Erscheinung, das Leben, die Lehren und Wunder, durch den Tod, die Auferstehung
und Himmelfahrt Christi, nebst den Gaben seines heiligen Geistes den Menschen
verliehen worden sind, \textbf{ungeachtet der Schriften, Arbeiten, Leiden und
Erduldungen des Martertodes seiner teueren Zeugen in allen Zeiten, nicht viel
mehr als der bloße Name vom wahren Christentum übrig geblieben zu sein
scheint.} Und wo nun die alte heidnische Natur der Menschen sich dieses Namens
anmaßt, oder ihr zügelloses Leben damit zu bedecken sucht, da sind die Bekenner
desselben in der Tat nichts anderes, als wirkliche, wiewohl verkleidete Heiden.
Denn wenn sie auch nicht dieselben Götzen der Heiden anbeten, so beten sie doch
Christum mit einem heidnischen Herzen an; und sie können auch nicht anders, so
lange sie in gleichen heidnischen Lüsten leben. So gehören also beide: der
Christ, der sich nicht selbst überwindet, und der zügellose Heide, zu einer und
derselben Religion\index{Heide}. Beide haben freilich verschiedene Gegenstände,
an welche sie ihre Gebete richten, allein ihre Anbetung ist doch nur erzwungen
und bloße Zeremonie. Denn die Gottheit, die sie im wahren Sinne verehren, ist
der Gott dieser Welt, der große Beherrscher der weltlichen Lüste und Begierden.
Vor ihm beugen sie sich mit allen Kräften der Seele und der Sinne. Was sollen
wir essen? Was sollen wir trinken? Was sollen wir anziehen, und wie sollen wir
unsere Zeit hinbringen? Auf welche Art können wir uns Reichtum erwerben?
Wodurch können wir unsere Macht vergrößern, unsere Besitzungen ausdehnen, unsere
Namen und Familien in der Welt berühmt machen und verewigen? -- Diese niedrige
Sinnlichkeit fasst der geliebte Apostel Johannes sehr kurz und nachdrucksvoll in
einigen Worten zusammen:
\textit{"`Fleischeslust, Augenlust und hochmütiges\editnote{'hofartiges' ersetzt durch 'hochmütiges'.}
Leben,"'} sagt er, \textit{"`sind nicht vom Vater, sondern von der Welt,}"'\biblecite{Johannes 2,16.}{Johannes 02@Johannes 2} die im Argen liegt.

\section{3. Abschnitt} \label{kap1_ab3}

Es ist eine traurige Feststellung, aber eine durchaus nicht zu leugnende
Wahrheit, dass diese weltlichen Lüste die Gegenstände des Nachsinnens, der Sorge
und der Unterhaltung des größten Teils der unglücklichen Christenheit ausmachen,
und -- was das Elend noch vergrößert -- mit der Zeit zugenommen haben. Denn, so
wie die Welt älter geworden ist, hat sie sich auch verschlimmert. Die Beispiele
früherer, ausschweifender Zeitalter, und die daraus zu ziehenden beklagenswerten
Folgerungen haben das unsrige Zeitalter nicht abgeschreckt, sondern vielmehr
noch gereizt, so dass die Menschen unserer Zeit den alten Vorrat von
Gottlosigkeit noch mehr angehäuft haben. Ja, sie haben die ihnen gegebenen bösen
Beispiele so sehr übertroffen, dass sie, statt in besseren Zeiten Fortschritte in
der Tugend zu machen, auf eine abscheuliche Art tief unter die Heiden
herabgesunken sind. --
Sie haben ihren Hochmut, ihre wollüstige
Ausgelassenheit, Unreinheit und Trunkenheit, ihr Fluchen, Schwören und Lügen,
ihr Neiden und Verleumden, ihre Grausamkeit, Falschheit, Habsucht,
Ungerechtigkeit und Unterdrückung, so allgemein verbreitet und mit einem
erfinderischen Geist so hoch getrieben, dass sie darin den Ungläubigen zum
Anstoße und Ärgernisse gedient und ihnen die stärkste Veranlassung gegeben
haben, die heilige Religion mit Verachtung zu betrachten, für welche sie durch
gute Beispiele der Christen hätten gewonnen werden können.

\section{4. Abschnitt} \label{kap1_ab4}

Diesen traurigen Abfall von der ursprünglichen Reinheit der ersten Zeiten des
Christentums, als der Ruhm desselben in dem reinen Lebenswandel seiner
Bekenner bestand, kann ich nicht anders als den zweiten und furchtbarsten Teil
des Trauerspiels betrachten,
\label{ref:01_04_zweite_kreuzigung}
\textbf{welches die Juden \index{Juden} mit dem
glorreichen Heilande des Menschengeschlechts begannen. Diese, die durch die
Macht der Unwissenheit und der großen Vorurteile, die sie gegen seine in den
Augen der Welt unansehnliche Erscheinung hatten, so verblendet waren, dass sie
ihn, als er erschien, nicht annehmen wollten, verfolgten ihn jedoch nur zwei
oder drei Jahre, bis sie ihn zuletzt an einem Tage kreuzigten.}
\textbf{Allein die Grausamkeit der falschen Christen ist von weit längerer
Dauer. Nachdem sie, wie Judas\personenindex{Judas}, zuerst ihn anerkannt und
dann viele Jahrhunderte hindurch auf das Schändlichste verraten haben, hören
sie nicht auf, ihn zu verfolgen und zu kreuzigen, indem sie von seiner Lehre,
welche Selbstüberwindung und Heiligkeit vorschreibt, in ihren Sitten fortwährend
abweichen, und durch ihren Lebenswandel ihrem Glaubensbekenntnisse beständig
widersprechen. Von solchen sagt uns der Verfasser der Epistel an die Hebräer,
\textit{"`dass sie ihnen selbst den Sohn Gottes von neuem wieder kreuzigen und
öffentlich zum Gespötte machen."'}}\biblecite{Hebräer 6,6.}{Hebräer 06@Hebräer 6}Johannes nennt ihre verunreinigten Herzen in seiner Offenbarung:
\textit{"`die Gassen des geistlich sogenannten Sodoms\index{Orte:!Sodom, geistlich} und Ägyptens\index{Orte:!Ägypten}, wo unser Herr gekreuzigt ist."'}\biblecite{Offenbarung 11,8.}{Offenbarung 11}.
Und so wie Christus ehemals sagte: "`\textit{des
Menschen Feinde werden seine eigenen Hausgenossen sein,}"'\biblecite{Matthäus 10,36.}{Matthäus 10}
so befinden sich jetzt die Feinde Christi vornehmlich unter seinen
eigenen Bekennern, unter welchen es nicht wenige gibt, die ihn anspeien, ans
Kreuz nageln und durchbohren und ihm Essig mit Galle vermischt zu trinken
geben.\biblecite{Matthäus 27.}{Matthäus 27}
Dieses ist auch nicht schwer einzusehen, da
diejenigen Menschen, die nach ihrer verdorbenen Natur und unter demselben bösen
Einflusse leben, worunter die gottlosen Juden standen, welche Christum äußerlich
kreuzigten, ihn gewiss innerlich kreuzigen\index{Kreuz!innerlich kreuzigen}, und alle, welche jetzt die
Erscheinung und Zucht seiner Gnade\index{Gnade} in ihren eigenen Herzen verwerfen, gleiche
Stammes und Geschlechtes mit jenen verhärteten Juden sind, die damals derselben
Gnade widerstanden, als sie in Christo erschien und durch ihn geoffenbart
ward.

\section{5. Abschnitt} \label{kap1_ab5}

\label{ref:01_05_in_suende_gleich}
\textbf{Die Sünde
ist, von einem Ende der Welt bis zum anderen, von einerlei Natur und
Beschaffenheit. Denn wenn auch ein Lügner kein Trunkenbold, oder ein Flucher
kein Hurer, und keiner von ihnen eigentlich ein Mörder ist, so gehören sie doch
alle zu \textit{einer Gemeinschaft}, sind alle Zweige aus einer und derselben
bösen Wurzel, alle eines Geschlechts.} Die Gottlosen haben nur einen
gemeinschaftlichen Vater, wie Christus den Bekennern des Judentums, die in
jenem Zeitalter die sichtbare Kirche ausmachten, frei erklärte, indem er ihre
Ansprüche auf Moses und Abraham \personenindex{Mose} \personenindex{Abraham}
verwarf, und ihnen gerade heraus sagte,
\textit{"`wer Sünde tue, sei der Sünde Knecht; sie täten die Werke des Teufels
und wären folglich des Teufels Kinder."'},\biblecite{Johannes 8,34-45.}{Johannes 08@Johannes 8}.
Diese Behauptung wird immer wahr bleiben,
so lange dieselben Gründe dafür vorhanden sind.
\textit{"`Wem ihr euch zum Gehorsam ergebt,"'} sagt Paulus, \textit{"`dessen
Knechte seid ihr."'}\biblecite{Römer 6,16.}{Römer 06@Römer 6}
Und Johannes sagt in seiner allgemeinen Epistel an die ersten Gemeinden:
\textit{"`Lasst euch von niemand betrügen, wer Sünde tut, der ist vom Teufel."'}\biblecite{Johannes 3,7+8.}{Johannes 03@Johannes 3}
-- War Judas \personenindex{Judas} darum ein besserer Christ, dass er
\textit{"`gegrüßet seist du, Meister!"'} ausrief, und Christum küsste?
Keinesweges. Es war vielmehr das Zeichen seines Verrats, die Losung, wodurch 
die blutdürstigen Juden Christum erkennen sollten, damit sie
ihn greifen konnten. Judas nannte Christus \textit{Meister} und verriet ihn
er küsste ihn, und verkaufte ihn zum Tode.
So verhält es sich mit der Religion der falschen \textbf{Namenschristen}
\index{Namenschristen} noch jetzt.
Fragt man sie, ob Christus ihr Herr sei,
so sind sie bereit auszurufen:
\textit{"`Behüte uns Gott, dass es anders wäre! Freilich ist er unser Herr!"'}
-- \textit{"`Wohlan denn! Haltet ihr aber auch seine Gebote?"'}
-- \textit{"`O Nein! Wie könnten wir das?"'}
-- \textit{"`Wie dürft ihr euch denn seine Jünger nennen?"'}
-- \textit{"`Es ist unmöglich"'}!
...antworten sie.
\textit{"`Wie kann man verlangen, dass wir seine Gebote halten sollen? Das kann
ja kein Mensch."'}
-- Wie? Es wäre unmöglich, das zu tun, ohne dessen Ausübung Christus es für
unmöglich erklärt, ein Christ zu sein? Ist Christus denn im Irrtum\editnote{'unbillig' durch 'in Irrtum' ersetzt.}? Wird er
\textit{"`da ernten wollen, wo er nicht gesät hat?"'}\biblecite{Matthäus 25,24.}{Matthäus 25}
oder etwas von uns verlangen, wozu er uns
keine Fähigkeit gab?
\index{Judaskuss} \label{ref:01_05_in_suende_verbinden}
-- \textbf{So geht es zu, dass die falschen Christen mit Judas zusammen, Christus ihren
Herrn und Meister nennen, zu gleicher Zeit aber mit dem bösen Haufen der Welt
sich verbinden,} um ihn zu verraten,
dass sie ihn umarmen und küssen, soweit ein scheinbares Namenbekenntnis
\index{Namenbekenntnis}\index{Bekenntnis} reicht, ihn aber treulos verkaufen, sobald es darauf
ankommt, ihre herrschende Leidenschaft, der sie am meisten nachhängen, zu
befriedigen.

\section{6. Abschnitt} \label{kap1_ab6}

Möchte doch keiner seine eigene Seele betrügen!
"`\textit{Man kann nicht Trauben von Dornen, oder Feigen von Disteln sammeln.}"'\biblecite{Matthäus 7,16.}{Matthäus 07@Matthäus 7}
Ein Wolf ist kein Lamm und ein Geier keine Taube.
Zu welcher äußeren Religionsform, zu welcher religiösen Gesellschaft, oder zu
welcher Kirche du dich auch bekennst, so ist es eine an dich und alle Menschen
gerichtete Wahrheit Gottes, dass diejenigen, welche die Form und den Schein der
Gottseligkeit haben, aber durch ihr ungöttliches Leben die Kraft derselben
verleugnen, nicht die wahre, sondern die falsche Kirche ausmachen, die, obgleich
sie sich den Titel der Braut\index{Braut Christi} des Lammes\index{Lamm!Gottes}, oder der Kirche Christi, beilegt,
dennoch jenes große Geheimnis, oder "`die geheimnisvolle \textit{Babylon}"'
\index{Orte:!Babylon} \index{Hure Babylon} ist, welche der heilige
Geist so passend
\textit{"`die Mutter der Hurerei und aller Gräuel auf Erden"'}\biblecite{Offenbarung 17,5.}{Offenbarung 17}
nennt, weil
sie von der christlichen Keuschheit und Reinheit ausgeartet ist, in alle Gräuel
des heidnischen Babylon, einer prachtvollen Stadt der Vorzeit, die als Sitz der
babylonischen Könige und der größten Hoffart und Üppigkeit in der damaligen
Welt berühmt war. Was nun dieses mystische Babylon damals war, das ist sie auch
noch jetzt: die größte Feindin der Sache und des Volks Gottes.

\section{7. Abschnitt} \label{kap1_ab7}

Es bleibt auch wahr, dass die, welche vom Fleische geboren sind,
diejenigen, welche aus dem Geiste geboren sind und die \textit{Beschneidung des
Herzens} \index{Beschneidung!des Herzens} erfahren haben,\biblecite{Galater 4,29.}{Galater 04@Galater 4}
hassen und verfolgen, weil sie nach
Babylons Erfindungen, Lehrarten und Vorschriften Gott nicht verehren und
anbeten, und weder ihre nichtigen Traditionen als Lehren annehmen, noch im Leben
und Wandel nach ihren verderbten Moden und Gebräuchen sich bequemen können.
Wo dieses nun der Fall ist, da verwandelt die Abtrünnige sich in eine
Verfolgerin.
-- Denn es ist nicht genug, dass sie selbst von der ersten Reinheit des
Christentums abgewichen ist;
Nein!
Andere sollen es ihr auch nachtun.
Darum lässt sie auch denen, die an ihrer Ausartung keinen Anteil haben oder ihr
Malzeichen\editnote{Anmerkung: Vermutlich die Anspielung auf Offenbarung 13,16.}
\index{Bibelstellen:!Offenbarung 13)}
nicht annehmen wollen, keine Ruhe.
-- Wer ist auch wohl weiser als sie, die Mutterkirche?\index{Kirche!Mutterkirche}
Und wer kann mit dem Tiere, auf dem sie reitet, streiten?

\medskip

Die Abtrünnigen und Abergläubigen\index{Aberglaube} sind immer stolz auf ihren Irrtum und
unduldsam gegen andere, die nicht ihrer Meinung sind.
Alle sollen ihnen beistimmen oder umkommen.
Daher werden
\textit{"`die erschlagenen Zeugen und das Blut der Seelen unter
dem Altare"'}\biblecite{Offenbarung 6,9.}{Offenbarung 06@Offenbarung 6}
innerhalb der Mauern dieser geheimnisvollen Hure Babylon, dieser großen, festen Stadt
der falschen Christen, gefunden, und von dem heiligen Geist in der Offenbarung
ihr zur Last gelegt werden.
Es ist freilich nicht zu bewundern, dass sie, die zuerst den Herrn kreuzigte,
hernach auch seine Knechte tötete,
aber höchst sonderbar und zugleich grausam ist es, dass sie ihren Bräutigam
töten, ihren Heiland ermorden kann,
da sie doch diese beiden Benennungen, die ihr so viel eingebracht haben, so sehr
zu lieben scheint, und auch durch dieselben,
-- wiewohl ohne allen gerechten Anspruch, --
sich immer noch zu empfehlen sucht.
Indessen sind ihre Kinder, durch ihren fortwährenden Ungehorsam gegen die
Offenbarung des göttlichen Lichts in ihren Seelen, so gänzlich unter die
Herrschaft der Finsternis geraten, dass sie vergessen haben, was der Mensch
einst war, oder was sie jetzt sein sollten, und wahres reines Christentum, wenn
sie es antreffen, nicht einmal kennen
wiewohl sie sich viel darauf einbilden, zu der Zahl der Bekenner desselben zu
gehören.
Ihre Begriffe vom Heil der Seele sind so fleischlich und falsch, dass sie Gutes
bös und Böses gut nennen.
Sie halten Menschen wie Teufel für Christen, und Heilige für Teufel.
-- Obgleich nun die Erwägung der gottlosen Ungebundenheit ihres Lebens, da
dieselbe ihr Verderben nach sich zieht, schon das tiefste Bedauern erregen muss,
so ist doch von allen Selbsttäuschungen, unter denen sie sich befinden,
hinsichtlich ihres ewigen Zustandes, die verderblichste diese, dass sie in dem
allgemeinen Wahn stehen, sie könnten Kinder Gottes sein, während sie im
Ungehorsam gegen seine heiligen Gebote leben.
Sie dürften sich für Jünger Jesu halten, obgleich sie sich weigern, sein Kreuz
zu tragen, und sie könnten sich auch als Glieder seiner wahren Kirche
betrachten, welche heilig und ohne Tadel sein sollen, ungeachtet sie ein
unheiliges und tadelhaftes Leben führen.
So sind sie mitten in ihren Sünden im Frieden und halten sich in ihren
Übertretungen für sicher.
Ihre eitle Hoffnung betäubt ihre bessere Überzeugung und erstickt jede zarte
Ermahnung zur Reue,
so dass also ihr Irrtum in Ansehung ihrer Pflichten gegen Gott ebenso
gefährlich als ihre Empörung gegen ihn ist. \label{ref:01_07_selbstbetrug}
-- \textbf{So wandeln sie an Abgründen und täuschen sich selbst mit
schmeichelhaften Vorstellungen, bis das Grab sie verschlingt und das
Gericht des lebendigen
Gottes sie aus ihrer Schlafsucht weckt}, wo dann in der Qual der Gottlosen, als
dem Lohne ihrer Werke, ihre armen unglücklichen Seelen ihren Irrtum empfinden
werden.

\section{8. Abschnitt} \label{kap1_ab8}

Dieses war von jeher das Schicksal aller weltlich gesinnten Christen -- ist es
noch und wird es immer sein.
Ein so furchtbares Ende, dass ich, wenn mich auch meine Pflichten gegen Gott und
meine Mitmenschen nicht aufforderten, schon als bloßer Mensch, und als einer,
der den Schrecken der Gerichte des Herrn in dem Wege und in der Bewirkung seiner
eigenen Seligkeit aus Erfahrung kennt, allein durch das Mitleid mich hinreichend
bewogen fühlen würde, diese Abhandlung zu schreiben, um die Bekenner des
Christentums gegen die abergläubischen Meinungen, Gebräuche und Lüste der Welt
zu warnen, und sie zu der Kenntnis des Kreuzes Christi und zum täglichen
Gehorsam gegen dasselbe, als dem einzigen uns von Christus angezeigten und
verordneten Mittel zur Seligkeit, einzuladen;
damit diejenigen, die sich jetzt des christlichen Namens bloß anmaßen, zum
wahren Besitze der Sache gelangen, und durch die Kraft des Kreuzes,
-- gegen welches sie jetzt unempfindlich und tot sind,
statt dass sie durch dasselbe der Welt gekreuzigt und abgestorben sein sollten,
-- Teilhaber an der Auferstehung in Christo Jesu werden und zu einem neuen
Leben kommen mögen.
Denn alle, die wirklich in Christus sind, das heißt:
die eine Erlösung durch ihn und eine Vereinigung mit ihm erfahren haben, sind
\textit{"`neu Geborene"'}\editnote{'neue Kreaturen' ist ersetzt durch 'neu Geborene'.}.\biblecite{Galater 6,14-16.}{Galater 06@Galater 6}
\index{Wiedergeboren}
Diese haben einen neuen Willen empfangen, womit sie den Willen Gottes und nicht
ihr eigenes Wollen vollbringen.
Diese können in der Wahrheit beten und sie verspotten Gott nicht, wenn sie
sagen:
\textit{"`Dein Wille geschehe auf Erden, wie im Himmel."'}\biblecite{Matthäus 6,10.}{Matthäus 06@Matthäus 6}
Diese haben ein neues Streben;
sie trachten nach dem, das droben ist, und ihr ewiger Schatz ist
Christus.\biblecite{Kolosser 3,2+3.}{Kolosser 03@Kolosser 3}
Sie haben einen neuen Glauben, der die Eigenschaft hat, dass er die Fallstricke
und Versuchungen des Geistes der Welt überwindet, wenn sie in ihnen selbst oder
durch andere erscheinen.
Sie haben endlich auch neue Werke, die nicht in abergläubischen Einrichtungen
oder menschlichen Erfindungen, sondern in reinen Früchten des Geistes Christus
\index{Geiste Christus} bestehen, welche derselbe in ihnen hervorbringt,
nämlich in Werken
\textit{"`der Liebe, der Freude, des Friedens, der Geduld, der
Freundlichkeit, der Gütigkeit, des Glaubens, der Sanftmut, der Keuschheit,
gegen welche das Gesetz nicht ist."'}
Von denen hingegen, die den Geiste Christus nicht haben und nach demselben nicht
wandeln, sagt uns der Apostel, dass sie nicht zu den seinigen
gehören\biblecite{Galater 5,22+23.}{Galater 05@Galater 5},
und auf solchen liegt der Zorn Gottes
und die Verdammung des göttlichen Gesetzes. \index{Gebote!Gesetz}
Denn, wenn, nach Paulus Lehre, "`\textit{nichts Verdammliches an denen ist, die
in Jesus Christus sind, die nicht nach dem Fleische, sondern nach dem Geiste
wandeln}"'\biblecite{Römer 8,9.}{Römer 08@Römer 8};
so sind, nach derselben Lehre, diejenigen, die nicht nach diesem heiligen Geiste
wandeln, auch nicht in Christus.
Diese können also auch weder wahren Anteil an ihm haben, noch gerechte
Ansprüche auf das durch ihn gebrachte Heil machen, und sind folglich der
Verdammnis unterworfen. \index{Verdammnis}

\section{9. Abschnitt} \label{kap1_ab9}

Es ist eine gewisse Wahrheit, dass die angebliche Religion der Gottlosen eine Lüge
ist.
"`\textit{Die Gottlosen,}"' sagt der Prophet, "`\textit{haben keinen Frieden.}"'\biblecite{Jesaja 48,22.}{Jesaja 48}
Wahren Gemütsfrieden können sie in der Tat auch nicht haben, da sie bei allen
ihren Werken des Ungehorsams in ihrem Gewissen bestraft und von ihrem eigenen
Herzen verdammt werden.
Sie mögen gehen, wohin sie wollen, ihre Gewissensvorwürfe gehen mit ihnen, und
oft verfolgt sie auch der Schrecken,
denn es ist ein beleidigter Gott, der sie beunruhigt und durch sein Licht ihnen
ihre Sünden der Reihe nach unter Augen stellt.
Zuweilen suchen sie ihn freilich durch ihre leibliche, selbstersonnene Andacht
und Anbetung zu versöhnen;
allein ihre Bemühungen sind vergeblich,
denn die wahre Gottesverehrung besteht darin, dass man den Willen Gottes tue,
den sie aber so oft übertreten.
Alles andere ist ein leeres Kompliment, wie es jener machte, der da sagte,
"`\textit{er wolle gehen, und doch nicht ging.}"'\biblecite{Matthäus 21,30.}{Matthäus 21}
Zu anderen Zeiten nehmen sie ihre Zuflucht zu Vergnügungen und Zerstreuungen in
Gesellschaften, um die Stimme des göttlichen Bestrafers in ihren Herzen zu
ersticken, oder seine Pfeile abzustumpfen, die beunruhigenden Gedanken zu
verscheuchen, und sich außerhalb des Bezirks dieses Störers ihrer Vergnügungen
in Sicherheit zu begeben.
Aber der Allmächtige erreicht sie dennoch früher oder später gewiss.
Diejenigen, welche die Bedingungen seiner Barmherzigkeit verwerfen, können
seiner endlichen Gerechtigkeit nicht entgehen.
Vergeblich werden dann die uneinsichtigen\editnote{'unbußfertigen' durch
'uneinsichtigen' ersetzt.} Empörer gegen sein Gesetz die Berge anrufen und in
den Höhlen der Erde Schutz suchen.
Sein alldurchforschendes Auge wird ihre dicksten Bedeckungen durchdringen und
in ihrem Dunkel ein Licht anzünden, das ihre mit Schuld belasteten Seelen mit
Schrecken erfüllen wird und welches sie nie werden auslöschen können.
Gewiss!
Ihr Ankläger ist bei ihnen, und sie können sich ebenso wenig von ihm, als von
sich selbst losmachen.
Er ist in ihrer Mitte und wird sich fest an sie halten.
Derselbe Geist, der den Geistern der Gerechten Zeugnis gibt, wird gegen die
ihrigen zeugen.
Ja, ihre eigenen Herzen werden sich laut gegen sie erheben.
-- "`\textit{Wenn uns unser Herz verdammt,}"' sagt Johannes, "`\textit{so ist
Gott noch größer als unser Herz, und er erkennt oder weiß alles;}"'\biblecite{Johannes 3,20.}{Johannes 03@Johannes 3} das heißt:
Wenn der Mensch der Verdammung seines eigenen Herzens nicht ausweichen kann, so
wird er auch gewiss den Gerichten Gottes nicht entgehen können,
da seine Macht unbegrenzt ist. \index{Gott!Allmacht}
An jenem Tage werden die stolzen und üppigen Christen einsehen lernen, dass Gott
die Person nicht ansieht,
dass alle Sekten und Namen sich in zwei Gattungen: \index{Sekten}
in Schafe und Böcke, nämlich in Gerechte und Ungerechte, auflösen werden.
Und selbst der Gerechte hat eine so genaue Prüfung durchzugehen, dass deshalb ein
heiliger Mann zu dem Ausrufe bewogen wurde: "`\textit{Wenn der Gerechte kaum
erhalten wird, wie will der Gottlose und Sünder erscheinen?}"'\biblecite{1
Petrus 4,18.}{Petrus 1 04@1. Petrus 4}.
Wenn also die Gedanken, Worte und Handlungen der Gerechten eine solche Prüfung
bestehen und vor dem unparteiischen Richter des Himmels und der Erde untersucht
werden müssen, wie sollte der Gottlose davon ausgenommen sein?
Nein!
Er, der nicht lügen kann, hat uns gesagt, dass viele als dann \textit{Herr!
Herr!} ausrufen, ihr Bekenntnis von ihm erheben und alle die Werke, die sie in
seinem Namen verrichtet haben, erzählen werden, um ihn gutmütig\editnote{'geneigt' durch 'gutmütig' ersetzt.} zu machen, und dennoch mit dem
schrecklichen Ausspruch verworfen werden sollen:
"`\textit{Weicht von mir, ihr Übeltäter; ich kenne euch nicht.}"'\biblecite{Matthäus 7, 23.}{Matthäus 07@Matthäus 7}
Als sagte er:
Geht nur fort, ihr Übeltäter!
Ihr habt euch zwar zu mir bekannt,
aber ich will euch dennoch nicht anerkennen,
denn euer eitles und böses Leben hat euch für mein heiliges Reich untüchtig
gemacht.
Gehet hin zu den Götzen, denen ihr gedient habt,
zu euern geliebten Lüsten, die ihr angebetet, und zu der argen Welt, deren
Freundschaft ihr so sehr gesucht, und die ihr so hoch verehrt habt,
lasst diese euch nun, wenn sie es können, von dem Zorn erretten, der, als der
gerechte Lohn eurer Werke, über euch ausbrechen wird.
-- So endigt das Werk derer, die auf den Sand bauen;
der Atem des Richters bläst es um, und sein Fall ist schrecklich.
-- O dann, dann wird es sein, dass die Gerechten den Vorzug vor den Gottlosen
haben werden!
Weshalb auch schon in alten Zeiten ein Abtrünniger ausrief:
"`\textit{O möchte meine Seele den Tod des Gerechten sterben und mein Ende wie
das seinige sein!}"'\biblecite{4.~Mose 23,10.}{Mose 4 23@4. Mose 23}
Ja, denn der Urteilsspruch lautet anders,
der Richter lächelt freundlich!
Er wirft einen Blick voller Liebe auf seine eigenen Schafe, und ladet sie mit
den holden Worten ein:
"`\textit{Kommt her, ihr Gesegneten meines Vaters!}"'\biblecite{Matthäus 25,34.}{Matthäus 25}
Ihr, die ihr durch geduldiges Ausharren im Wohltun schon lange der
Unsterblichkeit entgegen gesehen habt;
ihr seid die wahren Gefährten meiner Trübsale und meines Kreuzes gewesen und
habt mit unermüdeter Treue in der Unterwerfung unter meinen heiligen Willen
mutvoll bis ans Ende ausgehalten, indem ihr auf mich, den Urheber eures
köstlichen Glaubens, in Erwartung der Belohnung hinsaht, die ich denen, die
mich lieben und nicht müde werden, verheißen habe.
-- "`\textit{Nun geht ein, zu eures Herrn Freude und ererbt das Reich, das
vom Anfange der Welt her für euch bereitet ist.}"'\biblecite{Matthäus 25, 34.}{Matthäus 25}.

\section{10. Abschnitt} \label{kap1_ab10}

O Christenheit!
Es ist das inbrünstige Gebet meiner Seele, dass, nach allem deinem hohen
Bekenntnisse von Christus und von seiner sanften und heiligen Religion, dein
unpassendes und dem Leben Christus so unähnliches Leben dich an jenem großen
Gerichtstage der Welt nicht verwerflich mache und zuletzt um dein ewiges Heil
bringen möge.
Höre mich daher noch ein wenig an,
ich bitte dich darum.
Kann Christus wohl dein Herr sein, wenn du ihm keinen Gehorsam leistest?
Oder kannst du dich seine Dienerin nennen, wenn du ihm gar nicht dienst? Irre
dich nicht!
"`\textit{Was du säest, das wirst du auch ernten.}"'\biblecite{Galater 6,7.}{Galater 06@Galater 6}
Er ist gewiss dein Erlöser und Heiland nicht, so lange du seine Gnade in deinem
Herzen verwirfst, durch welche er dich erlösen und selig machen will.
Sage mir, wovon hat er dich erlöst? \index{Erlösung}
Hat er dich von deinen sündlichen Lüsten, von deinen weltlichen Begierden und
von deinem eitlen Wandel erlöst?
-- Ist dieses nicht geschehen, so ist er auch nicht dein Heiland und Erlöser;
denn, obgleich er sich allen zum Erlöser und Heilande darbietet, so kann er es
doch eigentlich und in der Tat nur für diejenigen sein, die sich durch ihn
erlösen und selig machen lassen,
und es können keine von ihm selig gemacht werden, die nicht aufhören wollen, in
den sündlichen Dingen zu leben, die sie von Gott trennen, und von welchen er sie
zu erlösen in die Welt kam.

\medskip

Christus ist gekommen, die Menschen von der Sünde und vom ewigen Tode, als dem
Lohne derselben, zu erretten.
Allein diejenigen, die sich nicht durch die in ihren Seelen wirkende Kraft
Christi von der Macht und Herrschaft, welche die Sünde über sie ausübt, erlösen
oder befreien lassen, können auch von dem ewigen Tode, dem sicheren Lohne der
Sünde, in der sie leben, nie errettet werden.

\medskip

Inwiefern also die Menschen über die bösen Neigungen und fleischlichen Lüste,
denen sie ergeben waren, den Sieg erlangt haben, insofern sind sie wahrhaft
erlöst und selig gemacht und wirkliche Zeugen
\label{ref:01_10_jesus_der_erloeser}
\textbf{der Erlösung, die durch
Jesus Christus geschieht.}
Dieses wichtige Werk des Erlösers wird auch durch seinen Namen angezeigt:
\textit{"`Seinen Namen,"'} sagte der Engel des Herrn, \textit{"`sollst du Jesus
nennen; denn er wird sein Volk selig machen von \textbf{seinen
Sünden.}"'}\biblecite{Matthäus 1,21.}{Matthäus 01@Matthäus 1}
Und Johannes sagte von Christo:
\textit{"`Siehe, das ist Gottes Lamm, welches die Sünden der Welt hinweg nimmt."'}\biblecite{Johannes 1,19.}{Johannes 01@Johannes 1}
Das heißt so viel als:
\textit{betrachtet ihn, den Gott gegeben hat, die Menschen zu erleuchten und
alle diejenigen von ihren Sünden zu befreien und selig zu machen, die ihn, sein
Licht und seine Gnade, in ihren Herzen aufnehmen, täglich sein Kreuz tragen, und
ihm nachfolgen}.
Das sind solche, die lieber dem Vergnügen der Befriedigung ihrer Lüste und
Begierden entsagen, als gegen die Erkenntnis, die er ihnen von seinem Willen
gegeben hat, sündigen, oder etwas tun, wovon sie wissen, dass sie es nicht tun
sollten.







