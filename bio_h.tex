\section*{H}

\articlesize

\begin{description}

 \item[Hanson, Elizabeth] $\ast$1684; \dag1737, neuenglische Quäkerin. Bekannt
 durch ihren Erlebnisbericht über ihre Zeit in indianischer Gefangenschaft.
 Gefangen genommen wurde sie im Sommer 1724. Der Bericht erschien 1728 und
 wurde im 18. Jahrhundert in Amerika und in England mehrfach nachgedruckt.

 \item[Helmont, Franciscus Mercurius van] $\ast$um~1614, \dag1699 (in Cölln
 bei Berlin). Quäker, Alchemist, Kabbalist, Theosoph, Arzt, Rosenkreuzer.
 \endnote{"`Helmont, Franciscus Mercurius van,"',
 Biographisch-Bibliographischen Kirchenlexikon, Band XXV (2005)
 Spalten~586-597 Autor: Claus Bernet,
 \texttt{http://www.bautz.de/bbkl/h/helmont\_f\_m.shtml}}

 \item[Hicks, Elias] $\ast$19.3.1748 \dag27.2.1830. Quäker, Wanderprediger,
 Abolitionisten. An seiner Person entzündete sich ein heftiger Streit, der
 dann in den  \textit{"`Hicksites-Orthodox-Schisma"'} (oder auch
 $\to$\textit{"`great separation"'}) der Quäker mündete.
 Daraus ging dann die $\to$Hicksites und die $\to$\textit{Orthodoxe Linie} hervor.


 
 \item[Hodgkin, Thomas] $\ast$17.8.1798 , \dag5.4.1866 in Jaffa, damals
 Palästina. Britischer Quaker, Arzt und Pathologe.

\item[Hold in the Light] Im Deutschen wird gesagt "`\textit{Im Licht halten}"'.
Synonym für "`\textit{jemanden in das Gebet einschlissen}"'.

 \item[Hoover, Herbert C.] $\ast$10.8.1874 \dag20.10.1964. Quaker und der 31.
 Präsident der Vereinigten Staaten von Amerika (zwischen 1929 und 1933). 1938
 traf Hoover als erster und einziger Quäker mit Adolf Hitler zusammen.
 Allerdings ohne im geringsten die Gefahr die von den Nationalsozialisten
 ausführlichenging, richti ein zu schätzen!

 \item[Hopkins, Johns] $\ast$19.5.1795, \dag24.12.1873. US-amerikanischer Quaker,
 Geschäftsmann und Philanthrop. Mit seiner beträchtlichen Hinterlassenschaft
 wurden die Universität und das Krankenhaus gegründet, die seinen Namen tragen.

 \item[Hutchinson, Anne] $\ast$Juli 1591, \dag~August 1643. War keine Quakerin. Wurde als Anne Marbury in England geboren; heiratete mit 21 Jahren William Hutchinson und siedelte mit ihm 1634 in die britische Kolonie Massachusetts in Nordamerika um. Sie trat hervor als selbstsichere Frau, was für ihre Zeit ungewöhnlich war. Sie Vorkämpferin des Feminismus und eine Kämpferin für Menschenrechte und Toleranz. Sie lehrte, dass Gott direkt zu jedermann spräche und nicht allein zum Klerus. Sie Beeinflusste die spätere Quäkerin Mary Dyer, die sich beide persönlich kannten.


 \end{description}

\normalsize
