\chapter{Literaturverzeichnis}


\section{Deutschsprachige Texte zu William Penn}

\begin{enumerate}
\item  Ernst Bunsen: William Penn. Die Zustände Englands 1644-1718, Leipzig
1854.
 \item Georg Schmidt-Ebers: William Penn. Dramatisches Bild der Gründung
Philadelphias in vier Abtheilungen, Leipzig 1850.
 \item (Charles Evans): Eine Gedrängte Biographische Skizze von William Penn,
Philadelphia (1876).
 \item V. J. Mann: Leben und Wirken William Penn’s. Gabe zur
zweihundertjährigen
Gedächtnißfeier seiner ersten Ankunft in Pennsylvanien, Reading (PA) 1882.
 \item Otto Schnizer:
Die ersten Quäker Georg Fox und William Penn,
Calw 1907 (Calwer Famielenbibliothek, 67).
 \item Ilse Lange: Aus dem Leben des Quäkers William Penn, Berlin 1924.
 \item Alfons Paquet: William Penn, Gründer von Pennsylvanien. Ein Schauspiel,
Augsburg 1927.
 \item Anneliese Hewig: William Penn, der Gründer von Pennsylvanien
(mit zwei Bildern) in Monatshefte der deutschen Freunde, 5, 1, 1928, S.~15-22.
 \item Wolfgang Krahmer: William Penn als Kolonisator von Pennsylvania,
insbesondere seiner Politik gegenüber den Indianern, Frankfurt a.M. 1928.
 \item Elisabeth Daepp: William Penn. Hg. von der Schweizerischen Religiösen
Gesellschaft der ‘Freunde’ (Quäker), Zürich, um 1930.
 \item Elisabeth Rotten (Hrsg.): William Penns Entwurf zum gegenwärtigen und
künftigen
Frieden Europas durch Schaffung eines europäischen Parlaments,
Reichstags oder Staatenbundes. Ein Völkerbund-Entwurf des 17. Jahrhunderts,
Bern 1936.
 \item Elizabeth Janet Gray: William Penn, Bad Pyrmont 1946.
 \item Emilia Fogelklou: William Penn. Ein Buch vom Staat und vom Gewissen,
Hamburg 1948.
 \item Emilia Fogelklou: Die Stellung von James Naylor und William Penn zur
Bibel, in: Der Quäker. Monatsschrift der deutschen Freunde, 23, 11, 1949, S.
173-175.
 \item Emilia Fogelklou: William Penn. Quäker und Staatengründer, Leipzig
1963.
 \item Egon Larsen: Kein Kreuz – keine Krone. William Penn, der Gründer
Pennsylvaniens, in: Ders.: Rebellen für die Freiheit, Berlin 1963, S. 13-36.
 \item Kurt Rose: Der Sohn des Admirals. William Penn – Aufbruch in die Neue
Welt, Basel 1991.
 \item Anselm Verbeek: Ein christlicher Radikaler. Vor 325 Jahren gründete der
Quäker William Penn Pennsylvania, in: Ökumenische Information, 10, 2006, S.
14-16.
\end{enumerate}


\section{Deutschsprachige Werke von William Penn}

\begin{enumerate}
 \item Forderung der Christenheit fürs Gericht (...). Welches alles in
Englischer Sprache geschrieben ist von Wilhelm Penn, und in die Hochteutsche
Sprache treulich transferirt, Amsterdam 1678.
 \item Eine Nachricht wegen der Landschaft Pennsilvania in America, welche
jüngstens unter dem grossen Siegel in Engelland an William Penn, etc. Sambt
den Freyheiten und der Macht, so zu behöriger guten Regierung derselben nötig
übergeben worden, und zum Unterricht derer,
so etwan bereits bewogen oder noch möchten bewogen werden,
ümb sich selbsten darhin zu begeben oder einige
Bediente und Gesinde an diesen Ort zu senden, hiermit Kund gethan wird,
Amsterdam 1681. Frankfurth 1683 (2).
 \item Beschreibung der in America neu-erfundenen Provinz Pennsylvanien. Derer
Inwohner, Gesetz, Arth, Sitten und Gebrauch; auch sämtlicher Reviren des
Landes, sonderlich der Haupt-Stadt Philadelphia. Alles glaubwürdig auß des
Gouverneurs darinnen erstatteten Nachricht, Hamburg 1684.
 \item Forderung der Christenheit fürs Gericht. Sampt Einer freundlichen
Heymsuchung in der Liebe Gottes, an alle die jenige unter allerley Secten und
Religionen, welche eine Begierde und Verlangen haben nach der wahren Erkändniß
Gottes; auff daß sie ihm in der Warheit und Gerechtigkeit möchten dienen und
anbeten, sie seyen auch wie sie wollen. Wie auch Ein Sendbrieff an alle die
jenigen, die unter der Christlichen Confession, und von den äußerlichen Secten
und Gemeinten oder Kirchen angesondert sind; Und auch zuletzt Ein Sendbrieff
an alle die jenige, die von dem Tag ihrer Heymsuchung empfindlich seyn geworden,
Amsterdam 1750.
 \item Forderung der Christenheit vors Gericht: samt Einer freundlichen
Heimsuchung in der Liebe Gottes, an alle diejenigen unter allerley Secten und
Religionen, welche eine Begierde und Verlangen haben nach der Wahren
Erkenntniß
Gottes, auf Daß sie
ihm in der Wahrheit und
Gerechtigkeit möchten
dienen und
anbeten, sie seyen auch wie sie wollen. Wie auch ein Sendbrief an alle
diejenigen die unter der Christlichen Confeßion und von den äußerlichen Secten
und Gemeinen oder Kirchen abgesondert sind. Und auch zuletzt Ein Sendbrief an
alle diejenigen, die von dem Tag ihrer Heimsuchung empfindlich seyn geworden,
Philadelphia 1791.

 \item Eine Freundliche Heimsuchung in der Liebe Gottes, welche die Welt
überwindet; An alle diejenigen, die ein Verlangen haben, Gott zu kennen, und
ihn
in Wahrheit und
Aufrichtigkeit anzubeten,
von was Secte,
oder Art von
Gottesdienst dieselbigen in der ganzen (so genannten) Christenwelt seyn mögen,
und vornemlich in Hoch- und Nieder-Teutschland: Begreifend Ein klar Gezeugniß
zu
dem Alten Apostolischen Leben, Weg, und Anbetung im Geist und in der Wahrheit;
die Gott in dieser Zeit auf der Erde wiederum wird aufrichten, und lebendig
machen. Nach der
Amsterdamer Ausgabe von
1678, Philadelphia 1791.

 \item Kurze Nachricht von der Entstehung und dem Fortgang der christlichen
Gesellschaft der Freunde
die man Quäker nennt,
worin 1.~ihr Hauptgrundsatz, 2.~ihre Lehre, 3.~ihr Gottesdienst, 4. ihr Kirchendienst, 5.~ihre Zucht und 6.~ihre Verfahrensart genau beschrieben ist. Hrsg. von Ludwig Seebohm, Pyrmont 1792.
 \item Kurze Nachricht von dem Ursprunge und Fortgange der Leute die man
Quäker nennet; worin zugleich ihr Hauptgrundsatz, ihre Lehren, ihr
Gottesdienst, ihr Lehramt und ihre Kirchenzucht deutlich vor Augen gelegt werden.
Nebst einer kurzen Erzehlung der ehemaligen Gnadeneröfnung Gottes
in der Welt, welche als eine Einleitung beygefüget ist, London 1793. London 1846 (2).
 \item Früchte der Einsamkeit. In Gedanken und Maximen über den menschlichen
Lebenswandel, Tübingen 1785.
 \item Zärtlicher Besuch in der Liebe Gottes, welche die Welt überwindet, an
Alle, die nach Gerechtigkeit hungern und dürsten und Gott in Wahrheit und
Auffrichtigkeit zu erkennen und zu verstehen suchen, Friedensthal 1802.
 \item Zärtlicher Rath an die Erweckten, ein Sendschreiben, Friedensthal 1802.
 \item Schlüssel zu den Grundsätzen der Freunde, die man Quaker nennt,
Friedensthal 1802.
 \item Wiederherstellung des ersten Christenthums, Friedensthal 1802.
 \item Früchte der Einsamkeit. In zwey Abtheilungen, Friedensthal 1803.
 \item Rath und Ermahnung an die hin und her zerstreueten und verborgenen
Christenthumsbekenner, die sich von den sichtbaren Secten und Gemeinschaften
in der so genannten Christenheit äußerlich
getrennet und abgesondert haben, Friedenthal 1803.
 \item Kurze Nachricht von der Entstehung, Ausbreitung, Lehre und Kirchenzucht
der Freunde, die man Quaker nennt, Friedensthal 1804.
 \item Früchte der Einsamkeit, Heidelberg 1913.
 \item Der Völkerbundplan. Hrsg. von Margarete Rothbarth, Berlin 1920
(Monographien zum Völkerbund, 9).
 \item Religionsgesellschaft der Freunde in Deutschland e.V. (Hg.): Botschaft
von William Penn. (Gründer von Pennsylvanien). Glaube, Hoffnung und Liebe,
welche die Welt überwinden, wohne in reichem Maße unter Euch! Pfullingen 1923.
 \item Ein Essay zum gegenwärtigen und zukünftigen Frieden von Europa durch
Schaffung eines europäischen Reichstags,
1693, in: Kurt von Raumer (Hg.): Ewiger Friede: Friedensrufe und Friedenspläne seit der Renaissance, München 1953, S. 321-341 (Orbis Academicus. Geschichte der politischen Ideen in Dokumenten und Darstellungen).
 \item Essay über den gegenwärtigen und zukünftigen Frieden von Europa, 1693,
in: Walter Wimmer; Ruth Wimmer (Hg.): Friedenszeugnisse aus vier
Jahrtausenden, Leipzig 1987, S. 77-78.
\end{enumerate}