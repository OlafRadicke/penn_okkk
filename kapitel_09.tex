
\chapter{Kapitel 9} \label{kap9}


\section{Zusammenfassung des 9. Kapitel}

\begin{description}
\item[1. Abschnitt] Die dritte böse Wirkung des Stolzes ist Begierde nach Ehre
und persönlicher Achtung. Nur zu viele machen sich derselben schuldig.
\dotfill \textit{Seite~\pageref{kap9_ab1}}\\
\item[2. Abschnitt] Mordochai wäre beinahe ein Opfer dieser schädlichen
Leidenschaft geworden. Sie hat schon großen Elend in ganzen Völkerschaften
angerichtet.
\dotfill \textit{Seite~\pageref{kap9_ab2}}\\
\item[3. Abschnitt] Die Welt hat sowohl von wahrer Ehre als von wahrer
Wissenschaft ganz falsche Begriffe.
\dotfill \textit{Seite~\pageref{kap9_ab3}}\\
\item[4. Abschnitt] Gründe, warum der Verfasser und die Gesellschaft der
Christen, zu welcher er sich bekennt, die in der Welt üblichen Zeichen der
Ehrenbezeigung nicht gebrauchen.
%hier fehlte am Ende der Punkt!! Darauf hatte ich noch nicht geachtet - kannst du dir mal die
%übrigen Kapitel ansehen, ob bei der Zusammenfassung am Beginn die Punkte gesetzt sind?
\dotfill \textit{Seite~\pageref{kap9_ab4}}\\
\item[5. Abschnitt] Ihr \textbf{erster Grund} ist ein Gefühl, dass sie in der
Zeit ihrer
Erweckung von der Unverträglichkeit solcher weltlichen Gebräuche mit dem Geist
und Leben den wahren Christentums erhielten, und ihre Überzeugung, dass Stolz
und Eigenliebe denselben die erste Entstehung gaben.
\dotfill \textit{Seite~\pageref{kap9_ab5}}\\
\item[6. Abschnitt] Es konnte folglich der Tadel anderer sie nicht bewegen, die
Ausübung dieser durch Gefühl und Überzeugung erlangten Grundsätze aufzugeben.
\dotfill \textit{Seite~\pageref{kap9_ab6}}\\
\item[7. Abschnitt] Sie tun es nicht, um Sekten zu stiften oder sich
auszuzeichnen.
\dotfill \textit{Seite~\pageref{kap9_ab7}}\\
\item[8. Abschnitt] Auch nicht um Formalität zu begünstigen, sie lassen nur die
eitlen Gebräuche fahren, und verhalten sich also bei ihrer Ausschließung
derselben bloß leidendlich.
\dotfill \textit{Seite~\pageref{kap9_ab8}}\\
\item[9. Abschnitt] Ihr Betragen hierin ist gewissermaßen ein Probierstein der
Welt.
\dotfill \textit{Seite~\pageref{kap9_ab9}}\\
\item[10. Abschnitt] Auch dient dieses Kreuz, welches mit dem Betragen der Welt
im Widerspruch steht, zu ihrer eigenen Prüfung.
\dotfill \textit{Seite~\pageref{kap9_ab10}}\\
\item[11. Abschnitt] Der \textbf{zweite Grund}, warum sie jene Gebräuche
unterlassen, ist
die gänzliche Leerheit und Nichtigkeit derselben.
\dotfill \textit{Seite~\pageref{kap9_ab11}}\\
\item[12. Abschnitt] Der Ausdruck: Ehre, so wie er in der heiligen Schrift
gebraucht wird, bezeichnet einen ganz anderen Begriff, als die Welt mit demselben
verbindet. Er bedeutet gewöhnlich Gehorsam,
\dotfill \textit{Seite~\pageref{kap9_ab12}}\\
\item[13. Abschnitt] zuweilen auch Beförderung oder Erhebung.
\dotfill \textit{Seite~\pageref{kap9_ab13}}\\
\item[14. Abschnitt] Beiläufige Erklärung des Ausdrucks: \textit{Torheit}, im
Sinne
der Schrift.
\dotfill \textit{Seite~\pageref{kap9_ab14}}\\
\item[15. Abschnitt] Das Wort Ehre ist oft auch gleichbedeutend mit Ruf.
\dotfill \textit{Seite~\pageref{kap9_ab15}}\\
\item[16. Abschnitt] unter Ehre wird hinsichtlich gewisser Ämter und
Eigenschaften auch Hochachtung verstanden.
\dotfill \textit{Seite~\pageref{kap9_ab16}}\\
\item[17. Abschnitt] Ehre bedeutet auch Beistand und Gunsterweisung höherer
Personen gegen Geringeren gegenüber
\footnote{\texttt{'gegenüber' wurde hinzugefügt}}.
\dotfill \textit{Seite~\pageref{kap9_ab17}}\\
\item[18. Abschnitt]  Den Menschen Ehre erweisen heißt auch, ihnen ohne
Unterschied des Standes und der Lebensverhältnisse mit Dienstfertigkeit und
Achtung zuvorkommen: \textit{"`Ehrt Jedermann"'}
\dotfill \textit{Seite~\pageref{kap9_ab18}}\\
\item[19. Abschnitt] Doch beschränkt in gewisser Hinsicht der Verfasser der
Psalmen die Erweisung der Ehre vornehmlich auf die Gottesfürchtigen.
\dotfill \textit{Seite~\pageref{kap9_ab19}}\\
\item[20. Abschnitt] Von dieser Art Ehre wird in den Gebräuchen der Welt aber
wenig gefunden.
\dotfill \textit{Seite~\pageref{kap9_ab20}}\\
\item[21. Abschnitt] Der \textbf{dritte Grund} ihrer Unterlassung jener
weltlichen
Ehrenbezeugungen ist der, dass nicht von wahrer Ehre in ihnen enthalten ist und
folglich denen, welchen Ehre gebührt, keine dadurch erwiesen wird.
\dotfill \textit{Seite~\pageref{kap9_ab21}}\\
\item[22. Abschnitt] Der Verfasser und seine Freunde sind für wahre Ehre.
\dotfill \textit{Seite~\pageref{kap9_ab22}}\\
\item[23. Abschnitt] Ein \textbf{vierter Grund} ist, dass, wenn jene Gebräuche
wahre Ehre
enthielten, auch der niederträchtigste Mensch sie anderen erweisen könnte,
welcher jedoch unmöglich ist.
\dotfill \textit{Seite~\pageref{kap9_ab23}}\\
\item[24. Abschnitt] \textbf{Fünftens} könnten dann auch feindselige,
heuchlerische und
rachsüchtige Menschen andere ehren, welches ebensowenig sein kann.
\dotfill \textit{Seite~\pageref{kap9_ab24}}\\
\item[25. Abschnitt] Der sechste Grund ist davon abgeleitet, dass wahre Ehre
eines uralten Ursprungs ist und früher als die falsche Ehrenbezeigung bestand.
\dotfill \textit{Seite~\pageref{kap9_ab25}}\\
\item[26. Abschnitt] Der \textbf{siebente Grund} ist von der Entstehung jener
falschen
und leeren Komplimente hergenommen, welche den ehrlichen geraden Landmann, der
sie nicht zu machen versteht, von der ihm gebührenden Achtung ausschließen.
\dotfill \textit{Seite~\pageref{kap9_ab26}}\\
\item[27. Abschnitt] Der \textbf{achte Grund} gegen diese falsche Ehre ist der,
das sie
für Geld zu haben ist, welcher mit der wahren Ehre der Fall nicht sein kann.
\dotfill \textit{Seite~\pageref{kap9_ab27}}\\
\item[28. Abschnitt] Der \textbf{neunte und letzte Grund} gegen dieselbe ist
endlich der,
dass die Lehren der heiligen Schrift den Gebräuch der falschen Ehrenerweisungen
den wahren Christen ausdrücklich untersagen.
\dotfill \textit{Seite~\pageref{kap9_ab28}}\\
\item[29. Abschnitt] Das Beispiel der Mardochai.
\dotfill \textit{Seite~\pageref{kap9_ab29}}\\
\item[30. Abschnitt] Unterredung eines Bischofs mit dem Verfasser über diesen
Gegenstand.
\dotfill \textit{Seite~\pageref{kap9_ab30}}\\
\item[31. Abschnitt] Elihus Beispiel im Buch Hiob.
\dotfill \textit{Seite~\pageref{kap9_ab31}}\\
\item[32. Abschnitt] Die Lehre, welche Christus seinen Jüngern in dieser
Hinsicht gab.
\dotfill \textit{Seite~\pageref{kap9_ab32}}\\
\item[33. Abschnitt] Paulus warnt vor der Gleichstellung der Welt.
\dotfill \textit{Seite~\pageref{kap9_ab33}}\\
\item[34. Abschnitt] Petrus verlangt, dass wir uns den Lüsten der Welt nicht
gleichstellen sollen.
\dotfill \textit{Seite~\pageref{kap9_ab34}}\\
\item[35. Abschnitt] Jakobus erklärt sich gegen das Ansehen der Person
\dotfill \textit{Seite~\pageref{kap9_ab35}}\\
\item[36. Abschnitt] Dessen ungeachtet wissen wahre Christen sich bescheiden und
anständig zu betragen.
\dotfill \textit{Seite~\pageref{kap9_ab36}}\\
\item[37. Abschnitt] Doch ist ihr Betragen, seiner Natur und seinen Beweggründen
nach, von dem der Welt sehr verschieden.
\dotfill \textit{Seite~\pageref{kap9_ab37}}\\
\item[38.-40. Abschnitt] Verschiedene Zeugnisse zu Gunsten unseres Benehmens und
unserer Verweigerung, uns nach den Gebräuchen der Welt zu richten.
\dotfill \textit{Seite~\pageref{kap9_ab38},~\pageref{kap9_ab39}~und~\pageref{kap9_ab40}}\\

\end{description}

\newpage


\section{1. Abschnitt} \label{kap9_ab1}

Die dritte üble Wirkung des Stolzes\index{Stolz} ist ein unmäßiges Trachten nach
persönlicher
Ehre und Auszeichnung\index{Titel!und Austeichnungen}. Der Stolze liebt und
strebt nach Macht, damit jedermann
ihm Huldigung und Ehre erweisen soll, und wer es hieran mangeln lässt, setzt
sich seinem Zorn\index{Zorn} und seiner Rache\index{Rache} aus. Diese böse
Eigenschaft des Stolzes hat
sich mehr oder weniger unter dem ganzen verderbten Menschengeschlecht
verbreitet, und ist immer eine Veranlassung zu großer
Gehässigkeit\index{Gehässigkeit} und zu
heftigen Beleidigungen in der Welt gewesen.

\section{2. Abschnitt} \label{kap9_ab2}

Die Urkunden der heiligen Schrift liefern uns ein auffallendes Beispiel der
Bosheit und Rachsucht, deren ein vom Stolz aufgeblasener Mensch fähig ist, wenn
ihm die Befriedigung seiner Eigenliebe verweigert wird.
Mardochai\index{Personen:!Mardochai} wäre beinah
erhenkt worden, und alle Juden standen in Gefahr, ihr Leben zu verlieren, weil
er sich weigerte, vor Haman\index{Personen:!Haman}, dem Günstlinge des Königs
Ahasverus\index{Personen:!Ahasverus, König}, die Knie zu
beugen\index{Knie beugen}. Sogar die Geschichte unserer Zeiten erzählt uns von
ähnlichen Vorfällen.
So sind z. B. schon dadurch, dass ein Schiff vor gewissen Häfen oder Besatzungen
die Segel nicht strich, die Flaggen nicht senkte, oder die gewöhnliche
Begrüßung\index{Begrüßung}
unterließ, ja durch noch weit geringere Dinge, zwischen Staaten und Reichen die
furchtbarsten Kriege\index{Krieg} veranlasst worden, die ungeheure Summen Geldes
und
entsetzlich viel Menschenblut gekostet haben. Gleiche traurige Folgen hat nicht
selten der Rangstreit unter Fürsten und ihren Gesandten nach sich gezogen. Ebenso 
hat auch der Neid und heftiger Streit, der unter Privatpersonen bloß dadurch
entstand, dass der eine oder der andere sich einbildete, ihm sei von der Ehre,
die durch Entblößung des Hauptes\index{Hut!abnehmen},
Verbeugung\index{Verbeugung} des Körpers und Beilegung der Titel\index{Titel}
an den Tag gelegt wird, seinem Rang oder Stande nach zu wenig widerfahren, oft
blutigen Zweikampf\index{Zweikampf} und Mord Verursacht. Mir selbst ist einmal
in Frankreich\index{Orte:!Frankreich}
der Fall begegnet, \label{kap9_ab2_duell_penn}
\footnote{Dieses geschah, als ich noch nicht zu der
Gesellschaft gehörte, zu welcher ich mich jetzt bekenne.}
\index{Duelierung} dass ich des Nachts um
elf Uhr auf dem Wege nach meiner Wohnung von einem Menschen mit entblößtem Degen
angegriffen wurde, der Genugtuung\index{Genugtuung} von mir forderte, weil ich
seine höfliche
Begrüßung mit dem Hut\index{Hut!abnehmen} nicht erwidert hätte, wiewohl ich in
Wahrheit ihn gar
nicht bemerkt hatte. Gesetzt nun, er hätte mich erstochen, indem er verschiedene
Ausfälle auf mich tat, oder ich hätte in meiner Selbstverteidigung ihn
getötet, als ich ihn im Beisein eines Dieners, des Grafen
Crawford\index{Personen:!Crawford, Grafen},
entwaffnete, so frage ich jeden Menschen von Verstand und Gewissen, ob die ganze
Zeremonie des Begrüßens es wert war, dass ein Mensch sein Leben darüber einbüßen
sollte, wenn man die Würde seiner Natur und die Wichtigkeit seines Lebens,
sowohl in Hinsicht auf seine Abhängigkeit von Gott, seinem Schöpfer und auf
seine eigene Wohlfahrt, als auch auf die Nützlichkeit seines Daseins für die
menschliche Gesellschaft, gehörig in Erwägung zieht.

\section{3. Abschnitt} \label{kap9_ab3}

Betrachten wir den wahren Grund der Sache, so finden wir, dass die von Gott
abgewichene Welt ebensosehr in Ansehung ihrer Begriffe von wahrer Ehre und
Achtung, als in anderen Dingen ausgeartet ist. Denn alles, was die Menschen in
ihrem entarteten Zustande einander als Zeichen der Ehre und Achtung erweisen,
ist im Grunde doch nur Schein und leere Zeremonie, so dass man dieselben Worte,
deren Paulus\index{Personen:!Paulus} sich von der falschen
Wissenschaft\index{Erkenntnis!falsche Wissenschaft} bediente, auch auf sie
anwenden und von ihnen sagen kann: Es sind fälschlich sogenannte Beweise von
Ehre und Achtung, die nichts von der Eigenschaft wahrer Ehre und wahrer
Hochachtung enthalten. Und da sie zuerst von ausgearteten Menschen, die gern
geehrt sein wollten, erfunden und eingeführt wurden, so ist es auch nur der
Stolz im Menschen, der sie gern hat und verlangt, und sich beleidigt findet,
wenn sie ihm verweigert werden. Hätten aber die Menschen rechte Begriffe von
einem wahrhaft christlichen Gemütszustande und von der
\textit{"`Ehre, die von Gott ist,"'}
\footnote{Johannes 5,44.}
\index{Bibelstellen:!Johannes 5)}
die Jesus\index{Personen:!Jesus} lehrt, so würden sie solche eitle
Ehre nicht begehren und noch weniger auf ihre Erweisung bestehen.

\section{4. Abschnitt} \label{kap9_ab4}

Hier sei es mir erlaubt, die Gründe anzugeben und näher zu erklären, die mich
und die christliche Gesellschaft\index{Religiöse Gesellschaft der Freunde
(Quäker)}, mit welcher ich in religiöser Verbindung lebe,
bewogen haben, verschiedene weltliche Gebräuche und Arten der Ehrenbezeigung,
auf welche in unsern Tagen so sehr gehalten wird, als eitle und törichte Dinge
zu unterlassen. Und lass mich dich bitten, Leser, setze alle Vorurteile und alle
Verachtungen bei Seite, und lies und erwäge mit der Gelassenheit und
Unparteilichkeit eines nüchternen, forschenden Geistes, was ich hier zu unserer
Verteidigung anführe. Sollten wir denn auch irren, so bedauere und belehre uns
lieber, als dass du unsere Einfalt verachtest oder beleidigst.

\section{5. Abschnitt} \label{kap9_ab5}

\label{ref:09_05_offenbarung} \textbf{Der erste und stärkste Beweggrund, der am mächtigsten auf unsere
Gemüther
wirkte, und uns bewog, die jetzt in der Welt üblichen Gebräuche des
Hutabnehmens\index{Hut!abnehmen}, der Verbeugung des Körpers oder der Kniee, und
der Beilegung
schmeichelhafter Titel und Beinamen, in unseren Begrüßungen zu unterlassen, war
eine durch Gottes Licht\index{Gott!Gottes Licht} und Geist uns gegebene Erkenntnis}
und Empfindung des
Abfalls der Christenheit von Gott, und eine klare Einsicht in die Ursache und
Wirkungen dieser großen und beklagenswerten Abweichung. Das erste, was diese
Entdeckung bei uns hervorbrachte, war ein tiefes Gefühl unseres eigenen Zustandes.
Wir sahen ihn, den wir mit unseren Übertretungen durchstochen hatten, und wurden
darüber in Trauer und Reue\index{Reue} versetzt. Ein Tag der Demütigung
überfiel uns, und
wir fanden kein Behagen mehr an den Freuden und Genüssen, die wir einst liebten.
Nun wurden unsere Handlungen, ehe sie vollzogen wurden, genau geprüft und
untersucht, und es wurde eine strenge Selbstprüfung\index{Selbst!-prüfung}
vorgenommen. Da verstanden
wir erst recht die Worte des Propheten:
\textit{"`Wer wird den Tag seiner Zukunft
erleiden mögen? Und wer wird bestehen, wenn er erscheinen wird? Denn er ist wie
das Feuer eines Goldschmieds und wie die Seife der Wäscher."'}
\footnote{Maleachi 3,2}
\index{Bibelstellen:!Maleachi 3)}
Und was der Apostel Petrus sagt:
\textit{"`Wenn der Gerechte kaum erhalten wird, wie will der Gottlose und Sünder
erscheinen?"'}
\footnote{1. Petrus 4,18}
\index{Bibelstellen:!1. Petrus 4)}
Darum,
\textit{"`weil wir wissen, dass der Herr zu fürchten ist,"'}
(nach dem Englischen:
\textit{"`weil wir den Schrecken des Herrn kennen,"'}) wie Paulus sagt,
\textit{"`so überreden wir die Menschen."'}
\footnote{1. Petrus 4,18}
\index{Bibelstellen:!1. Petrus 4)}
Und wozu? -- Sich loszureißen von dem Wesen
und dem Geist, von den verderblichen Lüsten und eitlen Gebräuchen einer argen Welt,
eingedenk der Worte Jesu:
\textit{"`dass der Mensch am Tage des Gerichts von jedem
unnützen Worte, das er geredet hat, Rechenschaft geben muss."'}
\footnote{Matthäus 13,36}
\index{Bibelstellen:!Matthäus 13)}

\medskip

\index{Selbst!-erkenntnis}Die großen Gemütsübungen, unter denen wir uns befanden,
und die
Niedergeschlagenheit unserer Geister fiel unseren Nachbarn und Bekannten bald
auf, und wir schämten uns auch nicht, zu gestehen, dass die Furcht des Herrn uns
so mächtig ergriffen hatte, weil wir so lange unter einem Bekenntnisse von
Religion den heiligen Geist Gottes betrübt hatten, der uns wegen unseres
Ungehorsams insgeheim bestrafte\index{Gebote!Bestrafung, innere}. Da wir in diesem
Zustand der Zerknirschung
und Reue den Gedanken, in unseren alten Sünden zu beharren, verabscheuten, so
fürchteten wir uns sogar, erlaubte Dinge zu gebrauchen, weil wir besorgt waren,
dass wir sie missbrauchen möchten. So erfuhren wir die Erfüllung der Worte des
Propheten:
"`Wie geht es denn, dass ich alle Männer sehe, ihre Hände auf ihren
Hüften haben, wie Frauen bei der Geburt, und dass alle Angesichter so bleich
sind?"'
\footnote{Jeremia 20,6}
\index{Bibelstellen:!Jeremia 20)}
Angst und Wehe hatten uns in der Tat ergriffen.
Unser Himmel schien zu zerschmelzen und unsere Erde aus ihren Angeln gehoben,
wir waren wie Menschen, \textit{"`auf welche,"'} wie der Apostel sagt,
\textit{"`das Ende der Welt gekommen war."'}\index{Welt!Ende der} Gott weiß es,
dass es in jenen Tagen so mit uns war. Das
Licht der Erscheinung Christi in unserne Herzen entdeckte jede Pflanze, die er
nicht selbst in uns gepflanzt hatte, und der Atem seines Mundes zerstörte sie.
Er war ein schneller Zeuge gegen jeden unreinen Gedanken\index{Gedanken!unreine}, gegen jedes
unfruchtbare Werk, und, gepriesen sei sein Name, wir ärgerten uns nicht an ihm
oder an seinen Gerichten\index{Gericht!Christi}. Da war es, dass wir über unser
ganzes Leben eine
strenge Prüfung\index{Selbst!-prüfung} anstellten, dass jeder
Gedanke\index{Gedanken}, jedes Wort, jede Handlung dem
Gericht unterworfen, in der Wurzel untersucht und in Hinsicht auf Zweck und
Folgen erwogen wurde.
\textit{"`Augenlust, Fleischeslust und hoffärtiges Leben,"'}
\footnote{1. Johannes 2,16}
\index{Bibelstellen:!1. Johannes 2)}
diese Grundpfeiler des Geheimnisses der Bosheit
in uns, waren den Augen unserer Seelen enthüllt. Unsere eigene Kenntnis des
bösen Sauerteigs\index{Sauerteig}, unsere eigenen Erfahrungen von seinen
verschiedenen
verderblichen Wirkungen und von dem, was er hervorgebracht und wie er gewirkt
hatte, gaben uns eine Einsicht in den Zustand anderer. Und was wir nun in uns
selbst nicht billigen konnten, ja, nicht durften leben und fortwiken lassen,
weil
wir offenbar einsahen, dass es in der Nacht des Abfalles aus einer bösen Wurzel
entsprungen war, das konnten wir auch in anderen nicht gut heißen und nähren.
Daher glaube ich in der Furcht und Gegenwart des allsehenden gerechten Gottes
behaupten zu können, dass, unter vielen andern Dingen, uns auch die gegenwärtig
in der Welt üblichen Ehrenbezeigungen zur Last geworden sind, da wir sie als
Früchte erkannt haben, die nicht im Garten Gottes\index{Gott!Garten Gottes}
gewachsen, sondern in der Zeit
der Finsternis\index{Finsternis!Zeit der} aus einer verderbten Wurzel entsprossen sind, die bloß dazu
dienen, dass sie den Stolz und die Torheit des Menschen nähren, und folglich nur
eitlen Gemüthern gefallen können.

\section{6. Abschnitt} \label{kap9_ab6}

Obgleich wir leicht voraussehen konnten, was für Stürme über uns ausbrechen und
welchen Vorwürfen wir uns aussetzen würden, wenn wir uns weigerten, jene
weltlichen Gebräuche noch länger zu beobachten, so waren wir doch weit entfernt,
uns dadurch in unserer Überzeugung wankend machen zu lassen, indem das
Betragen, welches die weltlich gesinnten Menschen gegen uns an den Tag legten,
uns noch mehr darin bestärkte. Denn wir fanden sie so voll hoher Begriffe von
sich selbst und so begierig nach Ehrenbezeigungen von ihren Mitmenschen, dass
sie, sobald wir aus zarter Gewissenhaftigkeit gegen Gott ihnen dieselben nicht
mehr, wie zuvor, erweisen konnten, um diese Dinge sich weit mehr, als um alle
anderen Stücke unseres christlichen Bekenntnisses bekümmerten, so verschieden von
dem ihrigen und so wichtig in Hinsicht auf das Heil der Seele sie auch sein
mochten. \label{ref:09_06_grundsaetze}
\textbf{Was auch immer die Ehre Gottes und unsere eigene Seligkeit
betreffen möchte, so war es in ihren Augen größere Ketzerei\index{Personen:!Ketzer} und Lästerung,
vor ihnen den
Hut nicht abzuziehen und ihnen ihre gewöhnlichen Ehrentitel nicht zu geben, oder
sich zu weigern, auf ihre Gesundheit zu trinken\index{Alkohol!trinken}, oder in
ein Karten- und
Würfelspiel\index{Spiele} sich mit ihnen einzulassen, als
unsere übrigen Grundsätze zu
behaupten, die, weil sie ihnen weniger auffielen, ihnen auch nicht so wichtig
und hinderlich zu sein schienen.}

\section{7. Abschnitt} \label{kap9_ab7}

\label{ref:09_07_vorwurff} \textbf{Man hat uns oft vorgeworfen, es sei
uns nur darum zu tun, die Beobachtung
gewisser äußerer Formen einzuführen, und unser schlichtes Betragen sei gleichsam
nur das farbige Band oder Kennzeichen der Partei, wodurch wir uns auszuzeichnen
suchten.} Allein ich erkläre in der Furcht des allmächtigen Gottes, dass dieses
nur Einbildungen und Auslegungen unempfindlicher Menschen sind, die das uns von
dem Herrn verliehene Gefühl von dem, was aus einer guten und bösen Wurzel im
menschlichen Herzen entspringt, nicht kennen. Wenn aber solche Tadler unserer
einfachen Sitte einst von der mächtigen Kraft Gottes in ihrem Inneren ergriffen
und erwerkt sein werden, und diese Dinge ihrem Ursprung und ihrer wahren Natur
und Eigenschaft nach beurteilen lernen, dann werden sie ihre eigene Last fühlen
und uns von der Torheit oder Heuchelei\index{Heuchelei!der Quäker}, der sie uns
beschuldigten, gern freisprechen.

\section{8. Abschnitt} \label{kap9_ab8}

\label{ref:09_08_vorwurff} \textbf{Auf den Vorwurf, den man uns macht, dass wir es 
in geringen Dingen sehr
genau
nehmen,} welches doch Leuten von so hohen Ansprüchen auf Freiheit des Geistes
nicht anstehe, antworte ich mit Gelassenheit, Wahrheit und Mäßigung, erstlich:
\textbf{Nichts ist gering, dessen Verrichtung oder Unterlassung Gott uns zur
Gewissenssache macht. Denn haben auch selbst unsere Gegner diesen Dingen, die
sie als unbedeutend betrachtet wissen wollen, und deren Geringfügigkeit sie uns
vorwerfen, schon oft ein so großes Gewicht beigelegt, dass sie uns, ihrer
Unterlassung wegen, geschlagen, eingekerkert, Gerechtigkeit versagt haben;
u.s.w.,} des Spottes und Tadels, womit man uns nicht selten deshalb
überhäuft
hat, nicht zu erwähnen. Ja, hätte es uns an Beweisen für die Wahrheit unserer
innern Überzeugung in Betreff dieser Stücke gemangelt, so würde in der Tat
das Betragen derer, die sich uns darin widersezten, uns überflüssig damit
versehen haben. Indessen muss es uns genügen,
\textit{"`dass die Weisheit von ihren Kindern gerechtfertigt wird."'}
\footnote{Matthäus 11,19}
\index{Bibelstellen:!Matthäus 11)}
Wir entsagen dem Gebrauch
der Dinge, von deren Eitelkeit und Nichtübereinstimmung mit dem wahren
Christentum wir überzeugt sind, und indem wir so gegen leere Formen und bloß
verneinend oder verweigernd verhalten, legen wir sie ab, führen aber keine neuen
dafür ein.

\section{9. Abschnitt} \label{kap9_ab9}

Die Welt hängt so sehr an Zeremonien und an der Außenseite der Dinge, dass es
deshalb wohl der göttlichen Weisheit in allen Zeitaltern gefallen hat, mit ihren
Gnadenausteilungen an die Menschen allezeit unter einer Gestalt hervorzutreten,
die ihren eingeführten Gebräuchen entgegen war, um dadurch sowohl den
menschlichen Erfindungen zu widersprechen, als auch die Aufrichtigkeit der
Anhänger und Bekenner der Wahrheit zu prüfen\index{Prüfung!des Glaubens}. Ja, ihre
gegenwärtige Erscheinung
dient auch der Welt zu einem Prüfstein. Sie zeigt, wieviel Gelassenheit,
Wohlwollen, Uneinvorgenommenheit und Mäßigung die Menschen besitzen, und wenn sie
an dem einfachen unansehnlichen Äußeren der Wahrheit, deren Schönheit innerlich
ist, sich nicht so sehr stossen, dass sie ihr den Eingang in ihre Herzen
verschließen, so kann dieses ihnen wichtige Aufschlüsse gewähren. Denn wer einen
kostbaren Edelstein darum ausschlägt, weil er ihm in einem einfachen
Kästchen dargeboten wird, der zeigt, dass er ihn nicht nach seinem wahren Wert
schätzt, und wird sich auch aus seinem Besitz nichts machen. Darum nenne ich die
einfache Sitte, zu welcher die Wahrheit ihre Bekenner in unseren Tagen anleitet,
einen \textit{Prüfstein}, weil dadurch offenbar wird, woran die Herzen und
Neigungen
der Menschen, bei allen ihren hohen Ansprüchen auf bessere Einsichten, dennoch
hängen.

\section{10. Abschnitt} \label{kap9_ab10}

\label{ref:09_10_spott} \textbf{Für dass Volk Gottes ist dieses einfache Betragen in der Tat 
eine große
Prüfung\index{Prüfung!des Glaubens},
weil es dadurch in die Notwendigkeit versetzt wird, seine Nichtübereinstimmung
mit den so allgemein üblich gewordenen und hochgeachteten Gebräuchen der Welt
öffentlich an den Tag zu legen, und sich dem Anstaunen, der Vorspottung und
Beleidigung der unwissenden Menge auszusetzen.} Es liegt aber auch ein
verborgener Schatz darin. \textbf{Es gewöhnt uns, Schmach und Verachtung zu
ertragen,
es lehrt uns, den falschen Ruhm\index{Ruhm!falscher} der Welt zu verachten,
stillschweigend die
Widersprüche und verächtliche Begegnung ihrer Verehrer zu erdulden, und endlich
mit christlicher Geduld und Sanftmut ihre Vorwürfe und Beleidigungen zu
besiegen.} Fügen wir noch hinzu, dass dein unweltliches Betragen dich von dem
Umgang mit deinen weltlich gesinnten Bekannten entwöhnt\index{Welt!Absonderung von}, wenn 
du dich als einen
Schwachkopf, Toren, Schwärmer u.s.w. von ihnen zurückgesetzt siehst, und dass
du eben dadurch vor einer gefährlichen Versuchung, vor dem mächtigen Einfluss
ihres schädlichen Beispiels, bewahrt wirst. Und endlich wirst du dadurch in
die Gesellschaft der Streiter des glorreichen, wiewohl verspotteten und
verfolgten Jesu\index{Verfolgung!Jesu} eingeführt, \textbf{um unter seiner Fahne
gegen Welt, Fleisch und Teufel\index{Personen:!Teufel}
zu kämpfen, damit du, nachdem du im Stande oder Erniedrigung und Demütigung
standhaft mit ihm gelitten hast, auch im Stande der Herrlichkeit mit ihm
regieren mögest, der seine armen, verachteten, aber getreuen Nachfolger
\textit{"`mit der Herrlichkeit verklärt, die er bei dem Vater hatte, ehe die
Welt
war."'}
\footnote{Johannes 17,5.}
\index{Bibelstellen:!Johannes 17)}
Dieses war nun der erste Grund, der uns bewog, den
Gebrauch weltlicher Ehrenbezeigungen abzuschaffen.}

\section{11. Abschnitt} \label{kap9_ab11}

\textbf{Unser zweiter Beweggrund} zur Unterlassung und Verweigerung der jetzt
üblichen
Art, einander Ehre oder Achtung in Anreden, Begrüßungen etc., zu erweisen,
entsprang aus der Erwägung der gänzliehen Leerheit und Eitelkeit dieser
Gebräuche, die, wenn wir auch annehmen, dass sie an sich nicht böse sind, doch
nichts von wahrer Ehre und Achtung enthalten. Denn so, wie Religion und
Gottesverehrung im Allgemeinen so sehr ausgeartet sind, dass sie mit der in der
\textbf{ersten Kirche}\index{Kirche!erste} gebräuchlichen Ausübung derselben
fast keine Ähnlichkeit mehr
haben, so ist es auch mit den Erweisungen wahrer Ehre und Achtung der Fall. Es
ist sowohl von jenen als von diesen nur wenig in der Welt anzutreffen, und von
den Gebräuchen, deren man sich jetzt bedient, um Ehre und Achtung an den Tag zu
legen, lässt sich keiner, weder nach der Vernunft noch nach der Schrift,
rechtfertigen.

\section{12. Abschnitt} \label{kap9_ab12}

\label{ref:09_12_ehre} \index{Ehre!Bedutung in der Bibel} \textbf{Das Wort Ehre wird in der Schrift
häufig und in verschiedener Bedeutung
gebraucht. Erstlich drückt es hier Gehorsam aus}\index{Gehorsam}, z. B. wenn
Gott sagt:
\textit{"`Wer mich ehrt,"'}
\footnote{1. Samuel 2,30.}
\index{Bibelstellen:!1. Samuel 2)}
das heißt: Wer meine Gebote hält. Ferner:
\textit{"`Ehrt den König,"'}
\footnote{1. Petrus 2,17}
\index{Bibelstellen:!1. Petrus 2)}
das heißt: Gehorcht dem König.
\textit{"`Ehre deinen Vater und deine Mutter!"'}
\footnote{2. Mose 20,12}
\index{Bibelstellen:!2. Mose 20)}
d. h. wie der
Apostel zu den Ephsesern sagt:
\textit{"`Ihr Kinder seid gehorsam eueren Eltern in dem
Herrn, denn das ist billig."'}
\footnote{Epheser 6,1+2.}
\index{Bibelstellen:!Epheser 6)}
Beachtet ihre Lehren und
ihren Rat! Vorausgesetzt nämlich, dass Herren und Eltern nur das befehlen, was
recht ist, weil sie sonst, wenn sie etwas Unerlaubtes verlangten, sich selbst
entehren würden, so wie Untertanen und Kinder ihre Vorgesetzten und Eltern
entehren, wenn sie ihnen in ungerechten Dingen gehorchen. In diesem Sinne
gebraucht auch Christus das Wort Ehre, wenn er sagt:
\textit{"`Ich habe keinen Teufel,
sondern ich ehre meinen Vater, und ihr verunehrt mich."'}
\footnote{Johannes 8,49.}
\index{Bibelstellen:!Johannes 8)}
d. h.: Ich tue den Willen meines Vaters in allem, was ich tue, ihr aber wollt
mich nicht hören, ihr verwerft meinen Rat, und wollt meiner Stimme nicht
gehorchen. Es war nicht Verweigerung des Hutabnehmens, des Kniebeugens oder der
leeren Titel, der er sie beschuldigte, nein, ihr Ungehorsam war es, den er ihnen
vorwarf, weil sie sich ihm, den Gott gesandt hatte, widersetzten und nicht an
ihn glaubten. Darin besstand die Unehre, die sie ihm erwiesen, indem sie ihn,
den Gott zum Heile der Welt verordnet hatte, als einen Betrüger behandelten. Und
leider gibt es auch in unseren Tagen nur zu viele, die ihn so entehren! Ferner
sagt Christus in demselben Sinne:
\textit{"`Damit sie alle den Sohn ehren, wie sie den
Vater ehren. Wer den Sohn nicht ehrt, der ehrt auch den Vater nicht, der ihn
gesandt hat."'}
\footnote{Johannes 5,23.}\index{Jesus!ehren}
\index{Bibelstellen:!Johannes 5)}
D. h.: Diejenigen, die Christo kein Gehör
geben, ihn nicht göttlich verehren und anbeten, und ihm nicht Gehorsam leisten,
die hören, verehren und gehorchen auch Gott nicht. Da sie vorgaben, an Gott zu
glauben, so hätten sie auch an ihn glauben müssen, wie er ihnen deutlich sagte.
Dieser Begriff von Ehre leuchtet auch sehr klar aus dem Beispiel des
Hauptmanns ein, dessen Glauben Christus so sehr lobte. Denn als derselbe unserem
Heiland eine Schilderung seiner Ehrenstellen
 gab, sagte er zu ihm:
\textit{"`Ich habe
Kriegesknechte unter mir, und wenn ich zu einem sage: Gehe hin, so geht er, und
zu einem andern: Komm her, so kommt er, und zu meinem Knechte: Tue das, so tut
er’s."'}
\footnote{Lukas 7,8.}
\index{Bibelstellen:!Lukas 7)}
Hierin setzte er die Ehre seines Postens und die
Achtung, die seine Soldaten\index{Personen:!Soldaten} ihm erwiesen, aber nicht in
Entblößung des Hauptes
und Beugung des Körpers. Auch sind solche Gebräuche bis jetzt unter den Soldaten
noch nicht üblich geworden, da sie weibisch\index{weibisch} und des männlichen
Ernstes unwürdig
sind.

\section{13. Abschnitt} \label{kap9_ab13}

Ferner wird das Wort Ehre in der Schrift gebraucht, um Beförderung zu einem
mächtigen und erhobenen Amt auszudrücken. So gebraucht es
David\index{Personen:!David}, wenn er von
Christo zu Gott sagt:
\textbf{"`Mit Ehre und Schmuck hast du ihn gekrönt, er hat große
Ehre an deiner Hilfel"'}
\footnote{Psalm 8,6.) Psalm 21,6.}
\index{Bibelstellen:!Psalm 8)}
\index{Bibelstellen:!Psalm 21)}
(oder an deinem Heil!) D.h.: Gott hat Christo Gewalt über alle seine Feinde
gegeben und ihn zu großer
Herrschaft erhoben. Salomo \index{Personen:!Salomo} sagt:
"`Die Furcht des Herrn ist Zucht zur Weirheit,
und ehe man zu Ehren kommt, muss man zuvor leiden."'
\footnote{Weisheit Salomo 15,33.}
\index{Bibelstellen:!Weisheit Salomo 15)}
D. h.: Ehe man befördert wird, muss man zuvor Demütigungen\index{Demütigungen}
erfahren. Und an
einem anderen Orte sagt er:
\textit{"`Wie der Schnee im Sommer und der Regen in der
Ernte, so reimt sich dem Narren Ehre nicht."'}
\footnote{Weisheit Salomo 26,1.}
\index{Bibelstellen:!Weisheit Salomo 26)}
Ein Narr ist
nämlich keines Vertauens fähig und folglich keiner Beförderung und Erhebung
würdig. Dazu sind Tugend, Weisheit, Rechtschaffenheit und Fleiß erforderlich,
die ein Narr nicht besitzt. Und sollen nun schmeichelhafte Titel und andere
jetzt übliche Zeichen der Ehrenbezeigung für wirkliche Ehre gelten, so muss
Salomos  Sprichwort unfehlbar zutreffen, wie es denn auch in unserem Zeitalter
häufig genug der Fall ist, wenn dergleichen Ehrerweisungen auf Leute
verschwendet werden, die Salomos Narren ähnlich sind, indem sie sich
\textit{"`der Zucht der Weisheit entziehen"'}
\footnote{Weisheit Salomos 13,18}
\index{Bibelstellen:!Weisheit Salomos 13)}
und die Furcht der
Herrn hassen, die doch nur allein den Menschen zu einem seiner Weisen machen
kann.

\section{14. Abschnitt} \label{kap9_ab14}

So wie Tugend und Weisheit im Grunde eines sind, haben auch Narrheit oder
Torheit und Gottlosigkeit oder Bosheit einerlei Bedeutung. So wird z.B.
Sichems\index{Personen:!Sichems} Schändung der Dina\index{Personen:!Dina}, einer
Tochter Jakobs\index{Personen:!Jakobs}, und die Empörung und
Gottlosigkeit der Israeliten\index{Personen:!Israeliten} in der Schrift Narrheit
und Torheit
genannt.
\footnote{1. Mose 34,7.) Josua 7,15}
\index{Bibelstellen:!1. Mose 34)}
\index{Bibelstellen:!Josua 7)}
David\index{Personen:!David} sagte,
\textit{"`seine Wunden stanken
und eiterten von seiner Torheit,"'}
\footnote{Psalm 38,6.}
\index{Bibelstellen:!Psalm 38)}
d. h. von seinen Sünden,
und an einem anderen Orte:
\textit{"`Ach dass der Herr seinem Volk und seinen Heiligen
Frieden zusagt, damit sie nicht wieder auf eine Torheit
geraten!"'}
\footnote{Psalm 85,9.}
\index{Bibelstellen:!Psalm 85)}
d. h. nicht wieder Böses tun.
\textit{"`Die Missetat des Gottlosen,"'}
sagt Salomo,
\textit{"`wird ihn fangen, und er wird mit den Stricken
seiner Sünde\index{Sünde} gehalten werden. Er wird sterben, weil er sich nicht
will ziehen lassen, und um seiner großen Torheit willen wird es ihm nicht wohl
gehen."'}
\footnote{Weisheit Salomo 5,22+23.}\footnote{\texttt{In der Schlachter Übersetzung:} \textit{"`Den Gottlosen nehmen seine eigenen Missetaten gefangen, und von den Stricken seiner Sünde wird er festgehalten. Er stirbt an Zuchtlosigkeit, und infolge seiner großen Torheit taumelt er dahin."' }}
Und Christus setzt Torheit oder
Unvernunft mit Gotteslästerung, Stolz, Diebstahl, Mord, Ehebruch, Bosheit u.s.w.
in eine Klasse. -- Ich führe diese Schriftsteller hier an, um den
Unterschied zu zeigen, der zwischen dem Sinne des heiligen Geistes und den
Begriffen, die man in den ersten Zeitaltern von Narrheit und Torheit und von
einem Narren hatte, der keine Ehre verdiente, und zwischen den Sinnen statt
findet, den man in unseren Tagen mit diesen Ausdrücken gewöhnlich verbindet,
damit uns das Missverhältnis zwischen der Ehre, welche der heilige Geist und
diejenigen, die durch ihn geleitet wurden, unter diesem Ausdruck verstanden,
und zwischen den Gebräuchen, die gegenwärtig unter den Bekennern der
christlichen Religion als Ehrerweisungen üblich sind, desto klarer einleuchten
möge.

\section{15. Abschnitt} \label{kap9_ab15}

Unter dem Worte Ehre wird ferner auch guter Ruf verstanden, und in diesem Sinne
nehmen wir es gleichfalls. Salomo sagt:
\textit{"`Ein holdseliges Weib erhalt die Ehre."'}
\footnote{Weisheit Salomos 11,16}
\index{Bibelstellen:!Weisheit Salomo 11)}
Das heißt: Sie bewahrt ihren guten Namen und
behauptet durch ihre Tugend den Ruf von ihrer Bescheidenheit und Keuschheit. An
einem anderen Ort sagt er:
\textit{"`Es ist dem Mann eine Ehre, vom Hader zu bleiben,"'}\footnote{\texttt{Ein 'Hader' ist ein Streit. In der Schlachter-Übersetzung heißt es:}  \textit{Abzulassen vom Streit ist für den Mann eine Ehre, jeder Narr aber stürzt sich hinein.}}
\footnote{Weisheit Salomo 20,3.}
\index{Bibelstellen:!Weisheit Salomo 20)}
d. h. es erwirbt ihm den Ruf eines weisen und
guten Mannes. Christus drückt denselben Sinn des Wortes aus, wenn er sagt:
\textit{"`Ein Prophet gilt nirgend weniger als in seinem Vaterlande."'}
\footnote{Matthäus 13,57}
\index{Bibelstellen:!Matthäus 13)}
(Nach dem Englischen:)\textit{"`Ein Prophet ist nicht ohne Ehre, außer in seinem
Vaterlande."'}) D. h. er genießt überall Zutrauen und Achtung, am wenigsten aber
zu
Hause. Paulus schrieb den Thessalonichern:
\textit{"`Ein jeder unter euch wisse sein
Gefäß, (seinen Leib) in Heiligung und Ehre zu erhalten,"'}
\footnote{1. Thessalonicher 4,4.}
\index{Bibelstellen:!1. Thessalonicher 4)}
d. h. in Keuschheit und Mäßigkeit. -- In allen diesen Stellen sind nun die
von uns abgelegten Gebräuche der Ehrenbezeigung nicht allein gar nicht
berücksichtigt, sondern vielmehr von ihrem Sinne gänzlich ausgeschlossen.

\section{16. Abschnitt} \label{kap9_ab16}

Auch wird in der Schrift das Wort Ehre in Hinsicht auf Ämter, Würden und
Eingenschaften angewendet. Z. B. wenn Paulus sagt:
\textit{"`Die Ältesten, die wohl
vorstehen, halte man zweifacher Ehre wert."'}
\footnote{1. Timotheus 5,17.}
\index{Bibelstellen:!1. Timotheus 5)}
D. h.: Sie
verdienen doppelte Achtung, Liebe und Wertschätzung, wenn sie heilig,
barmherzig, mäßig, friedsam, demütig u.s.w. sind, vornehmlich diejenigen, die
am Worte und in der Lehre arbeiten \index{Predigen}\index{Lehren}. In dieser
Hinsicht empfahl Paulus den
Epaphroditus an die Philipper:
\textit{"`So nehmt ihn nun auf in dem Herrn,"'} sagt er,
\textit{"`mit allen Freuden, und haltet solche in Ehren."'}
\footnote{Philipper 2,29}
\index{Bibelstellen:!Philipper 2)}
Damit
wollte er sagen: Schätzt und achtet solche in allem, was sie sagen und lehren,
welches die wahreste, natürlichste und überzeugendste Art ist, wirkliche Achtung
für einen Mann Gottes an den Tag zu legen, die auch Christus seinen Jüngern in
den Worten empfahl:
\textit{"`Liebt ihr mich, so haltet meine Gebote."'}
\footnote{Johannes 14,15.}
\index{Bibelstellen:!Johannes 14)}
Auch gebietet uns der Apostel,
\textit{"`die wahren Witwen zu ehren."'} Damit
sagt er, dass solche Frauen, die ein keusches Leben führen und durch Tugenden
sich auszeichnen, ehrwürdig und achtungsswert sind. So ist auch die Ehe ein
ehrwürdiger Stand, vorausgesetzt,\index{Ehe!Treue}
\textit{"`dass das Ehebett unbefleckt sei."'}
\footnote{Hebräer 13,4.}
\index{Bibelstellen:!Hebräer 13)}
Denn die Ehre des Ehestandes besteht in dem keuschen
Leben der Verehelichten.

\section{17. Abschnitt} \label{kap9_ab17}

Noch wird in der Schrift das Wort Ehre von höheren Personen auf niedrigere
angewendet, wie es aus der Frage des Ahasverus\index{Personen:!Ahasverus, König} an
Haman\index{Personen:!Haman} hervorgehet:
\textit{"`Was soll man
dem Mann tun, den der König gern ehren wollte?"'}
\footnote{Epheser 6,6.}
\index{Bibelstellen:!Epheser 6)}
Der König hatte ihn mächtig befördert und erhoben, wie es hernach auch Mardochai
tat. Und von den Juden\index{Personen:!Juden} lesen wir besonders,
\textit{"`dass ihnen Licht und Freude,
Wonne und Ehre widerfahren sei."'}
\footnote{Epheser 8,16.}
\index{Bibelstellen:!Epheser 8)}
Sie entgingen nämlich der
Verfolgung, die ihnen bevorstand, und durch die Vermittelung der
Esther\index{Personen:!Esther} und des
Mardochai\index{Personen:!Mardochai} genossen sie nicht nur Ruhe und Frieden,
sondern auch Wohlwollen und
Begünstigung. In diesem Sinne ermahnt auch Petrus die christlichen Ehemänner,
\textit{"`ihre Weiber zu ehren,"'}\index{Personen:!Frauen!ehren} das heißt, sie zu lieben,
zu schätzen, zu achten, zu
begünstigen und zu beschützen wegen ihrer Treue und Zuneigung gegen ihre Männer,
wegen der zärtlichen Pflege ihrer Kinder, und wegen ihres Fleißes und ihrer
Umsicht in ihren Familien um diese Ehre auszudrücken, bedarf es aber keiner
zeremoniellen Beobachtungen oder prächtigen Titel. -- So ehret auch Gott die
frommen und guten Menschen:
\textit{"`Wer mich ehrt, spricht der Herr, den will ich
auch ehren, wer mich verachtet, der soll wieder verachtet werden."'}
\footnote{1. Samuel 2,30.}
\index{Bibelstellen:!1. Samuel 2)}
Das heißt, ich will denen, die mich ehren und mir gehorchen, Gutes
erzeigen, ich will sie lieben, segnen, erhalten und beschützen, aber diejenigen,
die mich verachten, die meinem Geist widerstreben und mein Gesetz in ihren
Herzen übertreten\index{Gebote!im Herzen übertreten}, die sollen verachtet und
gering geschätzt werden, und weder
bei Gott noch bei gerechten Menschen in Gunst stehen. Auch geschieht es oft
noch unter den Menschen, dass, wenn ein Großer einen Armen besucht, sich um ihn
bekümmert und ihm hilft, der Letztere zu sagen pflegt, dieser große Mann habe
ihm die Ehre erzeigt, ihn zu besuchen und ihm in seiner Not zu helfen.

\section{18. Abschnitt} \label{kap9_ab18}

Ich will diese Beweise mit noch einer Schriftstelle schließen, die ebenso
vielumfassend als klar und passend ist. Petrus liefert sie uns in den
wenigen Worten:
\textit{"`Ehrt jedermann, liebt die Brüder!"'}
\footnote{1. Petrus 2,17.}
\index{Bibelstellen:!1. Petrus 2)}
das heißt, die Liebe, welche die Ehre noch übertrifft, soll sich vorzugsweise auf die
Brüder erstrecken, \label{ref:09_18_ehre}
\textbf{die Ehre aber, die in Achtung und Wertschätzung besteht,
bist du allen Menschen schuldig. Und bist du sie allen schuldig, so musst du sie
auch denen erweisen, die geringer sind als du.\index{Gleichheit}} Aber warum
ist es deine Pflicht,
alle Menschen zu ehren? -- Weil sie der edelste Teil der ganzen Schöpfung
Gottes, deine Mitmenschen und folglich deines eigenen Geschlechtes sind. Darum
hege natürliche Gefühle für sie, sei mitleidig, und hilf ihnen, wo du kannst,
zeige dich allezeit bereitwillig, ihnen wahre Achtung zu erweisen, und lass ihnen
alles Gute, und jede Unterstüzung, die in deiner Macht steht, widerfahren.

\section{19. Abschnitt} \label{kap9_ab19}

\index{Achtung!nicht erweisen} \label{ref:09_19_ehre}
\textbf{Doch scheint eine Beschränkung dieses Gebotes:
\textit{"`Jedermann zu ehren,"'} in jenen
Worten des gottseligen Davids zu liegen}, wo er sagt:
\textit{"`Herr, wer wird 
in deiner Hütte wohnen? Wer wird auf deinem heiligen Berge bleiben? -- Wer die
Gottlosen nicht achtet, sondern die Gottesfürchtigen ehrt."'}
\footnote{Psalm 15,1-4}
\index{Bibelstellen:!Psalm 15)}
\textbf{Hier werden nur diejenigen, die Gott von Herzen fürchten und lieben, der
Ehre würdig gehalten, und die Gottlosen\index{Personen:!Gottlosen nicht ehren} hingegen,
die Gott und sein Gesetz
sowohl in ihren eigenen Herzen als auch in anderen verachten, als Gegenstände der
Unehre betrachtet, die nur Geringachtung verdienen.}

\medskip

Zum Beschluss dieser Untersuchung des Sinnes, der dem Worte Ehre in der Schrift
beigelegt wird, will ich noch bemerken, dass aus allem, was davon gesagt ist,
\textbf{drei Hauptbegriffe von Ehre hervorgehen, nach welchen erstlich die Ehre,
die wir
unsern Oberen oder Vorgesetzten schuldig sind, in Gehorsam, zweitens die, welche
wir unseres gleichen zu erweisen haben, in Liebe und Achtung, und drittens
diejenige, die wir gegen Geringere oder gegen unsere Untergebenen beobachten
müssen, in Unterstützung, Schutz und Hilfe besteht.} Dieses ist die wahre Ehre,
die mit dem Willen Gottes übereinstimmt, und unter seinem Volk vor Zeiten
gebräuchlich war.

\section{20. Abschnitt} \label{kap9_ab20}

Wie wenig aber nun allen diesem in der leeren Zeremonie des Hutabnehmens, in
der Beugung des Rückens oder der Kniee, und in der Beilegung schmeichelhafter
Titel besteht, lasse ich den Wahrheit sprechendern Zeugen Gottes in jedem
unvoreingenommenen Herzen beurteilen. Denn ich darf mich in Ansehung des Wertes
oder Unwertes dieser Dinge nicht auf die Entscheidung des stolzen, von seiner
Eigenliebe bestochenen, selbstsüchtigen Menschen berufen, da dieser, so wenig er
auch geneigt ist, dergleichen Ehrenbezeigungen anderen zu erweisen, sie selbst so
sehr liebt und sucht, dass er vor übler Laune oder Zorn außer sich gerät, sobald
sie ihm verweigert werden. -- Demnach ist \label{ref:09_20_zeiter_grund}
\textbf{unser zweiter Grund, warum wir die in
der Welt als Zeichen von Achtung und Ehre gebräuchlichen Zeremonien abgeschafft
haben, dieser: Weil wir in der ganzen Schrift der Wahrheit nicht finden, dass
jemals eine derselben vom Geist Gottes geboten oder empfohlen worden ist.}

\section{21. Abschnitt} \label{kap9_ab21}

Der dritte Grund unserer Unterlassung solcher Ehrenbezeigungen ist der, dass
dadurch kein wahres Gefühl von Achtung und Ehre an den Tag gelegt werden kann,
weil sie an sich zweideutig sind, und oft dazu gebraucht werden, die Menschen zu
täuschen und die ihnen gebührende Achtung und Ehre zu umgehen, indem man ihnen
nichts statt etwas, Schein für Wesen gibt. Denn es ist nichts von den
Eigenschaften wahrer Achtung und Ehre, weder Gehorsam gegen Obere und
Vorgesetzte, noch Liebe und Wertschätzung gegen unsere gleichen, noch Hilfe,
Unterstützung und Güte gegen Geringere in ihnen enthalten.

\section{22. Abschnitt} \label{kap9_ab22}

\label{ref:09_20_koenig} \textbf{Wir erklären der ganzen Welt, dass wir
es mit wahrer Ehre und wahrer Achtung
halten. Wir ehren den König\index{Personen:!König, ehren}, unsere Eltern,
unsere Herren, unsere Obrigkeiten\index{Personen:!Obrigkeiten, ehren}
und Vorgesetzte\index{Personen:!Vorgesetzte, ehren}, uns untereinander, ja, alle
Menschen nach Gottes Vorschrift,
wie es die heiligen Männer und Frauen der Vorzeit getan haben. Aber die jetzt
üblichen Gebräuche, wodurch die Menschen einander zu ehren meinen, können wir
nicht anders als eitel, trügerisch und ganz zwecklos verweigern.}

\section{23. Abschnitt} \label{kap9_ab23}

Viertens haben wir für unsere Weigerung derselben noch einen triftigen Grund.
Wir bemerken nämlich, dass eitle, wüste und weltlich gesinnte Menschen dergleichen
Komplimente und Ehrenbezeigungen am meisten lieben und am bereitwilligsten
ausüben, so wie sie auch immer am fertigsten sind, die Einfachheit unseres
Betragens zu verspotten. Und da uns nun aus den bestimmten Zeugnissen der
heiligen Urkunden bekannt ist, dass solche Leute, die unter der Herrschaft einer
die Menschheit entehrenden Geistes leben, nicht im Stande sind, jemand wirklich
zu ehren, doch aber die Zeremonie des Hutabnehmens und der Verbeugung des
Körpers genau beobachten können, und auch nicht allein sehr freigiebig damit
sind, sondern auch eine besondere Fertigkeit darin besitzen, so dient uns
dieser zu einem klaren Beweis, dass solche Gebräuche, die jeder eitle Wüstling
mitmachen kann, nicht die rechten Mittel sind, wahre Ehre und Achtung an den Tag
zu legen.

\section{24. Abschnitt} \label{kap9_ab24}

Fünftens füge ich noch hinzu, dass diese Gebräuche oft sogar der
Heuchelei\index{Heuchelei} und
der Rachsucht zum Deckmantel dienen. Denn wie viele Menschen gibt es nicht, die
bei allen ihren Ehrenbezeigungen, die sie sich gegenseitig erweisen, dennoch
einander in ihren Herzen verachten. Ja, wieviel Groll, Neid, Bitterkeit,
heimliche Verleumdung und verborgene Anschläge, einander zu verderben, werden
nicht oft mit diesen eitlen Zeremonien bedeckt, die endlich der Zorn des
Menschen für seine List zu stark wird, die Heuchellarve abwirft und in offene
Beleidigung und Rache ausbricht. -- Dieses kann aber mit der Ehre, die uns in
der heiligen Schrift gelehrt wird, nie der Fall sein. Denn jemand aus Haß
Gehorsam zu leisten oder ihm Auszeichnung zu erweisen, würde gewiss etwas
Seltenes sein, und anderen Liebe, Hilfe, Dienste oder Unterstützung zu zeigen,
um sie zu betrügen oder sich an ihnen zu rächen, ist ganz unerhört. Mit solchen
echten Beweisen von Ehre und Achtung sind Heuchelei und Rachsucht durchaus
unvereinbar, da es sich vernünftigerweise nicht denken lässt, dass die Menschen,
um ihre bösen Gesinnungen zu bemänteln, Handlungen verrichten sollten, die
gerade das Gegenteil derselben an den Tag legen.

\section{25. Abschnitt} \label{kap9_ab25}

Unser sechster Grund ist der, dass wahre Ehrenbezeigung vom Anfange an gewesen,
der Gebrauch des Hutabnehmens und der leeren Titel und Komplimente aber eine
Erfindung neuerer Zeiten ist, und da also wahre Ehre schon bestand, ehe es Hüte
und Titel gab, so leuchtet es klar ein, dass dieselbe in dem Gebrauch dieser
Dinge nicht bestehen oder dadurch an den Tag gelegt werden kann. Auch ist die
Art und Weise, der man sich zu allen Zeiten bediente, um wahre Ehre und Achtung
auszudrücken, unstreitig noch immer die beste, und hierin werden wir gewiss
besser durch die Lehren der heiligen Schrift, als durch die Geschicklichkeit der
Tanzlehrer
\footnote{\texttt{'Tanzmeister' ersetzt duch 'Tanzlehrer'}}\index{Personen:!Tanzlehrer} unterrichtet.

\section{26. Abschnitt} \label{kap9_ab26}

Wenn, siebtens, wahre Ehre in solchen Zeremonien bestände, so würde natürlich
daraus folgen, dass diejenigen, die sie, der herrschenden Mode gemäß, am
genauesten zu beobachten verstanden, auch am fähigsten waren, anderen Ehre zu
erweisen, und dann dürfte folglich der Mensch seinen Maßstab für die Erweisung
wahrer Ehre nicht nach einem richtigen und vernünftigen inneren Pflichtgefühl
nehmen, sondern er müsste ihn nach der Vorschrift des Tanzlehrers einrichten,
dessen Kunst und Geschicklichkeit zu seiner Zeit am beliebtesten wäre. Solche
falschen Begriffe von Ehre sind es, welche so viele Eltern\index{Kinder!Erziehung}
zu bedeutenden
Ausgaben verleiten, um ihre Kinder in der Kunst unterrichten zu lassen, wie sie
anständige Komplimente oder geschickte Verbeugungen machen müssen, die man
irriger Weise zu Ehrenbezeigungen für ganz notwendig hält.
\index{Personen:!Landwirte} Und schließt nicht
auch dieser falsche Begriff von Ehre die rechtschaffenen, biederen Landleute
davon aus, die, indem sie den Acker bauen, pflügen, säen, ernten und ihre
Produkte zum Markt bringen, in allen Dingen ihrer Obrigkeit, ihren Gutsherren,
und ihren Eltern und Vorgesetzten nur Aufrichtigkeit und Ehrbarkeit gehorchen,
aber nur selten jene künstlichen Zeremonien beobachten, und wenn sie so tun,
sich doch so sonderbar und ungeschickt dabei benehmen, dass sie den
eingebildeten Weltlingen zu Gegenständen ihres Gespöttes und Gelächters dienen,
wiewohl in den Augen des Verständigen ihr Gehorsam und ihre Treue über die
Eitelkeit und Heuchelei ihrer Tadler weit erhaben ist? Die Menschen von
niedrigen und verkehrten Begriffen sind immer geschäftig, die wahre Ehre zu
verdrängen und die falsche an ihre Stelle zu setzen. Auch müssen wir noch
erwägen, dass sowohl denen, welche anderen zeremonielle Ehrenbezeigungen erweisen,
als auch denen, welchen sie erwiesen werden, mehr an der Art und Weise ihrer der
herrschenden Mode angemessenen Vollziehung, als an der dadurch auszudrückenden
Achtung und Ehre gelegen ist. Es ist daher etwas Gewöhnliches, von Personen, die
es in der Kunst, Komplimente zu machen, weit gebracht haben, zu hören: Er ist
ein sehr gebildeter\index{Personen:!Gebildete Leute}, feiner Mann! 
Oder: \label{ref:09_26_feines_benemen} \textbf{Sie ist
eine Person von sehr feinem
Benehmen! Aber worin besteht diese Bildung und dieses seine Benehmen? -- In
nichts anderem, als in einigen gezierten Stellungen und Krümmungen, die dem
Körper ganz unnatürlich sind, und jedermann lächerlich sein würden, wenn die Mode
sie nicht geböte.} Eine solche Macht hat die Eitelkeit über die Menschen in
unserem Weltteile gewonnen, dass ihr Benehmen in diesen Dingen den
morgenländischen Völkern zu einem verächtlichen Sprichwort geworden ist!

\section{27. Abschnitt} \label{kap9_ab27}

Achtens kann wahre Ehre nicht in Hüten, Verbeugungen und Titeln bestehen, weil
diese Dinge für Geld\index{Geld} zu haben sind. Daher gibt es so viele
Tanzschulen\index{Tanzschule},
Schauspielhäuser\index{Theater} und dergleichen, wohin man die Jugend schickt, um die
eitlen Moden und
Zeremonien der Welt zu erlernen, während man sie in den Dingen, welche die Ehre
Gottes betreffen, ganz unwissend lässt. So werden die zarten Gemüter der Jugend
mit den sichtbaren und vergänglichen Dingen so sehr angefüllt, dass sie, statt an
ihren Schöpfer zu denken, sich nur mit Tand und Possen, ja oft mit noch viel
schlimmeren Sachen beschäftigen, die nicht selten ihnen selbst Entehrung und
Entbehrung und ihren unverständigen Eltern lebenslänglichen Verdruss und Kummer
zuziehen. Möchten dafür die Eltern mit dem, was sie auf eine solche Erziehung
ihrer Kinder verwenden, durch Unterstützung der Armen Gott ehren! Sie würden
gewiss am Tag der endlichen Abrechnung besser dabei stehen.

\section{28. Abschnitt} \label{kap9_ab28}

Endlich können wir Verbeugungen, leere Titel und Hutabziehen auch aus der
Ursache nicht als wahre Ehrenbezeigungen anerkennen, weil solche Gebräuche in
früheren und späteren Zeiten von Gott, von seinem Sohn und von seinen Knechten
verboten worden sind, wie ich aus drei oder vier klaren Zeugnissen zu beweisen
mich bestreben werde.

\section{29. Abschnitt} \label{kap9_ab29}

Meinen ersten Beweis nehme ich aus der Geschichte des
Mardochai\index{Personen:!Mardochai} und des Hamans\index{Personen:!Hamans},
welche so sehr für uns spricht, dass sie, meines Erachtens, alle Widersprüche,
die man in dieser Sache schon gegen und vorgebracht hat, zum Schweigen bringen
muss. Haman war erster Staatsdiener und der Günstling des Königs Ahasverus. Der
Text sagt:
"`Der König setzte seinen Stuhl über alle Fürsten, die bei ihm waren,
und alle Knechte des Könige, beugten die Knie und beteten Haman an, denn der
König hatte es also geboten. Aber Mardochai beugte die Kniee nicht und betete
nicht an."'
\footnote{Esther 3,1+2.}
\index{Bibelstellen:!Esther 3)}
Dieses hatte zuerst für Mardochai so
schlimme Folgen, dass auf Hamans Befehl ein Galgen für ihn errichtet wurde. Allein
der Ausgang der Geschichte zeigt, dass Haman seine Erfindung selbst versuchen
und an dem für Mardochai errichteten Galgen seinen Stolz mit seinem Leben
bezahlen musste. War nun aber Mardechai, -- nach Art der Welt davon zu reden, und
ohne aus den für ihn günstigen Ausgang der Begebenheit zu sehen, -- nicht ein
ungeschliffener, oder doch wenigstens alberner, launischer oder eigensinniger
Mann, dass er um einer Kleinigkeit willen sich einer solchen Gefahr aussetzte?
Was würde es ihm geschadet haben, wenn er sich vor
Haman gebückt und den Mann geehrt hätte, dem der König Ehre erwirbt? Verachtete
er nicht den König selbst, indem er Haman die ihm gebührende Ehrenbezeigung
verweigerte? Der König hatte ja diese Auszeichnung des Hamans befohlen, und ist
nicht jedermann dem König Ehre und Gehorsam schuldig? Und sollte man nicht auch
sagen, Mardochai hätte ja bloß um des Königs willen sich vor Haman verbeugen und
doch dabei in seinem Herzen denken können, was ihm beliebte? Konnte er nicht
sagen, er beuge sich eigentlich nicht vor Haman, sondern vor dem Ansehn und der
Gewalt des Königs, und überdies war ja die ganze Sache nur eine unschuldige
Zeremonie. Allein, es scheint, Mardochai war ein zu gerader und fester Mann, und
besaß nicht List und Schlauheit genug, um Hamans Zorn auszuweichen.

Dem sei nun wie ihm wolle, Mardochai war ein vortrefflicher Mensch. Er fürchtete
Gott und übte Gerechtigkeit, und eben darum gefiel er Gott, und endlich auch dem
König, der wohl eher als Harman, Ursache gehabt hätte, auf ihn zu zürnen. Und
wir finden, dass der König ihn in allen Würden einsetzte, die Haman bekleidet
hatte, und ihn fast mit noch größeren Ehrenbezeigungen überhäufte. Es ist wahr,
zuerst lauteten die Nachrichten schlecht. Man ging mit nichts Geringerem um, als
Mardochai selbst und das ganze Volk der Juden um seinetwillen auszurotten. Aber
seine Demut und Rechtschaffenheit, sein Fasten\index{Fasten} und sein starken
Schreien zu
Gott trugen den Sieg über seine Feinde davon. Das Volk war erhalten, und der arme
verurteilte Mardochai war, nachdem er alles überstanden hatte, noch über die
Fürsten erhoben. O hierin liegt trefflicher Unterricht für alle, die um solcher
oder anderer Ursachen willen in geistlichen Übungen und Anfechtungen stehen.
Wer treulich in dem beharrt, was seiner festen Überzeugung\index{Überzeugung}
nach Gott von ihm
fordert\index{Anliegen!göttliches}, so sehr es auch dem Geist und Sinn der
Welt und seiner Eigenliebe
zuwider sein mag, der wird am Ende eine reiche Belohnung\index{Belohnung} darin
finden. -- Denkt,
Brüder, an den Trunk kalten Wassers!
\textit{"`Wir werden ernten, wenn wir nicht müde
werden!"'}
Erinnert euch, dass unser himmlischer Führer sich nicht vor dem
beugen
wollte, der zu ihm sagte:
\textit{"`Wenn du niederfällst und mich anbetest, so will ich
dir volle Herrlichkeit der Welt geben."'}
\footnote{Matthäus 4,9}
\index{Bibelstellen:!Matthäus 4)}
Und sollten wir
uns denn nun verneigen? Nein, nein! Wir müssen unserem heiligen Führer
nachfolgen.\index{Gott!Gott allein dienen}

\section{30. Abschnitt} \label{kap9_ab30}

Ehe ich jedoch diese Abteilung schließe, wird es nicht unpassend sein, wenn ich
noch eine Unterredung hinzufüge, die ich einst mit einem verstorbenen, nicht
wenig berühmten Bischof\index{Personen:!Bischof, Disput mit} über diesen Gegenstand
hatte. Er suchte, wie ich mich
erinnere, der Stärke meiner Beweise dadurch auszuweichen, dass er sagte:
Mardochai\index{Personen:!Mardochai} habe die Verleugung vor dem Haman nicht
aus der Ursache verweigert,
weil dieselbe ein Zeichen der Verehrung des königlichen Günstlings war, sondern
weil er (Mardochai) ein Vorbild von Christo, und Haman ein Unbeschnittener
gewesen sei, der sich eher vor ihm hätte verbeugen sollen. Hierauf erwiderte
ich, dass, indem ich zugab, Mardochai sei ein Vorbild von Christo gewesen, so
wie die Juden das Volk oder die Kirche Gottes vorstellten, und dass auch so wie
die Juden\index{Personen:!Juden} durch Mardochai erhalten wurden, die
Kirche\index{Kirche} durch Christum errettet
und erhalten werde, dieses nur zu meinen Gunsten entscheide, da aus demselben
Grund die geistlich Beschnittenen\index{Beschneidung!geistig} oder die
Nachfolger Christi\index{Personen:!Nachfolger!Christi}, sich nicht nach
den Sitten und Gebräuchen der geistlich Unbeschnittenen bequemen oder vor den
Kindern der Welt verbeugen dürfen. Daraus folgt nun noch ferner, dass solche
Gebräuche, wenn sie schon damals, als die Vorbilder noch unerfüllt waren, für
verwerflich gehalten wurden, gewiss auf keine Weise in der Zeit der Erscheinung
des Gegenbildes, oder des Wesens selbst, gebilligt und beobachtet werden dürfen.
Es leuchtet im Gegenteile klar ein, dass es unsere Pflicht ist, bei unserem
Grundsatz der Unterlassung jener weltlichen Zeremonien fest zu beharren, und
dass wir folglich in unserem Umgang mit Menschen
\textit{"`uns der Welt nicht gleichstellen dürfen,"'} \index{Welt!Umgang mit
der}
sondern in unserem ganzen Wandel erneuert und verändert
werden müssen, indem wir uns an unseren Mardochai anschließen, der sich nicht
geneigt oder verbeugt hat, und in dessen wahrer Nachfolge auch wir, als sein
Volk, uns ebenfalls den weltlichen Gebräuchen gemäß, nicht neigen und verbeugen
dürfen. Was denn auch unsere Schmach und unsere Leiden deshalb sein mögen, diese
werden ein Ende haben! Unser Heerführer Mardochai\index{Personen:!Heerführer
Mardochai}, der in allen Ländern im Tore
des Königs für sein Volk auftritt, wird uns endlich befreien und nur
seinetwillen werden wir vom König selbst begünstigt und geliebt werden. So
mächtig beweist der getreue Mardochai sich zuletzt! Darum lasst uns alle
aufsehen auf Jesus, den himmlischen Mardochai und wahren
Israel\index{Orte:!Israel}, der Macht bei
Gott hat, und in der Stunde der Versuchung\index{Versuchung} sich nicht beugte,
sondern mächtig
triumphierte, und daher ein Fürst in Ewigkeit ist, dessen Herrschaft nie ein Ende
nehmen wird.
\footnote{Jesaja 9,6+7}
\index{Bibelstellen:!Jesaja 9)}

\section{31. Abschnitt} \label{kap9_ab31}

Das nächste in den Urkunden der heiligen Schrift aufbewahrte Beispiel, welches
gegen diese weltlichen Gebräuche zeugt, finden wir beim Hiob, wo
Elihu\index{Personen:!Elihu} sagt:
\textit{"`Ich will niemands Person ansehen und will keinen Menschen rühmen
(nach der
englischen Bibel: keinen Menschen schmeichelhafte Titel geben,) denn ich weiß
nicht, wenn ich es täte, ob mich mein Schöpfer nicht über ein Kleinen
hinwegnehmen würde."'}
%Kleinen hinwegnehmen??
\footnote{Hiob 32,21+22}
\index{Bibelstellen:!Hiob 32)}
Hier wird natürlich die Frage
entstehen: Was für Titel als schmeichelhaft zu betrachten sind. Die Antwort
leuchtet aber schon von selbst ein, denn es sind unstreitig alle solche leere
und erdichteten Titel und Benennungen, die dem Menschen in der Absicht beigelegt
werden, um mehr aus ihm zu machen, als er wirklich ist.
\label{ref:09_31_heuchelei} \textbf{Z.B. wenn man jemand,
der vielleicht nach Ehre und Ansehen begierig ist, um seiner
Eigenliebe\index{Eigenliebe} zu
schmeicheln und dadurch seine Gewogenheit oder Zuneigung zu gewinnen, das
nennt, was er nicht ist, oder ihn über seinen wahren Namen und Stand, oder über
sein Amt und Verdienst erhebt, indem man ihm Würden oder Eigenschaften
zuschreibt, die er nicht besitzt. Dergleichen Titel und Beinamen sind zum Beispiel unter
den hohen Ständen "`Durchlauchtigster"' oder "`Allerdurchlauchtigster"',
"`Allerheiligster"', \textit{"`allerhöchste Majestät"'}, \textit{"`Eure
Exzellenz"'},,\textit{"`Euer
Gnaden"'}, \textit{"`Euer Herrlichkeit"'}, \textit{"`Euer Hochwürden"'},
\textit{"`Hochehrwürden"'},
\textit{"`Ehrwürden"'}, \textit{"`Euer Weisheit"'}, \textit{"`Wohlweisheit"'},
und mehrere andere
schmeichelhafte Ausdrücke und Benennungen, die man erfunden hat, um dem Ehrgeiz,
dem Stolz und der Eitelkeit des armen sterblichen Menschen zu schmeicheln.}
Aus demselben Grund werden auch unter der geringen Klasse die Menschen
genannt, was sie nicht sind, indem man sie Herr, mein Herr, u.s.w. betitelt,
und sie weise, gerecht und gut nennt, wenn sie keine Beweise von Weisheit,
Gerechtigkeit und Güte an den Tag legen.

\medskip

Dieses war schon unter den ausgearteten Juden\index{Personen:!Juden!ausgeartete}
gebräuchlich, denn wir finden, dass
einer von ihnen zu Christo kam und sagte:
\textit{"`Guter Meister, was muss ich tun,
dass ich das ewige Leben erbe?"'}
\footnote{Lukas 18,18}
\index{Bibelstellen:!Lukas 18)}
Der Ausdruck: \textit{guter
Meister}, war damals eine Art der Begrüßung oder Achtung bezeigenden Anrede, wie
es jetzt üblich ist, mein \textit{"`guter Herr, gnädiger Herr,"'} u.s.w. zu
jemanden zu
sagen. Was gab aber Christus dem jüdischen Obersten zur Antwort, und wie nahm er
seine Anrede auf?
\textit{"`Was nennest du mich gut,"' sagte Christus, "`Niemand ist gut
als nur der einzige Gott."'}
\footnote{Lukas 18,19} Er, der ein größeres Recht als
die ganze Menschheit zur Annahme einer solchen anrede hatte, verwarf dieselbe.
Und warum? Weil noch ein Größeres, als er da war, und er wohl einsah, dass
dieser Mann seine Anrede, dem damaligen Zeitgebrauch gemäß, nur an seine
Menschheit, aber nicht an seine in ihm wohnende Gottheit\index{Gott!innewohnende Gottheit} richtete. 
Darum wies
Christus seine Begrüßung von sich ab, indem er zugleich ihn und uns dadurch
belehrte, dass wir dergleichen unter den Menschen gebräuchliche Titel und
Beinamen weder selbst annehmen noch anderen geben sollen, denn da in dieser
Hinsicht Gott allein alles Gute zugeschrieben werden muss, so kann es nicht
anders als sündig sein, wenn wir das, was ihm gebührt, aus Schmeichelei dem
gefallenen Menschen zueignen.

\medskip

Ein so gerades und genaues Benehmen geziemte ihm, der zu dem großen Zweck
erschienen war, den Menschen aus dem beklagenswerten Zustand seiner Entartung
wieder zu der ursprünglichen Unschuld und Reinheit zurückzuführen, worin er sich
gleich nach seiner Erschaffung befand. Und er lehrt uns noch, behutsam zu sein,
wie wir uns mit unseren Anreden an die Menschen wenden, in jener feierlichen
Erklärung,
\textit{"`dass die Menschen am Tage des Gerichts von jedem unnützen Wort, das
sie geredet haben, Rechenschaft ablegen müssen."'}\index{Gericht!jüngstes}
\footnote{Matthäus 12,36}
\index{Bibelstellen:!Matthäus 12)}
Auch sollte es allen Menschen gegen die große Freiheit, welche sie sich hierin
erlauben, zur Warnung dienen, und das Zartgefühl unseres Gewissens hinlänglich
rechtfertigen, wenn es gehörig erwogen wird, dass der Mensch dem Allmächtigen
kaum eine größere Beleidigung zufügen kann, als wenn er die Ehrenbezeigungen,
die ihm gebühren, seinen Mitmenschen zueignet, der ein Geschöpf seines Wortes,
das Werk seiner Hand ist.
\textit{"`Der Herr ist ein eifriger Gott, und er will seine
Ehre keinem anderen geben."'} Überdies hat ein solches Betragen der Menschen so
viel ähnliches mit der Sünde jener aus Ehrgeiz gefallenen Engel, die größer und
besser zu sein trachteten, als der unbeschränkte Herr über alles sie gemacht und
ihnen ihre Stelle angewiesen hatte. Ja, es sieht der Abgötterei, -- jener
Sünde,
die unter dem Gesetz unverzeihlich war, -- so ähnlich, wenn wir einen Menschen
über seinen Stand und Beruf, worin er von Gott gesetzt ist, erheben, dass es kaum
begreiflich ist, wie männliche und weibliche Wesen, die sich Christen nennen,
wenn sie über das Unrecht und die Eitelkeit solcher Dinge ernstlich nachdenken,
noch langer in denselben verharren, sie sogar verteidigen, und was noch
schlimmer ist, andere, die aus zarter Gewissenhaftigkeit sich davon losgesagt
haben, deshalb tadeln und verlachen können. Elihu wollte, wie wir lesen, es
nicht wagen, sich damit zu befassen, denn er legte der Sache ein so großes
Gewicht bei, das er einen Grund seiner Unterlassung des Gebrauchs der
schmeichelhaften Titel mit den Worten darlegte:
\textit{"`damit mein Schöpfer mich nicht
über ein Kleines hinwegnehme"'} Das heißt: Aus Furcht, dass Gott mich töten möchte,
darf ich den Menschen keine Titel beilegen, die ihnen nicht gebühren, oder
welche nur ihrer Eigenliebe schmeicheln. Ich darf dem Geiste, dem
danach gelüstet, keine Nahrung geben. Gott muss gepriesen und erhoben, der
Mensch aber erniedrigt werden. Gott zürnt\index{Gott!zürnt} mit Eifer darüber,
wenn der Mensch
über seinen Stand erhoben wird. Er will, dass der Mensch in seiner Sphäre bleibe,
seinen Ursprung bedenke und
\textit{"`sich des Felsens erinnere, aus welchem er gehauen
ist."'}
\footnote{Jesaja 51,1}
\index{Bibelstellen:!Jesaja 51)}
Er soll wissen, dass alles, was er besitzt, nicht sein
eigen, sondern ihm nur anvertraut ist, dass es seinem Schöpfer gehört, der ihm
sein Dasein gab, und ihn erhält, was alles der Mensch in der Eitelkeit
seines Herzens so leicht vergisst. Damit ich ihm nun hierin durch Beilegung
schmeichelhafter Titel nicht behilflich bin, statt dass ich ihn, der Wahrheit und
Aufrichtigkeit gemäß, in seiner wirklichen Eigenschaft betrachten und anreden,
und nach seiner wahren Würde behandeln sollte, und damit ich mir dadurch das
Missfallen meines Gottes nicht zuziehe, und ihn nicht reize, mich in seinem Zorn
und Eifer durch einen schnellen oder frühzeitigen Tod hinwegzunehmen, so will,
-- so darf ich mich solcher Titel gegen niemand bedienen.

\section{32. Abschnitt} \label{kap9_ab32}

Könnten wir nun auch solche Beweise zur Rechtfertigung unserer Abschaffung der
leeren Titel und anderer eitlen Ehrenbezeigungen aus den Schriften des alten
Testaments nicht anführen, so sollte und müsste es doch allen Christen genügen,
dass jene Gebräuche von dem großen Herrn und Meister ihrer Religion streng gerügt
werden. Er ist in der Tat so weit entfernt, die Menschen in ihren
Ehrenbezeigungen zu bestärken, dass er sie ihnen vielmehr, -- so sehr sie sich
auch auf die herrschende Landessitte berufen mögen, -- als ein Zeichen ihrer
Abweichung vorwirft. Denn so warf er sie den Juden als einen Beweis ihres
Unglaubens mit den Worten vor:
\textit{"`Wie könnt ihr glauben, die ihr Ehre von einander
nehmt, und die Ehre, die allein von Gott ist, nicht sucht?"'}
\footnote{Johannes 5,44}
\index{Bibelstellen:!Johannes 5)}
Hier sehen wir, dass der Mangel ihres Glaubens an Christus die wirkende
Ursache war, warum sie weltliche und nicht himmlische Ehre allein suchten.
Dieses können wir uns auch leicht erklären, wenn wir erwägen, dass Selbstliebe
und Ehrfucht mit Christusliebe und Demuth unverträglich sind. Und da nun die
ausgearteten Juden nach Ruf und Ansehen in der Welt strebten, so war es ihnen
unmöglich, alles zu verlassen und Christus nachzufolgen, dessen Reich\index{Reich
Gottes} nicht von
dieser Welt ist, und der auf eine Art erschien, die den Gesinnungen und der
Stimmung der Menschen so sehr zuwider war, dass dieser Sinn in den oben
angeführten Worten unseres Herrn enthalten ist, geht deutlich aus der Erklärung
hervor, die er uns von jener Ehre gibt, welche die Juden\index{Personen:!Juden}
einander erwiesen und
von einander annahmen, und vor welcher er die Nachfolger seiner Demuth, die sein
Kreuz trugen\index{Kreuz!tragen}, warnte. Die Ausdrücke, deren er sich bediente,
und die er nicht
auf das Volk, sondern auf die gelehrten und großen Männer von Ansehn und Ehre
unter den Juden anwendete, sind diese:
\textit{"`Sie sitzen gern oben an bei Tisch
oder bei Gastmählern,"'}
\footnote{Matthäus 23,6}
\index{Bibelstellen:!Matthäus 23)}
d.h.: sie lieben, hinsichtlich des
Ranges und der Ehre, die vornehmsten Plätze,
\textit{"`und haben es gern, wenn sie gegrüßt werden,"'}\footnote{\texttt{Dieser Satz kommt im Original zwei mal vor. Offenbar ein Fehler. Wir haben ihn ein mal gestrichen.}}
d. h.: sie
finden Gefallen an den üblichen Begrüßungen und Ehrenbezeigungen, wie zum Beispiel in
unseren Tagen es die Entblößung des Hauptes und die Verbeugungen des Körpers sind,
und zwar \textit{"`auf dem Markt,"'} nämlich an bemerkbaren Orten, zum Beispiel auf
öffentlichen Wandelgängen, Sammelplätzen u.s.w. Und endlich sagt Christus,
haben sie es gern,
\textit{"`wenn sie von den Menschen Rabbi genannt
werden."'}\index{Personen:!Rabbi}
\footnote{Matthäus 23,6}
\index{Bibelstellen:!Matthäus 23)}
Diese Benennung war bei den Juden einer der
ausgezeichnetsten Titel, ein Wort, das große Ehre und Erhabenheit bezeichnete,
und mit den jetzt gebräuchlichen Ausdrücken: \textit{"'Euer Gnaden, Eure
Herrlichkeit, Hochwürdiger Vater"`} etc. als gleichbedeutend betrachtet werden
kann. Er war
über solche Menschen von hohem Rang und verfeinerter Bildung, über welche unser
Herr das Wehe aussprach, indem er ihr ehrsüchtiges Betragen als ein böses
Merkmal, woran sie erkennbar waren, und als die Ursache angab, die ihn zu seinen
Drohungen gegen sie bewog. Dabei lässt er es jedoch nicht bewenden, er verfolgt
diesen Punkt von Ehrbegierde vor allen andern, indem er seine Jünger dagegen
warnt, und ihnen ebenso bestimmt wie nachdrücklich gebietet:
\textit{"`Aber ihr sollt
euch nicht Rabbi nennen lassen, denn einer ist euer Meister, Christus, und ihr
alle seid Brüder. Auch sollt ihr euch nicht Meister nennen lassen. .... Der
Größte unter euch soll euer Diener sein, und wer sich selbst erhöht, der soll
erniedrigt werden."'}\index{Gleichheit}\index{Rangordnung}
\footnote{Matthäus 23,8-12}
\index{Bibelstellen:!Matthäus 23)}

\medskip

Es ist klar, dass diese Stellen sowohl eine ernste Bestrafung der weltlichen Ehre
überhaupt, als auch der besonderen Zweige und Ausdrücke derselben enthalten, und
dass diese Bestrafung, so deutlich als sie die Sprache der heiligen Schrift nur
bezeichnen kann, und insofern unser jetziges Zeitalter mit dem damaligen
Ähnlichkeit hat, sich auch bestimmt auf die in unseren Tagen üblichen
zeremoniellen Ehrenbezeigungen bezieht, wegen deren Unterlassung wir, als eine
religiöse Gesellschaft, oft schon nicht nur persönliche Verachtung, Verspottung
und Misshandlung, sondern auch nicht selten Schaden und Verlust an unserem
Vermögen erlitten haben. Möge Gott solche Verfolgungen den unverständigen
Urhebern derselben verzeihen!

\section{33. Abschnitt} \label{kap9_ab33}

Der Apostel Paulus gibt in seiner Epistel an die Römer eine überaus wichtige,
eindrückliche und mit dieser Lehre Christi genau übereinstimmende Ermahnung mit
den Worten:
\textit{"`Ich ermahne euch, liebe Brüder, durch die Barmherzigkeit Gottes,
dass ihr euere Leiber zu einen lebendigen, heiligen und Gott wohlgefälligen Opfer
begebet, (darstellet,) welches euer vernünftiger Gottesdienst ist. Und stellt
euch nicht dieser Welt gleich, sondern verändert euch durch Erneuerung eures
Sinnes, (Gemüths,) damit ihr prüfen mögt, welches der gute, der wohlgefällige
und der vollkommene Wille Gottes sei."'}
\footnote{Römer 12,1+2.}
\index{Bibelstellen:!Römer 12)}

\medskip

Der Apostel schrieb dieses an ein Volk, das von den Schlingen des äußeren Glanzes
und der Pracht der Welt ganz umgeben war. Rom war der Sitz
Cäsars\index{Personen:!Cäsar}, gleichsam
der Brennpunkt des Reichs, und die Gebieterin aller Erfindungen und Moden, die,
sowie jetzt die französischen, wo nicht der ganzen, doch wenigstens der
römischen Welt zu Vorschriften dienten. Daher auch das Sprichwort: \texttt{Cum
fueris Romae, Romano vivito more}: Wenn du in Rom\index{Orte:!Rom} bist, so musst
du nach römischer Sitte leben.

\medskip

\index{Welt!Absondern von}Der Apostel war aber anderer Meinung. Er warnte
die Christen in jener
Stadt), dass sie sich \textit{der Welt nicht gleichstellen}, das heißt, ihren eitlen
Moden
und Gebräuchen nicht folgen, sondern sich von ihnen losmachen sollten.
Dieses empfahl er ihnen ebenso nachdrücklich, denn jene Welt, der sie sich
nicht gleichstellen sollten, bestand in nichts anderes, als in dem verderbten und
entarteten Zustande der Menschen jener Zeit. Und daher ermahnte er auch die
Gläubigen, und zwar mit dem kräftigsten und eindringendsten Beweggrunde der
Barmherzigkeit Gottes, sich zu ändern, das heißt, der unter den Römern üblichen
Lebensweise zu entsagen und dem wohlgefälligen Willen Gottes gemäß zu leben.
Als hätte er gesagt:
\textit{"`Untersucht, was ihr tut und treibt, seht zu, ob es vor
Gott recht und ihm wohlgefällig ist. Richtet jeden Gedanken, jedes Wort, jede
Tat. Prüft, ob euer Tun und Lassen von Gott ist oder nicht,"'}
\footnote{Johannes 3,21}
\index{Bibelstellen:!Johannes 3)}
damit ihr auf diese Weise erkennen mögt, worin der gute, der
wohlgefällige und der vollkommene Wille Gottes\index{Gott!Willen} bestehe.

\section{34. Abschnitt} \label{kap9_ab34}

Der nächstfolgende Beweis, den wir zu unserer Verteidigung aus der Schrift
anführen, ist eine Stelle aus dem ersten Brief des Apostels "`Petrus"', den er
an die hin und wieder in Ponto, Galatien, Cappadocien, Asien und
Bithynien befindlichen, gläubigen Fremdlige schrieb, welche, nachdem sie
durch die Kraft des göttlichen Geistes gesammelt waren, die Gemeinen Christi
in jenen Weltgegenden ausmachten. Die Worte des Apostels sind diese:
\textit{"`Darum
begürtet die Lenden eures Gemüths, seid nüchtern und setzt euere Hoffnung ganz
auf die Gnade, die euch durch die Offenbarung Jesu Christi dargeboten wird,
als gehorsame Kinder, und stellt euch nicht gleich, wie vorhin, da ihr in
Unwissenheit nach den Lüsten lebtet."'}
\footnote{1. Petrus 1,13+14.}
\index{Bibelstellen:!1. Petrus 1)}
Das heißt:
Lasst euch nicht mehr in den eitlen Moden und Gebräuchen der Welt antreffen, nach
welchen ihr euch in eurer vorigen Unwissenheit richtetet und bequemtet, sondern,
da ihr einen einfacheren und vortrefflicheren Weg erkennt, so seid nun auch
nüchtern und inbrünstig im Geist und hofft bis ans Ende. Ermattet nicht und
gebt euere Sache nicht auf! -- Lasst die Welt immer spotten und traget das
Widersprechen der Sünder mit Standhaftigkeit, als gehorsame Kinder, damit ihr
bei der Offenbarung Jesu Christi die Belohnung Gottes empfangen mögt.

\medskip


In dieser Hinsicht nannte auch der Apostel die Gläubigen "`\textit{Fremdlinge,
leben als}"', \index{Personen:!Fremdlinge} eine sehr
passende figürliche Benennung für Leute, denen die Gebräuche und Gewohnheiten
der Welt fremd geworden waren, die einen Glauben und Sitten hatten, welche die
Welt nicht kannte, und denen es, als solchen Fremdlingen, würde schlecht
angestanden haben, wenn sie sich nach den eitlen Moden und Gebräuchen der Welt
hätten richten und bequemen wollen, da ihre Entfremdung darin bestand, dass sie
sich solchen Dingen, die ihnen zuvor gewöhnlich und eigentümlich waren,
gänzlich entzogen hatten, dass übrigens der Apostel mit dem Worte
\textit{Fremdling}
einen geistlichen Sinn verstand, leuchtet deutlich aus dem folgenden siebzehnten
Vers ein, wo er sagt:
\textit{"`So führt nun eueren Wandel, so lange ihr hier wallet,
mit Fucht."'}
Das heißt, \textbf{Verlebt euere Zeit hier auf Erden als Fremdlinge in der
Furcht Gottes, nicht nach der Sitte der Welt.} Dieser Sinn des Apostels erhellt sich
auch noch ferner aus seinen Ausdrücken im zweiten Kapitel, wo er die Gläubigen
ein eigentümliches Volk nennt, das heißt, ein besonderes, unterschiedenes, von der
Welt abgesondertes Volk, das mit ihren Moden und Gebräuchen keine Gemeinschaft
mehr hat. Sonst sehe ich nicht ein, wie der Apostel diese von den Gläubigen
hätte sagen können, denn sobald sie mit andern in ihren weltlichen Gebräuchen
und eitlen Ehrenbezeigungen gemeinschaftliche Sache machen, sind sie nicht mehr
ein eigentümliches von der Welt abgesondertes, sondern ein sich ihr
gleichstellendes und ihr gleichförmiges Volk.

\section{35. Abschnitt} \label{kap9_ab35}

Ich will meine aus der heiligen Schrift hergenommenen Beweise gegen den Gebrauch
weltlicher Ehrenbezeigungen mit einem bemerkenswerten Zeugnis des Apostels
Jakobus schließen, welches derselbe gegen die in der Welt übliche Unterscheidung
des Ansehens der Person im Allgemeinen ablegt.\index{Gleichheit}
\textbf{\textit{"`Liebe Brüder,"'} sagt er,
\textit{"`haltet nicht dafür, dass der Glaube an Jesus Christus, unseren Herrn der
Herrlichkeit, Ansehen der Person leide. Denn wenn in eure Versammlung ein Mann
mit einem goldenen Ring und mit einem herrlichen Kleide käme, und es käme auch
ein Armer in einem unsaubern Kleid, und ihr sähet auf den, der das herrliche
Kleid trüge, und spärchet zu ihm: Setze du dich her aufs Bette, und spräche zu
dem Armen: Stehe du dort, oder setze dich her zu meinen Füßen, und bedächtete es
nicht recht, würdet ihr dann nicht Richter und machet bösen Unterschied?"'}
(Indem sie wohl wüssten, dass sie unrecht handelten.) --\textit{ "`Wenn ihr das
königliche
Gesetz erfüllt nach der Schrift: Liebe deinen Nächsten als dich selbst, so tut
ihr wohl. Wenn ihr aber die Perlen anseht, so tut ihr Sünde und werdet
als Übertreter vom Gesetze bestraft."'}}
\footnote{Jakobus 2,1-4. und 8. und 9.}
\index{Bibelstellen:!Jakobus 2)}
Dieses ist
ein so volles und klares Zeugnis, dass es mir ebensowenig nötig zu scheint, ihm
noch etwas beizufügen, als es AnderEn schwer sein dürfte, gründliche Einwendungen
dagegen zu machen. Das erste, was der Apostel uns ausdrücklich erklärt, ist; dass
wir die Person nicht ansehen sollen, und das zweite, dass wir, wenn wir es tun,
sündigen und das Gesetz Gottes übertreten, und zwar auf unsere eigene Gefahr.
Dennoch werden vielleicht Einige sagen, \label{ref:09_35_staende_abschaffen}
\textbf{dass wir auf diese Weise allen
Unterschied der Stände unter den Menschen aufhöben, und dagegen eine Ehre und
Achtung einführten, die bloß auf gegenseitiger Liebe und Wertschätzung beruhe.
Wenn das die Folge dieser Lehre ist, so kann ich es nicht ändern, und der
Apostel Jakobus, der sie uns als christlich und apostolisch überliefert hat,
muss es verantworten.} Es hat aber auch ein Größerer als er seinen Jüngern, --
von denen Jakobus einer war, -- gesagt:
\textit{"`Ihr wisst, dass die weltlichen Fürsten
herrschen, u.s.w. So soll es unter euch nicht sein, sondern wenn jemand unter
euch gewaltig"'} oder der Vornehmste\\textit{"`sein will, der sei euer
Diener."'}
\footnote{Matthäus 20,25-27}
\index{Bibelstellen:!Matthäus 20)}
Das will heißen: Wer nach Ansehn und
Herrschaft strebt, und gern der Vornehmste sein will, der soll der Geringste
unter euch sein. Um überhaupt die Wahrheit zu sagen, so müssen wir gestehen, dass
sowohl in den früherer Zeitaltern vor der Erscheinung Christi auf Erden, als
auch in den zunächst darauf folgenden Zeiten, in den Sitten und Gebräuchen der
Menschen weit mehr Einfachheit herrschte, als in unseren Tagen anzutreffen ist.
Jene frühen Zeiten der Welt, so schlimm sie auch in anderen Dingen gewesen sein
mögen, waren doch mit dem jetzt so allgemeinen Gebrauche schmeichelhafter Titel
und anderer Torheiten fast gänzlich unbekannt, und wenn sie auch angewendet
wurden, so geschah es doch nur sehr selten. Sehen wir uns in der biblischen
Geschichte um, so finden wir kein Beispiel von solchen Benennungen, als:
\textit{Herr
Adam}, oder \textit{Herr von Adam!} obgleich er Herr der Welt war, noch:
\textit{Herr} oder
\textit{Herr von Noah}, der doch zweiter Herr der Erde war, oder: der
\textit{Herr von Abraham},
der Vater der Gläubigen, der \textit{Herr Jsaak, der Herr Jakob}, u.s.w. Noch
vielweniger: \textit{Herr Paulus, Herr Petrus, hochwürdiger Herr Apostel,} und
dergleichen und
nichts von \textit{Eure Heiligkeit, Euer Gnaden,} u.s.w. Selbst die Heiden
gebrauchten
weit einfachere Benennungen ihrer Personen und bedienten sich keiner so
schmeichelhaften Anreden und Zeremonien in ihrem Umgang, als jetzt unter den
Christen üblich sind. In keinem griechischen oder lateinischen Werk finden wir
die Ausdrücke:,\textit{Herr von Solon, Herr von Phoeion, Herr von Plato, Herr
von
Aristoteles, Herr von Scipio, Herr von Flaccius, Herr von Cato, Herr von Cicero}
obgleich diese Männer die größten Weisen und Helden der mächtigsten Reiche der
Welt waren. Ihre bloßen Namen waren hinreichend, sie vor anderen Menschen
auszuzeichnen, und ihre Tugenden, die sie in der Besorgung des öffentlichen
Wohls an den Tag legten, machten ihre Ehrentitel aus. Auch hat jener eitle
Gebrauch bei den lateinischen Schriftstellern sich noch nicht eingeschlichen,
die der Gewohnheit treu geblieben sind, die gelehrtesten und ausgezeichnetsten
Männer bloß bei ihren Namen anzuführen, und ihnen höchstens die Beinamen: weise
und würdig, zuzueignen, die auch wir, wenn ihre Handlungen sie ihnen beilegen,
ohne Gewissensskrupel ihnen gern zugestehen. So werden zum Beispiel die
Kirchenväter\index{Personen:!Kirchenväter}
nur auf diese Weise angeführt:
\textit{Polycarpus\index{Personen:!Polycarpus},
Ignatius\index{Personen:!Ignatius}, Irenäus\index{Personen:!Irenäu},
Cyprian\index{Personen:!Cyprian},
Tertulian\index{Personen:!Tertulian}, Origenes\index{Personen:!Origenes},
Arnobius\index{Personen:!Arnobius}, Lactantius\index{Personen:!Lactantius},
Chrysostomus\index{Personen:!Chrysostomus},
Hieronymus\index{Personen:!Hieronymus},} u.s.w.
So auch neuere Schriftsteller, z.B.:
\textit{Damascenus\index{Personen:!Damascenus},
Rabanus\index{Personen:!Rabanus}, Paschasius\index{Personen:!Paschasius},
Theophilactus\index{Personen:!Theophilactus},
Bernhard\index{Personen:!Bernhard},} u.s.w. und noch neuere, z.B.:
\textit{Luther\index{Personen:!Luther},
Melanchthon\index{Personen:!Melanchthon},
Calvin\index{Personen:!Calvin}, Beza\index{Personen:!Beza},
Zwingli\index{Personen:!Zwingli}, Marlorat\index{Personen:!Marlorat},
Bossius\index{Personen:!Bossius}, Scotius\index{Personen:!Scotius},
Dalleus\index{Personen:!Dalleus}, Amyralbus\index{Personen:!Amyralbus},} u.s.w.
Auch Schriftsteller unsers Vaterlandes finden wir ebenso angeführt, z. B.:
\textit{Gildas\index{Personen:!Gildas}, Beda\index{Personen:!Beda},
Alruinus\index{Personen:!Alruinus}, Horn\index{Personen:!Horn},
Bracton\index{Personen:!Bracton}, Grosteed\index{Personen:!Grosteed},
Littleton\index{Personen:!}, Cramer\index{Personen:!Cramer},
Ribley\index{Personen:!Ribley},
Jewel\index{Personen:!Jewel}, Whitaker\index{Personen:!Whitaker},
Selden\index{Personen:!Selden}} u.a.m. Da man nun, wie ich vermute, solche
einfache
Anführungen nicht für unhöflich oder unschicklich hält, warum sollen denn wir
unserer einfachen Sitte wegen so sehr verlacht und verspottet werden, wenn es
klar ist, dass unser Betragen in diesen Stücken sich auf einen rechtlichen
Gewissensskrupel gründet, den wir gegen die Verehrung des Stolzen in den Herzen
der Menschen haben, der auf eine eben so begierige als verderbliche Weise nach
Ansehen, Ehre und Größe strebt? Und warum soll uns dies von Bekennern der
christlichen Religion widerfahren, deren göttlicher Stifter seinen Nachfolgern
jene törichten Gebräuche in seiner Lehre ausdrücklich verboten und sie ebenso
bestimmt als jede andere ungöttliche Handlung verworfen hat? Ich bitte daher
alle, die solche eitlen Zeremonien noch lieben, gebrauchen und von anderen
erwarten, sehr ernstlich, dass sie dem, was ich darüber geschrieben habe, ein
ruhiges uneinvorgenommenes Nachdenken gönnen wollen.

\section{36. Abschnitt} \label{kap9_ab36}

\textbf{Die Christen sind indessen auch nicht so unbescheiden und ungebildet,
als die
Welt vielleicht glaubt.} Auch sie erweisen Achtung und Ehre, nur liegt der
Unterschied, der zwischen ihnen und der Welt hierin statt findet, sowohl in der
Natur und Eigenschaft der Ehre, die sie erweisen, als auch in den Beweggründen,
die sie dazu haben. Die Ehre der Welt besteht in einer leeren Zeremonie ohne
Kraft und Leben. Die Ehre der Christen hingegen ist etwas Wesentliches und
Wahres, sie mag nun durch Gehorsam gegen Obere und Vorgesetzte, oder durch Liebe
und Wertschätzung gegen ihresgleichen, oder durch Hilfe und Unterstüzung gegen
Geringere oder Untergebene an den Tag gelegt werden. Dann sind auch die
Beweggründe zu den Ehrenerweisungen bei beiden Teilen sehr verschieden.
\label{ref:09_36_ehre_erweisen} \textbf{Diejenigen der Welt sind schöne 
Kleider\index{Kleidung!schöne}, prächtige Titel oder großes
Vermögen, denn dieses sind die Gegenstände, welche die Kinder der Welt
\index{Personen:!Kinder!der Welt} lieben und
verehren. Die Beweggründe der Christen hingegen entspringen aus einem Gefühl
der Pflichten, die sie in den Augen Gottes anderen zu erweisen schuldig sind,}
und
zwar erstlich Eltern, Obrigkeit und Vorgesetzten, dann denen, die ihnen
weniger nahe sind, \textbf{und endlich allen Menschen, nach dem Maße ihrer
Tugend,
Weisheit und Gottseligkeit.}\index{Ehre!erweisen nach Maß der Tugent} Diese 
Ehre ist in der Tat sehr verschieden von der,
bei welcher die Menschen einander eine persönliche Rücksichten erweisen, indem sie
entweder die Personen anderer aus eigennützigen Absichten verehren, oder zu so
niedrigen Begriffen und Gefühlen von der Menschenwürde herabgesunken sind, dass sie
vor Reichtum oder prächtigen Kleidern sich beugen.

\section{37. Abschnitt} \label{kap9_ab37}

\label{ref:09_37_ehre_erweisen} \textbf{Wir geben gern zu, dass unsere Art,
Ehre zu erweisen, gewissermaßen so verborgen
wie unsere Religion\index{Religion!verborgene} ist, und dass beide, für weltlich gesinnte 
Gemüther, ebenso
wenig erkennbar als behaglich sind. Unser einfaches und gerades Benehmen fällt
ihnen als eigen und sonderbar auf, und geht, sozusagen, ganz gegen den Strich.}
Und so verhält es sich auch mit der christlichen Religion, und zwar aus
denselben Ursachen. Denn, hätte nicht, unter dem Namen des Christentums, ein
heidnisches Wesen schon so lange unter den Bekennern desselben geherrscht, so
würde es ihnen nicht so schwer sein, das Wahre von dem Falschen zu
unterscheiden, O! Mögen daher doch die Christen sich in dem Spiegel der
Gerechtigkeit\index{Gerechtigkeit!Spiegel der} beschauen, der ihnen ihre wahre Gestalt
 zeigt, und sie mit
gründlicher Selbstkenntnis versieht. Dann werden sie prüfen und erkennen
können, was in und an ihnen mit der Lehre und dem Leben Christi übereinstimmt,
dann werden sie im Stande sein, richtig zu beurteilen, ob sie wirkliche
Christen oder nur mit dem christlichen Namen getaufte 
Heiden\index{Personen:!getaufte Heiden} \index{Taufe} sind. Hier
folgen nun noch einige Zeugnisse zu Gunsten unseres Betragens aus älteren und
neueren Schriftstellern.

\section{38. Abschnitt} \label{kap9_ab38}

Marlorat\index{Personen:!Marlorat} gibt uns 
aus Luther\index{Personen:!Luther} und Calvin\index{Personen:!Calvin} eine Erklärung der oben 
angeführten
merkwürdigen Stelle in der ersten Episiel des Apostel Jakobus, worin er die
Gedanken jener ersten Reformatoren über das Ansehen der Person in folgenden
Worten ausdrückt:
\textit{"`die Person ansehen, will an diesem Orte soviel sagen, als
auf Tracht und Kleider Rücksicht nehmen. Der Apostel gibt dadurch zu erkennen,
dass ein solches Ansehen der Person dem wahren Glauben so sehr zuwider sei, dass
es sich gar nicht mit demselben vereinigen lasse, und wenn äußerer Glanz und
andere weltliche Rücksichten die Oberhand gewinnen, und das, was von Christo
ist, schwächen, so ist dieses ein Zeichen des sinkenen Glaubens. Denn die
Herrlichkeit und der Glanz Christi ist in einer wahrhaft frommen Seele so groß,
dass, in Vergleichung damit, alle Herrlichkeit der Welt für ein so göttlich
gesinntes Gemüth keinen Reiz und keine Schönheit hat. Der Apostel zeigt, dass ein
solches Ansehen der Person gegen das Licht im Menschen streitet, da alle, die
solchen Gebräuchen anhängen, in ihrem Innern dafür bestraft werden. Darum
müsse Heiligkeit die Ursache und der Beweggrund aller äußeren Achtungsbezeigung
sein, und niemand dürfe aus einem anderen Grunde, als seines heiligen Lebens
wegen, geehrt werden."'} So weit Marlorat. Ist nun aber diese Lehre in der
Wahrheit gegründet, so haben wir vollkommen Recht, wenn wir uns weigern, die
eitlen Ehrenbezeigungen weltlicher Menschen zu beachten.

\section{39. Abschnitt} \label{kap9_ab39}

Ich füge diesem Zeugnisse noch die Ermahnung bei, die ein gelehrter alter
Schriftsteller, der vor 1200 Jahren lebte und in großer Achtung stand, der
edle Matrone Celantia\index{Personen:!Celantia, Matrone}, erteilte. In einer Anweisung, 
wie sie in ihrem
Wohlstande und bei dem Genusse hoher Ehre leben müsse, gibt er ihr, unter
anderen religösen Erinnerungen, folgenden Unterricht:
\textit{"`Siehe nicht auf deinen
Adel und lass ihn dir nicht zu einer Ursache dienen, irgend jemand
vorzuschreiten. Betrachte nicht andere von geringerer Herkunft als dir
untergeordnete, denn unsere Religion erlaubt kein Ansehen der Person, und
lehrt uns, die Menschen nicht nach ihren äußeren Verhältnissen, sondern nach
ihrer inneren Gemüthsverfassung und Gesinnung zu würdigen. Hiernach halten wir
sie für edel oder unedel. In Gottes Augen ist der, welcher der Sünde nicht
dient, frei, und wer durch Tugend sich auszeichnet, edel. Gott hat die
Niedrigen und Verachteten der Welt erwählt, um dadurch die Großen zu demütigen.
Überdies ist es große Torheit, wenn jemand auf seinen vornehmen Stand sich
etwas einbildet, da alle vor Gott gleich sind. \label{ref:09_39_erettung}
\textbf{Die Erkaufung der Armen und der
Reichen kostete Christus ein gleiches Maß seinen Blutes.} Auch ist es ganz
unwesentlich, in was für einem Stande ein Mensch geboren ist, da die neue
Kreatur keinen Unterschied der Stände kennt. Wollen wir aber vergessen, dass wir
alle von einem Vater abstammen, so sollten wir uns doch wenigstens beständig
erinnern, dass wir nur einen Erlöser haben."'}

\section{40. Abschnitt} \label{kap9_ab40}

Da ich mich einmal darauf eingelassen habe, jene ebenso beliebten als nutzlosen
Gebräuche, die als eigentümliche Erzeugnisse eitler und stolzer Gemüther, auch
die Ergötzlichkeit derselben ausmachen, zu bestreiten, so will ich noch eine
bemerkenswerte Stelle anführen, wie ich sie bei dem berühmten Casanbon\index{Personen:!Casanbon} in
seiner Abhandlung über Sitten und Gebräuche vorfinde. Er berichtet uns daselbst
in der Kürze, was zwischen Sulpitius Severus\index{Personen:!Severus, Sulpitius} und Paulinus, 
dem Bischof von
Nola\index{Personen:!Paulinus, Bischofe von
Nola}, vorfiel. Paulinus war einer von den gottseligen Männern, die alles
hingaben, um Gefangene zu befreien, während andere gleichen Standes, um den
Charakter ihres Herrn an den Tag zu legen, oft viele zu Bettlern und Gefangenen
machten, indem sie die Plünderung und Einkerkerung solcher Christen
begünstigten, die der Stimme Gottes in ihrem Gewissen gehorsam waren. Paulinus
drückt sich nun so aus: \textit{"`Man hält unter uns seit einigen Jahren den nicht für
einen höflichen Menschen, der sich ein Gewissen daraus macht, oder sich weigert,
in Briefen an seinesgleichen oder an geringere sich als ihr Diener zu
unterschreiben."'} Auch erhielt einst Sulpitius Severus von Paulinus einen
scharfen Verweis, weil Letzterer in einem Briefe an ihn sich 
als "`\textit{seinen Diener}"'
unterzeichnet hatte. \textit{"`Hüte dich in Zukunft,"'} sagte Paulinus, \textit{"`dass du, der du
aus der Knechtschaft zur Freiheit berufen bist, dich nicht Diener eines Menschen
nennst, der dein Bruder und Mitknecht ist, denn es ist keine sündige
Schmeichelei, und kein Beweis von Demuth, wenn man sich gegen einen Menschen,
gegen seinen Sünder, der Ehrenbezeigung bedient, die nur dem einzigen Herrn, dem
einzigen Meister und einzigen Gott gebühren"'} Dieser Bischof hatte, wie es
scheint, dieselbe Gesinnung, die Christus mit den Worten ausdrückte: \textit{"`Was
nennst du mich gut? Niemand ist gut, als Gott allein."'} Und wir können hieraus
annehmen, wie jene apostolischen Bischöfe über die Höflichkeitsbezeigungen
dachten, die jetzt bei denen, die sich Christen, Bischöfe, ja, Nachfolger der
ersten christlichen Bischöfe nennen, in so hohem Ansehen stehen. Damals war es
Sünde, wenn jemand sieh derselben bediente, heutigen Tages betrachtet man sie
als Tugenden. Damals hielt man sie für Schmeichelei, jetzt gelten sie für
Beweise der Achtung, damals wurden sie scharf gerügt, und jetzt, -- ach! Jetzt
verdient jeder, der sie unterlässt, den strengsten Tadel. O der ungeheueren
Eitelkeit! Wie sehr, wie entsetzlich weit sind diejenigen, die sich Christen
nennen, von den einfachen Sitten der ersten Zeiten des Christentums und von
der Lebensweise der heiligen Männer und Frauen jener Tage abgewichen! Sie haben
sich in der Tat dem zügellosen Leben der Welt, die Gott nicht kennt, so sehr
ergeben, und sind ihren eitlen Zeremonien, -- die doch sowohl von der Schrift
und Vernunft, als auch durch das ihnen widersprechende Beispiel guter Menschen
verworfen worden, -- durch langen Gebrauch so sehr daran gewohnt geworden, dass sie
ihnen ganz natürlich zu sein scheinen. Und viele sehen so wenig weder die
Ursache noch die schädlichen Wirkungen dieser Torheiten ein, dass sie nicht
allein beständig in denselben fortleben, sondern sie sogar verteidigen,
und auf eine sehr unchristliche Weise diejenigen, welche sich den Gebräuchen
derselben nicht mehr erlauben dürfen, zu Gegenständen ihres Spottes und
Gelächters machen. -- Doch ich gehe zur Erklärung und Verteidigung eines andern
Stückes unserer einfachen Sitte über, welches nicht wenig dazu beiträgt, dass wir
den Leichtsinnigen, Eitlen und Unbesonnen unseren Zeitalters ein Stein des
Anstoßes sind.


