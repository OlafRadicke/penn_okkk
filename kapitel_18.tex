
\chapter{18. Kapitel} \label{kap18}
\section{Zusammenfassung des 18. Kapitel}

\begin{description}
\item[1. Abschnitt] Wenn man euch die eitlen Gebräuche etc. als gleichgültige
Dinge betrachten könnte, so müssen sie dennoch, um des Mißbrauchs willen, den
man davon macht, verworfen werden.
\\(\textit{Seite \pageref{kap18_ab1}})
\item[2. Abschnitt] Selbst Diejenigen, die sie mitmachen erkennen diesen
Mißbrauch, und sollten deshalb davon abstehen.
\\(\textit{Seite \pageref{kap18_ab2}})
\item[3. Abschnitt] Wer nur einigermaßen auf Ehrbarkeit Anspruch macht, sollte
schon um des schlechten Beispiels willen solcher Ungebundenheit entsagen. Ein
weiser Vater entzieht seinem Kinde den Gegenstand, an dem es zu sehr hängt; und
wir sind ja verpflichtet, über uns selbst und über unsern Nächsten zu wachen.
\\(\textit{Seite \pageref{kap18_ab3}})
\item[4. Abschnitt] Gott giebt in dem Beispiele mit der ehernen Schlange eine
Belehrung, daß wir die Sachen, mit welchen ein Mißbrauch getrieben wird, bei
Seite sezen sollen.
\\(\textit{Seite \pageref{kap18_ab4}})
\item[5. Abschnitt] Möchte auch der Gebrauch dieser Dinge zuweilen zweckmäßig
sein, so müssen sie dennoch, sobald sie ein böses Beispiel geben, verworfen
werden.
\\(\textit{Seite \pageref{kap18_ab5}})
\item[6. Abschnitt] Wer sich noch solche Vergnügungen erlaubt, zeige, das er sie
mehr als Christum und sein Kreuz liebt. -- Von den Nachteilen, die sie auf
Personen und Vermögen, auf Leib und Seele haben.
\\(\textit{Seite \pageref{kap18_ab6}})
\item[7. Abschnitt] Verständlige Leute wissen, das dieses wahr ist; was die
Schuldigen betrifft, so berufe ich mich auf den Zeugen Gottes in ihrem Herzen.
Ihr Zustand ist dem des geistigen Babylons ähnlich.
\\(\textit{Seite \pageref{kap18_ab7}})
\item[8. Abschnitt] Mäßigkeit im Essen und Trinken, und Einfachheit in der
Kleidung, tragen sehr zur Beförderung des Guten bei, wie der Apostel Paulus in
seinen Episteln behauptet.
\\(\textit{Seite \pageref{kap18_ab8}})
\item[9. Abschnitt] Mäßigkeit bereichert ein Land, und ist daher unter jeder
Regierung sowohl eine politische als religiöse Tugent.
\\(\textit{Seite \pageref{kap18_ab9}})
\item[10 Abschnitt] Wenn Man seine Pflichten gegen Gott erfüllt hat, so ist es
noch Zeit genug, an Erholung für sich selbst zu denken.
\\(\textit{Seite \pageref{kap18_ab10}})
\item[11. Abschnitt] Vorschlag für die Obrigkeiten und für alle Menschen, wie
sie ihre Zeit und ihr Geld zu besseren Zwecken verwenden können.
\\(\textit{Seite \pageref{kap18_ab11}})

\end{description}

\newpage

\section{1. Abschnitt} \label{kap18_ab1}

Sollten nun auch alle jene eitlen Moden, Gebräuche und Ergötzungen wirklich so
gleichgültig sein, wie sie als schädlich und verderblich bewiesen werden sind,
-- denn daß Jemand in der Vertheidigung derselben mehr als ihre Gleichgültigkeit
hatte behaupten wollen, ist mir nicht bekannt, -- so ist doch der Mißbrauch, der
damit getrieben wird, so groß, und der Einfluß desselben, der wie eine
ansteckende Seuche um sich greift, so nachtheilig, daß sie schon aus diesem
Grunde von allen Menschen, vornehnilich aber von Denen, deren Maßigkeit sie
bisher vor Uebertreibung derselben bewahrte, oder welche, obgleich sie sich
ihrer schuldig gemacht haben, doch Verstand genug besitzen, die Thorheit solcher
Unmäßigkeit einzusehen, gänzlich verworfen werden sollten. Was ist aber eine
gleichgültige Sache anders, als eine solche, die man mit gleicher Befugniß thun
oder lassen kann? Und wenn ich nun auch einräumen wurde, daß dieses hier der
Fall sei, so lehret uns dennoch Vernunft und Religion, daß bei jeder Sache,
sobald man sie mit solcher Begierde gebraucht, daß es ein Kreuz\index{Kreuz}
sein würde, ihr
zu entsagen, die Grenzen solcher Gleichgültigkeit überschritten werden, und man
folglich gestehen muß, daß sie dadurch etwas unentbehrliches geworden ist. Und
da nun auf diese Weise die Natur der Sache verletzt wird, so entstehet ein
vollkommner Mißbrauch derselben, und man kann sie folglich nicht mehr als
gleichgültig ansehen, sondern muß sie als unerlaubt betrachten.

\section{2. Abschnitt} \label{kap18_ab2}

Nun wird man mir zugeben, dass Mißbräuche aller der Dinge, die ich so ernstlich
angegriffen habe, unter Personen von jedem Alter, Geschlechte und Stande
angetroffen werden. \textbf{Aber Viele wollen ihnen darum nicht entsagen, weil
sie
dieselben, wie ich vernommen habe, aus dem Grunde für erlaubt halten, daß, --
wie sie sagen, -- der Mißbrauch, den Andere von einer Sache  machen, ihren
Gebrauch derselben nicht aufhebe.} Allein sie haben vielleicht vergessen, oder
wollen es sich nicht einfallen lassen, daß sie ja alle jene Gebräuche etc. für
gleichgültige Dinge erklärt haben; und wenn dem so ist, wie denn auch die
Eitelkeit selbst nicht mehr verlangt, -- so, sage ich --, kann keine Schlußfolge
klarer sein, als die, daß sie alle diese Dinge nothwendig verlassen müssen, weil
sie den Mißbrauch derselben eingestehen. \textbf{Denn, wenn sie so gleichgültig
sind,
daß man sie zu jeder Zeit nach Belieben mitmachen oder unterlassen kann; so
erfordert es gewiß unsere Pflicht, daß wir ihnen entsagen: da es klar am Tage
liegt, daß unser Gebrauch derselben ihr allgemeines Uebermaß noch vermehrt, und
Andere in ihrem Mißbrauche bestärkt, wenn sie sehen, daß Leute, die den Ruf
eines ehrbaren Lebenewandels behaupten wollen, solche Thorheiten nachahmen, oder
ihnen darin mit ihrern Beispiele vorangehen; denn das Beispiel macht immer einen
noch weit stärkern Eindruck als die Lehre.}
\footnote{Philipper 3,17}
\index{Bibelstellen:!Philipper 3)}

\section{3. Abschnitt} \label{kap18_ab3}

Jeder, der Anspruch auf ernstes Nachdenken macht, sollte sich genau prüfen, ob
er nicht aus irgend eine Art zur Beförderung der herrschenden Ausschweifungen
schon beigetragen habe, und daher keine Zeit verlieren, sich gänzlich von ihnen
loszumachen, damit, wo sein voriges Beispiel Andern darin zur Aufmunterung
diente, sein besserer Lebenswandel auch den Einfluß haben möge, ihrer
Unmäßigkeit Einhalt zu thun. \index{Erziehung}Weise Eltern\index{Personen:!Eltern}
entziehen ihren Kindern\index{Personen:!Kinder} die
Gegenstände, welche, so unschuldig sie auch an sich sein mögen, eine zu große
Gewalt über ihre schwachen Sinne ausüben, um sie davon zu entwöhnen. So pflegt
man auch einen krummen Stock,um ihn gerade zu machen, ebensosehr nach der
entgegengesetzten Seite zu biegen, damit er auf diese Art die gehörige Richtung
erhalte. \index{Vorbild sein}\textbf{Und gewiß, Diejenigen, welche mehr
Mäßigkeit als Andere besitzen,
sollten nicht vergessen, daß sie Haushälter Gottes sind, und daher diese ihnen
verliehene Gabe zum Besten ihres Nächsten anwenden. Nur der Brudermörder
Cain\index{Personen:!Cain}
fragte den Herrn:
\textit{"`Soll ich denn meines Bruders Hüter sein?"'}}
\footnote{1. Mose 4,9}
\index{Bibelstellen:!1. Mose 4)}
Denn dazu ist Jeder unumgänglich verpflichtet; und daher sollte auch Jeder
so klug sein, sich den Gebrauch solcher gleichgültigen Dinge zu versagen,
wodurch sein Nächster zu Thorheiten verleitet oder in denselben bestärkt werden
könnte.

\section{4. Abschnitt} \label{kap18_ab4}

Gott hat in dieser Hinsicht seinen Willen hinlänglich zu erkennen gegeben; denn,
obgleich die eherne Schlange\index{Schlange} eine seiner eigenen Anordnungen und
ein Vorbild von
Christo war, so befahl er doch mit großen: Mißfallen, daß sie zerbrochen werden
sollte, weil das Volk zu sehr daran hing. Sogar die Haine, so schön und angenehm
auch ihre Lage war, mußten zerstöret werden; und warum? weil man sie zu
Gegenständen der Abgötterei\index{Gott!Abgötterei} gemacht hatte. Und was ist ein
Abgott oder Götze
andere, als etwas, worauf daß menschliche Gemüth einen zu hohen Wert legt? Es
muß daher besser und vortheilhafter sein, einer gemißbrauchten Suche zu
entsagen, als eine gleichgültige zu gebrauchen.

\section{5. Abschnitt} \label{kap18_ab5}

Waren auch jene Dinge an sich selbst wirklich nützlich, welches sie eher
nothwendig als gleichgültig machen würde, so müßte dennoch, sobald Umstände sie
schädlich machten, auch diese ihre Nützlichkeit auf die Seite gesetzt werden;
wievielmehr sollte man sie denn aber nicht als bloß gleichgültige Dinge
verleugnen? \textbf{Wir müssen niemals unser Privatvergnügen der öffentlichen
Wohlfahrt
vorziehen, noch uns den Gebrauch gleichgültiger Dinge erlauben, wenn wir wissen,
daß es dem Gemeinwohle nachtheilig sein Würde, wie dieses wirklich der Fall mit
allen solchen Dingen ist, durch deren Gebrauch man, auss gelindeste genommen,
Andern ein böses Beispiel giebt und sie zur Nachahmung desselben reizt.} Daher
gebieten uns Vernunft und Pflicht, nur solche Dinge zu den unentbehrlichen zu
rechnen, die mit
\textit{"`einem göttlichen Leben und Wandel"'}
\footnote{2. Petrus 1,3.}
\index{Bibelstellen:!2. Petrus 1)}
übereinstümmen, und unsere Freiheit ihres Gebrauches nach dem, was unserm
Nächsten zutraglich ist, abzumessen. \textbf{Hier treten also zwei
Verpflichtungen ein;
die erste ist:}daß man in solchen Dingen kein Beispiel gebe, deren Gebrauch,
wenn er auch an sich selbst unschädlich ist, doch unserm Nächsten in dem
Mißbrauche derselben und in seiner übertriebenen Eitelkeit bestärken wurde; und
\textbf{die zweite} bestehet darin: daß man bei dem Gebrauche unerlaubter Dinge
aus
diejenigen religiösen Personen Rücksicht nehme, denen die eitlen Moden,
Gebrauche, Vergnügungen etc. anstößig sind, und sie daher gänzlich
vermeide.
\footnote{Römer 14,13 bis zu Ende}
\index{Bibelstellen:!Römer 14)}

\section{6. Abschnitt} \label{kap18_ab6}

Was nun Diejenigen betrifft, die aller meiner angeführten Grunde ungeachtet,
dennoch in allen diesen Thorheiten bleiben wollen, so muß es ihnen selbst
einleuchten, \textbf{daß sie keinen andern Beweggrund dazu haben können, als
weil sie
tief darin verwickelt und ihre Neigungen so sehr davon hingerissen sind, daß es
ihnen fast unmöglich ist, sich davon loszusagen, und weil sie, ungeachtet ihrer
öftern Versicherungen, daß ihr Herz an solchen Nichtigkeiten nicht hänge,
dennoch dieselben mehr als Christum und sein heiliges Kreuz\index{Kreuz}
lieben.} Wer seine
eigene Glückseligkeit so wenig berücksichtigt, wird auch schwerlich die
Wohlfahrt seiners. Nächsten zu befördern suchen.

\medskip

Wenn wir endlich die Wirkungen und Folgen der üppigen Moden, Gebräuche,
Ergötzungen und Zeitvertreibe der Menschen ernstlich erwägen, so kann es uns
nicht entgehen, zu bemerken, wieviel Eitelkeit, Stolz, Müßiggang und sowohl
Zeit- als Geldverschwendung, kurz, wieviel Elend sie schon in der Welt
verbreitet haben und noch verbreiten. Wie viele Männer sind nicht dadurch von
ihrer Mäßigkeit abgebracht, wie viele Weiber nicht von ihrer natürlichen
Sanftmuth und Unschuld zu leichten, ausschweifenden, frechen, ja schändlichen
Handlungen verleitet worden? Welche herrliche Besitzungen haben sie nicht in die
größte Schuldenlast gestürzt? Wie ist nicht so oft durch sie die Keuschheit in
die Schlingen der niederträchtigsten Verführer gerathen und die jugendliche
Gesundheit durch die unnatürlichsten Krankheiten plötzlich untergrabeir worden,
so daß die unglücklichen Sklaven\index{Personen:!Sklaven} ihrer zügellosen Begierden ihre
übrigen
Lebenstage unter bestandigen Qualen zubringen mußten, weiche die schrecklichen
aber unvermeidlichen Folgen ihrer Ausschweifungen waren. Und so groß ist das
Verderben der Menschen und die Macht des Lasiers, daß Einige während ihrer
Qualen die größte Müßigkeit angeloben, sobald diese aber vorüber sind, wieder zu
ihren vorigen Ausschweifungen zurückkehren.
\footnote{
Klagelieder 4,5)
Weisheit Salomos 12,17)
Hiob 21,13+14)
Psalm 35,23)
Psalm 37,10)
Kohelet 8,12)
Psalm 37,1+2)
Sprüche 2,22}
\index{Bibelstellen:!Klagelieder 4)}
\index{Bibelstellen:!Weisheit Salomos 12)}
\index{Bibelstellen:!Hiob 21)}
\index{Bibelstellen:!Psalm 35)}
\index{Bibelstellen:!Psalm 37)}
\index{Bibelstellen:!Kohelet 8)}
\index{Bibelstellen:!Sprüche 2)}

\section{7. Abschnitt} \label{kap18_ab7}

Daß dieses und noch weit mehr wirklich der Fall sei, wird, meiner Ueberzeugung
nach, kein Verstandiger, der einige Erfahrung besitzt, leugnen. Wie denn aber
irgend Jemand, der ein Gewissen hat, oder Gottesfurcht zu haben bekennt, nach
ernster Ueberlegung noch in der Tracht, in den Sitten, Gebräuchen und
Vergnügungen solcher Menschen, deren ganzes Leben mit Dingen, wie ich eine
berührt habe, beflekt ist, verweilen, oder sogar an ihren Ausschweifungen Theil
nehmen könne, überlasse ich dem gerechten Zeugen der Wahrheit in ihren eigenen
Herzen zu beurtheilen. Nein wahrlich! das heißt nicht der Stimme Gottes
gehorchen,  die zu allen Zeiten laut rief: \textit{"`Gehet aus!"'} und woraus
denn? -- Aus den Wegen, Moden und Gebräuchen, aus der Lebensweise und dem Geiste
\textit{"`Babylons."'}\index{Orte:!Babylon} Und was ist Babylon? --
\textit{"`Die große Stadt, worin alle jene eitlen,
thörichten, üppigen, überflüssigen und gottlosen Dinge getrieben werden"'},
gegen
welche die Schrift die furchtbarsten Gerichte\index{Gericht!Jüngstes} ankündigt,
indem sie alle
Unmäßigkeit und Zügellosigkeit der Menschen dem \textit{"`Becher der Bosheit"'}
zuschreibt,
aus welchem Babylon ihnen zu trinken gegeben hat, und woher alle sogenannte
gleichgültigen Dinge ihren Ursprung genommen haben.
\footnote{Jeremia 16,5-9)
Jesaja 3,13-16)
Jesaja 50,8)
Jesaja 15,6+7)
Amos 6,3-7}
\index{Bibelstellen:!Jeremia 16)}
\index{Bibelstellen:!Jesaja 3)}
\index{Bibelstellen:!Jesaja 50)}
\index{Bibelstellen:!Jesaja 15)}
\index{Bibelstellen:!Amos 6)}
Laßt uns hören, was Johannes in
seiner Offenbarung von ihr sagt:
\textit{"`Wieviel sie sich herrlich gemacht und ihren
Muthwillen gehabt hat, soviel schenket ihr Qual und Leid ein. -- Und es werden
sie beweinen und beklagen die Könige der Erde, die Muthwillen mit ihr getrieben
haben. Und die Kaufleute auf Erden werden weinen und Leid tragen, weil ihre
Waare Niemand mehr kaufen wird; die Waare des Goldes und Silbers, und der
Edelgesteine, und Perlen, und Seide, und Purpur, und Scharlach, und allerlei
Thienenholz, und von Erz, und von Eisen, und von Marmor und Zimmt, und Thymian,
(Räuchwerk) und Salben, und Weihrauch, und Wein, und Oel, und Semmelmehl, und
Weinen, und Vieh und Schafe, und Pferde und Wagen, und Leichname, und Seelen der
Menschen."'}
\footnote{Offenbarung 7.9+11-13}
\index{Bibelstellen:!Offenbarung)}
Hier seyen wir sowohl den Charakter
als auch das Gericht der Maßlosigkeit
\footnote{\texttt{'Üppigkeit' erstzt durch 'Maßlosigkeit'}}; denn obgleich diese
Worte, wie ich weiß,
noch eine andere als buchstäbliche Bedeutung
haben\index{Gebote!Bibelauslegung}\index{Bibelauslegung!Wörtliche}, so bezeichnen sie
doch auch
sehr deutlich die Pracht, den Ueberfluß, den Reichthum und das ganze müßige,
gemächliche, eitle und wollüstige Leben, welches unter den Einwohnern
Babylons\index{Orte:!Babylon}
herrscht. Aber wer will an dem schrecklichen Tage ihresz
Gerichtes\index{Gericht!Jüngstes} noch ihrem
Markte beiwohnen? Wer will dann ihre Schauspiele besuchen, ihre Moden nachahmen,
und mit ihren glänzenden Erfindungen Handel treiben? Gewiß Niemand; denn sie
soll gerichtet werden. Kein Vorwand wird sie entschuldigen, oder vor dem Zorne
des Richters schützen;
\textit{"`denn stark ist Gott der Herr, der sie richten wird."'}
\footnote{Offenbarung 18,8}
\index{Bibelstellen:!Offenbarung 18)}

\medskip

Sollte man auch diesen vernünftigen Vorstellungen kein Gehör geben, so werde ich
demungeachtet noch einen Theil des traurigen Schicksals Babylons zur fernern
Warnung anführen. Denn, meine Freunde, ihr müsset lernen, euch mit himmlischen
Dingen zu beschäftigen, und eilen, dem göttlichen Einflusse in eurem, Innern
Gehorsam zu leisten, der euch zu Betrachtungen ewiger Freuden anleitet, sonst
wird mit Babylon\index{Orte:!Babylon}, der Mutter aller Maßlosigkeit
\footnote{\texttt{'Üppigkeit' erstzt durch 'Maßlosigkeit'}} und Eitelkeit,
\textit{"`das Obst, an
welchem eure Seele Lust hat, von euch weichen, und Alles was köstlich und
herrlich ist, wird von euch weichen, und ihr werdet es nicht mehr
finden."'}
\footnote{Offenbarung 18,14}
\index{Bibelstellen:!Offenbarung 18)}
Nein! o! ihr Reichen! Nicht mehr! Darum, ihr
Bewohner der Erde!
\textit{"`sammelt euch Schätze im Himmel,"'}
\footnote{Lukas 12,23+43.}
\index{Bibelstellen:!Lukas 12)}
wohin nichts eindringen kann, das sie zuerstören könnte, und wo sehr bald die
Zeit sich in der Ewigkeit verlieren wird.

\section{8. Abschnitt} \label{kap18_ab8}

Doch hören meine Beweise gegen die schädlichen Thorheiten der Welt hier noch
nicht auf; ich muß noch zeigen, daß das Gegentheil derselben, nämlich:
Maßigkeit\index{Maßigkeit}
im Essen und Trinken und Einfachheit in der Kleidung, Verbunden mit einem
sanften, bescheidenen Wesen und ruhigen Gemüthe, daß bei jeder Gelegenheit durch
ein mit demselben ubereinstimmendes Betragen in heiliger Ehrbarkeit sich
ausspricht, alles Gute in der Welt stiftet und verbreitet; weshalb auch der
Apostel die Christen ermahnet:
\textit{"`Lasset kein unnützes Geschwätz aus eurem Munde
gehen, sondern, was zur Besserung dienet, und holdselig (erbaulich) zu hören
ist. Auch keine schändlichen Worte und Narrentheidinge (Possen) oder Scherz,
sondern vielmehr Danksagung. Lasset euch von Niemand verführen mit vergeblichen
(leeren) Worten; denn um dieser Dinge willen kommt der Zorn Gottes "`über die
Kinder des Unglaubens."'}
\footnote{
Kolosser 4,5+6)
1. Thessalonicher 4,11+12)
1. Petrus 3,1-4)
Epheser 4,4)
Epheser 5,3-6)
1. Timotheus 4,12.
Philipper 3,16-20}
\index{Bibelstellen:!Kolosser 4)}
\index{Bibelstellen:!1. Thessalonicher 4)}
\index{Bibelstellen:!1. Petrus 3)}
\index{Bibelstellen:!Epheser 4)}
\index{Bibelstellen:!Epheser 5)}
\index{Bibelstellen:!1. Timotheus 4)}
\index{Bibelstellen:!Philipper 3)}
\textbf{Wenn die Christen
beides Geschlechtes sich auf eine mit ihrem Bekenntnisse übereinstimmende Art
anständig kleideten, so würde die Unverschämtheit Anderer dadurch gezügelt und
ihrer Prachtliebe, ihrem Stolze und ihrer Eitelkeit ein sie bestrafeudes Muster
vorgehalten werden. Sie würden es nicht wagen, die allgemein anerkannte
Keuschheit anzugreifen; der göttliche Ernst der Christen würde ihre Frechheit
entwaffnen.} Der Tugend müßte man überall Achtung zollen, das Laster würde würde
sich furchtsam und beschämt zurückziehen müssen, und die Unmäßigkeit dürfte sich
nicht sehen lassen. Dieses würde aller Schwelgerei, aller Kleiderpracht, allem
Titelstolze, und dem verschwenderischen Leben ein Ende machen; die ursprüngliche
Unschuld und Einfalt zurückrufen, und jenes einfache, gerade, aufrichtige und
harmlose Leben wieder herstellen, wobei man nicht bekümmert ist, was man essen
oder trinken und womit man sich kleiden wolle, welches, wie Christnis uns sagt,
die Sorge der Heiden ist, und, leider! auch unter den heutigen
Christenthunmsbekennern, bei allem ihrem Geschwätze von Religion und
Frömmigkeit, so häufig angetroffen wird. Man würde vielmehr das Beispiel der
Alten nachahmen, die, mit mäßiger Sorge für die Bedürfnisse und Bequemlichkeiten
dieses Lebens, ihre Hauptsorge auf die Angelegenheiten den himmlischen
Reiches\index{Reich!himmlisches}
richteten, und sich mehr um ihren Wachsthum in der Gerechtigkeit als um die
Vermehrung ihres Reichthumts bekümmerten. "`Denn sie sammelten sich Schätze für
den Himmel und erduldeten Trübsale um des unvergänglichen Erbes
Willen"'
\footnote{
2. Petrus 2,12)
Sprüche 31,23-31)
Jakobus 2,2-9)
2. Petrus 3,11)
Psalm 26,6)
Lukas  12,22-30)
Matthäus 25,21}
\index{Bibelstellen:!2. Petrus 2)}
\index{Bibelstellen:!Sprüche 31)}
\index{Bibelstellen:!Jakobus 2)}
\index{Bibelstellen:!2. Petrus 3)}
\index{Bibelstellen:!Lukas  12)}
\index{Bibelstellen:!Matthäus 25)}

\section{9. Abschnitt} \label{kap18_ab9}

\textbf{Die Tugend der Mäßigkeit, die ich empfehle und vertheidige, ist aber
nicht
allein in religiöser, sondern auch in politischer\index{Politik} Hinsicht
nützlich; da es immer
einer jeden guten Regierung großen Vortheil gewährt, wenn sie die
Ausschweifungen der Unterthanen einschränkt und unterdrückt.} Das üppige Leben
erzeugt Weichlichkeit, Trägheit, Armuth\index{Armut} und Elend\index{Elend};
aber die Mäßigkeit beugt
solchen Uebeln vor und erhält das Land im Wohlstande.
\footnote{Sprüche 10,4)
Kohelet 10,16-18.}
\index{Bibelstellen:!Sprüche 10)}
\index{Bibelstellen:!Kohelet 10)}
\index{Ausenhandel}Sie bewahret vor ausländischen Thorheiten und
Ueberflüssigkeiten, und befördert die Verbesserung unserer eigenen Erzeugnisse,
wodurch wir, statt Andern schuldig zu sein, sie zu unsern Schuldnern machen
können. Durch Ausübung dieser Tugend können Leute die durch Ausschweifung, nicht
durch Wohlthätigkeit, ihre Güter in tiefe Schulden gestürzt hoben, in kurzer
Zeit von diesen Schulden wieder frei werden, die sonst, wie fressende Motten,
die größten Besitzungen verzehren. Die Mäßigkeit hilft Unbemittelten ihr
geringes Vermögen vermehren, indem sie ihnen nicht erlaubt, ihren mühsam
erworbenen Verdienst mit Puz und Kleiderpracht, thörichten Maifesten oder
Fastnachtsbelustigungen, Spielen, Tanzen\index{Tanz} und Schauspielen, mit
Schwelgen in
Schenken und Wirthshäusern\index{Wirthshäuser} und andern Ausbrüchen der
Unmäßigkeit durchzubringen,
welche Thorheiten besondere in unserm Vaterlande (England)\index{Orte:!England}
so sehr überhand
genommen haben, daß es dadurch mehr, als irgend ein anderes Land in der Welt,
lächerlich gemacht wird. Denn nirgend findet man, so viel ich weiß, mehr
betrügerische Marktschreier und Quacksalber\index{Personen:!Quacksalber}, Seiltänzer,
Taschendiebe, profane
Komödianten und Gaukler\index{Personen:!Gaukler}, zur großen Herabwürdigung der Religion
und zur Schande
der Regierung\index{Regierung}; wodurch dann das Volk zum Müßiggange, zur
Verschwendung und zu
Ausschweifungen verführt, der heilige Geist betrübt, und der Allmächtige gereizt
wird, seine Gerichte, die vor der Thür sind, herein brechen zu lassen und
endlich den Ausspruch zu thun:
\textit{"`Wer böse ist, sei immerhin böse."'}
\footnote{Offenbarung 22,11}
\index{Bibelstellen:!Offenbarung 22)}
\index{Zeugnis}\index{Missionierung}\index{Weg!der einzige}\textbf{-- Daher
können wir nicht andere, als unsere
Stimmen laut erheben, und sowohl durch unsere Lehre, als auch durch unsern
Lebenswandel, gegen die einen Mißbräuche unserer Nebenmenschen zeugen, damit, wo
möglich, Einige dadurch bewogen werden mögen, solche Thorheiten zu verlassen,
und den guten alten Pfad der Mäßigkeit, der Weisheit, des Ernstes; und der
Heiligkeit zu wählen; da dieser nur der einzige ist, der zu dein wahren Genusse
der Segnungen des Friedens und der Fülle in dieser Zeit, und der ewigen
Glückseligkeit in der künftigen Welt führet.}

\section{10. Abschnitt} \label{kap18_ab10}

Gesetzt, endlich, wir hätten auch keine der angeführten Grunde für uns, um die
im Lande herrschenden bösen Gebräuche und Gewohnheiten mit Recht zu verwerfen,
so werdet ihr uns doch erlauben, zu bemerken, daß er dann, wenn die Menschen
werden gelernt haben, ihren Schöpfer zu fürchten, ihn anzubeten und ihm zu
gehorchen; wenn sie die großen Schulden ihrer Abweichungen abgetragen und die
Last ihrer Untergebenen werden erleichtert haben; wenn vornehmlich die blassen
Gesichter mehr bemitleidet, die Hungrigen gesättiget, die Nakten bekleidet, die
schmachtenden Armen, die verlassenen Wittwen und hülflosen Waisen, -- welche
Gottes Geschöpfe und eure Mitmenschen sind, -- versorgt sein werden; alsdann,
sage ich, wenn ihr je können werdet, -- wird es noch Zeit genug für euch sein,
die Gleichgültigkeit und Unschädlichkeit eurer Vergnügungen zu
behaupten.
\footnote{
Kohelet 12,1
Psalm 37,21
Psalm 10,2
Psalm 4,2
Psalm 84,3+4
Sprüche 22,7+9
Jesaja 3,14+15}
\index{Bibelstellen:!Kohelet 12)}
\index{Bibelstellen:!Psalm 37)}
\index{Bibelstellen:!Psalm 10)}
\index{Bibelstellen:!Psalm 4)}
\index{Bibelstellen:!Psalm 84)}
\index{Bibelstellen:!Sprüche 22)}
\index{Bibelstellen:!Jesaja 3)}
\index{Soziale Ungerechtigkeit}\textbf{Daß aber der Schweiß und die mühsamen
Arbeiten
des Landmannes vom frühen Morgen bis in die späte Nacht, bei Hitze und Kälte,
bei Dürre und Nässe, für das Vergnügen, das Wohlbehagen und die Zeitvertreibe
einer kleinen Anzahl von Menschen zu sorgen bestimmt wären; daß neunzehn
Zwanzigstel\footnote{\texttt{...Etwas umständlich ausgedrückt. Mann könnte auch
einfach sagen: '5\% der Bevölkerung (nämlich die Oberschicht) lebt auf Kosten
von 95\% der hart arbeintenden Befölkerung, in Saus und Braus'}} der Bewohner
des Landes beständig den härtesten Arbeiten mit dem
Pfluge, dem Karn, dem Fuhrwerke und dem Dreschflegel unterworfen sein müßten, um
die ungezügelten Begierden der Maßlosigkeit
\footnote{\texttt{'Üppigkeit' erstzt durch 'Maßlosigkeit'}} und Schwelgerei des
einen Zwanzigstels
zu befriedigen, ist so weit entfernt, eine Anordnung des großen Regierers der
Welt und Gottes der Geister alles Fleisches zu sein, daß es im höchsten Grade
gotteslästerlich und abscheulich sein würde, nur zu denken, daß solche
schreiende Ungerechtigkeit von \textit{Ihm} und seinen göttlichen Einrichtungen,
und
nicht von der Unmäßigkeit der Menschen herrühre.}
\footnote{
Amos 5,11+12
Amos 8,4
1. Petrus 1,17
Jakobus 5,4+5
Psalm 41,1
Jakobus 2,15+16
Psalm 112,9}
\index{Bibelstellen:!Amos 5)}
\index{Bibelstellen:!Amos 8)}
\index{Bibelstellen:!1. Petrus 1)}
\index{Bibelstellen:!Jakobus 5)}
\index{Bibelstellen:!Psalm 41)}
\index{Bibelstellen:!Jakobus 2)}
\index{Bibelstellen:!Psalm 112)}
Von einer andern Seite
betrachtet, können für Wahr\footnote{\texttt{'billig' ersetzt durch 'für Wahr'}}
die Menschen auf das Mitleid oder auf die Hülfe und
den Beistand des allmächtigen Gottes keine Ansprüche machen, wenn sie mit ihren
Ausgaben für eitle Vergnügungen fortfahren, während sie die Bedürfnisse der
Unglücklichen unbefriedigt lassen; besonders wenn man erwägt, daß Gott sie nur
zu Haushaltern gesetzt hat, die einander dienen, helfen und unterstützen sollen.
Ja, diese Pflichten sind so bestimmt eingeschärft, daß wir, wenn wir sie
unterlassen, nicht ohne Grund den schrecklichen Ausspruch erwarten mussen:
\textit{"`Gehet von mir, ihr Verfluchten, in des ewige
Feuer."'}\index{Verflucht} u. s. w.
\footnote{Matthäus 25,34-41.}
\index{Bibelstellen:!Matthäus 25)}
\index{Mildtätigkeit}Wohingegen das Besuchen der Kranken und Gefangenen und die
Unterstützung der Durstigen in den Augen Jesu solche vortreffliche Eigenschaften
sind, daß er Diejenigen, die es daran nicht ermangeln ließen, mit dem
erfreulichen Außspruche selig preisen wird:
\textit{"`Kommet her, ihr Gesegneten meines
Vaters, ererbet das Reich, das für euch bereitet ist"'}, u. s. w. Die Großen der
Erde sollten also nicht, wie der Wal\footnote{\texttt{'Leviathan' ersetzt duch
'Wal'}} in Meere, die Kleinen zu ihrer Beute
machen, und noch vielweniger mit dein Leben und der mühsamen Arbeit der
Geringern Scherz treiben, um ihre zügellosen Neigungen zu befriedigen.

\section{11. Abschnitt} \label{kap18_ab11}

Sei es mir daher erlaubt, den bürgerlichen Obrigkeiten\index{Personen:!bürgerlichen
Obrigkeiten} einen Vorschlag zu thun,
der ihre ernste Betrachtung zu verdienen scheint. Nämlich: wenn in jedem
Gerichtsbezirke die Einwohner dazu aufgefordert und aufgemuntert würden, das
Geld, welches sie für eitle Ueberflüssigkeiten, z.B. Spitzen, Juwelen,
Stickereien, Besetzungen auf Kleider, kostbare Verzierungen an Mobilien, und für
unnöthige Dienerschaft ausgeben, nebst dem, was sie gewöhnlich bei Gastmählern,
Festen, oder beim Spiele u. s. w. ausgeben\footnote{\texttt{'aufgehen lassen'
ersetzt duch 'ausgeben'}}, in eine öffentliche Kasse zu
legen, so würde man bald ein hinreichendes Kapital zusammen bringen können,
womit verarmte Familien unterstützt, Arbeitshäuser für Arme\index{Sozialer
Wohnungsbau}, die noch zu
arbeiten fähig wären, erbauet, und Armenanstalten für alte und
gebrechliche\index{Altersheime}
Leute errichtet und unterhalten werden könnten. Dann
würden wir keine Bettler\index{Personen:!Bettler} im Lande haben, und man würde die Klagen
und das
Geschrei der Wittwen und Waisen nicht mehr hören. Es könnten unglückliche
Gefangene losgekauft und die Drangsale der Protestanten, die in andern Ländern
Verfolgung leiden, erleichtert werden\index{Personen:!Glaubensverfolgte}. Ja,
selbst dem öffentlichen Schatze\index{Sozialversicherung} könnte, in dringenden
Fällen, aus solchen
Sparbänken Unterstützung geleistet werden\footnote{\texttt{...Also sowas wie
eine Sozialversicherung}}. Dieses würde ein Opfer, ein Dienst
sein, der dem gerechten und gnädigen Gott gefiele; man gäbe dadurch andern
Ländern ein edles Beispiel der Weisheit und Mäßigkeit, und stiftete eine
unschätzbare Wohlthat für das Vaterland.
\footnote{Sprüche 14,21
Matthäus 19,21}
\index{Bibelstellen:!Sprüche 14)}
\index{Bibelstellen:!Matthäus 19)}
\footnote{Der wegen seiner großen
Menschenliebe allgemein geschätzte und vornehmlich wegen seiner unablässigen
edlen Bemühungen: die unglückliche Lage der armen Gefangenen überall zu
erleichtern, in ganz Europa berühmte John Howard\index{Personen:!Howard, John},
fand einst beim Abschlusse
einer Jahrrechnung\footnote{\texttt{...Also Jahresbilanz}} einen bedeutenden
Ueberschuß seiner Einnahme. Er bot die
ganze Summe seiner Frau an, um zu einer Reise nach London\index{Orte:!London}
oder zu irgend
einem andern ihr beliebigen Vergnügen Gebrauch davon zu machen. Die Edelmüthige
erwiederte: Was für eine nette Wohnung für eine arme Familie ließe sich doch
damit erbauen! Dieser schöne Wink fand ganz den Beifall
ihres guten Mannes, und das Geld ward zu dem von ihr vorgeschlagenen
wohlthätigen Zwecke verwendet. Philipper 4,8
\textbf{(Anmerkung des Uebersetzers)}}

\medskip

\textbf{Ach! warum bedarf es denn so vieler Ueberredung, um die Menschen zu dem
zu
bewegen, was ihre eigene Glückseligkeit so sehr erheischt. Hätten die
aufgeklärten: Wüstlinge unserer Zeit nur einigen Begriff von dem Edelsinne einen
heidnischen Cato\index{Personen:!Cato}, so würden sie sich lieber ihre
sinnlichen Genüsse
versagen, als ein so edles Unternehmen unversucht lassen!} Wenn sie aber nur
essen, trinken, spielen, ihre Gesundheit zu Grunde richten, ihr Vermögen
verschwenden, und besonders ihre kostbare Zeit, -- die, zur nothwendigen
Vorbereitung aus die Ewigkeit, dem Herrn gewidmet sein sollte, und deren Werth,
wenn sie es einsähen, mit keinen irdischen Gütern zu vergleichen ist, -- so
unwiederbringlich verlieren; ich sage, \textbf{wenn sie so ununterbrorchen sich
beständig mit niedrigen, armseligen Kleinigkeiten beschäftigen, was können sie
dann anders erwarten, als daß an dem großen Gerichtstage Gottes die
Heiden\index{Personen:!Heiden} sie
richten und die Lehren und Beispiele Jesu und seiner wahren Nachfolger sie
verurtheilen werden.} Ja, ihr endliches Loos wird um so schrecklicher sein; da
sie alle ihre eitlen Thorheiten und Ausschweifungen als Bekenner des
Christenthumes, als Verehrer Jesu begingen, dessen Leben, so wie seine
Religion, Selbstverleugnung lehrer, und mit dem Lebenswandel der mehrsten
Christen im beständigen Widerspruche stehet. Denn Er, der
Gottmensch\index{Personen:!Gottmensch}, war
demüthig, sie aber sind stolz; er vergab seinen Feinden, sie suchen sich an
ihnen zu rächen; \textbf{er war sanftmüthig, sie sind jähzornig; er liebte
Einfachheit,
sie lieben Pracht; er war enthalsam, sie sind schwelgerisch; er war keusch, sie
sind wollüstig; er war ein Pilger auf Erden; sie sind Weltbürger; kurz, er war
niedrig geboren, wurde in der Verborgenheit erzogen, und ärmlich  bedient; -- so
lebte er verachtet und starb gehaßt von seinem eigenen Volke. Kommt nun, ihr
vorgeblichen Bekenner und Nachfolger dieses gekreuzigten Jesu!
\textit{"`Prüfet euch,
untersuchet euch selbst, ob ihr im Glauben seid; denn wenn ihr nicht erkennet,
daß Jesus Christus in euch ist, und in euch regieret, so seid ihr zur
Seligkeit untüchtig."'}}\index{Christus!innerer}
\footnote{2. Korinther 13,5}
\index{Bibelstellen:!2. Korinther 13)}
\textit{"`Irret euch
nicht, Gott läßt sich nicht spotten und endlich mit einer erzwungenen Reue
befriedigen; sondern, was ihr säet, das werdet ihr ernten."'}
\footnote{Galater 6,7+8.}
\index{Bibelstellen:!Galater 6)}
Ich bitte euch, höre: mich! Denker daran, daß ihr zum Heile Gottes, zu
eurer Seligkeit eingeladen und gebeten wurdet! Ich sage nochmalßi
\textit{"`Was ihr säet, werdet ihr ernten."'}
\textbf{Seid ihr Feinde des Kreuzes\index{Kreuz} Christi? und daß seid
ihr, wenn ihr euch weigert, es zu tragen, und nach eurem eigenen Willen, aber
nicht nach eurer Pflicht handeln wollt; seid ihr
Unbeschnittene\index{Beschneidung} an Herzen und
Ohren? und auch das seid ihr, wenn ihr Ihn nicht hören und ihm die Thüre nicht
öffnen wollt, der an euren Herzen anklopft; dämpfet ihr den Geist, und
widerstrebet seinem Einflusse, der mit euch ringet, um euch Gott zuzuführen?}
und das thut ihr ohne Zweifel so lange, als ihr seine Anregungen, Bestrafungen
und Belehrungen nicht achtet und befolget: \textbf{so säet ihr auf das Fleisch,
um die
Lüste desselben zu vollbringen, und werdet vom Fleische nichts anders ernten,
als Früchte den Verderbens, nämlich: Wehe, Trübsal und Angst von Gott, der durch
Jesum Christum die Lebendigen und die Todten richten wird.} Wollt ihr aber
das heilige Kreuz. Christi täglich aufnehmen und auf den Geist säen? Wollt ihr
dem Lichte, der Gnade, die durch Jesum Christum kommt, und weilche er Allen
zur Bewirkung ihrer Seligkeit verliehen hat, Gehör geben? Wollt ihr eure
Gedanken, Worte und Werke, in dem Lichte\index{Licht} dieser göttlichen
Gnade\index{Gnade} prüfen und
abmessen, welche Alle, die sie lieben, lehret,
"`das ungöttliche Wesen und die
weltlichen Lüste zu überwinden
\footnote{\texttt{'verleugnen' ersetzt durch 'überwinden'}}, und mäßig, gerecht
und gottselig in dieser Welt
zu leben?"'
\footnote{Titus 2,11+12}
\index{Bibelstellen:!Titus 2)}
so könnet ihr mit fester Zuversicht auf
die selige Hoffnung und Erscheinung der Herrlichkeit des großen Gottes und
unsere Heilandes Jesu Christi warten.
\footnote{Titus 13}
\index{Bibelstellen:!Titus 13)}
Wohlan! Laßt es so
sein, ihr Christen! Entfliehet dem zukünftigen Zorne! Warum wollt ihr sterben?
Laßt es an der verflossenen Zeit genug sein! Bedenket, daß \textbf{\textit{ohne
Kreuz keine
Krone}}\index{ohne-Kreuz-keine-Krone} zu hoffen ist! Erkaufet die Zeit; denn die
Tage sind böse, und der
eurigen werden nur noch wenige sein.
\footnote{Epheser 5,16}
\index{Bibelstellen:!Epheser 5)}
Darum
\textit{"`begürtet die
Lenden eures Gemüths; seid nüchtern, wachet, betet, und beharret bis ans
Ende."'}
Und zu eurer Aufmunterung und zu eurem Troste erinnert euch,
\textit{"`daß Alle, die mit
Geduld in guten Werken nach dem ewigen Leben trachten, auch Preis, Ehre und
Unsterblichkeit in dem Reiche des Vaters ernten werden;"'}
\footnote{Römer 2,7}
\index{Bibelstellen:!Römer 2)}
denn sein ist das Reich, die Kraft und die Herrlichkeit in Ewigkeit.
\textbf{Amen!}\index{Amen}
