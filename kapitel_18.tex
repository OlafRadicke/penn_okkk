
\chapter{Kapitel 18} \label{kap18}
\section{Zusammenfassung des 18. Kapitel}

\begin{description}
\item[1. Abschnitt] Wenn man euch die eitlen Gebräuche etc. als gleichgültige
Dinge betrachten könnte, so müssen sie dennoch, um des Missbrauchs willen, den
man davon macht, verworfen werden.
\dotfill \textit{Seite~\pageref{kap18_ab1}}\\
\item[2. Abschnitt] Selbst diejenigen, die sie mitmachen, erkennen diesen
Missbrauch, und sollten deshalb davon abstehen.
\dotfill \textit{Seite~\pageref{kap18_ab2}}\\
\item[3. Abschnitt] Wer nur einigermaßen auf Ehrbarkeit Anspruch macht, sollte
\item[3. Abschnitt] Wer nur einigermaßen auf Ehrbarkeit Anspruch macht, sollte
schon um des schlechten Beispiels willen solcher Ungebundenheit entsagen. Ein
weiser Vater entzieht seinem Kind den Gegenstand, an dem es zu sehr hängt, und
wir sind ja verpflichtet, über uns selbst und über unseren Nächsten zu wachen.
\dotfill \textit{Seite~\pageref{kap18_ab3}}\\
\item[4. Abschnitt] Gott gibt in dem Beispiel mit der ehernen Schlange eine
Belehrung, dass wir die Sachen, mit welchen Missbrauch getrieben wird, bei
Seite setzen sollen.
\dotfill \textit{Seite~\pageref{kap18_ab4}}\\
\item[5. Abschnitt] Möchte auch der Gebrauch dieser Dinge zuweilen zweckmäßig
sein, so müssen sie dennoch, sobald sie ein böses Beispiel geben, verworfen
werden.
\dotfill \textit{Seite~\pageref{kap18_ab5}}\\
\item[6. Abschnitt] Wer sich noch solche Vergnügungen erlaubt, zeige, dass er
sie
mehr als Christus und sein Kreuz liebt. -- Von den Nachteilen, die sie auf
Personen, Vermögen, Leib und Seele haben.
\dotfill \textit{Seite~\pageref{kap18_ab6}}\\
\item[7. Abschnitt] Verständige Leute wissen, dass dieses wahr ist. Was die
Schuldigen betrifft, so berufe ich mich auf den Zeugen Gottes in ihrem Herzen.
Ihr Zustand ist dem des geistigen Babylons ähnlich.
\dotfill \textit{Seite~\pageref{kap18_ab7}}\\
\item[8. Abschnitt] Mäßigkeit im Essen und Trinken, und Einfachheit in der
Kleidung, tragen sehr zur Beförderung des Guten bei, wie der Apostel Paulus in
seinen Episteln behauptet.
\dotfill \textit{Seite~\pageref{kap18_ab8}}\\
\item[9. Abschnitt] Mäßigkeit bereichert ein Land, und ist daher unter jeder
Regierung sowohl eine politische als auch religiöse Tugend.
\dotfill \textit{Seite~\pageref{kap18_ab9}}\\
\item[10 Abschnitt] Wenn man seine Pflichten gegen Gott erfüllt hat, so ist es
noch Zeit genug, an Erholung für sich selbst zu denken.
\dotfill \textit{Seite~\pageref{kap18_ab10}}\\
\item[11. Abschnitt] Vorschlag für die Obrigkeiten und für alle Menschen, wie
sie ihre Zeit und ihr Geld zu besseren Zwecken verwenden können.
\dotfill \textit{Seite~\pageref{kap18_ab11}}\\

\end{description}

\newpage

\section{1. Abschnitt} \label{kap18_ab1}

Sollten nun auch alle jene eitlen Moden, Gebräuche und Ergötzungen wirklich so
gleichgültig sein, wie sie als schädlich und verderblich bewiesen worden sind,
-- denn dass jemand in der Verteidigung derselben mehr als ihre Gleichgültigkeit
hätte behaupten wollen, ist mir nicht bekannt,~-- so ist doch der Missbrauch,
der
damit getrieben wird, so groß, und der Einfluss desselben, der wie eine
ansteckende Seuche um sich greift, so nachteilig, dass sie schon aus diesem
Grunde von allen Menschen, vornehmlich aber von denen, deren Mäßigkeit sie
bisher vor Übertreibung derselben bewahrte, oder welche, obgleich sie sich
ihrer schuldig gemacht haben, doch Verstand genug besitzen, die Torheit solcher
Unmäßigkeit einzusehen, gänzlich verworfen werden sollten. Was ist aber eine
gleichgültige Sache anderes, als eine solche, die man mit gleicher Befugnis tun
oder lassen kann? Und wenn ich nun auch einräumen würde, dass dieses hier der
Fall sei, so lehrt uns dennoch Vernunft und Religion, dass bei jeder Sache,
sobald man sie mit solcher Begierde gebraucht, dass es ein Kreuz\index{Kreuz}
sein würde, ihr
zu entsagen, die Grenzen solcher Gleichgültigkeit überschritten werden, und man
folglich gestehen muss, dass sie dadurch etwas Unentbehrliches geworden ist. Und
da nun auf diese Weise die Natur der Sache verletzt wird, so entsteht ein
vollkommener Missbrauch derselben und man kann sie folglich nicht mehr als
gleichgültig ansehen, sondern muss sie als unerlaubt betrachten.

\section{2. Abschnitt} \label{kap18_ab2}

Nun wird man mir zugeben, dass Missbräuche aller der Dinge, die ich so ernstlich
angegriffen habe, unter Personen von jedem Alter, Geschlecht und Stande
angetroffen werden. \label{ref:18_02_einwand}\textbf{Aber viele wollen ihnen
darum nicht entsagen, weil
sie
dieselben, wie ich vernommen habe, aus dem Grunde für erlaubt halten, dass, --
wie sie sagen,~-- der Missbrauch, den andere von einer Sache machen, ihren
Gebrauch derselben nicht aufhebe.} Allein sie haben vielleicht vergessen, oder
wollen es sich nicht einfallen lassen, dass sie ja alle jene Gebräuche etc. für
gleichgültige Dinge erklärt haben, und wenn dem so ist, wie denn auch die
Eitelkeit selbst nicht mehr verlangt, -- so sage ich~--, kann keine Schlussfolge
klarer sein, als die, dass sie alle diese Dinge notwendig aufgeben müssen, weil
sie den Missbrauch derselben eingestehen. \textbf{Denn, wenn sie so gleichgültig
sind,
dass man sie zu jeder Zeit nach Belieben mitmachen oder unterlassen kann, so
erfordert es gewiss unsere Pflicht, dass wir ihnen entsagen: Da es klar am Tag
liegt, dass unser Gebrauch derselben ihr allgemeines Übermaß noch vermehrt, und
andere in ihrem Missbrauch bestärkt, wenn sie sehen, dass Leute, die den Ruf
eines ehrbaren Lebenswandels behaupten wollen, solche Torheiten nachahmen, oder
ihnen darin mit ihrem Beispiel vorangehen, denn das Beispiel macht immer einen
noch weit stärkeren Eindruck als die Lehre.}\biblecite{Philipper 3,17.}{Philipper 03@Philipper 3}

\section{3. Abschnitt} \label{kap18_ab3}

Jeder, der Anspruch auf ernstes Nachdenken macht, sollte sich genau prüfen, ob
er nicht aus irgend eine Art zur Beförderung der herrschenden Ausschweifungen
schon beigetragen habe, und daher keine Zeit verlieren, sich gänzlich von ihnen
loszumachen, damit, wo sein voriges Beispiel anderen zur Aufmunterung
diente, sein besserer Lebenswandel auch den Einfluss haben möge, ihrer
Unmäßigkeit Einhalt zu tun. \index{Erziehung}Weise
Eltern\index{Eltern}
entziehen ihren Kindern\index{Kinder} die
Gegenstände, welche, so unschuldig sie auch an sich sein mögen, eine zu große
Gewalt über ihre schwachen Sinne ausüben, um sie davon zu entwöhnen. So pflegt
man auch einen krummen Stock, um ihn gerade zu machen, ebensosehr nach der
entgegengesetzten Seite zu biegen, damit er auf diese Art die gehörige Richtung
erhält. \index{Vorbild} \label{ref:18_03_vorbild}\textbf{Und gewiss,
diejenigen, welche mehr
Mäßigkeit als andere besitzen,
sollten nicht vergessen, dass sie Haushälter Gottes sind, und daher diese ihnen
verliehene Gabe zum Besten ihres Nächsten anwenden. Nur der Brudermörder
Kain\personenindex{Kain}
fragte den Herrn:
\textit{"`Soll ich denn meines Bruders Hüter sein?"'}}\biblecite{1.~Mose 4,9.}{Mose 1 04@1. Mose 4}
Denn dazu ist jeder unumgänglich verpflichtet, und daher sollte auch jeder
so klug sein, sich den Gebrauch solcher gleichgültigen Dinge zu versagen,
wodurch sein Nächster zu Torheiten verleitet oder in denselben bestärkt werden
könnte.

\section{4. Abschnitt} \label{kap18_ab4}

Gott hat in dieser Hinsicht seinen Willen hinlänglich zu erkennen gegeben, denn,
obgleich die eherne Schlange\index{Schlange} eine seiner eigenen Anordnungen und
ein Vorbild von
Christo war, so befahl er doch mit großem Missfallen, dass sie zerbrochen werden
sollte, weil das Volk zu sehr daran hing. Sogar die Gärten, so schön und
angenehm
auch ihre Lage war, mussten zerstört werden, und warum? Weil man sie zu
Gegenständen der Abgötterei\index{Gott!Abgötterei} gemacht hatte. Und was ist
ein
Abgott oder Götze
anderes als etwas, worauf dass menschliche Gemüt einen zu hohen Wert legt? Es
muss daher besser und vorteilhafter sein, einer gemissbrauchten Sache zu
entsagen, als eine gleichgültige zu gebrauchen.

\section{5. Abschnitt} \label{kap18_ab5}

Waren auch jene Dinge an sich selbst wirklich nützlich, welches sie eher
notwendig als gleichgültig machen würde, so müsste dennoch, sobald Umstände sie
schädlich machten, auch diese ihre Nützlichkeit auf die Seite gesetzt werden,
wievielmehr sollte man sie denn aber nicht als bloß gleichgültige Dinge
verleugnen? \textbf{Wir müssen niemals unser Privatvergnügen der öffentlichen
Wohlfahrt
vorziehen, noch uns den Gebrauch gleichgültiger Dinge erlauben, wenn wir wissen,
dass es dem Gemeinwohl nachteilig sein würde, wie dieses wirklich der Fall mit
allen solchen Dingen ist, durch deren Gebrauch man, aufs gelindeste genommen,
anderen ein böses Beispiel gibt und sie zur Nachahmung desselben reizt.} Daher
gebieten uns Vernunft und Pflicht, nur solche Dinge zu den unentbehrlichen zu
rechnen, die mit
\textit{"`einem göttlichen Leben und Wandel"'}\biblecite{2.~Petrus 1,3.}{Petrus 2 01@2. Petrus 1}
übereinstimmen, und unsere Freiheit ihres Gebrauchs nach dem, was unserem
Nächsten zuträglich ist, abzumessen. \textbf{Hier treten also zwei
Verpflichtungen ein.
Die erste ist:}dass man in solchen Dingen kein Beispiel gebe, deren Gebrauch,
wenn er auch an sich selbst unschädlich ist, doch unserem Nächsten in dem
Missbrauch derselben und in seiner übertriebenen Eitelkeit bestärken würde, und
\textbf{die zweite} besteht darin, dass man bei dem Gebrauch unerlaubter Dinge
auf
diejenigen religiösen Personen Rücksicht nehme, denen die eitlen Moden,
Gebräuche, Vergnügungen etc. anstößig sind, und sie daher gänzlich
vermeide.\biblecite{Römer 14,13 bis zum Ende.}{Römer 14}

\section{6. Abschnitt} \label{kap18_ab6}

Was nun diejenigen betrifft, die aller meiner angeführten Grunde ungeachtet,
dennoch in allen diesen Torheiten bleiben wollen, so muss es ihnen selbst
einleuchten, \textbf{dass sie keinen anderen Beweggrund dazu haben können, als
weil sie
tief darin verwickelt und ihre Neigungen so sehr davon hingerissen sind, dass es
ihnen fast unmöglich ist, sich davon loszusagen, und weil sie, ungeachtet ihrer
öfteren Versicherungen, dass ihr Herz an solchen Nichtigkeiten nicht hänge,
dennoch dieselben mehr als Christum und sein heiliges Kreuz\index{Kreuz}
lieben.} Wer seine
eigene Glückseligkeit so wenig berücksichtigt, wird auch schwerlich die
Wohlfahrt seines Nächsten zu befördern suchen.

\medskip

Wenn wir endlich die Wirkungen und Folgen der üppigen Moden, Gebräuche,
Ergötzungen und Zeitvertreib der Menschen ernstlich erwägen, so kann es uns
nicht entgehen, zu bemerken, wieviel Eitelkeit, Stolz, Müßiggang und sowohl
Zeit- als auch Geldverschwendung, kurz, wieviel Elend sie schon in der Welt
verbreitet haben und noch verbreiten. Wie viele Männer sind nicht dadurch von
ihrer Mäßigkeit abgebracht, wie viele Frauen nicht von ihrer natürlichen
Sanftmut und Unschuld zu leichten, ausschweifenden, frechen, ja schändlichen
Handlungen verleitet worden? Welche herrliche Besitzungen haben sie nicht in die
größte Schuldenlast gestürzt? Wie ist nicht so oft durch sie die Keuschheit in
die Schlingen der niederträchtigsten Verführer geraten und die jugendliche
Gesundheit durch die unnatürlichsten Krankheiten plötzlich untergraben worden,
so dass die unglücklichen Sklaven\index{Sklave} ihrer zügellosen
Begierden ihre
übrigen
Lebenstage unter beständigen Qualen zubringen mussten, welche die schrecklichen
aber unvermeidlichen Folgen ihrer Ausschweifungen waren. Und so groß ist das
Verderben der Menschen und die Macht des Lasters, dass einige während ihrer
Qualen die größte Mäßigkeit geloben, sobald diese aber vorüber sind, wieder zu
ihren vorigen Ausschweifungen zurückkehren.\bibmarginpar{
Klagelieder 4,5)
Weisheit Salomos 12,17)
Hiob 21,13+14)
Psalm 35,23)
Psalm 37,10)
Kohelet 8,12)
Psalm 37,1+2)
Sprüche 2,22.}
\bibleindex{Klagelieder 04@Klagelieder 4}
\bibleindex{Weisheit Salomos 12}
\bibleindex{Hiob 21}
\bibleindex{Psalm 035@Psalm 35}
\bibleindex{Psalm 037@Psalm 37}
\bibleindex{Kohelet 08@Kohelet 8}
\bibleindex{Sprüche 02@Sprüche 2}

\section{7. Abschnitt} \label{kap18_ab7}

Dass dieses und noch weit mehr wirklich der Fall sei, wird, meiner Überzeugung
nach, kein Verständiger, der einige Erfahrung besitzt, leugnen. Wie denn aber
irgend jemand, der ein Gewissen hat oder Gottesfurcht zu haben bekennt, nach
ernster Überlegung noch in der Tracht, in den Sitten, Gebräuchen und
Vergnügungen solcher Menschen, deren ganzes Leben mit Dingen, wie ich eine
berührt habe, befleckt ist, verweilen, oder sogar an ihren Ausschweifungen Teil
nehmen könne, überlasse ich dem gerechten Zeugen der Wahrheit in ihren eigenen
Herzen zu beurteilen. Nein wahrlich! Das heißt nicht der Stimme Gottes
gehorchen, die zu allen Zeiten laut rief: \textit{"`Gehet aus!"'} und woraus
denn? -- Aus den Wegen, Moden und Gebräuchen, aus der Lebensweise und dem Geiste
\textit{"`Babylons."'}\index{Orte:!Babylon} Und was ist Babylon? --
\textit{"`Die große Stadt, worin alle jene eitlen,
törichten, üppigen, überflüssigen und gottlosen Dinge getrieben werden"'},
gegen
welche die Schrift die furchtbarsten Gerichte\index{Gericht!Jüngstes} ankündigt,
indem sie alle
Unmäßigkeit und Zügellosigkeit der Menschen dem \textit{"`Becher der Bosheit"'}
zuschreibt,
aus welchem Babylon ihnen zu trinken gegeben hat, und woher alle sogenannte
gleichgültigen Dinge ihren Ursprung genommen haben.\bibmarginpar{Jeremia 16,5-9)
Jesaja 3,13-16)
Jesaja 50,8)
Jesaja 15,6+7)
Amos 6,3-7.}
\bibleindex{Jeremia 16}
\bibleindex{Jesaja 03@Jesaja 3}
\bibleindex{Jesaja 50}
\bibleindex{Jesaja 15}
\bibleindex{Amos 06@Amos 6}
Lasst uns hören, was Johannes in
seiner Offenbarung von ihr sagt:
\textit{"`Wieviel sie sich herrlich gemacht und ihren
Mutwillen gehabt hat, soviel schenkt ihr Qual und Leid ein. -- Und es werden die
Könige der Erde
sie beweinen und beklagen, die Mutwillen mit ihr getrieben
haben. Und die Kaufleute auf Erden werden weinen und Leid tragen, weil ihre
Ware niemand mehr kaufen wird, die Ware des Goldes, Silbers und der
Edelsteine, und Perlen, Seide, Purpur, Scharlach und allerlei
Edelholz\editnote{'Thienenholz' ersetzt durch 'Edelholz'.}, und von Erz,
Eisen, Marmor, Zimt, Thymian,
(Räuchwerk) Salben, Weihrauch, Wein, Öl, Semmelmehl,
Weizen, Vieh, Schafe, Pferde, Wagen, Leichname und Seelen der
Menschen."'}\biblecite{Offenbarung 7.9+11-13.}{Offenbarung  07@Offenbarung  7}
Hier sein wir sowohl den Charakter
als auch das Gericht der Maßlosigkeit\editnote{'Üppigkeit' ersetzt durch
'Maßlosigkeit'.}; denn obgleich diese
Worte, wie ich weiß,
noch eine andere als buchstäbliche Bedeutung
haben\index{Gebote!Bibelauslegung}\index{Bibelauslegung!wörtliche}, so
bezeichnen sie
doch auch
sehr deutlich die Pracht, den Überfluss, den Reichtum und das ganze müßige,
gemächliche, eitle und wollüstige Leben, welches unter den Einwohnern
Babylons\index{Orte:!Babylon}
herrscht. Aber wer will an dem schrecklichen Tage ihres
Gerichts\index{Gericht!Jüngstes} noch ihrem
Markt beiwohnen? Wer will dann ihre Schauspiele besuchen, ihre Moden nachahmen,
und mit ihren glänzenden Erfindungen Handel treiben? Gewiss niemand, denn sie
soll gerichtet werden. Kein Vorwand wird sie entschuldigen oder vor dem Zorn
des Richters schützen,
\textit{"`denn stark ist Gott der Herr, der sie richten
wird."'}\biblecite{Offenbarung 18,8.}{Offenbarung 18}

\medskip

Sollte man auch diesen vernünftigen Vorstellungen kein Gehör geben, so werde ich
dem ungeachtet noch einen Teil des traurigen Schicksals Babylons zur ferneren
Warnung anführen. Denn, meine Freunde, ihr müsst lernen, euch mit himmlischen
Dingen zu beschäftigen, und eilen, dem göttlichen Einfluss in eurem Inneren
Gehorsam zu leisten, der euch zu Betrachtungen ewiger Freuden anleitet, sonst
wird mit Babylon\index{Orte:!Babylon}, der Mutter aller
Maßlosigkeit\editnote{'Üppigkeit' ersetzt durch 'Maßlosigkeit'.} und
Eitelkeit,
\textit{"`das Obst, an
welchem eure Seele Lust hat, von euch weichen, und alles was köstlich und
herrlich ist, wird von euch weichen, und ihr werdet es nicht mehr
finden."'}\biblecite{Offenbarung 18,14.}{Offenbarung 18}
Nein! o ihr Reichen! Nicht mehr! Darum, ihr
Bewohner der Erde!
\textit{"`sammelt euch Schätze im Himmel,"'}\biblecite{Lukas 12,23+43.}{Lukas 12}
wohin nichts eindringen kann, das sie zerstören könnte, und wo sehr bald die
Zeit sich in der Ewigkeit verlieren wird.

\section{8. Abschnitt} \label{kap18_ab8}

Doch hören meine Beweise gegen die schädlichen Torheiten der Welt hier noch
nicht auf, ich muss noch zeigen, dass das Gegenteil derselben, nämlich:
Mäßigkeit\index{Mäßigkeit}
im Essen und Trinken und Einfachheit in der Kleidung, verbunden mit einem
sanften, bescheidenen Wesen und ruhigen Gemüt, dass bei jeder Gelegenheit durch
ein mit demselben übereinstimmendes Betragen in heiliger Ehrbarkeit sich
ausspricht, alles Gute in der Welt stiftet und verbreitet, weshalb auch der
Apostel die Christen ermahnt:
\textit{"`Lasst kein unnützes Geschwätz aus eurem Munde
gehen, sondern, was zur Besserung dient, und holdselig (erbaulich) zu hören
ist. Auch keine schändlichen Worte und Narrenteidinge (Possen) oder Scherze,
sondern vielmehr Danksagung. Lasst euch von niemand verführen mit vergeblichen
(leeren) Worten, denn um dieser Dinge willen kommt der Zorn Gottes "`über die
Kinder des Unglaubens."'}\bibmarginpar{
Kolosser 4,5+6)
1.~Thessalonicher 4,11+12)
1.~Petrus 3,1-4)
Epheser 4,4)
Epheser 5,3-6)
1.~Timotheus 4,12)
Philipper 3,16-20.}
\bibleindex{Kolosser 04@Kolosser 4}
\bibleindex{Thessalonicher 1 04@1. Thessalonicher 4}
\bibleindex{Petrus 1 03@1. Petrus 3}
\bibleindex{Epheser 04@Epheser 4}
\bibleindex{Epheser 05@Epheser 5}
\bibleindex{Timotheus 1 04@1. Timotheus 4}
\bibleindex{Philipper 03@Philipper 3}
\label{ref:18_08_vorbild_kleidung}\textbf{Wenn die Christen
beiderlei Geschlechts sich auf eine mit ihrem Bekenntnisse übereinstimmende Art
anständig kleideten, so würde die Unverschämtheit anderer dadurch gezügelt und
ihrer Prachtliebe, ihrem Stolze und ihrer Eitelkeit ein sie bestrafendes Muster
vorgehalten werden. Sie würden es nicht wagen, die allgemein anerkannte
Keuschheit anzugreifen, der göttliche Ernst der Christen würde ihre Frechheit
entwaffnen.} Der Tugend müsste man überall Achtung zollen, das Laster würde
sich furchtsam und beschämt zurückziehen müssen, und die Unmäßigkeit dürfte sich
nicht sehen lassen. Dieses würde aller Schwelgerei, aller Kleiderpracht, allem
Titelstolz und dem verschwenderischen Leben ein Ende machen, die ursprüngliche
Unschuld und Einfalt zurückrufen und jenes einfache, gerade, aufrichtige und
harmlose Leben wieder herstellen, wobei man nicht bekümmert ist, was man essen
oder trinken und womit man sich kleiden wolle, welches, wie Christus uns sagt,
die Sorge der Heiden ist, und, leider, auch unter den heutigen
Christentumsbekennern, bei allem ihrem Geschwätz von Religion und
Frömmigkeit, so häufig angetroffen wird. Man würde vielmehr das Beispiel der
Alten nachahmen, die, mit mäßiger Sorge für die Bedürfnisse und Bequemlichkeiten
dieses Lebens, ihre Hauptsorge auf die Angelegenheiten des himmlischen
Reichs\index{Reich!himmlisches}
richteten, und sich mehr um ihr Wachstum in der Gerechtigkeit als um die
Vermehrung ihres Reichtums kümmerten. "`Denn sie sammelten sich Schätze für
den Himmel und erduldeten Trübsale um des unvergänglichen Erbes
Willen"'\bibmarginpar{
2.~Petrus 2,12)
Sprüche 31,23-31)
Jakobus 2,2-9)
2.~Petrus 3,11)
Psalm 26,6)
Lukas 12,22-30)
Matthäus 25,21.}
\bibleindex{Petrus 2 02@2. Petrus 2}
\bibleindex{Sprüche 31}
\bibleindex{Jakobus 02@Jakobus 2}
\bibleindex{Petrus 2 03@2. Petrus 3}
\bibleindex{Lukas 12}
\bibleindex{Matthäus 25}

\section{9. Abschnitt} \label{kap18_ab9}

\label{ref:18_09_gesellschaftlich} \textbf{Die Tugend der Mäßigkeit, die ich
empfehle und verteidige, ist aber nicht
allein in religiöser, sondern auch in politischer\index{Politik} Hinsicht
nützlich, da es immer
einer jeden guten Regierung großen Vorteil bringt, wenn sie die
Ausschweifungen der Untertanen einschränkt und unterdrückt.} Das üppige Leben
erzeugt Weichlichkeit, Trägheit, Armut\index{Armut} und Elend\index{Elend};
aber die Mäßigkeit beugt
solchen Übeln vor und erhält das Land im Wohlstand.\bibmarginpar{Sprüche 10,4)
Kohelet 10,16-18.}
\bibleindex{Sprüche 10}
\bibleindex{Kohelet 10}
\index{Außenhandel}Sie bewahrt vor ausländischen Torheiten und
Überflüssigkeiten, und befördert die Verbesserung unserer eigenen Erzeugnisse,
wodurch wir, statt anderen schuldig zu sein, sie zu unseren Schuldnern machen
können. Durch Ausübung dieser Tugend können Leute die durch Ausschweifung, nicht
durch Wohltätigkeit, ihre Güter in tiefe Schulden gestürzt haben, in kurzer
Zeit von diesen Schulden wieder frei werden, die sonst, wie fressende Motten,
die größten Besitzungen verzehren. Die Mäßigkeit hilft Unbemittelten, ihr
geringes Vermögen zu vermehren, indem sie ihnen nicht erlaubt, ihren mühsam
erworbenen Verdienst mit Putz und Kleiderpracht, törichten Maifesten oder
Fastnachtsbelustigungen, Spielen, Tanzen\index{Tanz} und Schauspielen, mit
Schwelgerei in
Kneipen und Wirtshäusern\index{Wirtshaus} und anderen Ausbrüchen der
Unmäßigkeit durchzubringen,
welche Torheiten besonders in unserem Vaterland (England)\index{Orte:!England}
so sehr überhand
genommen haben, dass es dadurch mehr, als irgend ein anderes Land in der Welt,
lächerlich gemacht wird. Denn nirgends findet man, so viel ich weiß, mehr
betrügerische Marktschreier und Quacksalber\index{Quacksalber},
Seiltänzer,
Taschendiebe, profane
Komödianten und Gaukler\index{Gaukler}, zur großen Herabwürdigung der
Religion
und zur Schande
der Regierung\index{Regierung}, wodurch dann das Volk zum Müßiggang, zur
Verschwendung und zu
Ausschweifungen verführt, der heilige Geist betrübt und der Allmächtige gereizt
wird, seine Gerichte, die vor der Tür stehen, hereinbrechen zu lassen und
endlich den Ausspruch zu tun:
\textit{"`Wer böse ist, sei immerhin böse."'}\biblecite{Offenbarung 22,11.}{Offenbarung 22}
\index{Zeugnis}\index{Missionierung}\index{Weg!der einzige}
\label{ref:18_09_lautstark} \textbf{-- Daher
können wir nicht anders, als unsere
Stimmen laut zu erheben und sowohl durch unsere Lehre, als auch durch unseren
Lebenswandel, gegen die Missbräuche unserer Mitmenschen zeugen, damit, wo
möglich, einige dadurch bewogen werden mögen, solche Torheiten aufzugeben,
und den guten alten Pfad der Mäßigkeit, der Weisheit, des Ernstes und der
Heiligkeit zu wählen, da dieser der einzige ist, der zu deinem wahren Genuss
der Segnungen des Friedens und der Fülle in dieser Zeit und der ewigen
Glückseligkeit in der künftigen Welt führt.}

\section{10. Abschnitt} \label{kap18_ab10}

Gesetzt, endlich, wir hätten auch keine der angeführten Gründe für uns, um die
im Land herrschenden bösen Gebräuche und Gewohnheiten mit Recht zu verwerfen,
so werdet ihr uns doch erlauben, zu bemerken, dass er dann, wenn die Menschen
gelernt haben, ihren Schöpfer zu fürchten, ihn anzubeten und ihm zu
gehorchen, wenn sie die großen Schulden ihrer Abweichungen abgetragen und die
Last ihrer Untergebenen erleichtert haben werden, wenn vornehmlich die blassen
Gesichter mehr bemitleidet, die Hungrigen gesättigt, die Nackten bekleidet, die
schmachtenden Armen, die verlassenen Witwen und hilflosen Waisen, -- welche
Gottes Geschöpfe und eure Mitmenschen sind,~-- versorgt sein werden, alsdann,
sage ich, wenn ihr je können werdet, -- wird es noch Zeit genug für euch sein,
die Gleichgültigkeit und Unschädlichkeit eurer Vergnügungen zu
behaupten.\bibmarginpar{
Kohelet 12,1)
Psalm 37,21)
Psalm 10,2)
Psalm 4,2)
Psalm 84,3+4)
Sprüche 22,7+9)
Jesaja 3,14+15.}
\bibleindex{Kohelet 12}
\bibleindex{Psalm 037@Psalm 37}
\bibleindex{Psalm 010@Psalm 10}
\bibleindex{Psalm 004@Psalm 4}
\bibleindex{Psalm 084@Psalm 84}
\bibleindex{Sprüche 22}
\bibleindex{Jesaja 03@Jesaja 3}
\index{Soziale Ungerechtigkeit}
\label{ref:18_10_ungeraechtikeit} \textbf{Dass aber der Schweiß und die mühsamen
Arbeiten
des Landmannes vom frühen Morgen bis in die späte Nacht, bei Hitze und Kälte,
bei Dürre und Nässe, für das Vergnügen, das Wohlbehagen und die Zeitvertreib
einer kleinen Anzahl von Menschen zu sorgen bestimmt wären, dass neunzehn
Zwanzigstel\editnote{...Etwas umständlich ausgedrückt. Mann könnte auch
einfach sagen: '5\% der Bevölkerung (nämlich die Oberschicht) lebt auf Kosten
von 95\% der hart arbeitenden Bevölkerung, in Saus und Braus'.} der Bewohner
des Landes beständig den härtesten Arbeiten mit dem
Pflug, dem Karren, dem Fuhrwerk und dem Dreschflegel unterworfen sein müssten,
um
die ungezügelten Begierden der Maßlosigkeit\editnote{'Üppigkeit' ersetzt
durch 'Maßlosigkeit'.} und Schwelgerei des
einen Zwanzigstels
zu befriedigen, ist so weit entfernt, eine Anordnung des großen Regierers der
Welt und Gottes der Geister alles Fleisches zu sein, dass es im höchsten Grade
gotteslästerlich und abscheulich sein würde, nur zu denken, dass solche
schreiende Ungerechtigkeit von \textit{Ihm} und seinen göttlichen Einrichtungen,
und
nicht von der Unmäßigkeit der Menschen herkomme.}\bibmarginpar{
Amos 5,11+12)
Amos 8,4)
1. Petrus 1,17)
Jakobus 5,4+5)
Psalm 41,1)
Jakobus 2,15+16)
Psalm 112,9.}
\bibleindex{Amos 05@Amos 5}
\bibleindex{Amos 08@Amos 8}
\bibleindex{Petrus 1 01@1. Petrus 1}
\bibleindex{Jakobus 05@Jakobus 5}
\bibleindex{Psalm 041@Psalm 41}
\bibleindex{Jakobus 02@Jakobus 2}
\bibleindex{Psalm 112}
Von einer anderen Seite
betrachtet, können fürwahr\editnote{'billig' ersetzt durch 'fürwahr'.}
die Menschen auf das Mitleid oder auf die Hilfe und
den Beistand des allmächtigen Gottes keine Ansprüche machen, wenn sie mit ihren
Ausgaben für eitle Vergnügungen fortfahren, während sie die Bedürfnisse der
Unglücklichen unbefriedigt lassen, besonders wenn man erwägt, dass Gott sie nur
zu Haushaltern gesetzt hat, die einander dienen, helfen und unterstützen sollen.
Ja, diese Pflichten sind so bestimmt eingeschärft, dass wir, wenn wir sie
unterlassen, nicht ohne Grund den schrecklichen Ausspruch erwarten mussen:
\textit{"`Geht von mir, ihr Verfluchten, in das ewige
Feuer."'}\index{Flucht} usw.\biblecite{Matthäus 25,34-41.}{Matthäus 25}
\index{Mildtätigkeit}Wohingegen das Besuchen der Kranken und Gefangenen und die
Unterstützung der Durstigen in den Augen Jesu solche vortreffliche Eigenschaften
sind, dass er diejenigen, die es daran nicht ermangeln ließen, mit dem
erfreulichen Ausspruche selig preisen wird:
\textit{"`Kommt her, ihr Gesegneten meines
Vaters, erbet das Reich, das für euch bereitet ist"'}, usw. Die Großen der
Erde sollten also nicht, wie der Wal\editnote{'Leviathan' ersetzt durch
'Wal'.} im Meer, die Kleinen zu ihrer Beute
machen, und noch vielweniger mit deinem Leben und der mühsamen Arbeit der
Geringern Scherz treiben, um ihre zügellosen Neigungen zu befriedigen.

\section{11. Abschnitt} \label{kap18_ab11}

Sei es mir daher erlaubt, den bürgerlichen
Obrigkeiten\index{Bürgerlichen Obrigkeiten} einen Vorschlag zu machen,
der ihre ernste Betrachtung zu verdienen scheint. Nämlich: Wenn in jedem
Gerichtsbezirk die Einwohner dazu aufgefordert und aufgemuntert würden, das
Geld, welches sie für eitle Überflüssigkeiten, zum Beispiel Spitzenware, Juwelen,
Stickereien, Besetzungen auf Kleider, kostbare Verzierungen an Möbeln und für
unnötige Dienerschaft ausgeben, nebst dem, was sie gewöhnlich bei Gastmälern,
Festen, oder beim Spiel usw. ausgeben\editnote{'aufgehen lassen'
ersetzt durch 'ausgeben'.}, in eine öffentliche Kasse zu
legen, so würde man bald ein hinreichendes Kapital zusammen bringen können,
womit verarmte Familien unterstützt, Arbeitshäuser für Arme\index{Sozialer
Wohnungsbau}, die noch zu
arbeiten fähig wären, erbaut, und Armenanstalten für alte und
gebrechliche\index{Altersheim}
Leute errichtet und unterhalten werden könnten. Dann
würden wir keine Bettler\index{Bettler} im Lande haben, und man würde
die Klagen
und das
Geschrei der Witwen und Waisen nicht mehr hören. Es könnten unglückliche
Gefangene losgekauft und die Drangsale der Protestanten, die in anderen Ländern
Verfolgung leiden, erleichtert werden\index{Glaubensverfolgter}. Ja,
selbst dem öffentlichen Schatz\index{Sozialversicherung} könnte, in dringenden
Fällen, aus solchen
Sparbänken Unterstützung geleistet werden\editnote{...Also soetwas wie
eine Sozialversicherung.}. Dieses würde ein Opfer, ein Dienst
sein, der dem gerechten und gnädigen Gott gefiele. Man gäbe dadurch anderen
Ländern ein edles Beispiel der Weisheit und Mäßigkeit und stiftete eine
unschätzbare Wohltat für das Vaterland.\bibmarginpar{Sprüche 14,21.)
Matthäus 19,21.}
\bibleindex{Sprüche 14}
\bibleindex{Matthäus 19}\footnote{Der wegen seiner großen
Menschenliebe allgemein geschätzte und vornehmlich wegen seiner unablässigen
edlen Bemühungen, die unglückliche Lage der armen Gefangenen überall zu
erleichtern, in ganz Europa berühmte John Howard\personenindex{Howard, John},
fand einst beim Abschluss
einer Jahresrechnung einen bedeutenden
Überschuß seiner Einnahmen. Er bot die
ganze Summe seiner Frau an, um zu einer Reise nach London\index{Orte:!London}
oder zu irgend
einem anderen ihrer beliebigen Vergnügen Gebrauch davon zu machen. Die
Edelmütige
erwiderte: Was für eine nette Wohnung für eine arme Familie ließe sich doch
damit erbauen! Dieser schöne Wink fand ganz den Beifall
ihres guten Mannes, und das Geld wurde zu dem von ihr vorgeschlagenen
wohltätigen Zweck verwendet. Philipper 4,8
\textbf{(Anmerkung des Übersetzers)}.}

\medskip

\label{ref:18_11_ueberredung}
\textbf{Ach! Warum bedarf es denn so vieler Überredung, um die Menschen zu dem
zu
bewegen, was ihre eigene Glückseligkeit so sehr erheischt? Hätten die
aufgeklärten Wüstlinge unserer Zeit nur einigen Begriff von dem Edelsinn eines
heidnischen Cato\personenindex{Cato}, so würden sie sich lieber ihre
sinnlichen Genüsse
versagen, als ein so edles Unternehmen unversucht lassen!} Wenn sie aber nur
essen, trinken, spielen, ihre Gesundheit zu Grunde richten, ihr Vermögen
verschwenden, und besonders ihre kostbare Zeit, -- die, zur notwendigen
Vorbereitung auf die Ewigkeit, dem Herrn gewidmet sein sollte, und deren Wert,
wenn sie es einsähen, mit keinen irdischen Gütern zu vergleichen ist,~-- so
unwiederbringlich verlieren, ich sage, \textbf{wenn sie so ununterbrochen sich
beständig mit niedrigen, armseligen Kleinigkeiten beschäftigen, was können sie
dann anderes erwarten, als dass an dem großen Gerichtstag Gottes die
Heiden\index{Heide} sie
richten und die Lehren und Beispiele Jesu und seiner wahren Nachfolger sie
verurteilen werden.} Ja, ihr endliches Los wird um so schrecklicher sein, da
sie alle ihre eitlen Torheiten und Ausschweifungen als Bekenner des
Christentumes, als Verehrer Jesu begingen, dessen Leben, so wie seine
Religion, Selbstverleugnung lehret, und mit dem Lebenswandel der mehrsten
Christen im beständigen Widerspruche stehet. Denn er, der
Gottmensch\index{Gottmensch}, war
demütig, sie aber sind stolz, er vergab seinen Feinden, sie suchen sich an
ihnen zu rächen, \textbf{er war sanftmütig, sie sind jähzornig, er liebte
Einfachheit,
sie lieben Pracht, er war enthaltsam, sie sind schwelgerisch, er war keusch, sie
sind wollüstig, er war ein Pilger auf Erden, sie sind Weltbürger, kurz, er war
niedrig geboren, wurde in der Verborgenheit erzogen, und ärmlich bedient, -- so
lebte er verachtet und starb gehaßt von seinem eigenen Volk. Kommt nun, ihr
vorgeblichen Bekenner und Nachfolger dieses gekreuzigten Jesu!
\textit{"`Prüft euch,
untersucht euch selbst, ob ihr im Glauben seid, denn wenn ihr nicht erkennt,
dass Jesus Christus in euch ist, und in euch regiert, so seid ihr zur
Seligkeit ungeeignet."'}}\index{Christus!innerer}\biblecite{2.~Korinther 13,5.}{Korinther 2 13@2. Korinther 13}
\textit{"`Irrt euch
nicht, Gott lässt sich nicht spotten und endlich mit einer erzwungenen Reue
befriedigen, sondern, was ihr sät, das werdet ihr ernten."'}\biblecite{Galater
6,7+8.}{Galater 06@Galater 6}
Ich bitte euch, höret mich! Denkt daran, dass ihr zum Heil Gottes, zu
euerer Seligkeit eingeladen und gebeten wurdet! Ich sage nochmal
\textit{"`Was ihr sät, werdet ihr ernten."'}
\label{ref:18_11_feinde_des_kreuzes}
\textbf{Seid ihr Feinde des Kreuzes\index{Kreuz} Christi? Und das seid
ihr, wenn ihr euch weigert, es zu tragen, und nach euerem eigenen Willen, aber
nicht nach euerer Pflicht handeln wollt, seid ihr
Unbeschnittene\index{Beschneidung} an Herzen und
Ohren? Und auch das seid ihr, wenn ihr ihn nicht hören und ihm die Türe nicht
öffnen wollt, der an eueren Herzen anklopft, dämpft ihr den Geist, und
widerstrebet seinem Einflusse, der mit euch ringt, um euch Gott zuzuführen?}
und das tut ihr ohne Zweifel so lange, als ihr seine Anregungen, Bestrafungen
und Belehrungen nicht achtet und befolgt: \textbf{so sät ihr auf das Fleisch,
um die
Lüste desselben zu vollbringen, und werdet vom Fleisch nichts anderes ernten,
als Früchte den Verderbens, nämlich: Wehe, Trübsal und Angst von Gott, der durch
Jesum Christum die Lebendigen und die Toten richten wird.} Wollt ihr aber
das heilige Kreuz Christi täglich aufnehmen und auf den Geist sän? Wollt ihr
dem Licht, der Gnade, die durch Jesum Christum kommt, und welche er allen
zur Bewirkung ihrer Seligkeit verliehen hat, Gehör geben? Wollt ihr euere
Gedanken, Worte und Werke, in dem Licht\index{Licht} dieser göttlichen
Gnade\index{Gnade} prüfen und
abmessen, welche alle, die sie lieben, lehrt,
"`das ungöttliche Wesen und die
weltlichen Lüste zu überwinden\editnote{'verleugnen' ersetzt durch
'überwinden'.}, und mäßig, gerecht
und gottselig in dieser Welt
leben?"'\biblecite{Titus 2,11+12.}{Titus 02@Titus 2}
so könnt ihr mit fester Zuversicht auf
die selige Hoffnung und Erscheinung der Herrlichkeit des großen Gottes und
unseres Heilands Jesu Christi warten.\biblecite{Titus 13.}{Titus 13}
Wohlan! Laßt es so
sein, ihr Christen! Entflieht dem zukünftigen Zorn! Warum wollt ihr sterben?
Laßt es an der verflossenen Zeit genug sein! Bedenkt, dass \textbf{\textit{ohne
Kreuz keine
Krone}} zu hoffen ist! Erkauft die Zeit, denn die
Tage sind böse, und der
eurigen werden nur noch wenige sein.\biblecite{Epheser 5,16.}{Epheser 05@Epheser 5}
Darum
\textit{"`begürtet die
Lenden eueres Gemüts, seid nüchtern, wacht, betet und harrt bis ans
Ende."'}
Und zu euerer Aufmunterung und zu euerem Trost erinnert euch,
\textit{"`das alle, die mit
Geduld in guten Werken nach dem ewigen Leben trachten, auch Preis, Ehre und
Unsterblichkeit in dem Reich des Vaters ernten werden,"'}\biblecite{Römer 2,7.}{Römer 02@Römer 2}
denn sein ist das Reich, die Kraft und die Herrlichkeit in Ewigkeit.
\textbf{Amen!}\index{Amen!}



